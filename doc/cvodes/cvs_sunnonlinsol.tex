%===============================================================================
\chapter{Description of the SUNNonlinearSolver module}\label{c:sunnonlinsol}
%===============================================================================
\index{SUNNonlinearSolver@\texttt{SUNNonlinearSolver} module}
{\sundials} packages are written in terms of generic nonlinear solver
operations defined by the {\sunnonlinsol} API and implemented by a
particular {\sunnonlinsol} module of type \Id{SUNNonlinearSolver}.
Users can supply their own {\sunnonlinsol} module, or use one of the
modules provided with {\sundials}. The following section presents the
{\sunnonlinsol} API and its implementation. The subsequent sections in
this chapter describe the {\sunnonlinsol} modules provided with
{\sundials}.


% ====================================================================
\section{The SUNNonlinearSolver API}
\label{s:sunnonlinsol_api}
% ====================================================================

The {\sunnonlinsol} API defines several nonlinear solver operations
that enable {\sundials} packages to utilize any {\sunnonlinsol}
implementation that provides the required functions. These functions
can be divided into three categories. The first are the core nonlinear
solver functions: get type, initialization, setup, solve, and free.
The second group of functions consists set routines to supply the
nonlinear solver with functions provided by the {\sundials} package
and to modify solver parameters. The final group consists of get
routines for retrieving nonlinear solver statistics. All of these
functions are defined in the header file
\id{sundials/sundials\_nonlinearsolver.h}.

% ====================================================================
\subsection{SUNNonlinearSolver core functions}
\label{ss:sunnonlinsol_corefn}
% ====================================================================
\ucfunction{SUNNonlinSolGetType}
{
  type = SUNNonlinSolGetType(NLS);
}
{
  The \textit{required} function \ID{SUNNonlinSolGetType} returns the
  type identifier for the nonlinear solver \id{NLS} and is used to
  determine the solver type.
}
{
  \begin{args}[NLS]
  \item[NLS] (\id{SUNNonlinearSolver})
    a {\sunnonlinsol} object.
  \end{args}
}
{
  The return value \id{type} (of type \id{int}) will be one of the
  following:
  \begin{args}[SUNNONLINEARSOLVER\_STATIONARY]
  \item[\Id{SUNNONLINEARSOLVER\_ROOTFIND}]
    \id{0}, the {\sunnonlinsol} module solves $F(y) = 0$.
  \item[\Id{SUNNONLINEARSOLVER\_STATIONARY}]
    \id{1}, the {\sunnonlinsol} module solves $G(y) = y$.
  \end{args}
}
{}
% --------------------------------------------------------------------
\ucfunction{SUNNonlinSolInitialize}
{
  retval = SUNNonlinSolInitialize(NLS);
}
{
  The \textit{required} function \ID{SUNNonlinSolInitialize} preforms
  nonlinear solver initialization.
}
{
  \begin{args}[NLS]
  \item[NLS] (\id{SUNNonlinearSolver})
    a {\sunnonlinsol} object.
  \end{args}
}
{
  The return value \id{retval} (of type \id{int}) should be zero for a
  successful call and a negative value for a failure.
}
{
  This function assumes all solver-specific options have been set prior
  to being called.
}
% --------------------------------------------------------------------
\ucfunction{SUNNonlinSolSetup}
{
  retval = SUNNonlinSolSetup(NLS, y, mem);
}
{
  The \textit{optional} function \ID{SUNNonlinSolSetup} preforms any
  solver setup needed before a nonlinear solve.
}
{
  \begin{args}[NLS]
  \item[NLS] (\id{SUNNonlinearSolver})
    a {\sunnonlinsol} object.
  \item[y] (\id{N\_Vector})
    the initial iteration passed to the nonlinear solver.
  \item[mem] (\id{void *})
    the {\sundials} package memory structure.
  \end{args}
}
{
  The return value \id{retval} (of type \id{int}) should be zero for a
  successful call and a negative value for a failure.
}
{
  {\sunnonlinsol} implementations that do not require setup may set
  this operation to \id{NULL}.
}
% --------------------------------------------------------------------
\ucfunction{SUNNonlinSolSolve}
{
  retval = SUNNonlinSolSolve(NLS, y0, y, w, tol, callSetup, mem);
}
{
  The \textit{required} function \ID{SUNNonlinSolSolve} solves the
  nonlinear system $F(y)=0$ or $G(y)=y$.
}
{
  \begin{args}[callSetup]
  \item[NLS] (\id{SUNNonlinearSolver})
    a {\sunnonlinsol} object.
  \item[y0] (\id{N\_Vector})
    the initial iterate for the nonlinear solver and should be left
    unchanged.
  \item[y] (\id{N\_Vector})
    the solution to the nonlinear system.
  \item[w] (\id{N\_Vector})
    the weight vector used in computing weighted norms.
  \item[tol] (\id{realtype})
    the specified tolerance in a weight root mean square norm.
  \item[callSetup] (\id{booleantype})
    a flag indicating if the linear solver setup function should be
    called. 
  \item[mem] (\id{void *})
    the {\sundials} package memory structure.
  \end{args}
}
{
  The return value \id{retval} (of type \id{int}) should be one of the
  following:
  \begin{args}[SUN\_NLS\_CONV\_RECVR]
  \item[\Id{SUN\_NLS\_SUCCESS}]
    the solve was successful.
  \item[\Id{SUN\_NLS\_CONV\_RECVR}]
    the solve failed to converge and the integrator should attempt to
    recover.
  \item[\id{*\_RHSFUNC\_RECVR}]
    the ODE right-hand side function returned a recoverable error
  \item[\id{*\_RES\_RECVR}]
    the DAE residual function returned a recoverable error
  \item[\id{*\_LSETUP\_RECVR}]
    the linear solver setup function returned a recoverable error
  \item[\id{*\_LSOLVE\_RECVR}]
    the linear solver solve function returned a recoverable error
  \item[\id{*\_MEM\_NULL}]
    the {\sundials} package memory was \id{NULL}
  \item[\id{*\_RHSFUNC\_FAIL}]
    the ODE right-hand side function returned an unrecoverable error
  \item[\id{*\_RES\_FAIL}]
    the DAE residual function returned an unrecoverable error
  \item[\id{*\_LSETUP\_FAIL}]
    the linear solver setup function returned an unrecoverable error
  \item[\id{*\_LSOLVE\_FAIL}]
    the linear solver solve function returned an unrecoverable error
  \end{args}
  In the above return codes \id{*} is a {\sundials} package specific
  prefix (\id{CV} for {\cvode} or {\cvodes}, \id{IDA} for {\ida} or
  {\idas}, and \id{ARK} for {\arkode}).
}
{}
% --------------------------------------------------------------------
\ucfunction{SUNNonlinSolFree}
{
  retval = SUNNonlinSolFree(NLS);
}
{
  The \textit{required} function \ID{SUNNonlinSolFree} frees memory
  allocated by the nonlinear solver.
}
{
  \begin{args}[NLS]
  \item[NLS] (\id{SUNNonlinearSolver})
    a {\sunnonlinsol} object.
  \end{args}
}
{
  The return value \id{retval} (of type \id{int}) should be zero for a
  successful call, and a negative value for a failure.
}
{}


% ====================================================================
\subsection{SUNNonlinearSolver set functions}
\label{ss:sunnonlinsol_setfn}
% ====================================================================
\ucfunction{SUNNonlinSetSysFn}
{
  retval = SUNNonlinSolSetSysFn(NLS, SysFn);
}
{
  The \textit{required} function \ID{SUNNonlinSolSetSysFn} is used
  to provide the nonlinear solver with the function defining the
  nonlinear system. This is the function $F(y)=0$ for
  \wtt{SUNNONLINERASOLVER\_ROOTFIND} modules or $G(y)=y$ for
  \wtt{SUNNONLINEARSOLVER\_STATIONARY} modules.
}
{
  \begin{args}[SysFn]
  \item[NLS] (\id{SUNNonlinearSolver})
    a {\sunnonlinsol} object.
  \item[SysFn] (\id{SUNNonlinSolSysFn})
    the function defining the nonlinear system. See
    \ref{ss:sunnonlinsol_sunsuppliedfn} for the definition of
    \id{SUNNonlinSolSysFn}.
  \end{args}
}
{
  The return value \id{retval} (of type \id{int}) should be zero for a
  successful call, and a negative value for a failure.
}
{}
% --------------------------------------------------------------------
\ucfunction{SUNNonlinSetLSetupFn}
{
  retval = SUNNonlinSolSetLSetupFn(NLS, LSetupFn);
}
{
  The \textit{optional} function \ID{SUNNonlinSolLSetupFn} is used to
  provide the nonlinear solver with a wrapper function to the
  {\sundials} package's linear solver setup function.
}
{
  \begin{args}[LSetupFn]
  \item[NLS] (\id{SUNNonlinearSolver})
    a {\sunnonlinsol} object.
  \item[LSetupFn] (\id{SUNNonlinSolLSetupFn})
    a wrapper function to the {\sundials} package's linear solver
    setup function. See \ref{ss:sunnonlinsol_sunsuppliedfn} for the
    definition of \id{SUNNonlinLSetupFn}.
  \end{args}
}
{
  The return value \id{retval} (of type \id{int}) should be zero for a
  successful call, and a negative value for a failure.
}
{
  {\sunnonlinsol} implementations not utilizing {\sunlinsol} modules
  may set this operation to \id{NULL}.
}
% --------------------------------------------------------------------
\ucfunction{SUNNonlinSetLSolveFn}
{
  retval = SUNNonlinSolLSolve(NLS, LSolveFn);
}
{
  The \textit{optional} function \ID{SUNNonlinSolLSolve} is used to
  provide the nonlinear solver with a wrapper function to the
  {\sundials} package's linear solver solve function.
}
{
  \begin{args}[LSolveFn]
  \item[NLS] (\id{SUNNonlinearSolver})
    a {\sunnonlinsol} object
  \item[LSolveFn] (\id{SUNNonlinSolLSolveFn})
    a wrapper function to the {\sundials} package's linear solver
    solve function. See \ref{ss:sunnonlinsol_sunsuppliedfn} for the
    definition of \id{SUNNonlinSolLSolveFn}.
  \end{args}
}
{
  The return value \id{retval} (of type \id{int}) should be zero for a
  successful call, and a negative value for a failure.
}
{
  {\sunnonlinsol} implementations not utilizing {\sunlinsol} modules
  may set this operation to \id{NULL}.
}
% --------------------------------------------------------------------
\ucfunction{SUNNonlinSetConvTestFn}
{
  retval = SUNNonlinSolSetConvTestFn(NLS, CTestFn);
}
{
  The \textit{optional} function \ID{SUNNonlinSolSetConvTestFn} is
  used to provide the nonlinear solver with a function for determining
  if the nonlinear solver iteration has converged.
}
{
  \begin{args}[CTestFn]
  \item[NLS] (\id{SUNNonlinearSolver})
    a {\sunnonlinsol} object.
  \item[CTestFn] (\id{SUNNonlinearSolConvTestFn})
    a {\sundials} package nonlinear solver convergence test function.
    See \ref{ss:sunnonlinsol_sunsuppliedfn} for the definition of
    \id{SUNNonlinSolConvTestFn}.
  \end{args}
}
{
  The return value \id{retval} (of type \id{int}) should be zero for a
  successful call, and a negative value for a failure.
}
{
  {\sunnonlinsol} implementations utilizing their own convergence test
  criteria may set this function to \id{NULL}.
}
% --------------------------------------------------------------------
\ucfunction{SUNNonlinSetMaxIters}
{
  retval = SUNNonlinSolSetMaxIters(NLS, maxiters);
}
{
  The \textit{optional} function \ID{SUNNonlinSolSetMaxIters} sets the
  maximum number of nonlinear solver iterations.
}
{
  \begin{args}[maxiters]
  \item[NLS] (\id{SUNNonlinearSolver})
    a {\sunnonlinsol} object.
  \item[maxiters] (\id{int})
    the maximum number of nonlinear iterations.
  \end{args}
}
{
  The return value \id{retval} (of type \id{int}) should be zero for a
  successful call, and a negative value for a failure
  (e.g., $\id{maxiters} < 1$).
}
{}


% ====================================================================
\subsection{SUNNonlinearSolver get functions}
\label{ss:sunnonlinsol_getfn}
% ====================================================================
\ucfunction{SUNNonlinGetNumIters}
{
  retval = SUNNonlinSolGetNumIters(NLS, numiters);
}
{
  The \textit{optional} function \ID{SUNNonlinSolGetNumIters} to get
  the total number of nonlinear solver iterations.
}
{
  \begin{args}[numiters]
  \item[NLS] (\id{SUNNonlinearSolver})
    a {\sunnonlinsol} object
  \item[numiters] (\id{long int})
    the total number of nonlinear solver iterations.
  \end{args}
}
{
  The return value \id{retval} (of type \id{int}) should be zero for a
  successful call, and a negative value for a failure.
}
{}


% ====================================================================
\subsection{SUNDIALS package provided SUNNonlinearSolver routines}
\label{ss:sunnonlinsol_sunsuppliedfn}
% ====================================================================

The {\sundials} packages provide the {\sunnonlinsol} module with
routines for evaluating the nonlinear system, calling the {\sunlinsol}
setup and solve functions, and testing the nonlinear iteration for
convergence. The function types for each of these functions are
defined in the header file \id{sundials/sundials\_nonlinearsolver.h},
and are described below. 
% --------------------------------------------------------------------
\usfunction{SUNNonlinSolSysFn}
{
  typedef int (*SUNNonlinSolSysFn)(N\_Vector y, N\_Vector F, void* mem);
}
{
  These functions evaluate the nonlinear system $F(y)$
  for \wtt{SUNNONLINEARSOLVER\_ROOTFIND} type modules or $G(y)$
  for \wtt{SUNNONLINEARSOLVER\_STATIONARY} type modules. Memory
  for \id{F} should already be allocated prior to calling this
  function. The vector \id{y} should be left unchanged.
}
{
  \begin{args}[mem]
  \item[y]
    is the state vector at which the nonlinear system should be evaluated.
  \item[F]
    is the output vector containing $F(y)$ or $G(y)$, depending on the
    solver type.
  \item[mem]
    is the {\sundials} package memory structure.
  \end{args}
}
{
  The return value \id{retval} (of type \id{int}) will be one of the
  following:
  \begin{args}[*\_RHSFUNC\_RECVR]
  \item[\id{*\_SUCCESS}]
    the function evaluation was successful
  \item[\id{*\_RHSFUNC\_RECVR}]
    the ODE right-hand side function returned a recoverable error
  \item[\id{*\_RES\_RECVR}]
    the DAE residual function returned a recoverable error
  \item[\id{*\_RHSFUNC\_FAIL}]
    the ODE right-hand side function returned an unrecoverable error
  \item[\id{*\_RES\_FAIL}]
    the DAE residual function returned an unrecoverable error
  \item[\id{*\_MEM\_NULL}]
    the {\sundials} package memory was \id{NULL}
  \end{args}
  In the above return codes \id{*} is a {\sundials} package specific
  prefix (\id{CV} for {\cvode} or {\cvodes}, \id{IDA} for {\ida} or
  {\idas}, and \id{ARK} for {\arkode}).
}
{}
% --------------------------------------------------------------------
\usfunction{SUNNonlinSolLSetupFn}
{
  typedef int (*SUNNonlinSolLSetupFn)(N\_Vector y, N\_Vector F, void* mem);
}
{
  These functions are wrappers to the {\sundials} package's function
  for setting up linear solves with {\sunlinsol} modules.
}
{
  \begin{args}[mem]
  \item[y]
    is the state vector at which the linear system should be setup.
  \item[F]
    is the value of the nonlinear system function at \id{y}.
  \item[mem]
    is the {\sundials} package memory structure.
  \end{args}
}
{
  The return value \id{retval} (of type \id{int}) will be one of the
  following:
  \begin{args}[*\_LSETUP\_RECVR]
  \item[\id{*\_SUCCESS}]
    the linear solver setup was successful
  \item[\id{*\_LSETUP\_RECVR}]
    the linear solver setup function returned a recoverable error
  \item[\id{*\_LSETUP\_FAIL}]
    the linear solver setup function returned an unrecoverable error
  \item[\id{*\_MEM\_NULL}]
    the {\sundials} package memory was \id{NULL}
  \end{args}
  In the above return codes \id{*} is a {\sundials} package specific
  prefix (\id{CV} for {\cvode} or {\cvodes}, \id{IDA} for {\ida} or
  {\idas}, and \id{ARK} for {\arkode}).
}
{
  {\sunnonlinsol} modules not utilizing {\sunlinsol} linear solvers
  may ignore these functions and will need to handle any linear system
  setup.
}
% --------------------------------------------------------------------
\usfunction{SUNNonlinSolLSolveFn}
{
  typedef int (*SUNNonlinSolLSolveFn)(N\_Vector y, N\_Vector b, void* mem);
}
{
  These functions are wrappers to the {\sundials} package's function
  for solving linear systems with {\sunlinsol} modules.
}
{
  \begin{args}[mem]
  \item[y]
    is the input vector containing the current nonlinear iteration.
  \item[b]
    contains the right-hand side vector for the linear solve on input
    and the solution to the linear system on output.
  \item[mem]
    is the {\sundials} package memory structure.
  \end{args}
}
{
  The return value \id{retval} (of type \id{int}) will be one of the
  following:
  \begin{args}[*\_LSOLVE\_RECVR]
  \item[\id{*\_SUCCESS}]
    the linear solve was successful
  \item[\id{*\_LSOLVE\_RECVR}]
    the linear solver solve function returned a recoverable error
  \item[\id{*\_LSOLVE\_FAIL}]
    the linear solver solve function returned an unrecoverable error
  \item[\id{*\_MEM\_NULL}]
    the {\sundials} package memory was \id{NULL}
  \end{args}
  In the above return codes \id{*} is a {\sundials} package specific
  prefix (\id{CV} for {\cvode} or {\cvodes}, \id{IDA} for {\ida} or
  {\idas}, and \id{ARK} for {\arkode}).
}
{
  {\sunnonlinsol} modules not utilizing {\sunlinsol} linear solvers
  may ignore these functions and will need to handle solving linear
  systems.
}
% --------------------------------------------------------------------
\usfunction{SUNNonlinSolConvTestFn}
{
  typedef int (*SUNNonlinSolConvTestFn)(&int m, realtype delnrm,\\
                                        &realtype tol, void* mem);
}
{
  These functions are {\sundials} package specific convergence tests of
  nonlinear solvers.
}
{
  \begin{args}[delnrm]
  \item[m]
    is the value of the nonlinear iteration counter, starting from
  zero for the initial iteration.
  \item[delnrm]
    is the WRMS norm of the iteration update vector $\delta^{(m)}$ in
    $y^{(m+1)} = y^{(m)} + \delta^{(m+1)}$.
  \item[tol]
    is the nonlinear solver tolerance.
  \item[mem]
    is the {\sundials} package memory structure.
  \end{args}
}
{
  The return value of this routine will be one of the following: 
  \begin{args}[SUN\_NLS\_CONV\_RECVR]
  \item[\id{SUN\_NLS\_SUCCESS}]
    the iteration is converged.
  \item[\id{SUN\_NLS\_CONTINUE}]
    the iteration has not converged, keep iterating.
  \item[\id{SUN\_NLS\_CONV\_RECVR}]
    the iteration appears to be diverging, try to recover.
  \item[\id{*\_MEM\_NULL}]
    the {\sundials} package memory was \id{NULL}
  \end{args}
  In the above return codes \id{*} is a {\sundials} package specific
  prefix (\id{CV} for {\cvode} or {\cvodes}, \id{IDA} for {\ida} or
  {\idas}, and \id{ARK} for {\arkode}).
}
{
  {\sunnonlinsol} modules utilizing their own convergence criteria may
  ignore these functions.
}


% ====================================================================
\subsection{SUNNonlinearSolver return codes}
\label{ss:sunnonlinsol_returncodes}
% ====================================================================

The functions provided to {\sunnonlinsol} modules and functions within
the {\sundials}-provided {\sunnonlinsol} implementations utilize a
common set of return codes, shown below in Table \ref{t:sunnonlinsol_returncodes}.

\newlength{\ColumnOneA}
\settowidth{\ColumnOneA}{\id{SUN\_NLS\_CONV\_RECVR}}
\newlength{\ColumnTwoA}
\settowidth{\ColumnTwoA}{\id{Value}}
\newlength{\ColumnThreeA}
\setlength{\ColumnThreeA}{\textwidth}
\addtolength{\ColumnThreeA}{-0.5in}
\addtolength{\ColumnThreeA}{-\ColumnOneA}
\addtolength{\ColumnThreeA}{-\ColumnTwoA}

\tablecaption{Description of the \id{SUNNonlinearSolver} return codes}\label{t:sunnonlinsol_returncodes}
\tablehead{\hline {\rule{0mm}{5mm}}{\bf Name} & {\bf Value} & {\bf Description} \\[3mm] \hline\hline}
\tabletail{\hline \multicolumn{3}{|r|}{\small\slshape continued on next page} \\ \hline}
\begin{xtabular}{|p{\ColumnOneA}|p{\ColumnTwoA}|p{\ColumnThreeA}|}
%%
\id{SUN\_NLS\_SUCCESS}     & \id{0}  & successful call or converged solve
\\[1mm]
%%
\id{SUN\_NLS\_CONTINUE}    & \id{1}  & the nonlinear solver is not
                                      converged, keep iterating 
\\[1mm]
%%
\id{SUN\_NLS\_CONV\_RECVR} & \id{2}  & the nonlinear solver appears to
                                       be diverging, try to recover
\\[1mm]
%%
\id{SUN\_NLS\_MEM\_NULL}   & \id{-1} & a memory argument is \id{NULL}
\\[1mm]
%%
\id{SUN\_NLS\_MEM\_FAIL}   & \id{-2} & a memory access or allocation failed
\\[1mm]
%%
\id{SUN\_NLS\_ILL\_INPUT}  & \id{-3} & an illegal input option was provided
\\
\end{xtabular}
\bigskip


% ====================================================================
\subsection{The generic SUNNonlinearSolver module}
\label{ss:sunnonlinsol_generic}
% ====================================================================

{\sundials} packages interact with a specific {\sunnonlinsol}
implementation through the generic {\sunnonlinsol} module on which all
other {\sunnonlinsol} are built. The \wtt{SUNNonlinearSolver} type is
a pointer to a structure containing an implementation-dependent
\textit{content} field and an \textit{ops} field. The type
\wtt{SUNNonlinearSolver} is defined as follows:
%%
%%
\begin{verbatim}
typedef struct _generic_SUNNonlinearSolver *SUNNonlinearSolver;

struct _generic_SUNNonlinearSolver {
  void *content;
  struct _generic_SUNNonlinearSolver_Ops *ops;
};
\end{verbatim}
%%
%%
where the \wtt{\_generic\_SUNLinearSolver\_Ops} structure is a list of 
pointers to the various actual linear solver operations provided by a
specific implementation. The \wtt{\_generic\_SUNLinearSolver\_Ops}
structure is defined as
%%
%%
\begin{verbatim}
struct _generic_SUNNonlinearSolver_Ops {
  SUNNonlinearSolver_Type (*gettype)(SUNNonlinearSolver);
  int                     (*initialize)(SUNNonlinearSolver);
  int                     (*setup)(SUNNonlinearSolver, N_Vector, void*);
  int                     (*solve)(SUNNonlinearSolver, N_Vector, N_Vector,
                                   N_Vector, realtype, booleantype, void*);
  int                     (*free)(SUNNonlinearSolver);
  int                     (*setsysfn)(SUNNonlinearSolver, SUNNonlinSolSysFn);
  int                     (*setlsetupfn)(SUNNonlinearSolver, SUNNonlinSolLSetupFn);
  int                     (*setlsolvefn)(SUNNonlinearSolver, SUNNonlinSolLSolveFn);
  int                     (*setctestfn)(SUNNonlinearSolver, SUNNonlinSolConvTestFn);
  int                     (*setmaxiters)(SUNNonlinearSolver, int);
  int                     (*getnumiters)(SUNNonlinearSolver, long int*);
};
\end{verbatim}
%%
%%
The generic {\sunnonlinsol} module defines and implements the nonlinear
solver operations defined in Sections \ref{ss:sunnonlinsol_corefn}
-- \ref{ss:sunnonlinsol_getfn}. These routines are in fact only
wrappers to the nonlinear solver operations provided by a particular
{\sunnonlinsol} implementation, which are accessed through the ops
field of the \id{SUNNonlinearSolver} structure. To illustrate this
point we show below the implementation of a typical nonlinear solver
operation from the generic \{sunnonlinsol} module,
namely \wtt{SUNNonlinSolInitialize}, which initializes a
{\sunnonlinsol} object for use after it has been created and
configured, and returns a flag denoting a successful/failed operation:
%%
%%
\begin{verbatim}
int SUNNonlinSolInitialize(SUNNonlinearSolver NLS)
{
  return ((int) NLS->ops->initialize(NLS));
}
\end{verbatim}


% ====================================================================
\subsection{Implementing a SUNNonlinearSolver Module}
\label{ss:sunnonlinsol_custom}
% ====================================================================

A {\sunnonlinsol} implementation \textit{must} do the following:
\begin{enumerate}
\item Specify the content of the {\sunnonlinsol} module.
\item Define and implement the required nonlinear solver operations
  defined in Sections \ref{ss:sunnonlinsol_corefn}
  -- \ref{ss:sunnonlinsol_getfn}. Note that the names of the module
  routines should be unique to that implementation in order to permit
  using more than one {\sunlinsol} module (each with different
  \wtt{SUNLinearSolver} internal data representations) in
  the same code.
\item Define and implement a user-callable constructor to create a
  \wtt{SUNNonlinearSolver} object.
\end{enumerate}
Additionally, a {\sunnonlinsol} implementation \textit{may} do the
following:
\begin{enumerate}
\item Define and implement additional user-callable routines acting on
  a newly created \wtt{SUNNonlinearSolver} object that are not defined
  in the {\sunnonlinsol} API. For example, routines to set various
  configuration options for tuning the performance of the nonlinear
  solver.
\item Provide functions as needed for the particular {\sunnonlinsol}
  implementation to access different parts in the {\em content}
  structure of the newly defined \wtt{SUNLinearSolver} object 
  (e.g., routines to return various statistics from the solver).
\end{enumerate}


% ====================================================================
\section{The SUNNonlinearSolver\_Newton implementation}\label{s:sunnonlinsol_newton}
This section describes the {\sunnonlinsol} implementation of Newton's method. To
access the {\sunnonlinsolnewton} module, include the header file
\id{sunnonlinsol/sunnonlinsol\_newton.h}. We note that the {\sunnonlinsolnewton}
module is accessible from {\sundials} integrators \textit{without} separately
linking to the \id{libsundials\_sunnonlinsolnewton} module library.

% ====================================================================
\subsection{SUNNonlinearSolver\_Newton description}
\label{ss:sunnonlinsolnewton_math}
% ====================================================================

To find the solution to
\begin{equation}\label{e:newton_sys}
  F(y) = 0 \,
\end{equation}
given an initial guess $y^{(0)}$, Newton's method computes a series of
approximate solutions
\begin{equation}
  y^{(m+1)} = y^{(m)} + \delta^{(m+1)}
\end{equation}
where $m$ is the Newton iteration index, and the Newton update $\delta^{(m+1)}$
is the solution of the linear system
\begin{equation}\label{e:newton_linsys}
  A(y^{(m)}) \delta^{(m+1)} = -F(y^{(m)}) \, ,
\end{equation}
in which $A$ is the Jacobian matrix
\begin{equation}\label{e:newton_mat}
  A \equiv \partial F / \partial y \, .
\end{equation}
Depending on the linear solver used, the {\sunnonlinsolnewton} module
will employ either a Modified Newton method, or an Inexact Newton
method~\cite{Bro:87,BrSa:90,DES:82,DeSc:96,Kel:95}. When used with a direct
linear solver, the Jacobian matrix $A$ is held constant during the Newton
iteration, resulting in a Modified Newton method. With a matrix-free iterative
linear solver, the iteration is an Inexact Newton method.

In both cases, calls to the integrator-supplied \id{SUNNonlinSolLSetupFn}
function are made infrequently to amortize the increased cost of
matrix operations (updating $A$ and its factorization within direct
linear solvers, or updating the preconditioner within iterative linear
solvers).  Specifically, {\sunnonlinsolnewton} will call the
\id{SUNNonlinSolLSetupFn} function in two instances:
\begin{itemize}
\item[(a)] when requested by the integrator (the input
  \id{callLSetSetup} is \id{SUNTRUE}) before attempting the Newton
  iteration, or
\item[(b)] when reattempting the nonlinear solve after a recoverable
  failure occurs in the Newton iteration with stale Jacobian
  information (\id{jcur} is \id{SUNFALSE}).  In this case,
  {\sunnonlinsolnewton} will set \id{jbad} to \id{SUNTRUE} before
  calling the \id{SUNNonlinSolLSetupFn} function.
\end{itemize}
Whether the Jacobian matrix $A$ is fully or partially updated depends
on logic unique to each integrator-supplied \id{SUNNonlinSolSetupFn}
routine. We refer to the discussion of nonlinear solver strategies
provided in Chapter \ref{s:math} for details on this decision.

The default maximum number of iterations and the stopping criteria for
the Newton iteration are supplied by the {\sundials} integrator when
{\sunnonlinsolnewton} is attached to it.  Both the maximum number of
iterations and the convergence test function may be modified by the
user by calling the \id{SUNNonlinSolSetMaxIters} and/or
\id{SUNNonlinSolSetConvTestFn} functions after attaching the
{\sunnonlinsolnewton} object to the integrator.

% ====================================================================
\subsection{SUNNonlinearSolver\_Newton functions}
\label{ss:sunnonlinsolnewton_functions}
% ====================================================================

The {\sunnonlinsolnewton} module provides the following constructors
for creating a \\ \noindent
% --------------------------------------------------------------------
\id{SUNNonlinearSolver} object.

\ucfunction{SUNNonlinSol\_Newton}
{
  NLS = SUNNonlinSol\_Newton(y);
}
{
  The function \ID{SUNNonlinSol\_Newton} creates a
  \id{SUNNonlinearSolver} object for use with {\sundials} integrators to
  solve nonlinear systems of the form $F(y) = 0$ using Newton's method.
}
{
  \begin{args}[y]
  \item[y] (\id{N\_Vector})
    a template for cloning vectors needed within the solver.
  \end{args}
}
{
  The return value \id{NLS} (of type \id{SUNNonlinearSolver}) will be
  a {\sunnonlinsol} object if the constructor exits successfully,
  otherwise \id{NLS} will be \id{NULL}.
}
{}
% --------------------------------------------------------------------
\ucfunction{SUNNonlinSol\_NewtonSens}
{
  NLS = SUNNonlinSol\_NewtonSens(count, y);
}
{
  The function \ID{SUNNonlinSol\_NewtonSens} creates a
  \id{SUNNonlinearSolver} object for use with {\sundials} sensitivity enabled
  integrators ({\cvodes} and {\idas}) to solve nonlinear systems of the form
  $F(y) = 0$ using Newton's method.
}
{
  \begin{args}[count]
  \item[count] (\id{int})
    the number of vectors in the nonlinear solve. When integrating a system
    containing \id{Ns} sensitivities the value of \id{count} is:
    \begin{itemize}
      \item \id{Ns+1} if using a \textit{simultaneous} corrector approach.
      \item \id{Ns} if using a \textit{staggered} corrector approach.
    \end{itemize}
  \item[y] (\id{N\_Vector})
    a template for cloning vectors needed within the solver.
  \end{args}
}
{
  The return value \id{NLS} (of type \id{SUNNonlinearSolver}) will be
  a {\sunnonlinsol} object if the constructor exits successfully,
  otherwise \id{NLS} will be \id{NULL}.
}
{}
% --------------------------------------------------------------------
The {\sunnonlinsolnewton} module implements all of the functions
defined in sections \ref{ss:sunnonlinsol_corefn} --
\ref{ss:sunnonlinsol_getfn} except for the \id{SUNNonlinSolSetup} function. The
{\sunnonlinsolnewton} functions have the same names as those defined
by the generic {\sunnonlinsol} API with \id{\_Newton} appended to the
function name. Unless using the {\sunnonlinsolnewton} module as a
standalone nonlinear solver the generic functions defined in sections
\ref{ss:sunnonlinsol_corefn} -- \ref{ss:sunnonlinsol_getfn} should be
called in favor of the {\sunnonlinsolnewton}-specific implementations.

The {\sunnonlinsolnewton} module also defines the following additional
user-callable function.
% --------------------------------------------------------------------
\ucfunction{SUNNonlinSolGetSysFn\_Newton}
{
  retval = SUNNonlinSolGetSysFn\_Newton(NLS, SysFn);
}
{
  The function \ID{SUNNonlinSolGetSysFn\_Newton} returns the residual function
  that defines the nonlinear system.
}
{
  \begin{args}[SysFn]
  \item[NLS] (\id{SUNNonlinearSolver})
    a {\sunnonlinsol} object
  \item[SysFn] (\id{SUNNonlinSolSysFn*})
    the function defining the nonlinear system.
  \end{args}
}
{
  The return value \id{retval} (of type \id{int}) should be zero for a
  successful call, and a negative value for a failure.
}
{
  This function is intended for users that wish to evaluate the
  nonlinear residual in a custom convergence test function for the
  {\sunnonlinsolnewton} module.  We note that {\sunnonlinsolnewton}
  will not leverage the results from any user calls to \id{SysFn}.
}


% ====================================================================
\subsection{SUNNonlinearSolver\_Newton content}
\label{ss:sunnonlinsolnewton_content}
% ====================================================================

The \textit{content} field of the {\sunnonlinsolnewton} module is the
following structure.
%%
%%
\begin{verbatim}
struct _SUNNonlinearSolverContent_Newton {

  SUNNonlinSolSysFn      Sys;
  SUNNonlinSolLSetupFn   LSetup;
  SUNNonlinSolLSolveFn   LSolve;
  SUNNonlinSolConvTestFn CTest;

  N_Vector    delta;
  booleantype jcur;
  int         curiter;
  int         maxiters;
  long int    niters;
};
\end{verbatim}
%%
%%
These entries of the \emph{content} field contain the following
information:
\begin{args}[maxiters]
  \item[Sys]      - the function for evaluating the nonlinear system,
  \item[LSetup]   - the package-supplied function for setting up the linear solver,
  \item[LSolve]   - the package-supplied function for performing a linear solve,
  \item[CTest]    - the function for checking convergence of the Newton
                    iteration,
  \item[delta]    - the Newton iteration update vector,
  \item[jcur]     - the Jacobian status (\id{SUNTRUE} = current,
                    \id{SUNFALSE} = stale),
  \item[curiter]  - the current number of iterations in the solve attempt,
  \item[maxiters] - the maximum number of Newton iterations allowed in
                    a solve, and
  \item[niters]   - the total number of nonlinear iterations across all
                    solves.
\end{args}


% ====================================================================
\subsection{SUNNonlinearSolver\_Newton Fortran interface}
\label{ss:sunnonlinsolnewton_fortran}
% ====================================================================

For {\sundials} integrators that include a Fortran interface, the
{\sunnonlinsolnewton} module also includes a Fortran-callable
function for creating a \id{SUNNonlinearSolver} object.
\ucfunction{FSUNNEWTONINIT}
{
  FSUNNEWTONINIT(code, ier);
}
{
  The function \ID{FSUNNEWTONINIT} can be called for Fortran programs
  to create a\\
  \id{SUNNonlinearSolver} object for use with {\sundials}
  integrators to solve nonlinear systems of the form $F(y) = 0$ with
  Newton's method.
}
{
  \begin{args}[code]
  \item[code] (\id{int*})
    is an integer input specifying the solver id (1 for {\cvode}, 2
    for {\ida}, 3 for {\kinsol}, and 4 for {\arkode}).
  \end{args}
}
{
  \id{ier} is a return completion flag equal to \id{0} for a success
  return and \id{-1} otherwise. See printed message for details in case
  of failure.
}
{}

% ====================================================================

% LocalWords:  API


%---------------------------------------------------------------------------
\section{CVODES SUNNonlinearSolver interface}
\label{s:sunnonlinsol_interface}
%---------------------------------------------------------------------------

As discussed in Chapter \ref{s:math} each integration step requires the
(approximate) solution of a nonlinear system. This system can be formulated as
the rootfinding problem
\begin{equation}
  F(y^n) \equiv y^n - h_n \beta_{n,0} f(t_n,y^n) - a_n = 0 \, ,
\end{equation}
or as the fixed-point problem
\begin{equation}
  G(y^n) \equiv h_n \beta_{n,0} f(t_n,y^n) + a_n = y^n \, ,
\end{equation}
where $a_n\equiv\sum_{i>0}(\alpha_{n,i}y^{n-i}+h_n\beta_{n,i} {\dot{y}}^{n-i})$.

Rather than solving the above nonlinear systems for the new state $y^n$
{\cvodes} reformulates the above problems to solve for the correction $y_{cor}$
to the predicted new state $y_{pred}$ so that $y^n = y_{pred} + y_{cor}$.
The nonlinear systems rewritten in terms of $y_{cor}$ are
\begin{equation} \label{eq:res_corrector}
  F(y_{cor}) \equiv y_{cor} - \gamma f(t_n, y^n) - \tilde{a}_n = 0 \, ,
\end{equation}
for the rootfinding problem and
\begin{equation} \label{eq:fp_corrector}
  G(y_{cor}) \equiv \gamma f(t_n, y^n) + \tilde{a}_n = y_{cor} \, .
\end{equation}
for the fixed-point problem.
Similarly in the forward sensitivity analysis case the combined state and
sensitivity nonlinear systems are also reformulated in terms of the correction
to the predicted state and sensitivities.

The nonlinear system functions provided by {\cvodes} to the nonlinear solver
module internally update the current value of the new state (and the
sensitvities) based on the input correction vector(s) i.e.,
$y^n = y_{pred} + y_{cor}$ and $s_i^n = s_{i,pred} + s_{i,cor}$. The updated
vector(s) are used when calling the right-hand side function and when setting
up linear solves (e.g., updating the Jacobian or preconditioner).

{\cvodes} provides several advanced functions that will not be needed by most
users, but might be useful for users who choose to provide their own
implementation of the \id{SUNNonlinearSolver} API. For example, such a user
might need access to the current value of $\gamma$ to compute Jacobian data.

\ucfunctionf{CVodeGetCurrentGamma}
{
  flag = CVodeGetCurrentGamma(cvode\_mem, \&gamma);
}
{
  The function \ID{CVodeGetCurrentGamma} returns the current
  value of the scalar $\gamma$.
}
{
  \begin{args}[cvode\_mem]
  \item[cvode\_mem] (\id{void *})
    pointer to the {\cvodes} memory block.
  \item[gamma] (\id{realtype *})
      the current value of the scalar $\gamma$ appearing in the
      Newton equation $M = I - \gamma J$.
  \end{args}
}
{
  The return value \id{flag} (of type \id{int}) is one of
  \begin{args}[CV\_MEM\_NULL]
  \item[CV\_SUCCESS]
    The optional output value has been successfully set.
  \item[CV\_MEM\_NULL]
    The \id{cvode\_mem} pointer is \id{NULL}.
  \end{args}
}
{}
%%
%%
\ucfunctionf{CVodeGetCurrentState}
{
  flag = CVodeGetCurrentState(cvode\_mem, \&y);
}
{
  The function \ID{CVodeGetCurrentState} returns the current state vector. When
  called within the computation of a step (i.e., during a nonlinear solve) this
  is $y^n = y_{pred} + y_{cor}$. Otherwise this is the current internal solution
  vector $y(t)$. In either case the corresponding solution time can be obtained
  from \id{CVodeGetCurrentTime}.
}
{
  \begin{args}[cvode\_mem]
  \item[cvode\_mem] (\id{void *})
    pointer to the {\cvodes} memory block.
  \item[y] (\id{N\_Vector *})
    pointer that is set to the current state vector
  \end{args}
}
{
  The return value \id{flag} (of type \id{int}) is one of
  \begin{args}[CV\_MEM\_NULL]
  \item[CV\_SUCCESS]
    The optional output value has been successfully set.
  \item[CV\_MEM\_NULL]
    The \id{cvode\_mem} pointer is \id{NULL}.
  \end{args}
}
{}
%%
%%
\ucfunctionf{CVodeGetCurrentStateSens}
{
  flag = CVodeGetCurrentStateSens(cvode\_mem, \&yS);
}
{
  The function \ID{CVodeGetCurrentStateSens} returns the current sensitivity
  state vector array.
}
{
  \begin{args}[cvode\_mem]
  \item[cvode\_mem] (\id{void *})
    pointer to the {\cvodes} memory block.
  \item[yS] (\id{N\_Vector **})
    pointer to the vector array that is set to the current sensitivity state
    vector array
  \end{args}
}
{
  The return value \id{flag} (of type \id{int}) is one of
  \begin{args}[CV\_MEM\_NULL]
  \item[CV\_SUCCESS]
    The optional output value has been successfully set.
  \item[CV\_MEM\_NULL]
    The \id{cvode\_mem} pointer is \id{NULL}.
  \end{args}
}
{}
%%
%%
\ucfunctionf{CVodeGetCurrentSensSolveIndex}
{
  flag = CVodeGetCurrentSensSolveIndex(cvode\_mem, \&index);
}
{
  The function \ID{CVodeGetCurrentSensSolveIndex} returns the index of the
  current sensitivity solve when using the \id{CV\_STAGGERED1} solver.
}
{
  \begin{args}[cvode\_mem]
  \item[cvode\_mem] (\id{void *})
    pointer to the {\cvodes} memory block.
  \item[index] (\id{int *})
    will be set to the index of the current sensitivity solve
  \end{args}
}
{
  The return value \id{flag} (of type \id{int}) is one of
  \begin{args}[CV\_MEM\_NULL]
  \item[CV\_SUCCESS]
    The optional output value has been successfully set.
  \item[CV\_MEM\_NULL]
    The \id{cvode\_mem} pointer is \id{NULL}.
  \end{args}
}
{}
%%
%%
\ucfunctionf{CVodeGetNonlinearSystemData}
{
  flag = CVodeGetNonlinearSystemData(&cvode\_mem, \&tn, \&ypred, \&yn, \&fn,\\
                                     &\&gamma, \&rl1, \&zn1, \&user\_data);
}
{
  The function \ID{CVodeGetNonlinearSystemData} returns all internal
  data required to construct the current nonlinear system
  \eqref{eq:res_corrector} or \eqref{eq:fp_corrector}.
}
{
  \begin{args}[cvode\_mem]
  \item[cvode\_mem] (\id{void *}) pointer to the {\cvodes} memory block.
  \item[tn] (\id{realtype*}) current value of the independent variable $t_n$.
  \item[ypred] (\id{N\_Vector*}) predicted state vector $y_{pred}$ at $t_n$.
    This vector must not be changed.
  \item[yn] (\id{N\_Vector*}) state vector $y^n$. This vector may be
    not current and may need to be filled (see the note below).
  \item[fn] (\id{N\_Vector*}) the right-hand side function evaluated at the
    current time and state, $f(t_n, y^n)$. This vector may be
    not current and may need to be filled (see the note below).
  \item[gamma] (\id{realtype*}) current value of $\gamma$.
  \item[rl1] (\id{realtype*}) a scaling factor used to compute $\tilde{a}_n = $
    \id{rl1 * zn1}.
  \item[zn1] (\id{N\_Vector*}) a vector used to compute $\tilde{a}_n = $
    \id{rl1 * zn1}.
  \item[user\_data] (\id{void**}) pointer to the user-defined data structures
  \end{args}
}
{
  The return value \id{flag} (of type \id{int}) is one of
  \begin{args}[CV\_MEM\_NULL]
  \item[CV\_SUCCESS]
    The optional output values have been successfully set.
  \item[CV\_MEM\_NULL]
    The \id{cvode\_mem} pointer is \id{NULL}.
  \end{args}
}
{
  This routine is intended for users who whish to attach a custom
  \id{SUNNonlinSolSysFn} (see \S\ref{ss:sunnonlinsol_sunsuppliedfn}) to an
  existing \id{SUNNonlinearSolver} object (through a call to
  \id{SUNNonlinSolSetSysFn}) or who need access to nonlinear system data to
  compute the nonlinear system fucntion as part of a custom
  \id{SUNNonlinearSolver} object.

  When supplying a custom \id{SUNNonlinSolSysFn} to an existing
  \id{SUNNonlinearSolver} object, the user should call
  \id{CVodeGetNonlinearSystemData} \textbf{inside} the nonlinear system
  function to access the requisite data for evaluting the nonlinear system
  function of their choosing. Additionlly, if the \id{SUNNonlinearSolver} object
  (existing or custom) leverages the \id{SUNNonlinSolLSetupFn} and/or
  \id{SUNNonlinSolLSolveFn} functions supplied by {\cvodes} (through calls to
  \id{SUNNonlinSolSetLSetupFn} and \id{SUNNonlinSolSetLSolveFn} respectively)
  the vectors \id{yn} and \id{fn} \textbf{must be filled} in by the user's
  \id{SUNNonlinSolSysFn} with the current state and corresponding evaluation of
  the right-hand side function respectively i.e.,
  \begin{align*}
    yn &= y_{pred} + y_{cor}, \\
    fn &= f\left(t_{n}, y^n\right),
  \end{align*}
  where $y_{cor}$ was the first argument supplied to the \id{SUNNonlinSolSysFn}.

  If this function is called as part of a custom linear solver (i.e., the
  default \id{SUNNonlinSolSysFn} is used) then the vectors \id{yn} and \id{fn}
  are only current when \id{CVodeGetNonlinearSystemData} is called after an
  evaluation of the nonlinear system function.
}
%%
%%
\ucfunctionf{CVodeGetNonlinearSystemDataSens}
{
  flag = CVodeGetNonlinearSystemDataSens(&cvode\_mem, \&tn, \&ySpred, \&ySn,\\
                                         &\&gamma, \&rlS1, \&znS1, \&user\_data);
}
{
  The function \ID{CVodeGetNonlinearSystemDataSens} returns all internal
  sensitivity data required to construct the current nonlinear system
  \eqref{eq:res_corrector} or \eqref{eq:fp_corrector}.
}
{
  \begin{args}[cvode\_mem]
  \item[cvode\_mem] (\id{void *}) pointer to the {\cvodes} memory block.
  \item[tn] (\id{realtype*}) current value of the independent variable $t_n$.
  \item[ySpred] (\id{N\_Vector**}) predicted state vectors $yS_{i,pred}$ at
    $t_n$ for $i = 0 \dots N_s - 1$.
    This vector must not be changed.
  \item[ySn] (\id{N\_Vector**}) state vectors $yS_i^n$ for
    $i = 0 \dots N_s - 1$. These vectors may be not current (see the note
    below).
  \item[gamma] (\id{realtype*}) current value of $\gamma$.
  \item[rlS1] (\id{realtype*}) a scaling factor used to compute $\tilde{a}S_n = $
    \id{rlS1 * znS1}.
  \item[znS1] (\id{N\_Vector**}) a vectors used to compute $\tilde{a}S_{i,n} = $
    \id{rlS1 * znS1}.
  \item[user\_data] (\id{void**}) pointer to the user-defined data structures
  \end{args}
}
{
  The return value \id{flag} (of type \id{int}) is one of
  \begin{args}[CV\_MEM\_NULL]
  \item[CV\_SUCCESS]
    The optional output values have been successfully set.
  \item[CV\_MEM\_NULL]
    The \id{cvode\_mem} pointer is \id{NULL}.
  \end{args}
}
{
  This routine is intended for users who whish to attach a custom
  \id{SUNNonlinSolSysFn} (see \S\ref{ss:sunnonlinsol_sunsuppliedfn}) to an
  existing \id{SUNNonlinearSolver} object (through a call to
  \id{SUNNonlinSolSetSysFn}) or who need access to nonlinear system data to
  compute the nonlinear system fucntion as part of a custom
  \id{SUNNonlinearSolver} object.

  When supplying a custom \id{SUNNonlinSolSysFn} to an existing
  \id{SUNNonlinearSolver} object, the user should call
  \id{CVodeGetNonlinearSystemDataSens} \textbf{inside} the nonlinear system
  function used in the sensitivity nonlinear solve to access the requisite data
  for evaluting the nonlinear system function of their choosing. This could be
  the same function used for solving for the new state (the simultaneous
  approach) or a different function (the staggered or stagggered1 approaches).
  Additionlly, the vectors \id{ySn} are only provided as additional
  worksapce and do not need to be filled in by the user's
  \id{SUNNonlinSolSysFn}.

  If this function is called as part of a custom linear solver (i.e., the
  default \id{SUNNonlinSolSysFn} is used) then the vectors \id{ySn}
  are only current when \id{CVodeGetNonlinearSystemDataSens} is called after an
  evaluation of the nonlinear system function.
}
%%
%%
\ucfunctionf{CVodeComputeState}
{
  flag = CVodeComputeState(cvode\_mem, ycor, yn);
}
{
  The function computes the current $y(t)$ vector based on stored prediction and
  the given correction vector from the nonlinear solver i.e.,
  $y^n = y_{pred} + y_{cor}$.
}
{
  \begin{args}[cvode\_mem]
    \item[cvode\_mem] (\id{void *}) pointer to the {\cvodes} memory block
    \item[ycor] (\id{N\_Vector}) the correction
    \item[yn] (\id{N\_Vector}) the output vector
  \end{args}
}
{
  The return value \id{flag} (of type \id{int}) is one of
  \begin{args}[CVODE\_MEM\_NULL]
  \item[CV\_SUCCESS]
    The optional output value has been successfully set.
  \item[CV\_MEM\_NULL]
    The \id{cvode\_mem} pointer is \id{NULL}.
  \end{args}
}
{}
%%
%%
\ucfunctionf{CVodeComputeStateSens}
{
  flag = CVodeComputeStateSens(cvode\_mem, yScor, ySn);
}
{
  The function computes the current sensitivity vector $yS(t)$ for all
  sensitivities based on stored prediction and the given correction vector from
  the nonlinear solver i.e., $yS^n = yS_{pred} + yS_{cor}$.
}
{
  \begin{args}[cvode\_mem]
    \item[cvode\_mem] (\id{void *}) pointer to the {\cvodes} memory block
    \item[yScor] (\id{N\_Vector*}) the correction
    \item[ySn] (\id{N\_Vector*}) the output vector
  \end{args}
}
{
  The return value \id{flag} (of type \id{int}) is one of
  \begin{args}[CVODE\_MEM\_NULL]
  \item[CV\_SUCCESS]
    The optional output value has been successfully set.
  \item[CV\_MEM\_NULL]
    The \id{cvode\_mem} pointer is \id{NULL}.
  \end{args}
}
{}
%%
%%
\ucfunctionf{CVodeComputeStateSens1}
{
  flag = CVodeComputeStateSens1(cvode\_mem, idx, yScor1, ySn1);
}
{
  The function computes the current sensitivity vector $yS_i(t)$ for the
  sensitivity at the given index based on stored prediction and the given
  correction vector from the nonlinear solver i.e.,
  $yS_i^n = yS_{i,pred} + yS_{i,cor}$.
}
{
  \begin{args}[cvode\_mem]
    \item[cvode\_mem] (\id{void *}) pointer to the {\cvodes} memory block
    \item[index] (\id{int}) the index of the sensitivity to update
    \item[yScor] (\id{N\_Vector}) the correction
    \item[ySn] (\id{N\_Vector}) the output vector
  \end{args}
}
{
  The return value \id{flag} (of type \id{int}) is one of
  \begin{args}[CVODE\_MEM\_NULL]
  \item[CV\_SUCCESS]
    The optional output value has been successfully set.
  \item[CV\_MEM\_NULL]
    The \id{cvode\_mem} pointer is \id{NULL}.
  \end{args}
}
{}

%---------------------------------------------------------------------------
% sunnonlinsol module sections
%---------------------------------------------------------------------------

This section describes the {\sunnonlinsol} implementation of Newton's method. To
access the {\sunnonlinsolnewton} module, include the header file
\id{sunnonlinsol/sunnonlinsol\_newton.h}. We note that the {\sunnonlinsolnewton}
module is accessible from {\sundials} integrators \textit{without} separately
linking to the \id{libsundials\_sunnonlinsolnewton} module library.

% ====================================================================
\subsection{SUNNonlinearSolver\_Newton description}
\label{ss:sunnonlinsolnewton_math}
% ====================================================================

To find the solution to
\begin{equation}\label{e:newton_sys}
  F(y) = 0 \,
\end{equation}
given an initial guess $y^{(0)}$, Newton's method computes a series of
approximate solutions
\begin{equation}
  y^{(m+1)} = y^{(m)} + \delta^{(m+1)}
\end{equation}
where $m$ is the Newton iteration index, and the Newton update $\delta^{(m+1)}$
is the solution of the linear system
\begin{equation}\label{e:newton_linsys}
  A(y^{(m)}) \delta^{(m+1)} = -F(y^{(m)}) \, ,
\end{equation}
in which $A$ is the Jacobian matrix
\begin{equation}\label{e:newton_mat}
  A \equiv \partial F / \partial y \, .
\end{equation}
Depending on the linear solver used, the {\sunnonlinsolnewton} module
will employ either a Modified Newton method, or an Inexact Newton
method~\cite{Bro:87,BrSa:90,DES:82,DeSc:96,Kel:95}. When used with a direct
linear solver, the Jacobian matrix $A$ is held constant during the Newton
iteration, resulting in a Modified Newton method. With a matrix-free iterative
linear solver, the iteration is an Inexact Newton method.

In both cases, calls to the integrator-supplied \id{SUNNonlinSolLSetupFn}
function are made infrequently to amortize the increased cost of
matrix operations (updating $A$ and its factorization within direct
linear solvers, or updating the preconditioner within iterative linear
solvers).  Specifically, {\sunnonlinsolnewton} will call the
\id{SUNNonlinSolLSetupFn} function in two instances:
\begin{itemize}
\item[(a)] when requested by the integrator (the input
  \id{callLSetSetup} is \id{SUNTRUE}) before attempting the Newton
  iteration, or
\item[(b)] when reattempting the nonlinear solve after a recoverable
  failure occurs in the Newton iteration with stale Jacobian
  information (\id{jcur} is \id{SUNFALSE}).  In this case,
  {\sunnonlinsolnewton} will set \id{jbad} to \id{SUNTRUE} before
  calling the \id{SUNNonlinSolLSetupFn} function.
\end{itemize}
Whether the Jacobian matrix $A$ is fully or partially updated depends
on logic unique to each integrator-supplied \id{SUNNonlinSolSetupFn}
routine. We refer to the discussion of nonlinear solver strategies
provided in Chapter \ref{s:math} for details on this decision.

The default maximum number of iterations and the stopping criteria for
the Newton iteration are supplied by the {\sundials} integrator when
{\sunnonlinsolnewton} is attached to it.  Both the maximum number of
iterations and the convergence test function may be modified by the
user by calling the \id{SUNNonlinSolSetMaxIters} and/or
\id{SUNNonlinSolSetConvTestFn} functions after attaching the
{\sunnonlinsolnewton} object to the integrator.

% ====================================================================
\subsection{SUNNonlinearSolver\_Newton functions}
\label{ss:sunnonlinsolnewton_functions}
% ====================================================================

The {\sunnonlinsolnewton} module provides the following constructors
for creating a \\ \noindent
% --------------------------------------------------------------------
\id{SUNNonlinearSolver} object.

\ucfunction{SUNNonlinSol\_Newton}
{
  NLS = SUNNonlinSol\_Newton(y);
}
{
  The function \ID{SUNNonlinSol\_Newton} creates a
  \id{SUNNonlinearSolver} object for use with {\sundials} integrators to
  solve nonlinear systems of the form $F(y) = 0$ using Newton's method.
}
{
  \begin{args}[y]
  \item[y] (\id{N\_Vector})
    a template for cloning vectors needed within the solver.
  \end{args}
}
{
  The return value \id{NLS} (of type \id{SUNNonlinearSolver}) will be
  a {\sunnonlinsol} object if the constructor exits successfully,
  otherwise \id{NLS} will be \id{NULL}.
}
{}
% --------------------------------------------------------------------
\ucfunction{SUNNonlinSol\_NewtonSens}
{
  NLS = SUNNonlinSol\_NewtonSens(count, y);
}
{
  The function \ID{SUNNonlinSol\_NewtonSens} creates a
  \id{SUNNonlinearSolver} object for use with {\sundials} sensitivity enabled
  integrators ({\cvodes} and {\idas}) to solve nonlinear systems of the form
  $F(y) = 0$ using Newton's method.
}
{
  \begin{args}[count]
  \item[count] (\id{int})
    the number of vectors in the nonlinear solve. When integrating a system
    containing \id{Ns} sensitivities the value of \id{count} is:
    \begin{itemize}
      \item \id{Ns+1} if using a \textit{simultaneous} corrector approach.
      \item \id{Ns} if using a \textit{staggered} corrector approach.
    \end{itemize}
  \item[y] (\id{N\_Vector})
    a template for cloning vectors needed within the solver.
  \end{args}
}
{
  The return value \id{NLS} (of type \id{SUNNonlinearSolver}) will be
  a {\sunnonlinsol} object if the constructor exits successfully,
  otherwise \id{NLS} will be \id{NULL}.
}
{}
% --------------------------------------------------------------------
The {\sunnonlinsolnewton} module implements all of the functions
defined in sections \ref{ss:sunnonlinsol_corefn} --
\ref{ss:sunnonlinsol_getfn} except for the \id{SUNNonlinSolSetup} function. The
{\sunnonlinsolnewton} functions have the same names as those defined
by the generic {\sunnonlinsol} API with \id{\_Newton} appended to the
function name. Unless using the {\sunnonlinsolnewton} module as a
standalone nonlinear solver the generic functions defined in sections
\ref{ss:sunnonlinsol_corefn} -- \ref{ss:sunnonlinsol_getfn} should be
called in favor of the {\sunnonlinsolnewton}-specific implementations.

The {\sunnonlinsolnewton} module also defines the following additional
user-callable function.
% --------------------------------------------------------------------
\ucfunction{SUNNonlinSolGetSysFn\_Newton}
{
  retval = SUNNonlinSolGetSysFn\_Newton(NLS, SysFn);
}
{
  The function \ID{SUNNonlinSolGetSysFn\_Newton} returns the residual function
  that defines the nonlinear system.
}
{
  \begin{args}[SysFn]
  \item[NLS] (\id{SUNNonlinearSolver})
    a {\sunnonlinsol} object
  \item[SysFn] (\id{SUNNonlinSolSysFn*})
    the function defining the nonlinear system.
  \end{args}
}
{
  The return value \id{retval} (of type \id{int}) should be zero for a
  successful call, and a negative value for a failure.
}
{
  This function is intended for users that wish to evaluate the
  nonlinear residual in a custom convergence test function for the
  {\sunnonlinsolnewton} module.  We note that {\sunnonlinsolnewton}
  will not leverage the results from any user calls to \id{SysFn}.
}


% ====================================================================
\subsection{SUNNonlinearSolver\_Newton content}
\label{ss:sunnonlinsolnewton_content}
% ====================================================================

The \textit{content} field of the {\sunnonlinsolnewton} module is the
following structure.
%%
%%
\begin{verbatim}
struct _SUNNonlinearSolverContent_Newton {

  SUNNonlinSolSysFn      Sys;
  SUNNonlinSolLSetupFn   LSetup;
  SUNNonlinSolLSolveFn   LSolve;
  SUNNonlinSolConvTestFn CTest;

  N_Vector    delta;
  booleantype jcur;
  int         curiter;
  int         maxiters;
  long int    niters;
};
\end{verbatim}
%%
%%
These entries of the \emph{content} field contain the following
information:
\begin{args}[maxiters]
  \item[Sys]      - the function for evaluating the nonlinear system,
  \item[LSetup]   - the package-supplied function for setting up the linear solver,
  \item[LSolve]   - the package-supplied function for performing a linear solve,
  \item[CTest]    - the function for checking convergence of the Newton
                    iteration,
  \item[delta]    - the Newton iteration update vector,
  \item[jcur]     - the Jacobian status (\id{SUNTRUE} = current,
                    \id{SUNFALSE} = stale),
  \item[curiter]  - the current number of iterations in the solve attempt,
  \item[maxiters] - the maximum number of Newton iterations allowed in
                    a solve, and
  \item[niters]   - the total number of nonlinear iterations across all
                    solves.
\end{args}


% ====================================================================
\subsection{SUNNonlinearSolver\_Newton Fortran interface}
\label{ss:sunnonlinsolnewton_fortran}
% ====================================================================

For {\sundials} integrators that include a Fortran interface, the
{\sunnonlinsolnewton} module also includes a Fortran-callable
function for creating a \id{SUNNonlinearSolver} object.
\ucfunction{FSUNNEWTONINIT}
{
  FSUNNEWTONINIT(code, ier);
}
{
  The function \ID{FSUNNEWTONINIT} can be called for Fortran programs
  to create a\\
  \id{SUNNonlinearSolver} object for use with {\sundials}
  integrators to solve nonlinear systems of the form $F(y) = 0$ with
  Newton's method.
}
{
  \begin{args}[code]
  \item[code] (\id{int*})
    is an integer input specifying the solver id (1 for {\cvode}, 2
    for {\ida}, 3 for {\kinsol}, and 4 for {\arkode}).
  \end{args}
}
{
  \id{ier} is a return completion flag equal to \id{0} for a success
  return and \id{-1} otherwise. See printed message for details in case
  of failure.
}
{}

This section describes the {\sunnonlinsol} implementation of a fixed point
(functional) iteration with optional Anderson acceleration. To access the
{\sunnonlinsolfixedpoint} module, include the header file
\id{sunnonlinsol/sunnonlinsol\_fixedpoint.h}. We note that the
{\sunnonlinsolfixedpoint} module is accessible from {\sundials} integrators
\textit{without} separately linking to the\\
\noindent\id{libsundials\_sunnonlinsolfixedpoint} module library. 

% ====================================================================
\subsection{SUNNonlinearSolver\_FixedPoint description}
\label{ss:sunnonlinsolfixedpoint_math}
% ====================================================================

To find the solution to
\begin{equation}\label{e:fixed_point_sys}
  G(y) = y \, 
\end{equation}
given an initial guess $y^{(0)}$, the fixed point iteration computes a series of
approximate solutions
\begin{equation}\label{e:fixed_point_iteration}
  y^{(n+1)} = G(y^{(n)})
\end{equation}
where $n$ is the iteration index. The convergence of this iteration may be
accelerated using Anderson's method \cite{Anderson65, Walker-Ni09, Fang-Saad09,
LWWY11}. With Anderson acceleration using subspace size $m$, the series of
approximate solutions can be formulated as the linear combination
\begin{equation}\label{e:accelerated_fixed_point_iteration}
  y^{(n+1)} = \sum_{i=0}^{m_n} \alpha_i^{(n)} G(y^{(n-m_n+i)})
\end{equation}
where $m_n = \min\{m,n\}$ and the factors
\begin{equation}
\alpha^{(n)} =(\alpha_0^{(n)}, \ldots, \alpha_{m_n}^{(n)})
\end{equation}
solve the minimization problem $\min_\alpha  \| F_n \alpha^T \|_2$ under the
constraint that $\sum_{i=0}^{m_n} \alpha_i = 1$ where 
\begin{equation}
F_{n} = (f_{n-m_n}, \ldots, f_{n}) 
\end{equation}
with $f_i = G(y^{(i)}) - y^{(i)}$. Due to this constraint, in the limit of $m=0$
the accelerated fixed point iteration formula
\eqref{e:accelerated_fixed_point_iteration} simplifies to the standard
fixed point iteration \eqref{e:fixed_point_iteration}.

Following the recommendations made in \cite{Walker-Ni09}, the
{\sunnonlinsolfixedpoint} implementation computes the series of approximate
solutions as
\begin{equation}\label{e:accelerated_fixed_point_iteration_impl}
y^{(n+1)} = G(y^{(n)})-\sum_{i=0}^{m_n-1} \gamma_i^{(n)} \Delta g_{n-m_n+i}
\end{equation}
with $\Delta g_i = G(y^{(i+1)}) - G(y^{(i)})$ and where the factors
\begin{equation}
\gamma^{(n)} =(\gamma_0^{(n)}, \ldots, \gamma_{m_n-1}^{(n)})
\end{equation}
solve the unconstrained minimization problem
 $\min_\gamma \| f_n - \Delta F_n \gamma^T \|_2$ where 
\begin{equation}
\Delta F_{n} = (\Delta f_{n-m_n}, \ldots, \Delta f_{n-1}),
\end{equation}
with $\Delta f_i = f_{i+1} - f_i$. The least-squares problem is solved by
applying a QR factorization to $\Delta F_n = Q_n R_n$ and solving
 $R_n \gamma = Q_n^T f_n$.

The acceleration subspace size $m$ is required when constructing the
{\sunnonlinsolfixedpoint} object.  The default maximum number of
iterations and the stopping criteria for the fixed point iteration are
supplied by the {\sundials} integrator when {\sunnonlinsolfixedpoint}
is attached to it.  Both the maximum number of iterations and the
convergence test function may be modified by the user by calling
\id{SUNNonlinSolSetMaxIters} and \id{SUNNonlinSolSetConvTest}
functions after attaching the {\sunnonlinsolfixedpoint} object to the
integrator.

% ====================================================================
\subsection{SUNNonlinearSolver\_FixedPoint functions}
\label{ss:sunnonlinsolfixedpoint_functions}
% ====================================================================

The {\sunnonlinsolfixedpoint} module provides the following constructor
for creating the\\ \noindent
\id{SUNNonlinearSolver} object.
% --------------------------------------------------------------------
\ucfunction{SUNNonlinSol\_FixedPoint}
{
  NLS = SUNNonlinSol\_FixedPoint(y, m);
}
{
  The function \ID{SUNNonlinSol\_FixedPoint} creates a
  \id{SUNNonlinearSolver} object for use with {\sundials} integrators to
  solve nonlinear systems of the form $G(y) = y$.
}
{
  \begin{args}[y]
  \item[y] (\id{N\_Vector})
    a template for cloning vectors needed within the solver
  \item[m] (\id{int})
    the number of acceleration vectors to use
  \end{args}
}
{
  The return value \id{NLS} (of type \id{SUNNonlinearSolver}) will be
  a {\sunnonlinsol} object if the constructor exits successfully,
  otherwise \id{NLS} will be \id{NULL}.
}
{}
% --------------------------------------------------------------------
Since the accelerated fixed point iteration
\eqref{e:fixed_point_iteration} does not require the setup or solution
of any linear systems, the {\sunnonlinsolfixedpoint} module implements
all of the functions defined in sections \ref{ss:sunnonlinsol_corefn} --
\ref{ss:sunnonlinsol_getfn} except for the \id{SUNNonlinSolveSetup},
\id{SUNNonlinSolSetLSetupFn}, and \\ \noindent
\id{SUNNonlinSolSetLSolveFn} functions, that are set to \id{NULL}.
The {\sunnonlinsolfixedpoint} functions have the same names as those
defined by the generic {\sunnonlinsol} API with \id{\_FixedPoint}
appended to the function name.  Unless using the
{\sunnonlinsolfixedpoint} module as a standalone nonlinear solver the
generic functions defined in sections \ref{ss:sunnonlinsol_corefn} --
\ref{ss:sunnonlinsol_getfn} should be called in favor of the
{\sunnonlinsolfixedpoint}-specific implementations. 

The {\sunnonlinsolfixedpoint} module also defines the following additional
user-callable function.
% --------------------------------------------------------------------
\ucfunction{SUNNonlinSolGetSysFn\_FixedPoint}
{
  retval = SUNNonlinSolGetSysFn\_FixedPoint(NLS, SysFn);
}
{
  The function \ID{SUNNonlinSolGetSysFn\_FixedPoint} returns the fixed-point
  function that defines the nonlinear system.
}
{
  \begin{args}[SysFn]
  \item[NLS] (\id{SUNNonlinearSolver})
    a {\sunnonlinsol} object
  \item[SysFn] (\id{SUNNonlinSolSysFn*})
    the function defining the nonlinear system.
  \end{args}
}
{
  The return value \id{retval} (of type \id{int}) should be zero for a
  successful call, and a negative value for a failure.
}
{
  This function is intended for users that wish to evaluate the
  fixed-point function in a custom convergence test function for the
  {\sunnonlinsolfixedpoint} module. We note that {\sunnonlinsolfixedpoint}
  will not leverage the results from any user calls to \id{SysFn}.
}


% ====================================================================
\subsection{SUNNonlinearSolver\_FixedPoint content}
\label{ss:sunnonlinsolfixedpoint_content}
% ====================================================================

The \textit{content} field of the {\sunnonlinsolfixedpoint} module is the
following structure.
%%
%%
\begin{verbatim}
struct _SUNNonlinearSolverContent_FixedPoint {

  SUNNonlinSolSysFn      Sys;
  SUNNonlinSolConvTestFn CTest;

  int       m;
  int      *imap;
  realtype *R;
  realtype *gamma;
  realtype *cvals;
  N_Vector *df;
  N_Vector *dg;
  N_Vector *q;
  N_Vector *Xvecs;
  N_Vector  yprev;
  N_Vector  gy;
  N_Vector  fold;
  N_Vector  gold;
  N_Vector  delta;
  int       curiter;
  int       maxiters;
  long int  niters;
};
\end{verbatim}
%%
%%
The following entries of the \emph{content} field are always
allocated:
\begin{args}[maxiters]
  \item[Sys]      - function for evaluating the nonlinear system,
  \item[CTest]    - function for checking convergence of the fixed point iteration,
  \item[yprev]    - \id{N\_Vector} used to store previous fixed-point iterate,
  \item[gy]       - \id{N\_Vector} used to store $G(y)$ in fixed-point algorithm,
  \item[delta]    - \id{N\_Vector} used to store difference between successive fixed-point iterates,
  \item[curiter]  - the current number of iterations in the solve attempt,
  \item[maxiters] - the maximum number of fixed-point iterations allowed in
                    a solve, and
  \item[niters]   - the total number of nonlinear iterations across all
                    solves.
  \item[m]        - number of acceleration vectors,
\end{args}
If Anderson acceleration is requested (i.e., $m>0$ in the call to
\ID{SUNNonlinSol\_FixedPoint}), then the following items are also
allocated within the \emph{content} field:
\begin{args}[Xvecs]
  \item[imap]  - index array used in acceleration algorithm (length \id{m})
  \item[R]     - small matrix used in acceleration algorithm (length \id{m*m})
  \item[gamma] - small vector used in acceleration algorithm (length \id{m})
  \item[cvals] - small vector used in acceleration algorithm (length \id{m+1})
  \item[df]    - array of \id{N\_Vectors} used in acceleration algorithm (length \id{m})
  \item[dg]    - array of \id{N\_Vectors} used in acceleration algorithm (length \id{m})
  \item[q]     - array of \id{N\_Vectors} used in acceleration algorithm (length \id{m})
  \item[Xvecs] - \id{N\_Vector} pointer array used in acceleration algorithm (length \id{m+1})
  \item[fold]  - \id{N\_Vector} used in acceleration algorithm
  \item[gold]  - \id{N\_Vector} used in acceleration algorithm
\end{args}


% ====================================================================
\subsection{SUNNonlinearSolver\_FixedPoint Fortran interface}
\label{ss:sunnonlinsolfixedpoint_fortran}
% ====================================================================

For {\sundials} integrators that include a Fortran interface, the
{\sunnonlinsolfixedpoint} module also includes a Fortran-callable
function for creating a \id{SUNNonlinearSolver} object.
\ucfunction{FSUNFIXEDPOINTINIT}
{
  FSUNFIXEDPOINTINIT(code, m, ier);
}
{
  The function \ID{FSUNFIXEDPOINTINIT} can be called for Fortran programs
  to create a\\
  \id{SUNNonlinearSolver} object for use with {\sundials}
  integrators to solve nonlinear systems of the form $G(y) = y$.
}
{
  \begin{args}[code]
  \item[code] (\id{int*})
    is an integer input specifying the solver id (1 for {\cvode}, 2
    for {\ida}, 3 for {\kinsol}, and 4 for {\arkode}).
  \item[m] (\id{int*})
    is an integer input specifying the number of acceleration vectors.
  \end{args}
}
{
  \id{ier} is a return completion flag equal to \id{0} for a success
  return and \id{-1} otherwise. See printed message for details in case
  of failure.
}
{}

% ====================================================================
\section{The SUNNonlinearSolver\_PetscSNES implementation}
\label{s:sunnonlinsolpetsc}
% ====================================================================
This section describes the {\sunnonlinsol} interface to the PETSc SNES nonlinear
solver(s). To enable the {\sunnonlinsolpetsc} module, SUNDIALS must be
configured to use PETSc. Instructions on how to do this are given in
Chapter~\ref{ss:building_with_petsc}. To access the module, users must include
the header file \id{sunnonlinsol/sunnonlinsol\_petscsnes.h}. The library to link
to is \id{libsundials\_sunnonlinsolpetsc}{\em .lib} where {\em .lib} is .so for
shared libaries and .a for static libraries. Users of the {\sunnonlinsolpetsc}
should also see the section {\nvecpetsc} \ref{ss:nvec_petsc} which discusses the
{\nvector} interface to the PETSc Vec API.

% ====================================================================
\subsection{SUNNonlinearSolver\_PetscSNES description}
\label{ss:sunnonlinsolpetsc_description}
% ====================================================================

The {\sunnonlinsolpetsc} implementation allows users to utilize a PETSc SNES
nonlinear solver to solve the nonlinear systems that arise in the {\sundials}
integrators. Since SNES uses the KSP linear solver interface underneath it, the
{\sunnonlinsolpetsc} implementation does not interface with {\sundials} linear
solvers. Instead, users should set nonlinear solver options, linear solver
options, and preconditioner options through the PETSc SNES, KSP, and PC APIs
\cite{petsc-user-ref}.

{\warn}

Important usage notes for the {\sunnonlinsolpetsc} implementation are provided
below:

\begin{itemize}
\item The {\sunnonlinsolpetsc} implementation handles calling \id{SNESSetFunction}
at construction. The actual residual function $F(y)$ is set by the {\sundials}
integrator when the {\sunnonlinsolpetsc} object is attached to it. Therefore, a
user should not call \id{SNESSetFunction} on a \id{SNES} object that is being
used with {\sunnonlinsolpetsc}. For these reasons, it is recommended, although
not always necessary, that the user calls \id{SUNNonlinSol\_PetscSNES} with the
new \id{SNES} object immediately after calling

\item The number of nonlinear iterations is tracked by {\sundials} separately
from the count kept by SNES. As such, the function \id{SUNNonlinSolGetNumIters}
reports the cumulative number of iterations across the lifetime of the
{\sunnonlinsol} object.

\item Some ``converged'' and ``diverged'' convergence reasons returned by SNES
are treated as recoverable convergence failures by {\sundials}. Therefore, the
count of convergence failures returned by \id{SUNNonlinSolGetNumConvFails} will
reflect the number of recoverable convergence failures as determined by
{\sundials}, and may differ from the count returned by
\id{SNESGetNonlinearStepFailures}.

\item The {\sunnonlinsolpetsc} module is not currently compatible with the
{\cvodes} or {\idas} staggered or simultaneous sensitivity strategies.
\end{itemize}

% ====================================================================
\subsection{SUNNonlinearSolver\_PetscSNES functions}
\label{ss:sunnonlinsolpetsc_functions}
% ====================================================================

The {\sunnonlinsolpetsc} module provides the following constructor
for creating a \id{SUNNonlinearSolver} object.
% --------------------------------------------------------------------
\ucfunction{SUNNonlinSol\_PetscSNES}
{
  NLS = SUNNonlinSol\_PetscSNES(y, snes);
}
{
  The function \ID{SUNNonlinSol\_PetscSNES} creates a \id{SUNNonlinearSolver}
  object that wraps a PETSc \id{SNES} object for use with {\sundials}.
  This will call \id{SNESSetFunction} on the provided \id{SNES} object.
}
{
  \begin{args}[snes]
  \item[snes] (\id{SNES})
    a PETSc \id{SNES} object
  \item[y] (\id{N\_Vector})
    a \id{N\_Vector} object of type {\nvecpetsc} that used as a template
    for the residual vector
  \end{args}
}
{
  A {\sunnonlinsol} object if the constructor exits successfully,
  otherwise \id{NLS} will be \id{NULL}.
}
{
  {\warn} This function calls \id{SNESSetFunction} and will overwrite
  whatever function was previously set. Users should not call
  \id{SNESSetFunction} on the \id{SNES} object provided to the constructor.
}

% --------------------------------------------------------------------
The {\sunnonlinsolpetsc} module implements all of the functions
defined in sections \ref{ss:sunnonlinsol_corefn} --
\ref{ss:sunnonlinsol_getfn} except for \id{SUNNonlinSolSetup},
\id{SUNNonlinSolSetLSetupFn},\newline \id{SUNNonlinSolSetLSolveFn},
\id{SUNNonlinSolSetConvTestFn}, and \id{SUNNonlinSolSetMaxIters}.

The {\sunnonlinsolpetsc} functions have the same names as those defined
by the generic {\sunnonlinsol} API with \id{\_PetscSNES} appended to the
function name. Unless using the {\sunnonlinsolpetsc} module as a
standalone nonlinear solver the generic functions defined in sections
\ref{ss:sunnonlinsol_corefn} -- \ref{ss:sunnonlinsol_getfn} should be
called in favor of the {\sunnonlinsolpetsc}-specific implementations.

The {\sunnonlinsolpetsc} module also defines the following additional
user-callable functions.
% --------------------------------------------------------------------
\ucfunction{SUNNonlinSolGetSNES\_PetscSNES}
{
  retval = SUNNonlinSolGetSNES\_PetscSNES(NLS, SNES* snes);
}
{
  The function \ID{SUNNonlinSolGetSNES\_PetscSNES} gets the
  SNES context that was wrapped.
}
{
  \begin{args}[SysFn]
  \item[NLS] (\id{SUNNonlinearSolver})
    a {\sunnonlinsol} object
  \item[snes] (\id{SNES*})
    a pointer to a PETSc \id{SNES} object that will be set upon return
  \end{args}
}
{
  The return value \id{retval} (of type \id{int}) should be zero for a
  successful call, and a negative value for a failure.
}
{}
% --------------------------------------------------------------------
\ucfunction{SUNNonlinSolGetPetscError\_PetscSNES}
{
  retval = SUNNonlinSolGetPetscError\_PetscSNES(NLS, PestcErrorCode* error);
}
{
  The function \ID{SUNNonlinSolGetPetscError\_PetscSNES} gets the last error
  code returned by the last internal call to a PETSc API function.
}
{
  \begin{args}[SysFn]
  \item[NLS] (\id{SUNNonlinearSolver})
    a {\sunnonlinsol} object
  \item[error] (\id{PestcErrorCode*})
    a pointer to a PETSc error integer that will be set upon return
  \end{args}
}
{
  The return value \id{retval} (of type \id{int}) should be zero for a
  successful call, and a negative value for a failure.
}
{}
% --------------------------------------------------------------------
\ucfunction{SUNNonlinSolGetSysFn\_PetscSNES}
{
  retval = SUNNonlinSolGetSysFn\_PetscSNES(NLS, SysFn);
}
{
  The function \ID{SUNNonlinSolGetSysFn\_PetscSNES} returns the residual
  function that defines the nonlinear system.
}
{
  \begin{args}[SysFn]
  \item[NLS] (\id{SUNNonlinearSolver})
    a {\sunnonlinsol} object
  \item[SysFn] (\id{SUNNonlinSolSysFn*})
    the function defining the nonlinear system
  \end{args}
}
{
  The return value \id{retval} (of type \id{int}) should be zero for a
  successful call, and a negative value for a failure.
}
{}

% ====================================================================
\subsection{SUNNonlinearSolver\_PetscSNES content}
\label{ss:sunnonlinsolpetsc_content}
% ====================================================================

The {\sunnonlinsolpetsc} module defines the {\textit{content} field of a
\id{SUNNonlinearSolver} as the following structure:
%%
%%
\begin{verbatim}
struct _SUNNonlinearSolverContent_PetscSNES {
  int sysfn_last_err;
  PetscErrorCode petsc_last_err;
  long int nconvfails;
  long int nni;
  void *imem;
  SNES snes;
  Vec r;
  N_Vector y, f;
  SUNNonlinSolSysFn Sys;
};
\end{verbatim}
%%
%%
These entries of the \emph{content} field contain the following
information:
\begin{args}[Sys]
  \item[sysfn\_last\_err]  - last error returned by the system defining function,
  \item[petsc\_last\_err]  - last error returned by PETSc
  \item[nconvfails]        - number of nonlinear converge failures (recoverable or not),
  \item[nni]               - number of nonlinear iterations,
  \item[imem]              - {\sundials} integrator memory,
  \item[snes]              - PETSc SNES context,
  \item[r]                 - the nonlinear residual,
  \item[y]                 - wrapper for PETSc vectors used in the system function,
  \item[f]                 - wrapper for PETSc vectors used in the system function,
  \item[Sys]               - nonlinear system definining function.
\end{args}

