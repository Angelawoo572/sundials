%===================================================================================
\section{Forward sensitivity analysis example problems}\label{s:fwd_ex}
%===================================================================================

For all the {\idas} examples, either of the two sensitivity method options,
\id{IDA\_SIMULTANEOUS} or \id{IDA\_STAGGERED}, can be used, 
and sensitivities may be included in the error test or not 
(\id{errconS} set on \id{TRUE} or \id{FALSE}, respectively, in the
call to \id{IDASetSensErrCon}).

Abbreviated descriptions of one serial example (\id{idasSlCrank\_FSA\_dns})
and a parallel one (\id{idasBruss\_FSA\_kry\_bbd\_p}) are provided in the following
two subsections.  For details on the other examples, the reader is
directed to the comments in their source files.

%--------------------------------------------------------------------
\subsection{A serial dense example: idasSlCrank\_FSA\_dns}
\label{ss:idasSlCrank_FSA_dns}

The \id{idasSlCrank\_FSA\_dns} program solves a system of index-2 DAEs, modeling a
planar slider-crank mechanism.  The problem is obtained through a stabilized index
reduction (Gear-Gupta-Leimkuhler) starting from the index-3 DAE equations of motion
derived using three generalized coordinates and two algebraic position constraints.

The dependent variables are: three generalized coordinates --- crank
angle $q$, connecting bar angle $p$, and slider location $x$; their
speeds --- $v_q$, $v_p$, $v_x$; and four multipliers
$\lambda_1, \lambda_2, \mu_1, \mu_2$.

The equations of the system are as follows:
\begin{equation*}
\begin{split}
  &dq/dt = v_q - a \sin(q) \mu_1 + a \cos(q) \mu_2 \\
  &dp/dt = v_p - \sin(p) \mu_1 + \cos(p) \mu_2 \\
  &dx/dt = v_x - \mu_1 \\
  &J_1 dv_q/dt = F_q - a \sin(q) \lambda_1 + a \cos(q) \lambda_2 \\
  &J_2 dv_p/dt = F_p - \sin(p) \lambda_1 + \cos(p) \lambda_2 \\
  &m_2 dv_x/dt = F_x - \lambda_1 \\
  &x = \cos(p) + a \cos(q) \\
  &0 = \sin(p) + a \sin(q) \\
  &v_x = -a \sin(q) v_q - \sin(p) v_p \\
  &0 = a \cos(q) v_q + \cos(p) v_p
\end{split}
\end{equation*}
Here $a = 0.5$ is the half-length of the crank, $J_1 = 1$ is the moment of
inertia of the crank, $J_2 = 2$ is the moment of inertial of the connecting rod,
and $m_2 = 1$ is the mass of the connecting rod.  The force terms are given by:
\begin{equation*}
\begin{split}
  &F_q = - (f/l) a [\sin(p-q)/2 + x \sin(q)]/2 \\
  &F_p = - (f/l) [ x \sin(p) - a \sin(p-q)/2]/2 - F_e \sin(p) \\
  &F_x = (f/l) [\cos(p)/2 - x + a \cos(q)/2 ] + F_e ~~~~ \mbox{where}  \\
  &\ell^2 = x^2 - x [\cos(p) + a\cos(q)] + (1 + a^2)/4 + a \cos(p-q)/2 \\
  &2 \ell {\ell}' = 2 x v_x - v_x [\cos(p) + a\cos(q)] + x [\sin(p)v_p + a\sin(q)v_q]
                   - a \sin(p-q) (v_p-v_q)/2 \\
  &f = k (\ell - \ell_0) + c {\ell}'
\end{split}
\end{equation*}
Here $F_e = 1$ is a constant external force, $k = 1$ is the spring constant,
$c = 1$ is the damper constant, and $\ell_0 = 1$ is the spring free length.
The time interval is $0 \leq t \leq t_f = 10$.  The initial conditions (at $t = 0$)
are set to consistent values, given as follows:
\begin{equation*}
\begin{split}
  &q = \pi/2 \\
  &p = \arcsin(-a) \\
  &x = \cos(p) \\
  &v_q = v_p = v_x = 0 \\
  &\lambda_1 = \lambda_2 = \mu_1 = \mu_2 = 0 \\
  &dq/dt = dp/dt = dx/dt = 0 \\
  &dv_q/dt = \left[F_q\right]_{t=0} / J_1 \\
  &dv_p/dt = \left[F_p\right]_{t=0} / J_2 \\
  &dv_x/dt = \left[F_x\right]_{t=0} / m_2 \\
  &d\lambda_1/dt = d\lambda_2/dt = d\mu_1/dt = d\mu_2/dt = 0
\end{split}
\end{equation*}

In addition to solving the DAE system, the program uses {\idas} to compute
the time-average of the kinetic energy, as a quadrature:

\begin{equation*}
\begin{split}
  &G = \int_0^{t_f} g(t,y) \\
  &g(t,y) = (1/2)[J_1 (v_q)^2 - a^2 (v_q)^2/8 + J_2 (v_p)^2 + m_2 (v_x)^2 ]
\end{split}
\end{equation*}

This program uses the Forward Sensitivity Analysis capabilities of {\idas}
to compute the sensitivities $dG/dk$ and $dG/dc$ with respect to the
parameters $k$ and $c$.


The following output is generated by \id{idasSlCrank\_FSA\_dns} when computing
sensitivities with the \id{IDA\_SIMULTANEOUS} method and full error
control (\id{idasSlCrank\_FSA\_dns -sensi sim t}):

\includeOutput{idasSlCrank\_FSA\_dns}{../../examples/idas/serial/idasSlCrank_FSA_dns.out}


%----------------------------------------------------------------------------------

\subsection{A parallel example using IDABBDPRE: idasBruss\_FSA\_kry\_bbd\_p}
\label{ss:idasBruss_FSA_kry_bbd_p}

The \id{idasBruss\_FSA\_kry\_bbd\_p} program solves the two-species time-dependent
PDE known as the Brusselator problem, using the {\idaspgmr} linear solver and the
{\idabbdpre} preconditioner.

The PDEs are as follows:
\begin{equation*}
\begin{split}
  &\partial u / \partial t = \epsilon_1 (\partial^2 u / \partial x^2
                              + \partial^2 u / \partial y^2)
                              + u^2 v - (B + 1) u + A \\
  &\partial v / \partial t = \epsilon_2 (\partial^2 v / \partial x^2
                              + \partial^2 v / \partial y^2)
                              - u^2 v - B  u
\end{split}
\end{equation*}
on the unit square in $(x,y)$ and for $0 \leq t \leq t_f = 1$.  The constants
involved are $\epsilon_1 = \epsilon_2 = 0.002$, $A = 1$, and $B = 3.4$.
The boundary conditions are Neumann (zero derivatives).  The initial conditions
are given by:
\begin{equation*}
\begin{split}
  &u = 1 - 0.5 \cos(\pi y) \\
  &v = 3.5 - 2.5 \cos(\pi x)
\end{split}
\end{equation*}

The PDEs are discretized by central differencing on a uniform 2D spatial mesh.
The boundary conditions are handled by copying values from the first interior
mesh line to a line of ghost values on each side of the square.
The system is actually implemented on submeshes, processor by processor.

Here the forward sensitivity capability in {\idas} is used to compute
solution sensitivities with respect the two parameters $\epsilon_i$.
From those, we compute the corresponding sensitivities of the
of the final spatial average of $u$,
\begin{equation*}
  g = \int \int u(x,y,t_f) ~dx~dy
\end{equation*}
by means of a spatial integration:
\begin{equation*}
  dg/d\epsilon_i = \int \int \partial u(x,y,t_f) / \partial \epsilon_i~dx~dy ~.
\end{equation*}


The following output is generated by \id{idasBruss\_FSA\_kry\_bbd\_p} when computing
sensitivities with the \id{IDA\_SIMULTANEOUS} method and full error control:

\id{mpirun -np 4 idasBruss\_FSA\_kry\_bbd\_p -sensi sim t}

\includeOutput{idasBruss\_FSA\_kry\_bbd\_p}{../../examples/idas/parallel/idasBruss_FSA_kry_bbd_p.out}
