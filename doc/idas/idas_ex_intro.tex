%===============================================================================
\section{Introduction}\label{s:ex_intro}
%===============================================================================

This report is intended to serve as a companion document to the User
Documentation of {\idas} \cite{idas_ug}.  It provides details, with
listings, on the example programs supplied with the {\idas} distribution
package.

The {\idas} distribution contains examples of the following types: 
serial and parallel examples of Initial Value Problem (IVP) integration, 
serial and parallel examples of forward sensitivity analysis (FSA), and 
serial and parallel examples of adjoint sensitivity analysis (ASA).
The names of all the examples are given in the following table.

\newlength{\colone}
\settowidth{\colone}{em*3}
\begin{center}
  \begin{tabular}{|p{\colone}|l|l|} \hline

    & Serial examples & Parallel examples \\ \hline

    IVP & \id{idasRoberts\_dns}                              & \id{idasHeat2D\_kry\_p}       \\
    {}  & \id{idasRoberts\_klu} \id{idasRoberts\_sps}        & \id{idasHeat2D\_kry\_bbd\_p}  \\
    {}  & \id{idasAkzoNob\_dns} \id{idasSlCrank\_dns}        & \id{idasFoodWeb\_kry\_p}      \\
    {}  & \id{idasHeat2D\_bnd}  \id{idasHeat2D\_kry}         & \id{idasFoodWeb\_kry\_bbd\_p} \\
    {}  & \id{idasFoodWeb\_bnd} \id{idasFoodWeb\_bnd\_omp}   & \id{idasBruss\_kry\_bbd\_p}   \\
    {}  & \id{idasFoodWeb\_kry\_omp} \id{idasKrylovDemo\_ls} & \\
    \hline

    FSA & \id{idasRoberts\_FSA\_dns}                            & \id{idasBruss\_FSA\_kry\_bbd\_p} \\
    {}  & \id{idasRoberts\_FSA\_klu} \id{idasRoberts\_FSA\_sps} & \id{idasHeat2D\_FSA\_kry\_bbd\_p} \\
    {}  & \id{idasSlCrank\_FSA\_dns} & \\
    \hline
    
    ASA & \id{idasRoberts\_ASAi\_dns}                             & \id{idasBruss\_ASAp\_kry\_bbd\_p} \\
    {}  & \id{idasRoberts\_ASAi\_klu} \id{idasRoberts\_ASAi\_sps} & \\
    {}  & \id{idasAkzoNob\_ASAi\_dns} \id{idasHessian\_ASA\_FSA}  & \\
    \hline

  \end{tabular}
\end{center}

\vspace{0.2in}\noindent
With the exception of ''demo''-type example files, the names of all the examples 
distributed with {\sundials} are of the form \verb![slv][PbName]_[SA]_[ls]_[prec]_[p]!, 
where
\begin{description}
\item [{[slv]}] identifies the solver (for {\idas} examples this is \id{idas});
\item [{[PbName]}] identifies the problem;
\item [{[SA]}] identifies sensitivity analysis examples. This field can be one
  of: \id{FSA} for forward sensitivity examples, \id{ASAi} for adjoint sensitivity
  examples using an integral-form model output, or \id{ASAp} for adjoint sensitivity
  examples using an pointwise model output;
\item [{[ls]}] identifies the linear solver module used;
\item [{[prec]}] indicates the {\idas} preconditioner module used
  (if applicable --- for examples using a Krylov linear solver
  and the {\idabbdpre} module, this will be \id{bbd});
\item [{[p]}] indicates an example using the parallel vector module {\nvecp}.
\end{description}

\vspace{0.2in}\noindent
The examples are briefly described next.
Note that the {\idas} distribution includes all of the {\ida} {\CC}
examples (denoted here as examples for IVP integration). More details on
these can be found in the {\ida} Example Program document~\cite{ida_ex}.

%%
%%--------------------------------------------------------------------------
%%

\vspace{0.2in}\noindent
Supplied in the {\em srcdir}\id{/examples/idas/serial} directory are the
following serial examples (using the {\nvecs} module):

\begin{itemize}

%% IVP integration examples

\item \id{idasRoberts\_dns}
  solves the Robertson chemical kinetics problem~\cite{Rob:66}, which consists
  of two differential equations and one algebraic constraint.  It also uses
  the rootfinding feature of {\idas}.

  The problem is solved with the {\idadense} linear solver using
  a user-supplied Jacobian.

\item \id{idasRoberts\_klu}
  is the same as \id{idasRoberts\_dns} but uses the KLU sparse direct linear solver.

\item \id{idasRoberts\_sps}
  is the same as \id{idasRoberts\_dns} but uses the SuperLUMT sparse direct linear solver
  (with one thread).

\item \id{idasAkzoNob\_dns}
  solves the Akzo-Nobel chemical kinetics problem, which consists
  of six nonlinear DAEs of index 1. 
  The problem originates from Akzo Nobel Central research in Arnhern,
  The Netherlands, and describes a chemical process in which two
  species are mixed, while carbon dioxide is continuously added.

  The problem is solved with the {\idadense} linear solver using the
  default difference quotient dense Jacobian approximation.

\item \id{idasHeat2D\_bnd}
  solves a 2-D heat equation, semidiscretized to a DAE on the unit square.

  This program solves the problem with the {\idaband} linear solver and
  the default difference-quotient Jacobian approximation. For purposes of
  illustration, \id{IDACalcIC} is called to compute correct values at the
  boundary, given incorrect values as input initial guesses. The constraint
  $u > 0.0$ is imposed for all components.

\item \id{idasHeat2D\_kry}
  solves the same 2-D heat equation problem as \id{idasHeat2D\_bnd}, with the Krylov
  linear solver {\idaspgmr}. The preconditioner uses only the diagonal elements
  of the Jacobian.

\item \id{idasFoodWeb\_bnd}
  solves a system of PDEs modeling a food web problem, with predator-prey
  interaction and diffusion, on the unit square in 2-D.

  The PDEs are discretized in space to a system of DAEs which are solved
  using the {\idaband} linear solver with the default difference-quotient 
  Jacobian approximation.

\item \id{idasSlCrank\_dns}
  solves a system of index-2 DAEs, modeling a planar slider-crank mechanism.

  The problem is obtained through a stabilized index reduction (Gear-Gupta-Leimkuhler)
  starting from the index-3 DAE equations of motion derived using three generalized
  coordinates and two algebraic position constraints.
  The program also computes the time-averaged kinetic energy as a quadrature.

\item \id{idasKrylovDemo\_ls}
  solves the same problem as \id{idasHeat2D\_kry}, with three Krylov linear solvers
  {\idaspgmr}, {\idaspbcg}, and {\idasptfqmr}.  The preconditioner uses only
  the diagonal elements of the Jacobian.

%% FSA examples

\item \id{idasRoberts\_FSA\_dns}
  solves the same kinetics problem as in \id{idasRoberts\_dns}.
  \newline
  {\idas} also computes both its solution and solution sensitivities with respect
  to the three reaction rate constants appearing in the model. 
  This program solves the problem with the {\idadense} linear solver, and a 
  user-supplied Jacobian routine.

\item \id{idasRoberts\_FSA\_klu}
  solves the same problem as in \id{idasRoberts\_FSA\_dns} but uses
  the sparse direct solver KLU.

\item \id{idasRoberts\_FSA\_sps}
  solves the same problem as in \id{idasRoberts\_FSA\_dns} but uses
  the sparse solver SuperLUMT.

\item \id{idasSlCrank\_FSA\_dns}
  solves a system of index-2 DAEs, modeling a planar slider-crank mechanism.

  This example computes both its solution and solution sensitivities with respect
  to the problem parameters $k$ (spring constant) and $c$ (damper constant), 
  and then uses them to evaluate the gradient of the cumulative kinetic energy
  of the system.

%% ASA examples

\item \id{idasRoberts\_ASAi\_dns}
  solves the same kinetics problem as in \id{idasRoberts\_dns}.
  \newline
  Here the adjoint capability of {\idas} is used to compute gradients
  of a functional of the solution with respect to the three
  reaction rate constants appearing in the model.
  This program solves both the forward and backward problems with the 
  {\idadense} linear solver, and user-supplied Jacobian routines.

\item \id{idasRoberts\_ASAi\_klu}
  solves the same problem as in \id{idasRoberts\_ASAi\_dns}, but
  uses the sparse direct solver KLU.

\item \id{idasRoberts\_ASAi\_sps}
  solves the same problem as in \id{idasRoberts\_ASAi\_dns}, but
  uses the sparse solver SuperLUMT.

\item \id{idasAkzoNob\_ASAi\_dns}
  solves the Akzo-Nobel chemical kinetics problem.
  \newline
  The adjoint capability of {\idas} is used to compute gradients with
  respect to the initial conditions of the integral over time of the 
  concentration of the first species.

%% Hessian computation example

\item \id{idasHessian\_ASA\_FSA}
  is an example of using the {\em forward-over-adjoint} method for
  computing 2nd-order derivative information, in the form of Hessian-times-vector
  products.

\end{itemize}

%%
%%-------------------------------------------------------------------------------
%%

\vspace{0.2in}\noindent 
Supplied in the {\em srcdir}\id{/examples/idas/parallel} directory are
the following parallel examples (using the {\nvecp} module):

\begin{itemize}

%% IVP integration examples

\item \id{idasHeat2D\_kry\_p}
  solves the same 2-D heat equation problem as \id{idasHeat2D\_kry}, with {\idaspgmr}
  in parallel, and with a user-supplied diagonal preconditioner,
  
\item \id{idasHeat2D\_kry\_bbd\_p}
  solves the same problem as \id{idasHeat2D\_kry\_p}.

  This program uses the {\idaspgmr} linear solver in parallel,
  and the band-block-diagonal preconditioner {\idabbdpre} with half-bandwidths 
  equal to $1$.

\item \id{idasFoodWeb\_kry\_p}
  solves the same food web problem as \id{idasFoodWeb\_bnd}, but with {\idaspgmr}
  and a user-supplied preconditioner.
  
  The preconditioner supplied to {\idaspgmr} is the block-diagonal part of 
  the Jacobian with $n_s \times n_s$ blocks arising from the reaction terms only
  ($n_s =$ number of species).

\item \id{idasFoodWeb\_kry\_bbd\_p}
  solves the same food web problem as \id{idasFoodWeb\_kry\_p}.

  This program solves the problem using {\idaspgmr} in parallel and the
  {\idabbdpre} preconditioner.

\item \id{idasBruss\_kry\_bbd\_p}
  solves the two-species time-dependent PDE known as the Brusselator problem,
  using the {\idaspgmr} linear solver and the {\idabbdpre} preconditioner.

  The PDEs are discretized by central differencing on a 2D spatial mesh.
  The system is actually implemented on submeshes, processor by processor.

%% FSA examples

\item \id{idasBruss\_FSA\_kry\_bbd\_p}
  solves the Brusselator problem with the forward sensitivity capability in {\idas}
  used to compute solution sensitivities with respect to two of the problem 
  parameters, and then the gradient of a model output functional, written as
  the final time value of the spatial integral of the first PDE component.

\item \id{idasHeat2D\_FSA\_kry\_bbd\_p}
  solves the same problem as \id{idaHeat2D\_kry\_p}, but using the {\idabbdpre}
  preconditioner, and with forward sensitivity enabled to compute the solution
  sensitivity with respect to two coefficients of the original PDE.
  
%% ASA examples

\item \id{idasBruss\_ASAp\_kry\_bbd\_p}
  solves the same problem as \id{idasBruss\_FSA\_kry\_bbd\_p} but using an adjoint
  sensitivity approach for computing the gradient of the model output functional.

\end{itemize}

%%
%%-------------------------------------------------------------------------------
%%

\vspace{0.2in}\noindent 
Supplied in the {\em srcdir}\id{/examples/idas/C\_openmp} directory are
the following examples, using the OpenMP NVECTOR module:

\begin{itemize}

\item \id{idasFoodWeb\_bnd\_omp} solves the same problem as in
\id{idasFoodWeb\_bnd} but uses the OpenMP module. 

\item \id{idasFoodWeb\_kry\_omp} solves the same problem as in
\id{idasFoodWeb\_kry} but uses the OpenMP module. 

\end{itemize}

%%
%%--------------------------------------------------------------------------------
%%

\vspace{0.2in}\noindent
In the following sections, we give detailed descriptions of some (but
not all) of the sensitivity analysis examples. We do not discuss the 
examples for IVP integration; for those, the interested reader should consult
the {\ida} Examples document~\cite{ida_ex}. Any {\ida} problem
will work with {\idas} with only two modifications: (1) the main program
should include the header file \id{idas.h} instead of \id{ida.h}, and
(2) the loader command must reference
{\em builddir}\id{/lib/libsundials\_idas.}{\em lib} instead of
{\em builddir}\id{/lib/libsundials\_ida.}{\em lib}.

We also give our output files for each of thes examples described below,
but users should be cautioned that their results may differ slightly from these.
Differences in solution values may differ within the tolerances, and differences
in cumulative counters, such as numbers of steps or Newton iterations, may differ
from one machine environment to another by as much as 10\% to 20\%.

In the descriptions below, we make frequent references to the {\idas}
User Guide~\cite{idas_ug}.  All citations to specific sections
(e.g. \S\ref{s:types}) are references to parts of that user guide, unless
explicitly stated otherwise.

\vspace{0.2in}\noindent {\bf Note}
The examples in the {\idas} distribution were written in such a way as
to compile and run for any combination of configuration options during
the installation of {\sundials} (see Appendix \ref{c:install} in the User Guide).
As a consequence, they contain portions of code that will not typically be present in a
user program. For example, all example programs make use of the
variables \id{SUNDIALS\_EXTENDED\_PRECISION} and \id{SUNDIALS\_DOUBLE\_PRECISION}
to test if the solver libraries
were built in extended or double precision, and use the appropriate conversion 
specifiers in \id{printf} functions. Similarly, all forward sensitivity
examples can be run with or without sensitivity computations enabled and,
in the former case, with various combinations of methods and error control 
strategies. This is achieved in these example through the program arguments.

