%===================================================================================
\chapter{Using IDAS for Forward Sensitivity Analysis}\label{s:forward}
%===================================================================================

This chapter describes the use of {\idas} to compute solution sensitivities using
forward sensitivity analysis. One of our main guiding principles was to design 
the {\idas} user interface for forward sensitivity analysis as an extension of
that for IVP integration. Assuming a user main program and user-defined support 
routines for IVP integration have already been defined, in order to perform 
forward sensitivity analysis the user only has to insert a few more calls 
into the main program and (optionally) define an additional routine which
computes the residuals for sensitivity systems (\ref{e:sens_eqns}). 
The only departure from this philosophy is due to the \id{IDAResFn} type definition
(\S\ref{ss:resFn}). Without changing the definition of this type, the
only way to pass values of the problem parameters to the DAE residual
function is to require the user data structure \id{user\_data} to contain a pointer
to the array of real parameters $p$.

{\idas} uses various constants for both input and output.  These are
defined as needed in this chapter, but for convenience are also listed
separately in Appendix \ref{c:constants}.

We begin with a brief overview, in the form of a skeleton user program.
Following that are detailed descriptions of the interface to the
various user-callable routines and of the user-supplied routines that were not already
described in Chapter \ref{s:simulation}.

%---------------------------------------------------------------------
\section{A skeleton of the user's main program}\label{s:forward_usage}
%---------------------------------------------------------------------

The following is a skeleton of the user's main program (or calling
program) as an application of {\idas}. The user program is to have these 
steps in the order indicated, unless otherwise noted.
For the sake of brevity, we defer many of the details to the later sections.
As in \S\ref{ss:skeleton_sim}, most steps are independent of the {\nvector}
implementation used.  For the steps that are not, refer to
Chapter \ref{s:nvector} for the specific names.
Differences between the user main program in \S\ref{ss:skeleton_sim} and
the one below start only at step (\ref{i:fwd_start}).
Steps that are unchanged from the skeleton program presented in
\S\ref{ss:skeleton_sim} are grayed out.

First, note that no additional header files need be included for
forward sensitivity analysis beyond those for IVP solution
(\S\ref{ss:skeleton_sim}).
%%
%%
\index{User main program!forward
sensitivity analysis} \begin{Steps}
  
\item 
  \textcolor{gray}{\bf Initialize parallel or multi-threaded environment,
  if appropriate}

\item
  \textcolor{gray}{\bf Set problem dimensions etc.}

\item
  \textcolor{gray}{\bf Set vectors of initial values}
 
\item
  \textcolor{gray}{\bf Create {\idas} object}

\item
  \textcolor{gray}{\bf Initialize {\idas}}

\item
  \textcolor{gray}{\bf Specify integration tolerances}

\item
  \textcolor{gray}{\bf Set optional inputs}

\item
  \textcolor{gray}{\bf Create matrix object}

\item
  \textcolor{gray}{\bf Create linear solver object}

\item
  \textcolor{gray}{\bf Set linear solver optional inputs}

\item
  \textcolor{gray}{\bf Attach linear solver module}

\item
  \textcolor{gray}{\bf Initialize quadrature problem, if not sensitivity-dependent}

\item \label{i:fwd_start}
  {\bf Define the sensitivity problem}

  \begin{itemize}

    \item Number of sensitivities (required)

      Set \id{Ns} $= N_s$, the number of parameters with respect to which sensitivities
      are to be computed.
  
    \item Problem parameters (optional)

      If {\idas} is to evaluate the residuals of the sensitivity 
      systems, set \id{p}, an array of \id{Np} real parameters upon which the IVP 
      depends. Only parameters with respect to which sensitivities are (potentially) 
      desired need to be included. 
      Attach \id{p} to the user data structure \id{user\_data}. 
      For example, \id{user\_data->p = p;}

      If the user provides a function to evaluate the sensitivity residuals,
      \id{p} need not be specified.

    \item Parameter list (optional)

      If {\idas} is to evaluate the sensitivity residuals,
      set \id{plist}, an array of \id{Ns} integers to specify the 
      parameters \id{p} with respect to which solution sensitivities are to be computed.
      If sensitivities with respect to the $j$-th parameter \id{p[j]}
      ($0 \leq $ \id{j} $<$ \id{Np}) are desired, set
      ${\text{plist}}_i = j$, for some $i=0,\ldots, N_s-1$.

      If \id{plist} is not specified, {\idas} will compute sensitivities with respect
      to the first \id{Ns} parameters; i.e., ${\text{plist}}_i = i ~~ (i=0,\ldots,N_s-1)$.

      If the user provides a function to evaluate the sensitivity residuals,
      \id{plist} need not be specified.

    \item Parameter scaling factors (optional)

      If {\idas} is to estimate tolerances for the sensitivity solution vectors (based
      on tolerances for the state solution vector) or if {\idas} is to evaluate 
      the residuals of the sensitivity systems using the internal difference-quotient
      function, the results will be more accurate if order of magnitude information is provided.

      Set \id{pbar}, an array of \id{Ns} positive scaling factors. Typically,
      if $p_i \ne 0$, the value ${\bar p}_i = |p_{\text{plist}_i}|$ can be used.

      If \id{pbar} is not specified, {\idas} will use ${\bar p}_i = 1.0$.

      If the user provides a function to evaluate the sensitivity residual and specifies
      tolerances for the sensitivity variables, \id{pbar} need not be specified.

    \end{itemize}

    Note that the names for \id{p}, \id{pbar}, \id{plist}, as well as the field
    {\em p} of \id{user\_data} are arbitrary, but they must agree with the arguments
    passed to \id{IDASetSensParams} below.

\item
  {\bf Set sensitivity initial conditions}

  Set the \id{Ns} vectors \id{yS0[i]} and \id{ypS0[i]} of initial values
  for sensitivities (for $i=0,\ldots,$ \id{Ns} $ - 1$), using the
  appropriate functions defined by the particular {\nvector} implementation chosen.

  First, create an array of \id{Ns} vectors by making the appropriate call

  \id{yS0 = N\_VCloneVectorArray\_***(Ns, y0);}

  or

  \id{yS0 = N\_VCloneVectorArrayEmpty\_***(Ns, y0);}

  Here the argument \id{y0} serves only to provide the \id{N\_Vector}
  type for cloning.

  Then, for each $i = 0,\ldots, $\id{Ns} $ - 1$, load initial values
  for the $i$-th sensitivity vector \id{yS0[i]}.

  Set the initial conditions for the \id{Ns} sensitivity derivative vectors \id{ypS0}
  of $\dot{y}$ similarly.

\item
  {\bf Activate sensitivity calculations}

  Call \id{flag = }\Id{IDASensInit}\id{(\ldots);} to activate forward 
  sensitivity computations and allocate internal memory for {\idas} related 
  to sensitivity calculations (see \S\ref{ss:sensi_init}).

\item
  {\bf Set sensitivity tolerances}

  Call \id{IDASensSStolerances}, \id{IDASensSVtolerances}, or \id{IDASensEEtolerances}.
  See \S\ref{sss:idafwdtolerances}.

\item
  {\bf Set sensitivity analysis optional inputs}

  Call \id{IDASetSens*} routines to change from their default values any
  optional inputs that control the behavior of {\idas} in computing forward 
  sensitivities.  See \S\ref{ss:sens_optional_input}.

\item
  \textcolor{gray}{\bf Correct initial values}

\item
  \textcolor{gray}{\bf Specify rootfinding problem}

\item
  \textcolor{gray}{\bf Advance solution in time}

\item
  {\bf Extract sensitivity solution}

  After each successful return from \id{IDASolve}, the solution of the
  original IVP is available in the \id{y} argument of \id{IDASolve},
  while the sensitivity solution can be extracted into \id{yS} and \id{ypS}
  (which can be the same as \id{yS0} and \id{ypS0}, respectively) by calling 
  one of the following routines: \Id{IDAGetSens}, \Id{IDAGetSens1}, 
  \Id{IDAGetSensDky} or \Id{IDAGetSensDky1} (see \S\ref{ss:sensi_get}).

\item
 \textcolor{gray}{\bf Get optional outputs}

\item
  \textcolor{gray}{\bf Deallocate memory for solution vector}

\item 
  {\bf Deallocate memory for sensitivity vectors}

  Upon completion of the integration, deallocate memory for the vectors contained
  in \id{yS0} and \id{ypS0}:

  \id{N\_VDestroyVectorArray\_***(yS0, Ns);}

  If \id{yS} was created from \id{realtype} arrays \id{yS\_i}, it is the
  user's responsibility to also free the space for the arrays \id{yS\_i},
  and likewise for \id{ypS}.

\item
  {\bf Free user data structure}

\item
  \textcolor{gray}{\bf Free solver memory}
  
\item
  \textcolor{gray}{\bf Free vector specification memory}

\item 
  \textcolor{gray}{\bf Free linear solver and matrix memory}

\item 
  \textcolor{gray}{\bf Finalize MPI, if used}
  
\end{Steps}

%-------------------------------------------------------------------------------
\section{User-callable routines for forward sensitivity analysis}
%-------------------------------------------------------------------------------

This section describes the {\idas} functions, in addition to those presented
in \S\ref{ss:callable_fct_sim}, that are called by the user to set up and solve
a forward sensitivity problem.

\subsection{Forward sensitivity initialization and deallocation functions}
\label{ss:sensi_init}
%%
Activation of forward sensitivity computation is done by calling
\id{IDASensInit}.
The form of the call to this routine is as follows:
%%
\ucfunction{IDASensInit}
{
  flag = IDASensInit(ida\_mem, Ns, ism, resS, yS0, ypS0);
}
{
  The routine \ID{IDASensInit} activates forward sensitivity computations and
  allocates internal memory related to sensitivity calculations.
}
{
  \begin{args}[ida\_mem]

  \item[ida\_mem] (\id{void *})
    pointer to the {\idas} memory block returned by \id{IDACreate}.
  \item[Ns] (\id{int}) 
    the number of sensitivities to be computed.
  \item[ism] (\id{int})
    \index{forward sensitivity analysis!correction strategies}
    a flag used to select the sensitivity solution method.  Its value can 
    be either \id{IDA\_SIMULTANEOUS} or \id{IDA\_STAGGERED}:
    \begin{itemize}
    \item In the \ID{IDA\_SIMULTANEOUS} approach, the state and sensitivity variables are
      corrected at the same time. If \id{IDA\_NEWTON} was selected as the nonlinear system 
      solution method, this amounts to performing a modified Newton iteration on the
      combined nonlinear system;
    \item In the \ID{IDA\_STAGGERED} approach, the correction step for the sensitivity
      variables takes place at the same time for all sensitivity equations, but only after 
      the correction of the state variables has converged and the state variables 
      have passed the local error test; 
    \end{itemize}
  \item[resS] (\Id{IDASensResFn})
    is the {\CC} function which computes the residual of the sensitivity DAE. 
    For full details see \S\ref{s:user_fct_fwd}.
  \item[yS0] (\id{N\_Vector *}) 
    a pointer to an array of \id{Ns} vectors containing the initial 
    values of the sensitivities of $y$.
  \item[ypS0] (\id{N\_Vector *}) 
    a pointer to an array of \id{Ns} vectors containing the initial 
    values of the sensitivities of $\dot{y}$.
  \end{args}
}
{
  The return value \id{flag} (of type \id{int}) will be one of the following:
  \begin{args}[IDA\_ILL\_INPUT]
  \item[\Id{IDA\_SUCCESS}]
    The call to \id{IDASensInit} was successful.
  \item[\Id{IDA\_MEM\_NULL}] 
    The {\idas} memory block was not initialized through a 
    previous call to \id{IDACreate}.
  \item[\Id{IDA\_MEM\_FAIL}] 
    A memory allocation request has failed.
  \item[\Id{IDA\_ILL\_INPUT}] 
    An input argument to \id{IDASensInit} has an illegal value.
  \end{args}
}
{
  Passing \id{resS}$=$\id{NULL} indicates using the default internal difference 
  quotient sensitivity residual routine.

  If an error occurred, \id{IDASensInit} also prints an error message to the
  file specified by the optional input \id{errfp}.
}
%%
\index{memory requirements!IDAS@{\idas} solver}
In terms of the problem size $N$, number of sensitivity vectors $N_s$, and maximum method 
order \id{maxord}, the size of the real workspace is increased as follows:
\begin{itemize}
\item Base value: \id{lenrw} $=$ \id{lenrw} $+$ (\id{maxord+5})$N_s N$
\item With \id{IDASensSVtolerances}: 
  \id{lenrw} $=$ \id{lenrw} $+ N_s N$ 
\end{itemize}
the size of the integer workspace is increased as follows:
\begin{itemize}
\item Base value: \id{leniw} $=$ \id{leniw} $+$ (\id{maxord+5})$N_s N_i$
\item With \id{IDASensSVtolerances}: \id{leniw} $=$ \id{leniw} $+ N_s N_i$, 
\end{itemize}
where $N_i$ is the number of integer words in one \id{N\_Vector}.

The routine \ID{IDASensReInit}, useful during the solution of a sequence of problems of 
same size, reinitializes the sensitivity-related internal memory 
and must follow a call to \Id{IDASensInit} (and maybe a call to \id{IDAReInit}). 
The number \id{Ns} of sensitivities is assumed to be unchanged since the call to 
\Id{IDASensInit}.
%%
The call to the \id{IDASensReInit} function has the form:
%%
\ucfunction{IDASensReInit}
{
  flag = IDASensReInit(ida\_mem, ism, yS0, ypS0);
}
{
  The routine \ID{IDASensReInit} reinitializes forward sensitivity computations.
}
{
  \begin{args}[ida\_mem]

  \item[ida\_mem] (\id{void *})
    pointer to the {\idas} memory block returned by \id{IDACreate}.

  \item[ism] (\id{int})
    \index{forward sensitivity analysis!correction strategies}
    a flag used to select the sensitivity solution method.  Its value can 
    be either \id{IDA\_SIMULTANEOUS} or \id{IDA\_STAGGERED}.
  \item[yS0] (\id{N\_Vector *}) 
    a pointer to an array of \id{Ns} variables of type \id{N\_Vector} containing the 
    initial values of the sensitivities of $y$.
  \item[ypS0] (\id{N\_Vector *}) 
    a pointer to an array of \id{Ns} variables of type \id{N\_Vector} containing the 
    initial values of the sensitivities of $\dot{y}$.
  \end{args}
}
{
  The return value \id{flag} (of type \id{int}) will be one of the following:
  \begin{args}[IDA\_ILL\_INPUT]
  \item[\Id{IDA\_SUCCESS}]
    The call to \id{IDAReInit} was successful.
  \item[\Id{IDA\_MEM\_NULL}] 
    The {\idas} memory block was not initialized through a 
    previous call to \id{IDACreate}.
  \item[\Id{IDA\_NO\_SENS}]
    Memory space for sensitivity integration was not allocated through a 
    previous call to \id{IDASensInit}.
  \item[\Id{IDA\_ILL\_INPUT}] 
    An input argument to \id{IDASensReInit} has an illegal value.    
  \item[\Id{IDA\_MEM\_FAIL}] 
    A memory allocation request has failed.
  \end{args}
}
{
  All arguments of \id{IDASensReInit} are the same as those of
  \id{IDASensInit}.

  If an error occurred, \id{IDASensReInit} also prints an error message to the
  file specified by the optional input \id{errfp}.
}
%%
%%
To deallocate all forward sensitivity-related memory (allocated in a prior call
to \id{IDASensInit}), the user must call
%%
\ucfunction{IDASensFree}
{
  IDASensFree(ida\_mem);
}
{
  The function \ID{IDASensFree} frees the memory allocated for forward sensitivity 
  computations by a previous call to \id{IDASensInit}.
}
{
  The argument is the pointer to the {\idas} memory block (of type \id{void *}).
}
{
  The function \id{IDASensFree} has no return value.
}
{
  In general, \id{IDASensFree} need not be called by the user as it is
  invoked automatically by \id{IDAFree}.

  After a call to \id{IDASensFree}, forward sensitivity computations can be reactivated
  only by calling \id{IDASensInit} again.
}
%%
%%
To activate and deactivate forward sensitivity calculations for successive {\idas} runs,
without having to allocate and deallocate memory, the following function is provided:
%%
\ucfunction{IDASensToggleOff}
{
  IDASensToggleOff(ida\_mem);
}
{
  The function \ID{IDASensToggleOff} deactivates forward sensitivity 
  calculations. It does {\em not} deallocate sensitivity-related memory.
}
{
  \begin{args}[ida\_mem]
  \item[ida\_mem] (\id{void *})
    pointer to the memory previously allocated by \id{IDAInit}.
  \end{args}
}
{
  The return value \id{flag} of \id{IDASensToggle} is one of:
  \begin{args}[IDA\_MEM\_NULL]
  \item[\Id{IDA\_SUCCESS}] 
    \id{IDASensToggleOff} was successful.
  \item[\Id{IDA\_MEM\_NULL}] 
    \id{ida\_mem} was \id{NULL}.
  \end{args}
}
{
  Since sensitivity-related memory is not deallocated, sensitivities can
  be reactivated at a later time (using \id{IDASensReInit}).
}

%%==============================================================================
\subsection{Forward sensitivity tolerance specification functions}
\label{sss:idafwdtolerances}
%%==============================================================================
%%
One of the following three functions must be called to specify the
integration tolerances for sensitivities. Note that this call must be made after 
the call to \id{IDASensInit}.
%%
\ucfunction{IDASensSStolerances}
{
  flag = IDASensSStolerances(ida\_mem, reltolS, abstolS);
}
{
  The function \ID{IDASensSStolerances} specifies scalar relative and absolute
  tolerances.
}
{
  \begin{args}[ida\_mem]
  \item[ida\_mem] (\id{void *})
    pointer to the {\idas} memory block returned by \id{IDACreate}.
  \item[reltolS] (\id{realtype})
    is the scalar relative error tolerance.
  \item[abstolS] (\id{realtype*})
    is a pointer to an array of length \id{Ns} containing the scalar absolute 
    error tolerances.
  \end{args}
}
{
  The return flag \id{flag} (of type \id{int}) will be one of the following:
  \begin{args}[IDA\_ILL\_INPUT]
  \item[\Id{IDA\_SUCCESS}]
    The call to \id{IDASStolerances} was successful.
  \item[\Id{IDA\_MEM\_NULL}] 
    The {\idas} memory block was not initialized through a previous call to
    \id{IDACreate}.
  \item[\Id{IDA\_NO\_SENS}] 
    The sensitivity allocation function \id{IDASensInit} has not been called.
  \item[\Id{IDA\_ILL\_INPUT}] 
    One of the input tolerances was negative.
  \end{args}
}
{}
%%
%%
\ucfunction{IDASensSVtolerances}
{
  flag = IDASensSVtolerances(ida\_mem, reltolS, abstolS);
}
{
  The function \ID{IDASensSVtolerances} specifies scalar relative tolerance and
  vector absolute tolerances.
}
{
  \begin{args}[ida\_mem]
  \item[ida\_mem] (\id{void *})
    pointer to the {\idas} memory block returned by \id{IDACreate}.
  \item[reltolS] (\id{realtype})
    is the scalar relative error tolerance.
  \item[abstolS] (\id{N\_Vector*})
    is an array of \id{Ns} variables of type \id{N\_Vector}. The \id{N\_Vector} from
    \id{abstolS[is]} specifies the vector tolerances for \id{is}-th sensitivity.
  \end{args}
}
{
  The return flag \id{flag} (of type \id{int}) will be one of the following:
  \begin{args}[IDA\_ILL\_INPUT]
  \item[\Id{IDA\_SUCCESS}]
    The call to \id{IDASVtolerances} was successful.
  \item[\Id{IDA\_MEM\_NULL}] 
    The {\idas} memory block was not initialized through a previous call to
    \id{IDACreate}.
  \item[\Id{IDA\_NO\_SENS}] 
    The sensitivity allocation function \id{IDASensInit} has not been called.
  \item[\Id{IDA\_ILL\_INPUT}] 
    The relative error tolerance was negative or one of the absolute tolerance
    vectors had a negative component.
  \end{args}
}
{
  This choice of tolerances is important when the absolute error tolerance needs to
  be different for each component of any vector \id{yS[i]}. 
}
%%
%%
\ucfunction{IDASensEEtolerances}
{
  flag = IDASensEEtolerances(ida\_mem);
}
{
  When \ID{IDASensEEtolerances} is called, {\idas} will estimate tolerances for
  sensitivity variables based on the tolerances supplied for states variables
  and the scaling factors $\bar p$.
}
{
  \begin{args}[ida\_mem]
  \item[ida\_mem] (\id{void *})
    pointer to the {\idas} memory block returned by \id{IDACreate}.
  \end{args}
}
{
  The return flag \id{flag} (of type \id{int}) will be one of the following:
  \begin{args}[IDA\_ILL\_INPUT]
  \item[\Id{IDA\_SUCCESS}]
    The call to \id{IDASensEEtolerances} was successful.
  \item[\Id{IDA\_MEM\_NULL}] 
    The {\idas} memory block was not initialized through a previous call to
    \id{IDACreate}.
  \item[\Id{IDA\_NO\_SENS}] 
    The sensitivity allocation function \id{IDASensInit} has not been called.
  \end{args}
}
{}


\subsection{Forward sensitivity initial condition calculation function}
\label{sss:idasenscalcic}

\id{IDACalcIC} also calculates corrected initial conditions for sensitivity
variables of a DAE system. When used for initial conditions calculation of the 
forward sensitivities, \id{IDACalcIC} must be preceded by successful calls to  
\id{IDASensInit} (or \id{IDASensReInit}) and should precede the call(s) to 
\id{IDASolve}.  For restrictions that apply for initial conditions
calculation of the state variables, see \S\ref{ss:idacalcic}.

Calling \id{IDACalcIC} is optional. It is only necessary when the  
initial conditions do not satisfy the sensitivity systems. Even if forward 
sensitivity analysis was enabled, the call to the initial conditions
calculation function \id{IDACalcIC} is exactly the same as for state 
variables.

\begin{verbatim}
  flag = IDACalcIC(ida_mem, icopt, tout1);
\end{verbatim}

See \S\ref{ss:idacalcic} for a list of possible return values.


%%==============================================================================
\subsection{IDAS solver function}
%%==============================================================================
%%
Even if forward sensitivity analysis was enabled, the call to the main solver 
function \id{IDASolve} is exactly the same as in \S\ref{sss:idasolve}. However, 
in this case the return value \id{flag} can also be one of the following:
\begin{args}[IDA\_REP\_SRES\_ERR]
\item[\Id{IDA\_SRES\_FAIL}]
  The sensitivity residual function failed in an unrecoverable manner.
\item[\Id{IDA\_REP\_SRES\_ERR}]
  The user's residual function repeatedly returned a recoverable error flag, 
  but the solver was unable to recover.
\end{args}

%--------------------------

\subsection{Forward sensitivity extraction functions}\label{ss:sensi_get}

If forward sensitivity computations have been initialized by a call to
\id{IDASensInit}, or reinitialized by a call to \id{IDASensReInit},
then {\idas} computes both a solution and sensitivities at time
\id{t}. However, \id{IDASolve} will still return only the solutions
$y$ and $\dot{y}$ in \id{yret} and \id{ypret}, respectively.  Solution
sensitivities can be obtained through one of the following functions:
%%
%%
\ucfunction{IDAGetSens}
{
  flag = IDAGetSens(ida\_mem, \&tret, yS);
}
{
  The function \ID{IDAGetSens} returns the sensitivity solution vectors after a
  successful return from \id{IDASolve}.
}
{
  \begin{args}[ida\_mem]
  \item[ida\_mem] (\id{void *})
    pointer to the memory previously allocated by \id{IDAInit}.
  \item[tret] (\id{realtype})
    the time reached by the solver (output).
  \item[yS] (\id{N\_Vector *})
    the array of \id{Ns} computed forward sensitivity vectors.
  \end{args}
}
{
  The return value \id{flag} of \id{IDAGetSens} is one of:
  \begin{args}[IDA\_MEM\_NULL]
  \item[\Id{IDA\_SUCCESS}] 
    \id{IDAGetSens} was successful.
  \item[\Id{IDA\_MEM\_NULL}] 
    \id{ida\_mem} was \id{NULL}.
  \item[\Id{IDA\_NO\_SENS}] 
    Forward sensitivity analysis was not initialized.
  \item[\Id{IDA\_BAD\_DKY}] 
    \id{yS} is \id{NULL}.
  \end{args}
}
{
  Note that the argument \id{tret} is an output for this function. Its value will be
  the same as that returned at the last \id{IDASolve} call.
}
%%
%%
\index{optional output!interpolated sensitivities}
The function \ID{IDAGetSensDky} computes the \id{k}-th derivatives of the interpolating 
polynomials for the sensitivity variables at time \id{t}.
This function is called by \id{IDAGetSens} with \id{k} $= 0$, but may also be called 
directly by the user.
%%
\ucfunction{IDAGetSensDky}
{
  flag = IDAGetSensDky(ida\_mem, t, k, dkyS);
}
{
  The function \ID{IDAGetSensDky} returns derivatives of the sensitivity solution 
  vectors after a successful return from \id{IDASolve}.
}
{
  \begin{args}[ida\_mem]
  \item[ida\_mem] (\id{void *})
    pointer to the memory previously allocated by \id{IDAInit}.
  \item[t] (\id{realtype})
    specifies the time at which sensitivity information is 
    requested. The time \id{t} must fall within the interval defined by the last 
    successful step taken by {\idas}.
  \item[k] (\id{int}) order of derivatives.
  \item[dkyS] (\id{N\_Vector *})
    array of \id{Ns} vectors containing the derivatives on output. The space for
    \id{dkyS} must be allocated by the user. 
  \end{args}
}
{
  The return value \id{flag} of \id{IDAGetSensDky} is one of:
  \begin{args}[IDA\_MEM\_NULL]
  \item[\Id{IDA\_SUCCESS}] 
    \id{IDAGetSensDky} succeeded.
  \item[\Id{IDA\_MEM\_NULL}] 
    \id{ida\_mem} was \id{NULL}.
  \item[\Id{IDA\_NO\_SENS}] 
    Forward sensitivity analysis was not initialized.
  \item[\Id{IDA\_BAD\_DKY}]
    \id{dkyS} or one of the vectors \id{dkyS[i]} is \id{NULL}.
  \item[\Id{IDA\_BAD\_K}]
    \id{k} is not in the range $0, 1, ..., klast$.
  \item[\Id{IDA\_BAD\_T}] 
    The time \id{t} is not in the allowed range.
  \end{args}
}
{}
%%
%%
Forward sensitivity solution vectors can also be extracted separately for 
each parameter in turn through the functions \id{IDAGetSens1} and
\id{IDAGetSensDky1}, defined as follows:
\ucfunction{IDAGetSens1}
{
  flag = IDAGetSens1(ida\_mem, \&tret, is, yS);
}
{
  The function \ID{IDAGetSens1} returns the \id{is}-th sensitivity solution vector
  after a successful return from \id{IDASolve}.
}
{
  \begin{args}[ida\_mem]
  \item[ida\_mem] (\id{void *})
    pointer to the memory previously allocated by \id{IDAInit}.
  \item[tret] (\id{realtype *})
    the time reached by the solver (output).
  \item[is] (\id{int}) specifies which sensitivity vector is to be returned
    ($0\le$\id{is}$< N_s$).
  \item[yS] (\id{N\_Vector})
    the computed forward sensitivity vector.
  \end{args}
}
{
  The return value \id{flag} of \id{IDAGetSens1} is one of:
  \begin{args}[IDA\_MEM\_NULL]
  \item[\Id{IDA\_SUCCESS}]
    \id{IDAGetSens1} was successful.
  \item[\Id{IDA\_MEM\_NULL}] 
    \id{ida\_mem} was \id{NULL}.
  \item[\Id{IDA\_NO\_SENS}] 
    Forward sensitivity analysis was not initialized.
  \item[\Id{IDA\_BAD\_IS}]
    The index \id{is} is not in the allowed range.
  \item[\Id{IDA\_BAD\_DKY}] 
    \id{yS} is \id{NULL}.
  \item[\Id{IDA\_BAD\_T}] 
    The time \id{t} is not in the allowed range.
  \end{args}
}
{
  Note that the argument \id{tret} is an output for this function. Its value will be
  the same as that returned at the last \id{IDASolve} call.
}
%%
%%
\ucfunction{IDAGetSensDky1}
{
  flag = IDAGetSensDky1(ida\_mem, t, k, is, dkyS);
}
{
  The function \ID{IDAGetSensDky1} returns the \id{k}-th derivative of the 
  \id{is}-th sensitivity solution vector after a successful return from \id{IDASolve}.
}
{
  \begin{args}[ida\_mem]
  \item[ida\_mem] (\id{void *})
    pointer to the memory previously allocated by \id{IDAInit}.
  \item[t] (\id{realtype})
    specifies the time at which sensitivity information is 
    requested. The time \id{t} must fall within the interval defined by the last 
    successful step taken by {\idas}.
  \item[k] (\id{int}) order of derivative.
  \item[is] (\id{int}) specifies the sensitivity derivative vector to be returned
    ($0\le$\id{is}$< N_s$).
  \item[dkyS] (\id{N\_Vector})
    the vector containing the derivative on output. The space for \id{dkyS} must be
    allocated by the user. 
  \end{args}
}
{
  The return value \id{flag} of \id{IDAGetSensDky1} is one of:
  \begin{args}[IDA\_MEM\_NULL]
  \item[\Id{IDA\_SUCCESS}] 
    \id{IDAGetQuadDky1} succeeded.
  \item[\Id{IDA\_MEM\_NULL}] 
    \id{ida\_mem} was \id{NULL}.
  \item[\Id{IDA\_NO\_SENS}] 
    Forward sensitivity analysis was not initialized.
  \item[\Id{IDA\_BAD\_DKY}] 
    \id{dkyS} is \id{NULL}.
  \item[\Id{IDA\_BAD\_IS}]
    The index \id{is} is not in the allowed range.
  \item[\Id{IDA\_BAD\_K}] 
    \id{k} is not in the range $0, 1, ..., klast$.
  \item[\Id{IDA\_BAD\_T}] 
    The time \id{t} is not in the allowed range.
  \end{args}
}
{}
%%
%%

%--------------------------
\subsection{Optional inputs for forward sensitivity analysis}
\label{ss:sens_optional_input}
\index{optional input!forward sensitivity|(}
Optional input variables that control the computation of sensitivities
can be changed from their default values through calls to \id{IDASetSens*}
functions. Table~\ref{t:optional_input_fwd} lists all forward sensitivity 
optional input functions in {\idas} which are described in detail in the 
remainder of this section. 
%%
%%
\begin{table}[b]
\centering
\caption{Forward sensitivity optional inputs}
\label{t:optional_input_fwd}
\medskip
\begin{tabular}{|l|l|l|}\hline
{\bf Optional input} & {\bf Routine name} & {\bf Default} \\
\hline
Sensitivity scaling factors & \id{IDASetSensParams} & \id{NULL} \\
DQ approximation method & \id{IDASetSensDQMethod} & centered,0.0 \\
Error control strategy & \id{IDASetSensErrCon} & \id{SUNFALSE} \\
Maximum no. of nonlinear iterations & \id{IDASetSensMaxNonlinIters} & 3 \\
\hline
\end{tabular}
\end{table}
%%
%%
\ucfunction{IDASetSensParams}
{
  flag = IDASetSensParams(ida\_mem, p, pbar, plist);
}
{
  The function \ID{IDASetSensParams} specifies problem parameter information
  for sensitivity calculations.
}
{
  \begin{args}[ida\_mem]
  \item[ida\_mem] (\id{void *})
    pointer to the {\idas} memory block.
  \item[p] (\id{realtype *})
    a pointer to the array of real problem parameters used to evaluate $F(t,y,\dot{y},p)$.
    If non-\id{NULL}, \id{p} must point to a field in the user's data structure
    \id{user\_data} passed to the user's residual function.
    (See \S\ref{s:forward_usage}).
  \item[pbar] (\id{realtype *})
    an array of \id{Ns} positive scaling factors. If non-\id{NULL}, \id{pbar} must
    have all its components $> 0.0$.
    (See \S\ref{s:forward_usage}).
  \item[plist] (\id{int *}) 
    an array of \id{Ns} non-negative indices to specify which components of
    \id{p} to use in estimating the sensitivity equations.
    If non-\id{NULL}, \id{plist} must have all components $\ge 0$.
    (See \S\ref{s:forward_usage}).
  \end{args}
}
{
  The return value \id{flag} (of type \id{int}) is one of:
  \begin{args}[IDA\_ILL\_INPUT]
  \item[\Id{IDA\_SUCCESS}] 
    The optional value has been successfully set.
  \item[\Id{IDA\_MEM\_NULL}]
    The \id{ida\_mem} pointer is \id{NULL}.
  \item[\Id{IDA\_NO\_SENS}]
    Forward sensitivity analysis was not initialized.
  \item[\Id{IDA\_ILL\_INPUT}]
    An argument has an illegal value.
  \end{args}
}
{
  {\warn}This function must be preceded by a call to \id{IDASensInit}.
}
%%
%%
\ucfunction{IDASetSensDQMethod}
{
  flag = IDASetSensDQMethod(ida\_mem, DQtype, DQrhomax);
}
{
  The function \ID{IDASetSensDQMethod} specifies the difference quotient strategy in
  the case in which the residual of the sensitivity equations are to
  be computed by {\idas}.
}
{
  \begin{args}[ida\_mem]
  \item[ida\_mem] (\id{void *})
    pointer to the {\idas} memory block.
  \item[DQtype] (\id{int})
    specifies the difference quotient type and can be either \ID{IDA\_CENTERED}
    or \ID{IDA\_FORWARD}.
  \item[DQrhomax] (\id{realtype})
    positive value of the selection parameter used in deciding switching between a simultaneous
    or separate approximation of the two terms in the sensitivity residual.
  \end{args}
}
{
  The return value \id{flag} (of type \id{int}) is one of:
  \begin{args}[IDA\_ILL\_INPUT]
  \item[\Id{IDA\_SUCCESS}]
    The optional value has been successfully set.
  \item[\Id{IDA\_MEM\_NULL}]
    The \id{ida\_mem} pointer is \id{NULL}.
  \item[\Id{IDA\_ILL\_INPUT}]
    An argument has an illegal value.
  \end{args}
}
{
  If \id{DQrhomax} $= 0.0$, then no switching is performed. The approximation is done simultaneously
  using either centered or forward finite differences, depending on the value of \id{DQtype}. 
  For values of \id{DQrhomax} $\ge 1.0$, the simultaneous approximation is used whenever the 
  estimated finite difference perturbations for states and parameters are within a factor of
  \id{DQrhomax}, and the separate approximation is used otherwise. Note that a value 
  \id{DQrhomax} $< 1.0$ will effectively disable switching.   
  See \S\ref{ss:fwd_sensi} for more details.

  The default value are \id{DQtype}$=$\id{IDA\_CENTERED} and \id{DQrhomax}$=0.0$. 
}
%%
%%
\ucfunction{IDASetSensErrCon}
{
  flag = IDASetSensErrCon(ida\_mem, errconS);
}
{
  The function \ID{IDASetSensErrCon} specifies the error control
  strategy for sensitivity variables.
}
{
  \begin{args}[ida\_mem]
  \item[ida\_mem] (\id{void *})
    pointer to the {\idas} memory block.
  \item[errconS] (\id{booleantype})
    specifies whether sensitivity variables are included (\id{SUNTRUE}) or not
    (\id{SUNFALSE}) in the error control mechanism.
  \end{args}
}
{
  The return value \id{flag} (of type \id{int}) is one of:
  \begin{args}[IDA\_MEM\_NULL]
  \item[\Id{IDA\_SUCCESS}] 
    The optional value has been successfully set.
  \item[\Id{IDA\_MEM\_NULL}]
    The \id{ida\_mem} pointer is \id{NULL}.
  \end{args}
}
{
  By default, \id{errconS} is set to \id{SUNFALSE}. 
  If \id{errconS}$=$\id{SUNTRUE} then both state variables and
  sensitivity variables are included in the error tests. 
  If \id{errconS}$=$\id{SUNFALSE} then the sensitivity variables are excluded from the 
  error tests. Note that, in any event, all variables are considered in the convergence 
  tests.
}
%%
%%
\ucfunction{IDASetSensMaxNonlinIters}
{
  flag = IDASetSensMaxNonlinIters(ida\_mem, maxcorS);
}
{
  The function \ID{IDASetSensMaxNonlinIters} specifies the maximum
  number of nonlinear solver iterations for sensitivity variables per step.
}
{
  \begin{args}[ida\_mem]
  \item[ida\_mem] (\id{void *})
    pointer to the {\idas} memory block.
  \item[maxcorS] (\id{int})
    maximum number of nonlinear solver iterations allowed per step $( > 0)$.
  \end{args}
}
{
  The return value \id{flag} (of type \id{int}) is one of:
  \begin{args}[IDA\_MEM\_NULL]
  \item[\Id{IDA\_SUCCESS}] 
    The optional value has been successfully set.
  \item[\Id{IDA\_MEM\_NULL}]
    The \id{ida\_mem} pointer is \id{NULL}.
  \end{args}
}
{
  The default value is $3$.
}
\index{optional input!forward sensitivity|)}
%%
%%
%%--------------------------

\subsection{Optional outputs for forward sensitivity analysis}
\label{ss:sens_optional_output}

\subsubsection{Main solver optional output functions}\label{sss:sens_optout_main}
\index{optional output!forward sensitivity|(}
Optional output functions that return statistics and solver performance information
related to forward sensitivity computations are listed in Table~\ref{t:optional_output_fwd}
and described in detail in the remainder of this section.
%%
%%
\begin{table}[b]
\centering
\caption{Forward sensitivity optional outputs}
\label{t:optional_output_fwd}
\medskip
\begin{tabular}{|l|l|}\hline
{\bf Optional output} & {\bf Routine name} \\
\hline
No. of calls to sensitivity residual function & \id{IDAGetSensNumResEvals} \\
No. of calls to residual function for sensitivity& \id{IDAGetNumResEvalsSens} \\
No. of sensitivity local error test failures & \id{IDAGetSensNumErrTestFails} \\
No. of calls to lin. solv. setup routine for sens. & \id{IDAGetSensNumLinSolvSetups} \\
Sensitivity-related statistics as a group & \id{IDAGetSensStats} \\
Error weight vector for sensitivity variables & \id{IDAGetSensErrWeights} \\
No. of sens. nonlinear solver iterations& \id{IDAGetSensNumNonlinSolvIters} \\
No. of sens. convergence failures& \id{IDAGetSensNumNonlinSolvConvFails} \\ 
Sens. nonlinear solver statistics as a group & \id{IDAGetSensNonlinSolvStats} \\
\hline
\end{tabular}
\end{table}
%%
%%
\ucfunction{IDAGetSensNumResEvals}
{
  flag = IDAGetSensNumResEvals(ida\_mem, \&nfSevals);
}
{
  The function \ID{IDAGetSensNumResEvals} returns the number of calls to the sensitivity
  residual function.
}
{
  \begin{args}[ida\_mem]
  \item[ida\_mem] (\id{void *})
    pointer to the {\idas} memory block.
  \item[nfSevals] (\id{long int})
    number of calls to the sensitivity residual function.
  \end{args}
}
{
  The return value \id{flag} (of type \id{int}) is one of:
  \begin{args}[IDA\_MEM\_NULL]
  \item[\Id{IDA\_SUCCESS}] 
    The optional output value has been successfully set.
  \item[\Id{IDA\_MEM\_NULL}]
    The \id{ida\_mem} pointer is \id{NULL}.
  \item[\Id{IDA\_NO\_SENS}]
    Forward sensitivity analysis was not initialized.
  \end{args}
}
{}
%%
%%
\ucfunction{IDAGetNumResEvalsSens}
{
  flag = IDAGetNumResEvalsSens(ida\_mem, \&nfevalsS);
}
{
  The function \ID{IDAGetNumResEvalsSEns} returns the number of calls to the
  user's residual function due to the internal finite difference approximation
  of the sensitivity residuals.
}
{
  \begin{args}[ida\_mem]
  \item[ida\_mem] (\id{void *})
    pointer to the {\idas} memory block.
  \item[nfevalsS] (\id{long int})
    number of calls to the user residual function for sensitivity residuals.
  \end{args}
}
{
  The return value \id{flag} (of type \id{int}) is one of:
  \begin{args}[IDA\_MEM\_NULL]
  \item[\Id{IDA\_SUCCESS}] 
    The optional output value has been successfully set.
  \item[\Id{IDA\_MEM\_NULL}]
    The \id{ida\_mem} pointer is \id{NULL}.
  \item[\Id{IDA\_NO\_SENS}]
    Forward sensitivity analysis was not initialized.
  \end{args}
}
{
  This counter is incremented only if the internal finite difference approximation
  routines are used for the evaluation of the sensitivity residuals.
}
%%
%%
\ucfunction{IDAGetSensNumErrTestFails}
{
  flag = IDAGetSensNumErrTestFails(ida\_mem, \&nSetfails);
}
{
  The function \ID{IDAGetSensNumErrTestFails} returns the number of local
  error test failures for the sensitivity variables that have occurred.
}
{
  \begin{args}[ida\_mem]
  \item[ida\_mem] (\id{void *})
    pointer to the {\idas} memory block.
  \item[nSetfails] (\id{long int})
    number of error test failures.
  \end{args}
}
{
  The return value \id{flag} (of type \id{int}) is one of:
  \begin{args}[IDA\_MEM\_NULL]
  \item[\Id{IDA\_SUCCESS}] 
    The optional output value has been successfully set.
  \item[\Id{IDA\_MEM\_NULL}]
    The \id{ida\_mem} pointer is \id{NULL}.
  \item[\Id{IDA\_NO\_SENS}]
    Forward sensitivity analysis was not initialized.
  \end{args}
}
{
  This counter is incremented only if the sensitivity variables have been included
  in the error test (see \id{IDASetSensErrCon} in \S\ref{ss:sens_optional_input}). 
  Even in that case, this counter is not incremented if the \id{ism}$=$\id{IDA\_SIMULTANEOUS} 
  sensitivity solution method has been used.
}
%%
%%
\ucfunction{IDAGetSensNumLinSolvSetups}
{
  flag = IDAGetSensNumLinSolvSetups(ida\_mem, \&nlinsetupsS);
}
{
  The function \ID{IDAGetSensNumLinSolvSetups} returns the number of calls
  to the linear solver setup function due to forward sensitivity calculations.
}
{
  \begin{args}[ida\_mem]
  \item[ida\_mem] (\id{void *})
    pointer to the {\idas} memory block.
  \item[nlinsetupsS] (\id{long int})
    number of calls to the linear solver setup function.
  \end{args}
}
{
  The return value \id{flag} (of type \id{int}) is one of:
  \begin{args}[IDA\_MEM\_NULL]
  \item[\Id{IDA\_SUCCESS}] 
    The optional output value has been successfully set.
  \item[\Id{IDA\_MEM\_NULL}]
    The \id{ida\_mem} pointer is \id{NULL}.
  \item[\Id{IDA\_NO\_SENS}]
    Forward sensitivity analysis was not initialized.
  \end{args}
}
{
  This counter is incremented only if Newton iteration has been used and
  staggered sensitivity solution method (\id{ism}$=$\id{IDA\_STAGGERED})
  was specified in the call to \id{IDASensInit} (see \S\ref{ss:sensi_init}).
}
%%
%%
\ucfunction{IDAGetSensStats}
{
  \begin{tabular}[t]{@{}r@{}l@{}}
    flag = IDAGetSensStats(&ida\_mem, \&nfSevals, \&nfevalsS, \&nSetfails,\\
                             &\&nlinsetupsS);
  \end{tabular}
}
{
  The function \ID{IDAGetSensStats} returns all of the above sensitivity-related solver
  statistics as a group.
}
{
  \begin{args}[ida\_mem]
  \item[ida\_mem] (\id{void *})
    pointer to the {\idas} memory block.
  \item[nfSevals] (\id{long int})
    number of calls to the sensitivity residual function.
  \item[nfevalsS] (\id{long int})
    number of calls to the user-supplied residual function.
  \item[nSetfails] (\id{long int})
    number of error test failures.
  \item[nlinsetupsS] (\id{long int})
    number of calls to the linear solver setup function.
  \end{args}
}
{
  The return value \id{flag} (of type \id{int}) is one of:
  \begin{args}[IDA\_MEM\_NULL]
  \item[\Id{IDA\_SUCCESS}] 
    The optional output values have been successfully set.
  \item[\Id{IDA\_MEM\_NULL}]
    The \id{ida\_mem} pointer is \id{NULL}.
  \item[\Id{IDA\_NO\_SENS}]
    Forward sensitivity analysis was not initialized.
  \end{args}
}
{}
%%
%%
\ucfunction{IDAGetSensErrWeights}
{
  flag = IDAGetSensErrWeights(ida\_mem, eSweight);
}
{
  The function \ID{IDAGetSensErrWeights} returns the sensitivity error weight
  vectors at the current time. These are the reciprocals of the $W_i$ of
  (\ref{e:errwt}) for the sensitivity variables.
}
{
  \begin{args}[ida\_mem]
  \item[ida\_mem] (\id{void *})
    pointer to the {\idas} memory block.
  \item[eSweight] (\id{N\_Vector\_S})
    pointer to the array of error weight vectors.
  \end{args}
}
{
  The return value \id{flag} (of type \id{int}) is one of:
  \begin{args}[IDA\_MEM\_NULL]
  \item[\Id{IDA\_SUCCESS}] 
    The optional output value has been successfully set.
  \item[\Id{IDA\_MEM\_NULL}]
    The \id{ida\_mem} pointer is \id{NULL}.
  \item[\Id{IDA\_NO\_SENS}]
    Forward sensitivity analysis was not initialized.
  \end{args}
}
{
  The user must allocate memory for \id{eweightS}.
}
%%
%%
\ucfunction{IDAGetSensNumNonlinSolvIters}
{
  flag = IDAGetSensNumNonlinSolvIters(ida\_mem, \&nSniters);
}
{
  The function \ID{IDAGetSensNumNonlinSolvIters} returns the
  number of nonlinear iterations performed for 
  sensitivity calculations.
}
{
  \begin{args}[ida\_mem]
  \item[ida\_mem] (\id{void *})
    pointer to the {\idas} memory block.
  \item[nSniters] (\id{long int})
    number of nonlinear iterations performed.
  \end{args}
}
{
  The return value \id{flag} (of type \id{int}) is one of:
  \begin{args}[IDA\_MEM\_NULL]
  \item[\Id{IDA\_SUCCESS}] 
    The optional output value has been successfully set.
  \item[\Id{IDA\_MEM\_NULL}]
    The \id{ida\_mem} pointer is \id{NULL}.
  \item[\Id{IDA\_NO\_SENS}]
    Forward sensitivity analysis was not initialized.
  \end{args}
}
{
  This counter is incremented only if \id{ism} was \id{IDA\_STAGGERED}
  in the call to \id{IDASensInit} (see \S\ref{ss:sensi_init}).
}
%%
%%
\ucfunction{IDAGetSensNumNonlinSolvConvFails}
{
  flag = IDAGetSensNumNonlinSolvConvFails(ida\_mem, \&nSncfails);
}
{
  The function \ID{IDAGetSensNumNonlinSolvConvFails} returns the
  number of nonlinear convergence failures that have occurred for
  sensitivity calculations.
}
{
  \begin{args}[ida\_mem]
  \item[ida\_mem] (\id{void *})
    pointer to the {\idas} memory block.
  \item[nSncfails] (\id{long int})
    number of nonlinear convergence failures.
  \end{args}
}
{
  The return value \id{flag} (of type \id{int}) is one of:
  \begin{args}[IDA\_MEM\_NULL]
  \item[\Id{IDA\_SUCCESS}] 
    The optional output value has been successfully set.
  \item[\Id{IDA\_MEM\_NULL}]
    The \id{ida\_mem} pointer is \id{NULL}.
  \item[\Id{IDA\_NO\_SENS}]
    Forward sensitivity analysis was not initialized.
  \end{args}
}
{
  This counter is incremented only if \id{ism} was \id{IDA\_STAGGERED}
  in the call to \id{IDASensInit} (see \S\ref{ss:sensi_init}).
}
%%
%%
\ucfunction{IDAGetSensNonlinSolvStats}
{
  flag = IDAGetSensNonlinSolvStats(ida\_mem, \&nSniters, \&nSncfails);
}
{
  The function \ID{IDAGetSensNonlinSolvStats} returns the sensitivity-related
  nonlinear solver statistics as a group.
}
{
  \begin{args}[ida\_mem]
  \item[ida\_mem] (\id{void *})
    pointer to the {\idas} memory block.
  \item[nSniters] (\id{long int})
    number of nonlinear iterations performed.
  \item[nSncfails] (\id{long int})
    number of nonlinear convergence failures.
  \end{args}
}
{
  The return value \id{flag} (of type \id{int}) is one of:
  \begin{args}[IDA\_MEM\_NULL]
  \item[\Id{IDA\_SUCCESS}] 
    The optional output values have been successfully set.
  \item[\Id{IDA\_MEM\_NULL}]
    The \id{ida\_mem} pointer is \id{NULL}.
  \item[\Id{IDA\_NO\_SENS}]
    Forward sensitivity analysis was not initialized.
  \end{args}
}
{}
%%
\index{optional output!forward sensitivity|)}
%%
%%-------------------------------------------------------------------
%%
\subsubsection{Initial condition calculation optional output functions}
\label{sss:sens_optout_iccalc}
\index{optional output!initial condition calculation|(}

The sensitivity consistent initial conditions found by {\idas} (after a successful
call to \id{IDACalcIC}) can be obtained by calling the following function:

\ucfunction{IDAGetSensConsistentIC}
{
  flag = IDAGetSensConsistentIC(ida\_mem, yyS0\_mod, ypS0\_mod);
}
{
  The function \ID{IDAGetSensConsistentIC} returns the corrected initial conditions
  calculated by \id{IDACalcIC} for sensitivities variables.
}
{
  \begin{args}[yyS0\_mod]
  \item[ida\_mem] (\id{void *})
    pointer to the {\idas} memory block.
  \item[yyS0\_mod] (\id{N\_Vector *})
    a pointer to an array of \id{Ns} vectors containing consistent sensitivity 
    vectors.
  \item[ypS0\_mod] (\id{N\_Vector *})
    a pointer to an array of \id{Ns} vectors containing consistent sensitivity
    derivative vectors.
  \end{args}
}
{
  The return value \id{flag} (of type \id{int}) is one of
  \begin{args}[IDA\_ILL\_INPUT]
  \item[IDA\_SUCCESS] 
    \id{IDAGetSensConsistentIC} succeeded.
  \item[\Id{IDA\_MEM\_NULL}]
    The \id{ida\_mem} pointer is \id{NULL}.
  \item[\Id{IDA\_NO\_SENS}]
    The function \id{IDASensInit} has not been previously called.
  \item[\Id{IDA\_ILL\_INPUT}]
    \id{IDASolve} has been already called.
  \end{args}
}
{
  If the consistent sensitivity vectors or consistent derivative vectors
  are not desired, pass \id{NULL} for the corresponding argument.

  {\warn} The user must allocate space for \id{yyS0\_mod} and \id{ypS0\_mod}
  (if not \id{NULL}).
}
%%
%%
%-------------------------------------------------------------------
\section{User-supplied routines for forward sensitivity analysis}
\label{s:user_fct_fwd}
%-------------------------------------------------------------------

In addition to the required and optional user-supplied routines described
in \S\ref{ss:user_fct_sim}, when using {\idas} for forward sensitivity analysis,
the user has the option of providing a routine that calculates the residual 
of the sensitivity equations (\ref{e:sens_eqns}).

By default, {\idas} uses difference quotient approximation routines for the 
residual of the sensitivity equations. However, {\idas} allows the option for 
user-defined sensitivity residual routines (which also provides a mechanism for 
interfacing {\idas} to routines generated by automatic differentiation).

\index{forward sensitivity analysis!residual evaluation|(}
\index{residual function!forward sensitivity|(}

The user may provide the residuals of the sensitivity equations
(\ref{e:sens_eqns}), for all sensitivity parameters at once, through a function 
of type \ID{IDASensResFn} defined by:
%%
\usfunction{IDASensResFn}
{
  typedef int (*IDASensResFn)(&int Ns, realtype t, \\
                              &N\_Vector yy, N\_Vector yp, N\_Vector resval, \\ 
                              &N\_Vector *yS, N\_Vector *ypS, \\
                              &N\_Vector *resvalS, void *user\_data,  \\
                              &N\_Vector tmp1, N\_Vector tmp2, N\_Vector tmp3);
}
{
  This function computes the sensitivity residual for all sensitivity equations.
  It must compute the vectors $(\dFdyI)s_i(t) + (\dFdypI) \dot{s}_i(t) + (\dFdpiI)$ 
  and store them in \id{resvalS[i]}. 
}
{
  \begin{args}[user\_data]
  \item[t]
    is the current value of the independent variable.
  \item[yy]
    is the current value of the state vector, $y(t)$.
  \item[yp]
    is the current value of $\dot{y}(t)$.
  \item[resval]
    contains the current value $F$ of the original DAE residual.
  \item[yS]
    contains the current values of the sensitivities $s_i$.
  \item[ypS]
    contains the current values of the sensitivity derivatives $\dot{s}_i$.
  \item[resvalS]
    contains the output sensitivity residual vectors.
  \item[user\_data]
    is a pointer to user data.
  \item[tmp1]
  \item[tmp2]
  \item[tmp3]
    are \id{N\_Vector}s of length $N$ which can be used as temporary storage.
  \end{args}
}
{
  An \id{IDASensResFn} should return 0 if successful, a positive value if a recoverable
  error occurred (in which case {\idas} will attempt to correct), or a negative 
  value if it failed unrecoverably (in which case the integration is halted and
  \Id{IDA\_SRES\_FAIL} is returned).
}
{
  There is one situation in which recovery is not possible even if
  \id{IDASensResFn} function returns a recoverable error flag.  That is
  when this occurs at the very first call to the \id{IDASensResFn}, in
  which case {\idas} returns \Id{IDA\_FIRST\_RES\_FAIL}.
}

%%
%%==============================================================================
\section{Integration of quadrature equations depending on forward sensitivities}
\label{s:forward_quad_usage}
%%==============================================================================
%%

{\idas} provides support for integration of quadrature equations that depends not only 
on the state variables but also on forward sensitivities.

The following is an overview of the sequence of calls in a user's main program in 
this situation. Steps that are unchanged from the skeleton program presented in 
\S\ref{s:forward_usage} are grayed out.  See also \S\ref{ss:purequads}.

\index{User main program!integration of sensitivitiy-dependent quadratures}
\begin{Steps}
  
\item 
  \textcolor{gray}{\bf Initialize parallel or multi-threaded environment}

\item
  \textcolor{gray}{\bf Set problem dimensions, etc.}

\item
  \textcolor{gray}{\bf Set vectors of initial values}
 
\item
  \textcolor{gray}{\bf Create {\idas} object}

\item
  \textcolor{gray}{\bf Initialize {\idas} solver}

\item
  \textcolor{gray}{\bf Specify integration tolerances}

\item
  \textcolor{gray}{\bf Set optional inputs}

\item
  \textcolor{gray}{\bf Create matrix object}

\item
  \textcolor{gray}{\bf Create linear solver object}

\item
  \textcolor{gray}{\bf Set linear solver optional inputs}

\item
  \textcolor{gray}{\bf Initialize sensitivity-independent quadrature problem}

\item\label{i:quad_sens_sens_def}
  \textcolor{gray}{\bf Define the sensitivity problem}

\item
  \textcolor{gray}{\bf Set sensitivity initial conditions}

\item
  \textcolor{gray}{\bf Activate sensitivity calculations}

\item
  \textcolor{gray}{\bf Set sensitivity analysis optional inputs}

\item
  {\bf Set vector of initial values for quadrature variables}

  Typically, the quadrature variables should be initialized to $0$.

\item\label{i:quad_sens_init}
  {\bf Initialize sensitivity-dependent quadrature integration}

  Call \id{IDAQuadSensInit} to specify the quadrature equation right-hand
  side function and to allocate internal memory related to quadrature integration. 
  See \S\ref{ss:quad_sens_init} for details.

\item\label{i:quad_sens_optional_input}
  {\bf Set optional inputs for sensitivity-dependent quadrature integration}

  Call \id{IDASetQuadSensErrCon} to indicate whether or not quadrature variables
  should be used in the step size control mechanism. If so, one of the 
  \id{IDAQuadSens*tolerances} functions  must be called to specify the integration 
  tolerances for quadrature variables.
  See \S\ref{ss:quad_sens_optional_input} for details.

\item\label{i:quad_sens_ida_solve}
  \textcolor{gray}{\bf Advance solution in time}

\item
  {\bf Extract sensitivity-dependent quadrature variables}

  Call \id{IDAGetQuadSens}, \id{IDAGetQuadSens1}, \id{IDAGetQuadSensDky} or 
  \id{IDAGetQuadSensDky1} to obtain the values of the quadrature variables or their 
  derivatives at the current time. See \S\ref{ss:quad_sens_get} for details.

\item
  \textcolor{gray}{\bf Get optional outputs}

\item
  \textcolor{gray}{\bf Extract sensitivity solution}

\item
  {\bf Get sensitivity-dependent quadrature optional outputs}

  Call \id{IDAGetQuadSens*} functions to obtain optional output related to
  the integration of sensitivity-dependent quadratures.
  See \S\ref{ss:quad_sens_optional_output} for details.

\item
  \textcolor{gray}{\bf Deallocate memory for solutions vector}

\item 
  \textcolor{gray}{\bf Deallocate memory for sensitivity vectors}

\item
  {\bf Deallocate memory for sensitivity-dependent quadrature variables}
  
\item
  \textcolor{gray}{\bf Free solver memory}

\item
  \textcolor{gray}{\bf Free vector specification memory}

\item
  \textcolor{gray}{\bf Free linear solver and matrix memory}
  
\item 
  \textcolor{gray}{\bf Finalize MPI, if used}
  
\end{Steps}
%%
Note: \id{IDAQuadSensInit} (step \ref{i:quad_sens_init} above) can be called and
quadrature-related optional inputs (step \ref{i:quad_sens_optional_input} above)
can be set, anywhere between steps \ref{i:quad_sens_sens_def} and
\ref{i:quad_sens_ida_solve}.

%%===================================================================================

\subsection{Sensitivity-dependent quadrature initialization and deallocation}
\label{ss:quad_sens_init}

The function \id{IDAQuadSensInit} activates integration of quadrature equations depending
on sensitivities and allocates internal memory related to these calculations.
If \id{rhsQS} is input as \id{NULL}, then {\idas} uses an internal function that
computes difference quotient approximations to the functions
$\bar q_i = (\partial q / \partial y) s_i + (\partial q / \partial \dot{y}) \dot{s}_i +
 \partial q / \partial p_i$, in the notation of (\ref{e:QUAD}).
The form of the call to this function is as follows:
%%
\ucfunction{IDAQuadSensInit}
{
flag = IDAQuadSensInit(ida\_mem, rhsQS, yQS0);
}
{
  The function \ID{IDAQuadSensInit} provides required problem specifications,
  allocates internal memory, and initializes quadrature integration.
}
{
  \begin{args}[ida\_mem]
  \item[ida\_mem] (\id{void *})
    pointer to the {\idas} memory block returned by \id{IDACreate}.
  \item[rhsQS] (\Id{IDAQuadSensRhsFn})
    is the {\CC} function which computes $f_{QS}$, the right-hand side of the 
    sensitivity-dependent quadrature equations
    (for full details see \S\ref{ss:user_fct_quad_sens}).
  \item[yQS0] (\id{N\_Vector *})
    contains the initial values of sensitivity-dependent quadratures.
  \end{args}
}
{
  The return value \id{flag} (of type \id{int}) will be one of the following:
  \begin{args}[IDA\_ILL\_INPUT]
  \item[\Id{IDA\_SUCCESS}]
    The call to \id{IDAQuadSensInit} was successful.
  \item[\Id{IDA\_MEM\_NULL}] 
    The {\idas} memory was not initialized by a prior call to \id{IDACreate}.
  \item[\Id{IDA\_MEM\_FAIL}] 
    A memory allocation request failed.
  \item[\Id{IDA\_NO\_SENS}] 
    The sensitivities were not initialized by a prior call to \id{IDASensInit}.
  \item[\Id{IDA\_ILL\_INPUT}]
    The parameter \id{yQS0} is \id{NULL}.
  \end{args}
}
{
 
 {\warn}Before calling \id{IDAQuadSensInit}, the user must enable the sensitivites
  by calling \\ \id{IDASensInit}.

  If an error occurred, \id{IDAQuadSensInit} also sends an error message to the
  error handler function.
}
%%
\index{memory requirements!IDAS@{\idas} solver}
In terms of the number of quadrature variables $N_q$ and maximum method order \id{maxord},
the size of the real workspace is increased as follows:
\begin{itemize}
\item Base value: \id{lenrw} $=$ \id{lenrw} $+$ (\id{maxord+5})$N_q$
\item If \id{IDAQuadSensSVtolerances} is called: \id{lenrw} $=$ \id{lenrw} $+ N_q N_s$ 
\end{itemize}
and the size of the integer workspace is increased as follows:
\begin{itemize}
\item Base value: \id{leniw} $=$ \id{leniw} $+$ (\id{maxord+5})$N_q$
\item If \id{IDAQuadSensSVtolerances} is called: \id{leniw} $=$ \id{leniw} $+ N_q N_s$ 
\end{itemize}

The function \id{IDAQuadSensReInit}, useful during the solution of a sequence 
of problems of same size, reinitializes the quadrature related internal memory 
and must follow a call to \Id{IDAQuadSensInit}. The number \id{Nq} of 
quadratures as well as the number \id{Ns} of sensitivities are assumed to be 
unchanged from the prior call to \id{IDAQuadSensInit}.
The call to the \id{IDAQuadSensReInit} function has the form:
%%
\ucfunction{IDAQuadSensReInit}
{
  flag = IDAQuadSensReInit(ida\_mem, yQS0);
}
{
  The function \ID{IDAQuadSensReInit} provides required problem specifications 
  and reinitializes the sensitivity-dependent quadrature integration.
}
{
  \begin{args}[ida\_mem]
  \item[ida\_mem] (\id{void *})
    pointer to the {\idas} memory block.
  \item[yQS0] (\id{N\_Vector *})
    contains the initial values of sensitivity-dependent quadratures.
  \end{args}
}
{
  The return value \id{flag} (of type \id{int}) will be one of the following:
  \begin{args}[IDA\_NO\_QUADSENS]
  \item[\Id{IDA\_SUCCESS}]
    The call to \id{IDAQuadSensReInit} was successful.
  \item[\Id{IDA\_MEM\_NULL}] 
    The {\idas} memory was not initialized by a prior call to \id{IDACreate}.
  \item[\Id{IDA\_NO\_SENS}] 
    Memory space for the sensitivity calculation was not allocated by a prior
    call to \id{IDASensInit}.
  \item[\Id{IDA\_NO\_QUADSENS}] 
    Memory space for the sensitivity quadratures integration was not allocated by a prior
    call to \id{IDAQuadSensInit}.
  \item[\Id{IDA\_ILL\_INPUT}] 
    The parameter \id{yQS0} is \id{NULL}.
  \end{args}
}
{
  If an error occurred, \id{IDAQuadSensReInit} also sends an error message to the
  error handler function.
}
%%
%%
\ucfunction{IDAQuadSensFree}
{
  IDAQuadSensFree(ida\_mem);
}
{
  The function \ID{IDAQuadSensFree} frees the memory allocated for sensitivity 
  quadrature integration.
}
{
  The argument is the pointer to the {\idas} memory block (of type \id{void *}).
}
{
  The function \id{IDAQuadSensFree} has no return value.
}
{
  In general, \id{IDAQuadSensFree} need not be called by the user as it is
  called automatically by \id{IDAFree}.
}


%%===================================================================================

\subsection{IDAS solver function}

Even if quadrature integration was enabled, the call to the main solver 
function \id{IDASolve} is exactly the same as in \S\ref{sss:idasolve}. However, in this
case the return value \id{flag} can also be one of the following:
\begin{args}[IDA\_FIRST\_QRHS\_FAIL]
\item[\Id{IDA\_QSRHS\_FAIL}]
  The sensitivity quadrature right-hand side function failed in an unrecoverable manner.
\item[\Id{IDA\_FIRST\_QSRHS\_ERR}]
  The sensitivity quadrature right-hand side function failed at the first call.
\item[\Id{IDA\_REP\_QSRHS\_ERR}]
  Convergence test failures occurred too many times due to repeated recoverable errors in
  the quadrature right-hand side function. The \id{IDA\_REP\_RES\_ERR} will also
  be returned if the quadrature right-hand side function had repeated recoverable errors
  during the estimation of an initial step size (assuming the sensitivity quadrature
  variables are included in the error tests).
  \end{args}

%%===================================================================================

\subsection{Sensitivity-dependent quadrature extraction functions}\label{ss:quad_sens_get}

If sensitivity-dependent quadratures have been initialized by a call to \id{IDAQuadSensInit},
or reinitialized by a call to \id{IDAQuadSensReInit}, then {\idas} computes a solution,
sensitivities, and quadratures depending on sensitivities at time \id{t}. However, 
\id{IDASolve} will still return only the solutions $y$ and $\dot{y}$. Sensitivity-dependent 
quadratures can be obtained using one of the following functions:
%%
%%
\ucfunction{IDAGetQuadSens}
{
  flag = IDAGetQuadSens(ida\_mem, \&tret, yQS);
}
{
  The function \ID{IDAGetQuadSens} returns the quadrature sensitivity
  solution vectors after a successful return from \id{IDASolve}.
}
{
  \begin{args}[ida\_mem]
  \item[ida\_mem] (\id{void *})
    pointer to the memory previously allocated by \id{IDAInit}.
  \item[tret] (\id{realtype})
    the time reached by the solver (output).
  \item[yQS] (\id{N\_Vector *})
    array of \id{Ns} computed sensitivity-dependent quadrature vectors.
  \end{args}
}
{
  The return value \id{flag} of \id{IDAGetQuadSens} is one of:
  \begin{args}[IDA\_NO\_QUADSENS]
  \item[\Id{IDA\_SUCCESS}] 
    \id{IDAGetQuadSens} was successful.
  \item[IDA\_MEM\_NULL] 
    \id{ida\_mem} was NULL.
  \item[IDA\_NO\_SENS]
    Sensitivities were not activated.
  \item[\Id{IDA\_NO\_QUADSENS}] 
    Quadratures depending on the sensitivities were not activated.
  \item[IDA\_BAD\_DKY]
    \id{yQS} or one of the \id{yQS[i]} is \id{NULL}.
  \end{args}
}
{}
%%
%%
\index{optional output!interpolated sensitivity-dep. quadratures}
The function \Id{IDAGetQuadSensDky} computes the \id{k}-th derivatives of the interpolating 
polynomials for the sensitivity-dependent quadrature variables at time \id{t}.
This function is called by \id{IDAGetQuadSens} with \id{k = 0}, but may also be called 
directly by the user.
%%
\ucfunction{IDAGetQuadSensDky}
{
  flag = IDAGetQuadSensDky(ida\_mem, t, k, dkyQS);
}
{
  The function \ID{IDAGetQuadSensDky} returns derivatives of the quadrature sensitivities 
  solution vectors after a successful return from \id{IDASolve}.
}
{
  \begin{args}[ida\_mem]
  \item[\id{ida\_mem}] (\id{void *})
    pointer to the memory previously allocated by \id{IDAInit}.
  \item[\id{t}] (\id{realtype})
    the time at which information is requested. The time \id{t} must fall within the 
    interval defined by the last successful step taken by {\idas}.
  \item[\id{k}] (\id{int}) order of the requested derivative.
  \item[\id{dkyQS}] (\id{N\_Vector *})
    array of \id{Ns} vectors containing the derivatives. This vector
    array must be allocated by the user. 
  \end{args}
}
{
  The return value \id{flag} of \id{IDAGetQuadSensDky} is one of:
  \begin{args}[IDA\_NO\_QUADSENS]
  \item[\Id{IDA\_SUCCESS}]
    \id{IDAGetQuadSensDky} succeeded.
  \item[\Id{IDA\_MEM\_NULL}]
    \id{ida\_mem} was \id{NULL}.
  \item[IDA\_NO\_SENS]
    Sensitivities were not activated.
  \item[\Id{IDA\_NO\_QUADSENS}] 
    Quadratures depending on the sensitivities were not activated.
  \item[\Id{IDA\_BAD\_DKY}] 
    \id{dkyQS} or one of the vectors \id{dkyQS[i]} is \id{NULL}.
  \item[\Id{IDA\_BAD\_K}]
    \id{k} is not in the range $0, 1, ..., klast$.
  \item[\Id{IDA\_BAD\_T}] 
    The time \id{t} is not in the allowed range.
  \end{args}
}
{}
%%
%%
Quadrature sensitivity solution vectors can also be extracted separately for 
each parameter in turn through the functions \id{IDAGetQuadSens1} and
\id{IDAGetQuadSensDky1}, defined as follows:
\ucfunction{IDAGetQuadSens1}
{
  flag = IDAGetQuadSens1(ida\_mem, \&tret, is, yQS);
}
{
  The function \ID{IDAGetQuadSens1} returns the \id{is}-th sensitivity 
  of quadratures after a successful return from \id{IDASolve}.
}
{
  \begin{args}[ida\_mem]
  \item[ida\_mem] (\id{void *})
    pointer to the memory previously allocated by \id{IDAInit}.
  \item[tret] (\id{realtype})
    the time reached by the solver (output).
  \item[is] (\id{int}) specifies which sensitivity vector is to be returned
    ($0\le$\id{is}$< N_s$).
  \item[yQS] (\id{N\_Vector})
    the computed sensitivity-dependent quadrature vector.
  \end{args}
}
{
  The return value \id{flag} of \id{IDAGetQuadSens1} is one of:
  \begin{args}[IDA\_NO\_QUADSENS]
  \item[\Id{IDA\_SUCCESS}]
    \id{IDAGetQuadSens1} was successful.
  \item[\Id{IDA\_MEM\_NULL}] 
    \id{ida\_mem} was \id{NULL}.
  \item[\Id{IDA\_NO\_SENS}] 
    Forward sensitivity analysis was not initialized.
  \item[\Id{IDA\_NO\_QUADSENS}] 
    Quadratures depending on the sensitivities were not activated.
  \item[\Id{IDA\_BAD\_IS}]
    The index \id{is} is not in the allowed range.
  \item[\Id{IDA\_BAD\_DKY}] 
    \id{yQS} is \id{NULL}.
  \end{args}
}
{}
%%
%%
\ucfunction{IDAGetQuadSensDky1}
{
  flag = IDAGetQuadSensDky1(ida\_mem, t, k, is, dkyQS);
}
{
  The function \ID{IDAGetQuadSensDky1} returns the \id{k}-th derivative of the 
  \id{is}-th sensitivity solution vector after a successful 
  return from \id{IDASolve}.
}
{
  \begin{args}[ida\_mem]
  \item[ida\_mem] (\id{void *})
    pointer to the memory previously allocated by \id{IDAInit}.
  \item[t] (\id{realtype})
    specifies the time at which sensitivity information is 
    requested. The time \id{t} must fall within the interval defined by the last 
    successful step taken by {\idas}.
  \item[k] (\id{int}) order of derivative.
  \item[is] (\id{int}) specifies the sensitivity derivative vector to be returned
    ($0\le$\id{is}$< N_s$).
  \item[dkyQS] (\id{N\_Vector})
    the vector containing the derivative. The space for \id{dkyQS} must be allocated by 
    the user. 
  \end{args}
}
{
  The return value \id{flag} of \id{IDAGetQuadSensDky1} is one of:
  \begin{args}[IDA\_NO\_QUADSENS]
  \item[\Id{IDA\_SUCCESS}] 
    \id{IDAGetQuadDky1} succeeded.
  \item[\Id{IDA\_MEM\_NULL}] 
    \id{ida\_mem} was \id{NULL}.
  \item[\Id{IDA\_NO\_SENS}] 
    Forward sensitivity analysis was not initialized.
  \item[\Id{IDA\_NO\_QUADSENS}] 
    Quadratures depending on the sensitivities were not activated.
  \item[\Id{IDA\_BAD\_DKY}] 
    \id{dkyQS} is \id{NULL}.
  \item[\Id{IDA\_BAD\_IS}]
    The index \id{is} is not in the allowed range.
  \item[\Id{IDA\_BAD\_K}] 
    \id{k} is not in the range $0, 1, ..., klast$.
  \item[\Id{IDA\_BAD\_T}] 
    The time \id{t} is not in the allowed range.
  \end{args}
}
{}
%%
%%


%%===================================================================================

\subsection{Optional inputs for sensitivity-dependent quadrature integration}
\label{ss:quad_sens_optional_input}
\index{optional input!sensitivity-dependent quadrature integration|(}
{\idas} provides the following optional input functions to control the integration
of sensitivity-dependent quadrature equations.
%%
%%
\ucfunction{IDASetQuadSensErrCon}
{
 flag = IDASetQuadSensErrCon(ida\_mem, errconQS)
}
{
  The function \ID{IDASetQuadSensErrCon} specifies whether or not the
  quadrature variables are to be used in the local error control mechanism.
  If they are, the user must specify the error tolerances for the quadrature
  variables by calling \id{IDAQuadSensSStolerances},
  \id{IDAQuadSensSVtolerances}, or \id{IDAQuadSensEEtolerances}. 
}
{
  \begin{args}[errconQS]
  \item[ida\_mem] (\id{void *})
    pointer to the {\idas} memory block.
  \item[errconQS] (\id{booleantype})
    specifies whether sensitivity quadrature variables are included (\id{SUNTRUE}) or not
    (\id{SUNFALSE}) in the error control mechanism. 
  \end{args}
}
{
  The return value \id{flag} (of type \id{int}) is one of:
  \begin{args}[IDA\_NO\_QUADSENS]
  \item[\Id{IDA\_SUCCESS}] 
    The optional value has been successfully set.
  \item[\Id{IDA\_MEM\_NULL}]
    The \id{ida\_mem} pointer is \id{NULL}.
  \item[IDA\_NO\_SENS]
    Sensitivities were not activated.
  \item[\Id{IDA\_NO\_QUADSENS}] 
    Quadratures depending on the sensitivities were not activated.
  \end{args}
}
{
  By default, \id{errconQS} is set to \id{SUNFALSE}. 

  {\warn}It is illegal to call \id{IDASetQuadSensErrCon} before a call 
  to \id{IDAQuadSensInit}.
}


If the quadrature variables are part of the step size control mechanism, 
one of the following functions must be called to specify the
integration tolerances for quadrature variables. 

\ucfunction{IDAQuadSensSStolerances}
{
 flag = IDAQuadSensSVtolerances(ida\_mem, reltolQS, abstolQS);
}
{
  The function \ID{IDAQuadSensSStolerances} specifies scalar relative and absolute
  tolerances.
}
{
  \begin{args}[abstolQS]
  \item[ida\_mem] (\id{void *})
    pointer to the {\idas} memory block.
  \item[reltolQS] (\id{realtype})
    \index{tolerances}
    is the scalar relative error tolerance.
  \item[abstolQS] (\id{realtype*})
    is a pointer to an array containing the \id{Ns} scalar absolute 
    error tolerances.
  \end{args}
}
{
  The return value \id{flag} (of type \id{int}) is one of:
  \begin{args}[IDA\_NO\_QUADSENS]
  \item[\Id{IDA\_SUCCESS}] 
    The optional value has been successfully set.
  \item[\Id{IDA\_MEM\_NULL}]
    The \id{ida\_mem} pointer is \id{NULL}.
  \item[IDA\_NO\_SENS]
    Sensitivities were not activated.
  \item[\Id{IDA\_NO\_QUADSENS}] 
    Quadratures depending on the sensitivities were not activated.
  \item[\Id{IDA\_ILL\_INPUT}] 
    One of the input tolerances was negative.
  \end{args}
}
{}

\ucfunction{IDAQuadSensSVtolerances}
{
 flag = IDAQuadSensSVtolerances(ida\_mem, reltolQS, abstolQS);
}
{
  The function \ID{IDAQuadSensSVtolerances} specifies scalar relative and 
  vector absolute tolerances.
}
{
  \begin{args}[reltolQS]
  \item[ida\_mem] (\id{void *})
    pointer to the {\idas} memory block.
  \item[reltolQS] (\id{realtype})
    \index{tolerances}
    is the scalar relative error tolerance.
  \item[abstolQS] (\id{N\_Vector*})
    is an array of \id{Ns} variables of type \id{N\_Vector}. The \id{N\_Vector} from
    \id{abstolS[is]} specifies the vector tolerances for \id{is}-th quadrature 
    sensitivity.
  \end{args}
}
{
  The return value \id{flag} (of type \id{int}) is one of:
  \begin{args}[IDA\_NO\_QUADSENS]
  \item[\Id{IDA\_SUCCESS}] 
    The optional value has been successfully set.
  \item[\Id{IDA\_NO\_QUAD}]
    Quadrature integration was not initialized.
  \item[\Id{IDA\_MEM\_NULL}]
    The \id{ida\_mem} pointer is \id{NULL}.
  \item[IDA\_NO\_SENS]
    Sensitivities were not activated.
  \item[\Id{IDA\_NO\_QUADSENS}] 
    Quadratures depending on the sensitivities were not activated.
  \item[\Id{IDA\_ILL\_INPUT}] 
    One of the input tolerances was negative.
  \end{args}
}
{}

\ucfunction{IDAQuadSensEEtolerances}
{
  flag = IDAQuadSensEEtolerances(ida\_mem);
}
{
  The function \ID{IDAQuadSensEEtolerances} specifies that the tolerances for
  the sensitivity-dependent quadratures should be estimated from those provided
  for the pure quadrature variables. 
}
{
  \begin{args}[ida\_mem]
  \item[ida\_mem] (\id{void *})
    pointer to the {\idas} memory block.
  \end{args}
}
{
  The return value \id{flag} (of type \id{int}) is one of:
  \begin{args}[IDA\_MEM\_NULL]
  \item[\Id{IDA\_SUCCESS}] 
    The optional value has been successfully set.
  \item[\Id{IDA\_MEM\_NULL}]
    The \id{ida\_mem} pointer is \id{NULL}.
  \item[IDA\_NO\_SENS]
    Sensitivities were not activated.
  \item[\Id{IDA\_NO\_QUADSENS}] 
    Quadratures depending on the sensitivities were not activated.
  \end{args}
}
{ 
  When \ID{IDAQuadSensEEtolerances}  is used, before calling \Id{IDASolve}, 
  integration of pure quadratures must be initialized (see \ref{ss:quad_init})
  and tolerances for pure quadratures must be also specified 
  (see \ref{ss:quad_optional_input}).
  
}
%%
\index{optional input!sensitivity-dependent quadrature integration|)}
%%
%%

%%===================================================================================

\subsection{Optional outputs for sensitivity-dependent quadrature integration}
\label{ss:quad_sens_optional_output}
\index{optional output!sensitivity-dependent quadrature integration|(}

{\idas} provides the following functions that can be used to obtain solver
performance information related to quadrature integration.

\ucfunction{IDAGetQuadSensNumRhsEvals}
{
  flag = IDAGetQuadSensNumRhsEvals(ida\_mem, \&nrhsQSevals);
}
{
  The function \ID{IDAGetQuadSensNumRhsEvals} returns the 
  number of calls made to the user's quadrature right-hand side function.
}
{
  \begin{args}[nrhsQSevals]
  \item[ida\_mem] (\id{void *})
    pointer to the {\idas} memory block.
  \item[nrhsQSevals] (\id{long int})
    number of calls made to the user's \id{rhsQS} function.
  \end{args}
}
{
  The return value \id{flag} (of type \id{int}) is one of:
  \begin{args}[IDA\_NO\_QUADSENS]
  \item[\Id{IDA\_SUCCESS}] 
    The optional output value has been successfully set.
  \item[\Id{IDA\_MEM\_NULL}]
    The \id{ida\_mem} pointer is \id{NULL}.
  \item[\Id{IDA\_NO\_QUADSENS}]
    Sensitivity-dependent quadrature integration has not been initialized.
  \end{args}
}
{}
%%
%%
\ucfunction{IDAGetQuadSensNumErrTestFails}
{
  flag = IDAGetQuadSensNumErrTestFails(ida\_mem, \&nQSetfails);
}
{
  The function \ID{IDAGetQuadSensNumErrTestFails} returns the
  number of local error test failures due to quadrature variables.
}
{
  \begin{args}[nQSetfails]
  \item[ida\_mem] (\id{void *})
    pointer to the {\idas} memory block.
  \item[nQSetfails] (\id{long int})
    number of error test failures due to quadrature variables.
  \end{args}
}
{
  The return value \id{flag} (of type \id{int}) is one of:
  \begin{args}[IDA\_NO\_QUADSENS]
  \item[\Id{IDA\_SUCCESS}] 
    The optional output value has been successfully set.
  \item[\Id{IDA\_MEM\_NULL}]
    The \id{ida\_mem} pointer is \id{NULL}.
  \item[\Id{IDA\_NO\_QUADSENS}]
    Sensitivity-dependent quadrature integration has not been initialized.
  \end{args}
}
{}
%%
%%
\ucfunction{IDAGetQuadSensErrWeights}
{
  flag = IDAGetQuadSensErrWeights(ida\_mem, eQSweight);
}
{
  The function \ID{IDAGetQuadSensErrWeights} returns the quadrature error weights 
  at the current time.
}
{
  \begin{args}[eQSweight]
  \item[ida\_mem] (\id{void *})
    pointer to the {\idas} memory block.
  \item[eQSweight] (\id{N\_Vector *})
    array of quadrature error weight vectors at the current time.
  \end{args}
}
{
  The return value \id{flag} (of type \id{int}) is one of:
  \begin{args}[IDA\_MEM\_NULL]
  \item[\Id{IDA\_SUCCESS}] 
    The optional output value has been successfully set.
  \item[\Id{IDA\_MEM\_NULL}]
    The \id{ida\_mem} pointer is \id{NULL}.
  \item[\Id{IDA\_NO\_QUADSENS}]
    Sensitivity-dependent quadrature integration has not been initialized.
  \end{args}
}
{
  {\warn}The user must allocate memory for \id{eQSweight}.

  If quadratures were not included in the error control mechanism (through a 
  call to \id{IDASetQuadSensErrCon} with \id{errconQS=SUNTRUE}), 
  \id{IDAGetQuadSensErrWeights} does not set the \id{eQSweight} vector.
}
%%
%%
\ucfunction{IDAGetQuadSensStats}
{
  flag = IDAGetQuadSensStats(ida\_mem, \&nrhsQSevals, \&nQSetfails);
}
{
  The function \ID{IDAGetQuadSensStats} returns the {\idas} integrator statistics
  as a group.
}
{
  \begin{args}[nrhsQSevals]
  \item[ida\_mem] (\id{void *})
    pointer to the {\idas} memory block.
  \item[nrhsQSevals] (\id{long int})
    number of calls to the user's \id{rhsQS} function.
  \item[nQSetfails] (\id{long int})
    number of error test failures due to quadrature variables.
  \end{args}
}
{
  The return value \id{flag} (of type \id{int}) is one of
  \begin{args}[IDA\_NO\_QUADSENS]
  \item[\Id{IDA\_SUCCESS}] 
    the optional output values have been successfully set.
  \item[\Id{IDA\_MEM\_NULL}]
    the \id{ida\_mem} pointer is \id{NULL}.
  \item[\Id{IDA\_NO\_QUADSENS}]
    Sensitivity-dependent quadrature integration has not been initialized.
  \end{args}
}
{}
\index{optional output!sensitivity-dependent quadrature integration|)}
%%
%%

%%===================================================================================

\subsection{User-supplied function for sensitivity-dependent quadrature integration}
\label{ss:user_fct_quad_sens}

\index{right-hand side function!sensitivity-dependent quadrature equations}
For the integration of sensitivity-dependent quadrature equations, the user must provide
a function that defines the right-hand side of the sensitivity quadrature equations.
For sensitivities of quadratures (\ref{e:QUAD}) with integrands $q$, the appropriate
right-hand side functions are given by
$\bar q_i = (\partial q / \partial y) s_i + (\partial q / \partial \dot{y}) \dot{s}_i +
 \partial q / \partial p_i$.
This user function must be of type \Id{IDAQuadSensRhsFn}, defined as follows:
\usfunction{IDAQuadSensRhsFn}
{
  typedef int (*IDAQuadSensRhsFn)(&int Ns, realtype t, N\_Vector yy,\\
                             &N\_Vector yp, N\_Vector *yyS, N\_Vector *ypS,\\
                             &N\_Vector rrQ, N\_Vector *rhsvalQS,\\
                             &void *user\_data, N\_Vector tmp1,\\
                             &N\_Vector tmp2, N\_Vector tmp3)
}
{
  This function computes the sensitivity quadrature equation right-hand side for a given value
  of the independent variable $t$ and state vector $y$.
}
{
  \begin{args}[user\_data]
  \item[Ns]
    is the number of sensitivity vectors.
  \item[t]
    is the current value of the independent variable.
  \item[yy]
    is the current value of the dependent variable vector, $y(t)$.
  \item[yp]
    is the current value of the dependent variable vector, $\dot{y}(t)$.
  \item[yyS] 
    is an array of \id{Ns} variables of type \id{N\_Vector} containing the
    dependent sensitivity vectors $s_i$.
  \item[ypS] 
    is an array of \id{Ns} variables of type \id{N\_Vector} containing the
    dependent sensitivity derivatives $\dot{s}_i$.
  \item[rrQ]
    is the current value of the quadrature right-hand side $q$.
  \item[rhsvalQS]
    contains the \id{Ns} output vectors.
  \item[user\_data]
    is the \Id{user\_data} pointer passed to \id{IDASetUserData}.   
  \item[tmp1]
  \item[tmp2]
  \item[tmp3]
    are \id{N\_Vector}s which can be used as temporary storage.
  \end{args}
}
{
  An \id{IDAQuadSensRhsFn} should return 0 if successful, a positive value if a recoverable
  error occurred (in which case {\idas} will attempt to correct), or a negative 
  value if it failed unrecoverably (in which case the integration is halted and
  \Id{IDA\_QRHS\_FAIL} is returned).
}
{
  Allocation of memory for \id{rhsvalQS} is automatically handled within {\idas}.

  Both \id{yy} and \id{yp} are of type \id{N\_Vector} and both \id{yyS} and \id{ypS}
  are pointers to an array containing \id{Ns} vectors of type \id{N\_Vector}.
  It is the user's responsibility to access the vector data consistently 
  (including the use of the correct accessor macros from each {\nvector} implementation). 
  For the sake of computational efficiency, the vector functions in the two {\nvector} 
  implementations provided with {\idas} do not perform any consistency checks with respect 
  to their \id{N\_Vector} arguments (see \S\ref{ss:nvec_ser} and \S\ref{ss:nvec_par}).

  There is one situation in which recovery is not possible even if
  \id{IDAQuadSensRhsFn} function returns a recoverable error flag.  That is when
  this occurs at the very first call to the \id{IDAQuadSensRhsFn}, in which case
  {\idas} returns \Id{IDA\_FIRST\_QSRHS\_ERR}).
}


%%-------------------------------------------------------------------
\section{Note on using partial error control}\label{ss:partial}
%%-------------------------------------------------------------------
%%
\index{partial error control!explanation of {\idas} behavior}
For some problems, when sensitivities are excluded from the error control test, 
the behavior of {\idas} may appear at first glance to be erroneous. One would
expect that, in such cases, the sensitivity variables would not influence in
any way the step size selection.

The short explanation of this behavior is that the step size selection
implemented by the error control mechanism in {\idas} is based on the 
magnitude of the correction calculated by the nonlinear solver. As mentioned
in \S\ref{ss:sensi_init}, even with partial error control selected
in the call to \id{IDASensInit}, the sensitivity variables are included
in the convergence tests of the nonlinear solver.

When using the simultaneous corrector method (\S\ref{ss:fwd_sensi}),
the nonlinear system that is solved at each step involves both the state
and sensitivity equations. In this case, it is easy to see how the sensitivity 
variables may affect the convergence rate of the nonlinear solver and therefore
the step size selection. 
The case of the staggered corrector approach is more subtle. The sensitivity 
variables at a given step are 
computed only once the solver for the nonlinear state equations has converged.
However, if the nonlinear system corresponding to the sensitivity equations
has convergence problems, {\idas} will attempt to improve the initial guess
by reducing the step size in order to provide a better prediction of the
sensitivity variables. Moreover, even if there are no convergence failures in
the solution of the sensitivity system, {\idas} may trigger a call to the
linear solver's setup routine which typically involves reevaluation of Jacobian
information (Jacobian approximation in the case of {\idadense} and {\idaband}, or
preconditioner data in the case of the Krylov solvers). The new Jacobian information
will be used by subsequent calls to the nonlinear solver for the state equations
and, in this way, potentially affect the step size selection.

When using the simultaneous corrector method it is not possible to decide whether 
nonlinear solver convergence failures or calls to the linear solver setup routine
have been triggered by convergence problems due to the state or the sensitivity 
equations.
When using one of the staggered corrector methods, however, these situations can be 
identified by carefully monitoring the diagnostic information provided through 
optional outputs. If there are no convergence
failures in the sensitivity nonlinear solver, and none
of the calls to the linear solver setup routine were made by the sensitivity
nonlinear solver, then the step size selection is
not affected by the sensitivity variables.

Finally, the user must be warned that the effect of appending sensitivity 
equations to a given system of DAEs on the step size selection 
(through the mechanisms described above) is problem-dependent and can therefore
lead to either an increase or decrease of the total number of steps that {\idas} takes
to complete the simulation. At first glance, one would expect that the impact
of the sensitivity variables, if any, would be in the direction of increasing the
step size and therefore reducing the total number of steps. The argument for this
is that the presence of the sensitivity variables in the convergence test of the
nonlinear solver can only lead to additional iterations (and therefore a smaller
iteration error), or to additional calls to the linear solver setup routine
(and therefore more up-to-date Jacobian information), both of which will lead
to larger steps being taken by {\idas}. However, this is true only locally.
Overall, a larger integration step taken at a given time may lead to 
step size reductions at later times, due to either nonlinear solver convergence
failures or error test failures.
