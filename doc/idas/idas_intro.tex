%===================================================================================
\chapter{Introduction}\label{s:intro}
%===================================================================================

{\idas} is part of a software family called {\sundials}:
SUite of Nonlinear and DIfferential/ALgebraic equation Solvers~\cite{HBGLSSW:05}.
This suite consists of {\cvode}, {\arkode}, {\kinsol}, and {\ida}, and variants
of these with sensitivity analysis capabilities, {\cvodes} and {\idas}.

{\idas} is a general purpose solver for the initial value problem (IVP) for
systems of differential-algebraic equations (DAEs).  The name IDAS
stands for Implicit Differential-Algebraic solver with Sensitivity capabilities.
{\idas} is an extension of the {\ida} solver within {\sundials}, itself
based on {\daspk}~\cite{BHP:94,BHP:98}; however, like all {\sundials} solvers,
{\idas} is written in ANSI-standard C rather than {\F}77.
Its most notable features are that,
(1) in the solution of the underlying nonlinear system at each time
step, it offers a choice of Newton/direct methods and a choice of
Inexact Newton/Krylov (iterative) methods;
(2) it is written in a {\em data-independent} manner in that it acts
on generic vectors and matrices without any assumptions on the
underlying organization of the data; and
(3) it provides a flexible, extensible framework for sensitivity analysis,
using either {\em forward} or {\em adjoint} methods.
Thus {\idas} shares significant modules previously written within CASC
at LLNL to support the ordinary differential equation (ODE) solvers
{\cvode}~\cite{cvode_ug,CoHi:96} and {\pvode}~\cite{ByHi:98,ByHi:99},
the DAE solver {\ida}~\cite{ida_ug} on which {\idas} is based, the
sensitivity-enabled ODE solver {\cvodes}~\cite{cvodes_ug,SeHi:05}, and
also the nonlinear system solver {\kinsol}~\cite{kinsol_ug}.

At present, {\idas} may utilize a variety of Krylov methods provided
in {\sundials} that can be used in conjuction with Newton iteration:
these include the GMRES (Generalized Minimal RESidual)~\cite{SaSc:86},
FGMRES (Flexible Generalized Minimum RESidual)~\cite{Saa:93},
Bi-CGStab (Bi-Conjugate Gradient Stabilized)~\cite{Van:92}, TFQMR
(Transpose-Free Quasi-Minimal Residual)~\cite{Fre:93}, and PCG
(Preconditioned Conjugate Gradient)~\cite{HeSt:52} linear iterative
methods.  As Krylov methods, these require little
matrix storage for solving the Newton equations as compared to direct
methods. However, the algorithms allow for a user-supplied preconditioner
matrix, and, for most problems, preconditioning is essential for an
efficient solution.

For very large DAE systems, the Krylov methods are preferable over
direct linear solver methods, and are often the only feasible choice.
Among the Krylov methods in {\sundials}, we recommend GMRES as the
best overall choice.  However, users are encouraged to compare all
options, especially if encountering convergence failures with GMRES.
Bi-CGFStab and TFQMR have an advantage in storage requirements, in
that the number of workspace vectors they require is fixed, while that
number for GMRES depends on the desired Krylov subspace size.  FGMRES
has an advantage in that it is designed to support preconditioners
that vary between iterations (e.g.~iterative methods).  PCG exhibits
rapid convergence and minimal workspace vectors, but only works for
symmetric linear systems.

\index{IDAS@{\idas}!relationship to {\ida}|(}
{\idas} is written with a functionality that is a superset of that
of {\ida}. Sensitivity analysis capabilities, both
forward and adjoint, have been added to the main integrator. Enabling
forward sensitivity computations in {\idas} will result in the
code integrating the so-called {\em sensitivity equations}
simultaneously with the original IVP, yielding both the solution and
its sensitivity with respect to parameters in the model. Adjoint
sensitivity analysis, most useful when the gradients of relatively few
functionals of the solution with respect to many parameters are
sought, involves integration of the original IVP forward in time
followed by the integration of the so-called {\em adjoint equations}
backward in time. {\idas} provides the infrastructure needed to
integrate any final-condition ODE dependent on the solution of the
original IVP (in particular the adjoint system).
\index{IDAS@{\idas}!relationship to {\ida}|)}

\index{IDAS@{\idas}!motivation for writing in C|(}
There are several motivations for choosing the {\CC} language for {\idas}.
First, a general movement away from {\F} and toward {\CC} in scientific
computing was apparent.  Second, the pointer, structure, and dynamic
memory allocation features in {\CC} are extremely useful in software of
this complexity, with the great variety of method options offered.
Finally, we prefer {\CC} over {\CPP} for {\idas} because of the wider
availability of {\CC} compilers, the potentially greater efficiency of {\CC},
and the greater ease of interfacing the solver to applications written
in extended {\F}.
\index{IDAS@{\idas}!motivation for writing in C|)}


\section{Changes from previous versions}

\subsection*{Changes in v4.7.0}

A new {\nvector} implementation based on the {\sycl} abstraction layer has been
added targeting Intel GPUs. At present the only {\sycl} compiler supported is
the DPC++ (Intel oneAPI) compiler. See Section \ref{ss:nvec_sycl} for more
details. This module is considered experimental and is subject to major changes
even in minor releases.

A new {\sunmatrix} and {\sunlinsol} implementation were added to interface
with the MAGMA linear algebra library. Both the matrix and the linear solver
support general dense linear systems as well as block diagonal linear systems,
and both are targeted at GPUs (AMD or NVIDIA). See Section \ref{ss:sunlinsol_magmadense}
for more details.

\subsection*{Changes in v4.6.1}

Fixed a bug in the {\sundials} CMake which caused an error
if the CMAKE\_CXX\_STANDARD and SUNDIALS\_RAJA\_BACKENDS options
were not provided.

Fixed some compiler warnings when using the IBM XL compilers.

\subsection*{Changes in v4.6.0}

A new {\nvector} implementation based on the AMD ROCm HIP platform has been
added. This vector can target NVIDIA or AMD GPUs. See \ref{ss:nvec_hip} for more
details. This module is considered experimental and is subject to change from
version to version.

The RAJA {\nvector} implementation has been updated to support the HIP backend
in addition to the CUDA backend. Users can choose the backend when configuring
SUNDIALS by using the \id{SUNDIALS\_RAJA\_BACKENDS} CMake variable. This module
remains experimental and is subject to change from version to version.

A new optional operation, \id{N\_VGetDeviceArrayPointer}, was added to the N\_Vector
API. This operation is useful for N\_Vectors that utilize dual memory spaces,
e.g. the native SUNDIALS CUDA N\_Vector.

The SUNMATRIX\_CUSPARSE and SUNLINEARSOLVER\_CUSOLVERSP\_BATCHQR implementations
no longer require the SUNDIALS CUDA N\_Vector. Instead, they require that the vector
utilized provides the \id{N\_VGetDeviceArrayPointer} operation, and that the pointer
returned by \id{N\_VGetDeviceArrayPointer} is a valid CUDA device pointer.

\subsection*{Changes in v4.5.0}

Refactored the {\sundials} build system. CMake 3.12.0 or newer is now required.
Users will likely see deprecation warnings, but otherwise the changes
should be fully backwards compatible for almost all users. {\sundials}
now exports CMake targets and installs a SUNDIALSConfig.cmake file.

Added support for SuperLU DIST 6.3.0 or newer.

\subsection*{Changes in v4.4.0}

Added the function \id{IDASetLSNormFactor} to specify the factor for
converting between integrator tolerances (WRMS norm) and linear solver
tolerances (L2 norm) i.e., \id{tol\_L2 = nrmfac * tol\_WRMS}.

Added a new function \Id{IDAGetNonlinearSystemData} which advanced users might
find useful if providing a custom \id{SUNNonlinSolSysFn}.

\textbf{This change may cause an error in existing user code}.
The \id{IDASolveF} function for forward integration with checkpointing is now
subject to a restriction on the number of time steps allowed to reach the output
time. This is the same restriction applied to the \id{IDASolve} function. The
default maximum number of steps is 500, but this may be changed using the
\id{IDASetMaxNumSteps} function. This change fixes a bug that could cause an
infinite loop in the \id{IDASolveF} function.

The expected behavior of \id{SUNNonlinSolGetNumIters} and
\id{SUNNonlinSolGetNumConvFails} in the {\sunnonlinsol} API have been updated to
specify that they should return the number of nonlinear solver iterations and
convergence failures in the most recent solve respectively rather than the
cumulative number of iterations and failures across all solves respectively. The
API documentation and {\sundials} provided {\sunnonlinsol} implementations have
been updated accordingly. As before, the cumulative number of nonlinear
iterations may be retreived by calling \id{IDAGetNumNonlinSolvIters},
or \id{IDAGetSensNumNonlinSolvIters}, the cumulative number of failures with
\id{IDAGetNumNonlinSolvConvFails} or \id{IDAGetSensNumNonlinSolvConvFails}, or
both with \id{IDAGetNonlinSolvStats} or \id{IDAGetSensNonlinSolvStats}.

A new API, \id{SUNMemoryHelper}, was added to support \textbf{GPU users} who
have complex memory management needs such as using memory pools. This is paired
with new constructors for the {\nveccuda} and {\nvecraja} modules that accept a
\id{SUNMemoryHelper} object. Refer to sections
\ref{s:gpu_model},\ref{s:sunmemory}, \ref{ss:nvec_cuda} and \ref{ss:nvec_raja}
for more information.

The \id{NVECTOR\_RAJA} module has been updated to mirror the \id{NVECTOR\_CUDA} module.
Notably, the update adds managed memory support to the \id{NVECTOR\_RAJA} module.
Users of the module will need to update any calls to the \id{N\_VMake\_Raja} function
because that signature was changed. This module remains experimental and is
subject to change from version to version.

The \id{NVECTOR\_TRILINOS} module has been updated to work with Trilinos 12.18+.
This update changes the local ordinal type to always be an \id{int}.

Added support for CUDA v11.


\subsection*{Changes in v4.3.0}

Fixed a bug in the iterative linear solver modules where an error is not
returned if the Atimes function is \id{NULL} or, if preconditioning is enabled,
the PSolve function is \id{NULL}.

Added the ability to control the {\cuda} kernel launch parameters for the
\id{NVECTOR\_CUDA} and \id{SUNMATRIX\_CUSPARSE} modules. These modules remain
experimental and are subject to change from version to version.
In addition, the \id{NVECTOR\_CUDA} kernels were rewritten to be more flexible.
Most users should see equivalent performance or some improvement, but a select
few may observe minor performance degradation with the default settings. Users
are encouraged to contact the {\sundials} team about any perfomance changes
that they notice.

Added new capabilities for monitoring the solve phase in the {\sunnonlinsolnewton}
and {\sunnonlinsolfixedpoint} modules, and the {\sundials} iterative linear solver
modules. {\sundials} must be built with the CMake option
\id{SUNDIALS\_BUILD\_WITH\_MONITORING} to use these capabilties.

Added the optional functions \id{IDASetJacTimesResFn} and
\id{IDASetJacTimesResFnB} to specify an alternative residual function
for computing Jacobian-vector products with the internal difference quotient
approximation.


\subsection*{Changes in v4.2.0}

Fixed a build system bug related to the Fortran 2003 interfaces when using the
IBM XL compiler. When building the Fortran 2003 interfaces with an XL compiler
it is recommended to set \id{CMAKE\_Fortran\_COMPILER} to \id{f2003},
\id{xlf2003}, or \id{xlf2003\_r}.

Fixed a linkage bug affecting Windows users that stemmed from dllimport/dllexport
attributes missing on some SUNDIALS API functions.

Fixed a memory leak from not deallocating the \id{atolSmin0} and \id{atolQSmin0}
arrays.

Added a new \id{SUNMatrix} implementation, \id{SUNMATRIX\_CUSPARSE}, that interfaces
to the sparse matrix implementation from the NVIDIA cuSPARSE library. In addition,
the \id{SUNLINSOL\_CUSOLVER\_BATCHQR} linear solver has been updated to
use this matrix, therefore, users of this module will need to update their code.
These modules are still considered to be experimental, thus they are subject to
breaking changes even in minor releases.

The functions \id{IDASetLinearSolutionScaling} and
\id{IDASetLinearSolutionScalingB} were added to enable or disable the scaling
applied to linear system solutions with matrix-based linear solvers to account
for a lagged value of $\alpha$ in the linear system matrix
$\frac{\partial F}{\partial y} + \alpha\frac{\partial F}{\partial \dot{y}}$.
Scaling is enabled by default when using a matrix-based linear solver.

\subsection*{Changes in v4.1.0}

Fixed a build system bug related to finding LAPACK/BLAS.

Fixed a build system bug related to checking if the KLU library works.

Added a new build system option, \id{CUDA\_ARCH}, that can be used to specify
the CUDA architecture to compile for.

Fixed a build system bug related to finding PETSc when using the CMake
variables \id{PETSC\_INCLUDES} and \id{PETSC\_LIBRARIES} instead of
\id{PETSC\_DIR}.

Added two utility functions, \id{SUNDIALSFileOpen} and \id{SUNDIALSFileClose}
for creating/destroying file pointers that are useful when using the Fortran
2003 interfaces.

\subsection*{Changes in v4.0.0}

\subsubsection*{Build system changes}

\begin{itemize}
\item Increased the minimum required CMake version to 3.5 for most {\sundials}
configurations, and 3.10 when CUDA or OpenMP with device offloading are enabled.
%
\item The CMake option \id{BLAS\_ENABLE} and the variable \id{BLAS\_LIBRARIES} have
been removed to simplify builds as {\sundials} packages do not use BLAS
directly. For third party libraries that require linking to BLAS, the path to
the BLAS library should be included in the \id{\_LIBRARIES} variable for the
third party library \textit{e.g.}, \id{SUPERLUDIST\_LIBRARIES} when enabling
SuperLU\_DIST.
%
\item Fixed a bug in the build system that prevented the {\nvecpthreads} module from
being built.
\end{itemize}

\subsubsection*{NVECTOR module changes}

\begin{itemize}
\item Two new functions were added to aid in creating custom {\nvector} objects. The
constructor \id{N\_VNewEmpty} allocates an ``empty'' generic {\nvector} with the
object's content pointer and the function pointers in the operations structure
initialized to \id{NULL}. When used in the constructor for custom objects this
function will ease the introduction of any new optional operations to the
{\nvector} API by ensuring only required operations need to be set.
Additionally, the function \id{N\_VCopyOps(w, v)} has been added to copy the
operation function pointers between vector objects. When used in clone routines
for custom vector objects these functions also will ease the introduction of
any new optional operations to the {\nvector} API by ensuring all operations
are copied when cloning objects. See \S\ref{ss:nvecutils} for more details.
%
\item Two new {\nvector} implementations, {\nvecmanyvector} and
{\nvecmpimanyvector}, have been created to support flexible partitioning
of solution data among different processing elements (e.g., CPU + GPU) or for
multi-physics problems that couple distinct MPI-based simulations together. This
implementation is accompanied by additions to user documentation and {\sundials}
examples. See \S\ref{ss:nvec_manyvector} and \S\ref{ss:nvec_mpimanyvector} for
more details.
%
\item One new required vector operation and ten new optional vector operations have
been added to the {\nvector} API. The new required operation, \id{N\_VGetLength},
returns the global length of an \id{N\_Vector}. The optional operations have
been added to support the new \newline\noindent
{\nvecmpimanyvector} implementation. The
operation \id{N\_VGetCommunicator} must be implemented by subvectors that are
combined to create an {\nvecmpimanyvector}, but is not used outside of
this context. The remaining nine operations are optional local reduction
operations intended to eliminate unnecessary latency when performing vector
reduction operations (norms, etc.) on distributed memory systems. The optional
local reduction vector operations are
\id{N\_VDotProdLocal},
\id{N\_VMaxNormLocal},
\id{N\_VMinLocal},
\id{N\_VL1NormLocal},
\id{N\_VWSqrSumLocal},
\id{N\_VWSqrSumMaskLocal},
\id{N\_VInvTestLocal},
\id{N\_VConstrMaskLocal}, and
\id{N\_VMinQuotientLocal}.
If an {\nvector} implementation defines any of the local operations as
\id{NULL}, then the {\nvecmpimanyvector} will call standard {\nvector}
operations to complete the computation. See \S\ref{ss:nveclocalops} for more
details.
%
\item An additional {\nvector} implementation, {\nvecmpiplusx}, has been created to
support the MPI+X paradigm where X is a type of on-node parallelism
(\textit{e.g.}, OpenMP, CUDA). The implementation is accompanied by additions to
user documentation and {\sundials} examples. See \S\ref{ss:nvec_mpiplusx} for
more details.
%
\item The \id{*\_MPICuda} and \id{*\_MPIRaja} functions have been removed from the
{\nveccuda} and {\nvecraja} implementations respectively. Accordingly, the
\id{nvector\_mpicuda.h}, \newline\noindent
\id{nvector\_mpiraja.h},
\id{libsundials\_nvecmpicuda.lib}, and \id{libsundials\_nvecmpicudaraja.lib}
files have been removed. Users should use the {\nvecmpiplusx} module coupled
in conjunction with the {\nveccuda} or {\nvecraja} modules to replace the
functionality. The necessary changes are minimal and should require few
code modifications. See the programs in \id{examples/ida/mpicuda} and
\id{examples/ida/mpiraja} for examples of how to use the {\nvecmpiplusx}
module with the {\nveccuda} and {\nvecraja} modules respectively.
%
\item Fixed a memory leak in the {\nvecpetsc} module clone function.
%
\item Made performance improvements to the {\nveccuda} module. Users who utilize a
non-default stream should no longer see default stream synchronizations
after memory transfers.
%
\item Added a new constructor to the {\nveccuda} module that allows a user to provide
custom allocate and free functions for the vector data array and internal
reduction buffer. See \S\ref{ss:nvec_cuda_functions} for more details.
%
\item Added new Fortran 2003 interfaces for most {\nvector} modules. See Chapter
\ref{s:nvector} for more details on how to use the interfaces.
%
\item Added three new {\nvector} utility functions,
\id{FN\_VGetVecAtIndexVectorArray},\newline\noindent
\id{FN\_VSetVecAtIndexVectorArray}, and
\id{FN\_VNewVectorArray},
for working with \id{N\_Vector} arrays when using the Fortran 2003 interfaces.
See \S\ref{ss:nvecutils} for more details.
\end{itemize}

\subsubsection*{SUNMatrix module changes}

\begin{itemize}
\item Two new functions were added to aid in creating custom {\sunmatrix} objects. The
constructor \id{SUNMatNewEmpty} allocates an ``empty'' generic {\sunmatrix} with
the object's content pointer and the function pointers in the operations
structure initialized to \id{NULL}. When used in the constructor for custom
objects this function will ease the introduction of any new optional operations
to the {\sunmatrix} API by ensuring only required operations need to be set.
Additionally, the function \id{SUNMatCopyOps(A, B)} has been added to copy the
operation function pointers between matrix objects. When used in clone routines
for custom matrix objects these functions also will ease the introduction of any
new optional operations to the {\sunmatrix} API by ensuring all operations are
copied when cloning objects. See \S\ref{ss:sunmatrix_utilities} for more details.
%
\item A new operation, \id{SUNMatMatvecSetup}, was added to the {\sunmatrix} API
to perform any setup necessary for computing a matrix-vector product. This
operation is useful for {\sunmatrix} implementations which need to prepare the
matrix itself, or communication structures before performing the matrix-vector
product. Users who have implemented custom {\sunmatrix} modules will need to at
least update their code to set the corresponding \id{ops} structure member,
\id{matvecsetup}, to \id{NULL}. See \S\ref{ss:sunmatrix_functions} for more
details.
%
\item The generic {\sunmatrix} API now defines error codes to be returned by
{\sunmatrix} operations. Operations which return an integer flag indiciating
success/failure may return different values than previously. See
\S\ref{ss:sunmatrix_ReturnCodes} for more details.
%
\item A new {\sunmatrix} (and {\sunlinsol}) implementation was added to
facilitate the use of the SuperLU\_DIST library with {\sundials}. See
\S\ref{ss:sunmat_slunrloc} for more details.
%
\item Added new Fortran 2003 interfaces for most {\sunmatrix} modules. See Chapter
\ref{s:sunmatrix} for more details on how to use the interfaces.
\end{itemize}

\subsubsection*{SUNLinearSolver module changes}

\begin{itemize}
\item A new function was added to aid in creating custom {\sunlinsol} objects.
The constructor \id{SUNLinSolNewEmpty} allocates an ``empty'' generic
{\sunlinsol} with the object's content pointer and the function pointers
in the operations structure initialized to \id{NULL}. When used in the
constructor for custom objects this function will ease the introduction of any
new optional operations to the {\sunlinsol} API by ensuring only required
operations need to be set. See \S\ref{ss:sunlinsol_custom} for more details.
%
\item The return type of the {\sunlinsol} API function \id{SUNLinSolLastFlag}
has changed from \id{long int} to \id{sunindextype} to be consistent with the
type used to store row indices in dense and banded linear solver modules.
%
\item Added a new optional operation to the {\sunlinsol} API,
\id{SUNLinSolGetID}, that returns a \newline\noindent
\id{SUNLinearSolver\_ID} for identifying the
linear solver module.
%
\item The {\sunlinsol} API has been updated to make the initialize and setup
functions optional.
%
\item A new {\sunlinsol} (and {\sunmatrix}) implementation was added to
facilitate the use of the SuperLU\_DIST library with {\sundials}. See
\S\ref{ss:sunlinsol_sludist} for more details.
%
\item Added a new {\sunlinsol} implementation,
\id{SUNLinearSolver\_cuSolverSp\_batchQR}, which leverages the NVIDIA cuSOLVER
sparse batched QR method for efficiently solving block diagonal linear systems
on NVIDIA GPUs. See \S\ref{ss:sunlinsol_cuspbqr} for more details.
%
\item Added three new accessor functions to the {\sunlinsolklu} module,
\id{SUNLinSol\_KLUGetSymbolic},
\id{SUNLinSol\_KLUGetNumeric}, and
\id{SUNLinSol\_KLUGetCommon},
to provide user access to the underlying KLU solver structures. See
\S\ref{ss:sunlinsol_klu_functions} for more details.
%
\item Added new Fortran 2003 interfaces for most {\sunlinsol} modules. See
Chapter \ref{s:sunlinsol} for more details on how to use the interfaces.
\end{itemize}

\subsubsection*{SUNNonlinearSolver module changes}

\begin{itemize}
\item A new function was added to aid in creating custom {\sunnonlinsol}
objects. The constructor \id{SUNNonlinSolNewEmpty} allocates an ``empty''
generic {\sunnonlinsol} with the object's content pointer and the function
pointers in the operations structure initialized to \id{NULL}. When used in the
constructor for custom objects this function will ease the introduction of any
new optional operations to the {\sunnonlinsol} API by ensuring only
required operations need to be set. See \S\ref{ss:sunnonlinsol_custom} for more
details.
%
\item To facilitate the use of user supplied nonlinear solver convergence test
functions the \newline\noindent
\id{SUNNonlinSolSetConvTestFn} function in the
{\sunnonlinsol} API has been updated to take a \id{void*} data pointer as
input. The supplied data pointer will be passed to the nonlinear solver
convergence test function on each call.
%
\item The inputs values passed to the first two inputs of the \id{SUNNonlinSolSolve}
function in the {\sunnonlinsol} have been changed to be the predicted
state and the initial guess for the correction to that state. Additionally,
the definitions of \id{SUNNonlinSolLSetupFn} and \id{SUNNonlinSolLSolveFn} in
the {\sunnonlinsol} API have been updated to remove unused input
parameters. For more information on the nonlinear system formulation see
\S\ref{s:sunnonlinsol_interface} and for more details on the API functions see
Chapter \ref{c:sunnonlinsol}.
%
\item Added a new {\sunnonlinsol} implementation, {\sunnonlinsolpetsc}, which
interfaces to the PETSc SNES nonlinear solver API. See
\S\ref{s:sunnonlinsolpetsc} for more details.
%
\item Added new Fortran 2003 interfaces for most {\sunnonlinsol} modules. See
Chapter \ref{c:sunnonlinsol} for more details on how to use the interfaces.
\end{itemize}

\subsubsection*{IDAS changes}

\begin{itemize}
\item A bug was fixed in the {\idas} linear solver interface where an incorrect
Jacobian-vector product increment was used with iterative solvers other than
{\sunlinsolspgmr} and {\sunlinsolspfgmr}.
%
\item Fixed a bug where the \id{IDASolveF} function would not return a root in
\id{IDA\_NORMAL\_STEP} mode if the root occurred after the desired output time.
%
\item Fixed a bug where the \id{IDASolveF} function would return the wrong flag
under certrain cirumstances.
%
\item Fixed a bug in \id{IDAQuadReInitB} where an incorrect memory structure was
passed to \id{IDAQuadReInit}.
%
\item Removed extraneous calls to \id{N\_VMin} for simulations where the scalar
valued absolute tolerance, or all entries of the vector-valued absolute
tolerance array, are strictly positive. In this scenario, {\idas} will remove at
least one global reduction per time step.
%
\item The IDALS interface has been updated to only zero the Jacobian matrix
before calling a user-supplied Jacobian evaluation function when the attached
linear solver has type \newline\noindent
\id{SUNLINEARSOLVER\_DIRECT}.
%
\item Added the new functions,
\id{IDAGetCurentCj},
\id{IDAGetCurrentY},
\id{IDAGetCurrentYp}, \newline\noindent
\id{IDAComputeCurrentY},
\id{IDAComputeCurrentYp},
\id{IDAGetCurrentYSens},
\id{IDAGetCurrentYpSens},
\id{IDAComputeCurrentYSens},
and \id{IDAComputeCurrentYpSens},
which may be useful to users who choose to provide their own nonlinear solver
implementations.
%
\item Added a Fortran 2003 interface to {\idas}. See Chapter~\ref{s:idasfort} for
more details.
\end{itemize}



\subsection*{Changes in v3.1.0}

An additional {\nvector} implementation was added for the
{\tpetra} vector from the {\trilinos} library to facilitate interoperability
between {\sundials} and {\trilinos}. This implementation is accompanied by
additions to user documentation and {\sundials} examples.

A bug was fixed where a nonlinear solver object could be freed twice in some use
cases.

The \id{EXAMPLES\_ENABLE\_RAJA} CMake option has been removed. The option \id{EXAMPLES\_ENABLE\_CUDA}
enables all examples that use CUDA including the RAJA examples with a CUDA back end (if the RAJA
{\nvector} is enabled).

The implementation header file \id{idas\_impl.h} is no longer installed. This means users
who are directly manipulating the \id{IDAMem} structure will need to update their code
to use {\idas}'s public API.

Python is no longer required to run \id{make test} and \id{make test\_install}.

\subsection*{Changes in v3.0.2}

Added information on how to contribute to {\sundials} and a contributing agreement.

Moved definitions of DLS and SPILS backwards compatibility functions to a source file.
The symbols are now included in the {\idas} library, \id{libsundials\_idas}.

\subsection*{Changes in v3.0.1}

No changes were made in this release.

\subsection*{Changes in v3.0.0}

{\idas}' previous direct and iterative linear solver interfaces,
{\idadls} and {\idaspils}, have been merged into a single unified linear
solver interface, {\idals}, to support any valid {\sunlinsol} module.
This includes the ``DIRECT'' and ``ITERATIVE'' types as well as the new
``MATRIX\_ITERATIVE'' type. Details regarding how {\idals} utilizes linear
solvers of each type as well as discussion regarding intended use cases for
user-supplied {\sunlinsol} implementations are included in
Chapter~\ref{s:sunlinsol}. All {\idas} example programs and the standalone
linear solver examples have been updated to use the unified linear solver
interface.

The unified interface for the new {\idals} module is very similar to the
previous {\idadls} and {\idaspils} interfaces. To minimize challenges in user
migration to the new names, the previous {\CC} routine names may still
be used; these will be deprecated in future releases, so we recommend that users
migrate to the new names soon.

The names of all constructor routines for {\sundials}-provided
{\sunlinsol} implementations have been updated to follow the naming convention
\id{SUNLinSol\_*} where \id{*} is the name of the linear solver. The new names
are
\id{SUNLinSol\_Band},
\id{SUNLinSol\_Dense},
\id{SUNLinSol\_KLU},
\id{SUNLinSol\_LapackBand},\newline
\id{SUNLinSol\_LapackDense},
\id{SUNLinSol\_PCG},
\id{SUNLinSol\_SPBCGS},
\id{SUNLinSol\_SPFGMR},
\id{SUNLinSol\_SPGMR},
\id{SUNLinSol\_SPTFQMR}, and
\id{SUNLinSol\_SuperLUMT}.  Solver-specific ``set'' routine names have
been similarly standardized.  To minimize challenges in user migration
to the new names, the previous routine names may still be used; these
will be deprecated in future releases, so we recommend that users
migrate to the new names soon. All {\idas} example programs and the standalone
linear solver examples have been updated to use the new naming convention.

The \id{SUNBandMatrix} constructor has been simplified to remove the
storage upper bandwidth argument.

{\sundials} integrators have been updated to utilize generic nonlinear solver
modules defined through the {\sunnonlinsol} API. This API will ease the addition
of new nonlinear solver options and allow for external or user-supplied
nonlinear solvers. The {\sunnonlinsol} API and {\sundials} provided modules are
described in Chapter~\ref{c:sunnonlinsol} and follow the same object oriented
design and implementation used by the {\nvector}, {\sunmatrix}, and {\sunlinsol}
modules. Currently two {\sunnonlinsol} implementations are provided,
{\sunnonlinsolnewton} and {\sunnonlinsolfixedpoint}. These replicate the
previous integrator specific implementations of a Newton iteration and a
fixed-point iteration (previously referred to as a functional iteration),
respectively. Note the {\sunnonlinsolfixedpoint} module can optionally utilize
Anderson's method to accelerate convergence. Example programs using each of
these nonlinear solver modules in a standalone manner have been added and all
{\idas} example programs have been updated to use generic {\sunnonlinsol}
modules.

% IDAS specific changes due to the NLS API
By default {\idas} uses the {\sunnonlinsolnewton} module. Since {\idas} previously
only used an internal implementation of a Newton iteration no changes are
required to user programs and functions for setting the nonlinear solver options
(e.g., \id{IDASetMaxNonlinIters}) or getting nonlinear solver statistics (e.g.,
\id{IDAGetNumNonlinSolvIters}) remain unchanged and internally call generic
{\sunnonlinsol} functions as needed. While {\sundials} includes a fixed-point
nonlinear solver module, it is not currently supported in {\idas}. For details on
attaching a user-supplied nonlinear solver to {\idas} see
Chapter~\ref{s:simulation}, \ref{s:forward}, and \ref{s:adjoint}.

Three fused vector operations and seven vector array operations have been added
to the {\nvector} API. These \textit{optional} operations are disabled by
default and may be activated by calling vector specific routines after creating
an {\nvector} (see Chapter \ref{s:nvector} for more details). The new operations
are intended to increase data reuse in vector operations, reduce parallel
communication on distributed memory systems, and lower the number of kernel
launches on systems with accelerators. The fused operations are
\id{N\_VLinearCombination},
\id{N\_VScaleAddMulti}, and
\id{N\_VDotProdMulti}
and the vector array operations are
\id{N\_VLinearCombinationVectorArray},
\id{N\_VScaleVectorArray},
\id{N\_VConstVectorArray},
\id{N\_VWrmsNormVectorArray},
\id{N\_VWrmsNormMaskVectorArray},
\id{N\_VScaleAddMultiVectorArray}, and\\
\id{N\_VLinearCombinationVectorArray}.
If an {\nvector} implementation defines any of these operations as \id{NULL},
then standard {\nvector} operations will automatically be called as necessary to
complete the computation.
\\
\\
\noindent Multiple updates to {\nveccuda} were made:
\begin{itemize}
  \item Changed \id{N\_VGetLength\_Cuda} to return the global vector length
        instead of the local vector length.
  \item Added \id{N\_VGetLocalLength\_Cuda} to return the local vector length.
  \item Added \id{N\_VGetMPIComm\_Cuda} to return the MPI communicator used.
  \item Removed the accessor functions in the namespace suncudavec.
  \item Changed the \id{N\_VMake\_Cuda} function to take a host data pointer and a device
        data pointer instead of an \id{N\_VectorContent\_Cuda} object.
  \item Added the ability to set the \id{cudaStream\_t} used for execution of the
        {\nveccuda} kernels. See the function \id{N\_VSetCudaStreams\_Cuda}.
  \item Added \id{N\_VNewManaged\_Cuda}, \id{N\_VMakeManaged\_Cuda}, and
        \id{N\_VIsManagedMemory\_Cuda} functions to accommodate using managed
        memory with the {\nveccuda}.
\end{itemize}
%
Multiple changes to {\nvecraja} were made:
\begin{itemize}
  \item Changed \id{N\_VGetLength\_Raja} to return the global vector length
        instead of the local vector length.
  \item Added \id{N\_VGetLocalLength\_Raja} to return the local vector length.
  \item Added \id{N\_VGetMPIComm\_Raja} to return the MPI communicator used.
  \item Removed the accessor functions in the namespace suncudavec.
\end{itemize}
A new {\nvector} implementation for leveraging OpenMP 4.5+ device offloading has
been added, {\nvecopenmpdev}. See \S\ref{ss:nvec_openmpdev} for more details.


\subsection*{Changes in v2.2.1}

The changes in this minor release include the following:
\begin{itemize}
\item Fixed a bug in the {\cuda} {\nvector} where the \id{N\_VInvTest} operation
  could write beyond the allocated vector data.
\item Fixed library installation path for multiarch systems. This fix changes the default
  library installation path to \id{CMAKE\_INSTALL\_PREFIX/CMAKE\_INSTALL\_LIBDIR}
  from \id{CMAKE\_INSTALL\_PREFIX/lib}.\\
  \id{CMAKE\_INSTALL\_LIBDIR} is automatically
  set, but is available as a CMake option that can modified.
\end{itemize}

\subsection*{Changes in v2.2.0}

Fixed a bug in {\idas} where the saved residual value used in the nonlinear
solve for consistent initial conditions was passed as temporary workspace and
could be overwritten.
\\
\\
\noindent Fixed a thread-safety issue when using ajdoint sensitivity analysis.
\\
\\
\noindent Fixed a problem with setting \id{sunindextype} which would occur
with some compilers (e.g. armclang) that did not define \id{\_\_STDC\_VERSION\_\_}.
\\
\\
\noindent Added hybrid MPI/CUDA and MPI/RAJA vectors to allow use of more
than one MPI rank when using a GPU system.  The vectors assume one GPU
device per MPI rank.
\\
\\
\noindent Changed the name of the {\raja} {\nvector} library to
\id{libsundials\_nveccudaraja.lib} from \newline
\id{libsundials\_nvecraja.lib} to better reflect that we only support {\cuda}
as a backend for {\raja} currently.
\\
\\
\noindent Several changes were made to the build system:
\begin{itemize}
\item CMake 3.1.3 is now the minimum required CMake version.
\item Deprecate the behavior of the \id{SUNDIALS\_INDEX\_TYPE} CMake option and
  added the \newline
  \id{SUNDIALS\_INDEX\_SIZE} CMake option to select the \id{sunindextype}
  integer size.
\item The native CMake FindMPI module is now used to locate an MPI installation.
\item If MPI is enabled and MPI compiler wrappers are not set, the build system
  will check if \id{CMAKE\_<language>\_COMPILER} can compile MPI programs before
  trying to locate and use an MPI installation.
\item The previous options for setting MPI compiler wrappers and the executable
  for running MPI programs have been have been depreated. The new options that
  align with those used in native CMake FindMPI module are
  \id{MPI\_C\_COMPILER}, \id{MPI\_CXX\_COMPILER}, \id{MPI\_Fortran\_COMPILER},
  and \id{MPIEXEC\_EXECUTABLE}.
\item When a Fortran name-mangling scheme is needed (e.g., \id{ENABLE\_LAPACK}
  is \id{ON}) the build system will infer the scheme from the Fortran
  compiler. If a Fortran compiler is not available or the inferred or default
  scheme needs to be overridden, the advanced options
  \id{SUNDIALS\_F77\_FUNC\_CASE} and \id{SUNDIALS\_F77\_FUNC\_UNDERSCORES} can
  be used to manually set the name-mangling scheme and bypass trying to infer
  the scheme.
\item Parts of the main CMakeLists.txt file were moved to new files in the
  \id{src} and \id{example} directories to make the CMake configuration file
  structure more modular.
\end{itemize}

\subsection*{Changes in v2.1.2}

The changes in this minor release include the following:
\begin{itemize}
\item Updated the minimum required version of CMake to 2.8.12 and enabled
  using rpath by default to locate shared libraries on OSX.
\item Fixed Windows specific problem where \id{sunindextype} was not correctly
  defined when using 64-bit integers for the {\sundials} index type. On Windows
  \id{sunindextype} is now defined as the MSVC basic type \id{\_\_int64}.
\item Added sparse SUNMatrix ``Reallocate'' routine to allow specification of
  the nonzero storage.
\item Updated the KLU {\sunlinsol} module to set constants for the two
  reinitialization types, and fixed a bug in the full reinitialization
  approach where the sparse SUNMatrix pointer would go out of scope on
  some architectures.
\item Updated the ``ScaleAdd'' and ``ScaleAddI'' implementations in the
  sparse SUNMatrix module to more optimally handle the case where the
  target matrix contained sufficient storage for the sum, but had the
  wrong sparsity pattern.  The sum now occurs in-place, by performing
  the sum backwards in the existing storage.  However, it is still more
  efficient if the user-supplied Jacobian routine allocates storage for
  the sum $I+\gamma J$ manually (with zero entries if needed).
\item Changed the LICENSE install path to \id{instdir/include/sundials}.
\end{itemize}

\subsection*{Changes in v2.1.1}

The changes in this minor release include the following:
\begin{itemize}
\item Fixed a potential memory leak in the {\spgmr} and {\spfgmr} linear
  solvers: if ``Initialize'' was called multiple times then the solver
  memory was reallocated (without being freed).

\item Updated KLU SUNLinearSolver module to use a \id{typedef} for the
  precision-specific solve function to be used (to avoid compiler
  warnings).

\item Added missing typecasts for some \id{(void*)} pointers (again, to
  avoid compiler warnings).

\item Bugfix in \id{sunmatrix\_sparse.c} where we had used \id{int}
  instead of \id{sunindextype} in one location.

\item Added missing \id{\#include <stdio.h>} in {\nvector} and {\sunmatrix}
  header files.

\item Added missing prototype for \id{IDASpilsGetNumJTSetupEvals}.

\item Fixed an indexing bug in the {\cuda} {\nvector} implementation of
  \id{N\_VWrmsNormMask} and revised the {\raja} {\nvector} implementation of
  \id{N\_VWrmsNormMask} to work with mask arrays using values other than zero or
  one. Replaced \id{double} with \id{realtype} in the {\raja} vector test functions.
\end{itemize}
In addition to the changes above, minor corrections were also made to the
example programs, build system, and user documentation.

\subsection*{Changes in v2.1.0}

Added {\nvector} print functions that write vector data to a specified
file (e.g., \id{N\_VPrintFile\_Serial}).

Added \id{make test} and \id{make test\_install} options to the build
system for testing {\sundials} after building with \id{make} and
installing with \id{make install} respectively.

\subsection*{Changes in v2.0.0}

All interfaces to matrix structures and linear solvers
have been reworked, and all example programs have been updated.
The goal of the redesign of these interfaces was to provide more encapsulation
and to ease interfacing of custom linear solvers and interoperability
with linear solver libraries.
Specific changes include:
\begin{itemize}
\item Added generic {\sunmatrix} module with three provided implementations:
        dense, banded and sparse.  These replicate previous SUNDIALS Dls and
        Sls matrix structures in a single object-oriented API.
\item Added example problems demonstrating use of generic {\sunmatrix} modules.
\item Added generic \id{SUNLinearSolver} module with eleven provided
        implementations: {\sundials} native dense, {\sundials} native banded,
        LAPACK dense, LAPACK band, KLU,
        SuperLU\_MT, SPGMR, SPBCGS, SPTFQMR, SPFGMR, and PCG.  These replicate
        previous SUNDIALS generic linear solvers in a single object-oriented
        API.
\item Added example problems demonstrating use of generic \id{SUNLinearSolver}
        modules.
\item Expanded package-provided direct linear solver (Dls) interfaces and
        scaled, preconditioned, iterative linear solver (Spils) interfaces
        to utilize generic {\sunmatrix} and \id{SUNLinearSolver} objects.
\item Removed package-specific, linear solver-specific, solver modules
        (e.g. \id{CVDENSE}, \id{KINBAND}, \id{IDAKLU}, \id{ARKSPGMR}) since their functionality
        is entirely replicated by the generic Dls/Spils interfaces and
        \id{SUNLinearSolver/SUNMATRIX} modules.  The exception is \id{CVDIAG}, a
        diagonal approximate Jacobian solver available to {\cvode} and {\cvodes}.
\item Converted all {\sundials} example problems and files to utilize the new generic
        {\sunmatrix} and \id{SUNLinearSolver} objects, along with updated Dls and
        Spils linear solver interfaces.
\item Added Spils interface routines to {\arkode}, {\cvode}, {\cvodes}, {\ida},
        and {\idas} to allow specification of a user-provided "JTSetup" routine.
        This change supports users who wish to set up data structures for
        the user-provided Jacobian-times-vector ("JTimes") routine, and
        where the cost of one JTSetup setup per Newton iteration can be
        amortized between multiple JTimes calls.
\end{itemize}

Two additional {\nvector} implementations were added -- one for
{\cuda} and one for {\raja} vectors.
These vectors are supplied to provide very basic support for running
on GPU architectures.  Users are advised that these vectors both move all data
to the GPU device upon construction, and speedup will only be realized if the
user also conducts the right-hand-side function evaluation on the device.
In addition, these vectors assume the problem fits on one GPU.
Further information about {\raja}, users are referred to the web site,
https://software.llnl.gov/RAJA/.
These additions are accompanied by additions to various interface functions
and to user documentation.

All indices for data structures were updated to a new \id{sunindextype} that
can be configured to be a 32- or 64-bit integer data index type.
\id{sunindextype} is defined to be \id{int32\_t} or \id{int64\_t} when portable types are
supported, otherwise it is defined as \id{int} or \id{long int}.
The Fortran interfaces continue to use \id{long int} for indices, except for
their sparse matrix interface that now uses the new \id{sunindextype}.
This new flexible capability for index types includes interfaces to
PETSc, hypre, SuperLU\_MT, and KLU with
either 32-bit or 64-bit capabilities depending how the user configures
{\sundials}.

To avoid potential namespace conflicts, the macros defining \id{booleantype}
values \id{TRUE} and \id{FALSE} have been changed to \id{SUNTRUE} and
\id{SUNFALSE} respectively.

Temporary vectors were removed from preconditioner setup and solve
routines for all packages.  It is assumed that all necessary data
for user-provided preconditioner operations will be allocated and
stored in user-provided data structures.

The file \id{include/sundials\_fconfig.h} was added.  This file contains
{\sundials} type information for use in Fortran programs.

The build system was expanded to support many of the xSDK-compliant keys.
The xSDK is a movement in scientific software to provide a foundation for the
rapid and efficient production of high-quality,
sustainable extreme-scale scientific applications.  More information can
be found at, https://xsdk.info.

Added functions \id{SUNDIALSGetVersion} and \id{SUNDIALSGetVersionNumber} to
get {\sundials} release version information at runtime.

In addition, numerous changes were made to the build system.
These include the addition of separate \id{BLAS\_ENABLE} and \id{BLAS\_LIBRARIES}
CMake variables, additional error checking during CMake configuration,
minor bug fixes, and renaming CMake options to enable/disable examples
for greater clarity and an added option to enable/disable Fortran 77 examples.
These changes included changing \id{EXAMPLES\_ENABLE} to \id{EXAMPLES\_ENABLE\_C},
changing \id{CXX\_ENABLE} to \id{EXAMPLES\_ENABLE\_CXX}, changing \id{F90\_ENABLE} to
\id{EXAMPLES\_ENABLE\_F90}, and adding an \id{EXAMPLES\_ENABLE\_F77} option.

A bug fix was done to add a missing prototype for \id{IDASetMaxBacksIC}
in \id{ida.h}.

Corrections and additions were made to the examples,
to installation-related files,
and to the user documentation.


\subsection*{Changes in v1.3.0}

Two additional {\nvector} implementations were added -- one for
Hypre (parallel) ParVector vectors, and one for PETSc vectors.  These
additions are accompanied by additions to various interface functions
and to user documentation.

Each {\nvector} module now includes a function, \id{N\_VGetVectorID},
that returns the {\nvector} module name.

An optional input function was added to set a maximum number
of linesearch backtracks in the initial condition calculation, and
four user-callable functions were added to support the use of LAPACK
linear solvers in solving backward problems for adjoint sensitivity
analysis.

For each linear solver, the various solver performance counters are
now initialized to 0 in both the solver specification function and in
solver \id{linit} function.  This ensures that these solver counters
are initialized upon linear solver instantiation as well as at the
beginning of the problem solution.

A bug in for-loop indices was fixed in \id{IDAAckpntAllocVectors}. A bug was
fixed in the interpolation functions used in solving backward problems.

A memory leak was fixed in the banded preconditioner interface.
In addition, updates were done to return integers from linear solver
and preconditioner 'free' functions.

In interpolation routines for backward problems, added logic to bypass
sensitivity interpolation if input sensitivity argument is NULL.

The Krylov linear solver Bi-CGstab was enhanced by removing a redundant
dot product.  Various additions and corrections were made to the
interfaces to the sparse solvers KLU and SuperLU\_MT, including support
for CSR format when using KLU.

New examples were added for use of the OpenMP vector and for use of
sparse direct solvers within sensitivity integrations.

Minor corrections and additions were made to the {\idas} solver, to the
examples, to installation-related files, and to the user documentation.


\subsection*{Changes in v1.2.0}

Two major additions were made to the linear system solvers that are
available for use with the {\idas} solver.  First, in the serial case,
an interface to the sparse direct solver KLU was added.
Second, an interface to SuperLU\_MT, the multi-threaded version of
SuperLU, was added as a thread-parallel sparse direct solver option,
to be used with the serial version of the {\nvector} module.
As part of these additions, a sparse matrix (CSC format) structure
was added to {\idas}.

Otherwise, only relatively minor modifications were made to {\idas}:

In \id{IDARootfind}, a minor bug was corrected, where the input
array \id{rootdir} was ignored, and a line was added to break out of
root-search loop if the initial interval size is below the tolerance
\id{ttol}.

In \id{IDALapackBand}, the line \id{smu = MIN(N-1,mu+ml)} was changed to
\id{smu = mu + ml} to correct an illegal input error for \id{DGBTRF/DGBTRS}.

An option was added in the case of Adjoint Sensitivity Analysis with
dense or banded Jacobian:  With a call to \id{IDADlsSetDenseJacFnBS} or
\id{IDADlsSetBandJacFnBS}, the user can specify a user-supplied Jacobian
function of type \id{IDADls***JacFnBS}, for the case where the backward
problem depends on the forward sensitivities.

A minor bug was fixed regarding the testing of the input \id{tstop} on
the first call to \id{IDASolve}.

For the Adjoint Sensitivity Analysis case in which the backward problem
depends on the forward sensitivities, options have been added to allow
for user-supplied \id{pset}, \id{psolve}, and \id{jtimes} functions.

In order to avoid possible name conflicts, the mathematical macro
and function names \id{MIN}, \id{MAX}, \id{SQR}, \id{RAbs}, \id{RSqrt},
\id{RExp}, \id{RPowerI}, and \id{RPowerR} were changed to
\id{SUNMIN}, \id{SUNMAX}, \id{SUNSQR}, \id{SUNRabs}, \id{SUNRsqrt},
\id{SUNRexp}, \id{SRpowerI}, and \id{SUNRpowerR}, respectively.
These names occur in both the solver and in various example programs.

In the User Guide, a paragraph was added in Section 6.2.1 on
\id{IDAAdjReInit}, and a paragraph was added in Section 6.2.9 on
\id{IDAGetAdjY}.

Two new {\nvector} modules have been added for thread-parallel computing
environments --- one for OpenMP, denoted \id{NVECTOR\_OPENMP},
and one for Pthreads, denoted \id{NVECTOR\_PTHREADS}.

With this version of {\sundials}, support and documentation of the
Autotools mode of installation is being dropped, in favor of the
CMake mode, which is considered more widely portable.

\subsection*{Changes in v1.1.0}

One significant design change was made with this release: The problem
size and its relatives, bandwidth parameters, related internal indices,
pivot arrays, and the optional output \id{lsflag} have all been
changed from type \id{int} to type \id{long int}, except for the
problem size and bandwidths in user calls to routines specifying
BLAS/LAPACK routines for the dense/band linear solvers.  The function
\id{NewIntArray} is replaced by a pair \id{NewIntArray}/\id{NewLintArray},
for \id{int} and \id{long int} arrays, respectively.  In a minor
change to the user interface, the type of the index \id{which} in
IDAS was changed from \id{long int} to \id{int}.

Errors in the logic for the integration of backward problems were
identified and fixed.

A large number of minor errors have been fixed.  Among these are the following:
A missing vector pointer setting was added in \id{IDASensLineSrch}.
In \id{IDACompleteStep}, conditionals around lines loading a new column of three
auxiliary divided difference arrays, for a possible order increase, were fixed.
After the solver memory is created, it is set to zero before being filled.
In each linear solver interface function, the linear solver memory is
freed on an error return, and the \id{**Free} function now includes a
line setting to NULL the main memory pointer to the linear solver memory.
A memory leak was fixed in two of the \id{IDASp***Free} functions.
In the rootfinding functions \id{IDARcheck1}/\id{IDARcheck2}, when an exact
zero is found, the array \id{glo} of $g$ values at the left endpoint is
adjusted, instead of shifting the $t$ location \id{tlo} slightly.
In the installation files, we modified the treatment of the macro
SUNDIALS\_USE\_GENERIC\_MATH, so that the parameter GENERIC\_MATH\_LIB is
either defined (with no value) or not defined.


\section{Reading this User Guide}\label{ss:reading}

The structure of this document is as follows:
\begin{itemize}
\item
  In Chapter \ref{s:math}, we give short descriptions of the numerical
  methods implemented by {\idas} for the solution of initial value problems
  for systems of DAEs, continue with short descriptions of preconditioning
  (\S\ref{s:preconditioning}) and rootfinding (\S\ref{s:rootfinding}), and
  then give an overview of the mathematical aspects of sensitivity analysis,
  both forward (\S\ref{ss:fwd_sensi}) and adjoint (\S\ref{ss:adj_sensi}).
\item
  The following chapter describes the structure of the {\sundials} suite
  of solvers (\S\ref{ss:sun_org}) and the software organization of the {\idas}
  solver (\S\ref{ss:idas_org}).
\item
  Chapter \ref{s:simulation} is the main usage document for {\idas}
  for simulation applications.  It includes a complete description of
  the user interface for the integration of DAE initial value problems.
  Readers that are not interested in using {\idas} for sensitivity
  analysis can then skip the next two chapters.
\item
  Chapter \ref{s:forward} describes the usage of {\idas} for forward
  sensitivity analysis as an extension of its IVP integration
  capabilities.  We begin with a skeleton of the user main program,
  with emphasis on the steps that are required in addition to those
  already described in Chapter \ref{s:simulation}.  Following that we
  provide detailed descriptions of the user-callable interface
  routines specific to forward sensitivity analysis and of the
  additonal optional user-defined routines.
\item
  Chapter \ref{s:adjoint} describes the usage of {\idas} for adjoint
  sensitivity analysis. We begin by describing the {\idas} checkpointing
  implementation for interpolation of the original IVP solution during
  integration of the adjoint system backward in time, and with
  an overview of a user's main program. Following that we provide complete
  descriptions of the user-callable interface routines for adjoint sensitivity
  analysis as well as descriptions of the required additional user-defined routines.
\item
  Chapter \ref{s:nvector} gives a brief overview of the generic
  {\nvector} module shared amongst the various components of
  {\sundials}, as well as details on the {\nvector}
  implementations provided with {\sundials}.
\item
Chapter \ref{s:sunmatrix} gives a brief overview of the generic
  {\sunmatrix} module shared among the various components of
  {\sundials}, and details on the {\sunmatrix} implementations
  provided with {\sundials}:
  a dense implementation (\S\ref{ss:sunmat_dense}),
  a banded implementation (\S\ref{ss:sunmat_band}) and
  a sparse implementation (\S\ref{ss:sunmat_sparse}).
\item
  Chapter \ref{s:sunlinsol} gives a brief overview of the generic
  {\sunlinsol} module shared among the various components of
  {\sundials}.  This chapter contains details on the {\sunlinsol}
  implementations provided with {\sundials}.
  The chapter
  also contains details on the {\sunlinsol} implementations provided
  with {\sundials} that interface with external linear solver
  libraries.
%% \item
%%   Chapter \ref{s:new_linsolv} describes the interfaces to the linear
%%   solver modules, so that a user can provide his/her own such module.
%% \item
%%   Chapter \ref{s:gen_linsolv} describes in detail the generic linear
%%   solvers shared by all {\sundials} solvers.
\item
  Chapter \ref{c:sunnonlinsol} describes the {\sunnonlinsol} API and nonlinear
  solver implementations shared among the various components of {\sundials}.
\item
  Finally, in the appendices, we provide detailed instructions for the installation
  of {\idas}, within the structure of {\sundials} (Appendix \ref{c:install}), as
  well as a list of all the constants used for input to and output from {\idas}
  functions
  (Appendix \ref{c:constants}).
\end{itemize}

Finally, the reader should be aware of the following notational conventions
in this user guide:  program listings and identifiers (such as \id{IDAInit})
within textual explanations appear in typewriter type style;
fields in {\CC} structures (such as {\em content}) appear in italics;
and packages or modules, such as {\idals}, are written in all capitals.
Usage and installation instructions that constitute important warnings
are marked with a triangular symbol {\warn} in the margin.

%---------------------------------
% License information section
\section{SUNDIALS Release License}

The SUNDIALS packages are released open source, under a BSD license.
The only requirements of the BSD license are preservation of copyright and a standard disclaimer of liability.
Our Copyright notice is below along with the license.

**PLEASE NOTE**  If you are using SUNDIALS with any third 
party libraries linked in (e.g., LaPACK, KLU, SuperLU\_MT, PETSc, 
or hypre), be sure to review the respective license of the package as 
that license may have more restrictive terms than the SUNDIALS license.  
For example, if someone builds SUNDIALS with a statically linked KLU, 
the build is subject to terms of the LGPL license (which is what 
KLU is released with) and *not* the SUNDIALS BSD license anymore.

\subsection{Copyright Notices}
All SUNDIALS packages except ARKode are subject to the following Copyright
notice.

\subsubsection{SUNDIALS Copyright}
Copyright (c) 2002-2016, Lawrence Livermore National Security. 
Produced at the Lawrence Livermore National Laboratory.
Written by A.C. Hindmarsh, D.R. Reynolds, R. Serban, C.S. Woodward,
S.D. Cohen, A.G. Taylor, S. Peles, L.E. Banks, and D. Shumaker. \\
UCRL-CODE-155951    (CVODE)\\
UCRL-CODE-155950    (CVODES)\\
UCRL-CODE-155952    (IDA)\\
UCRL-CODE-237203    (IDAS)\\
LLNL-CODE-665877    (KINSOL)\\
All rights reserved. 
 
\subsubsection{ARKode Copyright}
ARKode is subject to the following joint Copyright notice.
Copyright (c) 2015-2016, Southern Methodist University and 
Lawrence Livermore National Security
Written by D.R. Reynolds, D.J. Gardner, A.C. Hindmarsh, C.S. Woodward,
and J.M. Sexton.\\
LLNL-CODE-667205    (ARKODE) \\
All rights reserved.

\subsection{BSD License}
Redistribution and use in source and binary forms, with or without
modification, are permitted provided that the following conditions
are met:
 
1. Redistributions of source code must retain the above copyright
notice, this list of conditions and the disclaimer below.
 
2. Redistributions in binary form must reproduce the above copyright
notice, this list of conditions and the disclaimer (as noted below)
in the documentation and/or other materials provided with the
distribution.
 
3. Neither the name of the LLNS/LLNL nor the names of its contributors
may be used to endorse or promote products derived from this software
without specific prior written permission.
 
THIS SOFTWARE IS PROVIDED BY THE COPYRIGHT HOLDERS AND CONTRIBUTORS 
"AS IS" AND ANY EXPRESS OR IMPLIED WARRANTIES, INCLUDING, BUT NOT 
LIMITED TO, THE IMPLIED WARRANTIES OF MERCHANTABILITY AND FITNESS 
FOR A PARTICULAR PURPOSE ARE DISCLAIMED. IN NO EVENT SHALL 
LAWRENCE LIVERMORE NATIONAL SECURITY, LLC, THE U.S. DEPARTMENT OF 
ENERGY OR CONTRIBUTORS BE LIABLE FOR ANY DIRECT, INDIRECT, INCIDENTAL, 
SPECIAL, EXEMPLARY, OR CONSEQUENTIAL DAMAGES (INCLUDING, BUT NOT LIMITED 
TO, PROCUREMENT OF SUBSTITUTE GOODS OR SERVICES; LOSS OF USE, 
DATA, OR PROFITS; OR BUSINESS INTERRUPTION) HOWEVER CAUSED AND ON ANY 
THEORY OF LIABILITY, WHETHER IN CONTRACT, STRICT LIABILITY, OR TORT 
(INCLUDING NEGLIGENCE OR OTHERWISE) ARISING IN ANY WAY OUT OF THE USE 
OF THIS SOFTWARE, EVEN IF ADVISED OF THE POSSIBILITY OF SUCH DAMAGE.
\\
\\
Additional BSD Notice
\begin{enumerate}
\item{} 
This notice is required to be provided under our contract with
the U.S. Department of Energy (DOE). This work was produced at
Lawrence Livermore National Laboratory under Contract 
No. DE-AC52-07NA27344 with the DOE.

\item{} 
Neither the United States Government nor Lawrence Livermore 
National Security, LLC nor any of their employees, makes any warranty, 
express or implied, or assumes any liability or responsibility for the
accuracy, completeness, or usefulness of any information, apparatus,
product, or process disclosed, or represents that its use would not
infringe privately-owned rights.

\item{} 
Also, reference herein to any specific commercial products, process, 
or services by trade name, trademark, manufacturer or otherwise does 
not necessarily constitute or imply its endorsement, recommendation, 
or favoring by the United States Government or Lawrence Livermore 
National Security, LLC. The views and opinions of authors expressed 
herein do not necessarily state or reflect those of the United States 
Government or Lawrence Livermore National Security, LLC, and shall 
not be used for advertising or product endorsement purposes.

\end{enumerate}

%---------------------------------
