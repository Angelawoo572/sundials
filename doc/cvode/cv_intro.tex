%===================================================================================
\chapter{Introduction}\label{s:intro}
%===================================================================================

{\cvode} is part of a software family called {\sundials}: 
SUite of Nonlinear and DIfferential/ALgebraic equation Solvers~\cite{HBGLSSW:05}.  
This suite consists of {\cvode}, {\arkode}, {\kinsol}, and {\ida}, and variants
of these with sensitivity analysis capabilities.
%
%---------------------------------
\section{Historical Background}\label{ss:history}
%---------------------------------

\index{CVODE@{\cvode}!relationship to {\vode}, {\vodpk}|(}
{\F} solvers for ODE initial value problems are widespread and heavily used. 
Two solvers that have been written at LLNL in the past are {\vode}~\cite{BBH:89} 
and {\vodpk}~\cite{Byr:92}.
{\vode}\index{VODE@{\vode}} is a general purpose solver that includes methods for
both stiff and nonstiff systems, and in the stiff case uses direct methods (full or
banded) for the solution of the linear systems that arise at each implicit
step. Externally, {\vode} is very similar to the well known solver
{\lsode}\index{LSODE@{\lsode}}~\cite{RaHi:94}. {\vodpk}\index{VODPK@{\vodpk}}
is a variant of {\vode} that uses a preconditioned Krylov (iterative)
method, namely GMRES, for the solution of the linear systems. {\vodpk}
is a powerful tool for large stiff systems because it combines
established methods for stiff integration, nonlinear iteration, and
Krylov (linear) iteration with a problem-specific treatment of the
dominant source of stiffness, in the form of the user-supplied
preconditioner matrix~\cite{BrHi:89}.  The capabilities of both
{\vode} and {\vodpk} have been combined in the {\CC}-language package
{\cvode}\index{CVODE@{\cvode}}~\cite{CoHi:96}.

At present, {\cvode} may utilize a variety of Krylov methods provided
in {\sundials} that can be used in conjuction with Newton iteration:
these include the GMRES (Generalized Minimal RESidual)~\cite{SaSc:86},
FGMRES (Flexible Generalized Minimum RESidual)~\cite{Saa:93},
Bi-CGStab (Bi-Conjugate Gradient Stabilized)~\cite{Van:92}, TFQMR
(Transpose-Free Quasi-Minimal Residual)~\cite{Fre:93}, and PCG
(Preconditioned Conjugate Gradient)~\cite{HeSt:52} linear iterative
methods.  As Krylov methods, these require almost no  
matrix storage for solving the Newton equations as compared to direct 
methods. However, the algorithms allow for a user-supplied preconditioner
matrix, and for most problems preconditioning is essential for an
efficient solution.
For very large stiff ODE systems, the Krylov methods are preferable over
direct linear solver methods, and are often the only feasible choice.
Among the Krylov methods in {\sundials}, we recommend GMRES as the
best overall choice.  However, users are encouraged to compare all
options, especially if encountering convergence failures with GMRES.
Bi-CGStab and TFQMR have an advantage in storage requirements, in
that the number of workspace vectors they require is fixed, while that
number for GMRES depends on the desired Krylov subspace size.  FGMRES
has an advantage in that it is designed to support preconditioners
that vary between iterations (e.g.~iterative methods).  PCG exhibits
rapid convergence and minimal workspace vectors, but only works for
symmetric linear systems.

In the process of translating the {\vode} and {\vodpk} algorithms into
{\CC}, the overall {\cvode} organization has been changed considerably.
One key feature of the {\cvode} organization is that the linear system
solvers comprise a layer of code modules that is separated from the
integration algorithm, allowing for easy modification and expansion of
the linear solver array.  A second key feature is a separate module
devoted to vector operations; this facilitated the extension to
multiprosessor environments with minimal impacts on the rest of the
solver, resulting in {\pvode}\index{PVODE@{\pvode}}~\cite{ByHi:99},
the parallel variant of {\cvode}.  \index{CVODE@{\cvode}!relationship
to {\vode}, {\vodpk}|)}

\index{CVODE@{\cvode}!relationship to {\cvode}, {\pvode}|(} Around 2002,
the functionality of {\cvode} and {\pvode} were combined into one
single code, simply called {\cvode}. Development of this version of
{\cvode} was concurrent with a redesign of the vector operations
module across the {\sundials} suite. The key feature of the
{\nvector} module is that it is written in terms of abstract vector
operations with the actual vector kernels attached by a particular
implementation (such as serial or parallel) of {\nvector}. This allows
writing the {\sundials} solvers in a manner independent of the actual
{\nvector} implementation (which can be user-supplied), as well as
allowing more than one {\nvector} module linked into an executable file.
{\sundials} (and thus {\cvode}) is supplied with six different {\nvector}
implementations:
serial, MPI-parallel, and both openMP and Pthreads thread-parallel
{\nvector} implementations, a Hypre parallel implementation,
and a PetSC implementation.
\index{CVODE@{\cvode}!relationship to {\cvode}, {\pvode}|)}

\index{CVODE@{\cvode}!motivation for writing in C|(}
There are several motivations for choosing the {\CC} language for {\cvode}.
First, a general movement away from {\F} and toward {\CC} in scientific
computing was apparent.  Second, the pointer, structure, and dynamic
memory allocation features in C are extremely useful in software of
this complexity, with the great variety of method options offered.
Finally, we prefer {\CC} over {\CPP} for {\cvode} because of the wider
availability of {\CC} compilers, the potentially greater efficiency of {\CC},
and the greater ease of interfacing the solver to applications written
in extended {\F}.
\index{CVODE@{\cvode}!motivation for writing in C|)}

\section{Changes from previous versions}

\subsection*{Changes in v3.1.1}

Fixed a potential memory leak in the {\spgmr} and {\spfgmr} linear
solvers: if "Initialize" was called multiple times then the solver
memory was reallocated (without being freed).

\subsection*{Changes in v3.1.0}

Added {\nvector} print functions that write vector data to a specified
file (e.g., \id{N\_VPrintFile\_Serial}).

Added \id{make test} and \id{make test\_install} options to the build
system for testing {\sundials} after building with \id{make} and
installing with \id{make install} respectively.

\subsection*{Changes in v3.0.0}

All interfaces to matrix structures and linear solvers 
have been reworked, and all example programs have been updated. 
The goal of the redesign of these interfaces was to provide more encapsulation
and ease in interfacing custom linear solvers and interoperability 
with linear solver libraries.
Specific changes include:
\begin{itemize}
\item Added generic SUNMATRIX module with three provided implementations:
        dense, banded and sparse.  These replicate previous SUNDIALS Dls and
        Sls matrix structures in a single object-oriented API.
\item Added example problems demonstrating use of generic SUNMATRIX modules.
\item Added generic SUNLINEARSOLVER module with eleven provided
        implementations: dense, banded, LAPACK dense, LAPACK band, KLU,
        SuperLU\_MT, SPGMR, SPBCGS, SPTFQMR, SPFGMR, PCG.  These replicate
        previous SUNDIALS generic linear solvers in a single object-oriented
        API.
\item Added example problems demonstrating use of generic SUNLINEARSOLVER
        modules.
\item Expanded package-provided direct linear solver (Dls) interfaces and
        scaled, preconditioned, iterative linear solver (Spils) interfaces
        to utilize generic SUNMATRIX and SUNLINEARSOLVER objects.
\item Removed package-specific, linear solver-specific, solver modules
        (e.g. CVDENSE, KINBAND, IDAKLU, ARKSPGMR) since their functionality
        is entirely replicated by the generic Dls/Spils interfaces and
        SUNLINEARSOLVER/SUNMATRIX modules.  The exception is CVDIAG, a
        diagonal approximate Jacobian solver available to CVODE and CVODES.
\item Converted all SUNDIALS example problems to utilize new generic
        SUNMATRIX and SUNLINEARSOLVER objects, along with updated Dls and
        Spils linear solver interfaces.
\item Added Spils interface routines to ARKode, CVODE, CVODES, IDA and
        IDAS to allow specification of a user-provided "JTSetup" routine.
        This change supports users who wish to set up data structures for
        the user-provided Jacobian-times-vector ("JTimes") routine, and
        where the cost of one JTSetup setup per Newton iteration can be
        amortized between multiple JTimes calls.
\end{itemize}

Two additional {\nvector} implementations were added -- one for
CUDA and one for RAJA vectors.  
These vectors are supplied to provide very basic support for running
on GPU architectures.  Users are advised that these vectors both move all data
to the GPU device upon construction, and speedup will only be realized if the
user also conducts the right-hand-side function evaluation on the device.
In addition, these vectors assume the problem fits on one GPU.
Further information about RAJA, users are referred to th web site, 
https://software.llnl.gov/RAJA/.
These additions are accompanied by additions to various interface functions
and to user documentation.

All indices for data structures were updated to a new \id{sunindextype} that
can be configured to be a 32- or 64-bit integer data index type. 
\id{sunindextype} is defined to be \id{int32\_t} or \id{int64\_t} when portable types are
supported, otherwise it is defined as \id{int} or \id{long int}.
The Fortran interfaces continue to use \id{long int} for indices, except for 
their sparse matrix interface that now uses the new \id{sunindextype}.
This new flexible capability for index types includes interfaces to 
PETSc, hypre, SuperLU\_MT, and KLU with 
either 32-bit or 64-bit capabilities depending how the user configures 
{\sundials}.

To avoid potential namespace conflicts, the macros defining \id{booleantype}
values \id{TRUE} and \id{FALSE} have been changed to \id{SUNTRUE} and
\id{SUNFALSE} respectively.

Temporary vectors were removed from preconditioner setup and solve
routines for all packages.  It is assumed that all necessary data
for user-provided preconditioner operations will be allocated and
stored in user-provided data structures.

The file \id{include/sundials\_fconfig.h} was added. This file contains 
{\sundials} type information for use in Fortran programs.

Added functions \id{SUNDIALSGetVersion} and \id{SUNDIALSGetVersionNumber} to
get {\sundials} release version information at runtime.

The build system was expanded to support many of the xSDK-compliant keys. 
The xSDK is a movement in scientific software to provide a foundation for the
rapid and efficient production of high-quality, 
sustainable extreme-scale scientific applications.  More information can
be found at, https://xsdk.info.

In addition, numerous changes were made to the build system.
These include the addition of separate \id{BLAS\_ENABLE} and \id{BLAS\_LIBRARIES} 
CMake variables, additional error checking during CMake configuration,
minor bug fixes, and renaming CMake options to enable/disable examples 
for greater clarity and an added option to enable/disable Fortran 77 examples.
These changes included changing \id{EXAMPLES\_ENABLE} to \id{EXAMPLES\_ENABLE\_C}, 
changing \id{CXX\_ENABLE} to \id{EXAMPLES\_ENABLE\_CXX}, changing \id{F90\_ENABLE} to 
\id{EXAMPLES\_ENABLE\_F90}, and adding an \id{EXAMPLES\_ENABLE\_F77} option.

A bug fix was made in \id{CVodeFree} to call \id{lfree} unconditionally 
(if non-NULL).
 
Corrections and additions were made to the examples, 
to installation-related files,
and to the user documentation.


\subsection*{Changes in v2.9.0}

Two additional {\nvector} implementations were added -- one for
Hypre (parallel) ParVector vectors, and one for {\petsc} vectors.  These
additions are accompanied by additions to various interface functions
and to user documentation.

Each {\nvector} module now includes a function, \id{N\_VGetVectorID},
that returns the {\nvector} module name.

For each linear solver, the various solver performance counters are
now initialized to 0 in both the solver specification function and in
solver \id{linit} function.  This ensures that these solver counters
are initialized upon linear solver instantiation as well as at the
beginning of the problem solution.

In {\fcvode}, corrections were made to three Fortran interface
functions.  Missing Fortran interface routines were added so that 
users can supply the sparse Jacobian routine when using sparse direct 
solvers.

A memory leak was fixed in the banded preconditioner interface.
In addition, updates were done to return integers from linear solver 
and preconditioner 'free' functions.

The Krylov linear solver Bi-CGstab was enhanced by removing a redundant
dot product.  Various additions and corrections were made to the
interfaces to the sparse solvers KLU and SuperLU\_MT, including support
for CSR format when using KLU.

New examples were added for use of the openMP vector and for use of 
sparse direct solvers from Fortran.

Minor corrections and additions were made to the {\cvode} solver, to the
Fortran interfaces, to the examples, to installation-related files,
and to the user documentation.

\subsection*{Changes in v2.8.0}

Two major additions were made to the linear system solvers that are
available for use with the {\cvode} solver.  First, in the serial case,
an interface to the sparse direct solver KLU was added.
Second, an interface to SuperLU\_MT, the multi-threaded version of
SuperLU, was added as a thread-parallel sparse direct solver option,
to be used with the serial version of the NVECTOR module.
As part of these additions, a sparse matrix (CSC format) structure 
was added to {\cvode}.

Otherwise, only relatively minor modifications were made to the
{\cvode} solver:

In \id{cvRootfind}, a minor bug was corrected, where the input
array \id{rootdir} was ignored, and a line was added to break out of
root-search loop if the initial interval size is below the tolerance
\id{ttol}.

In \id{CVLapackBand}, the line \id{smu = MIN(N-1,mu+ml)} was changed to
\id{smu = mu + ml} to correct an illegal input error for \id{DGBTRF/DGBTRS}.

In order to eliminate or minimize the differences between the sources
for private functions in {\cvode} and {\cvodes}, the names of 48
private functions were changed from \id{CV**} to \id{cv**}, and a few
other names were also changed.

Two minor bugs were fixed regarding the testing of input on the first
call to \id{CVode} -- one involving \id{tstop} and one involving the
initialization of \id{*tret}.

In order to avoid possible name conflicts, the mathematical macro
and function names \id{MIN}, \id{MAX}, \id{SQR}, \id{RAbs}, \id{RSqrt},
\id{RExp}, \id{RPowerI}, and \id{RPowerR} were changed to
\id{SUNMIN}, \id{SUNMAX}, \id{SUNSQR}, \id{SUNRabs}, \id{SUNRsqrt},
\id{SUNRexp}, \id{SRpowerI}, and \id{SUNRpowerR}, respectively.
These names occur in both the solver and in various example programs.

The example program \id{cvAdvDiff\_diag\_p} was added to illustrate
the use of \id{CVDiag} in parallel.

In the FCVODE optional input routines \id{FCVSETIIN} and \id{FCVSETRIN},
the optional fourth argument \id{key\_length} was removed, with
hardcoded key string lengths passed to all \id{strncmp} tests.

In all FCVODE examples, integer declarations were revised so that
those which must match a C type \id{long int} are declared \id{INTEGER*8},
and a comment was added about the type match.  All other integer
declarations are just \id{INTEGER}.  Corresponding minor corrections were
made to the user guide.

Two new {\nvector} modules have been added for thread-parallel computing
environments --- one for openMP, denoted \id{NVECTOR\_OPENMP},
and one for Pthreads, denoted \id{NVECTOR\_PTHREADS}.

With this version of {\sundials}, support and documentation of the
Autotools mode of installation is being dropped, in favor of the
CMake mode, which is considered more widely portable.

\subsection*{Changes in v2.7.0}

One significant design change was made with this release: The problem
size and its relatives, bandwidth parameters, related internal indices,
pivot arrays, and the optional output \id{lsflag} have all been
changed from type \id{int} to type \id{long int}, except for the
problem size and bandwidths in user calls to routines specifying
BLAS/LAPACK routines for the dense/band linear solvers.  The function
\id{NewIntArray} is replaced by a pair \id{NewIntArray}/\id{NewLintArray},
for \id{int} and \id{long int} arrays, respectively.

A large number of minor errors have been fixed.  Among these are the following:
In \id{CVSetTqBDF}, the logic was changed to avoid a divide by zero.
After the solver memory is created, it is set to zero before being filled.
In each linear solver interface function, the linear solver memory is
freed on an error return, and the \id{**Free} function now includes a
line setting to NULL the main memory pointer to the linear solver memory.
In the rootfinding functions \id{CVRcheck1}/\id{CVRcheck2}, when an exact
zero is found, the array \id{glo} of $g$ values at the left endpoint is
adjusted, instead of shifting the $t$ location \id{tlo} slightly.
In the installation files, we modified the treatment of the macro
SUNDIALS\_USE\_GENERIC\_MATH, so that the parameter GENERIC\_MATH\_LIB is
either defined (with no value) or not defined.

\subsection*{Changes in v2.6.0}

Two new features were added in this release: (a) a new linear solver module,
based on Blas and Lapack for both dense and banded matrices, and (b) an option
to specify which direction of zero-crossing is to be monitored while performing
rootfinding. 

The user interface has been further refined. Some of the API changes involve:
(a) a reorganization of all linear solver modules into two families (besides 
the existing family of scaled preconditioned iterative linear solvers,
the direct solvers, including the new Lapack-based ones, were also organized 
into a {\em direct} family); (b) maintaining a single pointer to user data,
optionally specified through a \id{Set}-type function; and (c) a general 
streamlining of the preconditioner modules distributed with the solver.

\subsection*{Changes in v2.5.0}

The main changes in this release involve a rearrangement of the entire 
{\sundials} source tree (see \S\ref{ss:sun_org}). At the user interface 
level, the main impact is in the mechanism of including {\sundials} header
files which must now include the relative path (e.g. \id{\#include <cvode/cvode.h>}).
Additional changes were made to the build system: all exported header files are
now installed in separate subdirectories of the instaltion {\em include} directory.

The functions in the generic dense linear solver (\id{sundials\_dense} and
\id{sundials\_smalldense}) were modified to work for rectangular $m \times n$
matrices ($m \le n$), while the factorization and solution functions were
renamed to \id{DenseGETRF}/\id{denGETRF} and \id{DenseGETRS}/\id{denGETRS}, 
respectively.
The factorization and solution functions in the generic band linear solver were 
renamed \id{BandGBTRF} and \id{BandGBTRS}, respectively.

\subsection*{Changes in v2.4.0}

{\cvspbcg} and {\cvsptfqmr} modules have been added to interface with the
Scaled Preconditioned Bi-CGstab ({\spbcg}) and Scaled Preconditioned
Transpose-Free Quasi-Minimal Residual ({\sptfqmr}) linear solver modules,
respectively (for details see Chapter \ref{s:simulation}). Corresponding
additions were made to the {\F} interface module {\fcvode}.
At the same time, function type names for Scaled Preconditioned Iterative
Linear Solvers were added for the user-supplied Jacobian-times-vector and
preconditioner setup and solve functions.

The deallocation functions now take as arguments the address of the respective 
memory block pointer.

To reduce the possibility of conflicts, the names of all header files have
been changed by adding unique prefixes (\id{cvode\_} and \id{sundials\_}).
When using the default installation procedure, the header files are exported
under various subdirectories of the target \id{include} directory. For more
details see Appendix \ref{c:install}.

\subsection*{Changes in v2.3.0}

The user interface has been further refined. Several functions used
for setting optional inputs were combined into a single one.  An optional
user-supplied routine for setting the error weight vector was added.
Additionally, to resolve potential variable scope issues, all
SUNDIALS solvers release user data right after its use. The build
systems has been further improved to make it more robust.

\subsection*{Changes in v2.2.1}

The changes in this minor {\sundials} release affect only the build system.

\subsection*{Changes in v2.2.0}

The major changes from the previous version involve a redesign of the
user interface across the entire {\sundials} suite. We have eliminated the
mechanism of providing optional inputs and extracting optional statistics 
from the solver through the \id{iopt} and \id{ropt} arrays. Instead,
{\cvode} now provides a set of routines (with prefix \id{CVodeSet})
to change the default values for various quantities controlling the
solver and a set of extraction routines (with prefix \id{CVodeGet})
to extract statistics after return from the main solver routine.
Similarly, each linear solver module provides its own set of {\id{Set}-}
and {\id{Get}-type} routines. For more details see \S\ref{ss:optional_input}
and \S\ref{ss:optional_output}.

Additionally, the interfaces to several user-supplied routines
(such as those providing Jacobians and preconditioner information) 
were simplified by reducing the number
of arguments. The same information that was previously accessible
through such arguments can now be obtained through {\id{Get}-type}
functions.

The rootfinding feature was added, whereby the roots of a set of given
functions may be computed during the integration of the ODE system.

Installation of {\cvode} (and all of {\sundials}) has been completely 
redesigned and is now based on configure scripts.


\section{Reading this User Guide}\label{ss:reading}

This user guide is a combination of general usage instructions. 
Specific example programs are provided as a separate document.  
We expect that some readers will want to
concentrate on the general instructions, while others will refer
mostly to the examples, and the organization is intended to
accommodate both styles.

There are different possible levels of usage of {\cvode}. The most
casual user, with a small IVP problem only, can get by with reading
\S\ref{ss:ivp_sol}, then Chapter \ref{s:simulation} through
\S\ref{sss:cvode} only, and looking at examples in~\cite{cvode_ex}.

In a different direction, a more expert user with an IVP problem may want to
(a) use a package preconditioner (\S\ref{ss:preconds}), 
(b) supply his/her own Jacobian or preconditioner routines (\S\ref{ss:user_fct_sim}),
(c) do multiple runs of problems of the same size (\S\ref{sss:cvreinit}), 
(d) supply a new {\nvector} module (Chapter \ref{s:nvector}), or even 
(e) supply new {\sunlinsol} and/or {\sunmatrix} modules (Chapters
\ref{s:sunmatrix} and \ref{s:sunlinsol}).

The structure of this document is as follows:
\begin{itemize}
\item
  In Chapter \ref{s:math}, we give short descriptions of the numerical
  methods implemented by {\cvode} for the solution of initial value
  problems for systems of ODEs, and continue with short descriptions of
  preconditioning (\S\ref{s:preconditioning}), stability limit detection
  (\S\ref{s:bdf_stab}), and rootfinding (\S\ref{ss:rootfinding}).
\item
  The following chapter describes the structure of the {\sundials} suite
  of solvers (\S\ref{ss:sun_org}) and the software organization of the {\cvode}
  solver (\S\ref{ss:cvode_org}). 
\item
  Chapter \ref{s:simulation} is the main usage document for {\cvode} for
  {\CC} applications.  It includes a complete description of the user interface
  for the integration of ODE initial value problems.
\item
  In Chapter \ref{s:fcmix}, we describe {\fcvode}, an interface module
  for the use of {\cvode} with {\F} applications.
\item
  Chapter \ref{s:nvector} gives a brief overview of the generic
  {\nvector} module shared among the various components of
  {\sundials}, and details on the {\nvector} implementations
  provided with {\sundials}.
\item
  Chapter \ref{s:sunmatrix} gives a brief overview of the generic
  {\sunmatrix} module shared among the various components of
  {\sundials}, and details on the {\sunmatrix} implementations
  provided with {\sundials}: 
  a dense implementation (\S\ref{ss:sunmat_dense}),
  a banded implementation (\S\ref{ss:sunmat_band}) and
  a sparse implementation (\S\ref{ss:sunmat_sparse}).
\item
  Chapter \ref{s:sunlinsol} gives a brief overview of the generic
  {\sunlinsol} module shared among the various components of
  {\sundials}.  This chapter contains details on the {\sunlinsol}
  implementations provided with {\sundials}.
  The chapter
  also contains details on the {\sunlinsol} implementations provided
  with {\sundials} that interface with external linear solver
  libraries.
%% \item
%%   Chapter \ref{s:new_linsolv} describes the interfaces to the linear
%%   solver modules, so that a user can provide his/her own such module.
%% \item
%%   Chapter \ref{s:gen_linsolv} describes in detail the generic linear
%%   solvers shared by all {\sundials} solvers.
\item
  Finally, in the appendices, we provide detailed instructions for the installation
  of {\cvode}, within the structure of {\sundials} (Appendix \ref{c:install}), as well
  as a list of all the constants used for input to and output from {\cvode} functions
  (Appendix \ref{c:constants}).
\end{itemize}

Finally, the reader should be aware of the following notational conventions
in this user guide:  program listings and identifiers (such as \id{CVodeInit}) 
within textual explanations appear in typewriter type style; 
fields in {\CC} structures (such as {\em content}) appear in italics;
and packages or modules, such as {\cvdls}, are written in all capitals. 
Usage and installation instructions that constitute important warnings
are marked with a triangular symbol {\warn} in the margin.

\paragraph{Acknowledgments.}
We wish to acknowledge the contributions to previous versions of the
{\cvode} and {\pvode} codes and their user guides by Scott D. Cohen~\cite{CoHi:94}
and George D. Byrne~\cite{ByHi:98}.

%---------------------------------
% License information section
\section{SUNDIALS Release License}

The SUNDIALS packages are released open source, under a BSD license.
The only requirements of the BSD license are preservation of copyright and a standard disclaimer of liability.
Our Copyright notice is below along with the license.

**PLEASE NOTE**  If you are using SUNDIALS with any third 
party libraries linked in (e.g., LaPACK, KLU, SuperLU\_MT, PETSc, 
or hypre), be sure to review the respective license of the package as 
that license may have more restrictive terms than the SUNDIALS license.  
For example, if someone builds SUNDIALS with a statically linked KLU, 
the build is subject to terms of the LGPL license (which is what 
KLU is released with) and *not* the SUNDIALS BSD license anymore.

\subsection{Copyright Notices}
All SUNDIALS packages except ARKode are subject to the following Copyright
notice.

\subsubsection{SUNDIALS Copyright}
Copyright (c) 2002-2016, Lawrence Livermore National Security. 
Produced at the Lawrence Livermore National Laboratory.
Written by A.C. Hindmarsh, D.R. Reynolds, R. Serban, C.S. Woodward,
S.D. Cohen, A.G. Taylor, S. Peles, L.E. Banks, and D. Shumaker. \\
UCRL-CODE-155951    (CVODE)\\
UCRL-CODE-155950    (CVODES)\\
UCRL-CODE-155952    (IDA)\\
UCRL-CODE-237203    (IDAS)\\
LLNL-CODE-665877    (KINSOL)\\
All rights reserved. 
 
\subsubsection{ARKode Copyright}
ARKode is subject to the following joint Copyright notice.
Copyright (c) 2015-2016, Southern Methodist University and 
Lawrence Livermore National Security
Written by D.R. Reynolds, D.J. Gardner, A.C. Hindmarsh, C.S. Woodward,
and J.M. Sexton.\\
LLNL-CODE-667205    (ARKODE) \\
All rights reserved.

\subsection{BSD License}
Redistribution and use in source and binary forms, with or without
modification, are permitted provided that the following conditions
are met:
 
1. Redistributions of source code must retain the above copyright
notice, this list of conditions and the disclaimer below.
 
2. Redistributions in binary form must reproduce the above copyright
notice, this list of conditions and the disclaimer (as noted below)
in the documentation and/or other materials provided with the
distribution.
 
3. Neither the name of the LLNS/LLNL nor the names of its contributors
may be used to endorse or promote products derived from this software
without specific prior written permission.
 
THIS SOFTWARE IS PROVIDED BY THE COPYRIGHT HOLDERS AND CONTRIBUTORS 
"AS IS" AND ANY EXPRESS OR IMPLIED WARRANTIES, INCLUDING, BUT NOT 
LIMITED TO, THE IMPLIED WARRANTIES OF MERCHANTABILITY AND FITNESS 
FOR A PARTICULAR PURPOSE ARE DISCLAIMED. IN NO EVENT SHALL 
LAWRENCE LIVERMORE NATIONAL SECURITY, LLC, THE U.S. DEPARTMENT OF 
ENERGY OR CONTRIBUTORS BE LIABLE FOR ANY DIRECT, INDIRECT, INCIDENTAL, 
SPECIAL, EXEMPLARY, OR CONSEQUENTIAL DAMAGES (INCLUDING, BUT NOT LIMITED 
TO, PROCUREMENT OF SUBSTITUTE GOODS OR SERVICES; LOSS OF USE, 
DATA, OR PROFITS; OR BUSINESS INTERRUPTION) HOWEVER CAUSED AND ON ANY 
THEORY OF LIABILITY, WHETHER IN CONTRACT, STRICT LIABILITY, OR TORT 
(INCLUDING NEGLIGENCE OR OTHERWISE) ARISING IN ANY WAY OUT OF THE USE 
OF THIS SOFTWARE, EVEN IF ADVISED OF THE POSSIBILITY OF SUCH DAMAGE.
\\
\\
Additional BSD Notice
\begin{enumerate}
\item{} 
This notice is required to be provided under our contract with
the U.S. Department of Energy (DOE). This work was produced at
Lawrence Livermore National Laboratory under Contract 
No. DE-AC52-07NA27344 with the DOE.

\item{} 
Neither the United States Government nor Lawrence Livermore 
National Security, LLC nor any of their employees, makes any warranty, 
express or implied, or assumes any liability or responsibility for the
accuracy, completeness, or usefulness of any information, apparatus,
product, or process disclosed, or represents that its use would not
infringe privately-owned rights.

\item{} 
Also, reference herein to any specific commercial products, process, 
or services by trade name, trademark, manufacturer or otherwise does 
not necessarily constitute or imply its endorsement, recommendation, 
or favoring by the United States Government or Lawrence Livermore 
National Security, LLC. The views and opinions of authors expressed 
herein do not necessarily state or reflect those of the United States 
Government or Lawrence Livermore National Security, LLC, and shall 
not be used for advertising or product endorsement purposes.

\end{enumerate}

%---------------------------------
