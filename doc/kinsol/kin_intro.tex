%===================================================================================
\chapter{Introduction}\label{s:intro}
%===================================================================================

{\kinsol} is part of a software family called {\sundials}: SUite
of Nonlinear and DIfferential/ALgebraic equation Solvers. This
suite consists of {\cvode}, {\kinsol}, and {\ida}, and variants of
these.
%
{\kinsol}\index{KINSOL@{\kinsol}!brief description of} is a
general-purpose nonlinear system solver based on Newton-Krylov
solver technology.

%---------------------------------
\section{Historical Background}\label{ss:history}
%---------------------------------

\index{KINSOL@{\kinsol}!relationship to NKSOL}  The first
nonlinear solver packages based on Newton-Krylov methods were
written in {\F}.  In particular, the NKSOL package, written at
LLNL, was the first Newton-Krylov solver package written for
solution of systems arising in solution of partial differential
equations~\cite{BrSa:90}.  This {\F} code made use of Newton's
method to solve the discrete nonlinear systems and applied a
preconditioned Krylov linear solver for solution of the Jacobian
system at each nonlinear iteration.  The key to the Newton-Krylov
method was that the matrix-vector multiplies required by the
Krylov method could effectively be approximated by a finite
difference of the nonlinear system-defining function, preventing a
requirement for the formation of the actual Jacobian matrix.
Significantly less memory was required for the solver as a result.

In the late 1990's, there was a push at LLNL to rewrite the
nonlinear solver into {\C} and port it to distributed memory
parallel machines.  Both Newton and Krylov methods are easily
implemented in parallel, and this effort gave rise to the
{\kinsol} package. {\kinsol} is similar to NKSOL in functionality,
except that it provides for more options in the choice of linear
system tolerances and has a more modular design to provide
flexibility for future enhancements.
\index{KINSOL@{\kinsol}!relationship to NKSOL}

At present, {\kinsol} contains three Krylov methods: the GMRES
(Generalized Minimal RESidual)~\cite{SaSc:86}, Bi-CGStab
(Bi-Conjugate Gradient Stabilized)~\cite{Van:92}, and TFQMR
(Transpose-Free Quasi-Minimal Residual)~\cite{Fre:93} linear
iterative methods. As Krylov methods, these require almost no
matrix storage as compared to direct methods. However, the
algorithms allow for a user-supplied preconditioner matrix,
and for most problems preconditioning is essential for an
efficient solution. For very large nonlinear algebraic systems,
the Krylov methods are preferable over direct linear solver methods,
and are often the only feasible choice. Among the three Krylov
methods in {\kinsol}, we recommend GMRES as the best overall
choice. However, users are encouraged to compare all three,
especially if encountering convergence failures with GMRES.
Bi-CGStab and TFQMR have an advantage in storage requirements,
in that the number of workspace vectors they require is fixed,
while that number for GMRES depends on the desired Krylov
subspace size.

In the process of translating NKSOL into {\C}, the overall
{\kinsol} organization has been changed considerably. One key
feature of the {\kinsol} organization is that a separate module
devoted to vector operations has been created.  This module
facilitated extension to multiprosessor environments with minimal
impact on the rest of the solver. The new vector module design is
shared across the {\sundials} suite. This {\nvector} module is
written in terms of abstract vector operations with the actual
routines attached by a particular implementation (such as serial
or parallel) of {\nvector}. This allows writing the {\sundials}
solvers in a manner independent of the actual {\nvector}
implementation (which can be user-supplied), as well as allowing
more than one {\nvector} module linked into an executable file.

\index{KINSOL@{\kinsol}!motivation for writing in C|(} There are
several motivations for choosing the {\C} language for {\kinsol}.
First, a general movement away from {\F} and toward {\C} in
scientific computing is apparent. Second, the pointer, structure,
and dynamic memory allocation features in C are extremely useful
in software of this complexity, with the great variety of method
options offered. Finally, we prefer {\C} over {\CPP} for {\kinsol}
because of the wider availability of {\C} compilers, the
potentially greater efficiency of {\C}, and the greater ease of
interfacing the solver to applications written in {\F}.
\index{KINSOL@{\kinsol}!motivation for writing in C|)}

\section{Changes from previous versions}

\subsection*{Changes in v2.6.0}

This release introduces a new linear solver module, based on Blas and Lapack 
for both dense and banded matrices.

The user interface has been further refined. Some of the API changes involve:
(a) a reorganization of all linear solver modules into two families (besides 
the already present family of scaled preconditioned iterative linear solvers,
the direct solvers, including the new Lapack-based ones were also organized 
into a {\em direct} family); (b) maintaining a single pointer to user data,
optionally specified through a \id{Set}-type function; (c) a general 
streamlining of the band-block-diagonal preconditioner module distributed 
with the solver.

\subsection*{Changes in v2.5.0}

The main changes in this release involve a rearrangement of the entire 
{\sundials} source tree (see \S\ref{ss:sun_org}). At the user interface 
level, the main impact is in the mechanism of including {\sundials} header
files which must now include the relative path (e.g. \id{\#include <cvode/cvode.h>}).
Additional changes were made to the build system: all exported header files are
now installed in separate subdirectories of the instaltion {\em include} directory.

The functions in the generic dense linear solver (\id{sundials\_dense} and
\id{sundials\_smalldense}) were modified to work for rectangular $m \times n$
matrices ($m \le n$), while the factorization and solution functions were
renamed to \id{DenseGETRF}/\id{denGETRF} and \id{DenseGETRS}/\id{denGETRS}, 
respectively.
The factorization and solution functions in the generic band linear solver were 
renamed \id{BandGBTRF} and \id{BandGBTRS}, respectively.

\subsection*{Changes in v2.4.0}

{\kinspbcg}, {\kinsptfqmr}, {\kindense}, and {\kinband} modules have been
added to interface with the Scaled Preconditioned Bi-CGStab ({\spbcg}),
Scaled Preconditioned Transpose-Free Quasi-Minimal Residual ({\sptfqmr}),
{\dense}, and {\band} linear solver modules, respectively (for details
see Chapter \ref{c:usage}). Corresponding additions were made to the {\F}
interface module {\fkinsol}. At the same time, function type names for
Scaled Preconditioned Iterative Linear Solvers were added for the
user-supplied Jacobian-times-vector and preconditioner setup and solve
functions.

Regarding the {\F} interface module {\fkinsol}, optional inputs are now
set using \id{FKINSETIIN} (integer inputs), \id{FKINSETRIN} (real inputs),
and \id{FKINSETVIN} (vector inputs). Optional outputs are still obtained
from the \id{IOUT} and \id{ROUT} arrays which are owned by the user
and passed as arguments to \id{FKINMALLOC}.

The {\kindense} and {\kinband} linear solver modules include support for
nonlinear residual monitoring which can be used to control Jacobian
updating.

To reduce the possibility of conflicts, the names of all header files have
been changed by adding unique prefixes (\id{kinsol\_} and \id{sundials\_}).
When using the default installation procedure, the header files are exported
under various subdirectories of the target \id{include} directory. For more
details see~\S\ref{c:install}.

\subsection*{Changes in v2.3.0}

The user interface has been further refined. Several functions used for setting optional
inputs were combined into a single one. Additionally, to resolve potential variable scope
issues, all SUNDIALS solvers release user data right after its use. The build systems has 
been further improved to make it more robust.

\subsection*{Changes in v2.2.1}

The changes in this minor {\sundials} release affect only the build system.

\subsection*{Changes in v2.2.0}

The major changes from the previous version involve a redesign of
the user interface across the entire {\sundials} suite. We have
eliminated the mechanism of providing optional inputs and
extracting optional statistics from the solver through the
\id{iopt} and \id{ropt} arrays. Instead, {\kinsol} now provides a
set of routines (with prefix \id{KINSet}) to change the default
values for various quantities controlling the solver and a set of
extraction routines (with prefix \id{KINGet}) to extract
statistics after return from the main solver routine. Similarly,
each linear solver module provides its own set of {\id{set}-} and
{\id{get}-type} routines. For more details see Chapter \ref{c:usage}.

Additionally, the interfaces to several user-supplied routines
(such as those providing Jacobian-vector products and
preconditioner information) were simplified by reducing the number
of arguments. The same information that was previously accessible
through such arguments can now be obtained through {\id{set}-type}
functions.

\section{Reading this User Guide}\label{ss:reading}

This user guide is a combination of general usage instructions and specific
examples. We expect that some readers will want to concentrate on the general
instructions, while others will refer mostly to the examples, and the
organization is intended to accommodate both styles.

There are different possible levels of usage of {\kinsol}. The most casual
user, with a small nonlinear system, can get by with reading all of 
Chapter~\ref{s:math}, then Chapter~\ref{c:usage} through \S\ref{sss:kinsol} only, and
looking at examples in~\cite{kinsol_ex}. In a different direction, a
more expert user with a nonlinear system may want to (a) use a package
preconditioner (\S\ref{sss:kinbbdpre}), (b) supply his/her own Jacobian or
preconditioner routines (\S\ref{ss:user_fct_sol}), (c) supply a new
{\nvector} module (Chapter~\ref{s:nvector}), or even (d) supply a different
linear solver module (\S\ref{ss:kinsol_org} and Chapter~\ref{s:gen_linsolv}).

The structure of this document is as follows:
\begin{itemize}
\item
  In Chapter \ref{s:math}, we provide short descriptions of the numerical
  methods implemented by {\kinsol} for the solution of nonlinear systems.
\item
  The following chapter describes the structure of the {\sundials} suite
  of solvers (\S\ref{ss:sun_org}) and the software organization of the {\kinsol}
  solver (\S\ref{ss:kinsol_org}).
\item
  Chapter \ref{c:usage} is the main usage document for {\kinsol} for {\C} applications.
  It includes a complete description of the user interface for the solution
  of nonlinear algebraic systems.
\item
  In Chapter \ref{s:fcmix}, we describe {\fkinsol}, an interface module for the
  use of {\kinsol} with {\F} applications.
\item
  Chapter \ref{s:nvector} gives a brief overview of the generic {\nvector} module
  shared among the various components of {\sundials}, and details on the two
  {\nvector} implementations provided with {\sundials}: a serial implementation
  (\S\ref{ss:nvec_ser}) and a parallel implementation based on {\mpi} \index{MPI}
  (\S\ref{ss:nvec_par}).
\item
  Chapter \ref{s:new_linsolv} describes the interfaces to the linear solver
  modules, so that a user can provide his/her own such module.
\item
  Chapter \ref{s:gen_linsolv} describes in detail the generic linear solvers
  shared by all {\sundials} solvers.
\item
  Finally, in the appendices, we provide detailed instructions for the installation
  of {\kinsol}, within the structure of {\sundials} (Appendix \ref{c:install}), as well
  as a list of all the constants used for input to and output from {\kinsol} functions
  (Appendix \ref{c:constants}).
\end{itemize}

Finally, the reader should be aware of the following notational
conventions in this user guide:  program listings and identifiers
(such as \id{KINInit}) within textual explanations appear in
typewriter type style; fields in {\C} structures (such as {\em
content}) appear in italics; and packages or modules are written
in all capitals. In the index, page numbers that appear in bold
indicate the main reference for that entry.

\paragraph{Acknowledgments.}
We wish to acknowledge the contributions to previous versions of the
{\kinsol} code and user guide of Allan G. Taylor.
