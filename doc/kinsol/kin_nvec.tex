%===================================================================================
\chapter{Description of the NVECTOR module}\label{s:nvector}
%===================================================================================
\index{NVECTOR@\texttt{NVECTOR} module}
% This is a shared SUNDIALS TEX file with description of
% the generic nvector abstraction
%
The {\sundials} solvers are written in a data-independent manner. 
They all operate on generic vectors (of type \Id{N\_Vector}) through a set of
operations defined by the particular {\nvector} implementation.
Users can provide their own specific implementation of the {\nvector} module,
or use one of four provided within {\sundials} -- a serial implementation and
three parallel implementations.  The generic operations are described below.
In the sections following, the implementations provided with {\sundials}
are described.

The generic \ID{N\_Vector} type is a pointer to a structure that has an 
implementation-dependent {\em content} field containing the 
description and actual data of the vector, and an {\em ops} field 
pointing to a structure with generic vector operations.
The type \id{N\_Vector} is defined as
%%
%%
\begin{verbatim}
typedef struct _generic_N_Vector *N_Vector;

struct _generic_N_Vector {
    void *content;
    struct _generic_N_Vector_Ops *ops;
};
\end{verbatim}
%%
%%
The \id{\_generic\_N\_Vector\_Ops} structure is essentially a list of pointers to
the various actual vector operations, and is defined as
%%
\begin{verbatim}
struct _generic_N_Vector_Ops {
  N_Vector    (*nvclone)(N_Vector);
  N_Vector    (*nvcloneempty)(N_Vector);
  void        (*nvdestroy)(N_Vector);
  void        (*nvspace)(N_Vector, long int *, long int *);
  realtype*   (*nvgetarraypointer)(N_Vector);
  void        (*nvsetarraypointer)(realtype *, N_Vector);
  void        (*nvlinearsum)(realtype, N_Vector, realtype, N_Vector, N_Vector); 
  void        (*nvconst)(realtype, N_Vector);
  void        (*nvprod)(N_Vector, N_Vector, N_Vector);
  void        (*nvdiv)(N_Vector, N_Vector, N_Vector);
  void        (*nvscale)(realtype, N_Vector, N_Vector);
  void        (*nvabs)(N_Vector, N_Vector);
  void        (*nvinv)(N_Vector, N_Vector);
  void        (*nvaddconst)(N_Vector, realtype, N_Vector);
  realtype    (*nvdotprod)(N_Vector, N_Vector);
  realtype    (*nvmaxnorm)(N_Vector);
  realtype    (*nvwrmsnorm)(N_Vector, N_Vector);
  realtype    (*nvwrmsnormmask)(N_Vector, N_Vector, N_Vector);
  realtype    (*nvmin)(N_Vector);
  realtype    (*nvwl2norm)(N_Vector, N_Vector);
  realtype    (*nvl1norm)(N_Vector);
  void        (*nvcompare)(realtype, N_Vector, N_Vector);
  booleantype (*nvinvtest)(N_Vector, N_Vector);
  booleantype (*nvconstrmask)(N_Vector, N_Vector, N_Vector);
  realtype    (*nvminquotient)(N_Vector, N_Vector);
};
\end{verbatim}




The generic {\nvector} module defines and implements the vector operations 
acting on \id{N\_Vector}.
These routines are nothing but wrappers for the vector operations defined by
a particular {\nvector} implementation, which are accessed through the {\em ops}
field of the \id{N\_Vector} structure. To illustrate this point we
show below the implementation of a typical vector operation from the
generic {\nvector} module, namely \id{N\_VScale}, which performs the scaling of a
vector \id{x} by a scalar \id{c}:
%%
%%
\begin{verbatim}
void N_VScale(realtype c, N_Vector x, N_Vector z) 
{
   z->ops->nvscale(c, x, z);
}
\end{verbatim}
%%
%%
Table \ref{t:nvecops} contains a complete list of all vector operations defined
by the generic {\nvector} module.

Finally, note that the generic {\nvector} module defines the functions
\ID{N\_VCloneVectorArray} and \ID{N\_VCloneEmptyVectorArray}.  Both functions
create (by cloning) an array of \id{count} variables of type \id{N\_Vector}, each
of the same type as an existing \id{N\_Vector}. Their prototypes are
\begin{verbatim}
N_Vector *N_VCloneVectorArray(int count, N_Vector w);
N_Vector *N_VCloneEmptyVectorArray(int count, N_Vector w);
\end{verbatim}
and their definitions are based on the implementation-specific \id{N\_VClone} and
\id{N\_VCloneEmpty} operations, respectively.

An array of variables of type \id{N\_Vector} can be destroyed by
calling \ID{N\_VDestroyVectorArray}, whose prototype is
\begin{verbatim}
void N_VDestroyVectorArray(N_Vector *vs, int count);
\end{verbatim}
and whose definition is based on the implementation-specific \id{N\_VDestroy} operation.


A particular implementation of the {\nvector} module must:
\begin{itemize}
\item Specify the {\em content} field of \id{N\_Vector}.
\item Define and implement the vector operations. 
  Note that the names of these routines should be unique to that implementation in order 
  to permit using more than one {\nvector} module (each with different \id{N\_Vector} 
  internal data representations) in the same code.
\item Define and implement user-callable constructor and destructor
  routines to create and free an \id{N\_Vector} with
  the new {\em content} field and with {\em ops} pointing to the
  new vector operations.
\item Optionally, define and implement additional user-callable routines
  acting on the newly defined \id{N\_Vector} (e.g., a routine to print
  the content for debugging purposes).
\item Optionally, provide accessor macros as needed for that particular implementation to 
  be used to access different parts in the {\em content} field of the newly defined \id{N\_Vector}.
\end{itemize}



%---------------------------------------------------------------------------
% Table of vector kernels
%---------------------------------------------------------------------------
\newpage

\newlength{\colone}
\settowidth{\colone}{\id{N\_VGetArrayPointer}}
\newlength{\coltwo}
\setlength{\coltwo}{\textwidth}
\addtolength{\coltwo}{-0.5in}
\addtolength{\coltwo}{-\colone}

\tablecaption{Description of the NVECTOR operations}\label{t:nvecops}
\tablefirsthead{\hline {\rule{0mm}{5mm}}{\bf Name} & {\bf Usage and Description} \\[3mm] \hline\hline}
\tablehead{\hline \multicolumn{2}{|l|}{\small\slshape continued from last page} \\
           \hline {\rule{0mm}{5mm}}{\bf Name} & {\bf Usage and  Description} \\[3mm] \hline\hline}
\tabletail{\hline \multicolumn{2}{|r|}{\small\slshape continued on next page} \\ \hline}
\tablelasttail{\hline}
\begin{supertabular}{|p{\colone}|p{\coltwo}|}
%%
\id{N\_VClone} & \id{v = N\_VClone(w);} \\ 
& Creates a new \id{N\_Vector} of the same type as an existing vector \id{w} and sets the
{\em ops} field.
It does not copy the vector, but rather allocates storage for the new vector.
\\[2mm]
%%
\id{N\_VCloneEmpty} & \id{v = N\_VCloneEmpty(w);} \\ 
& Creates a new \id{N\_Vector} of the same type as an existing vector \id{w} and sets the
{\em ops} field.
It does not allocate storage for data.
\\[2mm]
%%
\id{N\_VDestroy} & \id{N\_VDestroy(v);} \\
& Destroys the \id{N\_Vector} \id{v} and frees memory allocated for its
internal data.
\\[2mm]
%%
\id{N\_VSpace} & \id{N\_VSpace(nvSpec, \&lrw, \&liw);} \\
& Returns storage requirements for one \id{N\_Vector}.
\id{lrw} contains the number of realtype words and \id{liw}
contains the number of integer words.
This function is advisory only, for use in determining a user's total
space requirements; it could be a dummy function in a user-supplied
{\nvector} module if that information is not of interest.
\\[2mm]
%%
\id{N\_VGetArrayPointer} & \id{vdata = N\_VGetArrayPointer(v);} \\
& Returns a pointer to a \id{realtype} array from the \id{N\_Vector} \id{v}.
Note that this assumes that the internal data in \id{N\_Vector} is
a contiguous array of \id{realtype}.
This routine is only used in the solver-specific interfaces to the dense and
banded (serial) linear solvers, the sparse linear solvers (serial and
threaded), and in the interfaces to the banded (serial)
and band-block-diagonal (parallel) preconditioner modules provided with {\sundials}.
\\[2mm]
%%
\id{N\_VSetArrayPointer} & \id{N\_VSetArrayPointer(vdata, v);} \\
& Overwrites the data in an \id{N\_Vector} with a given array of \id{realtype}.
Note that this assumes that the internal data in \id{N\_Vector} is
a contiguous array of \id{realtype}.
This routine is only used in the interfaces to the dense (serial) linear
solver, hence need not exist in a user-supplied {\nvector} module for a
parallel environment.
\\[2mm]
%%
\id{N\_VLinearSum} & \id{N\_VLinearSum(a, x, b, y, z);} \\
& Performs the operation $z = a x + b y$, where $a$ and $b$ are \id{realtype} 
scalars and $x$ and $y$ are of type \id{N\_Vector}:
$z_i = a x_i + b y_i, \: i=0,\ldots,n-1$.
\\[2mm]
%%
\id{N\_VConst} & \id{N\_VConst(c, z);} \\
& Sets all components of the \id{N\_Vector} \id{z} to \id{realtype} \id{c}:
$z_i = c,\: i=0,\ldots,n-1$.
\\[2mm]
%%
\id{N\_VProd} & \id{N\_VProd(x, y, z);} \\
& Sets the \id{N\_Vector} \id{z} to be the component-wise product of the
\id{N\_Vector} inputs \id{x} and \id{y}:
$z_i = x_i y_i,\: i=0,\ldots,n-1$.
\\[2mm]
%%
\id{N\_VDiv} & \id{N\_VDiv(x, y, z);} \\
& Sets the \id{N\_Vector} \id{z} to be the component-wise ratio of the
\id{N\_Vector} inputs \id{x} and \id{y}:
$z_i = x_i / y_i,\: i=0,\ldots,n-1$. The $y_i$ may not be tested 
for $0$ values. It should only be called with a \id{y} that is
guaranteed to have all nonzero components.
\\[2mm]
%%
\id{N\_VScale} & \id{N\_VScale(c, x, z);} \\
& Scales the \id{N\_Vector} \id{x} by the \id{realtype} scalar \id{c} 
and returns the result in \id{z}:
$z_i = c x_i , \: i=0,\ldots,n-1$.
\\[2mm]
%%
\id{N\_VAbs} & \id{N\_VAbs(x, z);} \\
& Sets the components of the \id{N\_Vector} \id{z} to be the absolute
values of the components of the \id{N\_Vector} \id{x}:
$y_i = | x_i | , \: i=0,\ldots,n-1$.
\\[2mm]
%%
\id{N\_VInv} & \id{N\_VInv(x, z);} \\
& Sets the components of the \id{N\_Vector} \id{z} to be the inverses
of the components of the \id{N\_Vector} \id{x}:
$z_i = 1.0 /  x_i  , \: i=0,\ldots,n-1$. This routine
may not check for division by $0$. It should be called only with an 
\id{x} which is guaranteed to have all nonzero components.
\\[2mm]
%%
\id{N\_VAddConst} & \id{N\_VAddConst(x, b, z);} \\
& Adds the \id{realtype} scalar \id{b} to all components of \id{x} 
and returns the result in the \id{N\_Vector} \id{z}:
$z_i = x_i + b , \: i=0,\ldots,n-1$.
\\[2mm]
%%
\id{N\_VDotProd} & \id{d = N\_VDotProd(x, y);} \\
& Returns the value of the ordinary dot product of \id{x} and \id{y}:
$d=\sum_{i=0}^{n-1} x_i y_i$.
\\[2mm]
%%
\id{N\_VMaxNorm} & \id{m = N\_VMaxNorm(x);} \\
& Returns the maximum norm of the \id{N\_Vector} \id{x}:
$m = \max_{i} | x_i |$.
\\[2mm]
%%
\id{N\_VWrmsNorm} & \id{m = N\_VWrmsNorm(x, w)} \\
& Returns the weighted root-mean-square norm of the \id{N\_Vector} \id{x} with
\id{realtype} weight vector \id{w}:
$m = \sqrt{\left( \sum_{i=0}^{n-1} (x_i w_i)^2 \right) / n}$.
\\[2mm]
%%
\id{N\_VWrmsNormMask} & \id{m = N\_VWrmsNormMask(x, w, id);} \\
& Returns the weighted root mean square norm of the \id{N\_Vector} \id{x} with
\id{realtype} weight vector \id{w} built using only 
the elements of \id{x} corresponding to
nonzero elements of the \id{N\_Vector} \id{id}:\\
&$m = \sqrt{\left( \sum_{i=0}^{n-1} (x_i w_i \text{sign}(id_i))^2 \right) / n}$.
\\[2mm]
%%
\id{N\_VMin} & \id{m = N\_VMin(x);} \\
& Returns the smallest element of the \id{N\_Vector} \id{x}:
$m = \min_i x_i $.
\\[2mm]
%%
\id{N\_VWL2Norm} & \id{m = N\_VWL2Norm(x, w);} \\
& Returns the weighted Euclidean $\ell_2$ norm of the \id{N\_Vector} \id{x}
with \id{realtype} weight vector \id{w}: 
$m = \sqrt{\sum_{i=0}^{n-1} (x_i w_i)^2}$.
\\[2mm]
%%
\id{N\_VL1Norm} & \id{m = N\_VL1Norm(x);} \\
& Returns the $\ell_1$ norm of the \id{N\_Vector} \id{x}:
$m = \sum_{i=0}^{n-1} | x_i |$.
\\[2mm]
%%
\id{N\_VCompare} & \id{N\_VCompare(c, x, z);} \\
& Compares the components of the \id{N\_Vector} \id{x} to the \id{realtype}
scalar \id{c} and returns an \id{N\_Vector} \id{z} such that:
$z_i = 1.0$ if $| x_i | \ge c$ and $z_i = 0.0$ otherwise.
\\[2mm]
%%
\id{N\_VInvTest} & \id{t = N\_VInvTest(x, z);} \\
& Sets the components of the \id{N\_Vector} \id{z} to be the inverses
of the components of the \id{N\_Vector} \id{x}, with prior testing
for zero values:
$z_i = 1.0 /  x_i  , \: i=0,\ldots,n-1$.
This routine returns a boolean assigned to \id{TRUE} if all 
components of \id{x} are
nonzero (successful inversion) and returns \id{FALSE} otherwise.  
\\[2mm]
%%
\id{N\_VConstrMask} & \id{t = N\_VConstrMask(c, x, m);} \\
& Performs the following constraint tests:
$x_i > 0$ if $c_i=2$,
$x_i \ge 0$ if $c_i=1$,
$x_i \le 0$ if $c_i=-1$,
$x_i < 0$ if $c_i=-2$.
There is no constraint on $x_i$ if $c_i=0$.
This routine returns a boolean assigned to \id{FALSE} if any element failed
the constraint test and assigned to \id{TRUE} if all passed.  It also sets a
mask vector \id{m}, with elements equal to $1.0$ where the constraint 
test failed, and $0.0$ where the test passed.
This routine is used only for constraint checking.
\\[2mm]
%%
\id{N\_VMinQuotient} & \id{minq = N\_VMinQuotient(num, denom);} \\
& This routine returns the minimum of the quotients obtained   
by term-wise dividing \id{num}$_i$ by \id{denom}$_i$. 
A zero element in \id{denom} will be skipped. 
If no such quotients are found, then the large value 
\Id{BIG\_REAL} (defined in the header file \id{sundials\_types.h})
is returned. 
\\
%%
\end{supertabular}
\bigskip

%---------------------------------------------------------------------------
\section{The NVECTOR\_SERIAL implementation}\label{ss:nvec_ser}
% This is a shared SUNDIALS TEX file with description of
% the serial nvector implementation
%
The {\nvecs} implementation of the {\nvector} module
defines the {\em content} field of \id{NV\_Spec} to be a structure 
containing the length of the vector:
\begin{verbatim}
struct _NV_SpecContent_Serial {
  long int length;
};
\end{verbatim}
The {\em tag} field of \id{NV\_Spec} is set to \id{serial}.
The {\em content} field of \id{N\_Vector} is defined to be a structure containing
the length of the vector and a pointer to the beginning of a contiguous data array:
\begin{verbatim} 
struct _N_VectorContent_Serial {
  long int length;
  realtype   *data;
};
\end{verbatim}

The {\nvecs} implementation provides user-callable routines \ID{NV\_SpecInit\_Serial} 
to create a structure of type \id{NV\_Spec} whose {\em content} field is 
of type \id{struct \_NV\_SpecSerialContent}, and \ID{NV\_SpecFree\_Serial} 
to deallocate the space used by such a structure. 

\noindent The form of the call to \id{NV\_SpecInit\_Serial} is
\begin{verbatim}
nvspec = NV_SpecInit_Serial(vec_length);
\end{verbatim}
If successful, \id{NV\_SpecInit\_Serial} returns a pointer of type
\id{NV\_Spec}. \index{nvSpec@{\tt nvSpec}}This pointer should in turn be passed in any user
calls to \id{N\_VNew} to create a new \id{N\_Vector} of this type.
A vector specification object \id{nvspec} returned by \id{NV\_SpecInit\_Serial}
can be freed by calling:
\begin{verbatim}
NV_SpecFree_Serial(nvspec);
\end{verbatim}
In addition to these two routines, {\nvecs} defines serial implementations of all 
vector kernels listed in Table \ref{t:nvecops}, as well as the following macros
that can be used to access the contents of \id{NV\_Spec} and \id{N\_Vector}
or to create and destroy \id{N\_Vector}'s with component array data allocated
by the user. The suffix \id{\_S} in the names denotes serial version.
\begin{itemize}

\item \ID{NS\_CONTENT\_S}, \ID{NV\_CONTENT\_S}

      These macros give access to the contents of the serial 
      vector specification and \id{N\_Vector}, respectively.           
                                                               
      The assignment \id{ns\_cont = NS\_CONTENT\_S(nvspec)} sets       
      \id{ns\_cont} to be a pointer to the serial vector
      specification content structure (of type \id{struct \_NV\_SpecSerialContent}).

      The assignment \id{v\_cont = NV\_CONTENT\_S(v)} sets       
      \id{v\_cont} to be a pointer to the serial \id{N\_Vector} content    
      structure of type \id{struct \_N\_VectorSerialContent}.

\item \ID{NV\_DATA\_S}, \ID{NV\_LENGTH\_S}

      These macros give individual access to the parts of    
      the content of a serial \id{N\_Vector}.                        
                                                               
      The assignment \id{v\_data = NV\_DATA\_S(v)} sets \id{v\_data} to be     
      a pointer to the first component of the data for the \id{N\_Vector} \id{v}. 
      The assignment \id{NV\_DATA\_S(v) = v\_data} sets the component array of \id{v} to     
      be \id{v\_data} by storing the pointer \id{v\_data}.                   
                                                               
      The assignment \id{v\_len = NV\_LENGTH\_S(v)} sets \id{v\_len} to be     
      the length of \id{v}. On the other hand, the call \id{NV\_LENGTH\_S(v) = len\_v} 
      sets the length of \id{v} to be \id{len\_v}.

\item \ID{NV\_Ith\_S}

      This macro gives access to the individual components of the data
      array of an \id{N\_Vector}.

      The assignment \id{r = NV\_Ith\_S(v,i)} sets \id{r} to be the value of 
      the \id{i}-th component of \id{v}. The assignment \id{NV\_Ith\_S(v,i) = r}   
      sets the value of the \id{i}-th component of \id{v} to be \id{r}.        

      Here $i$ ranges from $0$ to $n-1$ for a vector of length $n$.

\item \ID{NV\_MAKE\_S}, \ID{NV\_DISPOSE\_S}

      These companion macros are used to create and          
      destroy an \id{N\_Vector} with a component array \id{vdata} 
      allocated by the user.                                   
                                                               
      The call \id{NV\_MAKE\_S(v,v\_data,nvspec)} makes \id{v} an        
      \id{N\_Vector} with component array pointer \id{v\_data}. The length of the  
      array is taken from \id{nvspec}.                             
      \id{NV\_MAKE\_S} stores the pointer \id{v\_data} so that changes      
      made by the user to the elements of \id{v\_data} are           
      simultaneously reflected in \id{v}. There is no copying of    
      elements.                                                
                                                               
      The call \id{NV\_DISPOSE\_S(v)} frees all memory associated     
      with \id{v} except for its component array. This memory was   
      allocated by the user and, therefore, should be          
      deallocated by the user.   

\item \ID{NVS\_MAKE\_S}, \ID{NVS\_DISPOSE\_S}
                             
      These companion macros are used to create and destroy  
      an array of \id{N\_Vector}'s with component array \id{vs\_data} 
      (of type \id{realtype **}) allocated by the user, for use in
      sensitivity calculations.

                                                               
      The call \id{NVS\_MAKE\_S(vs,vs\_data,ns,nvspec)} makes   
      \id{vs} an array of \id{ns} \id{N\_Vector}'s, with \id{vs[i]} 
      having component array pointer \id{vs\_data[i]} and length taken from \id{nvspec}.    
      \id{NVS\_MAKE\_S} stores the pointers \id{vs\_data[i]} so that        
      changes made by the user to the elements of \id{vs\_data} are  
      simultaneously reflected in \id{vs}. There is no copying of   
      elements.                                                
                                                               
      The call \id{NVS\_DISPOSE\_S(vs,ns)} frees all memory associated   
      with \id{vs} except for its components' component array.      
      This memory was allocated by the user and, therefore,    
      should be deallocated by the user.                       

\end{itemize}

\noindent {\bf Notes}                                                      
           
\begin{itemize}
                                        
\item
  Users who use the make/dispose macros must                 
  \id{\#include<stdlib.h>} since these macros expand to calls to     
  \id{malloc} and \id{free}.                                             
  
\item
  When looping over the components of an \id{N\_Vector} \id{v}, it is     
  more efficient to first obtain the component array via       
  \id{v\_data = NV\_DATA\_S(v)} and then access \id{v\_data[i]} within the     
  loop than it is to use \id{NV\_Ith\_S(v,i)} within the loop.        
                                     
\item                          
  \id{NV\_MAKE\_S} and \id{NV\_DISPOSE\_S} are similar to the \id{N\_VNew} and  
  \id{N\_VFree} implemented by {\nvecs}, while \id{NVS\_MAKE\_S} and 
  \id{NVS\_DISPOSE\_S}  are similar to  \id{N\_VNew\_S} and \id{N\_VFree\_S}. 
  The difference is one of responsibility for component memory     
  allocation and deallocation. \id{N\_VNew} allocates memory  
  for the \id{N\_Vector} components and \id{N\_VFree} frees the     
  component memory allocated by \id{N\_VNew}. For \id{NV\_MAKE\_S}   
  and \id{NV\_DISPOSE\_S}, the component memory is allocated and      
  freed by the user of this package. Similar remarks hold for  
  \id{NVS\_MAKE\_S},  \id{NVS\_DISPOSE\_S} and \id{N\_VNew\_S},              
  \id{N\_VFree\_S}.                                            

\item
  To maximize efficiency, vector kernels in the {\nvecs} implementation
  that have more than one \id{N\_Vector} argument do not check for
  consistent internal representation of these vectors. It is the user's 
  responsibility to ensure that such routines are called with \id{N\_Vector}
  arguments that were all created with the \id{NV\_Spec} structure returned
  by \id{NV\_SpecInit\_Serial}.

\end{itemize}


%---------------------------------------------------------------------------
\section{The NVECTOR\_PARALLEL implementation}\label{ss:nvec_par}
% This is a shared SUNDIALS TEX file with description of
% the parallel nvector implementation
%
The parallel implementation of the {\nvector} module defines the {\em content} 
field of \id{N\_Vector} to be a structure containing the global and local lengths 
of the vector, a pointer to the beginning of a contiguous local data array, and
an MPI communicator.
%%
\begin{verbatim} 
struct _N_VectorContent_Parallel {
  long int local_length;
  long int global_length;
  realtype *data;
  MPI_Comm comm;
};
\end{verbatim}
%%
%%--------------------------------------------
%%
The following seven macros are provided to access the content of a {\nvecp}
vector. The suffix \id{\_P} in the names denotes parallel version.
\begin{itemize}

\item 
  \ID{NV\_CONTENT\_P}

  This macro gives access to the contents of the parallel
  vector \id{N\_Vector}.
  
  The assignment \id{v\_cont = NV\_CONTENT\_P(v)} sets       
  \id{v\_cont} to be a pointer to the \id{N\_Vector} content    
  structure of type \id{struct \_N\_VectorParallelContent}.
  
  Implementation:
  
  \verb|#define NV_CONTENT_P(v) ( (N_VectorContent_Parallel)(v->content) )|
  
\item 
  \ID{NV\_OWN\_DATA\_P}, \ID{NV\_DATA\_P}, 
  \ID{NV\_LOCLENGTH\_P}, \ID{NV\_GLOBLENGTH\_P}
  
  These macros give individual access to the parts of    
  the content of a parallel \id{N\_Vector}.                        
  
  The assignment \id{v\_data = NV\_DATA\_P(v)} sets \id{v\_data} to be     
  a pointer to the first component of the local data for the \id{N\_Vector} \id{v}. 
  The assignment \id{NV\_DATA\_P(v) = v\_data} sets the component array of 
  \id{v} to be \id{v\_data} by storing the pointer \id{v\_data}.                   
  
  The assignment \id{v\_llen = NV\_LOCLENGTH\_P(v)} sets \id{v\_llen} to be     
  the length of the local part of \id{v}. 
  The call \id{NV\_LENGTH\_P(v) = llen\_v} sets      
  the local length of \id{v} to be \id{llen\_v}.
  
  The assignment \id{v\_glen = NV\_GLOBLENGTH\_P(v)} sets \id{v\_glen} to  
  be the global length of the vector \id{v}.                    
  The call \id{NV\_GLOBLENGTH\_P(v) = glen\_v} sets the global       
  length of \id{v} to be \id{glen\_v}.
  
  Implementation:
  
  \verb|#define NV_OWN_DATA_P(v)   ( NV_CONTENT_P(v)->own_data )|

  \verb|#define NV_DATA_P(v)       ( NV_CONTENT_P(v)->data )|

  \verb|#define NV_LOCLENGTH_P(v)  ( NV_CONTENT_P(v)->local_length )|

  \verb|#define NV_GLOBLENGTH_P(v) ( NV_CONTENT_P(v)->global_length )|
  
\item \ID{NV\_COMM\_P}

  This macro provides access to the MPI communicator used by the {\nvecp}
  vectors.

  Implementation:

  \verb|#define NV_COMM_P(v) ( NV_CONTENT_P(v)->comm )|

\item \ID{NV\_Ith\_P}

  This macro gives access to the individual components of the local data
  array of an \id{N\_Vector}.

  The assignment \id{r = NV\_Ith\_P(v,i)} sets \id{r} to be the value of 
  the \id{i}-th component of the local part of \id{v}. 
  The assignment \id{NV\_Ith\_P(v,i) = r}   
  sets the value of the \id{i}-th component of the local part of \id{v} 
  to be \id{r}.        
  
  Here $i$ ranges from $0$ to $n-1$, where $n$ is the local length.
      
  Implementation:

  \verb|#define NV_Ith_P(v,i) ( NV_DATA_P(v)[i] )|

\end{itemize}
%%
%%--------------------------------------------
%%
The {\nvecp} module defines serial implementations of all vector operations listed 
in Table \ref{t:nvecops} and provides the following user-callable routines:
%%
%%
\begin{itemize}

%%--------------------------------------

\item  \ID{N\_VNew\_Parallel}
  
  This function creates and allocates memory for a parallel vector.
 
  Prototype

\begin{verbatim}
N_Vector N_VNew_Parallel(MPI_Comm comm, 
                         long int local_length, 
                         long int global_length);
\end{verbatim}
  
%%--------------------------------------

\item \ID{N\_VNewEmpty\_Parallel}
 
  This function creates a new parallel \id{N\_Vector} with an empty (\id{NULL}) data array.
 
  Prototype

\begin{verbatim}
N_Vector N_VNewEmpty_Parallel(MPI_Comm comm, 
                              long int local_length, 
                              long int global_length);
\end{verbatim}

  
%%--------------------------------------

\item \ID{N\_VCloneEmpty\_Parallel}
 
  This function creates a new parallel \id{N\_Vector} with an empty (\id{NULL}) data array
  by using an existing \id{N\_Vector} as a template.
 
  Prototype

  \verb|N_Vector N_VCloneEmpty_Parallel(N_Vector w);|
  
%%--------------------------------------

\item \ID{N\_VMake\_Parallel}
  
  This function creates and allocates memory for a parallel vector
  with user-provided data array.
 
  Prototype

\begin{verbatim}
N_Vector N_VMake_Parallel(MPI_Comm comm, 
                          long int local_length,
                          long int global_length,
                          realtype *v_data);
\end{verbatim}

%%--------------------------------------

\item \ID{N\_VNewVectorArray\_Parallel}
 
  This function creates an array of \id{count} parallel vectors.
 
\begin{verbatim}
N_Vector *N_VNewVectorArray_Parallel(int count, 
                                     MPI_Comm comm, 
                                     long int local_length,
                                     long int global_length);
\end{verbatim}

%%--------------------------------------

\item \ID{N\_VNewVectorArrayEmpty\_Parallel}
 
  This function creates an array of \id{count} parallel vectors,
  each with an empty (\id{NULL}) data array.
 
\begin{verbatim}
N_Vector *N_VNewVectorArrayEmpty_Parallel(int count, 
                                          MPI_Comm comm, 
                                          long int local_length,
                                          long int global_length);
\end{verbatim}

%%--------------------------------------

\item \ID{N\_VDestroyVectorArray\_Parallel}
 
 This function frees memory allocated for the array of \id{count} parallel \id{N\_Vector} 
 created with \id{N\_VNewVectorArray\_Parallel} or with \id{N\_VNewVectorArrayEmpty\_Parallel}.

 Prototype

 \verb|void N_VDestroyVectorArray_Parallel(N_Vector *vs, int count);|


%%--------------------------------------

\item \ID{N\_VPrint\_Parallel}
  
  This function prints the content of a parallel vector to stdout.
 
  Prototype
  
  \verb|void N_VPrint_Parallel(N_Vector v);|


\end{itemize}
%%
%%------------------------------------
%%
\paragraph{\bf Notes} 
           
\begin{itemize}
                                        
\item
  When looping over the components of an \id{N\_Vector} \id{v}, it is     
  more efficient to first obtain the local component array via       
  \id{v\_data = NV\_DATA\_P(v)} and then access \id{v\_data[i]} within the     
  loop than it is to use \id{NV\_Ith\_P(v,i)} within the loop.        
                                                               
\item
  The {\nvecp} constructor functions \id{N\_VNewEmpty\_Parallel}, \id{N\_VMake\_Parallel}, 
  \id{N\_VCloneEmpty\_Parallel}, and \id{N\_VNewVectorArrayEmpty\_Parallel}
  set the field {\em own\_data} $=$ \id{FALSE}. 
  The functions \id{N\_VDestroy\_Parallel} and \id{N\_VDestroyVectorArray\_Parallel}
  will not attempt to free the pointer {\em data} for any \id{N\_Vector} with
  {\em own\_data} set to \id{FALSE}. In such a case, it is the user's responsibility to
  deallocate the {\em data} pointer.

\item
  To maximize efficiency, vector operations in the {\nvecp} implementation
  that have more than one \id{N\_Vector} argument do not check for
  consistent internal representation of these vectors. It is the user's 
  responsability to ensure that such routines are called with \id{N\_Vector}
  arguments that were all created with the \id{NV\_Spec} structure returned
  by \id{NV\_SpecInit\_Parallel}.

\end{itemize}



%---------------------------------------------------------------------------
\section{The NVECTOR\_OPENMP implementation}\label{ss:nvec_openmp}
%% This is a shared SUNDIALS TEX file with a description of the
%% OpenMP nvector implementation
%%

The OpenMP implementation of the {\nvector} module provided with {\sundials},
{\nvecopenmp}, defines the {\em content} field of \id{N\_Vector} to be a structure 
containing the length of the vector, a pointer to the beginning of a contiguous 
data array, and a boolean flag {\em own\_data} which specifies the ownership 
of {\em data}.  Operations on the vector are threaded using OpenMP, 
the number of threads used is based on the supplied argument in 
the vector constructor.
%%
\begin{verbatim} 
struct _N_VectorContent_OpenMP {
  long int length;
  booleantype own_data;
  realtype *data;
  int num_threads;
};
\end{verbatim}
%%
%%--------------------------------------------
%%
The following six macros are provided to access the content of an {\nvecopenmp}
vector. The suffix \id{\_OMP} in the names denotes OpenMP version.
%%
\begin{itemize}

\item \ID{NV\_CONTENT\_OMP}                             
    
  This routine gives access to the contents of the OpenMP
  vector \id{N\_Vector}.
  
  The assignment \id{v\_cont} $=$ \id{NV\_CONTENT\_OMP(v)} sets           
  \id{v\_cont} to be a pointer to the OpenMP \id{N\_Vector} content  
  structure.                                             
                                                            
  Implementation: 
  
  \verb|#define NV_CONTENT_OMP(v) ( (N_VectorContent_OpenMP)(v->content) )|
  
\item \ID{NV\_OWN\_DATA\_OMP}, \ID{NV\_DATA\_OMP}, \ID{NV\_LENGTH\_OMP}, \ID{NV\_NUM_THREADS\_OMP}


  These macros give individual access to the parts of    
  the content of a OpenMP \id{N\_Vector}.                        
                                                               
  The assignment \id{v\_data = NV\_DATA\_OMP(v)} sets \id{v\_data} to be     
  a pointer to the first component of the data for the \id{N\_Vector} \id{v}. 
  The assignment \id{NV\_DATA\_OMP(v) = v\_data} sets the component array of \id{v} to     
  be \id{v\_data} by storing the pointer \id{v\_data}.                   
  
  The assignment \id{v\_len = NV\_LENGTH\_OMP(v)} sets \id{v\_len} to be     
  the length of \id{v}. On the other hand, the call \id{NV\_LENGTH\_OMP(v) = len\_v} 
  sets the length of \id{v} to be \id{len\_v}.
                                                            
  The assignment \id{v\_num\_threads = NV\_NUM\_THREADS\_OMP(v)} sets \id{v\_num\_threads} to be     
  the number of threads from \id{v}. On the other hand, the call \id{NV\_NUM\_THREADS\_OMP(v) = num\_threads\_v} 
  sets the number of threads for \id{v} to be \id{num\_threads\_v}.
                                                            
  Implementation: 
  
  \verb|#define NV_OWN_DATA_OMP(v) ( NV_CONTENT_OMP(v)->own_data )|

  \verb|#define NV_DATA_OMP(v) ( NV_CONTENT_OMP(v)->data )|
  
  \verb|#define NV_LENGTH_OMP(v) ( NV_CONTENT_OMP(v)->length )|

  \verb|#define NV_NUM_THREADS_OMP(v) ( NV_CONTENT_OMP(v)->num_threads )|

\item \ID{NV\_Ith\_OMP}                                               
                                                            
  This macro gives access to the individual components of the data
  array of an \id{N\_Vector}.

  The assignment \id{r = NV\_Ith\_OMP(v,i)} sets \id{r} to be the value of 
  the \id{i}-th component of \id{v}. The assignment \id{NV\_Ith\_OMP(v,i) = r}   
  sets the value of the \id{i}-th component of \id{v} to be \id{r}.        
  
  Here $i$ ranges from $0$ to $n-1$ for a vector of length $n$.

  Implementation:

  \verb|#define NV_Ith_OMP(v,i) ( NV_DATA_OMP(v)[i] )|

\end{itemize}
%%
%%----------------------------------------------
%%
The {\nvecopenmp} module defines OpenMP implementations of all vector operations listed 
in Table \ref{t:nvecops}. Their names are obtained from those in Table \ref{t:nvecops} by
appending the suffix \id{\_OpenMP}. The module {\nvecopenmp} provides the following additional
user-callable routines:
%%
\begin{itemize}

%%--------------------------------------

\item \ID{N\_VNew\_OpenMP}

  This function creates and allocates memory for a OpenMP \id{N\_Vector}.
  Arguments are the vector length and number of threads.

  \verb|N_Vector N_VNew_OpenMP(long int vec_length, int num_threads);|

%%--------------------------------------

\item \ID{N\_VNewEmpty\_OpenMP}

  This function creates a new OpenMP \id{N\_Vector} with an empty (\id{NULL}) data array.

  

  \verb|N_Vector N_VNewEmpty_OpenMP(long int vec_length, int num_threads);|

%%--------------------------------------

\item \ID{N\_VMake\_OpenMP}

 This function creates and allocates memory for a OpenMP vector
 with user-provided data array.

 

 \verb|N_Vector N_VMake_OpenMP(long int vec_length, realtype *v_data, int num_threads);|

%%--------------------------------------

\item \ID{N\_VCloneVectorArray\_OpenMP}

 This function creates (by cloning) an array of \id{count} OpenMP vectors.

 

 \verb|N_Vector *N_VCloneVectorArray_OpenMP(int count, N_Vector w);|

%%--------------------------------------

\item \ID{N\_VCloneEmptyVectorArray\_OpenMP}

 This function creates (by cloning) an array of \id{count} OpenMP vectors, each with an
 empty (\id{NULL}) data array.

 

 \verb|N_Vector *N_VCloneEmptyVectorArray_OpenMP(int count, N_Vector w);|

%%--------------------------------------

\item \ID{N\_VDestroyVectorArray\_OpenMP}

 This function frees memory allocated for the array of \id{count} variables of type
 \id{N\_Vector} created with \id{N\_VCloneVectorArray\_OpenMP} or with
 \id{N\_VCloneEmptyVectorArray\_OpenMP}.

 

 \verb|void N_VDestroyVectorArray_OpenMP(N_Vector *vs, int count);|

%%--------------------------------------

\item \ID{N\_VPrint\_OpenMP}

 This function prints the content of a OpenMP vector to \id{stdout}.

 
 
 \verb|void N_VPrint_OpenMP(N_Vector v);|

\end{itemize}
%%
%%------------------------------------
%%
\paragraph{\bf Notes}                                                      
           
\begin{itemize}
                                        
\item
  When looping over the components of an \id{N\_Vector} \id{v}, it is     
  more efficient to first obtain the component array via       
  \id{v\_data = NV\_DATA\_OMP(v)} and then access \id{v\_data[i]} within the     
  loop than it is to use \id{NV\_Ith\_OMP(v,i)} within the loop.        

\item
  {\warn}\id{N\_VNewEmpty\_OpenMP}, \id{N\_VMake\_OpenMP}, 
  and \id{N\_VCloneEmptyVectorArray\_OpenMP} set the field 
  {\em own\_data} $=$ \id{FALSE}. 
  \id{N\_VDestroy\_OpenMP} and \id{N\_VDestroyVectorArray\_OpenMP}
  will not attempt to free the pointer {\em data} for any \id{N\_Vector} with
  {\em own\_data} set to \id{FALSE}. In such a case, it is the user's responsibility to
  deallocate the {\em data} pointer.
                                     
\item
  {\warn}To maximize efficiency, vector operations in the {\nvecopenmp} implementation
  that have more than one \id{N\_Vector} argument do not check for
  consistent internal representation of these vectors. It is the user's 
  responsibility to ensure that such routines are called with \id{N\_Vector}
  arguments that were all created with the same internal representations.

\end{itemize}


%---------------------------------------------------------------------------
\section{The NVECTOR\_PTHREADS implementation}\label{ss:nvec_pthreads}
%% This is a shared SUNDIALS TEX file with a description of the
%% Pthreads nvector implementation
%%

In situations where a user has a multi-core processing unit capable of
running multiple parallel threads with shared memory, {\sundials} provides
an implementation of {\nvector} using OpenMP, called {\nvecopenmp}, and
an implementation using Pthreads, called {\nvecpthreads}.  
Testing has shown that vectors should be of length at least $100,000$ 
before the overhead associated with creating and using the threads is
made up by the parallelism in the vector calculations. 

The Pthreads {\nvector} implementation provided with {\sundials}, denoted
{\nvecpthreads}, defines the {\em content} field of \id{N\_Vector} to be a structure 
containing the length of the vector, a pointer to the beginning of a contiguous 
data array, a boolean flag {\em own\_data} which specifies the ownership 
of {\em data}, and the number of threads.  
Operations on the vector are threaded using POSIX threads 
(Pthreads).
%%
\begin{verbatim} 
struct _N_VectorContent_Pthreads {
  long int length;
  booleantype own_data;
  realtype *data;
  int num_threads;
};
\end{verbatim}
%%
%%--------------------------------------------

The header file to be included when using this module is \id{nvector\_pthreads.h}.

The following six macros are provided to access the content of an {\nvecpthreads}
vector. The suffix \id{\_PT} in the names denotes the Pthreads version.
%%
\begin{itemize}

\item \ID{NV\_CONTENT\_PT}                             
    
  This routine gives access to the contents of the Pthreads
  vector \id{N\_Vector}.
  
  The assignment \id{v\_cont} $=$ \id{NV\_CONTENT\_PT(v)} sets           
  \id{v\_cont} to be a pointer to the Pthreads \id{N\_Vector} content  
  structure.                                             
                                                            
  Implementation: 
  
  \verb|#define NV_CONTENT_PT(v) ( (N_VectorContent_Pthreads)(v->content) )|
  
\item \ID{NV\_OWN\_DATA\_PT}, \ID{NV\_DATA\_PT}, \ID{NV\_LENGTH\_PT}, \ID{NV\_NUM\_THREADS\_PT}


  These macros give individual access to the parts of    
  the content of a Pthreads \id{N\_Vector}.                        
                                                               
  The assignment \id{v\_data = NV\_DATA\_PT(v)} sets \id{v\_data} to be     
  a pointer to the first component of the data for the \id{N\_Vector} \id{v}. 
  The assignment \id{NV\_DATA\_PT(v) = v\_data} sets the component array of \id{v} to     
  be \id{v\_data} by storing the pointer \id{v\_data}.                   
  
  The assignment \id{v\_len = NV\_LENGTH\_PT(v)} sets \id{v\_len} to be     
  the length of \id{v}. On the other hand, the call \id{NV\_LENGTH\_PT(v) = len\_v} 
  sets the length of \id{v} to be \id{len\_v}.
                                                            
  The assignment \id{v\_num\_threads = NV\_NUM\_THREADS\_PT(v)} sets \id{v\_num\_threads} to be     
  the number of threads from \id{v}. On the other hand, the call \id{NV\_NUM\_THREADS\_PT(v) = num\_threads\_v} 
  sets the number of threads for \id{v} to be \id{num\_threads\_v}.
                                                            
  Implementation: 
  
  \verb|#define NV_OWN_DATA_PT(v) ( NV_CONTENT_PT(v)->own_data )|

  \verb|#define NV_DATA_PT(v) ( NV_CONTENT_PT(v)->data )|
  
  \verb|#define NV_LENGTH_PT(v) ( NV_CONTENT_PT(v)->length )|

  \verb|#define NV_NUM_THREADS_PT(v) ( NV_CONTENT_PT(v)->num_threads )|

\item \ID{NV\_Ith\_PT}                                               
                                                            
  This macro gives access to the individual components of the data
  array of an \id{N\_Vector}.

  The assignment \id{r = NV\_Ith\_PT(v,i)} sets \id{r} to be the value of 
  the \id{i}-th component of \id{v}. The assignment \id{NV\_Ith\_PT(v,i) = r}   
  sets the value of the \id{i}-th component of \id{v} to be \id{r}.        
  
  Here $i$ ranges from $0$ to $n-1$ for a vector of length $n$.

  Implementation:

  \verb|#define NV_Ith_PT(v,i) ( NV_DATA_PT(v)[i] )|

\end{itemize}
%%
%%----------------------------------------------
%%
The {\nvecpthreads} module defines Pthreads implementations of all vector operations listed 
in Table \ref{t:nvecops}. Their names are obtained from those in Table \ref{t:nvecops}
by appending the suffix \id{\_Pthreads} (e.g. \id{N\_VDestroy\_Pthreads}).
The module {\nvecpthreads} provides the following additional user-callable routines:
%%
\begin{itemize}

%%--------------------------------------

\item \ID{N\_VNew\_Pthreads}

  This function creates and allocates memory for a Pthreads \id{N\_Vector}.
  Arguments are the vector length and number of threads.

  

  \verb|N_Vector N_VNew_Pthreads(long int vec_length, int num_threads);|

%%--------------------------------------

\item \ID{N\_VNewEmpty\_Pthreads}

  This function creates a new Pthreads \id{N\_Vector} with an empty (\id{NULL}) data array.

  

  \verb|N_Vector N_VNewEmpty_Pthreads(long int vec_length, int num_threads);|

%%--------------------------------------

\item \ID{N\_VMake\_Pthreads}

 This function creates and allocates memory for a Pthreads vector
 with user-provided data array.

 (This function does {\em not} allocate memory for \id{v\_data} itself.)

 \verb|N_Vector N_VMake_Pthreads(long int vec_length, realtype *v_data, int num_threads);|

%%--------------------------------------

\item \ID{N\_VCloneVectorArray\_Pthreads}

 This function creates (by cloning) an array of \id{count} Pthreads vectors.

 

 \verb|N_Vector *N_VCloneVectorArray_Pthreads(int count, N_Vector w);|

%%--------------------------------------

\item \ID{N\_VCloneVectorArrayEmpty\_Pthreads}

 This function creates (by cloning) an array of \id{count} Pthreads vectors, each with an
 empty (\id{NULL}) data array.

 

 \verb|N_Vector *N_VCloneVectorArrayEmpty_Pthreads(int count, N_Vector w);|

%%--------------------------------------

\item \ID{N\_VDestroyVectorArray\_Pthreads}

 This function frees memory allocated for the array of \id{count} variables of type
 \id{N\_Vector} created with \id{N\_VCloneVectorArray\_Pthreads} or with
 \id{N\_VCloneVectorArrayEmpty\_Pthreads}.

 

 \verb|void N_VDestroyVectorArray_Pthreads(N_Vector *vs, int count);|

%%--------------------------------------

\item \ID{N\_VGetLength\_Pthreads}

 This function returns the number of vector elements.

 
 
 \verb|long int N_VGetLength_Pthreads(N_Vector v);|

%%--------------------------------------

\item \ID{N\_VPrint\_Pthreads}

 This function prints the content of a Pthreads vector to \id{stdout}.

 
 
 \verb|void N_VPrint_Pthreads(N_Vector v);|

\end{itemize}
%%
%%------------------------------------
%%
\paragraph{\bf Notes}                                                      
           
\begin{itemize}
                                        
\item
  When looping over the components of an \id{N\_Vector} \id{v}, it is     
  more efficient to first obtain the component array via       
  \id{v\_data = NV\_DATA\_PT(v)} and then access \id{v\_data[i]} within the     
  loop than it is to use \id{NV\_Ith\_PT(v,i)} within the loop.        

\item
  {\warn}\id{N\_VNewEmpty\_Pthreads}, \id{N\_VMake\_Pthreads}, 
  and \id{N\_VCloneVectorArrayEmpty\_Pthreads} set the field 
  {\em own\_data} $=$ \id{FALSE}. 
  \id{N\_VDestroy\_Pthreads} and \id{N\_VDestroyVectorArray\_Pthreads}
  will not attempt to free the pointer {\em data} for any \id{N\_Vector} with
  {\em own\_data} set to \id{FALSE}. In such a case, it is the user's responsibility to
  deallocate the {\em data} pointer.
                                     
\item
  {\warn}To maximize efficiency, vector operations in the {\nvecpthreads} implementation
  that have more than one \id{N\_Vector} argument do not check for
  consistent internal representation of these vectors. It is the user's 
  responsibility to ensure that such routines are called with \id{N\_Vector}
  arguments that were all created with the same internal representations.

\end{itemize}

For solvers that include a Fortran interface module, the {\nvecpthreads}
module also includes a Fortran-callable function
\id{FNVINITPTS(code, NEQ, NUMTHREADS, IER)}, to initialize this
module.  Here \id{code} is an input solver id
(1 for {\cvode}, 2 for {\ida}, 3 for {\kinsol}, 4 for {\arkode}); NEQ is
the problem size (declared so as to match C type \id{long int});
NUMTHREADS is the number of threads; and IER is an error return flag
equal 0 for success and -1 for failure.


%---------------------------------------------------------------------------
\section{The NVECTOR\_PARHYP implementation}\label{ss:nvec_parhyp}
% This is a shared SUNDIALS TEX file with description of
% the MPI parallel hypre nvector implementation
%
The {\nvecph} implementation of the {\nvector} module provided with
{\sundials} is a wrapper around {\hypre}'s ParVector class. 
Most of the vector kernels simply call {\hypre} vector operations. 
The implementation defines the {\em content} field of \id{N\_Vector} to 
be a structure containing the global and local lengths of the vector, a 
pointer to an object of type \id{hypre\_ParVector}, an {\mpi} communicator, 
and a boolean flag {\em own\_parvector} indicating ownership of the
{\hypre} parallel vector object {\em x}.
%%
%%
\begin{verbatim}
struct _N_VectorContent_ParHyp {
  long int local_length;
  long int global_length;
  booleantype own_parvector;
  MPI_Comm comm;
  hypre_ParVector *x;
};
\end{verbatim}
%%
%%--------------------------------------------

\noindent
The header file to be included when using this module is \id{nvector\_parhyp.h}.
Unlike native {\sundials} vector types, {\nvecph} does not provide macros 
to access its member variables.

%%
%%--------------------------------------------
%%
The {\nvecph} module defines parhyp implementations of all vector operations listed 
in Table \ref{t:nvecops}, except for \verb|N_VSetArrayPointer|, because setting raw 
data pointers is handled by low-level {\hypre} functions. Implementation of 
\verb|N_VGetArrayPointer| is provided, but its use is strongly discouraged (we 
consider removing it, as well). When access to raw vector data is needed, it is 
recommended to extract {\hypre} vector first, and then use {\hypre} 
methods to access the raw data. 

The names of parhyp methods are obtained from those in Table \ref{t:nvecops}
by appending the suffix \id{\_ParHyp} (e.g. \id{N\_VDestroy\_ParHyp}).
The module {\nvecph} provides the following additional user-callable routines:
%%
%%
\begin{itemize}

%%--------------------------------------

\item \ID{N\_VNewEmpty\_ParHyp}
 
  This function creates a new parhyp \id{N\_Vector} with pointer to {\hypre} 
  vector set to \id{NULL}.
 
  

\begin{verbatim}
N_Vector N_VNewEmpty_ParHyp(MPI_Comm comm, 
                            long int local_length, 
                            long int global_length);
\end{verbatim}

  
%%--------------------------------------

\item \ID{N\_VMake\_ParHyp}
  
  This function creates \verb|N_Vector| wrapper around an existing
{\hypre} parallel vector.
 
(This function does {\em not} allocate memory for \id{x} itself.)  

\begin{verbatim}
N_Vector N_VMake_ParHyp(hypre_ParVector *x);
\end{verbatim}

%%--------------------------------------

\item \ID{N\_VCloneVectorArray\_ParHyp}
 
  This function creates (by cloning) an array of \id{count} parallel vectors.
 
\begin{verbatim}
N_Vector *N_VCloneVectorArray_ParHyp(int count, N_Vector w);
\end{verbatim}

%%--------------------------------------

\item \ID{N\_VCloneVectorArrayEmpty\_ParHyp}
 
  This function creates (by cloning) an array of \id{count} parallel vectors,
  each with an empty (\id{NULL}) data array.
 
\begin{verbatim}
N_Vector *N_VCloneVectorArrayEmpty_ParHyp(int count, N_Vector w);
\end{verbatim}

%%--------------------------------------

\item \ID{N\_VDestroyVectorArray\_ParHyp}
 
 This function frees memory allocated for the array of \id{count}  variables of
 type \id{N\_Vector} created with \id{N\_VCloneVectorArray\_ParHyp} or with
 \id{N\_VCloneVectorArrayEmpty\_ParHyp}.
 

 \verb|void N_VDestroyVectorArray_ParHyp(N_Vector *vs, int count);|


%%--------------------------------------

\item \ID{N\_VGetVector\_ParHyp}
  
  This function returns pointer to the underlying {\hypre} vector.
 
    
  \verb|hypre_ParVector *N_VPrint_ParHyp(N_Vector v);|


%%--------------------------------------

\item \ID{N\_VPrint\_ParHyp}
  
  This function prints the content of a parhyp vector to stdout.
 
    
  \verb|void N_VPrint_ParHyp(N_Vector v);|


\end{itemize}
%%
%%------------------------------------
%%
\paragraph{\bf Notes} 
           
\begin{itemize}
                                        
\item
  When there is a need to access components of an \id{N\_Vector} \id{v}, 
  it is recommeded to extract {\hypre} vector via       
  \id{x\_vec = N\_VGetVector(v)} and then access components using 
  {\hypre} functions.        
                                                               
\item
  {\warn}\id{N\_VNewEmpty\_ParHyp}, \id{N\_VMake\_ParHyp}, 
  and \id{N\_VCloneVectorArrayEmpty\_ParHyp} set the field 
  {\em own\_parvector} $=$ \id{FALSE}. 
  \id{N\_VDestroy\_ParHyp} and \id{N\_VDestroyVectorArray\_ParHyp}
  will not attempt to delete underlying {\hypre} vector for any \id{N\_Vector} 
  with {\em own\_parvector} set to \id{FALSE}. In such a case, it is the 
  user's responsibility to delete the underlying vector.

\item
  {\warn}To maximize efficiency, vector operations in the {\nvecph} implementation
  that have more than one \id{N\_Vector} argument do not check for
  consistent internal representation of these vectors. It is the user's 
  responsibility to ensure that such routines are called with \id{N\_Vector}
  arguments that were all created with the same internal representations.

\end{itemize}

% For solvers that include a Fortran interface module, the {\nvecph} module
% also includes a Fortran-callable function
% \id{FNVINITPH(COMM, code, NLOCAL, NGLOBAL, IER)},
% to initialize this {\nvecph} module.  Here \id{COMM} is the MPI communicator,
% \id{code} is an input solver id (1 for {\cvode}, 2 for {\ida}, 3 for {\kinsol},
% 4 for {\arkode}); \id{NLOCAL} and \id{NGLOBAL} are the local and global
% vector sizes, respectively (declared so as to match C type \id{long int});
% and IER is an error return flag equal 0 for success and -1 for failure.
% 
% {\warn}Note: If the header file \id{sundials\_config.h} defines
% \id{SUNDIALS\_MPI\_COMM\_F2C} to be $1$ (meaning the {\mpi}
% implementation used to build {\sundials} includes the
% \id{MPI\_Comm\_f2c} function), then \id{COMM} can be any valid
% {\mpi} communicator. Otherwise, \id{MPI\_COMM\_WORLD} will be used, so
% just pass an integer value as a placeholder.


%---------------------------------------------------------------------------

\section{NVECTOR Examples}\label{ss:nvec_examples}

There are \id{NVector} examples that may be installed for each
implementation: serial, parallel, OpenMP, and Pthreads.  Each
implementation makes use of the functions in \id{test\_nvector.c}.
These example functions show simple usage of the \id{NVector} family
of functions. The input to the examples are the vector length, number
of threads (if threaded implementation), and a print timing flag.

\noindent The following is a list of the example functions in \id{test\_nvector.c}:
\begin{itemize}
\item \id{Test\_N\_VClone}: Creates clone of vector and checks validity of clone.  
\item \id{Test\_N\_VCloneEmpty}: Creates clone of empty vector and checks validity of clone.  
\item \id{Test\_N\_VCloneVectorArray}: Creates clone of vector array and checks validity of cloned array.  
\item \id{Test\_N\_VCloneVectorArray}: Creates clone of empty vector array and checks validity of cloned array.  
\item \id{Test\_N\_VGetArrayPointer}: Get array pointer. 
\item \id{Test\_N\_VSetArrayPointer}: Allocate new vector, set pointer to new vector array, and check values. 
\item \id{Test\_N\_VLinearSum} Case 1a: Test y =  x + y 
\item \id{Test\_N\_VLinearSum} Case 1b: Test y = -x + y 
\item \id{Test\_N\_VLinearSum} Case 1c: Test y = ax + y
\item \id{Test\_N\_VLinearSum} Case 2a: Test x =  x + y
\item \id{Test\_N\_VLinearSum} Case 2b: Test x =  x - y
\item \id{Test\_N\_VLinearSum} Case 2c: Test x =  x + by
\item \id{Test\_N\_VLinearSum} Case 3:  Test z =  x + y
\item \id{Test\_N\_VLinearSum} Case 4a: Test z =  x - y
\item \id{Test\_N\_VLinearSum} Case 4b: Test z = -x + y
\item \id{Test\_N\_VLinearSum} Case 5a: Test z =  x + by
\item \id{Test\_N\_VLinearSum} Case 5b: Test z = ax + y
\item \id{Test\_N\_VLinearSum} Case 6a: Test z = -x + by
\item \id{Test\_N\_VLinearSum} Case 6b: Test z = ax - y
\item \id{Test\_N\_VLinearSum} Case 7:  Test z = a(x + y)
\item \id{Test\_N\_VLinearSum} Case 8:  Test z = a(x - y)
\item \id{Test\_N\_VLinearSum} Case 9:  Test z = ax + by
\item \id{Test\_N\_VConst}: Fill vector with constant and check result.
\item \id{Test\_N\_VProd}: Test vector multiply: z = x * y
\item \id{Test\_N\_VDiv}: Test vector division: z = x / y
\item \id{Test\_N\_VScale}: Case 1: scale: x = cx
\item \id{Test\_N\_VScale}: Case 2: copy: z = x
\item \id{Test\_N\_VScale}: Case 3: negate: z = -x
\item \id{Test\_N\_VScale}: Case 4: combination: z = cx
\item \id{Test\_N\_VAbs}: Create absolute value of vector. 
\item \id{Test\_N\_VAddConst}: add constant vector: z = c + x
\item \id{Test\_N\_VDotProd}: Calculate dot product of two vectors.
\item \id{Test\_N\_VMaxNorm}: Create vector with known values, find and validate max norm.
\item \id{Test\_N\_VWrmsNorm}: Create vector of known values, find and validate weighted root mean square.
\item \id{Test\_N\_VWrmsNormMask}: Case 1: Create vector of known values,
      find and validate weighted root mean square using all elements.
\item \id{Test\_N\_VWrmsNormMask}: Case 2: Create vector of known values,
      find and validate weighted root mean square using no elements.
\item \id{Test\_N\_VMin}: Create vector, find and validate the min.
\item \id{Test\_N\_VWL2Norm}: Create vector, find and validate the weighted Euclidean L2 norm.
\item \id{Test\_N\_VL1Norm}: Create vector, find and validate the L1 norm.
\item \id{Test\_N\_VCompare}: Compare vector with constant returning and validating comparison vector.
\item \id{Test\_N\_VInvTest}: Test z[i] = 1 / x[i]
\item \id{Test\_N\_VConstrMask}: Test mask of vector x with vector c.
\item \id{Test\_N\_VMinQuotient}: Fill two vectors with known values. Calculate and validate minimum quotient.
\end{itemize}



%---------------------------------------------------------------------------
\section{NVECTOR functions used by KINSOL}

In Table \ref{t:nvecuse} below, we list the vector functions in the 
{\nvector} module used within the {\kinsol} package.
The table also shows, for each function, which of the code modules uses
the function. The {\kinsol} column shows function usage within the main
solver module, while the remaining five columns show function
usage within each of the {\kinsol} linear solver interfaces, 
the {\kinbbdpre} preconditioner module, and the {\fkinsol} module.
Here {\kinls} stands for the generic linear solver interface in {\kinsol}.

At this point, we should emphasize that the {\kinsol} user does not need to know 
anything about the usage of vector functions by the {\kinsol} code modules in order 
to use {\kinsol}. The information is presented as an implementation detail for the 
interested reader.

\begin{table}[htb]
\centering
\caption{List of vector functions usage by {\kinsol} code modules}\label{t:nvecuse}
\medskip
\begin{tabular}{|r|c|c|c|c|c|} \hline
                                            &
\begin{sideways}{\kinsol}    \end{sideways} &
\begin{sideways}{\kinls}     \end{sideways} &
\begin{sideways}{\kinbbdpre} \end{sideways} &
\begin{sideways}{\fkinsol}   \end{sideways} \\ \hline\hline
%                           KINSOL   LS   BPRE  FKINSOL
\id{N\_VGetVectorID}        &     &     &     &     \\ \hline
\id{N\_VClone}              & \cm &     & \cm &     \\ \hline
\id{N\_VCloneEmpty}         &     &     &     & \cm \\ \hline
\id{N\_VDestroy}            & \cm &     & \cm & \cm \\ \hline
\id{N\_VSpace}              & \cm &  2  &     &     \\ \hline
\id{N\_VGetArrayPointer}    &     &  1  & \cm & \cm \\ \hline
\id{N\_VSetArrayPointer}    &     &  1  &     & \cm \\ \hline
\id{N\_VLinearSum}          & \cm & \cm &     &     \\ \hline
\id{N\_VConst}              &     & \cm &     &     \\ \hline
\id{N\_VProd}               & \cm & \cm &     &     \\ \hline
\id{N\_VDiv}                & \cm &     &     &     \\ \hline
\id{N\_VScale}              & \cm & \cm & \cm &     \\ \hline
\id{N\_VAbs}                & \cm &     &     &     \\ \hline
\id{N\_VInv}                & \cm &     &     &     \\ \hline
\id{N\_VDotProd}            & \cm & \cm &     &     \\ \hline
\id{N\_VMaxNorm}            & \cm &     &     &     \\ \hline
\id{N\_VMin}                & \cm &     &     &     \\ \hline
\id{N\_VWL2Norm}            & \cm & \cm &     &     \\ \hline
\id{N\_VL1Norm}             &     &  3  &     &     \\ \hline
\id{N\_VConstrMask}         & \cm &     &     &     \\ \hline
\id{N\_VMinQuotient}        & \cm &     &     &     \\ \hline
\hline
\id{N\_VLinearCombination}  & \cm & \cm &     &     \\ \hline 
\id{N\_VDotProdMulti}       & \cm &     &     &     \\ \hline 
\end{tabular}
\end{table}

Special cases (numbers match markings in table):
\begin{enumerate}
\item These routines are only required if an internal
  difference-quotient routine for constructing dense or band
  Jacobian matrices is used.
\item This routine is optional, and is only used in estimating
  space requirements for {\ida} modules for user feedback.
\item These routines are only required if the internal
  difference-quotient routine for approximating the Jacobian-vector
  product is used.
\end{enumerate}

Each {\sunlinsol} object may require additional {\nvector} routines
not listed in the table above.  Please see the the relevant
descriptions of these modules in Sections
\ref{ss:sunlinsol_dense}-\ref{ss:sunlinsol_pcg} for additional detail
on their {\nvector} requirements.

The vector functions listed in Table \ref{t:nvecops} that are {\em not} used by
{\kinsol} are \id{N\_VAddConst}, \id{N\_VWrmsNorm}, \id{N\_VWrmsNormMask},
\id{N\_VCompare}, \id{N\_VInvTest}, \id{N\_VGetCommunicator}, and \id{N\_VGetLength}.
Therefore a user-supplied {\nvector} module for {\kinsol} could omit these
functions.

The optional function \id{N\_VLinearCombination} is only used when
Anderson acceleration is enabled or the {\spbcg}, {\sptfqmr},
{\spgmr}, or {\spfgmr} linear solvers are used. \id{N\_VDotProd} is
only used when Anderson acceleration is enabled or Classical
Gram-Schmidt is used with {\spgmr} or {\spfgmr}. The remaining
operations from Tables \ref{t:nvecfusedops} and \ref{t:nvecarrayops}
are unused and a user-supplied {\nvector} module for {\kinsol} could
omit these operations.

%---------------------------------------------------------------------------
% nvector module sections
%---------------------------------------------------------------------------

% This is a shared SUNDIALS TEX file with description of
% the serial nvector implementation
%
The {\nvecs} implementation of the {\nvector} module
defines the {\em content} field of \id{NV\_Spec} to be a structure 
containing the length of the vector:
\begin{verbatim}
struct _NV_SpecContent_Serial {
  long int length;
};
\end{verbatim}
The {\em tag} field of \id{NV\_Spec} is set to \id{serial}.
The {\em content} field of \id{N\_Vector} is defined to be a structure containing
the length of the vector and a pointer to the beginning of a contiguous data array:
\begin{verbatim} 
struct _N_VectorContent_Serial {
  long int length;
  realtype   *data;
};
\end{verbatim}

The {\nvecs} implementation provides user-callable routines \ID{NV\_SpecInit\_Serial} 
to create a structure of type \id{NV\_Spec} whose {\em content} field is 
of type \id{struct \_NV\_SpecSerialContent}, and \ID{NV\_SpecFree\_Serial} 
to deallocate the space used by such a structure. 

\noindent The form of the call to \id{NV\_SpecInit\_Serial} is
\begin{verbatim}
nvspec = NV_SpecInit_Serial(vec_length);
\end{verbatim}
If successful, \id{NV\_SpecInit\_Serial} returns a pointer of type
\id{NV\_Spec}. \index{nvSpec@{\tt nvSpec}}This pointer should in turn be passed in any user
calls to \id{N\_VNew} to create a new \id{N\_Vector} of this type.
A vector specification object \id{nvspec} returned by \id{NV\_SpecInit\_Serial}
can be freed by calling:
\begin{verbatim}
NV_SpecFree_Serial(nvspec);
\end{verbatim}
In addition to these two routines, {\nvecs} defines serial implementations of all 
vector kernels listed in Table \ref{t:nvecops}, as well as the following macros
that can be used to access the contents of \id{NV\_Spec} and \id{N\_Vector}
or to create and destroy \id{N\_Vector}'s with component array data allocated
by the user. The suffix \id{\_S} in the names denotes serial version.
\begin{itemize}

\item \ID{NS\_CONTENT\_S}, \ID{NV\_CONTENT\_S}

      These macros give access to the contents of the serial 
      vector specification and \id{N\_Vector}, respectively.           
                                                               
      The assignment \id{ns\_cont = NS\_CONTENT\_S(nvspec)} sets       
      \id{ns\_cont} to be a pointer to the serial vector
      specification content structure (of type \id{struct \_NV\_SpecSerialContent}).

      The assignment \id{v\_cont = NV\_CONTENT\_S(v)} sets       
      \id{v\_cont} to be a pointer to the serial \id{N\_Vector} content    
      structure of type \id{struct \_N\_VectorSerialContent}.

\item \ID{NV\_DATA\_S}, \ID{NV\_LENGTH\_S}

      These macros give individual access to the parts of    
      the content of a serial \id{N\_Vector}.                        
                                                               
      The assignment \id{v\_data = NV\_DATA\_S(v)} sets \id{v\_data} to be     
      a pointer to the first component of the data for the \id{N\_Vector} \id{v}. 
      The assignment \id{NV\_DATA\_S(v) = v\_data} sets the component array of \id{v} to     
      be \id{v\_data} by storing the pointer \id{v\_data}.                   
                                                               
      The assignment \id{v\_len = NV\_LENGTH\_S(v)} sets \id{v\_len} to be     
      the length of \id{v}. On the other hand, the call \id{NV\_LENGTH\_S(v) = len\_v} 
      sets the length of \id{v} to be \id{len\_v}.

\item \ID{NV\_Ith\_S}

      This macro gives access to the individual components of the data
      array of an \id{N\_Vector}.

      The assignment \id{r = NV\_Ith\_S(v,i)} sets \id{r} to be the value of 
      the \id{i}-th component of \id{v}. The assignment \id{NV\_Ith\_S(v,i) = r}   
      sets the value of the \id{i}-th component of \id{v} to be \id{r}.        

      Here $i$ ranges from $0$ to $n-1$ for a vector of length $n$.

\item \ID{NV\_MAKE\_S}, \ID{NV\_DISPOSE\_S}

      These companion macros are used to create and          
      destroy an \id{N\_Vector} with a component array \id{vdata} 
      allocated by the user.                                   
                                                               
      The call \id{NV\_MAKE\_S(v,v\_data,nvspec)} makes \id{v} an        
      \id{N\_Vector} with component array pointer \id{v\_data}. The length of the  
      array is taken from \id{nvspec}.                             
      \id{NV\_MAKE\_S} stores the pointer \id{v\_data} so that changes      
      made by the user to the elements of \id{v\_data} are           
      simultaneously reflected in \id{v}. There is no copying of    
      elements.                                                
                                                               
      The call \id{NV\_DISPOSE\_S(v)} frees all memory associated     
      with \id{v} except for its component array. This memory was   
      allocated by the user and, therefore, should be          
      deallocated by the user.   

\item \ID{NVS\_MAKE\_S}, \ID{NVS\_DISPOSE\_S}
                             
      These companion macros are used to create and destroy  
      an array of \id{N\_Vector}'s with component array \id{vs\_data} 
      (of type \id{realtype **}) allocated by the user, for use in
      sensitivity calculations.

                                                               
      The call \id{NVS\_MAKE\_S(vs,vs\_data,ns,nvspec)} makes   
      \id{vs} an array of \id{ns} \id{N\_Vector}'s, with \id{vs[i]} 
      having component array pointer \id{vs\_data[i]} and length taken from \id{nvspec}.    
      \id{NVS\_MAKE\_S} stores the pointers \id{vs\_data[i]} so that        
      changes made by the user to the elements of \id{vs\_data} are  
      simultaneously reflected in \id{vs}. There is no copying of   
      elements.                                                
                                                               
      The call \id{NVS\_DISPOSE\_S(vs,ns)} frees all memory associated   
      with \id{vs} except for its components' component array.      
      This memory was allocated by the user and, therefore,    
      should be deallocated by the user.                       

\end{itemize}

\noindent {\bf Notes}                                                      
           
\begin{itemize}
                                        
\item
  Users who use the make/dispose macros must                 
  \id{\#include<stdlib.h>} since these macros expand to calls to     
  \id{malloc} and \id{free}.                                             
  
\item
  When looping over the components of an \id{N\_Vector} \id{v}, it is     
  more efficient to first obtain the component array via       
  \id{v\_data = NV\_DATA\_S(v)} and then access \id{v\_data[i]} within the     
  loop than it is to use \id{NV\_Ith\_S(v,i)} within the loop.        
                                     
\item                          
  \id{NV\_MAKE\_S} and \id{NV\_DISPOSE\_S} are similar to the \id{N\_VNew} and  
  \id{N\_VFree} implemented by {\nvecs}, while \id{NVS\_MAKE\_S} and 
  \id{NVS\_DISPOSE\_S}  are similar to  \id{N\_VNew\_S} and \id{N\_VFree\_S}. 
  The difference is one of responsibility for component memory     
  allocation and deallocation. \id{N\_VNew} allocates memory  
  for the \id{N\_Vector} components and \id{N\_VFree} frees the     
  component memory allocated by \id{N\_VNew}. For \id{NV\_MAKE\_S}   
  and \id{NV\_DISPOSE\_S}, the component memory is allocated and      
  freed by the user of this package. Similar remarks hold for  
  \id{NVS\_MAKE\_S},  \id{NVS\_DISPOSE\_S} and \id{N\_VNew\_S},              
  \id{N\_VFree\_S}.                                            

\item
  To maximize efficiency, vector kernels in the {\nvecs} implementation
  that have more than one \id{N\_Vector} argument do not check for
  consistent internal representation of these vectors. It is the user's 
  responsibility to ensure that such routines are called with \id{N\_Vector}
  arguments that were all created with the \id{NV\_Spec} structure returned
  by \id{NV\_SpecInit\_Serial}.

\end{itemize}

% This is a shared SUNDIALS TEX file with description of
% the parallel nvector implementation
%
The parallel implementation of the {\nvector} module defines the {\em content} 
field of \id{N\_Vector} to be a structure containing the global and local lengths 
of the vector, a pointer to the beginning of a contiguous local data array, and
an MPI communicator.
%%
\begin{verbatim} 
struct _N_VectorContent_Parallel {
  long int local_length;
  long int global_length;
  realtype *data;
  MPI_Comm comm;
};
\end{verbatim}
%%
%%--------------------------------------------
%%
The following seven macros are provided to access the content of a {\nvecp}
vector. The suffix \id{\_P} in the names denotes parallel version.
\begin{itemize}

\item 
  \ID{NV\_CONTENT\_P}

  This macro gives access to the contents of the parallel
  vector \id{N\_Vector}.
  
  The assignment \id{v\_cont = NV\_CONTENT\_P(v)} sets       
  \id{v\_cont} to be a pointer to the \id{N\_Vector} content    
  structure of type \id{struct \_N\_VectorParallelContent}.
  
  Implementation:
  
  \verb|#define NV_CONTENT_P(v) ( (N_VectorContent_Parallel)(v->content) )|
  
\item 
  \ID{NV\_OWN\_DATA\_P}, \ID{NV\_DATA\_P}, 
  \ID{NV\_LOCLENGTH\_P}, \ID{NV\_GLOBLENGTH\_P}
  
  These macros give individual access to the parts of    
  the content of a parallel \id{N\_Vector}.                        
  
  The assignment \id{v\_data = NV\_DATA\_P(v)} sets \id{v\_data} to be     
  a pointer to the first component of the local data for the \id{N\_Vector} \id{v}. 
  The assignment \id{NV\_DATA\_P(v) = v\_data} sets the component array of 
  \id{v} to be \id{v\_data} by storing the pointer \id{v\_data}.                   
  
  The assignment \id{v\_llen = NV\_LOCLENGTH\_P(v)} sets \id{v\_llen} to be     
  the length of the local part of \id{v}. 
  The call \id{NV\_LENGTH\_P(v) = llen\_v} sets      
  the local length of \id{v} to be \id{llen\_v}.
  
  The assignment \id{v\_glen = NV\_GLOBLENGTH\_P(v)} sets \id{v\_glen} to  
  be the global length of the vector \id{v}.                    
  The call \id{NV\_GLOBLENGTH\_P(v) = glen\_v} sets the global       
  length of \id{v} to be \id{glen\_v}.
  
  Implementation:
  
  \verb|#define NV_OWN_DATA_P(v)   ( NV_CONTENT_P(v)->own_data )|

  \verb|#define NV_DATA_P(v)       ( NV_CONTENT_P(v)->data )|

  \verb|#define NV_LOCLENGTH_P(v)  ( NV_CONTENT_P(v)->local_length )|

  \verb|#define NV_GLOBLENGTH_P(v) ( NV_CONTENT_P(v)->global_length )|
  
\item \ID{NV\_COMM\_P}

  This macro provides access to the MPI communicator used by the {\nvecp}
  vectors.

  Implementation:

  \verb|#define NV_COMM_P(v) ( NV_CONTENT_P(v)->comm )|

\item \ID{NV\_Ith\_P}

  This macro gives access to the individual components of the local data
  array of an \id{N\_Vector}.

  The assignment \id{r = NV\_Ith\_P(v,i)} sets \id{r} to be the value of 
  the \id{i}-th component of the local part of \id{v}. 
  The assignment \id{NV\_Ith\_P(v,i) = r}   
  sets the value of the \id{i}-th component of the local part of \id{v} 
  to be \id{r}.        
  
  Here $i$ ranges from $0$ to $n-1$, where $n$ is the local length.
      
  Implementation:

  \verb|#define NV_Ith_P(v,i) ( NV_DATA_P(v)[i] )|

\end{itemize}
%%
%%--------------------------------------------
%%
The {\nvecp} module defines serial implementations of all vector operations listed 
in Table \ref{t:nvecops} and provides the following user-callable routines:
%%
%%
\begin{itemize}

%%--------------------------------------

\item  \ID{N\_VNew\_Parallel}
  
  This function creates and allocates memory for a parallel vector.
 
  Prototype

\begin{verbatim}
N_Vector N_VNew_Parallel(MPI_Comm comm, 
                         long int local_length, 
                         long int global_length);
\end{verbatim}
  
%%--------------------------------------

\item \ID{N\_VNewEmpty\_Parallel}
 
  This function creates a new parallel \id{N\_Vector} with an empty (\id{NULL}) data array.
 
  Prototype

\begin{verbatim}
N_Vector N_VNewEmpty_Parallel(MPI_Comm comm, 
                              long int local_length, 
                              long int global_length);
\end{verbatim}

  
%%--------------------------------------

\item \ID{N\_VCloneEmpty\_Parallel}
 
  This function creates a new parallel \id{N\_Vector} with an empty (\id{NULL}) data array
  by using an existing \id{N\_Vector} as a template.
 
  Prototype

  \verb|N_Vector N_VCloneEmpty_Parallel(N_Vector w);|
  
%%--------------------------------------

\item \ID{N\_VMake\_Parallel}
  
  This function creates and allocates memory for a parallel vector
  with user-provided data array.
 
  Prototype

\begin{verbatim}
N_Vector N_VMake_Parallel(MPI_Comm comm, 
                          long int local_length,
                          long int global_length,
                          realtype *v_data);
\end{verbatim}

%%--------------------------------------

\item \ID{N\_VNewVectorArray\_Parallel}
 
  This function creates an array of \id{count} parallel vectors.
 
\begin{verbatim}
N_Vector *N_VNewVectorArray_Parallel(int count, 
                                     MPI_Comm comm, 
                                     long int local_length,
                                     long int global_length);
\end{verbatim}

%%--------------------------------------

\item \ID{N\_VNewVectorArrayEmpty\_Parallel}
 
  This function creates an array of \id{count} parallel vectors,
  each with an empty (\id{NULL}) data array.
 
\begin{verbatim}
N_Vector *N_VNewVectorArrayEmpty_Parallel(int count, 
                                          MPI_Comm comm, 
                                          long int local_length,
                                          long int global_length);
\end{verbatim}

%%--------------------------------------

\item \ID{N\_VDestroyVectorArray\_Parallel}
 
 This function frees memory allocated for the array of \id{count} parallel \id{N\_Vector} 
 created with \id{N\_VNewVectorArray\_Parallel} or with \id{N\_VNewVectorArrayEmpty\_Parallel}.

 Prototype

 \verb|void N_VDestroyVectorArray_Parallel(N_Vector *vs, int count);|


%%--------------------------------------

\item \ID{N\_VPrint\_Parallel}
  
  This function prints the content of a parallel vector to stdout.
 
  Prototype
  
  \verb|void N_VPrint_Parallel(N_Vector v);|


\end{itemize}
%%
%%------------------------------------
%%
\paragraph{\bf Notes} 
           
\begin{itemize}
                                        
\item
  When looping over the components of an \id{N\_Vector} \id{v}, it is     
  more efficient to first obtain the local component array via       
  \id{v\_data = NV\_DATA\_P(v)} and then access \id{v\_data[i]} within the     
  loop than it is to use \id{NV\_Ith\_P(v,i)} within the loop.        
                                                               
\item
  The {\nvecp} constructor functions \id{N\_VNewEmpty\_Parallel}, \id{N\_VMake\_Parallel}, 
  \id{N\_VCloneEmpty\_Parallel}, and \id{N\_VNewVectorArrayEmpty\_Parallel}
  set the field {\em own\_data} $=$ \id{FALSE}. 
  The functions \id{N\_VDestroy\_Parallel} and \id{N\_VDestroyVectorArray\_Parallel}
  will not attempt to free the pointer {\em data} for any \id{N\_Vector} with
  {\em own\_data} set to \id{FALSE}. In such a case, it is the user's responsibility to
  deallocate the {\em data} pointer.

\item
  To maximize efficiency, vector operations in the {\nvecp} implementation
  that have more than one \id{N\_Vector} argument do not check for
  consistent internal representation of these vectors. It is the user's 
  responsability to ensure that such routines are called with \id{N\_Vector}
  arguments that were all created with the \id{NV\_Spec} structure returned
  by \id{NV\_SpecInit\_Parallel}.

\end{itemize}


%% This is a shared SUNDIALS TEX file with a description of the
%% OpenMP nvector implementation
%%

The OpenMP implementation of the {\nvector} module provided with {\sundials},
{\nvecopenmp}, defines the {\em content} field of \id{N\_Vector} to be a structure 
containing the length of the vector, a pointer to the beginning of a contiguous 
data array, and a boolean flag {\em own\_data} which specifies the ownership 
of {\em data}.  Operations on the vector are threaded using OpenMP, 
the number of threads used is based on the supplied argument in 
the vector constructor.
%%
\begin{verbatim} 
struct _N_VectorContent_OpenMP {
  long int length;
  booleantype own_data;
  realtype *data;
  int num_threads;
};
\end{verbatim}
%%
%%--------------------------------------------
%%
The following six macros are provided to access the content of an {\nvecopenmp}
vector. The suffix \id{\_OMP} in the names denotes OpenMP version.
%%
\begin{itemize}

\item \ID{NV\_CONTENT\_OMP}                             
    
  This routine gives access to the contents of the OpenMP
  vector \id{N\_Vector}.
  
  The assignment \id{v\_cont} $=$ \id{NV\_CONTENT\_OMP(v)} sets           
  \id{v\_cont} to be a pointer to the OpenMP \id{N\_Vector} content  
  structure.                                             
                                                            
  Implementation: 
  
  \verb|#define NV_CONTENT_OMP(v) ( (N_VectorContent_OpenMP)(v->content) )|
  
\item \ID{NV\_OWN\_DATA\_OMP}, \ID{NV\_DATA\_OMP}, \ID{NV\_LENGTH\_OMP}, \ID{NV\_NUM_THREADS\_OMP}


  These macros give individual access to the parts of    
  the content of a OpenMP \id{N\_Vector}.                        
                                                               
  The assignment \id{v\_data = NV\_DATA\_OMP(v)} sets \id{v\_data} to be     
  a pointer to the first component of the data for the \id{N\_Vector} \id{v}. 
  The assignment \id{NV\_DATA\_OMP(v) = v\_data} sets the component array of \id{v} to     
  be \id{v\_data} by storing the pointer \id{v\_data}.                   
  
  The assignment \id{v\_len = NV\_LENGTH\_OMP(v)} sets \id{v\_len} to be     
  the length of \id{v}. On the other hand, the call \id{NV\_LENGTH\_OMP(v) = len\_v} 
  sets the length of \id{v} to be \id{len\_v}.
                                                            
  The assignment \id{v\_num\_threads = NV\_NUM\_THREADS\_OMP(v)} sets \id{v\_num\_threads} to be     
  the number of threads from \id{v}. On the other hand, the call \id{NV\_NUM\_THREADS\_OMP(v) = num\_threads\_v} 
  sets the number of threads for \id{v} to be \id{num\_threads\_v}.
                                                            
  Implementation: 
  
  \verb|#define NV_OWN_DATA_OMP(v) ( NV_CONTENT_OMP(v)->own_data )|

  \verb|#define NV_DATA_OMP(v) ( NV_CONTENT_OMP(v)->data )|
  
  \verb|#define NV_LENGTH_OMP(v) ( NV_CONTENT_OMP(v)->length )|

  \verb|#define NV_NUM_THREADS_OMP(v) ( NV_CONTENT_OMP(v)->num_threads )|

\item \ID{NV\_Ith\_OMP}                                               
                                                            
  This macro gives access to the individual components of the data
  array of an \id{N\_Vector}.

  The assignment \id{r = NV\_Ith\_OMP(v,i)} sets \id{r} to be the value of 
  the \id{i}-th component of \id{v}. The assignment \id{NV\_Ith\_OMP(v,i) = r}   
  sets the value of the \id{i}-th component of \id{v} to be \id{r}.        
  
  Here $i$ ranges from $0$ to $n-1$ for a vector of length $n$.

  Implementation:

  \verb|#define NV_Ith_OMP(v,i) ( NV_DATA_OMP(v)[i] )|

\end{itemize}
%%
%%----------------------------------------------
%%
The {\nvecopenmp} module defines OpenMP implementations of all vector operations listed 
in Table \ref{t:nvecops}. Their names are obtained from those in Table \ref{t:nvecops} by
appending the suffix \id{\_OpenMP}. The module {\nvecopenmp} provides the following additional
user-callable routines:
%%
\begin{itemize}

%%--------------------------------------

\item \ID{N\_VNew\_OpenMP}

  This function creates and allocates memory for a OpenMP \id{N\_Vector}.
  Arguments are the vector length and number of threads.

  \verb|N_Vector N_VNew_OpenMP(long int vec_length, int num_threads);|

%%--------------------------------------

\item \ID{N\_VNewEmpty\_OpenMP}

  This function creates a new OpenMP \id{N\_Vector} with an empty (\id{NULL}) data array.

  

  \verb|N_Vector N_VNewEmpty_OpenMP(long int vec_length, int num_threads);|

%%--------------------------------------

\item \ID{N\_VMake\_OpenMP}

 This function creates and allocates memory for a OpenMP vector
 with user-provided data array.

 

 \verb|N_Vector N_VMake_OpenMP(long int vec_length, realtype *v_data, int num_threads);|

%%--------------------------------------

\item \ID{N\_VCloneVectorArray\_OpenMP}

 This function creates (by cloning) an array of \id{count} OpenMP vectors.

 

 \verb|N_Vector *N_VCloneVectorArray_OpenMP(int count, N_Vector w);|

%%--------------------------------------

\item \ID{N\_VCloneEmptyVectorArray\_OpenMP}

 This function creates (by cloning) an array of \id{count} OpenMP vectors, each with an
 empty (\id{NULL}) data array.

 

 \verb|N_Vector *N_VCloneEmptyVectorArray_OpenMP(int count, N_Vector w);|

%%--------------------------------------

\item \ID{N\_VDestroyVectorArray\_OpenMP}

 This function frees memory allocated for the array of \id{count} variables of type
 \id{N\_Vector} created with \id{N\_VCloneVectorArray\_OpenMP} or with
 \id{N\_VCloneEmptyVectorArray\_OpenMP}.

 

 \verb|void N_VDestroyVectorArray_OpenMP(N_Vector *vs, int count);|

%%--------------------------------------

\item \ID{N\_VPrint\_OpenMP}

 This function prints the content of a OpenMP vector to \id{stdout}.

 
 
 \verb|void N_VPrint_OpenMP(N_Vector v);|

\end{itemize}
%%
%%------------------------------------
%%
\paragraph{\bf Notes}                                                      
           
\begin{itemize}
                                        
\item
  When looping over the components of an \id{N\_Vector} \id{v}, it is     
  more efficient to first obtain the component array via       
  \id{v\_data = NV\_DATA\_OMP(v)} and then access \id{v\_data[i]} within the     
  loop than it is to use \id{NV\_Ith\_OMP(v,i)} within the loop.        

\item
  {\warn}\id{N\_VNewEmpty\_OpenMP}, \id{N\_VMake\_OpenMP}, 
  and \id{N\_VCloneEmptyVectorArray\_OpenMP} set the field 
  {\em own\_data} $=$ \id{FALSE}. 
  \id{N\_VDestroy\_OpenMP} and \id{N\_VDestroyVectorArray\_OpenMP}
  will not attempt to free the pointer {\em data} for any \id{N\_Vector} with
  {\em own\_data} set to \id{FALSE}. In such a case, it is the user's responsibility to
  deallocate the {\em data} pointer.
                                     
\item
  {\warn}To maximize efficiency, vector operations in the {\nvecopenmp} implementation
  that have more than one \id{N\_Vector} argument do not check for
  consistent internal representation of these vectors. It is the user's 
  responsibility to ensure that such routines are called with \id{N\_Vector}
  arguments that were all created with the same internal representations.

\end{itemize}

%% This is a shared SUNDIALS TEX file with a description of the
%% Pthreads nvector implementation
%%

In situations where a user has a multi-core processing unit capable of
running multiple parallel threads with shared memory, {\sundials} provides
an implementation of {\nvector} using OpenMP, called {\nvecopenmp}, and
an implementation using Pthreads, called {\nvecpthreads}.  
Testing has shown that vectors should be of length at least $100,000$ 
before the overhead associated with creating and using the threads is
made up by the parallelism in the vector calculations. 

The Pthreads {\nvector} implementation provided with {\sundials}, denoted
{\nvecpthreads}, defines the {\em content} field of \id{N\_Vector} to be a structure 
containing the length of the vector, a pointer to the beginning of a contiguous 
data array, a boolean flag {\em own\_data} which specifies the ownership 
of {\em data}, and the number of threads.  
Operations on the vector are threaded using POSIX threads 
(Pthreads).
%%
\begin{verbatim} 
struct _N_VectorContent_Pthreads {
  long int length;
  booleantype own_data;
  realtype *data;
  int num_threads;
};
\end{verbatim}
%%
%%--------------------------------------------

The header file to be included when using this module is \id{nvector\_pthreads.h}.

The following six macros are provided to access the content of an {\nvecpthreads}
vector. The suffix \id{\_PT} in the names denotes the Pthreads version.
%%
\begin{itemize}

\item \ID{NV\_CONTENT\_PT}                             
    
  This routine gives access to the contents of the Pthreads
  vector \id{N\_Vector}.
  
  The assignment \id{v\_cont} $=$ \id{NV\_CONTENT\_PT(v)} sets           
  \id{v\_cont} to be a pointer to the Pthreads \id{N\_Vector} content  
  structure.                                             
                                                            
  Implementation: 
  
  \verb|#define NV_CONTENT_PT(v) ( (N_VectorContent_Pthreads)(v->content) )|
  
\item \ID{NV\_OWN\_DATA\_PT}, \ID{NV\_DATA\_PT}, \ID{NV\_LENGTH\_PT}, \ID{NV\_NUM\_THREADS\_PT}


  These macros give individual access to the parts of    
  the content of a Pthreads \id{N\_Vector}.                        
                                                               
  The assignment \id{v\_data = NV\_DATA\_PT(v)} sets \id{v\_data} to be     
  a pointer to the first component of the data for the \id{N\_Vector} \id{v}. 
  The assignment \id{NV\_DATA\_PT(v) = v\_data} sets the component array of \id{v} to     
  be \id{v\_data} by storing the pointer \id{v\_data}.                   
  
  The assignment \id{v\_len = NV\_LENGTH\_PT(v)} sets \id{v\_len} to be     
  the length of \id{v}. On the other hand, the call \id{NV\_LENGTH\_PT(v) = len\_v} 
  sets the length of \id{v} to be \id{len\_v}.
                                                            
  The assignment \id{v\_num\_threads = NV\_NUM\_THREADS\_PT(v)} sets \id{v\_num\_threads} to be     
  the number of threads from \id{v}. On the other hand, the call \id{NV\_NUM\_THREADS\_PT(v) = num\_threads\_v} 
  sets the number of threads for \id{v} to be \id{num\_threads\_v}.
                                                            
  Implementation: 
  
  \verb|#define NV_OWN_DATA_PT(v) ( NV_CONTENT_PT(v)->own_data )|

  \verb|#define NV_DATA_PT(v) ( NV_CONTENT_PT(v)->data )|
  
  \verb|#define NV_LENGTH_PT(v) ( NV_CONTENT_PT(v)->length )|

  \verb|#define NV_NUM_THREADS_PT(v) ( NV_CONTENT_PT(v)->num_threads )|

\item \ID{NV\_Ith\_PT}                                               
                                                            
  This macro gives access to the individual components of the data
  array of an \id{N\_Vector}.

  The assignment \id{r = NV\_Ith\_PT(v,i)} sets \id{r} to be the value of 
  the \id{i}-th component of \id{v}. The assignment \id{NV\_Ith\_PT(v,i) = r}   
  sets the value of the \id{i}-th component of \id{v} to be \id{r}.        
  
  Here $i$ ranges from $0$ to $n-1$ for a vector of length $n$.

  Implementation:

  \verb|#define NV_Ith_PT(v,i) ( NV_DATA_PT(v)[i] )|

\end{itemize}
%%
%%----------------------------------------------
%%
The {\nvecpthreads} module defines Pthreads implementations of all vector operations listed 
in Table \ref{t:nvecops}. Their names are obtained from those in Table \ref{t:nvecops}
by appending the suffix \id{\_Pthreads} (e.g. \id{N\_VDestroy\_Pthreads}).
The module {\nvecpthreads} provides the following additional user-callable routines:
%%
\begin{itemize}

%%--------------------------------------

\item \ID{N\_VNew\_Pthreads}

  This function creates and allocates memory for a Pthreads \id{N\_Vector}.
  Arguments are the vector length and number of threads.

  

  \verb|N_Vector N_VNew_Pthreads(long int vec_length, int num_threads);|

%%--------------------------------------

\item \ID{N\_VNewEmpty\_Pthreads}

  This function creates a new Pthreads \id{N\_Vector} with an empty (\id{NULL}) data array.

  

  \verb|N_Vector N_VNewEmpty_Pthreads(long int vec_length, int num_threads);|

%%--------------------------------------

\item \ID{N\_VMake\_Pthreads}

 This function creates and allocates memory for a Pthreads vector
 with user-provided data array.

 (This function does {\em not} allocate memory for \id{v\_data} itself.)

 \verb|N_Vector N_VMake_Pthreads(long int vec_length, realtype *v_data, int num_threads);|

%%--------------------------------------

\item \ID{N\_VCloneVectorArray\_Pthreads}

 This function creates (by cloning) an array of \id{count} Pthreads vectors.

 

 \verb|N_Vector *N_VCloneVectorArray_Pthreads(int count, N_Vector w);|

%%--------------------------------------

\item \ID{N\_VCloneVectorArrayEmpty\_Pthreads}

 This function creates (by cloning) an array of \id{count} Pthreads vectors, each with an
 empty (\id{NULL}) data array.

 

 \verb|N_Vector *N_VCloneVectorArrayEmpty_Pthreads(int count, N_Vector w);|

%%--------------------------------------

\item \ID{N\_VDestroyVectorArray\_Pthreads}

 This function frees memory allocated for the array of \id{count} variables of type
 \id{N\_Vector} created with \id{N\_VCloneVectorArray\_Pthreads} or with
 \id{N\_VCloneVectorArrayEmpty\_Pthreads}.

 

 \verb|void N_VDestroyVectorArray_Pthreads(N_Vector *vs, int count);|

%%--------------------------------------

\item \ID{N\_VGetLength\_Pthreads}

 This function returns the number of vector elements.

 
 
 \verb|long int N_VGetLength_Pthreads(N_Vector v);|

%%--------------------------------------

\item \ID{N\_VPrint\_Pthreads}

 This function prints the content of a Pthreads vector to \id{stdout}.

 
 
 \verb|void N_VPrint_Pthreads(N_Vector v);|

\end{itemize}
%%
%%------------------------------------
%%
\paragraph{\bf Notes}                                                      
           
\begin{itemize}
                                        
\item
  When looping over the components of an \id{N\_Vector} \id{v}, it is     
  more efficient to first obtain the component array via       
  \id{v\_data = NV\_DATA\_PT(v)} and then access \id{v\_data[i]} within the     
  loop than it is to use \id{NV\_Ith\_PT(v,i)} within the loop.        

\item
  {\warn}\id{N\_VNewEmpty\_Pthreads}, \id{N\_VMake\_Pthreads}, 
  and \id{N\_VCloneVectorArrayEmpty\_Pthreads} set the field 
  {\em own\_data} $=$ \id{FALSE}. 
  \id{N\_VDestroy\_Pthreads} and \id{N\_VDestroyVectorArray\_Pthreads}
  will not attempt to free the pointer {\em data} for any \id{N\_Vector} with
  {\em own\_data} set to \id{FALSE}. In such a case, it is the user's responsibility to
  deallocate the {\em data} pointer.
                                     
\item
  {\warn}To maximize efficiency, vector operations in the {\nvecpthreads} implementation
  that have more than one \id{N\_Vector} argument do not check for
  consistent internal representation of these vectors. It is the user's 
  responsibility to ensure that such routines are called with \id{N\_Vector}
  arguments that were all created with the same internal representations.

\end{itemize}

For solvers that include a Fortran interface module, the {\nvecpthreads}
module also includes a Fortran-callable function
\id{FNVINITPTS(code, NEQ, NUMTHREADS, IER)}, to initialize this
module.  Here \id{code} is an input solver id
(1 for {\cvode}, 2 for {\ida}, 3 for {\kinsol}, 4 for {\arkode}); NEQ is
the problem size (declared so as to match C type \id{long int});
NUMTHREADS is the number of threads; and IER is an error return flag
equal 0 for success and -1 for failure.

% This is a shared SUNDIALS TEX file with description of
% the MPI parallel hypre nvector implementation
%
The {\nvecph} implementation of the {\nvector} module provided with
{\sundials} is a wrapper around {\hypre}'s ParVector class. 
Most of the vector kernels simply call {\hypre} vector operations. 
The implementation defines the {\em content} field of \id{N\_Vector} to 
be a structure containing the global and local lengths of the vector, a 
pointer to an object of type \id{hypre\_ParVector}, an {\mpi} communicator, 
and a boolean flag {\em own\_parvector} indicating ownership of the
{\hypre} parallel vector object {\em x}.
%%
%%
\begin{verbatim}
struct _N_VectorContent_ParHyp {
  long int local_length;
  long int global_length;
  booleantype own_parvector;
  MPI_Comm comm;
  hypre_ParVector *x;
};
\end{verbatim}
%%
%%--------------------------------------------

\noindent
The header file to be included when using this module is \id{nvector\_parhyp.h}.
Unlike native {\sundials} vector types, {\nvecph} does not provide macros 
to access its member variables.

%%
%%--------------------------------------------
%%
The {\nvecph} module defines parhyp implementations of all vector operations listed 
in Table \ref{t:nvecops}, except for \verb|N_VSetArrayPointer|, because setting raw 
data pointers is handled by low-level {\hypre} functions. Implementation of 
\verb|N_VGetArrayPointer| is provided, but its use is strongly discouraged (we 
consider removing it, as well). When access to raw vector data is needed, it is 
recommended to extract {\hypre} vector first, and then use {\hypre} 
methods to access the raw data. 

The names of parhyp methods are obtained from those in Table \ref{t:nvecops}
by appending the suffix \id{\_ParHyp} (e.g. \id{N\_VDestroy\_ParHyp}).
The module {\nvecph} provides the following additional user-callable routines:
%%
%%
\begin{itemize}

%%--------------------------------------

\item \ID{N\_VNewEmpty\_ParHyp}
 
  This function creates a new parhyp \id{N\_Vector} with pointer to {\hypre} 
  vector set to \id{NULL}.
 
  

\begin{verbatim}
N_Vector N_VNewEmpty_ParHyp(MPI_Comm comm, 
                            long int local_length, 
                            long int global_length);
\end{verbatim}

  
%%--------------------------------------

\item \ID{N\_VMake\_ParHyp}
  
  This function creates \verb|N_Vector| wrapper around an existing
{\hypre} parallel vector.
 
(This function does {\em not} allocate memory for \id{x} itself.)  

\begin{verbatim}
N_Vector N_VMake_ParHyp(hypre_ParVector *x);
\end{verbatim}

%%--------------------------------------

\item \ID{N\_VCloneVectorArray\_ParHyp}
 
  This function creates (by cloning) an array of \id{count} parallel vectors.
 
\begin{verbatim}
N_Vector *N_VCloneVectorArray_ParHyp(int count, N_Vector w);
\end{verbatim}

%%--------------------------------------

\item \ID{N\_VCloneVectorArrayEmpty\_ParHyp}
 
  This function creates (by cloning) an array of \id{count} parallel vectors,
  each with an empty (\id{NULL}) data array.
 
\begin{verbatim}
N_Vector *N_VCloneVectorArrayEmpty_ParHyp(int count, N_Vector w);
\end{verbatim}

%%--------------------------------------

\item \ID{N\_VDestroyVectorArray\_ParHyp}
 
 This function frees memory allocated for the array of \id{count}  variables of
 type \id{N\_Vector} created with \id{N\_VCloneVectorArray\_ParHyp} or with
 \id{N\_VCloneVectorArrayEmpty\_ParHyp}.
 

 \verb|void N_VDestroyVectorArray_ParHyp(N_Vector *vs, int count);|


%%--------------------------------------

\item \ID{N\_VGetVector\_ParHyp}
  
  This function returns pointer to the underlying {\hypre} vector.
 
    
  \verb|hypre_ParVector *N_VPrint_ParHyp(N_Vector v);|


%%--------------------------------------

\item \ID{N\_VPrint\_ParHyp}
  
  This function prints the content of a parhyp vector to stdout.
 
    
  \verb|void N_VPrint_ParHyp(N_Vector v);|


\end{itemize}
%%
%%------------------------------------
%%
\paragraph{\bf Notes} 
           
\begin{itemize}
                                        
\item
  When there is a need to access components of an \id{N\_Vector} \id{v}, 
  it is recommeded to extract {\hypre} vector via       
  \id{x\_vec = N\_VGetVector(v)} and then access components using 
  {\hypre} functions.        
                                                               
\item
  {\warn}\id{N\_VNewEmpty\_ParHyp}, \id{N\_VMake\_ParHyp}, 
  and \id{N\_VCloneVectorArrayEmpty\_ParHyp} set the field 
  {\em own\_parvector} $=$ \id{FALSE}. 
  \id{N\_VDestroy\_ParHyp} and \id{N\_VDestroyVectorArray\_ParHyp}
  will not attempt to delete underlying {\hypre} vector for any \id{N\_Vector} 
  with {\em own\_parvector} set to \id{FALSE}. In such a case, it is the 
  user's responsibility to delete the underlying vector.

\item
  {\warn}To maximize efficiency, vector operations in the {\nvecph} implementation
  that have more than one \id{N\_Vector} argument do not check for
  consistent internal representation of these vectors. It is the user's 
  responsibility to ensure that such routines are called with \id{N\_Vector}
  arguments that were all created with the same internal representations.

\end{itemize}

% For solvers that include a Fortran interface module, the {\nvecph} module
% also includes a Fortran-callable function
% \id{FNVINITPH(COMM, code, NLOCAL, NGLOBAL, IER)},
% to initialize this {\nvecph} module.  Here \id{COMM} is the MPI communicator,
% \id{code} is an input solver id (1 for {\cvode}, 2 for {\ida}, 3 for {\kinsol},
% 4 for {\arkode}); \id{NLOCAL} and \id{NGLOBAL} are the local and global
% vector sizes, respectively (declared so as to match C type \id{long int});
% and IER is an error return flag equal 0 for success and -1 for failure.
% 
% {\warn}Note: If the header file \id{sundials\_config.h} defines
% \id{SUNDIALS\_MPI\_COMM\_F2C} to be $1$ (meaning the {\mpi}
% implementation used to build {\sundials} includes the
% \id{MPI\_Comm\_f2c} function), then \id{COMM} can be any valid
% {\mpi} communicator. Otherwise, \id{MPI\_COMM\_WORLD} will be used, so
% just pass an integer value as a placeholder.

% This is a shared SUNDIALS TEX file with description of
% the PETSc nvector wrapper implementation
%
The {\nvecpetsc} is an {\nvector} wrapper around PETSc vector. It defines the 
{\em content} field of \id{N\_Vector} to be a structure containing the global 
and local lengths of the vector, a pointer to the PETSc vector,
an {\mpi} communicator, and a boolean flag {\em own\_data} indicating ownership of 
the wrapped PETSc vector.
%%
\begin{verbatim} 
struct _N_VectorContent_petsc {
  long int local_length;
  long int global_length;
  booleantype own_data;
  Vec *pvec;
  MPI_Comm comm;
};
\end{verbatim}
%%
%%--------------------------------------------
%%
Note that PETSc vector wrapper requires {\sundials} to be built with {\mpi} support.
The following seven macros are provided to access the content of a {\nvecpetsc}
vector. The suffix \id{\_PTC} in the names denotes the PETSc wrapper 
version.
\begin{itemize}

\item 
  \ID{NV\_CONTENT\_PTC}

  This macro gives access to the contents of the parallel
  vector \id{N\_Vector}.
  
  The assignment \id{v\_cont = NV\_CONTENT\_PTC(v)} sets       
  \id{v\_cont} to be a pointer to the \id{N\_Vector} content    
  structure of type \id{struct \_N\_VectorParallelContent}.
  
  Implementation:
  
  \verb|#define NV_CONTENT_PTC(v) ( (N_VectorContent_petsc)(v->content) )|
  
\item 
  \ID{NV\_OWN\_DATA\_PTC}, \ID{NV\_PVEC\_PTC}, 
  \ID{NV\_LOCLENGTH\_PTC}, \ID{NV\_GLOBLENGTH\_PTC}
  
  These macros give individual access to the parts of    
  the content of a parallel \id{N\_Vector}.                        
  
  The assignment \id{v\_pvec = NV\_PVEC\_PTC(v)} sets the pointer to PETSc vector 
  \id{v\_pvec} to the address of the PETSc vector wrapped by \id{N\_Vector} \id{v}. 
  
  The assignment \id{v\_llen = NV\_LOCLENGTH\_PTC(v)} sets \id{v\_llen} to be     
  the length of the local part of \id{v}. 
  The call \id{NV\_LENGTH\_PTC(v) = llen\_v} sets      
  the local length of \id{v} to be \id{llen\_v}.
  
  The assignment \id{v\_glen = NV\_GLOBLENGTH\_PTC(v)} sets \id{v\_glen} to  
  be the global length of the vector \id{v}.                    
  The call \id{NV\_GLOBLENGTH\_PTC(v) = glen\_v} sets the global       
  length of \id{v} to be \id{glen\_v}.
  
  {\warn}Local and global vector lengths will be obsoleted by PETSc's own methods 
  for setting and retrieving vector length. 
  
  Implementation:
  
  \verb|#define NV_OWN_DATA_PTC(v)   ( NV_CONTENT_PTC(v)->own_data )|

  \verb|#define NV_PVEC_PTC(v)       ( NV_CONTENT_PTC(v)->pvec )|

  \verb|#define NV_LOCLENGTH_PTC(v)  ( NV_CONTENT_PTC(v)->local_length )|

  \verb|#define NV_GLOBLENGTH_PTC(v) ( NV_CONTENT_PTC(v)->global_length )|
  
\item \ID{NV\_COMM\_PTC}

  This macro provides access to the {\mpi} communicator used by the {\nvecpetsc}
  vectors.

  Implementation:

  \verb|#define NV_COMM_PTC(v) ( NV_CONTENT_PTC(v)->comm )|

\end{itemize}
%%
%%--------------------------------------------
%%
The {\nvecpetsc} module defines parallel implementations of all vector operations listed 
in Table \ref{t:nvecops}, except for \verb|N_VGetArrayPointer| and 
\verb|N_VSetArrayPointer|. The names of vector operations are obtained from those in 
Table \ref{t:nvecops} by appending the suffix \id{\_petsc}. The module {\nvecpetsc} 
provides the following additional user-callable routines:
%%
%%
\begin{itemize}

%%--------------------------------------

\item  \ID{N\_VNew\_petsc}
  
  This function creates and allocates new {\nvector} wrapper and PETSc 
  vector within.
 
  

\begin{verbatim}
N_Vector N_VNew_petsc(MPI_Comm comm, 
                      long int local_length, 
                      long int global_length);
\end{verbatim}
  
%%--------------------------------------

\item \ID{N\_VNewEmpty\_petsc}
 
  This function creates a new {\nvector} wrapper with pointer to wrapped 
  PETSc vector set to (\id{NULL}).
 
  

\begin{verbatim}
N_Vector N_VNewEmpty_petsc(MPI_Comm comm, 
                           long int local_length, 
                           long int global_length);
\end{verbatim}

  
%%--------------------------------------


\item \ID{N\_VCloneVectorArray\_petsc}
 
  This function creates (by cloning) an array of \id{count} parallel vectors.
 
\begin{verbatim}
N_Vector *N_VCloneVectorArray_petsc(int count, N_Vector w);
\end{verbatim}

%%--------------------------------------

\item \ID{N\_VCloneEmptyVectorArray\_petsc}
 
  This function creates (by cloning) an array of \id{count} parallel vectors,
  each with pointers to PETSc vectors set to (\id{NULL}).
 
\begin{verbatim}
N_Vector *N_VCloneEmptyVectorArray_petsc(int count, N_Vector w);
\end{verbatim}

%%--------------------------------------

\item \ID{N\_VDestroyVectorArray\_petsc}
 
 This function frees memory allocated for the array of \id{count} variables of
 type \id{N\_Vector} created with \id{N\_VCloneVectorArray\_petsc} or with
 \id{N\_VCloneEmptyVectorArray\_petsc}.
 

 \verb|void N_VDestroyVectorArray_petsc(N_Vector *vs, int count);|


%%--------------------------------------

\item \ID{N\_VPrint\_petsc}
  
  This function prints the content of the wrapped PETSc vector to stdout.
 
    
  \verb|void N_VPrint_petsc(N_Vector v);|


\end{itemize}
%%
%%------------------------------------
%%
\paragraph{\bf Notes} 
           
\begin{itemize}
                                        
\item
  {\warn}\id{N\_VNewEmpty\_petsc}, \id{N\_VMake\_petsc}, 
  and \id{N\_VCloneEmptyVectorArray\_petsc} set the field 
  {\em own\_data} $=$ \id{FALSE}. 
  \id{N\_VDestroy\_petsc} and \id{N\_VDestroyVectorArray\_petsc}
  will not attempt to free the pointer {\em pvec} for any \id{N\_Vector} with
  {\em own\_data} set to \id{FALSE}. In such a case, it is the user's responsibility to
  deallocate the {\em pvec} pointer.

\item
  {\warn}To maximize efficiency, vector operations in the {\nvecpetsc} implementation
  that have more than one \id{N\_Vector} argument do not check for
  consistent internal representation of these vectors. It is the user's 
  responsibility to ensure that such routines are called with \id{N\_Vector}
  arguments that were all created with the same internal representations.

\end{itemize}


% This is a shared SUNDIALS TEX file with description of
% the CUDA nvector implementation
%
The {\nveccuda} module is an experimental {\nvector} implementation in the {\cuda} language.
The module allows for {\sundials} vector kernels to run on GPU devices. It is intended for users
who are already familiar with {\cuda} and GPU programming. Building this vector
module requires a CUDA compiler and, by extension, a C++ compiler. The class \id{Vector}
in namespace \id{suncudavec} manages vector data layout:
\begin{verbatim}
template <class T, class I>
class Vector {
  I size_;
  I mem_size_;
  T* h_vec_;
  T* d_vec_;
  ThreadPartitioning<T, I>* partStream_;
  ThreadPartitioning<T, I>* partReduce_;
  bool ownPartitioning_;

  ...
};
\end{verbatim}

The class members are vector size (length), size of the vector data memory block, pointers
to vector data on the host and the device, pointers to \id{ThreadPartitioning}
implementations that handle thread partitioning for streaming and
reduction vector kernels, and a boolean flag that signals if the
vector owns the thread partitioning. The class \id{Vector} inherits from the empty structure
\begin{verbatim}
struct _N_VectorContent_Cuda {
};
\end{verbatim}
to interface the C++ class with the {\nvector} C code. When instantiated, the class
\id{Vector} will allocate memory on both the host and the device. Due to the rapid
progress of {\cuda} development, we expect that the \id{suncudavec::Vector}
class will change frequently in future {\sundials} releases. The code is
structured so that it can tolerate significant changes in the
\id{suncudavec::Vector} class without requiring changes to the user API.

%%
%%--------------------------------------------

The header file to include when using this module is \id{nvector\_cuda.h}.
The installed module library to link to is
\id{libsundials\_nveccuda.\textit{lib}}
where \id{\em.lib} is typically \id{.so} for shared libraries and \id{.a}
for static libraries.

Unlike other native {\sundials} vector types, {\nveccuda} does not provide macros
to access its member variables. Instead, user should use standalone functions in
namespace \id{suncudavec}.
\begin{itemize}

\item
  \ID{getDevData(N\_Vector v)}

  This function takes \id{N\_Vector} as an argument and returns raw pointer to vector
  data on the device (GPU). It is users responsibility to ensure correct vector is
  passed as the argument.

\item
  \ID{getHostData(N\_Vector v)}

  This function takes \id{N\_Vector} as an argument and returns raw pointer to vector
  data on the host (CPU memory). It is users responsibility to ensure correct vector is
  passed as the argument.

\item \ID{getSize(N\_Vector v)}

  Returns vector's local length.


\item \ID{getGlobalSize(N\_Vector v)}

  Returns vector's global length.


\item \ID{getMPICom(N\_Vector v)}

  Takes \id{N\_Vector} as an argument and returns sundials communicator of type
  \id{SUNDIALS\_Comm}.

\end{itemize}

%Note that {\nveccuda} requires {\sundials} to be built with {\mpi} support.

%%
%%--------------------------------------------
%%
The {\nveccuda} module defines implementations of all vector operations listed
in Tables \ref{t:nvecops}, \ref{t:nvecfusedops}, and \ref{t:nvecarrayops}, except
for \id{N\_VGetArrayPointer} and \id{N\_VSetArrayPointer}.
As such, this vector cannot be used with {\sundials} Fortran interfaces,
nor with {\sundials} direct solvers and preconditioners. Instead,
the {\nveccuda} module provides separate functions to access data on the host
and on the device. It also provides methods for copying from the host to
the device and vice versa. Usage examples of {\nveccuda} are provided in
some example programs for {\cvode} \cite{cvode_ex}.

The names of vector operations are obtained from those in Tables \ref{t:nvecops},
\ref{t:nvecfusedops}, and \ref{t:nvecarrayops} by appending the suffix \id{\_Cuda}
(e.g. \id{N\_VDestroy\_Cuda}).
The module {\nveccuda}  provides the following additional user-callable routines:
%%
%%
\begin{itemize}


%%--------------------------------------

\item \ID{N\_VNew\_MPI\_Cuda}

  This function creates and allocates memory for a {\cuda} \id{N\_Vector}.
  The memory is allocated on both host and device. Its arguments are local
  and global vector lengths, as well as the {\sundials} communicator. If
  {\sundials} is built with MPI support, the communicator is MPI communicator.
  Otherwise, it is an integer number and is ignored by \id{N\_VNew\_MPI\_Cuda}.
  If {\sundials} is built without MPI support local and global vector
  lengths must be the same.

\begin{verbatim}
N_Vector N_VNew_Cuda(SUNDIALS_Comm comm,
                     sunindextype local_length,
                     sunindextype global_length);
\end{verbatim}


%%--------------------------------------

\item \ID{N\_VNew\_Cuda}

  This function creates and allocates memory for a {\cuda} \id{N\_Vector}
  on a single node. The memory is allocated on both host and device.
  Its only argument is vector length. Use this constructor when {\sundials}
  is built without MPI support.

\begin{verbatim}
N_Vector N_VNew_Cuda(sunindextype length);
\end{verbatim}


%%--------------------------------------

\item \ID{N\_VNewEmpty\_Cuda}

  This function creates a new {\nvector} wrapper with the pointer to
  the wrapped {\cuda} vector set to (\id{NULL}). It is used by the
  \id{N\_VNew\_Cuda}, \id{N\_VMake\_Cuda}, and \id{N\_VClone\_Cuda}
  implementations.

\begin{verbatim}
N_Vector N_VNewEmpty_Cuda(sunindextype vec_length);
\end{verbatim}


%%--------------------------------------

\item \ID{N\_VMake\_Cuda}

  This function creates and allocates memory for an {\nveccuda}
  wrapper around a user-provided \id{suncudavec::Vector} class.
  Its only argument is of type \id{N\_VectorContent\_Cuda}, which
  is the pointer to the class.

\begin{verbatim}
N_Vector N_VMake_Cuda(N_VectorContent_Cuda c);
\end{verbatim}

%%--------------------------------------


\item \ID{N\_VGetLength\_Cuda}

 This function returns the length of the vector.

 \verb|sunindextype N_VGetLength_Cuda(N_Vector v);|

%%--------------------------------------

\item \ID{N\_VGetHostArrayPointer\_Cuda}

 This function returns a pointer to the vector data on the host.

 \verb|realtype *N_VGetHostArrayPointer_Cuda(N_Vector v);|


%%--------------------------------------

\item \ID{N\_VGetDeviceArrayPointer\_Cuda}

 This function returns a pointer to the vector data on the device.

 \verb|realtype *N_VGetDeviceArrayPointer_Cuda(N_Vector v);|


%%--------------------------------------

\item \ID{N\_VCopyToDevice\_Cuda}

 This function copies host vector data to the device.

 \verb|realtype *N_VCopyToDevice_Cuda(N_Vector v);|


%%--------------------------------------

\item \ID{N\_VCopyFromDevice\_Cuda}

 This function copies vector data from the device to the host.

 \verb|realtype *N_VCopyFromDevice_Cuda(N_Vector v);|


%%--------------------------------------

\item \ID{N\_VPrint\_Cuda}

  This function prints the content of a {\cuda} vector to \id{stdout}.

  \verb|void N_VPrint_Cuda(N_Vector v);|

%%--------------------------------------

\item \ID{N\_VPrintFile\_Cuda}

  This function prints the content of a {\cuda} vector to \id{outfile}.

  \verb|void N_VPrintFile_Cuda(N_Vector v, FILE *outfile);|


\end{itemize}
%%
%%------------------------------------
%%
\paragraph{\bf Notes}

\begin{itemize}

\item
  When there is a need to access components of an \id{N\_Vector\_Cuda}, \id{v},
  it is recommeded to use functions \id{N\_VGetDeviceArrayPointer\_Cuda} or
  \id{N\_VGetHostArrayPointer\_Cuda}.

% \item
%   {\warn}Unlike in other {\nvector} implementations, vector data will always be
%   deleted when invoking \id{N\_VDestroy\_Cuda} and \id{N\_VDestroyVectorArray\_Cuda},
%   even when the vector is created using \id{N\_VMake\_Cuda}. It is user's responsibility
%   to track memory allocations and deletions when using \id{N\_VMake\_Cuda}.

\item
  {\warn}To maximize efficiency, vector operations in the {\nveccuda} implementation
  that have more than one \id{N\_Vector} argument do not check for
  consistent internal representations of these vectors. It is the user's
  responsibility to ensure that such routines are called with \id{N\_Vector}
  arguments that were all created with the same internal representations.

\end{itemize}


% This is a shared SUNDIALS TEX file with description of
% the CUDA nvector implementation
%
The {\nvecraja} module is an experimental {\nvector} implementation using the
\href{https://software.llnl.gov/RAJA/}{\raja} hardware abstraction layer.
In this implementation, {\raja}
allows for {\sundials} vector kernels to run on GPU devices. The module is intended for users
who are already familiar with {\raja} and GPU programming. Building this vector
module requires a C++11 compliant compiler and a CUDA software development toolkit.
Besides the {\cuda} backend, {\raja} has other backends such as serial, OpenMP,
and OpenAC. These backends are not used in this {\sundials} release.
Class \id{Vector} in namespace \id{sunrajavec} manages the vector data layout:
\begin{verbatim}
template <class T, class I>
class Vector {
  I size_;
  I mem_size_;
  T* h_vec_;
  T* d_vec_;

  ...
};
\end{verbatim}
The class members are: vector size (length), size of the vector data memory block,
and pointers to vector data on the host and on the device. The class \id{Vector}
inherits from an empty structure
\begin{verbatim}
struct _N_VectorContent_Raja {
};
\end{verbatim}
to interface the C++ class with the {\nvector} C code. When instantiated, the class
\id{Vector} will allocate memory on both the host and the device. Due to the rapid
progress of {\raja} development, we expect that the \id{sunrajavec::Vector}
class will change frequently in future {\sundials} releases. The code is
structured so that it can tolerate significant changes in the
\id{sunrajavec::Vector} class without requiring changes to the user API.

%%
%%--------------------------------------------

The {\nvecraja} module can be utilized for single-node parallelism or in a distributed context with MPI.
The header file to include when using this module for single-node parallelism is \id{nvector\_raja.h}.
The header file to include when using this module in the distributed case is \id{nvector\_mpiraja.h}.
Note that only the {\nvecraja} constructor signature differs between the two header files.
The installed module libraries to link to are \id{libsundials\_nvecraja.\textit{lib}} in the single-node case,
or \id{libsundials\_nvecmpicudaraja.\textit{lib}} in the distributed case. Only one one of these libraries may be linked to when creating an executable or library. {\sundials} must be built with
MPI support if the distributed library is desired. The extension, \id{\em.lib}, is typically \id{.so} for shared libraries and \id{.a} for static libraries.

Unlike other native {\sundials} vector types, {\nvecraja} does not provide macros
to access its member variables. Instead, user should use the accessor functions in
the namespace \id{sunrajavec}.
\begin{itemize}

\item
  \ID{getDevData(N\_Vector v)}

  This function takes a \id{N\_Vector} as an argument and returns a raw pointer to the vector
  data on the device (GPU). It is the user's responsibility to ensure that the vector argument 
  is of the correct \id{N\_Vector} type.

\item
  \ID{getHostData(N\_Vector v)}

  This function takes a \id{N\_Vector} as an argument and returns a raw pointer to the vector
  data on the host (CPU memory). It is the user's responsibility to ensure that the vector argument 
  is of the correct \id{N\_Vector} type.

\item \ID{getSize(N\_Vector v)}

  Returns the vector's local length.


\item \ID{getGlobalSize(N\_Vector v)}

  Returns the vector's global length.


\item \ID{getMPIComm(N\_Vector v)}

  Takes a \id{N\_Vector} as an argument and returns a sundials communicator of type
  \id{SUNDIALS\_Comm}.

\end{itemize}

%Note that {\nvecraja} requires {\sundials} to be built with {\mpi} support.

%%
%%--------------------------------------------
%%
The {\nvecraja} module defines the implementations of all vector operations listed
in Tables \ref{t:nvecops}, \ref{t:nvecfusedops}, and \ref{t:nvecarrayops}, except
for \id{N\_VDotProdMulti}, \id{N\_VWrmsNormVectorArray}, and \\ \noindent
\id{N\_VWrmsNormMaskVectorArray} as support for arrays of reduction vectors is not
yet supported in {\raja}. These function will be added to the {\nvecraja}
implementation in the future. Additionally the vector operations \id{N\_VGetArrayPointer} and
\id{N\_VSetArrayPointer} are not implemented by the {\raja} vector.
As such, this vector cannot be used with the {\sundials} Fortran interfaces,
nor with the {\sundials} direct solvers and preconditioners.
The {\nvecraja} module provides separate functions to access data on the host
and on the device. It also provides methods for copying data from the host to
the device and vice versa. Usage examples of {\nvecraja} are provided in
some example programs for {\cvode} \cite{cvode_ex}.

The names of vector operations are obtained from those in Tables \ref{t:nvecops},
\ref{t:nvecfusedops}, and \ref{t:nvecarrayops}, by appending the suffix \id{\_Raja}
(e.g. \id{N\_VDestroy\_Raja}).
The module {\nvecraja}  provides the following additional user-callable routines:
%%
%%
\begin{itemize}


%%--------------------------------------

\item \ID{N\_VNew\_Raja}

  \textit{Note: this function signature is defined in the header \id{nvector\_mpiraja.h}
    and should be used when using this module in a distributed context.}
  This function creates and allocates memory for a {\raja} \id{N\_Vector}.
  The memory is allocated on both host and device. Its arguments are local
  and global vector lengths, as well as the {\mpi} communicator. Use this 
  constructor with the \id{libsundials\_nvecmpicudaraja}.\text{lib} library.

\begin{verbatim}
N_Vector N_VNew_Raja(MPI_Comm comm,
                     sunindextype local_length,
                     sunindextype global_length);
\end{verbatim}


%%--------------------------------------

\item \ID{N\_VNew\_Raja}

  \textit{Note: this function signature is defined in the header \id{nvector\_raja.h}
    and should be used when using this module for single-node parallelism.}
  This function creates and allocates memory for a {\raja} \id{N\_Vector}
  on a single node. The memory is allocated on both host and device.
  Its only argument is vector length. Use this constructor with the 
  \id{libsundials\_nveccudaraja}.\text{lib} library.

\begin{verbatim}
N_Vector N_VNew_Raja(sunindextype length);
\end{verbatim}


%%--------------------------------------

\item \ID{N\_VNewEmpty\_Raja}

  This function creates a new {\nvector} wrapper with the pointer to
  the wrapped {\raja} vector set to (\id{NULL}). It is used by the
  \id{N\_VNew\_Raja}, \id{N\_VMake\_Raja}, and \id{N\_VClone\_Raja}
  implementations.

\begin{verbatim}
N_Vector N_VNewEmpty_Raja(sunindextype vec_length);
\end{verbatim}


%%--------------------------------------

\item \ID{N\_VMake\_Raja}

  This function creates and allocates memory for an {\nvecraja}
  wrapper around a user-provided \id{sunrajavec::Vector} class.
  Its only argument is of type \id{N\_VectorContent\_Raja}, which
  is the pointer to the class.

\begin{verbatim}
N_Vector N_VMake_Raja(N_VectorContent_Raja c);
\end{verbatim}

%%--------------------------------------


\item \ID{N\_VGetLength\_Raja}

 This function returns the length of the vector.

 \verb|sunindextype N_VGetLength_Raja(N_Vector v);|

%%--------------------------------------

\item \ID{N\_VGetHostArrayPointer\_Raja}

 This function returns a pointer to the vector data on the host.

 \verb|realtype *N_VGetHostArrayPointer_Raja(N_Vector v);|


%%--------------------------------------

\item \ID{N\_VGetDeviceArrayPointer\_Raja}

 This function returns a pointer to the vector data on the device.

 \verb|realtype *N_VGetDeviceArrayPointer_Raja(N_Vector v);|


%%--------------------------------------

\item \ID{N\_VCopyToDevice\_Raja}

 This function copies host vector data to the device.

 \verb|realtype *N_VCopyToDevice_Raja(N_Vector v);|


%%--------------------------------------

\item \ID{N\_VCopyFromDevice\_Raja}

 This function copies vector data from the device to the host.

 \verb|realtype *N_VCopyFromDevice_Raja(N_Vector v);|


%%--------------------------------------

\item \ID{N\_VPrint\_Raja}

  This function prints the content of a {\raja} vector to \id{stdout}.

  \verb|void N_VPrint_Raja(N_Vector v);|

%%--------------------------------------

\item \ID{N\_VPrintFile\_Raja}

  This function prints the content of a {\raja} vector to \id{outfile}.

  \verb|void N_VPrintFile_Raja(N_Vector v, FILE *outfile);|


\end{itemize}
By default all fused and vector array operations are disabled in the {\nvecraja}
module. The following additional user-callable routines are provided to
enable or disable fused and vector array operations for a specific vector. To
ensure consistency across vectors it is recommended to first create a vector
with \id{N\_VNew\_Raja}, enable/disable the desired operations for that vector
with the functions below, and create any additional vectors from that vector
using \id{N\_VClone}. This guarantees the new vectors will have the same
operations enabled/disabled as cloned vectors inherit the same enable/disable
options as the vector they are cloned from while vectors created with
\id{N\_VNew\_Raja} will have the default settings for the {\nvecraja} module.
\begin{itemize}

%%--------------------------------------

\item \ID{N\_VEnableFusedOps\_Raja}

This function enables (\id{SUNTRUE}) or disables (\id{SUNFALSE}) all fused and
vector array operations in the {\raja} vector. The return value is \id{0} for
success and \id{-1} if the input vector or its \id{ops} structure are \id{NULL}.

\verb|int N_VEnableFusedOps_Raja(N_Vector v, booleantype tf);|

%%--------------------------------------

\item \ID{N\_VEnableLinearCombination\_Raja}

This function enables (\id{SUNTRUE}) or disables (\id{SUNFALSE}) the linear
combination fused operation in the {\raja} vector. The return value is \id{0} for
success and \id{-1} if the input vector or its \id{ops} structure are \id{NULL}.

\verb|int N_VEnableLinearCombination_Raja(N_Vector v, booleantype tf);|

%%--------------------------------------

\item \ID{N\_VEnableScaleAddMulti\_Raja}

This function enables (\id{SUNTRUE}) or disables (\id{SUNFALSE}) the scale and
add a vector to multiple vectors fused operation in the {\raja} vector. The
return value is \id{0} for success and \id{-1} if the input vector or its
\id{ops} structure are \id{NULL}.

\verb|int N_VEnableScaleAddMulti_Raja(N_Vector v, booleantype tf);|

%%--------------------------------------

%% \item \ID{N\_VEnableDotProdMulti\_Raja}

%% This function enables (\id{SUNTRUE}) or disables (\id{SUNFALSE}) the multiple
%% dot products fused operation in the {\raja} vector. The return value is \id{0}
%% for success and \id{-1} if the input vector or its \id{ops} structure are
%% \id{NULL}.

%% \verb|int N_VEnableDotProdMulti_Raja(N_Vector v, booleantype tf);|

%%--------------------------------------

\item \ID{N\_VEnableLinearSumVectorArray\_Raja}

This function enables (\id{SUNTRUE}) or disables (\id{SUNFALSE}) the linear sum
operation for vector arrays in the {\raja} vector. The return value is \id{0} for
success and \id{-1} if the input vector or its \id{ops} structure are \id{NULL}.

\verb|int N_VEnableLinearSumVectorArray_Raja(N_Vector v, booleantype tf);|

%%--------------------------------------

\item \ID{N\_VEnableScaleVectorArray\_Raja}

This function enables (\id{SUNTRUE}) or disables (\id{SUNFALSE}) the scale
operation for vector arrays in the {\raja} vector. The return value is \id{0} for
success and \id{-1} if the input vector or its \id{ops} structure are \id{NULL}.

\verb|int N_VEnableScaleVectorArray_Raja(N_Vector v, booleantype tf);|

%%--------------------------------------

\item \ID{N\_VEnableConstVectorArray\_Raja}

This function enables (\id{SUNTRUE}) or disables (\id{SUNFALSE}) the const
operation for vector arrays in the {\raja} vector. The return value is \id{0} for
success and \id{-1} if the input vector or its \id{ops} structure are \id{NULL}.

\verb|int N_VEnableConstVectorArray_Raja(N_Vector v, booleantype tf);|

%%--------------------------------------

%% \item \ID{N\_VEnableWrmsNormVectorArray\_Raja}

%% This function enables (\id{SUNTRUE}) or disables (\id{SUNFALSE}) the WRMS norm
%% operation for vector arrays in the {\raja} vector. The return value is \id{0} for
%% success and \id{-1} if the input vector or its \id{ops} structure are \id{NULL}.

%% \verb|int N_VEnableWrmsNormVectorArray_Raja(N_Vector v, booleantype tf);|

%%--------------------------------------

%% \item \ID{N\_VEnableWrmsNormMaskVectorArray\_Raja}

%% This function enables (\id{SUNTRUE}) or disables (\id{SUNFALSE}) the masked WRMS
%% norm operation for vector arrays in the {\raja} vector. The return value is
%% \id{0} for success and \id{-1} if the input vector or its \id{ops} structure are
%% \id{NULL}.

%% \verb|int N_VEnableWrmsNormMaskVectorArray_Raja(N_Vector v, booleantype tf);|

%%--------------------------------------

\item \ID{N\_VEnableScaleAddMultiVectorArray\_Raja}

This function enables (\id{SUNTRUE}) or disables (\id{SUNFALSE}) the scale and
add a vector array to multiple vector arrays operation in the {\raja} vector. The
return value is \id{0} for success and \id{-1} if the input vector or its
\id{ops} structure are \id{NULL}.

\verb|int N_VEnableScaleAddMultiVectorArray_Raja(N_Vector v, booleantype tf);|

%%--------------------------------------

\item \ID{N\_VEnableLinearCombinationVectorArray\_Raja}

This function enables (\id{SUNTRUE}) or disables (\id{SUNFALSE}) the linear
combination operation for vector arrays in the {\raja} vector. The return value
is \id{0} for success and \id{-1} if the input vector or its \id{ops} structure
are \id{NULL}.

\verb|int N_VEnableLinearCombinationVectorArray_Raja(N_Vector v, booleantype tf);|

%%--------------------------------------
\end{itemize}
%%
%%------------------------------------
%%
\paragraph{\bf Notes}

\begin{itemize}

\item
  When there is a need to access components of an \id{N\_Vector\_Raja}, \id{v},
  it is recommeded to use functions \id{N\_VGetDeviceArrayPointer\_Raja} or
  \id{N\_VGetHostArrayPointer\_Raja}.

% \item
%   {\warn}Unlike in other {\nvector} implementations, vector data will always be
%   deleted when invoking \id{N\_VDestroy\_Raja} and \id{N\_VDestroyVectorArray\_Raja},
%   even when the vector is created using \id{N\_VMake\_Raja}. It is user's responsibility
%   to track memory allocations and deletions when using \id{N\_VMake\_Raja}.

\item
  {\warn}To maximize efficiency, vector operations in the {\nvecraja} implementation
  that have more than one \id{N\_Vector} argument do not check for
  consistent internal representations of these vectors. It is the user's
  responsibility to ensure that such routines are called with \id{N\_Vector}
  arguments that were all created with the same internal representations.

\end{itemize}


%% This is a shared SUNDIALS TEX file with a description of the
%% OpenMPDEV nvector implementation
%%
\section{The NVECTOR\_OPENMPDEV implementation}\label{ss:nvec_openmpdev}

In situations where a user has access to a device such as a GPU for
offloading computation, {\sundials} provides an {\nvector} implementation using
OpenMP device offloading, called {\nvecopenmpdev}.

The {\nvecopenmpdev} implementation defines the \textit{content} field
of the \id{N\_Vector} to be a structure  containing the length of the vector, a pointer
to the beginning of a contiguous  data array on the host, a pointer to the beginning of
a contiguous data array on the device, and a boolean flag \id{own\_data} which specifies
the ownership of host and device data arrays.
%%
\begin{verbatim}
struct _N_VectorContent_OpenMPDEV {
  sunindextype length;
  booleantype own_data;
  realtype *host_data;
  realtype *dev_data;
};
\end{verbatim}
%%
%%--------------------------------------------

The header file to include when using this module is \id{nvector\_openmpdev.h}.
The installed module library to link to is
\id{libsundials\_nvecopenmpdev.\textit{lib}}
where \id{\em.lib} is typically \id{.so} for shared libraries and \id{.a}
for static libraries.


% ====================================================================
\subsection{NVECTOR\_OPENMPDEV accessor macros}
\label{ss:nvec_openmpdev_macros}
% ====================================================================

The following macros are provided to access the content of an {\nvecopenmpdev}
vector.
%%
\begin{itemize}

\item \ID{NV\_CONTENT\_OMPDEV}

  This routine gives access to the contents of the {\nvecopenmpdev}
  vector \id{N\_Vector}.

  The assignment \id{v\_cont} $=$ \id{NV\_CONTENT\_OMPDEV(v)} sets
  \id{v\_cont} to be a pointer to the {\nvecopenmpdev} \id{N\_Vector} content
  structure.

  Implementation:

  \verb|#define NV_CONTENT_OMPDEV(v) ( (N_VectorContent_OpenMPDEV)(v->content) )|

\item \ID{NV\_OWN\_DATA\_OMPDEV}, \ID{NV\_DATA\_HOST\_OMPDEV}, \ID{NV\_DATA\_DEV\_OMPDEV}, \ID{NV\_LENGTH\_OMPDEV}

  These macros give individual access to the parts of
  the content of an {\nvecopenmpdev} \id{N\_Vector}.

  The assignment \id{v\_data = NV\_DATA\_HOST\_OMPDEV(v)} sets \id{v\_data} to be
  a pointer to the first component of the data on the host for the \id{N\_Vector} \id{v}.
  The assignment \id{NV\_DATA\_HOST\_OMPDEV(v) = v\_data} sets the host component array of \id{v} to
  be \id{v\_data} by storing the pointer \id{v\_data}.

  The assignment \id{v\_dev\_data = NV\_DATA\_DEV\_OMPDEV(v)} sets \id{v\_dev\_data} to be
  a pointer to the first component of the data on the device for the \id{N\_Vector} \id{v}.
  The assignment \id{NV\_DATA\_DEV\_OMPDEV(v) = v\_dev\_data} sets the device component array of \id{v} to
  be \id{v\_dev\_data} by storing the pointer \id{v\_dev\_data}.

  The assignment \id{v\_len = NV\_LENGTH\_OMPDEV(v)} sets \id{v\_len} to be
  the length of \id{v}. On the other hand, the call \id{NV\_LENGTH\_OMPDEV(v) = len\_v}
  sets the length of \id{v} to be \id{len\_v}.

  Implementation:

  \verb|#define NV_OWN_DATA_OMPDEV(v) ( NV_CONTENT_OMPDEV(v)->own_data )|

  \verb|#define NV_DATA_HOST_OMPDEV(v) ( NV_CONTENT_OMPDEV(v)->host_data )|

  \verb|#define NV_DATA_DEV_OMPDEV(v) ( NV_CONTENT_OMPDEV(v)->dev_data )|

  \verb|#define NV_LENGTH_OMPDEV(v) ( NV_CONTENT_OMPDEV(v)->length )|

\end{itemize}


% ====================================================================
\subsection{NVECTOR\_OPENMPDEV functions}
\label{ss:nvec_openmpdev_functions}
% ====================================================================%

The {\nvecopenmpdev} module defines OpenMP device offloading implementations of
all vector operations listed in Tables \ref{t:nvecops}, \ref{t:nvecfusedops},
\ref{t:nvecarrayops}, and \ref{t:nveclocalops}, except for \id{N\_VGetArrayPointer} and
\id{N\_VSetArrayPointer}. As such, this vector cannot be used with the
{\sundials} Fortran interfaces, nor with the {\sundials} direct solvers and
preconditioners. It also provides methods for copying from the host to
the device and vice versa.

The names of vector operations are obtained from those in Tables
\ref{t:nvecops}, \ref{t:nvecfusedops}, \ref{t:nvecarrayops}, and
\ref{t:nveclocalops} by appending the
suffix \id{\_OpenMPDEV} (e.g. \id{N\_VDestroy\_OpenMPDEV}). The module
{\nvecopenmpdev} provides the following additional user-callable routines:
%%--------------------------------------
\sunmodfun{N\_VNew\_OpenMPDEV}
{
  This function creates and allocates memory for an {\nvecopenmpdev} \id{N\_Vector}.
}
{
  N\_Vector N\_VNew\_OpenMPDEV(sunindextype vec\_length)
}
%%--------------------------------------
\sunmodfun{N\_VNewEmpty\_OpenMPDEV}
{
  This function creates a new {\nvecopenmpdev} \id{N\_Vector} with an empty
  (\id{NULL}) host and device data arrays.
}
{
  N\_Vector N\_VNewEmpty\_OpenMPDEV(sunindextype vec\_length)
}
%%--------------------------------------
\sunmodfun{N\_VMake\_OpenMPDEV}
{
 This function creates an {\nvecopenmpdev} vector with user-supplied vector data
 arrays \id{h\_vdata} and \id{d\_vdata}. This function does not allocate memory for
 data itself.
}
{
  N\_Vector N\_VMake\_OpenMPDEV(sunindextype vec\_length, realtype *h\_vdata,
  \newlinefill{N\_Vector N\_VMake\_OpenMPDEV}
  realtype *d\_vdata)
}
%%--------------------------------------
\sunmodfun{N\_VCloneVectorArray\_OpenMPDEV}
{
 This function creates (by cloning) an array of \id{count} {\nvecopenmpdev} vectors.
}
{
 N\_Vector *N\_VCloneVectorArray\_OpenMPDEV(int count, N\_Vector w)
}
%%--------------------------------------
\sunmodfun{N\_VCloneVectorArrayEmpty\_OpenMPDEV}
{
 This function creates (by cloning) an array of \id{count} {\nvecopenmpdev} vectors, each with an
 empty (\id{NULL}) data array.
}
{
 N\_Vector *N\_VCloneVectorArrayEmpty\_OpenMPDEV(int count, N\_Vector w)
}
%%--------------------------------------
\sunmodfun{N\_VDestroyVectorArray\_OpenMPDEV}
{
 This function frees memory allocated for the array of \id{count} variables of type
 \id{N\_Vector} created with \id{N\_VCloneVectorArray\_OpenMPDEV} or with \newline
 \id{N\_VCloneVectorArrayEmpty\_OpenMPDEV}.
}
{
 void N\_VDestroyVectorArray\_OpenMPDEV(N\_Vector *vs, int count)
}
%%--------------------------------------
\sunmodfun{N\_VGetHostArrayPointer\_OpenMPDEV}
{
 This function returns a pointer to the host data array.
}
{
 realtype *N\_VGetHostArrayPointer\_OpenMPDEV(N\_Vector v)
}
%%--------------------------------------
\sunmodfun{N\_VGetDeviceArrayPointer\_OpenMPDEV}
{
 This function returns a pointer to the device data array.
}
{
 realtype *N\_VGetDeviceArrayPointer\_OpenMPDEV(N\_Vector v)
}
%%--------------------------------------
\sunmodfun{N\_VPrint\_OpenMPDEV}
{
 This function prints the content of an {\nvecopenmpdev} vector to \id{stdout}.
}
{
 void N\_VPrint\_OpenMPDEV(N\_Vector v)
}
%%--------------------------------------
\sunmodfun{N\_VPrintFile\_OpenMPDEV}
{
 This function prints the content of an {\nvecopenmpdev} vector to \id{outfile}.
}
{
 void N\_VPrintFile\_OpenMPDEV(N\_Vector v, FILE *outfile)
}
%%--------------------------------------
\sunmodfun{N\_VCopyToDevice\_OpenMPDEV}
{
 This function copies the content of an {\nvecopenmpdev} vector's host data array
 to the device data array.
}
{
 void N\_VCopyToDevice\_OpenMPDEV(N\_Vector v)
}
%%--------------------------------------
\sunmodfun{N\_VCopyFromDevice\_OpenMPDEV}
{
 This function copies the content of an {\nvecopenmpdev} vector's device data array
 to the host data array.
}
{
  void N\_VCopyFromDevice\_OpenMPDEV(N\_Vector v)
}
%%--------------------------------------

By default all fused and vector array operations are disabled in the {\nvecopenmpdev}
module. The following additional user-callable routines are provided to
enable or disable fused and vector array operations for a specific vector. To
ensure consistency across vectors it is recommended to first create a vector
with \id{N\_VNew\_OpenMPDEV}, enable/disable the desired operations for that vector
with the functions below, and create any additional vectors from that vector
using \id{N\_VClone}. This guarantees the new vectors will have the same
operations enabled/disabled as cloned vectors inherit the same enable/disable
options as the vector they are cloned from while vectors created with
\id{N\_VNew\_OpenMPDEV} will have the default settings for the {\nvecopenmpdev} module.
%%--------------------------------------
\sunmodfun{N\_VEnableFusedOps\_OpenMPDEV}
{
  This function enables (\id{SUNTRUE}) or disables (\id{SUNFALSE}) all fused and
  vector array operations in the {\nvecopenmpdev} vector. The return value is \id{0} for
  success and \id{-1} if the input vector or its \id{ops} structure are \id{NULL}.
}
{
  int N\_VEnableFusedOps\_OpenMPDEV(N\_Vector v, booleantype tf)
}
%%--------------------------------------
\sunmodfun{N\_VEnableLinearCombination\_OpenMPDEV}
{
  This function enables (\id{SUNTRUE}) or disables (\id{SUNFALSE}) the linear
  combination fused operation in the {\nvecopenmpdev} vector. The return value is \id{0} for
  success and \id{-1} if the input vector or its \id{ops} structure are \id{NULL}.
}
{
  int N\_VEnableLinearCombination\_OpenMPDEV(N\_Vector v, booleantype tf)
}
%%--------------------------------------
\sunmodfun{N\_VEnableScaleAddMulti\_OpenMPDEV}
{
  This function enables (\id{SUNTRUE}) or disables (\id{SUNFALSE}) the scale and
  add a vector to multiple vectors fused operation in the {\nvecopenmpdev} vector. The
  return value is \id{0} for success and \id{-1} if the input vector or its
  \id{ops} structure are \id{NULL}.
}
{
  int N\_VEnableScaleAddMulti\_OpenMPDEV(N\_Vector v, booleantype tf)
}
%%--------------------------------------
\sunmodfun{N\_VEnableDotProdMulti\_OpenMPDEV}
{
  This function enables (\id{SUNTRUE}) or disables (\id{SUNFALSE}) the multiple
  dot products fused operation in the {\nvecopenmpdev} vector. The return value is \id{0}
  for success and \id{-1} if the input vector or its \id{ops} structure are
  \id{NULL}.
}
{
  int N\_VEnableDotProdMulti\_OpenMPDEV(N\_Vector v, booleantype tf)
}
%%--------------------------------------
\sunmodfun{N\_VEnableLinearSumVectorArray\_OpenMPDEV}
{
  This function enables (\id{SUNTRUE}) or disables (\id{SUNFALSE}) the linear sum
  operation for vector arrays in the {\nvecopenmpdev} vector. The return value is \id{0} for
  success and \id{-1} if the input vector or its \id{ops} structure are \id{NULL}.
}
{
  int N\_VEnableLinearSumVectorArray\_OpenMPDEV(N\_Vector v, booleantype tf)
}
%%--------------------------------------
\sunmodfun{N\_VEnableScaleVectorArray\_OpenMPDEV}
{
  This function enables (\id{SUNTRUE}) or disables (\id{SUNFALSE}) the scale
  operation for vector arrays in the {\nvecopenmpdev} vector. The return value is \id{0} for
  success and \id{-1} if the input vector or its \id{ops} structure are \id{NULL}.
}
{
  int N\_VEnableScaleVectorArray\_OpenMPDEV(N\_Vector v, booleantype tf)
}
%%--------------------------------------
\sunmodfun{N\_VEnableConstVectorArray\_OpenMPDEV}
{
  This function enables (\id{SUNTRUE}) or disables (\id{SUNFALSE}) the const
  operation for vector arrays in the {\nvecopenmpdev} vector. The return value is \id{0} for
  success and \id{-1} if the input vector or its \id{ops} structure are \id{NULL}.
}
{
  int N\_VEnableConstVectorArray\_OpenMPDEV(N\_Vector v, booleantype tf)
}
%%--------------------------------------
\sunmodfun{N\_VEnableWrmsNormVectorArray\_OpenMPDEV}
{
  This function enables (\id{SUNTRUE}) or disables (\id{SUNFALSE}) the WRMS norm
  operation for vector arrays in the {\nvecopenmpdev} vector. The return value is \id{0} for
  success and \id{-1} if the input vector or its \id{ops} structure are \id{NULL}.
}
{
  int N\_VEnableWrmsNormVectorArray\_OpenMPDEV(N\_Vector v, booleantype tf)
}
%%--------------------------------------
\sunmodfun{N\_VEnableWrmsNormMaskVectorArray\_OpenMPDEV}
{
  This function enables (\id{SUNTRUE}) or disables (\id{SUNFALSE}) the masked WRMS
  norm operation for vector arrays in the {\nvecopenmpdev} vector. The return value is
  \id{0} for success and \id{-1} if the input vector or its \id{ops} structure are
  \id{NULL}.
}
{
  int N\_VEnableWrmsNormMaskVectorArray\_OpenMPDEV(N\_Vector v,
  \newlinefill{int N\_VEnableWrmsNormMaskVectorArray\_OpenMPDEV}
  booleantype tf)
}
%%--------------------------------------
\sunmodfun{N\_VEnableScaleAddMultiVectorArray\_OpenMPDEV}
{
  This function enables (\id{SUNTRUE}) or disables (\id{SUNFALSE}) the scale and
  add a vector array to multiple vector arrays operation in the {\nvecopenmpdev} vector. The
  return value is \id{0} for success and \id{-1} if the input vector or its
  \id{ops} structure are \id{NULL}.
}
{
  int N\_VEnableScaleAddMultiVectorArray\_OpenMPDEV(N\_Vector v,
  \newlinefill{int N\_VEnableScaleAddMultiVectorArray\_OpenMPDEV}
  booleantype tf)
}
%%--------------------------------------
\sunmodfun{N\_VEnableLinearCombinationVectorArray\_OpenMPDEV}
{
  This function enables (\id{SUNTRUE}) or disables (\id{SUNFALSE}) the linear
  combination operation for vector arrays in the {\nvecopenmpdev} vector. The return value
  is \id{0} for success and \id{-1} if the input vector or its \id{ops} structure
  are \id{NULL}.
}
{
  int N\_VEnableLinearCombinationVectorArray\_OpenMPDEV(N\_Vector v,
  \newlinefill{int N\_VEnableLinearCombinationVectorArray\_OpenMPDEV}
  booleantype tf)
}
%%
%%------------------------------------
%%
\paragraph{\bf Notes}

\begin{itemize}

\item
  When looping over the components of an \id{N\_Vector} \id{v}, it is
  most efficient to first obtain the component array via
  \id{h\_data = NV\_DATA\_HOST\_OMPDEV(v)} for the host array or \newline
  \id{d\_data = NV\_DATA\_DEV\_OMPDEV(v)} for the device array and then access
  \id{h\_data[i]} or \id{d\_data[i]} within the loop.

\item
  When accessing individual components of an \id{N\_Vector} \id{v} on
  the host remember to first copy the array
  back from the device with \id{N\_VCopyFromDevice\_OpenMPDEV(v)}
  to ensure the array is up to date.

\item
  {\warn}\id{N\_VNewEmpty\_OpenMPDEV}, \id{N\_VMake\_OpenMPDEV},
  and \id{N\_VCloneVectorArrayEmpty\_OpenMPDEV} set the field
  \id{own\_data} $=$ \id{SUNFALSE}.
  \id{N\_VDestroy\_OpenMPDEV} and \id{N\_VDestroyVectorArray\_OpenMPDEV}
  will not attempt to free the pointer {\em data} for any \id{N\_Vector} with
  \id{own\_data} set to \id{SUNFALSE}. In such a case, it is the user's responsibility to
  deallocate the {\em data} pointer.

\item
  {\warn}To maximize efficiency, vector operations in the {\nvecopenmpdev} implementation
  that have more than one \id{N\_Vector} argument do not check for
  consistent internal representation of these vectors. It is the user's
  responsibility to ensure that such routines are called with \id{N\_Vector}
  arguments that were all created with the same internal representations.

\end{itemize}

% This is a shared SUNDIALS TEX file with description of
% the Trilinos nvector wrapper implementation
%
\section{The NVECTOR\_TRILINOS implementation}\label{ss:nvec_trilinos}

The {\nvectrilinos} module is an {\nvector} wrapper around the {\trilinos}
\href{https://github.com/trilinos/Trilinos}{Tpetra} vector. The interface
to Tpetra is implemented in the \id{Sundials::TpetraVectorInterface} class. This
class simply stores a reference counting pointer to a Tpetra vector and
inherits from an empty structure
%%
\begin{verbatim}
struct _N_VectorContent_Trilinos {};
\end{verbatim}
%%
%%--------------------------------------------
to interface the C++ class with the {\nvector} C code.
A pointer to an instance of this class is kept in the \id{content} field
of the \id{N\_Vector} object, to ensure that the Tpetra vector
is not deleted for as long as the \id{N\_Vector} object exists.

The Tpetra vector type in the \id{Sundials::TpetraVectorInterface} class is defined
as:
\begin{verbatim}
  typedef Tpetra::Vector<realtype, sunindextype, sunindextype> vector_type;
\end{verbatim}
The Tpetra vector will use the {\sundials}-specified \id{realtype} as its scalar
type, and it will use \id{sunindextype} as the global and the local ordinal types.
This type definition will use Tpetra's default node type. Available Kokkos node
types in {\trilinos} 12.14 release are serial (single thread), OpenMP, Pthread,
and {\cuda}. The default node type is selected when building the Kokkos package.
For example, the Tpetra vector will use a {\cuda} node if Tpetra was built with
{\cuda} support and the {\cuda} node was selected as the default when Tpetra was
built.

The header file to include when using this module is \id{nvector\_trilinos.h}.
The installed module library to link to is
\id{libsundials\_nvectrilinos.\textit{lib}}
where \id{\em.lib} is typically \id{.so} for shared libraries and \id{.a}
for static libraries.


% ====================================================================
\subsection{NVECTOR\_TRILINOS functions}
\label{ss:nvec_trilinos_functions}
% ====================================================================

The {\nvectrilinos} module defines implementations of all vector operations listed
in Tables \ref{t:nvecops}, \ref{t:nveclocalops}, and
\ref{t:nveclocalops}, except for \verb|N_VGetArrayPointer| and
\verb|N_VSetArrayPointer|. As such, this vector cannot be used with
{\sundials} Fortran interfaces, nor with the {\sundials} direct
solvers and preconditioners. When access to raw vector data is needed, it is
recommended to extract the {\trilinos} Tpetra vector first, and then use Tpetra vector
methods to access the data. Usage examples of {\nvectrilinos} are provided in
example programs for {\ida} \cite{ida_ex}.

The names of vector operations are obtained from those in
Tables \ref{t:nvecops}, \ref{t:nveclocalops}, and \ref{t:nveclocalops} by appending the
suffix \id{\_Trilinos} (e.g. \id{N\_VDestroy\_Trilinos}).
Vector operations call existing \id{Tpetra::Vector} methods when available. Vector
operations specific to {\sundials} are implemented as standalone functions in the namespace
\id{Sundials::TpetraVector}, located in the file \id{SundialsTpetraVectorKernels.hpp}.
The module {\nvectrilinos} provides the following additional user-callable functions:
%%
%%
\begin{itemize}


%%--------------------------------------

\item \ID{N\_VGetVector\_Trilinos}

  This C++ function takes an \id{N\_Vector} as the argument and returns a reference
  counting pointer to the underlying Tpetra vector. This is a standalone function
  defined in the global namespace.

\begin{verbatim}
Teuchos::RCP<vector_type> N_VGetVector_Trilinos(N_Vector v);
\end{verbatim}


%%--------------------------------------

\item \ID{N\_VMake\_Trilinos}

  This C++ function creates and allocates memory for an {\nvectrilinos}
  wrapper around a user-provided Tpetra vector. This is a standalone function
  defined in the global namespace.

\begin{verbatim}
N_Vector N_VMake_Trilinos(Teuchos::RCP<vector_type> v);
\end{verbatim}


\end{itemize}
%%
%%------------------------------------
%%
\paragraph{\bf Notes}

\begin{itemize}

\item

  The template parameter \id{vector\_type} should be set as:\\
  \verb|  typedef Sundials::TpetraVectorInterface::vector_type vector_type|\\
   This will ensure that data types used in Tpetra vector match those in {\sundials}.

\item
  When there is a need to access components of an \id{N\_Vector\_Trilinos}, \id{v},
  it is recommeded to extract the {\trilinos} vector object via
  \id{x\_vec = N\_VGetVector\_Trilinos(v)} and then access components using
  the appropriate {\trilinos} functions.

\item
  The functions \id{N\_VDestroy\_Trilinos} and \id{N\_VDestroyVectorArray\_Trilinos}
  only delete the \id{N\_Vector} wrapper. The underlying Tpetra vector object will exist for as long as
  there is at least one reference to it.

\end{itemize}

% This is a shared SUNDIALS TEX file with description of
% the ManyVector nvector implementation
%
\section{The NVECTOR\_MANYVECTOR implementation}\label{ss:nvec_manyvector}

The {\nvecmanyvector} implementation of the {\nvector} module provided
with {\sundials} is designed to facilitate problems with an inherent
data partitioning for the solution vector within a computational node.
These data partitions are entirely user-defined, through construction
of distinct {\nvector} modules for each component, that are then
combined together to form the {\nvecmanyvector}.  We envision two
generic use cases for this implementation:
\begin{itemize}
\item[A.] \emph{Heterogeneous computational architectures}: for users
  who wish to partition data on a node between different computing
  resources, they may create architecture-specific subvectors for each
  partition.  For example, a user could create one serial component
  based on {\nvecs}, another component for GPU accelerators based on
  {\nveccuda}, and another threaded component based on {\nvecopenmp}.
\item[B.] \emph{Structure of arrays (SOA) data layouts}: for users who
  wish to create separate subvectors for each solution component,
  e.g., in a Navier-Stokes simulation they could have separate
  subvectors for density, velocities and pressure, which are combined
  together into a single {\nvecmanyvector} for the overall ``solution''.
\end{itemize}
We note that the above use cases are not mutually exclusive, and the
{\nvecmanyvector} implementation should support arbitrary combinations
of these cases.

The {\nvecmanyvector} implementation is designed to
work with any {\nvector} subvectors that implement the minimum
\emph{required} set of operations.
Additionally, {\nvecmanyvector} sets no limit on the number of
subvectors that may be attached (aside from the limitations of using
\id{sunindextype} for indexing, and standard per-node memory
limitations).  However, while this ostensibly supports subvectors
with one entry each (i.e., one subvector for each solution entry), we
anticipate that this extreme situation will hinder performance due to
non-stride-one memory accesses and increased function call overhead.
We therefore recommend a relatively coarse partitioning of the
problem, although actual performance will likely be
problem-dependent.

As a final note, in the coming years we plan to introduce additional
algebraic solvers and time integration modules that will leverage the
problem partitioning enabled by {\nvecmanyvector}.  However, even at
present we anticipate that users will be able to leverage such data
partitioning in their problem-defining ODE right-hand side, DAE
residual, or nonlinear solver residual functions.


% ====================================================================
\subsection{NVECTOR\_MANYVECTOR structure}
\label{ss:nvec_manyvector_structure}
% ====================================================================

The {\nvecmanyvector} implementation defines the {\em content} field
of \id{N\_Vector} to be a structure containing the number of
subvectors comprising the ManyVector, the global length of the
ManyVector (including all subvectors), a pointer to
the beginning of the array of subvectors, and a boolean flag
\id{own\_data} indicating ownership of the subvectors that populate
\id{subvec\_array}.
%%
\begin{verbatim}
struct _N_VectorContent_ManyVector {
  sunindextype  num_subvectors;  /* number of vectors attached      */
  sunindextype  global_length;   /* overall manyvector length       */
  N_Vector*     subvec_array;    /* pointer to N_Vector array       */
  booleantype   own_data;        /* flag indicating data ownership  */
};
\end{verbatim}
%%
%%--------------------------------------------

The header file to include when using this module is
\id{nvector\_manyvector.h}. The installed module library to link against is
\id{libsundials\_nvecmanyvector.\textit{lib}} where \id{\em.lib} is typically
\id{.so} for shared libraries and \id{.a} for static libraries.



% ====================================================================
\subsection{NVECTOR\_MANYVECTOR functions}
\label{ss:nvec_manyvector_functions}
% ====================================================================

The {\nvecmanyvector} module implements all vector operations listed
in Tables \ref{ss:nvecops}, \ref{ss:nvecfusedops}, \ref{ss:nvecarrayops},
and \ref{ss:nveclocalops}, except for \id{N\_VGetArrayPointer},
\id{N\_VSetArrayPointer}, \id{N\_VScaleAddMultiVectorArray}, and
\id{N\_VLinearCombinationVectorArray}.  As such, this vector cannot be
used with the {\sundials} Fortran-77 interfaces, nor with the
{\sundials} direct solvers and preconditioners. Instead, the \\
{\nvecmanyvector} module provides functions to access subvectors,
whose data may in turn be accessed according to their {\nvector}
implementations.

The names of vector operations are obtained from those in Tables
\ref{ss:nvecops}, \ref{ss:nvecfusedops}, \ref{ss:nvecarrayops}, and
\ref{ss:nveclocalops} by appending the suffix \id{\_ManyVector}
(e.g. \id{N\_VDestroy\_ManyVector}).
The module {\nvecmanyvector} provides the following additional
user-callable routines:
%%--------------------------------------
\sunmodfunf{N\_VNew\_ManyVector}
{
  This function creates a ManyVector from a set of existing {\nvector}
  objects.

  This routine will copy all \id{N\_Vector} pointers from the input
  \id{vec\_array}, so the user may modify/free that pointer array
  after calling this function.  However, this routine does \emph{not}
  allocate any new subvectors, so the underlying {\nvector} objects
  themselves should not be destroyed before the ManyVector that
  contains them.

  Upon successful completion, the new ManyVector is returned;
  otherwise this routine returns \id{NULL} (e.g., a memory allocation
  failure occurred).

  Users of the Fortran 2003 interface to this function will first need
  to use the generic \id{N\_Vector} utility functions
  \id{N\_VNewVectorArray}, and \id{N\_VSetVecAtIndexVectorArray} to create
  the \id{N\_Vector*} argument. This is further explained in 
  Chapter~\ref{sss:fortran2003_nvarrays}, and the functions are documented
  in Chapter~\ref{ss:nvecutils}.
}
{
  N\_Vector N\_VNew\_ManyVector(sunindextype num\_subvectors,
  \newlinefill{N\_Vector N\_VNew\_ManyVector}
  N\_Vector *vec\_array);
}
%%--------------------------------------
\sunmodfunf{N\_VGetSubvector\_ManyVector}
{
  This function returns the \id{vec\_num} subvector from the {\nvector}
  array.
}
{
  N\_Vector N\_VGetSubvector\_ManyVector(N\_Vector v, sunindextype vec\_num);
}
%%--------------------------------------
\sunmodfunf{N\_VGetSubvectorArrayPointer\_ManyVector}
{
  This function returns the data array pointer for the \id{vec\_num}
  subvector from the {\nvector} array.

  If the input \id{vec\_num} is invalid, or if the subvector does not
  support the \id{N\_VGetArrayPointer} operation, then \id{NULL} is returned.
}
{
  realtype *N\_VGetSubvectorArrayPointer\_ManyVector(N\_Vector v, sunindextype vec\_num);
}
%%--------------------------------------
\sunmodfunf{N\_VSetSubvectorArrayPointer\_ManyVector}
{
  This function sets the data array pointer for the \id{vec\_num}
  subvector from the {\nvector} array.

  If the input \id{vec\_num} is invalid, or if the subvector does not
  support the \id{N\_VSetArrayPointer} operation, then this routine
  returns \id{-1}; otherwise it returns \id{0}.
}
{
  int N\_VSetSubvectorArrayPointer\_ManyVector(realtype *v\_data, N\_Vector v, sunindextype vec\_num);
}
%%--------------------------------------
\sunmodfunf{N\_VGetNumSubvectors\_ManyVector}
{
  This function returns the overall number of subvectors in the
  ManyVector object.
}
{
  sunindextype N\_VGetNumSubvectors\_ManyVector(N\_Vector v);
}
%%--------------------------------------
By default all fused and vector array operations are disabled in the {\nvecmanyvector}
module, except for \id{N\_VWrmsNormVectorArray} and
\id{N\_VWrmsNormMaskVectorArray}, that are enabled by default. The
following additional user-callable routines are provided to enable or
disable fused and vector array operations for a specific vector. To
ensure consistency across vectors it is recommended to first create a
vector with \id{N\_VNew\_ManyVector},
enable/disable the desired operations for that vector with the
functions below, and create any additional vectors from that vector
using \id{N\_VClone}. This guarantees that the new vectors will have
the same operations enabled/disabled, since cloned vectors inherit
those configuration options from the vector they are cloned from, while
vectors created with \id{N\_VNew\_ManyVector}
will have the default settings for the
{\nvecmanyvector} module.  We note that these routines \emph{do not}
call the corresponding routines on subvectors, so those should be set up
as desired \emph{before} attaching them to the ManyVector in
\id{N\_VNew\_ManyVector}.
%%--------------------------------------
\sunmodfunf{N\_VEnableFusedOps\_ManyVector}
{
  This function enables (\id{SUNTRUE}) or disables (\id{SUNFALSE}) all fused and
  vector array operations in the ManyVector. The return value is \id{0} for
  success and \id{-1} if the input vector or its \id{ops} structure are \id{NULL}.
}
{
  int N\_VEnableFusedOps\_ManyVector(N\_Vector v, booleantype tf);
}
%%--------------------------------------
\sunmodfunf{N\_VEnableLinearCombination\_ManyVector}
{
  This function enables (\id{SUNTRUE}) or disables (\id{SUNFALSE}) the linear
  combination fused operation in the ManyVector. The return value is \id{0} for
  success and \id{-1} if the input vector or its \id{ops} structure are \id{NULL}.
}
{
  int N\_VEnableLinearCombination\_ManyVector(N\_Vector v, booleantype tf);
}
%%--------------------------------------
\sunmodfunf{N\_VEnableScaleAddMulti\_ManyVector}
{
  This function enables (\id{SUNTRUE}) or disables (\id{SUNFALSE}) the scale and
  add a vector to multiple vectors fused operation in the ManyVector. The
  return value is \id{0} for success and \id{-1} if the input vector or its
  \id{ops} structure are \id{NULL}.
}
{
  int N\_VEnableScaleAddMulti\_ManyVector(N\_Vector v, booleantype tf);
}
%%--------------------------------------
\sunmodfunf{N\_VEnableDotProdMulti\_ManyVector}
{
  This function enables (\id{SUNTRUE}) or disables (\id{SUNFALSE}) the multiple
  dot products fused operation in the ManyVector. The return value is \id{0}
  for success and \id{-1} if the input vector or its \id{ops} structure are
  \id{NULL}.
}
{
  int N\_VEnableDotProdMulti\_ManyVector(N\_Vector v, booleantype tf);
}
%%--------------------------------------
\sunmodfunf{N\_VEnableLinearSumVectorArray\_ManyVector}
{
  This function enables (\id{SUNTRUE}) or disables (\id{SUNFALSE}) the linear sum
  operation for vector arrays in the ManyVector. The return value is \id{0} for
  success and \id{-1} if the input vector or its \id{ops} structure are \id{NULL}.
}
{
  int N\_VEnableLinearSumVectorArray\_ManyVector(N\_Vector v, booleantype tf);
}
%%--------------------------------------
\sunmodfunf{N\_VEnableScaleVectorArray\_ManyVector}
{
  This function enables (\id{SUNTRUE}) or disables (\id{SUNFALSE}) the scale
  operation for vector arrays in the ManyVector. The return value is \id{0} for
  success and \id{-1} if the input vector or its \id{ops} structure are \id{NULL}.
}
{
  int N\_VEnableScaleVectorArray\_ManyVector(N\_Vector v, booleantype tf);
}
%%--------------------------------------
\sunmodfunf{N\_VEnableConstVectorArray\_ManyVector}
{
  This function enables (\id{SUNTRUE}) or disables (\id{SUNFALSE}) the const
  operation for vector arrays in the ManyVector. The return value is \id{0} for
  success and \id{-1} if the input vector or its \id{ops} structure are \id{NULL}.
}
{
  int N\_VEnableConstVectorArray\_ManyVector(N\_Vector v, booleantype tf);
}
%%--------------------------------------
\sunmodfunf{N\_VEnableWrmsNormVectorArray\_ManyVector}
{
  This function enables (\id{SUNTRUE}) or disables (\id{SUNFALSE}) the WRMS norm
  operation for vector arrays in the ManyVector. The return value is \id{0} for
  success and \id{-1} if the input vector or its \id{ops} structure are \id{NULL}.
}
{
  int N\_VEnableWrmsNormVectorArray\_ManyVector(N\_Vector v, booleantype tf);
}
%%--------------------------------------
\sunmodfunf{N\_VEnableWrmsNormMaskVectorArray\_ManyVector}
{
  This function enables (\id{SUNTRUE}) or disables (\id{SUNFALSE}) the masked WRMS
  norm operation for vector arrays in the ManyVector. The return value is
  \id{0} for success and \id{-1} if the input vector or its \id{ops} structure are
  \id{NULL}.
}
{
  int N\_VEnableWrmsNormMaskVectorArray\_ManyVector(N\_Vector v, booleantype tf);
}
%%
%%------------------------------------
%%
\paragraph{\bf Notes}

\begin{itemize}

\item
  {\warn}\id{N\_VNew\_ManyVector} sets the field {\em own\_data} $=$
  \id{SUNFALSE}.  \id{N\_VDestroy\_ManyVector} will not attempt to call
  \id{N\_VDestroy} on any subvectors contained in the subvector array
  for any \id{N\_Vector} with {\em own\_data} set to \id{SUNFALSE}. In
  such a case, it is the user's responsibility to deallocate the
  subvectors.

\item
  {\warn}To maximize efficiency, arithmetic vector operations in the
  {\nvecmanyvector} implementation that have more than one
  \id{N\_Vector} argument do not check for consistent internal
  representation of these vectors. It is the user's responsibility to
  ensure that such routines are called with \id{N\_Vector} arguments
  that were all created with the same subvector representations.

\end{itemize}


% This is a shared SUNDIALS TEX file with description of
% the MPIManyVector nvector implementation
%
\section{The NVECTOR\_MPIMANYVECTOR implementation}\label{ss:nvec_mpimanyvector}

The {\nvecmpimanyvector} implementation of the {\nvector} module provided
with {\sundials} is designed to facilitate problems with an inherent
data partitioning for the solution vector, and when using
distributed-memory parallel architectures.  As such, the MPIManyVector
implementation supports all use cases allowed by the MPI-unaware
ManyVector implementation, as well as partitioning data between nodes
in a parallel environment.  These data partitions are entirely
user-defined, through construction of distinct {\nvector} modules for
each component, that are then combined together to form the
{\nvecmpimanyvector}.  We envision three generic use cases for this
implementation:
\begin{itemize}
\item[A.] \emph{Heterogeneous computational architectures (single-node or
  multi-node)}: for users who wish to partition data on a node between
  different computing resources, they may create architecture-specific
  subvectors for each partition.  For example, a user could create one
  MPI-parallel component based on {\nvecp}, another single-node
  component for GPU accelerators based on {\nveccuda}, and another
  threaded single-node component based on {\nvecopenmp}.
\item[B.] \emph{Process-based multiphysics decompositions (multi-node)}: for
  users who wish to combine separate simulations together, e.g., where
  one subvector resides on one subset of MPI processes, while another
  subvector resides on a different subset of MPI processes, and where
  the user has created a MPI \emph{intercommunicator} to connect these
  distinct process sets together.
\item[C.] \emph{Structure of arrays (SOA) data layouts (single-node or
  multi-node)}: for users who wish to create separate subvectors for
  each solution component, e.g., in a Navier-Stokes simulation they
  could have separate subvectors for density, velocities and
  pressure, which are combined together into a single
  {\nvecmpimanyvector} for the overall ``solution''. 
\end{itemize}
We note that the above use cases are not mutually exclusive, and the
{\nvecmpimanyvector} implementation should support arbitrary combinations
of these cases.

The {\nvecmpimanyvector} implementation is designed to work with any
{\nvector} subvectors that implement the minimum \emph{required} set
of operations, however significant performance benefits may be
obtained when subvectors additionally implement the optional local
reduction operations listed in Table \ref{t:nveclocalops}.

Additionally, {\nvecmpimanyvector} sets no limit on the number of
subvectors that may be attached (aside from the limitations of using
\id{sunindextype} for indexing, and standard per-node memory
limitations).  However, while this ostensibly supports subvectors
with one entry each (i.e., one subvector for each solution entry), we
anticipate that this extreme situation will hinder performance due to
non-stride-one memory accesses and increased function call overhead.
We therefore recommend a relatively coarse partitioning of the
problem, although actual performance will likely be
problem-dependent.

As a final note, in the coming years we plan to introduce additional
algebraic solvers and time integration modules that will leverage the
problem partitioning enabled by {\nvecmpimanyvector}.  However, even at
present we anticipate that users will be able to leverage such data
partitioning in their problem-defining ODE right-hand side, DAE
residual, or nonlinear solver residual functions.


% ====================================================================
\subsection{NVECTOR\_MPIMANYVECTOR structure}
\label{ss:nvec_mpimanyvector_structure}
% ====================================================================


The {\nvecmpimanyvector} implementation defines the {\em content} field
of \id{N\_Vector} to be a structure containing the MPI communicator
(or \id{MPI\_COMM\_NULL} if running on a single-node), the number of
subvectors comprising the MPIManyVector, the global length of the
MPIManyVector (including all subvectors on all MPI tasks), a pointer to
the beginning of the array of subvectors, and a boolean flag
\id{own\_data} indicating ownership of the subvectors that populate
\id{subvec\_array}.
%%
\begin{verbatim} 
struct _N_VectorContent_MPIManyVector {
  MPI_Comm      comm;            /* overall MPI communicator        */
  sunindextype  num_subvectors;  /* number of vectors attached      */
  sunindextype  global_length;   /* overall mpimanyvector length    */
  N_Vector*     subvec_array;    /* pointer to N_Vector array       */
  booleantype   own_data;        /* flag indicating data ownership  */
};
\end{verbatim}
%%
%%--------------------------------------------

The header file to include when using this module is
\id{nvector\_mpimanyvector.h}. The installed module library to link against is
\id{libsundials\_nvecmpimanyvector.\textit{lib}} where \id{\em.lib} is typically
\id{.so} for shared libraries and \id{.a} for static libraries.

\warn\textbf{Note:} If {\sundials} is configured with MPI disabled, then the
MPIManyVector library will not be built.  Furthermore, any user codes
that include \id{nvector\_mpimanyvector.h} \emph{must} be compiled
using an MPI-aware compiler (whether the specific user code utilizes
MPI or not).  We note that the {\nvecmanyvector} implementation is
designed for ManyVector use cases in an MPI-unaware environment.


% ====================================================================
\subsection{NVECTOR\_MPIMANYVECTOR functions}
\label{ss:nvec_mpimanyvector_functions}
% ====================================================================

The {\nvecmpimanyvector} module implements all vector operations listed 
in Tables \ref{t:nvecops}, \ref{t:nvecfusedops}, \ref{t:nvecarrayops},
and \ref{t:nveclocalops}, except for \id{N\_VGetArrayPointer},
\id{N\_VSetArrayPointer}, \id{N\_VScaleAddMultiVectorArray}, and
\id{N\_VLinearCombinationVectorArray}.  As such, this vector cannot be
used with the {\sundials} Fortran-77 interfaces, nor with the
{\sundials} direct solvers and preconditioners. Instead, the \\
{\nvecmpimanyvector} module provides functions to access subvectors,
whose data may in turn be accessed according to their {\nvector}
implementations.

The names of vector operations are obtained from those in Tables
\ref{t:nvecops}, \ref{t:nvecfusedops}, \ref{t:nvecarrayops}, and
\ref{t:nveclocalops} by appending the suffix \id{\_MPIManyVector} 
(e.g. \id{N\_VDestroy\_MPIManyVector}).
The module {\nvecmpimanyvector} provides the following additional
user-callable routines:
%%--------------------------------------
\sunmodfun{N\_VNew\_MPIManyVector}
{
  This function creates an MPIManyVector from a set of existing {\nvector}
  objects, under the requirement that all MPI-aware subvectors use the
  same MPI communicator (this is checked internally).  If none of the
  subvectors are MPI-aware, then this may equivalently be used to
  describe data partitioning within a single node.  We note that this
  routine is designed to support use cases A and C above.

  This routine will copy all \id{N\_Vector} pointers from the input
  \id{vec\_array}, so the user may modify/free that pointer array
  after calling this function.  However, this routine does \emph{not}
  allocate any new subvectors, so the underlying {\nvector} objects
  themselves should not be destroyed before the MPIManyVector that
  contains them.

  Upon successful completion, the new MPIManyVector is returned;
  otherwise this routine returns \id{NULL} (e.g., if two MPI-aware
  subvectors use different MPI communicators).
}
{
  N\_Vector N\_VNew\_MPIManyVector(sunindextype num\_subvectors,
  \newlinefill{N\_Vector N\_VNew\_MPIManyVector}
  N\_Vector *vec\_array);
}
%%--------------------------------------
\sunmodfun{N\_VMake\_MPIManyVector}
{
  This function creates an MPIManyVector from a set of existing {\nvector}
  objects, and a user-created MPI communicator that ``connects'' these
  subvectors.  Any MPI-aware subvectors may use different MPI
  communicators than the input \id{comm}.  We note that this routine
  is designed to support any combination of the use cases above.

  The input \id{comm} should be the memory reference to this
  user-created MPI communicator.  We note that since many {\mpi}
  implementations \id{\#define} \id{MPI\_COMM\_WORLD} to be a specific
  integer \emph{value} (that has no memory reference), users who wish
  to supply \id{MPI\_COMM\_WORLD} to this routine should first
  set a specific \id{MPI\_Comm} variable to \id{MPI\_COMM\_WORLD}
  before passing in the reference, e.g.

  \hspace{0.5in} \texttt{MPI\_Comm comm;}\vspace{-0.5em}
  
  \hspace{0.5in} \texttt{comm = MPI\_COMM\_WORLD;}\vspace{-0.5em}
  
  \hspace{0.5in} \texttt{N\_Vector x;}\vspace{-0.5em}
  
  \hspace{0.5in} \texttt{x = N\_VMake\_MPIManyVector(\&comm, ...);}

  This routine will internally call \id{MPI\_Comm\_dup} to create a
  copy of the input \id{comm}, so the user-supplied \id{comm} argument
  need not be retained after the call to \id{N\_VMake\_MPIManyVector}.

  If all subvectors are MPI-unaware, then the input \id{comm} argument
  should be \id{NULL}, although in this case, it would be simpler to
  call \Id{N\_VNew\_MPIManyVector} instead.
  
  This routine will copy all \id{N\_Vector} pointers from the input
  \id{vec\_array}, so the user may modify/free that pointer array
  after calling this function.  However, this routine does \emph{not}
  allocate any new subvectors, so the underlying {\nvector} objects
  themselves should not be destroyed before the MPIManyVector that
  contains them.

  Upon successful completion, the new MPIManyVector is returned;
  otherwise this routine returns \id{NULL} (e.g., if the input
  \id{vec\_array} is \id{NULL}).
}
{
  N\_Vector N\_VMake\_MPIManyVector(MPI\_Comm *comm, 
  sunindextype num\_subvectors,
  \newlinefill{N\_Vector N\_VMake\_MPIManyVector}
  N\_Vector *vec\_array);
}
%%--------------------------------------
\sunmodfun{N\_VGetSubvector\_MPIManyVector}
{
  This function returns the \id{vec\_num} subvector from the {\nvector}
  array.
}
{
  N\_Vector N\_VGetSubvector\_MPIManyVector(N\_Vector v, sunindextype vec\_num);
}
%%--------------------------------------
\sunmodfun{N\_VGetSubvectorArrayPointer\_MPIManyVector}
{
  This function returns the data array pointer for the \id{vec\_num}
  subvector from the {\nvector} array.

  If the input \id{vec\_num} is invalid, or if the subvector does not
  support the \id{N\_VGetArrayPointer} operation, then \id{NULL} is returned.
}
{
  realtype *N\_VGetSubvectorArrayPointer\_MPIManyVector(N\_Vector v, sunindextype vec\_num);
}
%%--------------------------------------
\sunmodfun{N\_VSetSubvectorArrayPointer\_MPIManyVector}
{
  This function sets the data array pointer for the \id{vec\_num}
  subvector from the {\nvector} array.

  If the input \id{vec\_num} is invalid, or if the subvector does not
  support the \id{N\_VSetArrayPointer} operation, then this routine
  returns \id{-1}; otherwise it returns \id{0}.
}
{
  int N\_VSetSubvectorArrayPointer\_MPIManyVector(realtype *v\_data, N\_Vector v, sunindextype vec\_num);
}
%%--------------------------------------
\sunmodfun{N\_VGetNumSubvectors\_MPIManyVector}
{
  This function returns the overall number of subvectors in the
  MPIManyVector object.
}
{
  sunindextype N\_VGetNumSubvectors\_MPIManyVector(N\_Vector v);
}
%%--------------------------------------
By default all fused and vector array operations are disabled in the {\nvecmpimanyvector}
module, except for \id{N\_VWrmsNormVectorArray} and
\id{N\_VWrmsNormMaskVectorArray}, that are enabled by default. The
following additional user-callable routines are provided to enable or
disable fused and vector array operations for a specific vector. To
ensure consistency across vectors it is recommended to first create a
vector with \id{N\_VNew\_MPIManyVector} or \id{N\_VMake\_MPIManyVector},
enable/disable the desired operations for that vector with the
functions below, and create any additional vectors from that vector
using \id{N\_VClone}. This guarantees that the new vectors will have
the same operations enabled/disabled, since cloned vectors inherit
those configuration options from the vector they are cloned from, while
vectors created with \id{N\_VNew\_MPIManyVector} and
\id{N\_VMake\_MPIManyVector} will have the default settings for the
{\nvecmpimanyvector} module.  We note that these routines \emph{do not} 
call the corresponding routines on subvectors, so those should be set up
as desired \emph{before} attaching them to the MPIManyVector in
\id{N\_VNew\_MPIManyVector} or \id{N\_VMake\_MPIManyVector}.
%%--------------------------------------
\sunmodfun{N\_VEnableFusedOps\_MPIManyVector}
{
  This function enables (\id{SUNTRUE}) or disables (\id{SUNFALSE}) all fused and
  vector array operations in the MPIManyVector. The return value is \id{0} for
  success and \id{-1} if the input vector or its \id{ops} structure are \id{NULL}.
}
{
  int N\_VEnableFusedOps\_MPIManyVector(N\_Vector v, booleantype tf);
}
%%--------------------------------------
\sunmodfun{N\_VEnableLinearCombination\_MPIManyVector}
{
  This function enables (\id{SUNTRUE}) or disables (\id{SUNFALSE}) the linear
  combination fused operation in the MPIManyVector. The return value is \id{0} for
  success and \id{-1} if the input vector or its \id{ops} structure are \id{NULL}.
}
{
  int N\_VEnableLinearCombination\_MPIManyVector(N\_Vector v, booleantype tf);
}
%%--------------------------------------
\sunmodfun{N\_VEnableScaleAddMulti\_MPIManyVector}
{
  This function enables (\id{SUNTRUE}) or disables (\id{SUNFALSE}) the scale and
  add a vector to multiple vectors fused operation in the MPIManyVector. The
  return value is \id{0} for success and \id{-1} if the input vector or its
  \id{ops} structure are \id{NULL}.
}
{
  int N\_VEnableScaleAddMulti\_MPIManyVector(N\_Vector v, booleantype tf);
}
%%--------------------------------------
\sunmodfun{N\_VEnableDotProdMulti\_MPIManyVector}
{
  This function enables (\id{SUNTRUE}) or disables (\id{SUNFALSE}) the multiple
  dot products fused operation in the MPIManyVector. The return value is \id{0}
  for success and \id{-1} if the input vector or its \id{ops} structure are
  \id{NULL}.
}
{
  int N\_VEnableDotProdMulti\_MPIManyVector(N\_Vector v, booleantype tf);
}
%%--------------------------------------
\sunmodfun{N\_VEnableLinearSumVectorArray\_MPIManyVector}
{
  This function enables (\id{SUNTRUE}) or disables (\id{SUNFALSE}) the linear sum
  operation for vector arrays in the MPIManyVector. The return value is \id{0} for
  success and \id{-1} if the input vector or its \id{ops} structure are \id{NULL}.
}
{
  int N\_VEnableLinearSumVectorArray\_MPIManyVector(N\_Vector v, booleantype tf);
}
%%--------------------------------------
\sunmodfun{N\_VEnableScaleVectorArray\_MPIManyVector}
{
  This function enables (\id{SUNTRUE}) or disables (\id{SUNFALSE}) the scale
  operation for vector arrays in the MPIManyVector. The return value is \id{0} for
  success and \id{-1} if the input vector or its \id{ops} structure are \id{NULL}.
}
{
  int N\_VEnableScaleVectorArray\_MPIManyVector(N\_Vector v, booleantype tf);
}
%%--------------------------------------
\sunmodfun{N\_VEnableConstVectorArray\_MPIManyVector}
{
  This function enables (\id{SUNTRUE}) or disables (\id{SUNFALSE}) the const
  operation for vector arrays in the MPIManyVector. The return value is \id{0} for
  success and \id{-1} if the input vector or its \id{ops} structure are \id{NULL}.
}
{
  int N\_VEnableConstVectorArray\_MPIManyVector(N\_Vector v, booleantype tf);
}
%%--------------------------------------
\sunmodfun{N\_VEnableWrmsNormVectorArray\_MPIManyVector}
{
  This function enables (\id{SUNTRUE}) or disables (\id{SUNFALSE}) the WRMS norm
  operation for vector arrays in the MPIManyVector. The return value is \id{0} for
  success and \id{-1} if the input vector or its \id{ops} structure are \id{NULL}.
}
{
  int N\_VEnableWrmsNormVectorArray\_MPIManyVector(N\_Vector v, booleantype tf);
}
%%--------------------------------------
\sunmodfun{N\_VEnableWrmsNormMaskVectorArray\_MPIManyVector}
{
  This function enables (\id{SUNTRUE}) or disables (\id{SUNFALSE}) the masked WRMS
  norm operation for vector arrays in the MPIManyVector. The return value is
  \id{0} for success and \id{-1} if the input vector or its \id{ops} structure are
  \id{NULL}.
}
{
  int N\_VEnableWrmsNormMaskVectorArray\_MPIManyVector(N\_Vector v, booleantype tf);
}
%%
%%------------------------------------
%%
\paragraph{\bf Notes} 
           
\begin{itemize}
                                        
\item
  {\warn}\id{N\_VNew\_MPIManyVector} and \id{N\_VMake\_MPIManyVector} set
  the field {\em own\_data} $=$ \id{SUNFALSE}.  \\
  \id{N\_VDestroy\_MPIManyVector} will not attempt to call
  \id{N\_VDestroy} on any subvectors contained in the subvector array
  for any \id{N\_Vector} with {\em own\_data} set to \id{SUNFALSE}. In
  such a case, it is the user's responsibility to deallocate the
  subvectors.

\item
  {\warn}To maximize efficiency, arithmetic vector operations in the
  {\nvecmpimanyvector} implementation that have more than one
  \id{N\_Vector} argument do not check for consistent internal
  representation of these vectors. It is the user's responsibility to
  ensure that such routines are called with \id{N\_Vector} arguments
  that were all created with the same subvector representations.

\end{itemize}


% This is a shared SUNDIALS TEX file with description of
% the mpiplusx nvector implementation
%
\section{The NVECTOR\_MPIPLUSX implementation}\label{ss:nvec_mpiplusx}

The {\nvecmpiplusx} implementation of the {\nvector} module provided
with {\sundials} is designed to facilitate the MPI+X paradigm, where
X is some form of on-node (local) parallelism (e.g. OpenMP, CUDA).
This paradigm is becoming increasingly popular with the rise of
heterogeneous computing architectures.

The {\nvecmpiplusx} implementation is designed to work with any {\nvector} that
implements the minimum \emph{required} set of operations. However, it is not
recommended to use the {\nvecp}, {\nvecph}, {\nvecpetsc}, or {\nvectrilinos}
implementations underneath the {\nvecmpiplusx} module since they already provide
MPI capabilities.

% ====================================================================
\subsection{NVECTOR\_MPIPLUSX structure}
\label{ss:nvec_mpiplusx_structure}
% ====================================================================

The {\nvecmpiplusx} implementation is a thin wrapper around the
{\nvecmpimanyvector}. Accordingly, it adopts the same content structure
as defined in Section~\ref{ss:nvec_mpimanyvector_structure}. 

The header file to include when using this module is
\id{nvector\_mpiplusx.h}. The installed module library to link against is
\id{libsundials\_nvecmpiplusx.\textit{lib}} where \id{\em.lib} is typically
\id{.so} for shared libraries and \id{.a} for static libraries.

\warn\textbf{Note:} If {\sundials} is configured with MPI disabled, then the
mpiplusx library will not be built.  Furthermore, any user codes
that include \id{nvector\_mpiplusx.h} \emph{must} be compiled
using an MPI-aware compiler.

% ====================================================================
\subsection{NVECTOR\_MPIPLUSX functions}
\label{ss:nvec_mpiplusx_functions}
% ====================================================================

The {\nvecmpiplusx} module adopts all vector operations listed
in Tables \ref{ss:nvecops}, \ref{ss:nvecfusedops}, \ref{ss:nvecarrayops},
and \ref{ss:nveclocalops}, from the {\nvecmpimanyvector} (see section
\ref{ss:nvec_mpimanyvector_functions}) except for \id{N\_VGetArrayPointer}
and \id{N\_VSetArrayPointer}; the module provides its own implementation
of these functions that call the local vector implementations. Therefore,
the {\nvecmpiplusx} module implements all of the operations listed in the
referenced sections except for \id{N\_VScaleAddMultiVectorArray}, and
\id{N\_VLinearCombinationVectorArray}. Accordingly, it's compatibility
with the {\sundials} Fortran-77 interface, and with the {\sundials}
direct solvers and preconditioners depends on the local vector implementation.

The module {\nvecmpiplusx} provides the following additional
user-callable routines:
%%--------------------------------------
\sunmodfun{N\_VMake\_MPIPlusX}
{
  This function creates an MPIPlusX vector from an existing local
  (i.e. on-node) {\nvector} object, and a user-created MPI communicator.

  The input \id{comm} should be the memory reference to this
  user-created MPI communicator.  We note that since many {\mpi}
  implementations \id{\#define} \id{MPI\_COMM\_WORLD} to be a specific
  integer \emph{value} (that has no memory reference), users who wish
  to supply \id{MPI\_COMM\_WORLD} to this routine should first
  set a specific \id{MPI\_Comm} variable to \id{MPI\_COMM\_WORLD}
  before passing in the reference, e.g.

  \hspace{0.5in} \texttt{MPI\_Comm comm;}\vspace{-0.5em}
  
  \hspace{0.5in} \texttt{comm = MPI\_COMM\_WORLD;}\vspace{-0.5em}
  
  \hspace{0.5in} \texttt{N\_Vector x;}\vspace{-0.5em}
  
  \hspace{0.5in} \texttt{x = N\_VMake\_MPIPlusX(\&comm, ...);}

  This routine will internally call \id{MPI\_Comm\_dup} to create a
  copy of the input \id{comm}, so the user-supplied \id{comm} argument
  need not be retained after the call to \id{N\_VMake\_MPIPlusX}.

  This routine will copy the \id{N\_Vector} pointer to the input
  \id{local\_vector}, so the underlying local {\nvector} object
  should not be destroyed before the mpiplusx that contains it.

  Upon successful completion, the new MPIPlusX is returned;
  otherwise this routine returns \id{NULL} (e.g., if the input
  \id{local\_vector} is \id{NULL}).
}
{
  N\_Vector N\_VMake\_MPIPlusX(MPI\_Comm *comm, 
  \newlinefill{N\_Vector N\_VMake\_MPIPlusX}
  N\_Vector *local\_vector);
}
%%--------------------------------------
\sunmodfun{N\_VGetLocalVector\_MPIPlusX}
{
  This function returns the local vector underneath the 
  the MPIPlusX {\nvector}.
}
{
  N\_Vector N\_VGetLocalVector\_MPIPlusX(N\_Vector v);
}
\sunmodfun{N\_VGetArrayPointer\_MPIPlusX}
{
  This function returns the data array pointer for the local vector
  if the local vector implements the \id{N\_VGetArrayPointer} operation;
  otherwise it returns \id{NULL}.
}
{
  realtype* N\_VGetLocalVector\_MPIPlusX(N\_Vector v);
}
\sunmodfun{N\_VSetArrayPointer\_MPIPlusX}
{
  This function sets the data array pointer for the local vector
  if the local vector implements the \id{N\_VSetArrayPointer} operation.
}
{
  void N\_VSetArrayPointer\_MPIPlusX(realtype *data, N\_Vector v);
}
%%--------------------------------------
The {\nvecmpiplusx} module does not implement any fused or vector array
operations. Instead users should enable/disable fused operations on the
local vector.
%%
%%------------------------------------
%%
\paragraph{\bf Notes} 
           
\begin{itemize}
                                        
\item
  {\warn}\id{N\_VMake\_MPIPlusX} sets the field {\em own\_data} $=$ \id{SUNFALSE}. \\
  and \id{N\_VDestroy\_MPIPlusX} will not call \id{N\_VDestroy} on the local
  vector. In this case, it is the user's responsibility to deallocate the local vector.

\item
  {\warn}To maximize efficiency, arithmetic vector operations in the
  {\nvecmpiplusx} implementation that have more than one
  \id{N\_Vector} argument do not check for consistent internal
  representation of these vectors. It is the user's responsibility to
  ensure that such routines are called with \id{N\_Vector} arguments
  that were all created with the same local vector representations.

\end{itemize}


\section{NVECTOR Examples}\label{ss:nvec_examples}

There are \id{NVector} examples that may be installed for the
implementations provided with {\sundials}. Each
implementation makes use of the functions in \id{test\_nvector.c}.
These example functions show simple usage of the \id{NVector} family
of functions. The input to the examples are the vector length, number
of threads (if threaded implementation), and a print timing flag.

\noindent The following is a list of the example functions in \id{test\_nvector.c}:
\begin{itemize}
\item \id{Test\_N\_VClone}: Creates clone of vector and checks validity of clone.
\item \id{Test\_N\_VCloneEmpty}: Creates clone of empty vector and checks validity of clone.
\item \id{Test\_N\_VCloneVectorArray}: Creates clone of vector array and checks validity of cloned array.
\item \id{Test\_N\_VCloneVectorArray}: Creates clone of empty vector array and checks validity of cloned array.
\item \id{Test\_N\_VGetArrayPointer}: Get array pointer.
\item \id{Test\_N\_VSetArrayPointer}: Allocate new vector, set pointer to new vector array, and check values.
\item \id{Test\_N\_VGetLength}: Compares self-reported length to calculated length.
\item \id{Test\_N\_VGetCommunicator}: Compares self-reported communicator to the one used in constructor; or for MPI-unaware vectors it ensures that NULL is reported.
\item \id{Test\_N\_VLinearSum} Case 1a: Test y =  x + y
\item \id{Test\_N\_VLinearSum} Case 1b: Test y = -x + y
\item \id{Test\_N\_VLinearSum} Case 1c: Test y = ax + y
\item \id{Test\_N\_VLinearSum} Case 2a: Test x =  x + y
\item \id{Test\_N\_VLinearSum} Case 2b: Test x =  x - y
\item \id{Test\_N\_VLinearSum} Case 2c: Test x =  x + by
\item \id{Test\_N\_VLinearSum} Case 3:  Test z =  x + y
\item \id{Test\_N\_VLinearSum} Case 4a: Test z =  x - y
\item \id{Test\_N\_VLinearSum} Case 4b: Test z = -x + y
\item \id{Test\_N\_VLinearSum} Case 5a: Test z =  x + by
\item \id{Test\_N\_VLinearSum} Case 5b: Test z = ax + y
\item \id{Test\_N\_VLinearSum} Case 6a: Test z = -x + by
\item \id{Test\_N\_VLinearSum} Case 6b: Test z = ax - y
\item \id{Test\_N\_VLinearSum} Case 7:  Test z = a(x + y)
\item \id{Test\_N\_VLinearSum} Case 8:  Test z = a(x - y)
\item \id{Test\_N\_VLinearSum} Case 9:  Test z = ax + by
\item \id{Test\_N\_VConst}: Fill vector with constant and check result.
\item \id{Test\_N\_VProd}: Test vector multiply: z = x * y
\item \id{Test\_N\_VDiv}: Test vector division: z = x / y
\item \id{Test\_N\_VScale}: Case 1: scale: x = cx
\item \id{Test\_N\_VScale}: Case 2: copy: z = x
\item \id{Test\_N\_VScale}: Case 3: negate: z = -x
\item \id{Test\_N\_VScale}: Case 4: combination: z = cx
\item \id{Test\_N\_VAbs}: Create absolute value of vector.
\item \id{Test\_N\_VAddConst}: add constant vector: z = c + x
\item \id{Test\_N\_VDotProd}: Calculate dot product of two vectors.
\item \id{Test\_N\_VMaxNorm}: Create vector with known values, find and validate the max norm.
\item \id{Test\_N\_VWrmsNorm}: Create vector of known values, find and validate the weighted root mean square.
\item \id{Test\_N\_VWrmsNormMask}: Create vector of known values, find and validate the weighted root mean square using all elements except one.
\item \id{Test\_N\_VMin}: Create vector, find and validate the min.
\item \id{Test\_N\_VWL2Norm}: Create vector, find and validate the weighted Euclidean L2 norm.
\item \id{Test\_N\_VL1Norm}: Create vector, find and validate the L1 norm.
\item \id{Test\_N\_VCompare}: Compare vector with constant returning and validating comparison vector.
\item \id{Test\_N\_VInvTest}: Test z[i] = 1 / x[i]
\item \id{Test\_N\_VConstrMask}: Test mask of vector x with vector c.
\item \id{Test\_N\_VMinQuotient}: Fill two vectors with known values. Calculate and validate minimum quotient.
\item \id{Test\_N\_VLinearCombination} Case 1a: Test x = a x
\item \id{Test\_N\_VLinearCombination} Case 1b: Test z = a x
\item \id{Test\_N\_VLinearCombination} Case 2a: Test x = a x + b y
\item \id{Test\_N\_VLinearCombination} Case 2b: Test z = a x + b y
\item \id{Test\_N\_VLinearCombination} Case 3a: Test x = x + a y + b z
\item \id{Test\_N\_VLinearCombination} Case 3b: Test x = a x + b y + c z
\item \id{Test\_N\_VLinearCombination} Case 3c: Test w = a x + b y + c z
\item \id{Test\_N\_VScaleAddMulti} Case 1a: y = a x + y
\item \id{Test\_N\_VScaleAddMulti} Case 1b: z = a x + y
\item \id{Test\_N\_VScaleAddMulti} Case 2a: Y[i] = c[i] x + Y[i], i = 1,2,3
\item \id{Test\_N\_VScaleAddMulti} Case 2b: Z[i] = c[i] x + Y[i], i = 1,2,3
\item \id{Test\_N\_VDotProdMulti} Case 1: Calculate the dot product of two vectors
\item \id{Test\_N\_VDotProdMulti} Case 2: Calculate the dot product of one vector with three other vectors in a vector array.
\item \id{Test\_N\_VLinearSumVectorArray} Case 1: z = a x + b y
\item \id{Test\_N\_VLinearSumVectorArray} Case 2a: Z[i] = a X[i] + b Y[i]
\item \id{Test\_N\_VLinearSumVectorArray} Case 2b: X[i] = a X[i] + b Y[i]
\item \id{Test\_N\_VLinearSumVectorArray} Case 2c: Y[i] = a X[i] + b Y[i]
\item \id{Test\_N\_VScaleVectorArray} Case 1a: y = c y
\item \id{Test\_N\_VScaleVectorArray} Case 1b: z = c y
\item \id{Test\_N\_VScaleVectorArray} Case 2a: Y[i] = c[i] Y[i]
\item \id{Test\_N\_VScaleVectorArray} Case 2b: Z[i] = c[i] Y[i]
\item \id{Test\_N\_VScaleVectorArray} Case 1a: z = c
\item \id{Test\_N\_VScaleVectorArray} Case 1b: Z[i] = c
\item \id{Test\_N\_VWrmsNormVectorArray} Case 1a: Create a vector of know values, find and validate the weighted root mean square norm.
\item \id{Test\_N\_VWrmsNormVectorArray} Case 1b: Create a vector array of three vectors of know values, find and validate the weighted root mean square norm of each.
\item \id{Test\_N\_VWrmsNormMaskVectorArray} Case 1a: Create a vector of know values, find and validate the weighted root mean square norm using all elements except one.
\item \id{Test\_N\_VWrmsNormMaskVectorArray} Case 1b: Create a vector array of three vectors of know values, find and validate the weighted root mean square norm of each using all elements except one.
\item \id{Test\_N\_VScaleAddMultiVectorArray} Case 1a: y = a x + y
\item \id{Test\_N\_VScaleAddMultiVectorArray} Case 1b: z = a x + y
\item \id{Test\_N\_VScaleAddMultiVectorArray} Case 2a: Y[j][0] = a[j] X[0] + Y[j][0]
\item \id{Test\_N\_VScaleAddMultiVectorArray} Case 2b: Z[j][0] = a[j] X[0] + Y[j][0]
\item \id{Test\_N\_VScaleAddMultiVectorArray} Case 3a: Y[0][i] = a[0] X[i] + Y[0][i]
\item \id{Test\_N\_VScaleAddMultiVectorArray} Case 3b: Z[0][i] = a[0] X[i] + Y[0][i]
\item \id{Test\_N\_VScaleAddMultiVectorArray} Case 4a: Y[j][i] = a[j] X[i] + Y[j][i]
\item \id{Test\_N\_VScaleAddMultiVectorArray} Case 4b: Z[j][i] = a[j] X[i] + Y[j][i]
\item \id{Test\_N\_VLinearCombinationVectorArray} Case 1a: x = a x
\item \id{Test\_N\_VLinearCombinationVectorArray} Case 1b: z = a x
\item \id{Test\_N\_VLinearCombinationVectorArray} Case 2a: x = a x + b y
\item \id{Test\_N\_VLinearCombinationVectorArray} Case 2b: z = a x + b y
\item \id{Test\_N\_VLinearCombinationVectorArray} Case 3a: x = a x + b y + c z
\item \id{Test\_N\_VLinearCombinationVectorArray} Case 3b: w = a x + b y + c z
\item \id{Test\_N\_VLinearCombinationVectorArray} Case 4a: X[0][i] = c[0] X[0][i]
\item \id{Test\_N\_VLinearCombinationVectorArray} Case 4b: Z[i] = c[0] X[0][i]
\item \id{Test\_N\_VLinearCombinationVectorArray} Case 5a: X[0][i] = c[0] X[0][i] + c[1] X[1][i]
\item \id{Test\_N\_VLinearCombinationVectorArray} Case 5b: Z[i] = c[0] X[0][i] + c[1] X[1][i]
\item \id{Test\_N\_VLinearCombinationVectorArray} Case 6a: X[0][i] = X[0][i] + c[1] X[1][i] + c[2] X[2][i]
\item \id{Test\_N\_VLinearCombinationVectorArray} Case 6b: X[0][i] = c[0] X[0][i] + c[1] X[1][i] + c[2] X[2][i]
\item \id{Test\_N\_VLinearCombinationVectorArray} Case 6c: Z[i] = c[0] X[0][i] + c[1] X[1][i] + c[2] X[2][i]
\item \id{Test\_N\_VDotProdLocal}: Calculate MPI task-local portion of the dot product of two vectors.
\item \id{Test\_N\_VMaxNormLocal}: Create vector with known values, find and validate the MPI task-local portion of the max norm.
\item \id{Test\_N\_VMinLocal}: Create vector, find and validate the MPI task-local min.
\item \id{Test\_N\_VL1NormLocal}: Create vector, find and validate the MPI task-local portion of the L1 norm.
\item \id{Test\_N\_VWSqrSumLocal}: Create vector of known values, find and validate the MPI task-local portion of the weighted squared sum of two vectors.
\item \id{Test\_N\_VWSqrSumMaskLocal}: Create vector of known values, find and validate the MPI task-local portion of the weighted squared sum of two vectors, using all elements except one.
\item \id{Test\_N\_VInvTestLocal}: Test the MPI task-local portion of z[i] = 1 / x[i]
\item \id{Test\_N\_VConstrMaskLocal}: Test the MPI task-local portion of the mask of vector x with vector c.
\item \id{Test\_N\_VMinQuotientLocal}: Fill two vectors with known values. Calculate and validate the MPI task-local minimum quotient.
\end{itemize}

