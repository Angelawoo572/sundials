%===================================================================================
\section{C++ example problems}\label{s:ex_cpp}
%===================================================================================

\subsection{A parallel matrix-free example: kin\_heat2D\_nonlin\_p}\label{ss:kin_heat2D_nonlin_p}

As an illustration of the use of the {\kinsol} package for the
solution of nonlinear systems in parallel, 
we give a sample program called \id{kin\_heat2D\_nonlin\_p.cpp}.
It uses the {\kinsol} \id{KINFP} iteration 
and the {\nvecp} module (which provides a parallel implementation of {\nvector})
for the solution of the following test problem.

This problem involves solving a steady-state 2D heat equation with an additional
nonlinear term defined by $c(u)$:
\begin{equation}
    b = k_x u_{xx} + k_y u_{yy} + c(u) \quad \text{in} \quad \mathcal{D}
        = [0,1] \times [0,1]
\end{equation}
subject to the boundary conditions
\begin{equation}
    u(0,y) = u(1,y) = u(x,0) = u(x,1) = 0 \quad \text{on} \quad \partial \mathcal{D}.
\end{equation}
%%
We chose the analytical solution to be
\begin{equation}
    u_{exact} = u(x,y) = \sin^2(\pi x) \sin^2(\pi y),
\end{equation}
hence, we define the static term $b$ as follows
\begin{equation}
\begin{aligned}
    b = &k_x 2 \pi^2 (\cos^2(\pi x) - \sin^2(\pi x)) \sin^2(\pi y) \\
        &+ k_y 2 \pi^2 (\cos^2(\pi y) - \sin^2(\pi y)) \sin^2(\pi x) + c(u_{exact})
\end{aligned}
\end{equation}

The spatial derivatives are computed using second-order centered
differences, with the data distributed over $nx\times ny$ points
on a uniform spatial grid.  
The problem is set up to use spatial grid parameters $nx=64$ and
$ny=64$, with heat conductivity parameters $k_x=1.0$ and
$k_y=1.0$.  

This problem is solved via a Fixed Point Iteration, where the fixed point format
of the problem is set up by adding $u$ to both sides of the equation, as
seen here
\begin{equation}
    b + u = k_x u_{xx} + k_y u_{yy} + c(u) + u.
\end{equation}
Under this format, the fixed point problem to be solved is given by
\begin{equation}
    u = k_x u_{xx} + k_y u_{yy} + c(u) + u - b.
\end{equation}

This program solves the problem with Fixed Point Iteration, 
with the option of using Anderson Acceleration.
%%
The problem is run using a scalar relative
tolerance of $rtol=10^{-8}$, and a starting vector
containing all ones. 
The following table contains all available input parameters
when running the example problem.
%%
% Something like the following for the KIN memory calls
%%

\begin{center}
\begin{tabular}{ |p{5cm}||p{10cm}| }
\hline
\multicolumn{2}{|c|}{Input Parameters} \\
\hline
flag & option \\
\hline
{\tt --mesh <nx> <ny>} & mesh points in the x and y directions\\
{\tt --np <npx> <npy>} & number of MPI processes in the x and y directions\\
{\tt --domain <xu> <yu>} & domain upper bound in the x and y direction\\
{\tt --k <kx> <ky>} & diffusion coefficients\\
{\tt --rtol <rtol>} & relative tolerance\\
{\tt --maa <maa>} & number of previous residuals for Anderson Acceleration \\
{\tt --damping <damping>} & damping parameter for Anderson Acceleration \\
{\tt --orthaa <orthaa>} & orthogonalization routine used in Anderson Acceleration \\
{\tt --maxits <maxits>} & max number of iterations \\
{\tt --c <cu>} & nonlinear function choice (integer between 1 - 17)\\
{\tt --timing} & print timing data\\
{\tt --help} & print available input parameters and exit\\
\hline
\end{tabular}
\end{center}

The following nonlinear functions $c(u)$ can be set as the $c(u)$ argument via the
{\tt --c} flag.

\begin{center}
\begin{tabular}{ |p{2cm}||p{10cm}| }
\hline
\multicolumn{2}{|c|}{Input Parameter: Function Flag Options} \\
\hline
flag & function \\
\hline
{\tt --c 1} & $ c(u) = u $ \\
{\tt --c 2} & $ c(u) = u^3 - u $ \\
{\tt --c 3} & $ c(u) = u - u^2 $ \\
{\tt --c 4} & $ c(u) = e^u $ \\
{\tt --c 5} & $ c(u) = u^4 $ \\
{\tt --c 6} & $ c(u) = \cos^2(u) - \sin^2(u) $ \\
{\tt --c 7} & $ c(u) = \cos^2(u) - \sin^2(u) - e^u $ \\
{\tt --c 8} & $ c(u) = e^uu^4 - ue^{\cos(u)} $ \\
{\tt --c 9} & $ c(u) = e^{(\cos^2(u))} $ \\
{\tt --c 10} & $ c(u) = 10(u - u^2) $ \\
{\tt --c 11} & $ c(u) = -13 + u + ((5-u)u - 2)u $ \\
{\tt --c 12} & $ c(u) = \sqrt{5}(u - u^2) $ \\
{\tt --c 13} & $ c(u) = (u - e^u)^2 + (u + u \sin(u) - \cos(u))^2 $ \\
{\tt --c 14} & $ c(u) = u + ue^u + ue^{-u} $ \\
{\tt --c 15} & $ c(u) = u + ue^u + ue^{-u} + (u - e^u)^2 $ \\
{\tt --c 16} & $ c(u) = u + ue^u + ue^{-u} + (u - e^u)^2 + (u + u\sin(u) - \cos(u))^2 $ \\
{\tt --c 17} & $ c(u) = u + ue^{-u} + e^u (u + \sin(u) - \cos(u))^3 $ \\
\hline
\end{tabular}
\end{center}
%-----------------------------------------------------------------------------------

\subsection{A parallel example using {\hypre}: kin\_heat2D\_nonlin\_hypre\_pfmg}\label{ss:kin_heat2D_nonlin_hypre_pfmg}

As an illustration of the use of the {\kinsol} package for the
solution of nonlinear systems in parallel and using {\hypre} linear solvers, 
we give a sample program called \id{kin\_heat2D\_nonlin\_hypre\_pfmg.cpp}.
It uses the {\kinsol} \id{KINFP} iteration
and the {\nvecp} module (which provides a parallel implementation of {\nvector})
for the solution of the following test problem.

This problem involves solving a steady-state 2D heat equation with an additional
nonlinear term defined by $c(u)$:
\begin{equation}
    b = k_x u_{xx} + k_y u_{yy} + c(u) \quad \text{in} \quad \mathcal{D}
        = [0,1] \times [0,1]
\end{equation}
subject to the boundary conditions
\begin{equation}
    u(0,y) = u(1,y) = u(x,0) = u(x,1) = 0 \quad \text{on} \quad \partial \mathcal{D}.
\end{equation}
%%
We chose the analytical solution to be
\begin{equation}
    u_{exact} = u(x,y) = \sin^2(\pi x) \sin^2(\pi y),
\end{equation}
hence, we define the static term $b$ as follows
\begin{equation}
\begin{aligned}
    b = &k_x 2 \pi^2 (\cos^2(\pi x) - \sin^2(\pi x)) \sin^2(\pi y) \\
        &+ k_y 2 \pi^2 (\cos^2(\pi y) - \sin^2(\pi y)) \sin^2(\pi x) + c(u_{exact})
\end{aligned}
\end{equation}

The spatial derivatives are computed using second-order centered
differences, with the data distributed over $nx\times ny$ points
on a uniform spatial grid.  
The problem is set up to use spatial grid parameters $nx=64$ and
$ny=64$, with heat conductivity parameters $k_x=1.0$ and
$k_y=1.0$.  

This problem is solved via Fixed Point Iteration, where the fixed point format
of the problem is set up by implementing the Laplacian as a matrix-vector product, 
\begin{equation}
    b = A u + c(u),
\end{equation}
then moving $c(u)$ to the other side of the equation 
and solving for $u$ results 
in the following fixed point formulation
\begin{equation}
    u = A^{-1} (b - c(u)).
\end{equation}

This program solves the problem with a Fixed Point Iteration, 
with the option of using Anderson Acceleration.
%%
The problem is run using a scalar relative
tolerance of $rtol=10^{-8}$, and a starting
vector containing all ones.
The linear system solve is executed using the
\id{SUNLINSOL\_PCG} linear solver with
the {\hypre} PFMG preconditioner. The setup of the
linear solver can be found in the
\id{Setup\_LS} function, and setup of the {\hypre}
preconditioner can be found in the
\id{Setup\_Hypre} function within the main file.

All input parameter flags available for Example 
\ref{ss:kin_heat2D_nonlin_p} are also available for this problem.
In addition, all runtime flags controlling the linear solver
and {\hypre} related parameters are set
using the flags in the following table. 
%%
\begin{center}
\begin{tabular}{ |p{6cm}||p{10cm}| }
\hline
\multicolumn{2}{|c|}{Linear Solver and {\hypre} Related Input Parameters} \\
\hline
flag & option \\
\hline
{\tt --lsinfo} & output residual history for PCG\\
{\tt --liniters <liniters>} & max number of iterations for PCG\\
{\tt --epslin <epslin>} & linear tolerance for PCG\\
{\tt --pfmg\_relax <pfmg\_relax>} & relaxation type in PFMG\\
{\tt --pfmg\_nrelax <pfmg\_nrelax>} & pre/post relaxation sweeps in PFMG\\
\hline
\end{tabular}
\end{center}

