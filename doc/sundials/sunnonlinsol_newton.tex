% ====================================================================
\section{The SUNNonlinearSolver\_Newton implementation}
\label{s:sunnonlinsol_newton}
% ====================================================================

This section describes the {\sunnonlinsol} implementation of Newton's method. To
access the {\sunnonlinsolnewton} module, include the header file
\id{sunnonlinsol/sunnonlinsol\_newton.h}. We note that the {\sunnonlinsolnewton}
module is accessible from {\sundials} integrators \textit{without} separately
linking to the \id{libsundials\_sunnonlinsolnewton} module library.

% ====================================================================
\subsection{SUNNonlinearSolver\_Newton description}
\label{ss:sunnonlinsolnewton_math}
% ====================================================================

To find the solution to
\begin{equation}\label{e:newton_sys}
  F(y) = 0 \,
\end{equation}
given an initial guess $y^{(0)}$, Newton's method computes a series of
approximate solutions
\begin{equation}
  y^{(m+1)} = y^{(m)} + \delta^{(m+1)}
\end{equation}
where $m$ is the Newton iteration index, and the Newton update $\delta^{(m+1)}$
is the solution of the linear system
\begin{equation}\label{e:newton_linsys}
  A(y^{(m)}) \delta^{(m+1)} = -F(y^{(m)}) \, ,
\end{equation}
in which $A$ is the Jacobian matrix
\begin{equation}\label{e:newton_mat}
  A \equiv \partial F / \partial y \, .
\end{equation}
Depending on the linear solver used, the {\sunnonlinsolnewton} module
will employ either a Modified Newton method, or an Inexact Newton
method~\cite{Bro:87,BrSa:90,DES:82,DeSc:96,Kel:95}. When used with a direct
linear solver, the Jacobian matrix $A$ is held constant during the Newton
iteration, resulting in a Modified Newton method. With a matrix-free iterative
linear solver, the iteration is an Inexact Newton method.

In both cases, calls to the integrator-supplied \id{SUNNonlinSolLSetupFn}
function are made infrequently to amortize the increased cost of
matrix operations (updating $A$ and its factorization within direct
linear solvers, or updating the preconditioner within iterative linear
solvers).  Specifically, {\sunnonlinsolnewton} will call the
\id{SUNNonlinSolLSetupFn} function in two instances:
\begin{itemize}
\item[(a)] when requested by the integrator (the input
  \id{callLSetSetup} is \id{SUNTRUE}) before attempting the Newton
  iteration, or
\item[(b)] when reattempting the nonlinear solve after a recoverable
  failure occurs in the Newton iteration with stale Jacobian
  information (\id{jcur} is \id{SUNFALSE}).  In this case,
  {\sunnonlinsolnewton} will set \id{jbad} to \id{SUNTRUE} before
  calling the \id{SUNNonlinSolLSetupFn} function.
\end{itemize}
Whether the Jacobian matrix $A$ is fully or partially updated depends
on logic unique to each integrator-supplied \id{SUNNonlinSolSetupFn}
routine. We refer to the discussion of nonlinear solver strategies
provided in Chapter \ref{s:math} for details on this decision.

The default maximum number of iterations and the stopping criteria for
the Newton iteration are supplied by the {\sundials} integrator when
{\sunnonlinsolnewton} is attached to it.  Both the maximum number of
iterations and the convergence test function may be modified by the
user by calling the \id{SUNNonlinSolSetMaxIters} and/or
\id{SUNNonlinSolSetConvTestFn} functions after attaching the
{\sunnonlinsolnewton} object to the integrator.

% ====================================================================
\subsection{SUNNonlinearSolver\_Newton functions}
\label{ss:sunnonlinsolnewton_functions}
% ====================================================================

The {\sunnonlinsolnewton} module provides the following constructors
for creating a \newline
\id{SUNNonlinearSolver} object.
% --------------------------------------------------------------------
\ucfunctionf{SUNNonlinSol\_Newton}
{
  NLS = SUNNonlinSol\_Newton(y);
}
{
  The function \ID{SUNNonlinSol\_Newton} creates a
  \id{SUNNonlinearSolver} object for use with {\sundials} integrators to
  solve nonlinear systems of the form $F(y) = 0$ using Newton's method.
}
{
  \begin{args}[y]
  \item[y] (\id{N\_Vector})
    a template for cloning vectors needed within the solver.
  \end{args}
}
{
  The return value \id{NLS} (of type \id{SUNNonlinearSolver}) will be
  a {\sunnonlinsol} object if the constructor exits successfully,
  otherwise \id{NLS} will be \id{NULL}.
}
{}
% --------------------------------------------------------------------
\ucfunctionf{SUNNonlinSol\_NewtonSens}
{
  NLS = SUNNonlinSol\_NewtonSens(count, y);
}
{
  The function \ID{SUNNonlinSol\_NewtonSens} creates a
  \id{SUNNonlinearSolver} object for use with {\sundials} sensitivity enabled
  integrators ({\cvodes} and {\idas}) to solve nonlinear systems of the form
  $F(y) = 0$ using Newton's method.
}
{
  \begin{args}[count]
  \item[count] (\id{int})
    the number of vectors in the nonlinear solve. When integrating a system
    containing \id{Ns} sensitivities the value of \id{count} is:
    \begin{itemize}
      \item \id{Ns+1} if using a \textit{simultaneous} corrector approach.
      \item \id{Ns} if using a \textit{staggered} corrector approach.
    \end{itemize}
  \item[y] (\id{N\_Vector})
    a template for cloning vectors needed within the solver.
  \end{args}
}
{
  The return value \id{NLS} (of type \id{SUNNonlinearSolver}) will be
  a {\sunnonlinsol} object if the constructor exits successfully,
  otherwise \id{NLS} will be \id{NULL}.
}
{}
% --------------------------------------------------------------------
The {\sunnonlinsolnewton} module implements all of the functions
defined in sections \ref{ss:sunnonlinsol_corefn} --
\ref{ss:sunnonlinsol_getfn} except for the \id{SUNNonlinSolSetup} function. The
{\sunnonlinsolnewton} functions have the same names as those defined
by the generic {\sunnonlinsol} API with \id{\_Newton} appended to the
function name. Unless using the {\sunnonlinsolnewton} module as a
standalone nonlinear solver the generic functions defined in sections
\ref{ss:sunnonlinsol_corefn} -- \ref{ss:sunnonlinsol_getfn} should be
called in favor of the {\sunnonlinsolnewton}-specific implementations.

The {\sunnonlinsolnewton} module also defines the following additional
user-callable function.
% --------------------------------------------------------------------
\ucfunctionf{SUNNonlinSolGetSysFn\_Newton}
{
  retval = SUNNonlinSolGetSysFn\_Newton(NLS, SysFn);
}
{
  The function \ID{SUNNonlinSolGetSysFn\_Newton} returns the residual function
  that defines the nonlinear system.
}
{
  \begin{args}[SysFn]
  \item[NLS] (\id{SUNNonlinearSolver})
    a {\sunnonlinsol} object
  \item[SysFn] (\id{SUNNonlinSolSysFn*})
    the function defining the nonlinear system.
  \end{args}
}
{
  The return value \id{retval} (of type \id{int}) should be zero for a
  successful call, and a negative value for a failure.
}
{
  This function is intended for users that wish to evaluate the
  nonlinear residual in a custom convergence test function for the
  {\sunnonlinsolnewton} module.  We note that {\sunnonlinsolnewton}
  will not leverage the results from any user calls to \id{SysFn}.
}


% ====================================================================
\subsection{SUNNonlinearSolver\_Newton Fortran interfaces}
\label{ss:sunnonlinsolnewton_fortran}
% ====================================================================

The {\sunnonlinsolnewton} module provides a {\F} 2003 module as well as
{\F} 77 style interface functions for use from {\F} applications.

\subsubsection*{FORTRAN 2003 interface module}
The \ID{fsunnonlinsol\_newton\_mod} {\F} module defines interfaces to all
{\sunnonlinsolnewton} {\CC} functions using the intrinsic \id{iso\_c\_binding}
module which provides a standardized mechanism for interoperating with {\CC}. As
noted in the {\CC} function descriptions above, the interface functions are
named after the corresponding {\CC} function, but with a leading `F'. For
example, the function \id{SUNNonlinSol\_Newton} is interfaced as
\id{FSUNNonlinSol\_Newton}.

The {\F} 2003 {\sunnonlinsolnewton} interface module can be accessed with the \id{use}
statement, i.e. \id{use fsunnonlinsol\_newton\_mod}, and linking to the library \newline
\id{libsundials\_fsunnonlinsolnewton\_mod}.{\em lib} in addition to the {\CC}
library. For details on where the library and module file
\id{fsunnonlinsol\_newton\_mod.mod} are installed see Appendix \ref{c:install}.
We note that the module is accessible from the {\F} 2003 {\sundials} integrators
\textit{without} separately linking to the
\id{libsundials\_fsunnonlinsolnewton\_mod} library.

\subsubsection*{FORTRAN 77 interface functions}
For {\sundials} integrators that include a {\F} 77 interface, the
{\sunnonlinsolnewton} module also includes a Fortran-callable
function for creating a \id{SUNNonlinearSolver} object.
%
% --------------------------------------------------------------------
%
\ucfunction{FSUNNEWTONINIT}
{
  FSUNNEWTONINIT(code, ier);
}
{
  The function \ID{FSUNNEWTONINIT} can be called for Fortran programs
  to create a\\
  \id{SUNNonlinearSolver} object for use with {\sundials}
  integrators to solve nonlinear systems of the form $F(y) = 0$ with
  Newton's method.
}
{
  \begin{args}[code]
  \item[code] (\id{int*})
    is an integer input specifying the solver id (1 for {\cvode}, 2
    for {\ida}, and 4 for {\arkode}).
  \end{args}
}
{
  \id{ier} is a return completion flag equal to \id{0} for a success
  return and \id{-1} otherwise. See printed message for details in case
  of failure.
}
{}


% ====================================================================
\subsection{SUNNonlinearSolver\_Newton content}
\label{ss:sunnonlinsolnewton_content}
% ====================================================================

The {\sunnonlinsolnewton} module defines the {\textit{content} field of a
\id{SUNNonlinearSolver} as the following structure:
%%
%%
\begin{verbatim}
struct _SUNNonlinearSolverContent_Newton {

  SUNNonlinSolSysFn      Sys;
  SUNNonlinSolLSetupFn   LSetup;
  SUNNonlinSolLSolveFn   LSolve;
  SUNNonlinSolConvTestFn CTest;

  N_Vector    delta;
  booleantype jcur;
  int         curiter;
  int         maxiters;
  long int    niters;
  long int    nconvfails;
  void*       ctest_data;
};
\end{verbatim}
%%
%%
These entries of the \emph{content} field contain the following
information:
\begin{args}[ctest\_data]
  \item[Sys]         - the function for evaluating the nonlinear system,
  \item[LSetup]      - the package-supplied function for setting up the linear solver,
  \item[LSolve]      - the package-supplied function for performing a linear solve,
  \item[CTest]       - the function for checking convergence of the Newton
                       iteration,
  \item[delta]       - the Newton iteration update vector,
  \item[jcur]        - the Jacobian status (\id{SUNTRUE} = current,
                       \id{SUNFALSE} = stale),
  \item[curiter]     - the current number of iterations in the solve attempt,
  \item[maxiters]    - the maximum number of Newton iterations allowed in
                       a solve,
  \item[niters]      - the total number of nonlinear iterations across all
                       solves,
  \item[nconvfails]  - the total number of nonlinear convergence failures across
                       all solves, and
  \item[ctest\_data] - the data pointer passed to the convergence test function.
\end{args}
