The {\sunnonlinsolnewton} module provides a {\sunnonlinsol}
implementation of Newton's method with the following constructor for
creating the \id{SUNNonlinearSolver} object.
% --------------------------------------------------------------------
\ucfunction{SUNNonlinSol\_Newton}
{
  NLS = SUNNonlinSol\_Newton(y);
}
{
  The function \ID{SUNNonlinSol\_Newton} creates a
  \id{SUNNonlinearSolver} object for use with {\sundials} packages to
  solve nonlinear system of the form $F(y) = 0$ with Newton's method.
}
{
  \begin{args}[y]
  \item[y] (\id{N\_Vector})
    an \id{N\_Vector} used as a template for cloning vectors need within
    the solver
  \end{args}
}
{
  The return value \id{NLS} (of type \id{SUNNonlinearSolver}) will be
  a {\sunnonlinsol} object if constructor exits successfully,
  otherwise \id{NLS} will be \id{NULL}.
}
{}
% --------------------------------------------------------------------
To access the constructor and other {\sunnonlinsolnewton} functions,
include the header file \wtt{sunlinsol/sunnonlinsol\_newton.h}. Note
the {\sunnonlinsolnewton} module is accessible from {\sundials}
solvers \textit{without} linking to the
\id{libsundials\_sunnonlinsolnewton} module library.

The {\sunnonlinsolnewton} module implements all of the functions
defined in Sections \ref{ss:sunnonlinsol_corefn} --
\ref{ss:sunnonlinsol_getfn} with \id{\_Newton} appended to the
function name. The \textit{content} field of the {\sunnonlinsolnewton}
module is the following structure.
%%
%%
\begin{verbatim} 
struct _SUNNonlinearSolverContent_Newton {

  SUNNonlinSolSysFn      Sys;
  SUNNonlinSolLSetupFn   LSetup;
  SUNNonlinSolLSolveFn   LSolve;
  SUNNonlinSolConvTestFn CTest;

  N_Vector delta;
  int      maxiters;
  long int niters;
};
\end{verbatim}
%%
%%
These entries of the \emph{content} field contain the following
information:
\begin{args}[maxiters]
  \item[Sys] - the function for evaluating the nonlinear system,
  \item[LSetup] - the function for setting up the linear solver,
  \item[LSolve] - the function for preforming linear solves,
  \item[CTest] - the function for checking convergence of the Newton
    iteration,
  \item[delta] - the Newton iteration update vector,
  \item[maxiters] - the maximum number of Newton iterations allows in
    a solve, and
  \item[niters] - the total number of nonlinear iterations across all
    solves.
\end{args}

For {\sundials} packages that include a Fortran interface, the
{\sunnonlinsolnewton} module also includes a Fortran-callable
function for creating a \id{SUNNonlinearSolver} object.
\ucfunction{FSUNNEWTONINIT}
{
  SUNNEWTONINIT(code, ier);
}
{
  The function \ID{SUNNEWTONINIT} can be called for Fortran programs
  to create a \id{SUNNonlinearSolver} object for use with {\sundials}
  packages to solve nonlinear system of the form $F(y) = 0$ with
  Newton's method.
}
{
  \begin{args}[code]
  \item[code] (\id{int*})
    is an integer input specifying the solver id (1 for {\cvode}, 2
    for {\ida}, 3 for {\kinsol}, and 4 for {\arkode}).
  \end{args}
}
{
  \id{ier} is a return completion flag equal to \id{0} for a success
  return and \id{-1} otherwise. See printed message for details in case
  of failure.
}
{}

% LocalWords:  Fortran SUNNEWTONINIT Fortran
