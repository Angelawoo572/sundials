This section describes the {\sundials} {\sunnonlinsol} implementation
of Newton's method. When used with a direct linear solver this module
is a modified Newton method which updates the Jacobian
infrequently. With an iterative linear solver this module is an
inexact Newton method which updates the preconditioner infrequently.

To access the {\sunnonlinsolnewton} module, include the header
file \id{sunnonlinsol/sunnonlinsol\_newton.h}. Note the
{\sunnonlinsolnewton} module is accessible from {\sundials}
packages \textit{without} linking to the
\id{libsundials\_sunnonlinsolnewton} module library.


% ====================================================================
\subsection{SUNNonlinerSolver\_Newton functions}
\label{ss:sunnonlinsolnewton_functions}
% ====================================================================

The {\sunnonlinsolnewton} module provides the following constructor
for creating the \id{SUNNonlinearSolver} object.  
% --------------------------------------------------------------------
\ucfunction{SUNNonlinSol\_Newton}
{
  NLS = SUNNonlinSol\_Newton(y);
}
{
  The function \ID{SUNNonlinSol\_Newton} creates a
  \id{SUNNonlinearSolver} object for use with {\sundials} packages to
  solve nonlinear systems of the form $F(y) = 0$ with Newton's method.
}
{
  \begin{args}[y]
  \item[y] (\id{N\_Vector})
    an \id{N\_Vector} used as a template for cloning vectors need within
    the solver
  \end{args}
}
{
  The return value \id{NLS} (of type \id{SUNNonlinearSolver}) will be
  a {\sunnonlinsol} object if constructor exits successfully,
  otherwise \id{NLS} will be \id{NULL}.
}
{}
% --------------------------------------------------------------------
The {\sunnonlinsolnewton} module implements all of the functions
defined in sections \ref{ss:sunnonlinsol_corefn} -- 
\ref{ss:sunnonlinsol_getfn} with \id{\_Newton} appended to the
function name.

The {\sunnonlinsolnewton} module also defines the following additional
user-callable function.
% --------------------------------------------------------------------
\ucfunction{SUNNonlinSolGetSysFn\_Newton}
{
  retval = SUNNonlinSolGetSysFn\_Newton(NLS, SysFn);
}
{
  The \textit{optional} function \ID{SUNNonlinSolGetSysFn\_Newton}
  returns the function defining the nonlinear system.
}
{
  \begin{args}[SysFn]
  \item[NLS] (\id{SUNNonlinearSolver})
    a {\sunnonlinsol} object
  \item[SysFn] (\id{SUNNonlinSolSysFn*})
    the function defining the nonlinear system.
  \end{args}
}
{
  The return value \id{retval} (of type \id{int}) should be zero for a
  successful call, and a negative value for a failure.
}
{
  This function is intended for users that which to evaluate the
  nonlinear system in a custom convergence test function with
  {\sundials} provided nonlinear solver modules. Note that {\sundials}
  provided nonlinear solvers will not leverage any extra nonlinear
  system evaluations in the nonlinear solve.
}


% ====================================================================
\subsection{SUNNonlinerSolver\_Newton content}
\label{ss:sunnonlinsolnewton_content}
% ====================================================================

The \textit{content} field of the {\sunnonlinsolnewton} module is the
following structure.
%%
%%
\begin{verbatim} 
struct _SUNNonlinearSolverContent_Newton {

  SUNNonlinSolSysFn      Sys;
  SUNNonlinSolLSetupFn   LSetup;
  SUNNonlinSolLSolveFn   LSolve;
  SUNNonlinSolConvTestFn CTest;

  N_Vector    delta;
  booleantype jcur;
  int         mnewt;
  int         maxiters;
  long int    niters;
};
\end{verbatim}
%%
%%
These entries of the \emph{content} field contain the following
information:
\begin{args}[maxiters]
  \item[Sys] - the function for evaluating the nonlinear system,
  \item[LSetup] - the function for setting up the linear solver,
  \item[LSolve] - the function for preforming linear solves,
  \item[CTest] - the function for checking convergence of the Newton
    iteration,
  \item[delta] - the Newton iteration update vector,
  \item[jcur] - the Jacobian status (\id{SUNTRUE} =
  current, \id{SUNFALSE} = stale),
  \item[mnewt] - the current number of iterations in the solve attempt,
  \item[maxiters] - the maximum number of Newton iterations allows in
    a solve, and
  \item[niters] - the total number of nonlinear iterations across all
    solves.
\end{args}


% ====================================================================
\subsection{SUNNonlinerSolver\_Newton Fortran interface}
\label{ss:sunnonlinsolnewton_fortran}
% ====================================================================

For {\sundials} packages that include a Fortran interface, the
{\sunnonlinsolnewton} module also includes a Fortran-callable
function for creating a \id{SUNNonlinearSolver} object.
\ucfunction{FSUNNEWTONINIT}
{
  SUNNEWTONINIT(code, ier);
}
{
  The function \ID{SUNNEWTONINIT} can be called for Fortran programs
  to create a \id{SUNNonlinearSolver} object for use with {\sundials}
  packages to solve nonlinear system of the form $F(y) = 0$ with
  Newton's method.
}
{
  \begin{args}[code]
  \item[code] (\id{int*})
    is an integer input specifying the solver id (1 for {\cvode}, 2
    for {\ida}, 3 for {\kinsol}, and 4 for {\arkode}).
  \end{args}
}
{
  \id{ier} is a return completion flag equal to \id{0} for a success
  return and \id{-1} otherwise. See printed message for details in case
  of failure.
}
{}
