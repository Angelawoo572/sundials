{\sundials} packages are written in terms of generic nonlinear solver
operations defined by the {\sunnonlinsol} API and implemented by a
particular {\sunnonlinsol} module of type \Id{SUNNonlinearSolver}.
Users can supply their own {\sunnonlinsol} module, or use one of the
modules provided with {\sundials}. The following section presents the
{\sunnonlinsol} API and its implementation. The subsequent sections in
this chapter describe the {\sunnonlinsol} modules provided with
{\sundials}.


% ====================================================================
\section{The SUNNonlinearSolver API}
\label{s:sunnonlinsol_api}
% ====================================================================

The {\sunnonlinsol} API defines several nonlinear solver operations
that enable {\sundials} packages to untilize any {\sunnonlinsol}
implementation that provides the required functions. These functions
can be divided into three catagories. The first are the core nonlinear
solver functions: get type, initialization, setup, solve, and free.
The second group of functions consists set functions to supply the
nonlinear solver with functions provided by the {\sundials} pacakge
and to modify solver paramters. The final group consists of get
functions for retreiving nonlinear solver statistics. All of these
functions are defined in the header
file \id{sundials/sundials\_nonlinearsolver.h}.

% ====================================================================
\subsection{SUNNonlinearSolver core functions}
\label{ss:sunnonlinsol_corefn}
% ====================================================================
\ucfunction{SUNNonlinSolGetType}
{
  type = SUNNonlinSolGetType(NLS);
}
{
  The \textit{required} function \ID{SUNNonlinSolGetType} returns the
  type identifier for the nonlinear solver \id{NLS} and is used to
  determine the solver type (rootfind or stationary).
}
{
  \begin{args}[NLS]
  \item[NLS] (\id{SUNNonlinearSolver})
    a {\sunnonlinsol} object.
  \end{args}
}
{
  The return value \id{type} (of type \id{int}) will be one of the
  following:
  \begin{args}[SUNNONLINEARSOLVER\_STATIONARY]
  \item[\Id{SUNNONLINEARSOLVER\_ROOTFIND}]
    \id{0}, the {\sunnonlinsol} module solves $F(y) = 0$.
  \item[\Id{SUNNONLINEARSOLVER\_STATIONARY}]
    \id{1}, the {\sunnonlinsol} module solves $G(y) = y$.
  \end{args}
}
{}
% --------------------------------------------------------------------
\ucfunction{SUNNonlinSolInitialize}
{
  retval = SUNNonlinSolInitialize(NLS);
}
{
  The \textit{required} function \ID{SUNNonlinSolInitialize} preforms
  nonlinear solver initialization. This function assumes all
  solver-specific options have been set prior to being called.
}
{
  \begin{args}[NLS]
  \item[NLS] (\id{SUNNonlinearSolver})
    a {\sunnonlinsol} object.
  \end{args}
}
{
  The return value \id{retval} (of type \id{int}) should be zero for a
  successful call and a negative value for a failure.
}
{}
% --------------------------------------------------------------------
\ucfunction{SUNNonlinSolSetup}
{
  retval = SUNNonlinSolSetup(NLS, y, mem);
}
{
  The \textit{optional} function \ID{SUNNonlinSolSetup} preforms any
  solver setup needed before a nonlinear solve.
}
{
  \begin{args}[NLS]
  \item[NLS] (\id{SUNNonlinearSolver})
    a {\sunnonlinsol} object.
  \item[y] (\id{N\_Vector})
    the initial iteration passed to the nonlinear solver.
  \item[mem] (\id{void *})
    the {\sundials} package memory structure.
  \end{args}
}
{
  The return value \id{retval} (of type \id{int}) should be zero for a
  successful call and a negative value for a failure.
}
{
  {\sunnonlinsol} implementations that do not require setup may set
  this operation to \id{NULL}.
}
% --------------------------------------------------------------------
\ucfunction{SUNNonlinSolSolve}
{
  retval = SUNNonlinSolSolve(NLS, y0, y, w, tol, callSetup, mem);
}
{
  The \textit{required} function \ID{SUNNonlinSolSolve} solves the
  nonlinear system $F(y)=0$ or $G(y)=y$.
}
{
  \begin{args}[callSetup]
  \item[NLS] (\id{SUNNonlinearSolver})
    a {\sunnonlinsol} object.
  \item[y0] (\id{N\_Vector})
    the initial iterate for the nonlinear solver.
  \item[y] (\id{N\_Vector})
    the solution to the nonlinear system.
  \item[w] (\id{N\_Vector})
    the weight vector used in computing weighted norms.
  \item[tol] (\id{realtype})
    the specified tolerance in a weight root mean square norm.
  \item[callSetup] (\id{booleantype})
    a flag indicating if the linear solver should be setup.
  \item[mem] (\id{void *})
    the {\sundials} package memory structure.
  \end{args}
}
{
  The return value \id{retval} (of type \id{int}) should be zero if
  the solve was successful, a positive value if a recoverable error
  occured, and a negative value for an unrecoverable failure.
}
{
  In the context of time integration, the failure of the nonlinear
  solver to converge is considered a \textit{recoverable} error and a
  positive value should be returned in this case. The time integrator
  will then reattempt the step (and the nonlinear solve) with a
  reduced time step size.
}
% --------------------------------------------------------------------
\ucfunction{SUNNonlinSolFree}
{
  retval = SUNNonlinSolFree(NLS);
}
{
  The \textit{required} function \ID{SUNNonlinSolFree} frees memory
  allocated by the nonlinear solver.
}
{
  \begin{args}[NLS]
  \item[NLS] (\id{SUNNonlinearSolver})
    a {\sunnonlinsol} object.
  \end{args}
}
{
  The return value \id{retval} (of type \id{int}) should be zero for a
  successful call, and a negative value for a failure.
}
{}


% ====================================================================
\subsection{SUNNonlinearSolver set functions}
\label{ss:sunnonlinsol_setfn}
% ====================================================================
\ucfunction{SUNNonlinSetSysFn}
{
  retval = SUNNonlinSolSetSysFn(NLS, SysFn);
}
{
  The \textit{required} function \ID{SUNNonlinSolSetSysFn} is used
  to provide the nonlinear solver with the function defining the
  nonlinear system. This is the function $F(y)=0$ for
  \wtt{SUNNONLINERASOLVER\_ROOTFIND} modules or $G(y)=y$ for
  \wtt{SUNNONLINEARSOLVER\_STATIONARY} modules.
}
{
  \begin{args}[SysFn]
  \item[NLS] (\id{SUNNonlinearSolver})
    a {\sunnonlinsol} object.
  \item[SysFn] (\id{SUNNonlinSolSysFn})
    the function defining the nonlinear system. See
    \ref{ss:sunnonlinsol_sunsuppliedfn} for the definition of
    \id{SUNNonlinSolSysFn}.
  \end{args}
}
{
  The return value \id{retval} (of type \id{int}) should be zero for a
  successful call, and a negative value for a failure.
}
{}
% --------------------------------------------------------------------
\ucfunction{SUNNonlinSetLSetupFn}
{
  retval = SUNNonlinSolSetLSetupFn(NLS, LSetupFn);
}
{
  The \textit{optional} function \ID{SUNNonlinSolLSetupFn} is used to
  provide the nonlinear solver with a wrapper function to the
  {\sundials} package's linear solver setup function.
}
{
  \begin{args}[LSetupFn]
  \item[NLS] (\id{SUNNonlinearSolver})
    a {\sunnonlinsol} object.
  \item[LSetupFn] (\id{SUNNonlinSolLSetupFn})
    a wrapper function to the {\sundials} package's linear solver
    setup function. See \ref{ss:sunnonlinsol_sunsuppliedfn} for the
    definition of \id{SUNNonlinLSetupFn}.
  \end{args}
}
{
  The return value \id{retval} (of type \id{int}) should be zero for a
  successful call, and a negative value for a failure.
}
{
  {\sunnonlinsol} implementations not utilizing {\sunlinsol} modules
  may set this operation to \id{NULL}.
}
% --------------------------------------------------------------------
\ucfunction{SUNNonlinSetLSolveFn}
{
  retval = SUNNonlinSolLSolve(NLS, LSolveFn);
}
{
  The \textit{optional} function \ID{SUNNonlinSolLSolve} is used to
  provide the nonlinear solver with a wrapper function to the
  {\sundials} package's linear solver solve function.
}
{
  \begin{args}[LSolveFn]
  \item[NLS] (\id{SUNNonlinearSolver})
    a {\sunnonlinsol} object
  \item[LSolveFn] (\id{SUNNonlinSolLSolveFn})
    a wrapper function to the {\sundials} package's linear solver
    solve function. See \ref{ss:sunnonlinsol_sunsuppliedfn} for the
    definition of \id{SUNNonlinSolLSolveFn}.
  \end{args}
}
{
  The return value \id{retval} (of type \id{int}) should be zero for a
  successful call, and a negative value for a failure.
}
{
  {\sunnonlinsol} implementations not utilizing {\sunlinsol} modules
  may set this operation to \id{NULL}.
}
% --------------------------------------------------------------------
\ucfunction{SUNNonlinSetConvTestFn}
{
  retval = SUNNonlinSolSetConvTestFn(NLS, CTestFn);
}
{
  The \textit{optional} function \ID{SUNNonlinSolSetConvTestFn} is
  used to provide the nonlinear solver with a function for determining
  if the nonlinear solver iteration has converged.
}
{
  \begin{args}[CTestFn]
  \item[NLS] (\id{SUNNonlinearSolver})
    a {\sunnonlinsol} object.
  \item[CTestFn] (\id{SUNNonlinearSolConvTestFn})
    a {\sundials} package nonlinear solver convergence test function.
    See \ref{ss:sunnonlinsol_sunsuppliedfn} for the definition of
    \id{SUNNonlinSolConvTestFn}.
  \end{args}
}
{
  The return value \id{retval} (of type \id{int}) should be zero for a
  successful call, and a negative value for a failure.
}
{
  {\sunnonlinsol} implementations utilizing their own convergence test
  criteria may set this function to \id{NULL}.
}
% --------------------------------------------------------------------
\ucfunction{SUNNonlinSetMaxIters}
{
  retval = SUNNonlinSolSetMaxIters(NLS, maxiters);
}
{
  The \textit{optional} function \ID{SUNNonlinSolSetMaxIters} sets the
  maximum number of nonlinear solver iterations.
}
{
  \begin{args}[maxiters]
  \item[NLS] (\id{SUNNonlinearSolver})
    a {\sunnonlinsol} object.
  \item[maxiters] (\id{int})
    the maximum number of nonlinear iterations.
  \end{args}
}
{
  The return value \id{retval} (of type \id{int}) should be zero for a
  successful call, and a negative value for a failure
  (e.g., $\id{maxiters} < 1$).
}
{}


% ====================================================================
\subsection{SUNNonlinearSolver get functions}
\label{ss:sunnonlinsol_getfn}
% ====================================================================
\ucfunction{SUNNonlinGetNumIters}
{
  retval = SUNNonlinSolGetNumIters(NLS, numiters);
}
{
  The \textit{optional} function \ID{SUNNonlinSolGetNumIters} to get
  the total number of nonlinear solver iterations.
}
{
  \begin{args}[numiters]
  \item[NLS] (\id{SUNNonlinearSolver})
    a {\sunnonlinsol} object
  \item[numiters] (\id{long int})
    the total number of nonlinear solver iterations.
  \end{args}
}
{
  The return value \id{retval} (of type \id{int}) should be zero for a
  successful call, and a negative value for a failure.
}
{}


% ====================================================================
\subsection{SUNDIALS package provided SUNNonlinearSolver routines}
\label{ss:sunnonlinsol_sunsuppliedfn}
% ====================================================================

The {\sundials} packages provide the {\sunnonlinsol} module with
routines for evaluating the nonlinear system, calling the {\sunlinsol}
steup and solve functions, and testing the nonlinear iteration for
convergence. The function types for each of these functions are
defined in the header file \id{sundials/sundials\_nonlinearsolver.h},
and are described below. 
% --------------------------------------------------------------------
\usfunction{SUNNonlinSolSysFn}
{
  typedef int (*SUNNonlinSolSysFn)(N\_Vector y, N\_Vector F, void* mem);
}
{
  These functions evaluate the nonlinear system $F(y)$
  for \wtt{SUNNONLINEARSOLVER\_ROOTFIND} type modules or $G(y)$
  for \wtt{SUNNONLINEARSOLVER\_STATIONARY} type modules. Memory
  for \id{F} should already be allocated prior to calling this
  function. The vector \id{y} should be left unchanged.
}
{
  \begin{args}[mem]
  \item[y]
    is the state vector at which the nonlinear system should be evaluated.
  \item[F]
    is the output vector containing $F(y)$ or $G(y)$, depending on the
    solver type.
  \item[mem]
    is the {\sundials} package memory structure.
  \end{args}
}
{
  The return value of this routine will be zero if the evaluation was
  successful, a postive value if a recoverable error occured, or a
  negative value if an unrecoverable error occured.
}
{}
% --------------------------------------------------------------------
\usfunction{SUNNonlinSolLSetupFn}
{
  typedef int (*SUNNonlinSolLSetupFn)(N\_Vector y, N\_Vector F, void* mem);
}
{
  These functions are wrappers to the {\sundials} package's function
  for setting up linear solves with {\sunlinsol} modules.
}
{
  \begin{args}[mem]
  \item[y]
    is the state vector at which the linear system should be setup.
  \item[F]
    is the value of the nonlinear system function at \id{y}.
  \item[mem]
    is the {\sundials} package memory structure.
  \end{args}
}
{
  The return value of this routine will be zero if the setup was
  successful, a postive value if a recoverable error occured, or a
  negative value if an unrecoverable error occured.
}
{
  {\sunnonlinsol} modules not utilizing {\sunlinsol} linear solvers
  may ignore these functions and will need to handle any linear system
  setup.
}
% --------------------------------------------------------------------
\usfunction{SUNNonlinSolLSolveFn}
{
  typedef int (*SUNNonlinSolLSolveFn)(N\_Vector y, N\_Vector b, void* mem);
}
{
  These functions are wrappers to the {\sundials} package's function
  for solving linear systems with {\sunlinsol} modules.
}
{
  \begin{args}[mem]
  \item[y]
    is the input vector containing the current nonlinear iteration.
  \item[b]
    contains the right-hand side vector for the linear solve on input
    and the solution to the linear system on output.
  \item[mem]
    is the {\sundials} package memory structure.
  \end{args}
}
{
  The return value of this routine will be zero if the linear solve
  was successful, a postive value if a recoverable error occured, or a
  negative value if an unrecoverable error occured.
}
{
  {\sunnonlinsol} modules not utilizing {\sunlinsol} linear solvers
  may ignore these functions and will need to handle solving linear
  systems.
}
% --------------------------------------------------------------------
\usfunction{SUNNonlinSolConvTestFn}
{
  typedef int (*SUNNonlinSolConvTestFn)(&int m, realtype delnrm,\\
                                        &realtype tol, void* mem);
}
{
  These functions are {\sundials} package specifc convergence tests of
  nonlinear solvers.
}
{
  \begin{args}[delnrm]
  \item[m]
    is the value of the nonlinear iteration counter, starting from
  zero for the inital iteration.
  \item[delnrm]
    is the WRMS norm of the iteration update vector $\delta^{(m)}$ in
    $y^{(m+1)} = y^{(m)} + \delta^{(m+1)}$.
  \item[tol]
    is the nonlinear solver tolerance.
  \item[mem]
    is the {\sundials} package memory structure.
  \end{args}
}
{
  The return value of this routine will be one of the following: 
  \begin{args}[SUN\_NLS\_CONV\_RECVR]
  \item[\id{SUN\_NLS\_SUCCESS}]
    the iteration is converged.
  \item[\id{SUN\_NLS\_CONTINUE}]
    the iteration is not converged.
  \item[\id{SUN\_NLS\_CONV\_RECVR}]
    the iteration appears to be diverging.
  \end{args}
  or a negative value if an unrecoverable error occured
  (i.e., \id{mem} is \id{NULL}).
}
{
  {\sunnonlinsol} modules utilizing their own convergence criteria may
  ignore these functions.
}


% ====================================================================
\subsection{SUNNonlinearSolver return codes}
\label{ss:sunnonlinsol_returncodes}
% ====================================================================

The functions provided to {\sunnonlinsol} modules and functions within
the {\sundials}-provided {\sunnonlinsol} implementations utilize a
common set of return codes, shown below in Table \ref{t:sunnonlinsol_returncodes}.

\newlength{\ColumnOneA}
\settowidth{\ColumnOneA}{\id{SUN\_NLS\_CONV\_RECVR}}
\newlength{\ColumnTwoA}
\settowidth{\ColumnTwoA}{\id{Value}}
\newlength{\ColumnThreeA}
\setlength{\ColumnThreeA}{\textwidth}
\addtolength{\ColumnThreeA}{-0.5in}
\addtolength{\ColumnThreeA}{-\ColumnOneA}
\addtolength{\ColumnThreeA}{-\ColumnTwoA}

\tablecaption{Description of the \id{SUNNonlinearSolver} return codes}\label{t:sunnonlinsol_returncodes}
\tablehead{\hline {\rule{0mm}{5mm}}{\bf Name} & {\bf Value} & {\bf Description} \\[3mm] \hline\hline}
\tabletail{\hline \multicolumn{3}{|r|}{\small\slshape continued on next page} \\ \hline}
\begin{xtabular}{|p{\ColumnOneA}|p{\ColumnTwoA}|p{\ColumnThreeA}|}
%%
\id{SUN\_NLS\_SUCCESS}     & \id{0}  & successful call or converged solve
\\[1mm]
%%
\id{SUN\_NLS\_CONTINUE}    & \id{1}  & the nonlinear solver is not
                                      converged, keep iterating 
\\[1mm]
%%
\id{SUN\_NLS\_CONV\_RECVR} & \id{2}  & the nonlinear solver appears to
                                       be diverging, try to recover
\\[1mm]
%%
\id{SUN\_NLS\_MEM\_NULL}   & \id{-1} & a memory argument is \id{NULL}
\\[1mm]
%%
\id{SUN\_NLS\_MEM\_FAIL}   & \id{-2} & a memory access or allocation failed
\\[1mm]
%%
\id{SUN\_NLS\_ILL\_INPUT}  & \id{-3} & an illegal input option was provided
\\
\end{xtabular}
\bigskip


% ====================================================================
\subsection{The generic SUNNonlinearSolver module}
\label{ss:sunnonlinsol_generic}
% ====================================================================

{\sundials} packages interact with a specific {\sunnonlinsol}
implementation through the generic {\sunnonlinsol} module on which all
other {\sunnonlinsol} are built. The \wtt{SUNNonlinearSolver} type is
a pointer to a structure containing an implementation-dependent
\textit{content} field and an \textit{ops} field. The type
\wtt{SUNNonlinearSolver} is defined as follows:
%%
%%
\begin{verbatim}
typedef struct _generic_SUNNonlinearSolver *SUNNonlinearSolver;

struct _generic_SUNNonlinearSolver {
  void *content;
  struct _generic_SUNNonlinearSolver_Ops *ops;
};
\end{verbatim}
%%
%%
where the \wtt{\_generic\_SUNLinearSolver\_Ops} structure is a list of 
pointers to the various actual linear solver operations provided by a
specific implemenation. The \wtt{\_generic\_SUNLinearSolver\_Ops}
structure is defined as
%%
%%
\begin{verbatim}
struct _generic_SUNNonlinearSolver_Ops {
  SUNNonlinearSolver_Type (*gettype)(SUNNonlinearSolver);
  int                     (*initialize)(SUNNonlinearSolver);
  int                     (*setup)(SUNNonlinearSolver, N_Vector, void*);
  int                     (*solve)(SUNNonlinearSolver, N_Vector, N_Vector,
                                   N_Vector, realtype, booleantype, void*);
  int                     (*free)(SUNNonlinearSolver);
  int                     (*setsysfn)(SUNNonlinearSolver, SUNNonlinSolSysFn);
  int                     (*setlsetupfn)(SUNNonlinearSolver, SUNNonlinSolLSetupFn);
  int                     (*setlsolvefn)(SUNNonlinearSolver, SUNNonlinSolLSolveFn);
  int                     (*setctestfn)(SUNNonlinearSolver, SUNNonlinSolConvTestFn);
  int                     (*setmaxiters)(SUNNonlinearSolver, int);
  int                     (*getnumiters)(SUNNonlinearSolver, long int*);
};
\end{verbatim}
%%
%%
The generic sunlinsol module defines and implements the nonlinear
solver operations defined in Sections \ref{ss:sunnonlinsol_corefn}
-- \ref{ss:sunnonlinsol_getfn}. These routines are in fact only
wrappers to the nonlinear solver operations provided by a particular
{\sunnonlinsol} implementation, which are accessed through the ops
field of the \id{SUNNonlinearSolver} structure. To illustrate this
point we show below the implementation of a typical nonlinear solver
operation from the generic \{sunnonlinsol} module,
namely \wtt{SUNNonlinSolInitialize}, which initializes a
{\sunnonlinsol} object for use after it has been created and
configured, and returns a flag denoting a successful/failed operation:
%%
%%
\begin{verbatim}
int SUNNonlinSolInitialize(SUNNonlinearSolver NLS)
{
  return ((int) NLS->ops->initialize(NLS));
}
\end{verbatim}


% ====================================================================
\subsection{Implementing a SUNNonlinearSolver Module}
\label{ss:sunnonlinsol_custom}
% ====================================================================

A {\sunnonlinsol} implementation \textit{must} do the following:
\begin{enumerate}
\item Specify the content of the {\sunnonlinsol} module.
\item Define and implement the required nonlinear solver operations
  defined in Sections \ref{ss:sunnonlinsol_corefn}
  -- \ref{ss:sunnonlinsol_getfn}. Note that the names of the module
  routines should be unique to that implementation in order to permit
  using more than one {\sunlinsol} module (each with different
  \wtt{SUNLinearSolver} internal data representations) in
  the same code.
\item Define and implement a user-callable constructor to create a
  \wtt{SUNNonlinearSolver} object.
\end{enumerate}
Additionally, a {\sunnonlinsol} implementation \textit{may} do the
following:
\begin{enumerate}
\item Define and implement additional user-callable routines acting on
  a newly created \wtt{SUNNonlinearSolver} object that are not defined
  in the {\sunnonlinsol} API. For example, routines to set various
  configuration options for tuning the performance of the nonlinear
  solver.
\item Provide functions as needed for the particular {\sunnonlinsol}
  implementation to access different parts in the {\em content}
  structure of the newly defined \wtt{SUNLinearSolver} object 
  (e.g., routines to return various statistics from the solver).
\end{enumerate}

