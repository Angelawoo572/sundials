%% This is a shared SUNDIALS TEX file with a description of the
%% diagonal sunmatrix implementation
%%

The diagonal implementation of the {\sunmatrix} module provided with
{\sundials}, {\sunmatdiag}, defines the {\em content} field
of \id{SUNMatrix} to be the following structure:
%%
\begin{verbatim} 
struct _SUNMatrixContent_Diagonal {
  N_Vector d;
};
\end{verbatim}
%%
The entry of the \emph{content} field contains the following
information:
\begin{description}
  \item[d] - generic {\nvector} object containing the diagonal of the
  {\sunmatrix}.
\end{description}

\noindent The header file to be included when using this module 
is \id{sunmatrix/sunmatrix\_diagonal.h}. \\

\noindent The following macros are provided to access the
content of a {\sunmatdiag} matrix. The prefix \id{SM\_} in the names
denotes that these macros are for \emph{SUNMatrix} implementations,
and the suffix \id{\_DIAG} denotes that these are specific to
the \emph{diagonal} version.
%%
\begin{itemize}

\item \ID{SM\_CONTENT\_DIAG}
    
  This routine gives access to the contents of the
  diagonal \id{SUNMatrix}.
  
  The assignment \id{A\_cont} $=$ \id{SM\_CONTENT\_DIAG(A)} sets
  \id{A\_cont} to be a pointer to the diagonal \id{SUNMatrix} content
  structure.                                             
                                                            
  Implementation: 
  
  \verb|#define SM_CONTENT_DIAG(A)  ( (SUNMatrixContent_Diagonal)(A->content) )|
  
\item \ID{SM\_DATA\_DIAG}
                                                            
  This macro gives access to the {\nvector} \id{d} that defines the
  diagonal matrix.

  The assignment \id{A\_data = SM\_DATA\_DIAG(A)} sets \id{A\_data} to be     
  the {\nvector} storing the matrix diagonal.  The
  assignment \id{SM\_DATA\_DIAG(A) = A\_data} sets the {\nvector}
  matrix diagonal to \id{A\_data}.
  
  Implementation:

  \verb|#define SM_DATA_DIAG(A)     ( SM_CONTENT_DIAG(A)->d )|

\end{itemize}
%%
%%----------------------------------------------
%%
The {\sunmatdiag} module defines diagonal implementations of all
matrix operations listed in Table \ref{t:sunmatops}. Their names are
obtained from those in Table \ref{t:sunmatops} by appending the
suffix \id{\_Diagonal} (e.g. \id{SUNMatCopy\_Diagonal}). 
The module {\sunmatdiag} provides the following additional
user-callable routines: 
%%
\begin{itemize}

%%--------------------------------------

\item \ID{SUNDiagonalMatrix}

  This constructor function creates and allocates memory for a diagonal \id{SUNMatrix}.
  Its argument is a template {\nvector} object.

  \verb|SUNMatrix SUNDiagonalMatrix(N_Vector tmpl);|

%%--------------------------------------

\item \ID{SUNDiagonalMatrix\_Diag}

  This function returns the {\nvector} containing the diagonal of
  the \id{SUNMatrix}. 
 
  \verb|N_Vector SUNDiagonalMatrix_Diag(SUNMatrix A);|

\end{itemize}
%%
%%------------------------------------
%%
\paragraph{\bf Notes}                                                      
           
\begin{itemize}
                                        
\item
  To access the components of a diagonal \id{SUNMatrix} \id{A}, you
  must first retrieve the {\nvector} containing the matrix diagonal
  via \id{A\_data = SM\_DATA\_DIAG(A)} or\\
  \id{A\_data = SUNDiagonalMatrix\_Diag(A)}, and then access the
  entries of the {\nvector} using whichever macros or functions are
  appropriate for that {\nvector} implementation.

\item
  Within the \id{SUNMatMatvec\_Diagonal} routine, internal consistency
  checks are performed to ensure that the matrix is called with
  consistent {\nvector} implementations.  This consistency check
  merely ensures that the matrix diagonal, as well as the input 
  {\nvector} objects \id{x} and \id{y} all have an
  identical \id{N\_Vector\_ID}.  As such, this routine may be used
  with any {\sundials}-supplied or user-supplied {\nvector}
  implementation. 

\end{itemize}

For solvers that include a Fortran interface module, the {\sunmatdiag}
module also includes the Fortran-callable
function \id{FSUNDiagonalMatInit(code, ier)} to initialize
this {\sunmatdiag} module for a given {\sundials} solver.
Here \id{code} is an integer input solver id (1 for {\cvode}, 2 for {\ida}, 3
for {\kinsol}, 4 for {\arkode}); and \id{ier} is an error return flag 
equal to 0 for success and -1 for failure. Both \id{code} and \id{ier}
are declared to match C type \id{int}. This routine must be
called \emph{after} a corresponding {\nvector} has been initialized by
a call to \id{FNVInit*}. Additionally, when using {\arkode} with a
non-identity mass matrix, the Fortran-callable
function \id{FSUNDiagonalMassMatInit(ier)} initializes this
{\sunmatdiag} module for storing the mass matrix.
