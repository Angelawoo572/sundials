%% This is a shared SUNDIALS TEX file with a description of the
%% pcg sunlinsol implementation
%%

The {\pcg} (Preconditioned Conjugate Gradient \cite{HeSt:52})
implementation of the {\sunlinsol} module provided with {\sundials},
{\sunlinsolpcg}, is an iterative linear solver that is designed to be
compatible with any {\nvector} implementation (serial, threaded,
parallel, and user-supplied) that supports a minimal subset of operations
(\id{N\_VClone}, \id{N\_VDotProd}, \id{N\_VScale}, \id{N\_VLinearSum},
\id{N\_VProd}, and \id{N\_VDestroy}).  Unlike the {\spgmr} and {\spfgmr}
algorithms, {\pcg} requires a fixed amount of memory that does not
increase with the number of allowed iterations.

Unlike all of the other iterative linear solvers supplied with
{\sundials}, {\pcg} should only be used on \emph{symmetric} linear
systems (e.g.~mass matrix linear systems encountered in
{\arkode}). As a result, the explanation of the role of scaling and
preconditioning matrices given in general must be modified in this
scenario.  The {\pcg} algorithm solves a linear system $Ax = b$ where  
$A$ is a symmetric ($A^T=A$), real-valued matrix.  Preconditioning is
allowed, and is applied in a symmetric fashion on both the right and
left.  Scaling is also allowed and is applied symmetrically.  We
denote the preconditioner and scaling matrices as follows:
\begin{itemize}
\item $P$ is the preconditioner (assumed symmetric),
\item $S$ is a diagonal matrix of scale factors.
\end{itemize}
The matrices $A$ and $P$ are not required explicitly; only routines
that provide $A$ and $P^{-1}$ as operators are required.  The diagonal
of the matrix $S$ is held in a single {\nvector}, supplied by the user.

In this notation, {\pcg} applies the underlying CG algorithm to the
equivalent transformed system 
\begin{equation}
  \label{eq:transformed_linear_systemPCG}
  \tilde{A} \tilde{x} = \tilde{b}
\end{equation}
where
\begin{align}
  \notag
  \tilde{A} &= S P^{-1} A P^{-1} S,\\
  \label{eq:transformed_linear_system_componentsPCG}
  \tilde{b} &= S P^{-1} b,\\
  \notag
  \tilde{x} &= S^{-1} P x.
\end{align} 
The scaling matrix must be chosen so that the vectors $SP^{-1}b$ and
$S^{-1}Px$ have dimensionless components.

The stopping test for the PCG iterations is on the L2 norm of the
scaled preconditioned residual:
\begin{align*}
  &\| \tilde{b} - \tilde{A} \tilde{x} \|_2  <  \delta\\
  \Leftrightarrow\quad &\\
  &\| S P^{-1} b - S P^{-1} A x \|_2  <  \delta\\
  \Leftrightarrow\quad &\\
  &\| P^{-1} b - P^{-1} A x \|_S  <  \delta
\end{align*}
where $\| v \|_S = \sqrt{v^T S^T S v}$, with an input tolerance $\delta$.

The {\sunlinsolpcg} module defines the {\em content} field of a
\id{SUNLinearSolver} to be the following structure:
%%
\begin{verbatim} 
struct _SUNLinearSolverContent_PCG {
  int maxl;
  int pretype;
  int numiters;
  realtype resnorm;
  long int last_flag;
  ATimesFn ATimes;
  void* ATData;
  PSetupFn Psetup;
  PSolveFn Psolve;
  void* PData;
  N_Vector s;
  N_Vector r;
  N_Vector p;
  N_Vector z;
  N_Vector Ap;
};
\end{verbatim}
%%
These entries of the \emph{content} field contain the following
information:
\begin{description}
  \item[maxl] - number of {\pcg} iterations to allow (default is 5),
  \item[pretype] - flag for use of preconditioning (default is none),
  \item[numiters] - number of iterations from the most-recent solve,
  \item[resnorm] - final linear residual norm from the most-recent solve,
  \item[last\_flag] - last error return flag from an internal function,
  \item[ATimes] - function pointer to perform $Av$ product,
  \item[ATData] - pointer to structure for \id{ATimes},
  \item[Psetup] - function pointer to preconditioner setup routine,
  \item[Psolve] - function pointer to preconditioner solve routine,
  \item[PData] - pointer to structure for \id{Psetup} and \id{Psolve},
  \item[s] - vector pointer for supplied scaling matrix
    (default is \id{NULL}),
  \item[r] - a {\nvector} which holds the preconditioned linear system
    residual,
  \item[p, z, Ap] - {\nvector}s used for workspace by the
    {\pcg} algorithm. 
\end{description}

This solver is constructed to perform the following operations:
\begin{itemize}
\item During construction all {\nvector} solver data is allocated,
  with vectors cloned from a template {\nvector} that is input, and
  default solver parameters are set.
\item User-facing ``set'' routines may be called to modify default
  solver parameters.
\item Additional ``set'' routines are called by the {\sundials} solver
  that interfaces with {\sunlinsolpcg} to supply the 
  \id{ATimes}, \id{PSetup}, and \id{Psolve} function pointers and
  \id{s} scaling vector.
\item In the ``initialize'' call, the solver parameters are checked
  for validity.
\item In the ``setup'' call, any non-\id{NULL} \id{PSetup} function is
  called.  Typically, this is provided by the {\sundials} solver
  itself, that translates between the generic \id{PSetup} function and
  the solver-specific routine (solver-supplied or user-supplied).
\item In the ``solve'' call the {\pcg} iteration is performed.  This
  will include scaling and preconditioning if those options have been
  supplied.
\end{itemize}

\noindent The header file to include when using this module 
is \id{sunlinsol/sunlinsol\_pcg.h}. The {\sunlinsolpcg} module
is accessible from all {\sundials} solvers \textit{without}
linking to the \\
\id{libsundials\_sunlinsolpcg} module library. \\

%%
%%----------------------------------------------
%%

\noindent The {\sunlinsolpcg} module defines implementations of all
``iterative'' linear solver operations listed in Table
\ref{t:sunlinsolops}:
\begin{itemize}
\item \id{SUNLinSolGetType\_PCG}
\item \id{SUNLinSolInitialize\_PCG}
\item \id{SUNLinSolSetATimes\_PCG}
\item \id{SUNLinSolSetPreconditioner\_PCG}
\item \id{SUNLinSolSetScalingVectors\_PCG} -- since {\pcg} only
  supports symmetric scaling, the second {\nvector} argument to this
  function is ignored
\item \id{SUNLinSolSetup\_PCG}
\item \id{SUNLinSolSolve\_PCG}
\item \id{SUNLinSolNumIters\_PCG}
\item \id{SUNLinSolResNorm\_PCG}
\item \id{SUNLinSolResid\_PCG}
\item \id{SUNLinSolLastFlag\_PCG}
\item \id{SUNLinSolSpace\_PCG}
\item \id{SUNLinSolFree\_PCG}
\end{itemize}
The module {\sunlinsolpcg} provides the following additional
user-callable routines: 
%%
\begin{itemize}

%%--------------------------------------

\item \ID{SUNPCG}

  This constructor function creates and allocates memory for a {\pcg}
  \id{SUNLinearSolver}.  Its arguments are an {\nvector}, a flag
  indicating to use preconditioning, and the number of linear
  iterations to allow. 

  This routine will perform consistency checks to ensure that it is
  called with a consistent {\nvector} implementation (i.e.~that it
  supplies the requisite vector operations).  If \id{y} is
  incompatible then this routine will return \id{NULL}.

  A \id{maxl} argument that is $\le0$ will result in the default
  value (5).

  Since the {\pcg} algorithm is designed to only support symmetric
  preconditioning, then any of the \id{pretype} inputs \id{PREC\_LEFT}
  (1), \id{PREC\_RIGHT} (2), or \id{PREC\_BOTH} (3) will result in use
  of the symmetric preconditioner;  any other integer input will
  result in the default (no preconditioning).
  Although some {\sundials} solvers are designed to only work
  with left preconditioning ({\ida} and {\idas}) and others with only
  right preconditioning ({\kinsol}), {\pcg} should \emph{only} be used
  with these packages when the linear systems are known to
  be \emph{symmetric}.  Since the scaling of matrix rows and columns
  must be identical in a symmetric matrix, symmetric preconditioning
  should work appropriately even for packages designed with one-sided
  preconditioning in mind.

  \verb|SUNLinearSolver SUNPCG(N_Vector y, int pretype, int maxl);|

%%--------------------------------------

\item \ID{SUNPCGSetPrecType}

  This function updates the flag indicating use of preconditioning.
  As above, any one of the input values, \id{PREC\_LEFT} (1),
  \id{PREC\_RIGHT} (2), or \id{PREC\_BOTH} (3) will enable
  preconditioning; \id{PREC\_NONE} (0) disables preconditioning.

  This routine will return with one of the error codes
  \id{SUNLS\_ILL\_INPUT} (illegal \id{pretype}), \id{SUNLS\_MEM\_NULL}
  (\id{S} is \id{NULL}), or \id{SUNLS\_SUCCESS}.
  
  \verb|int SUNPCGSetPrecType(SUNLinearSolver S, int pretype);|

%%--------------------------------------

\item \ID{SUNPCGSetMaxl}

  This function updates the number of linear solver iterations to
  allow. 

  A \id{maxl} argument that is $\le0$ will result in the default
  value (5).

  This routine will return with one of the error codes
  \id{SUNLS\_MEM\_NULL} (\id{S} is \id{NULL}) or \id{SUNLS\_SUCCESS}.
  
  \verb|int SUNPCGSetMaxl(SUNLinearSolver S, int maxl);|

\end{itemize}
%%
%%------------------------------------
%%
For solvers that include a Fortran interface module, the
{\sunlinsolpcg} module also includes the Fortran-callable
function \id{FSUNPCGInit(code, pretype, maxl, ier)} to initialize
this {\sunlinsolpcg} module for a given {\sundials} solver.
Here \id{code} is an integer input solver id (1 for {\cvode}, 2 for {\ida}, 3
for {\kinsol}, 4 for {\arkode}); \id{pretype} and \id{maxl} are the
same as for the C function \ID{SUNPCG}; \id{ier} is an error return
flag equal to 0 for success and -1 for failure.  All of these input
arguments should be declared so as to match C type \id{int}.  This
routine must be called \emph{after} the {\nvector} object has been
initialized.  Additionally, when using {\arkode} with a non-identity
mass matrix, the Fortran-callable function 
\id{FSUNMassPCGInit(pretype, maxl, ier)} initializes this
{\sunlinsolpcg} module for solving mass matrix linear systems.

The \id{SUNPCGSetPrecType} and \id{SUNPCGSetMaxl} routines also
support Fortran interfaces for the system and mass matrix solvers (all
arguments should be commensurate with a C \id{int}):
\begin{itemize}
\item \id{FSUNPCGSetPrecType(code, pretype, ier)}
\item \id{FSUNMassPCGSetPrecType(pretype, ier)}
\item \id{FSUNPCGSetMaxl(code, maxl, ier)}
\item \id{FSUNMassPCGSetMaxl(maxl, ier)}
\end{itemize}
