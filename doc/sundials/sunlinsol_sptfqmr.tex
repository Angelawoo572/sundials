%% This is a shared SUNDIALS TEX file with a description of the
%% sptfqmr sunlinsol implementation
%%

The {\sptfqmr} (Scaled, Preconditioned, Transpose-Free Quasi-Minimum
Residual \cite{Fre:93}) implementation of the {\sunlinsol} module 
provided with {\sundials}, {\sunlinsolsptfqmr}, is an iterative linear
solver that is designed to be compatible with any {\nvector}
implementation (serial, threaded, parallel, and user-supplied) that
supports a minimal subset of operations (\id{N\_VClone}, 
\id{N\_VDotProd}, \id{N\_VScale}, \id{N\_VLinearSum}, \id{N\_VProd},
\id{N\_VConst}, \id{N\_VDiv}, and \id{N\_VDestroy}).  Unlike the
{\spgmr} and {\spfgmr} algorithms, {\sptfqmr} requires a fixed amount of
memory that does not increase with the number of allowed iterations.

The {\sunlinsolsptfqmr} module defines the {\em content} field of a
\id{SUNLinearSolver} to be the following structure:
%%
\begin{verbatim} 
struct _SUNLinearSolverContent_SPTFQMR {
  int maxl;
  int pretype;
  int numiters;
  realtype resnorm;
  long int last_flag;
  ATimesFn ATimes;
  void* ATData;
  PSetupFn Psetup;
  PSolveFn Psolve;
  void* PData;
  N_Vector s1;
  N_Vector s2;
  N_Vector r_star;
  N_Vector q;
  N_Vector d;
  N_Vector v;
  N_Vector p;
  N_Vector *r;
  N_Vector u;
  N_Vector vtemp1;
  N_Vector vtemp2;
  N_Vector vtemp3;
};
\end{verbatim}
%%
These entries of the \emph{content} field contain the following
information:
\begin{description}
  \item[maxl] - number of TFQMR iterations to allow (default is 5),
  \item[pretype] - flag for type of preconditioning to employ
    (default is none),
  \item[numiters] - number of iterations from the most-recent solve,
  \item[resnorm] - final linear residual norm from the most-recent solve,
  \item[last\_flag] - last error return flag from an internal function,
  \item[ATimes] - function pointer to perform $Av$ product,
  \item[ATData] - pointer to structure for \id{ATimes},
  \item[Psetup] - function pointer to preconditioner setup routine,
  \item[Psolve] - function pointer to preconditioner solve routine,
  \item[PData] - pointer to structure for \id{Psetup} and \id{Psolve},
  \item[s1, s2] - vector pointers for supplied scaling matrices
    (default is \id{NULL}),
  \item[r\_star] - a {\nvector} which holds the initial scaled,
    preconditioned linear system residual,
  \item[q, d, v, p, u] - {\nvector}s used for workspace by the SPTFQMR
    algorithm,
  \item [r] - array of two {\nvector}s used for workspace within the
    SPTFQMR algorithm,
  \item[vtemp1, vtemp2, vtemp3] - temporary vector storage.
\end{description}

This solver is constructed to perform the following operations:
\begin{itemize}
\item During construction all {\nvector} solver data is allocated,
  with vectors cloned from a template {\nvector} that is input, and
  default solver parameters are set.
\item User-facing ``set'' routines may be called to modify default
  solver parameters.
\item Additional ``set'' routines are called by the {\sundials} solver
  that interfaces with {\sunlinsolsptfqmr} to supply the 
  \id{ATimes}, \id{PSetup}, and \id{Psolve} function pointers and
  \id{s1} and \id{s2} scaling vectors.
\item In the ``initialize'' call, the solver parameters are checked
  for validity.
\item In the ``setup'' call, any non-\id{NULL} 
  \id{PSetup} function is called.  Typically, this is provided by
  the {\sundials} solver itself, that translates between the
  generic \id{PSetup} function and the
  solver-specific routine (solver-supplied or user-supplied).
\item In the ``solve'' call the TFQMR iteration is performed.  This
  will include scaling and preconditioning if those options have been
  supplied.
\end{itemize}

\noindent The header file to be included when using this module 
is \id{sunlinsol/sunlinsol\_sptfqmr.h}. \\
%%
%%----------------------------------------------
%%
The {\sunlinsolsptfqmr} module defines implementations of all
``iterative'' linear solver operations listed in Table
\ref{t:sunlinsolops}:
\begin{itemize}
\item \id{SUNLinSolGetType\_SPTFQMR}
\item \id{SUNLinSolInitialize\_SPTFQMR}
\item \id{SUNLinSolSetATimes\_SPTFQMR}
\item \id{SUNLinSolSetPreconditioner\_SPTFQMR}
\item \id{SUNLinSolSetScalingVectors\_SPTFQMR}
\item \id{SUNLinSolSetup\_SPTFQMR}
\item \id{SUNLinSolSolve\_SPTFQMR}
\item \id{SUNLinSolNumIters\_SPTFQMR}
\item \id{SUNLinSolResNorm\_SPTFQMR}
\item \id{SUNLinSolResid\_SPTFQMR}
\item \id{SUNLinSolLastFlag\_SPTFQMR}
\item \id{SUNLinSolSpace\_SPTFQMR}
\item \id{SUNLinSolFree\_SPTFQMR}
\end{itemize}
The module {\sunlinsolsptfqmr} provides the following additional
user-callable routines: 
%%
\begin{itemize}

%%--------------------------------------

\item \ID{SUNSPTFQMR}

  This constructor function creates and allocates memory for a {\sptfqmr}
  \id{SUNLinearSolver}.  Its arguments are an {\nvector}, the desired
  type of preconditioning, and the number of linear iterations to
  allow.

  This routine will perform consistency checks to ensure that it is
  called with a consistent {\nvector} implementation (i.e.~that it
  supplies the requisite vector operations).  If \id{y} is
  incompatible, then this routine will return \id{NULL}.

  A \id{maxl} argument that is $\le0$ will result in the default
  value (5).

  Allowable inputs for \id{pretype} are \id{PREC\_NONE} (0),
  \id{PREC\_LEFT} (1), \id{PREC\_RIGHT} (2) and \id{PREC\_BOTH} (3);
  any other integer input will result in the default (no
  preconditioning).
  We note that some {\sundials} solvers are designed to only work
  with left preconditioning ({\ida} and {\idas}) and others with only
  right preconditioning ({\kinsol}). While it is possible to configure
  a {\sunlinsolsptfqmr} object to use any of the preconditioning options
  with these solvers, this use mode is not supported and may result in
  inferior performance.

  \verb|SUNLinearSolver SUNSPTFQMR(N_Vector y, int pretype, int maxl);|

%%--------------------------------------

\item \ID{SUNSPTFQMRSetPrecType}

  This function updates the type of preconditioning to use.  Supported
  values are \id{PREC\_NONE} (0), \id{PREC\_LEFT} (1),
  \id{PREC\_RIGHT} (2), and \id{PREC\_BOTH} (3).  

  This routine will return with one of the error codes
  \id{SUNLS\_ILL\_INPUT} (illegal \id{pretype}), \id{SUNLS\_MEM\_NULL}
  (\id{S} is \id{NULL}), or \id{SUNLS\_SUCCESS}.
  
  \verb|int SUNSPTFQMRSetPrecType(SUNLinearSolver S, int pretype);|

%%--------------------------------------

\item \ID{SUNSPTFQMRSetMaxl}

  This function updates the number of linear solver iterations to allow.  

  A \id{maxl} argument that is $\le0$ will result in the default
  value (5).

  This routine will return with one of the error codes
  \id{SUNLS\_MEM\_NULL} (\id{S} is \id{NULL}) or \id{SUNLS\_SUCCESS}.
  
  \verb|int SUNSPTFQMRSetMaxl(SUNLinearSolver S, int maxl);|

\end{itemize}
%%
%%------------------------------------
%%
For solvers that include a Fortran interface module, the
{\sunlinsolsptfqmr} module also includes the Fortran-callable
function \id{FSUNSPTFQMRInit(code, pretype, maxl, ier)} to initialize
this {\sunlinsolsptfqmr} module for a given {\sundials} solver.
Here \id{code} is an integer input solver id (1 for {\cvode}, 2 for {\ida}, 3
for {\kinsol}, 4 for {\arkode}); \id{pretype} and \id{maxl} are the
same as for the C function \ID{SUNSPTFQMR}; \id{ier} is an error return
flag equal to 0 for success and -1 for failure.  All of these input
arguments should be declared so as to match C type \id{int}.  This
routine must be called \emph{after} the {\nvector} object has been
initialized.  Additionally, when using {\arkode} with a non-identity
mass matrix, the Fortran-callable function 
\id{FSUNMassSPTFQMRInit(pretype, maxl, ier)} initializes this
{\sunlinsolsptfqmr} module for solving mass matrix linear systems.

The \id{SUNSPTFQMRSetPrecType} and \id{SUNSPTFQMRSetMaxl} routines also
support Fortran interfaces for the system and mass matrix solvers (all
arguments should be commensurate with a C \id{int}):
\begin{itemize}
\item \id{FSUNSPTFQMRSetPrecType(code, pretype, ier)} 
\item \id{FSUNMassSPTFQMRSetPrecType(pretype, ier)}
\item \id{FSUNSPTFQMRSetMaxl(code, maxl, ier)}
\item \id{FSUNMassSPTFQMRSetMaxl(maxl, ier)}
\end{itemize}
