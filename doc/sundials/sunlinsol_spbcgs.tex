% ====================================================================
\section{The SUNLinearSolver\_SPBCGS implementation}
\label{ss:sunlinsol_spbcgs}
% ====================================================================

This section describes the {\sunlinsol} implementation of the {\spbcgs}
(Scaled, Preconditioned, Bi-Conjugate Gradient, Stabilized \cite{Van:92})
iterative linear solver. The {\sunlinsolspbcgs} module is designed to be
compatible with any {\nvector} implementation that supports a minimal subset
of operations (\id{N\_VClone}, \id{N\_VDotProd}, \id{N\_VScale},
\id{N\_VLinearSum}, \id{N\_VProd}, \id{N\_VDiv}, and
\id{N\_VDestroy}). Unlike the {\spgmr} and {\spfgmr} algorithms, {\spbcgs}
requires a fixed amount of memory that does not increase with the number of
allowed iterations.

To access the {\sunlinsolspbcgs} module, include the header file
\id{sunlinsol/sunlinsol\_spbcgs.h}. We note that the {\sunlinsolspbcgs} module is
accessible from {\sundials} packages \textit{without} separately linking to
the \id{libsundials\_sunlinsolspbcgs} module library.


% ====================================================================
\subsection{SUNLinearSolver\_SPBCGS description}
\label{ss:sunlinsol_spbcgs_description}
% ====================================================================

This solver is constructed to perform the following operations:
\begin{itemize}
\item During construction all {\nvector} solver data is allocated,
  with vectors cloned from a template {\nvector} that is input, and
  default solver parameters are set.
\item User-facing ``set'' routines may be called to modify default
  solver parameters.
\item Additional ``set'' routines are called by the {\sundials} solver
  that interfaces with {\sunlinsolspbcgs} to supply the
  \id{ATimes}, \id{PSetup}, and \id{Psolve} function pointers and
  \id{s1} and \id{s2} scaling vectors.
\item In the ``initialize'' call, the solver parameters are checked
  for validity.
\item In the ``setup'' call, any non-\id{NULL}
  \id{PSetup} function is called.  Typically, this is provided by
  the {\sundials} solver itself, that translates between the
  generic \id{PSetup} function and the
  solver-specific routine (solver-supplied or user-supplied).
\item In the ``solve'' call the {\spbcgs} iteration is performed.  This
  will include scaling and preconditioning if those options have been
  supplied.
\end{itemize}


% ====================================================================
\subsection{SUNLinearSolver\_SPBCGS functions}
\label{ss:sunlinsol_spbcgs_functions}
% ====================================================================

The {\sunlinsolspbcgs} module provides the following user-callable constructor
for creating a \newline \id{SUNLinearSolver} object.
%
% --------------------------------------------------------------------
%
\ucfunctiondf{SUNLinSol\_SPBCGS}
{
  LS = SUNLinSol\_SPBCGS(y, pretype, maxl);
}
{
  The function \ID{SUNLinSol\_SPBCGS} creates and allocates memory for
  a {\spbcgs} \newline \id{SUNLinearSolver} object.
}
{
  \begin{args}[pretype]
  \item[y] (\id{N\_Vector})
    a template for cloning vectors needed within the solver
  \item[pretype] (\id{int})
    flag indicating the desired type of preconditioning, allowed
    values are:
    \begin{itemize}
    \item \id{PREC\_NONE} (0)
    \item \id{PREC\_LEFT} (1)
    \item \id{PREC\_RIGHT} (2)
    \item \id{PREC\_BOTH} (3)
    \end{itemize}
    Any other integer input will result in the default (no
    preconditioning).
  \item[maxl] (\id{int})
    the number of linear iterations to allow. Values $\le0$ will
    result in the default value (5).
  \end{args}
}
{
  This returns a \id{SUNLinearSolver} object.  If either \id{y} is
  incompatible then this routine will return \id{NULL}.
}
{
  This routine will perform consistency checks to ensure that it is
  called with a consistent {\nvector} implementation (i.e.~that it
  supplies the requisite vector operations).  If \id{y} is
  incompatible, then this routine will return \id{NULL}.

  We note that some {\sundials} solvers are designed to only work
  with left preconditioning ({\ida} and {\idas}) and others with only
  right preconditioning ({\kinsol}). While it is possible to configure
  a {\sunlinsolspbcgs} object to use any of the preconditioning options
  with these solvers, this use mode is not supported and may result in
  inferior performance.
}
{SUNSPBCGS}
%
% --------------------------------------------------------------------
%
The {\sunlinsolspbcgs} module defines implementations of all
``iterative'' linear solver operations listed in Sections
\ref{ss:sunlinsol_CoreFn} -- \ref{ss:sunlinsol_GetFn}:
\begin{itemize}
\item \id{SUNLinSolGetType\_SPBCGS}
\item \id{SUNLinSolInitialize\_SPBCGS}
\item \id{SUNLinSolSetATimes\_SPBCGS}
\item \id{SUNLinSolSetPreconditioner\_SPBCGS}
\item \id{SUNLinSolSetScalingVectors\_SPBCGS}
\item \id{SUNLinSolSetup\_SPBCGS}
\item \id{SUNLinSolSolve\_SPBCGS}
\item \id{SUNLinSolNumIters\_SPBCGS}
\item \id{SUNLinSolResNorm\_SPBCGS}
\item \id{SUNLinSolResid\_SPBCGS}
\item \id{SUNLinSolLastFlag\_SPBCGS}
\item \id{SUNLinSolSpace\_SPBCGS}
\item \id{SUNLinSolFree\_SPBCGS}
\end{itemize}
All of the listed operations are callable via the {\F} 2003 interface module
by prepending an `F' to the function name.

The {\sunlinsolspbcgs} module also defines the following additional
user-callable functions.
%
% --------------------------------------------------------------------
%
\ucfunctiondf{SUNLinSol\_SPBCGSSetPrecType}
{
  retval = SUNLinSol\_SPBCGSSetPrecType(LS, pretype);
}
{
  The function \ID{SUNLinSol\_SPBCGSSetPrecType} updates the type of
  preconditioning to use in the {\sunlinsolspbcgs} object.
}
{
  \begin{args}[pretype]
  \item[LS] (\id{SUNLinearSolver})
    the {\sunlinsolspbcgs} object to update
  \item[pretype] (\id{int})
    flag indicating the desired type of preconditioning, allowed
    values match those discussed in \id{SUNLinSol\_SPBCGS}.
  \end{args}
}
{
  This routine will return with one of the error codes
  \id{SUNLS\_ILL\_INPUT} (illegal \id{pretype}), \id{SUNLS\_MEM\_NULL}
  (\id{S} is \id{NULL}) or \id{SUNLS\_SUCCESS}.
}
{}
{SUNSPBCGSSetPrecType}
%
% --------------------------------------------------------------------
%
\ucfunctiondf{SUNLinSol\_SPBCGSSetMaxl}
{
  retval = SUNLinSol\_SPBCGSSetMaxl(LS, maxl);
}
{
  The function \ID{SUNLinSol\_SPBCGSSetMaxl} updates the number of
  linear solver iterations to allow.
}
{
  \begin{args}[maxl]
  \item[LS] (\id{SUNLinearSolver})
    the {\sunlinsolspbcgs} object to update
  \item[maxl] (\id{int})
    flag indicating the number of iterations to allow. Values $\le0$
    will result in the default value (5).
  \end{args}
}
{
  This routine will return with one of the error codes
  \id{SUNLS\_MEM\_NULL} (\id{S} is \id{NULL}) or \id{SUNLS\_SUCCESS}.
}
{}
{SUNSPBCGSSetMaxl}


% ====================================================================
\subsection{SUNLinearSolver\_SPBCGS Fortran interfaces}
\label{ss:sunlinsol_spbcgs_fortran}
% ====================================================================

The {\sunlinsolspbcgs} module provides a {\F} 2003 module as well as {\F} 77
style interface functions for use from {\F} applications.

\subsubsection*{FORTRAN 2003 interface module}
The \ID{fsunlinsol\_spbcgs\_mod} {\F} module defines interfaces to all
{\sunlinsolspbcgs} {\CC} functions using the intrinsic \id{iso\_c\_binding}
module which provides a standardized mechanism for interoperating with {\CC}. As
noted in the {\CC} function descriptions above, the interface functions are
named after the corresponding {\CC} function, but with a leading `F'. For
example, the function \id{SUNLinSol\_SPBCGS} is interfaced as
\id{FSUNLinSol\_SPBCGS}.

The {\F} 2003 {\sunlinsolspbcgs} interface module can be accessed with the \id{use}
statement, i.e. \id{use fsunlinsol\_spbcgs\_mod}, and linking to the library
\id{libsundials\_fsunlinsolspbcgs\_mod}.{\em lib} in addition to the {\CC} library.
For details on where the library and module file
\id{fsunlinsol\_spbcgs\_mod.mod} are installed see Appendix \ref{c:install}.
We note that the module is accessible from the {\F} 2003 {\sundials} integrators
\textit{without} separately linking to the
\id{libsundials\_fsunlinsolspbcgs\_mod} library.

\subsubsection*{FORTRAN 77 interface functions}
For solvers that include a {\F} 77 interface module, the
{\sunlinsolspbcgs} module also includes a Fortran-callable function
for creating a \id{SUNLinearSolver} object.
%
% --------------------------------------------------------------------
%
\ucfunction{FSUNSPBCGSINIT}
{
  FSUNSPBCGSINIT(code, pretype, maxl, ier)
}
{
  The function \ID{FSUNSPBCGSINIT} can be called for Fortran programs
  to create a {\sunlinsolspbcgs} object.
}
{
  \begin{args}[pretype]
  \item[code] (\id{int*})
    is an integer input specifying the solver id (1 for {\cvode}, 2
    for {\ida}, 3 for {\kinsol}, and 4 for {\arkode}).
  \item[pretype] (\id{int*})
    flag indicating desired preconditioning type
  \item[maxl] (\id{int*})
    flag indicating number of iterations to allow
  \end{args}
}
{
  \id{ier} is a return completion flag equal to \id{0} for a success
  return and \id{-1} otherwise. See printed message for details in case
  of failure.
}
{
  This routine must be called \emph{after} the {\nvector} object has
  been initialized.

  Allowable values for \id{pretype} and \id{maxl} are the same as for
  the {\CC} function \newline \ID{SUNLinSol\_SPBCGS}. 
}
Additionally, when using {\arkode} with a non-identity
mass matrix, the {\sunlinsolspbcgs} module includes a Fortran-callable
function for creating a \id{SUNLinearSolver} mass matrix solver
object.
%
% --------------------------------------------------------------------
%
\ucfunction{FSUNMASSSPBCGSINIT}
{
  FSUNMASSSPBCGSINIT(pretype, maxl, ier)
}
{
  The function \ID{FSUNMASSSPBCGSINIT} can be called for Fortran programs
  to create a {\sunlinsolspbcgs} object for mass matrix linear systems.
}
{
  \begin{args}[pretype]
  \item[pretype] (\id{int*})
    flag indicating desired preconditioning type
  \item[maxl] (\id{int*})
    flag indicating number of iterations to allow
  \end{args}
}
{
  \id{ier} is a \id{int} return completion flag equal to \id{0} for a success
  return and \id{-1} otherwise. See printed message for details in case
  of failure.
}
{
  This routine must be called \emph{after} the {\nvector} object has
  been initialized.

  Allowable values for \id{pretype} and \id{maxl} are the same as for
  the {\CC} function \newline \ID{SUNLinSol\_SPBCGS}. 
}
%
% --------------------------------------------------------------------
%
The \id{SUNLinSol\_SPBCGSSetPrecType} and
\id{SUNLinSol\_SPBCGSSetMaxl} routines also support Fortran interfaces
for the system and mass matrix solvers.
%
% --------------------------------------------------------------------
%
\ucfunction{FSUNSPBCGSSETPRECTYPE}
{
  FSUNSPBCGSSETPRECTYPE(code, pretype, ier)
}
{
  The function \ID{FSUNSPBCGSSETPRECTYPE} can be called for Fortran
  programs to change the type of preconditioning to use.
}
{
  \begin{args}[pretype]
  \item[code] (\id{int*})
    is an integer input specifying the solver id (1 for {\cvode}, 2
    for {\ida}, 3 for {\kinsol}, and 4 for {\arkode}).
  \item[pretype] (\id{int*})
    flag indicating the type of preconditioning to use.
  \end{args}
}
{
  \id{ier} is a \id{int} return completion flag equal to \id{0} for a success
  return and \id{-1} otherwise. See printed message for details in case
  of failure.
}
{
  See \id{SUNLinSol\_SPBCGSSetPrecType} for complete further documentation of
  this routine.
}
%
% --------------------------------------------------------------------
%
\ucfunction{FSUNMASSSPBCGSSETPRECTYPE}
{
  FSUNMASSSPBCGSSETPRECTYPE(pretype, ier)
}
{
  The function \ID{FSUNMASSSPBCGSSETPRECTYPE} can be called for Fortran
  programs to change the type of preconditioning for mass matrix
  linear systems.
}
{
  The arguments are identical to \id{FSUNSPBCGSSETPRECTYPE} above, except that
  \id{code} is not needed since mass matrix linear systems only arise
  in {\arkode}.
}
{
  \id{ier} is a \id{int} return completion flag equal to \id{0} for a success
  return and \id{-1} otherwise. See printed message for details in case
  of failure.
}
{
  See \id{SUNLinSol\_SPBCGSSetPrecType} for complete further documentation of
  this routine.
}
%
% --------------------------------------------------------------------
%
\ucfunction{FSUNSPBCGSSETMAXL}
{
  FSUNSPBCGSSETMAXL(code, maxl, ier)
}
{
  The function \ID{FSUNSPBCGSSETMAXL} can be called for Fortran
  programs to change the maximum number of iterations to allow.
}
{
  \begin{args}[maxl]
  \item[code] (\id{int*})
    is an integer input specifying the solver id (1 for {\cvode}, 2
    for {\ida}, 3 for {\kinsol}, and 4 for {\arkode}).
  \item[maxl] (\id{int*})
    the number of iterations to allow.
  \end{args}
}
{
  \id{ier} is a \id{int} return completion flag equal to \id{0} for a success
  return and \id{-1} otherwise. See printed message for details in case
  of failure.
}
{
  See \id{SUNLinSol\_SPBCGSSetMaxl} for complete further
  documentation of this routine.
}
%
% --------------------------------------------------------------------
%
\ucfunction{FSUNMASSSPBCGSSETMAXL}
{
  FSUNMASSSPBCGSSETMAXL(maxl, ier)
}
{
  The function \ID{FSUNMASSSPBCGSSETMAXL} can be called for Fortran
  programs to change the type of preconditioning for mass matrix
  linear systems.
}
{
  The arguments are identical to \id{FSUNSPBCGSSETMAXL} above, except that
  \id{code} is not needed since mass matrix linear systems only arise
  in {\arkode}.
}
{
  \id{ier} is a \id{int} return completion flag equal to \id{0} for a success
  return and \id{-1} otherwise. See printed message for details in case
  of failure.
}
{
  See \id{SUNLinSol\_SPBCGSSetMaxl} for complete further
  documentation of this routine.
}
%
% --------------------------------------------------------------------
%

% ====================================================================
\subsection{SUNLinearSolver\_SPBCGS content}
\label{ss:sunlinsol_spbcgs_content}
% ====================================================================

The {\sunlinsolspbcgs} module defines the \textit{content} field of a
\id{SUNLinearSolver} as the following structure:
%%
\begin{verbatim}
struct _SUNLinearSolverContent_SPBCGS {
  int maxl;
  int pretype;
  int numiters;
  realtype resnorm;
  long int last_flag;
  ATimesFn ATimes;
  void* ATData;
  PSetupFn Psetup;
  PSolveFn Psolve;
  void* PData;
  N_Vector s1;
  N_Vector s2;
  N_Vector r;
  N_Vector r_star;
  N_Vector p;
  N_Vector q;
  N_Vector u;
  N_Vector Ap;
  N_Vector vtemp;
};
\end{verbatim}
%%
These entries of the \emph{content} field contain the following
information:
\begin{args}[last\_flag]
  \item[maxl] - number of {\spbcgs} iterations to allow (default is 5),
  \item[pretype] - flag for type of preconditioning to employ
    (default is none),
  \item[numiters] - number of iterations from the most-recent solve,
  \item[resnorm] - final linear residual norm from the most-recent solve,
  \item[last\_flag] - last error return flag from an internal function,
  \item[ATimes] - function pointer to perform $Av$ product,
  \item[ATData] - pointer to structure for \id{ATimes},
  \item[Psetup] - function pointer to preconditioner setup routine,
  \item[Psolve] - function pointer to preconditioner solve routine,
  \item[PData] - pointer to structure for \id{Psetup} and \id{Psolve},
  \item[s1, s2] - vector pointers for supplied scaling matrices
    (default is \id{NULL}),
  \item[r] - a {\nvector} which holds the current scaled,
    preconditioned linear system residual,
  \item[r\_star] - a {\nvector} which holds the initial scaled,
    preconditioned linear system residual,
  \item[p, q, u, Ap, vtemp] - {\nvector}s used for workspace by the
    {\spbcgs} algorithm.
\end{args}

