%% This is a shared SUNDIALS TEX file with a description of the
%% spbcgs sunlinsol implementation
%%

The {\spbcg} (Scaled, Preconditioned, Bi-Conjugate Gradient,
Stabilized \cite{Van:92}) implementation of the {\sunlinsol} module 
provided with {\sundials}, {\sunlinsolspbcgs}, is an iterative linear
solver that is designed to be compatible with any {\nvector}
implementation (serial, threaded, parallel, user-supplied) that
supports a minimal subset of operations (\id{N\_VClone}, 
\id{N\_VDotProd}, \id{N\_VScale}, \id{N\_VLinearSum}, \id{N\_VProd},
\id{N\_VDiv} and \id{N\_VDestroy}).  Unlike the {\spgmr} and {\spfgmr}
algorithms, {\spbcg} requires a fixed amount of memory that does not
scale with the number of allowed iterations.

The {\sunlinsolspbcgs} module defines the {\em content} field of a
\id{SUNLinearSolver} to be the following structure:
%%
\begin{verbatim} 
struct _SUNLinearSolverContent_SPBCGS {
  int maxl;
  int pretype;
  int numiters;
  realtype resnorm;
  long int last_flag;
  ATSetupFn ATSetup;
  ATimesFn ATimes;
  void* ATData;
  PSetupFn Psetup;
  PSolveFn Psolve;
  void* PData;
  N_Vector s1;
  N_Vector s2;
  N_Vector r;
  N_Vector r_star;
  N_Vector p;
  N_Vector q;
  N_Vector u;
  N_Vector Ap;
  N_Vector vtemp;
};
\end{verbatim}
%%
These entries of the \emph{content} field contain the following
information:
\begin{description}
  \item[maxl] - number of {\spbcg} iterations to allow (default is 5)
  \item[pretype] - flag for type of preconditioning to employ
    (default is none)
  \item[numiters] - number of iterations from most-recent solve
  \item[resnorm] - final linear residual norm from most-recent solve
  \item[last\_flag] - last error return flag from internal function
  \item[ATSetup] - function pointer to setup routine for \id{ATimes} data
  \item[ATimes] - function pointer to perform $Av$ product
  \item[ATData] - pointer to structure for \id{ATSetup}, \id{ATimes}
  \item[Psetup] - function pointer to preconditioner setup routine
  \item[Psolve] - function pointer to preconditioner solve routine
  \item[PData] - pointer to structure for \id{Psetup}, \id{Psolve}
  \item[s1, s2] - vector pointers for supplied scaling matrices
    (default are \id{NULL})
  \item[r] - a {\nvector} which holds the current scaled,
    preconditioned linear system residual 
  \item[r\_star] - a {\nvector} which holds the initial scaled,
    preconditioned linear system residual 
  \item[p, q, u, Ap, vtemp] - {\nvector}s used for workspace by the
    {\spbcg} algorithm. 
\end{description}

This solver is constructed to perform the following operations:
\begin{itemize}
\item During construction all {\nvector} solver data is allocated,
  with vectors cloned from a template {\nvector} that is input, and
  default solver parameters are set.
\item User-facing ``set'' routines may be called to modify default
  solver parameters.
\item Additional ``set'' routines are called by the {\sundials} solver
  that interfaces with {\sunlinsolspbcgs} to supply the \id{ATSetup},
  \id{ATimes}, \id{PSetup} and \id{Psolve} function pointers and
  \id{s1} and \id{s2} scaling vectors.
\item In the ``initialize'' call, the solver parameters are checked
  for validity.
\item In the ``setup'' call, any non-\id{NULL} \id{ATSetup} and
  \id{PSetup} functions are called.  Typically, these are provided by
  the {\sundials} solvers themselves, that translate between the
  generic \id{ATSetup} and \id{PSetup} functions and the
  solver-specific routines (solver-supplied or user-supplied).
\item In the ``solve'' call the {\spbcg} iteration is performed.  This
  will include scaling and preconditioning if those options have been
  supplied.
\end{itemize}

\noindent The header file to be included when using this module 
is \id{sunlinsol/sunlinsol\_spbcgs.h}. \\
%%
%%----------------------------------------------
%%
The {\sunlinsolspbcgs} module defines implementations of all
``iterative'' linear solver operations listed in Table
\ref{t:sunlinsolops}:
\begin{itemize}
\item \id{SUNLinSolGetType\_SPBCGS}
\item \id{SUNLinSolInitialize\_SPBCGS}
\item \id{SUNLinSolSetATimes\_SPBCGS}
\item \id{SUNLinSolSetPreconditioner\_SPBCGS}
\item \id{SUNLinSolSetScalingVectors\_SPBCGS}
\item \id{SUNLinSolSetup\_SPBCGS}
\item \id{SUNLinSolSolve\_SPBCGS}
\item \id{SUNLinSolNumIters\_SPBCGS}
\item \id{SUNLinSolResNorm\_SPBCGS}
\item \id{SUNLinSolResid\_SPBCGS}
\item \id{SUNLinSolLastFlag\_SPBCGS}
\item \id{SUNLinSolSpace\_SPBCGS}
\item \id{SUNLinSolFree\_SPBCGS}
\end{itemize}
The module {\sunlinsolspbcgs} provides the following additional
user-callable routines: 
%%
\begin{itemize}

%%--------------------------------------

\item \ID{SUNSPBCGS}

  This function creates and allocates memory for a {\spbcg}
  \id{SUNLinearSolver}.  Its arguments are an {\nvector}, the desired
  type of preconditioning, and the number of linear iterations to allow.

  This routine will perform consistency checks to ensure that it is
  called with a consistent {\nvector} implementation (i.e.~that it
  supplies the requisite vector operations).  If \id{y} is
  incompatible then this routine will return \id{NULL}.

  A \id{maxl} argument that is $\le0$ will result in the default
  value (5).

  Allowable inputs for \id{pretype} are \id{PREC\_NONE} (0),
  \id{PREC\_LEFT} (1), \id{PREC\_RIGHT} (2) and \id{PREC\_BOTH} (3);
  any other integer input will result in the default (no
  preconditioning).  We note that some {\sundials} solvers are
  designed to only work with right preconditioning ({\kinsol}, {\ida},
  {\idas}).  While it is possible to configure a {\sunlinsolspbcgs}
  object to use \id{PREC\_LEFT} or \id{PREC\_BOTH} with these solvers,
  this use mode is not supported and may result in inferior
  performance.

  \verb|SUNLinearSolver SUNSPBCGS(N_Vector y, int pretype, int maxl);|

%%--------------------------------------

\item \ID{SUNSPBCGSSetPrecType}

  This function updates the type of preconditioning to use.  Supported
  values are \id{PREC\_NONE} (0), \id{PREC\_LEFT} (1),
  \id{PREC\_RIGHT} (2) and \id{PREC\_BOTH} (3).  

  This routine will return with one of the error codes
  \id{SUNLS\_ILL\_INPUT} (illegal \id{pretype}), \id{SUNLS\_MEM\_NULL}
  (\id{S} is \id{NULL}) or \id{SUNLS\_SUCCESS}.
  
  \verb|int SUNSPBCGSSetPrecType(SUNLinearSolver S, int pretype);|

%%--------------------------------------

\item \ID{SUNSPBCGSSetMaxl}

  This function updates the number of linear solver iterations to allow.  

  A \id{maxl} argument that is $\le0$ will result in the default
  value (5).

  This routine will return with one of the error codes
  \id{SUNLS\_MEM\_NULL} (\id{S} is \id{NULL}) or \id{SUNLS\_SUCCESS}.
  
  \verb|int SUNSPBCGSSetMaxl(SUNLinearSolver S, int maxl);|

\end{itemize}
%%
%%------------------------------------
%%
For solvers that include a Fortran interface module, the
{\sunlinsolspbcgs} module also includes the Fortran-callable
function \id{FSUNSPBCGSInit(code, pretype, maxl, ier)} to initialize
this {\sunlinsolspbcgs} module for a given {\sundials} solver.
Here \id{code} is an input solver id (1 for {\cvode}, 2 for {\ida}, 3
for {\kinsol}, 4 for {\arkode}); \id{pretype} and \id{maxl} are the
same as for the C function \ID{SUNSPBCGS}; \id{ier} is an error return
flag equal 0 for success and -1 for failure.  All of these input
arguments should be declared so as to match C type \id{int}).  This
routine must be called \emph{after} the {\nvector} object has been
initialized.  Additionally, when using {\arkode} with non-identity
mass matrix, the Fortran-callable function 
\id{FSUNMassSPBCGSInit(pretype, maxl, ier)} initializes this
{\sunlinsolspbcgs} module for solving mass matrix linear systems.

The \id{SUNSPBCGSSetPrecType} and \id{SUNSPBCGSSetMaxl} routines also
support Fortran interfaces for the system and mass matrix solvers:
\begin{itemize}
\item \id{FSUNSPBCGSSetPrecType(code, pretype, ier)} -- all arguments
  should be commensurate with a C \id{int}
\item \id{FSUNMassSPBCGSSetPrecType(pretype, ier)}
\item \id{FSUNSPBCGSSetMaxl(code, maxl, ier)} -- all arguments
  should be commensurate with a C \id{int}
\item \id{FSUNMassSPBCGSSetMaxl(maxl, ier)}
\end{itemize}
