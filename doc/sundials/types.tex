% This is a shared SUNDIALS TEX file with description of
% types used in llntyps.h
%
\index{portability}
The \ID{sundials\_types.h} file contains the definition of the type \ID{realtype},
which is used by the {\sundials} solvers for all floating-point data, the definition 
of the integer type \ID{sunindextype}, which is used for vector and matrix indices,
and \ID{booleantype}, which is used for certain logic operations within {\sundials}.


\subsection{Floating point types}

The type \id{realtype} can be \id{float}, \id{double}, or \id{long double}, with
the default being \id{double}.
The user can change the precision of the {\sundials} solvers arithmetic at the
configuration stage (see \S\ref{ss:configuration_options_nix}).

Additionally, based on the current precision, \id{sundials\_types.h} defines 
\Id{BIG\_REAL} to be the largest value representable as a \id{realtype},
\Id{SMALL\_REAL} to be the smallest value representable as a \id{realtype}, and
\Id{UNIT\_ROUNDOFF} to be the difference between $1.0$ and the minimum \id{realtype}
greater than $1.0$.

Within {\sundials}, real constants are set by way of a macro called
\Id{RCONST}.  It is this macro that needs the ability to branch on the
definition \id{realtype}.  In ANSI {\CC}, a floating-point constant with no
suffix is stored as a \id{double}.  Placing the suffix ``F'' at the
end of a floating point constant makes it a \id{float}, whereas using the suffix
``L'' makes it a \id{long double}.  For example,
\begin{verbatim}
#define A 1.0
#define B 1.0F
#define C 1.0L
\end{verbatim}
defines \id{A} to be a \id{double} constant equal to $1.0$, \id{B} to be a
\id{float} constant equal to $1.0$, and \id{C} to be a \id{long double} constant
equal to $1.0$.  The macro call \id{RCONST(1.0)} automatically expands to \id{1.0}
if \id{realtype} is \id{double}, to \id{1.0F} if \id{realtype} is \id{float},
or to \id{1.0L} if \id{realtype} is \id{long double}.  {\sundials} uses the
\id{RCONST} macro internally to declare all of its floating-point constants. 

A user program which uses the type \id{realtype} and the \id{RCONST} macro
to handle floating-point constants is precision-independent except for
any calls to precision-specific standard math library
functions.  (Our example programs use both \id{realtype} and
\id{RCONST}.)  Users can, however, use the type \id{double}, \id{float}, or
\id{long double} in their code (assuming that this usage is consistent
with the typedef for \id{realtype}).  Thus, a previously existing
piece of ANSI {\CC} code can use {\sundials} without modifying the code
to use \id{realtype}, so long as the {\sundials} libraries use the
correct precision (for details see \S\ref{ss:configuration_options_nix}).


\subsection{Integer types used for vector and matrix indices}

The type \id{sunindextype} can be either a 32- or 64-bit \emph{signed} integer.
The default is the portable \id{int64\_t} type, and the user can change it
to \id{int32\_t} at the configuration stage. The configuration system
will detect if the compiler does not support portable types, and will
replace \id{int32\_t} and \id{int64\_t} with \id{int} and \id{long int},
respectively, to ensure use of the desired sizes on Linux, Mac OS X, and Windows
platforms. {\sundials} currently does not support \emph{unsigned} integer types 
for vector and matrix indices, although these could be added in the future if there 
is sufficient demand.

A user program which uses \id{sunindextype} to handle vector and matrix indices
will work with both index storage types except for any calls to index storage-specific
external libraries. (Our \id{C} and \id{C++} example programs use \id{sunindextype}.)
Users can, however, use any one of \id{int}, \id{long int}, \id{int32\_t}, \id{int64\_t} or
\id{long long int} in their code, assuming that this usage is consistent with the typedef
for \id{sunindextype} on their architecture). Thus, a previously existing piece of ANSI
{\CC} code can use {\sundials} without modifying the code to use \id{sunindextype},
so long as the {\sundials} libraries use the appropriate index storage type (for details
see \S\ref{ss:configuration_options_nix}).
