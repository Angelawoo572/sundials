%% This is a shared SUNDIALS TEX file with a description of the
%% Pthreads nvector implementation
%%

The Pthreads implementation of the {\nvector} module provided with {\sundials},
{\nvecpthreads}, defines the {\em content} field of \id{N\_Vector} to be a structure 
containing the length of the vector, a pointer to the beginning of a contiguous 
data array, and a boolean flag {\em own\_data} which specifies the ownership 
of {\em data}.  Operations on the vector are threaded using POSIX threads 
(Pthreads), the number of threads used is based on the supplied argument in 
the vector constructor.
%%
\begin{verbatim} 
struct _N_VectorContent_Pthreads {
  long int length;
  booleantype own_data;
  realtype *data;
  int num_threads;
};
\end{verbatim}
%%
%%--------------------------------------------
%%
The following five macros are provided to access the content of an {\nvecpthreads}
vector. The suffix \id{\_PT} in the names denotes Pthreads version.
%%
\begin{itemize}

\item \ID{NV\_CONTENT\_PT}                             
    
  This routine gives access to the contents of the Pthreads
  vector \id{N\_Vector}.
  
  The assignment \id{v\_cont} $=$ \id{NV\_CONTENT\_PT(v)} sets           
  \id{v\_cont} to be a pointer to the Pthreads \id{N\_Vector} content  
  structure.                                             
                                                            
  Implementation: 
  
  \verb|#define NV_CONTENT_PT(v) ( (N_VectorContent_Pthreads)(v->content) )|
  
\item \ID{NV\_OWN\_DATA\_PT}, \ID{NV\_DATA\_PT}, \ID{NV\_LENGTH\_PT}


  These macros give individual access to the parts of    
  the content of a Pthreads \id{N\_Vector}.                        
                                                               
  The assignment \id{v\_data = NV\_DATA\_PT(v)} sets \id{v\_data} to be     
  a pointer to the first component of the data for the \id{N\_Vector} \id{v}. 
  The assignment \id{NV\_DATA\_PT(v) = v\_data} sets the component array of \id{v} to     
  be \id{v\_data} by storing the pointer \id{v\_data}.                   
  
  The assignment \id{v\_len = NV\_LENGTH\_PT(v)} sets \id{v\_len} to be     
  the length of \id{v}. On the other hand, the call \id{NV\_LENGTH\_PT(v) = len\_v} 
  sets the length of \id{v} to be \id{len\_v}.
                                                            
  Implementation: 
  
  \verb|#define NV_OWN_DATA_PT(v) ( NV_CONTENT_PT(v)->own_data )|

  \verb|#define NV_DATA_PT(v) ( NV_CONTENT_PT(v)->data )|
  
  \verb|#define NV_LENGTH_PT(v) ( NV_CONTENT_PT(v)->length )|

  \verb|#define NV_NUM_THREADS_PT(v) ( NV_CONTENT_PT(v)->num_threads )|

\item \ID{NV\_Ith\_PT}                                               
                                                            
  This macro gives access to the individual components of the data
  array of an \id{N\_Vector}.

  The assignment \id{r = NV\_Ith\_PT(v,i)} sets \id{r} to be the value of 
  the \id{i}-th component of \id{v}. The assignment \id{NV\_Ith\_PT(v,i) = r}   
  sets the value of the \id{i}-th component of \id{v} to be \id{r}.        
  
  Here $i$ ranges from $0$ to $n-1$ for a vector of length $n$.

  Implementation:

  \verb|#define NV_Ith_PT(v,i) ( NV_DATA_PT(v)[i] )|

\end{itemize}
%%
%%----------------------------------------------
%%
The {\nvecpthreads} module defines Pthreads implementations of all vector operations listed 
in Table \ref{t:nvecops}. Their names are obtained from those in Table \ref{t:nvecops} by
appending the suffix \id{\_Pthreads}. The module {\nvecpthreads} provides the following additional
user-callable routines:
%%
\begin{itemize}

%%--------------------------------------

\item \ID{N\_VNew\_Pthreads}

  This function creates and allocates memory for a Pthreads \id{N\_Vector}.
  Arguments are the vector length and number of threads.

  

  \verb|N_Vector N_VNew_Pthreads(long int vec_length, int num_threads);|

%%--------------------------------------

\item \ID{N\_VNewEmpty\_Pthreads}

  This function creates a new Pthreads \id{N\_Vector} with an empty (\id{NULL}) data array.

  

  \verb|N_Vector N_VNewEmpty_Pthreads(long int vec_length, int num_threads);|

%%--------------------------------------

\item \ID{N\_VMake\_Pthreads}

 This function creates and allocates memory for a Pthreads vector
 with user-provided data array.

 

 \verb|N_Vector N_VMake_Pthreads(long int vec_length, realtype *v_data, int num_threads);|

%%--------------------------------------

\item \ID{N\_VCloneVectorArray\_Pthreads}

 This function creates (by cloning) an array of \id{count} Pthreads vectors.

 

 \verb|N_Vector *N_VCloneVectorArray_Pthreads(int count, N_Vector w);|

%%--------------------------------------

\item \ID{N\_VCloneEmptyVectorArray\_Pthreads}

 This function creates (by cloning) an array of \id{count} Pthreads vectors, each with an
 empty (\id{NULL}) data array.

 

 \verb|N_Vector *N_VCloneEmptyVectorArray_Pthreads(int count, N_Vector w);|

%%--------------------------------------

\item \ID{N\_VDestroyVectorArray\_Pthreads}

 This function frees memory allocated for the array of \id{count} variables of type
 \id{N\_Vector} created with \id{N\_VCloneVectorArray\_Pthreads} or with
 \id{N\_VCloneEmptyVectorArray\_Pthreads}.

 

 \verb|void N_VDestroyVectorArray_Pthreads(N_Vector *vs, int count);|

%%--------------------------------------

\item \ID{N\_VPrint\_Pthreads}

 This function prints the content of a Pthreads vector to \id{stdout}.

 
 
 \verb|void N_VPrint_Pthreads(N_Vector v);|

\end{itemize}
%%
%%------------------------------------
%%
\paragraph{\bf Notes}                                                      
           
\begin{itemize}
                                        
\item
  When looping over the components of an \id{N\_Vector} \id{v}, it is     
  more efficient to first obtain the component array via       
  \id{v\_data = NV\_DATA\_PT(v)} and then access \id{v\_data[i]} within the     
  loop than it is to use \id{NV\_Ith\_PT(v,i)} within the loop.        

\item
  {\warn}\id{N\_VNewEmpty\_Pthreads}, \id{N\_VMake\_Pthreads}, 
  and \id{N\_VCloneEmptyVectorArray\_Pthreads} set the field 
  {\em own\_data} $=$ \id{FALSE}. 
  \id{N\_VDestroy\_Pthreads} and \id{N\_VDestroyVectorArray\_Pthreads}
  will not attempt to free the pointer {\em data} for any \id{N\_Vector} with
  {\em own\_data} set to \id{FALSE}. In such a case, it is the user's responsibility to
  deallocate the {\em data} pointer.
                                     
\item
  {\warn}To maximize efficiency, vector operations in the {\nvecpthreads} implementation
  that have more than one \id{N\_Vector} argument do not check for
  consistent internal representation of these vectors. It is the user's 
  responsibility to ensure that such routines are called with \id{N\_Vector}
  arguments that were all created with the same internal representations.

\end{itemize}
