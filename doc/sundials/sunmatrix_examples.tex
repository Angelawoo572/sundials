\section{SUNMatrix Examples}\label{ss:sunmat_examples}

There are \id{SUNMatrix} examples that may be installed for each
implementation: dense, banded, and sparse.  Each implementation
makes use of the functions in \id{test\_sunmatrix.c}. 
These example functions show simple usage of the \id{SUNMatrix} family
of functions.  The inputs to the examples depend on the matrix type,
and are output to \texttt{stdout} if the example is run without the
appropriate number of command-line arguments. 

\noindent The following is a list of the example functions in \id{test\_sunmatrix.c}:
\begin{itemize}
\item \id{Test\_SUNMatGetID}: Verifies the returned matrix ID against
      the value that should be returned.
\item \id{Test\_SUNMatClone}: Creates clone of an existing matrix,
      copies the data, and checks that their values match.  
\item \id{Test\_SUNMatZero}: Zeros out an existing matrix and checks
      that each entry equals 0.0.
\item \id{Test\_SUNMatCopy}: Clones an input matrix, copies its data
      to a clone, and verifies that all values match.
\item \id{Test\_SUNMatScaleAdd}: Given an input matrix $A$ and an
      input identity matrix $I$, this test clones and copies $A$ to a new
      matrix $B$, computes $B = -B+B$, and verifies that the resulting
      matrix entries equal 0.0.  Additionally, if the matrix is
      square, this test clones and copies $A$ to a new matrix $D$, clones and
      copies $I$ to a new matrix $C$, computes $D = D+I$ and
      $C = C+A$ using \id{SUNMatScaleAdd}, and then verifies that
      $C==D$.
\item \id{Test\_SUNMatScaleAddI}: Given an input matrix $A$ and an
      input identity matrix $I$, this clones and copies $I$ to a new
      matrix $B$, computes $B = -B+I$ using \id{SUNMatScaleAddI}, and
      verifies that the resulting matrix entries equal 0.0.
\item \id{Test\_SUNMatMatvec} Given an input matrix $A$ and input
      vectors $x$ and $y$ such that $y=Ax$, this test has different
      behavior depending on whether $A$ is square.  If it is square,
      it clones and copies $A$ to a new matrix $B$, computes
      $B = 3B+I$ using \id{SUNMatScaleAddI}, clones $y$ to new vectors
      $w$ and $z$, computes $z = Bx$ using \id{SUNMatMatvec}, computes
      $w = 3y+x$ using \id{N\_VLinearSum}, and verifies that $w==z$.
      If $A$ is not square, it just clones $y$ to a new vector $z$,
      computes $z=Ax$ using \id{SUNMatMatvec}, and verifies that
      $y==z$.
\item \id{Test\_SUNMatSpace} verifies that \id{SUNMatSpace} can be
      called, and outputs the results to \texttt{stdout}.
\end{itemize}
