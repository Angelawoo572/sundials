% ====================================================================
\section{The SUNNonlinearSolver\_PetscSNES implementation}
\label{s:sunnonlinsolpetsc}
% ====================================================================
This section describes the {\sunnonlinsol} interface to the PETSc SNES nonlinear
solver(s). To enable the {\sunnonlinsolpetsc} module, SUNDIALS must be
configured to use PETSc. Instructions on how to do this are given in
Chapter~\ref{ss:building_with_petsc}. To access the module, users must include
the header file \id{sunnonlinsol/sunnonlinsol\_petscsnes.h}. The library to link
to is \id{libsundials\_sunnonlinsolpetsc}{\em .lib} where {\em .lib} is .so for
shared libaries and .a for static libraries. Users of the {\sunnonlinsolpetsc}
should also see the section {\nvecpetsc} \ref{ss:nvec_petsc} which discusses the
{\nvector} interface to the PETSc Vec API.

% ====================================================================
\subsection{SUNNonlinearSolver\_PetscSNES description}
\label{ss:sunnonlinsolpetsc_description}
% ====================================================================

The {\sunnonlinsolpetsc} implementation allows users to utilize a PETSc SNES
nonlinear solver to solve the nonlinear systems that arise in the {\sundials}
integrators. Since SNES uses the KSP linear solver interface underneath it, the
{\sunnonlinsolpetsc} implementation does not interface with {\sundials} linear
solvers. Instead, users should set nonlinear solver options, linear solver
options, and preconditioner options through the PETSc SNES, KSP, and PC APIs
\cite{petsc-user-ref}.

{\warn}

Important usage notes for the {\sunnonlinsolpetsc} implementation are provided
below:

\begin{itemize}
\item The {\sunnonlinsolpetsc} implementation handles calling \id{SNESSetFunction}
at construction. The actual residual function $F(y)$ is set by the {\sundials}
integrator when the {\sunnonlinsolpetsc} object is attached to it. Therefore, a
user should not call \id{SNESSetFunction} on a \id{SNES} object that is being
used with {\sunnonlinsolpetsc}. For these reasons, it is recommended, although
not always necessary, that the user calls \id{SUNNonlinSol\_PetscSNES} with the
new \id{SNES} object immediately after calling

\item The number of nonlinear iterations is tracked by {\sundials} separately
from the count kept by SNES. As such, the function \id{SUNNonlinSolGetNumIters}
reports the cumulative number of iterations across the lifetime of the
{\sunnonlinsol} object.

\item Some ``converged'' and ``diverged'' convergence reasons returned by SNES
are treated as recoverable convergence failures by {\sundials}. Therefore, the
count of convergence failures returned by \id{SUNNonlinSolGetNumConvFails} will
reflect the number of recoverable convergence failures as determined by
{\sundials}, and may differ from the count returned by
\id{SNESGetNonlinearStepFailures}.

\item The {\sunnonlinsolpetsc} module is not currently compatible with the
{\cvodes} or {\idas} staggered or simultaneous sensitivity strategies.
\end{itemize}

% ====================================================================
\subsection{SUNNonlinearSolver\_PetscSNES functions}
\label{ss:sunnonlinsolpetsc_functions}
% ====================================================================

The {\sunnonlinsolpetsc} module provides the following constructor
for creating a \id{SUNNonlinearSolver} object.
% --------------------------------------------------------------------
\ucfunction{SUNNonlinSol\_PetscSNES}
{
  NLS = SUNNonlinSol\_PetscSNES(y, snes);
}
{
  The function \ID{SUNNonlinSol\_PetscSNES} creates a \id{SUNNonlinearSolver}
  object that wraps a PETSc \id{SNES} object for use with {\sundials}.
  This will call \id{SNESSetFunction} on the provided \id{SNES} object.
}
{
  \begin{args}[snes]
  \item[snes] (\id{SNES})
    a PETSc \id{SNES} object
  \item[y] (\id{N\_Vector})
    a \id{N\_Vector} object of type {\nvecpetsc} that used as a template
    for the residual vector
  \end{args}
}
{
  A {\sunnonlinsol} object if the constructor exits successfully,
  otherwise \id{NLS} will be \id{NULL}.
}
{
  {\warn} This function calls \id{SNESSetFunction} and will overwrite
  whatever function was previously set. Users should not call
  \id{SNESSetFunction} on the \id{SNES} object provided to the constructor.
}

% --------------------------------------------------------------------
The {\sunnonlinsolpetsc} module implements all of the functions
defined in sections \ref{ss:sunnonlinsol_corefn} --
\ref{ss:sunnonlinsol_getfn} except for \id{SUNNonlinSolSetup},
\id{SUNNonlinSolSetLSetupFn},\newline \id{SUNNonlinSolSetLSolveFn},
\id{SUNNonlinSolSetConvTestFn}, and \id{SUNNonlinSolSetMaxIters}.

The {\sunnonlinsolpetsc} functions have the same names as those defined
by the generic {\sunnonlinsol} API with \id{\_PetscSNES} appended to the
function name. Unless using the {\sunnonlinsolpetsc} module as a
standalone nonlinear solver the generic functions defined in sections
\ref{ss:sunnonlinsol_corefn} -- \ref{ss:sunnonlinsol_getfn} should be
called in favor of the {\sunnonlinsolpetsc}-specific implementations.

The {\sunnonlinsolpetsc} module also defines the following additional
user-callable functions.
% --------------------------------------------------------------------
\ucfunction{SUNNonlinSolGetSNES\_PetscSNES}
{
  retval = SUNNonlinSolGetSNES\_PetscSNES(NLS, SNES* snes);
}
{
  The function \ID{SUNNonlinSolGetSNES\_PetscSNES} gets the
  SNES context that was wrapped.
}
{
  \begin{args}[SysFn]
  \item[NLS] (\id{SUNNonlinearSolver})
    a {\sunnonlinsol} object
  \item[snes] (\id{SNES*})
    a pointer to a PETSc \id{SNES} object that will be set upon return
  \end{args}
}
{
  The return value \id{retval} (of type \id{int}) should be zero for a
  successful call, and a negative value for a failure.
}
{}
% --------------------------------------------------------------------
\ucfunction{SUNNonlinSolGetPetscError\_PetscSNES}
{
  retval = SUNNonlinSolGetPetscError\_PetscSNES(NLS, PestcErrorCode* error);
}
{
  The function \ID{SUNNonlinSolGetPetscError\_PetscSNES} gets the last error
  code returned by the last internal call to a PETSc API function.
}
{
  \begin{args}[SysFn]
  \item[NLS] (\id{SUNNonlinearSolver})
    a {\sunnonlinsol} object
  \item[error] (\id{PestcErrorCode*})
    a pointer to a PETSc error integer that will be set upon return
  \end{args}
}
{
  The return value \id{retval} (of type \id{int}) should be zero for a
  successful call, and a negative value for a failure.
}
{}
% --------------------------------------------------------------------
\ucfunction{SUNNonlinSolGetSysFn\_PetscSNES}
{
  retval = SUNNonlinSolGetSysFn\_PetscSNES(NLS, SysFn);
}
{
  The function \ID{SUNNonlinSolGetSysFn\_PetscSNES} returns the residual
  function that defines the nonlinear system.
}
{
  \begin{args}[SysFn]
  \item[NLS] (\id{SUNNonlinearSolver})
    a {\sunnonlinsol} object
  \item[SysFn] (\id{SUNNonlinSolSysFn*})
    the function defining the nonlinear system
  \end{args}
}
{
  The return value \id{retval} (of type \id{int}) should be zero for a
  successful call, and a negative value for a failure.
}
{}

% ====================================================================
\subsection{SUNNonlinearSolver\_PetscSNES content}
\label{ss:sunnonlinsolpetsc_content}
% ====================================================================

The {\sunnonlinsolpetsc} module defines the {\textit{content} field of a
\id{SUNNonlinearSolver} as the following structure:
%%
%%
\begin{verbatim}
struct _SUNNonlinearSolverContent_PetscSNES {
  int sysfn_last_err;
  PetscErrorCode petsc_last_err;
  long int nconvfails;
  long int nni;
  void *imem;
  SNES snes;
  Vec r;
  N_Vector y, f;
  SUNNonlinSolSysFn Sys;
};
\end{verbatim}
%%
%%
These entries of the \emph{content} field contain the following
information:
\begin{args}[Sys]
  \item[sysfn\_last\_err]  - last error returned by the system defining function,
  \item[petsc\_last\_err]  - last error returned by PETSc
  \item[nconvfails]        - number of nonlinear converge failures (recoverable or not),
  \item[nni]               - number of nonlinear iterations,
  \item[imem]              - {\sundials} integrator memory,
  \item[snes]              - PETSc SNES context,
  \item[r]                 - the nonlinear residual,
  \item[y]                 - wrapper for PETSc vectors used in the system function,
  \item[f]                 - wrapper for PETSc vectors used in the system function,
  \item[Sys]               - nonlinear system definining function.
\end{args}
