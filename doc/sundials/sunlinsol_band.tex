%% This is a shared SUNDIALS TEX file with a description of the
%% band sunlinsol implementation
%%
\section{The SUNLinearSolver\_Band implementation}\label{ss:sunlinsol_band}

The band implementation of the {\sunlinsol} module provided with
{\sundials}, {\sunlinsolband}, is designed to be used with the
corresponding {\sunmatband} matrix type, and one of the serial or
shared-memory {\nvector} implementations ({\nvecs}, {\nvecopenmp} or
{\nvecpthreads}).


%---------------------------------------------------------------------------
\subsection{{\sunlinsolband} usage}\label{ss:sunlinsol_band_usage}

The header file to include when using this module is
\id{sunlinsol/sunlinsol\_band.h}. The {\sunlinsolband} module 
is accessible from all {\sundials} solvers \textit{without}
linking to the \\ \noindent
\id{libsundials\_sunlinsolband} module library.

The module {\sunlinsolband} provides the following user-callable constructor routine: 
%%
% --------------------------------------------------------------------
\ucfunction{SUNLinSol\_Band}
{
  LS = SUNLinSol\_Band(y, A);
}
{
  The function \ID{SUNLinSol\_Band} creates and allocates memory for
  a band \id{SUNLinearSolver} object.
}
{
  \begin{args}[y]
  \item[y] (\id{N\_Vector})
    a template for cloning vectors needed within the solver
  \item[A] (\id{SUNMatrix})
    a {\sunmatband} matrix template for cloning matrices needed
    within the solver 
  \end{args}
}
{
  This returns a \id{SUNLinearSolver} object.  If either \id{A} or
  \id{y} are incompatible then this routine will return \id{NULL}.
}
{
  This routine will perform consistency checks to ensure that it is
  called with consistent {\nvector} and {\sunmatrix} implementations.
  These are currently limited to the {\sunmatdense} matrix type and
  the {\nvecs}, {\nvecopenmp}, and {\nvecpthreads} vector types.  As
  additional compatible matrix and vector implementations are added to
  {\sundials}, these will be included within this compatibility check.

  Additionally, this routine will verify that the input matrix \id{A}
  is allocated with appropriate upper bandwidth storage for the $LU$
  factorization.
}
% --------------------------------------------------------------------
%%
For backwards compatibility, we also provide the wrapper functions:
\begin{itemize}

\item \ID{SUNBandLinearSolver}

  Wrapper function for \ID{SUNLinSol\_Band}, with identical input and
  output arguments.

\end{itemize}
%%
%%------------------------------------
%%
For solvers that include a Fortran interface module, the {\sunlinsolband}
module also includes a Fortran-callable function for creating a
\id{SUNLinearSolver} object.
\ucfunction{FSUNBANDLINSOLINIT}
{
  FSUNBANDLINSOLINIT(code, ier)
}
{
  The function \ID{FSUNBANDLINSOLINIT} can be called for Fortran programs
  to create a band \id{SUNLinearSolver} object.
}
{
  \begin{args}[code]
  \item[code] (\id{int*})
    is an integer input specifying the solver id (1 for {\cvode}, 2
    for {\ida}, 3 for {\kinsol}, and 4 for {\arkode}).
  \end{args}
}
{
  \id{ier} is a return completion flag equal to \id{0} for a success
  return and \id{-1} otherwise. See printed message for details in case
  of failure.
}
{
  This routine must be
  called \emph{after} both the {\nvector} and {\sunmatrix} objects have
  been initialized.
}
Additionally, when using {\arkode} with a non-identity
mass matrix, the {\sunlinsolband} module includes a Fortran-callable
function for creating a \id{SUNLinearSolver} mass matrix solver
object.
\ucfunction{FSUNMASSBANDLINSOLINIT}
{
  FSUNMASSBANDLINSOLINIT(ier)
}
{
  The function \ID{FSUNMASSBANDLINSOLINIT} can be called for Fortran programs
  to create a band \id{SUNLinearSolver} object for mass matrix linear
  systems.
}
{
}
{
  \id{ier} is a \id{int} return completion flag equal to \id{0} for a success
  return and \id{-1} otherwise. See printed message for details in case
  of failure.
}
{
  This routine must be
  called \emph{after} both the {\nvector} and {\sunmatrix} mass-matrix
  objects have been initialized.
}

%---------------------------------------------------------------------------
\subsection{{\sunlinsolband} description}\label{ss:sunlinsol_band_description}



The {\sunlinsolband} module defines the {\em
content} field of a \id{SUNLinearSolver} to be the following structure:
%%
\begin{verbatim} 
struct _SUNLinearSolverContent_Band {
  sunindextype N;
  sunindextype *pivots;
  long int last_flag;
};
\end{verbatim}
%%
These entries of the \emph{content} field contain the following
information:
\begin{description}
  \item[N] - size of the linear system,
  \item[pivots] - index array for partial pivoting in LU factorization,
  \item[last\_flag] - last error return flag from internal function evaluations.
\end{description}

This solver is constructed to perform the following operations:
\begin{itemize}
\item The ``setup'' call performs a $LU$ factorization with
  partial (row) pivoting, $PA=LU$, where $P$ is a permutation matrix,
  $L$ is a lower triangular matrix with 1's on the diagonal, and $U$
  is an upper triangular matrix.  This factorization is stored
  in-place on the input {\sunmatband} object $A$, with pivoting
  information encoding $P$ stored in the \id{pivots} array.
\item The ``solve'' call performs pivoting and forward and
  backward substitution using the stored \id{pivots} array and the
  $LU$ factors held in the {\sunmatband} object.
\item
  {\warn} $A$ must be allocated to accommodate the increase in upper
  bandwidth that occurs during factorization.  More precisely, if $A$
  is a band matrix with upper bandwidth \id{mu} and lower bandwidth
  \id{ml}, then the upper triangular factor $U$ can have upper
  bandwidth as big as \id{smu = MIN(N-1,mu+ml)}. The lower triangular
  factor $L$ has lower bandwidth \id{ml}.
\end{itemize}


%%
%%----------------------------------------------
%%

\noindent The {\sunlinsolband} module defines band implementations of all
``direct'' linear solver operations listed in Sections
\ref{ss:sunlinsol_CoreFn}-\ref{ss:sunlinsol_GetFn}:
\begin{itemize}
\item \id{SUNLinSolGetType\_Band}
\item \id{SUNLinSolInitialize\_Band} -- this does nothing, since all
  consistency checks are performed at solver creation.
\item \id{SUNLinSolSetup\_Band} -- this performs the $LU$ factorization.
\item \id{SUNLinSolSolve\_Band} -- this uses the $LU$ factors
  and \id{pivots} array to perform the solve.
\item \id{SUNLinSolLastFlag\_Band}
\item \id{SUNLinSolSpace\_Band} -- this only returns information for
  the storage \emph{within} the solver object, i.e.~storage
  for \id{N}, \id{last\_flag}, and \id{pivots}.
\item \id{SUNLinSolFree\_Band}
\end{itemize}
