%% This is a shared SUNDIALS TEX file with a description of the
%% OpenMP nvector implementation
%%
\section{The NVECTOR\_OPENMP implementation}\label{ss:nvec_openmp}

In situations where a user has a multi-core processing unit capable of
running multiple parallel threads with shared memory, {\sundials} provides
an implementation of {\nvector} using OpenMP, called {\nvecopenmp}, and
an implementation using Pthreads, called {\nvecpthreads}.
Testing has shown that vectors should be of length at least $100,000$
before the overhead associated with creating and using the threads is
made up by the parallelism in the vector calculations.

The OpenMP {\nvector} implementation provided with {\sundials},
{\nvecopenmp}, defines the {\em content} field of \id{N\_Vector} to be a structure
containing the length of the vector, a pointer to the beginning of a contiguous
data array, a boolean flag {\em own\_data} which specifies the ownership
of {\em data}, and the number of threads.
Operations on the vector are threaded using OpenMP.
%%
\begin{verbatim}
struct _N_VectorContent_OpenMP {
  sunindextype length;
  booleantype own_data;
  realtype *data;
  int num_threads;
};
\end{verbatim}
%%
%%--------------------------------------------
%%

The header file to include when using this module is \id{nvector\_openmp.h}.
The installed module library to link to is
\id{libsundials\_nvecopenmp.\textit{lib}}
where \id{\em.lib} is typically \id{.so} for shared libraries and \id{.a}
for static libraries.
The {\F} module file to use when using the {\F} 2003 interface to
this module is \id{fnvector\_openmp\_mod.mod}.


% ====================================================================
\subsection{NVECTOR\_OPENMP accessor macros}
\label{ss:nvec_openmp_macros}
% ====================================================================

The following macros are provided to access the content of an {\nvecopenmp}
vector. The suffix \id{\_OMP} in the names denotes the OpenMP version.
%%
\begin{itemize}

\item \ID{NV\_CONTENT\_OMP}

  This routine gives access to the contents of the OpenMP
  vector \id{N\_Vector}.

  The assignment \id{v\_cont} $=$ \id{NV\_CONTENT\_OMP(v)} sets
  \id{v\_cont} to be a pointer to the OpenMP \id{N\_Vector} content
  structure.

  Implementation:

  \verb|#define NV_CONTENT_OMP(v) ( (N_VectorContent_OpenMP)(v->content) )|

\item \ID{NV\_OWN\_DATA\_OMP}, \ID{NV\_DATA\_OMP}, \ID{NV\_LENGTH\_OMP}, \ID{NV\_NUM\_THREADS\_OMP}


  These macros give individual access to the parts of
  the content of a OpenMP \id{N\_Vector}.

  The assignment \id{v\_data = NV\_DATA\_OMP(v)} sets \id{v\_data} to be
  a pointer to the first component of the data for the \id{N\_Vector} \id{v}.
  The assignment \id{NV\_DATA\_OMP(v) = v\_data} sets the component array of \id{v} to
  be \id{v\_data} by storing the pointer \id{v\_data}.

  The assignment \id{v\_len = NV\_LENGTH\_OMP(v)} sets \id{v\_len} to be
  the length of \id{v}. On the other hand, the call \id{NV\_LENGTH\_OMP(v) = len\_v}
  sets the length of \id{v} to be \id{len\_v}.

  The assignment \id{v\_num\_threads = NV\_NUM\_THREADS\_OMP(v)} sets \id{v\_num\_threads} to be
  the number of threads from \id{v}. On the other hand, the call \id{NV\_NUM\_THREADS\_OMP(v) = num\_threads\_v}
  sets the number of threads for \id{v} to be \id{num\_threads\_v}.

  Implementation:

  \verb|#define NV_OWN_DATA_OMP(v) ( NV_CONTENT_OMP(v)->own_data )|

  \verb|#define NV_DATA_OMP(v) ( NV_CONTENT_OMP(v)->data )|

  \verb|#define NV_LENGTH_OMP(v) ( NV_CONTENT_OMP(v)->length )|

  \verb|#define NV_NUM_THREADS_OMP(v) ( NV_CONTENT_OMP(v)->num_threads )|

\item \ID{NV\_Ith\_OMP}

  This macro gives access to the individual components of the data
  array of an \id{N\_Vector}.

  The assignment \id{r = NV\_Ith\_OMP(v,i)} sets \id{r} to be the value of
  the \id{i}-th component of \id{v}. The assignment \id{NV\_Ith\_OMP(v,i) = r}
  sets the value of the \id{i}-th component of \id{v} to be \id{r}.

  Here $i$ ranges from $0$ to $n-1$ for a vector of length $n$.

  Implementation:

  \verb|#define NV_Ith_OMP(v,i) ( NV_DATA_OMP(v)[i] )|

\end{itemize}


% ====================================================================
\subsection{NVECTOR\_OPENMP functions}
\label{ss:nvec_openmp_functions}
% ====================================================================

The {\nvecopenmp} module defines OpenMP implementations of all vector operations listed
in Tables \ref{t:nvecops}, \ref{t:nvecfusedops}, \ref{t:nvecarrayops},
and \ref{t:nveclocalops}. Their names are obtained from those in these
tables by appending the suffix \id{\_OpenMP} (e.g. \id{N\_VDestroy\_OpenMP}).
All the standard vector operations listed in \ref{t:nvecops} with the suffix
\id{\_OpenMP} appended are callable via the {\F} 2003 interface by prepending an
`F' (e.g. \id{FN\_VDestroy\_OpenMP}).

The module {\nvecopenmp} provides the following additional user-callable routines:
%%--------------------------------------
\sunmodfunf{N\_VNew\_OpenMP}
{
 This function creates and allocates memory for a OpenMP \id{N\_Vector}.
 Arguments are the vector length and number of threads.
}
{
 N\_Vector N\_VNew\_OpenMP(sunindextype vec\_length, int num\_threads)
}

%%--------------------------------------

\sunmodfunf{N\_VNewEmpty\_OpenMP}
{
  This function creates a new OpenMP \id{N\_Vector} with an empty (\id{NULL}) data array.
}
{
  N\_Vector N\_VNewEmpty\_OpenMP(sunindextype vec\_length, int num\_threads)
}
%%--------------------------------------
\sunmodfunf{N\_VMake\_OpenMP}
{
 This function creates and allocates memory for a OpenMP vector
 with user-provided data array. This function does {\em not} allocate memory for
 \id{v\_data} itself.
}
{
  N\_Vector N\_VMake\_OpenMP(sunindextype vec\_length, realtype *v\_data,
  \newlinefill{N\_Vector N\_VMake\_OpenMP}
  int num\_threads);
}
%%--------------------------------------
\sunmodfun{N\_VCloneVectorArray\_OpenMP}
{
  This function creates (by cloning) an array of \id{count} OpenMP vectors.
}
{
  N\_Vector *N\_VCloneVectorArray\_OpenMP(int count, N\_Vector w)
}
%%--------------------------------------
\sunmodfun{N\_VCloneVectorArrayEmpty\_OpenMP}
{
  This function creates (by cloning) an array of \id{count} OpenMP vectors, each with an
  empty (\id{NULL}) data array.
}
{
  N\_Vector *N\_VCloneVectorArrayEmpty\_OpenMP(int count, N\_Vector w)
}
%%--------------------------------------
\sunmodfun{N\_VDestroyVectorArray\_OpenMP}
{
  This function frees memory allocated for the array of \id{count} variables of type
  \id{N\_Vector} created with \id{N\_VCloneVectorArray\_OpenMP} or with
  \id{N\_VCloneVectorArrayEmpty\_OpenMP}.
}
{
 void N\_VDestroyVectorArray\_OpenMP(N\_Vector *vs, int count)
}
%%--------------------------------------
\sunmodfunf{N\_VPrint\_OpenMP}
{
  This function prints the content of an OpenMP vector to \id{stdout}.
}
{
  void N\_VPrint\_OpenMP(N\_Vector v)
}
%%--------------------------------------
\sunmodfun{N\_VPrintFile\_OpenMP}
{
  This function prints the content of an OpenMP vector to \id{outfile}.
}
{
  void N\_VPrintFile\_OpenMP(N\_Vector v, FILE *outfile)
}
%%--------------------------------------

By default all fused and vector array operations are disabled in the {\nvecopenmp}
module. The following additional user-callable routines are provided to
enable or disable fused and vector array operations for a specific vector. To
ensure consistency across vectors it is recommended to first create a vector
with \id{N\_VNew\_OpenMP}, enable/disable the desired operations for that vector
with the functions below, and create any additional vectors from that vector
using \id{N\_VClone}. This guarantees the new vectors will have the same
operations enabled/disabled as cloned vectors inherit the same enable/disable
options as the vector they are cloned from while vectors created with
\id{N\_VNew\_OpenMP} will have the default settings for the {\nvecopenmp} module.
%%--------------------------------------
\sunmodfun{N\_VEnableFusedOps\_OpenMP}
{
  This function enables (\id{SUNTRUE}) or disables (\id{SUNFALSE}) all fused and
  vector array operations in the OpenMP vector. The return value is \id{0} for
  success and \id{-1} if the input vector or its \id{ops} structure are \id{NULL}.
}
{
  int N\_VEnableFusedOps\_OpenMP(N\_Vector v, booleantype tf)
}
%%--------------------------------------
\sunmodfun{N\_VEnableLinearCombination\_OpenMP}
{
  This function enables (\id{SUNTRUE}) or disables (\id{SUNFALSE}) the linear
  combination fused operation in the OpenMP vector. The return value is \id{0} for
  success and \id{-1} if the input vector or its \id{ops} structure are \id{NULL}.
}
{
  int N\_VEnableLinearCombination\_OpenMP(N\_Vector v, booleantype tf)
}
%%--------------------------------------
\sunmodfun{N\_VEnableScaleAddMulti\_OpenMP}
{
  This function enables (\id{SUNTRUE}) or disables (\id{SUNFALSE}) the scale and
  add a vector to multiple vectors fused operation in the OpenMP vector. The
  return value is \id{0} for success and \id{-1} if the input vector or its
  \id{ops} structure are \id{NULL}.
}
{
  int N\_VEnableScaleAddMulti\_OpenMP(N\_Vector v, booleantype tf)
}
%%--------------------------------------
\sunmodfun{N\_VEnableDotProdMulti\_OpenMP}
{
  This function enables (\id{SUNTRUE}) or disables (\id{SUNFALSE}) the multiple
  dot products fused operation in the OpenMP vector. The return value is \id{0}
  for success and \id{-1} if the input vector or its \id{ops} structure are
  \id{NULL}.
}
{
  int N\_VEnableDotProdMulti\_OpenMP(N\_Vector v, booleantype tf)
}
%%--------------------------------------
\sunmodfun{N\_VEnableLinearSumVectorArray\_OpenMP}
{
  This function enables (\id{SUNTRUE}) or disables (\id{SUNFALSE}) the linear sum
  operation for vector arrays in the OpenMP vector. The return value is \id{0} for
  success and \id{-1} if the input vector or its \id{ops} structure are \id{NULL}.
}
{
  int N\_VEnableLinearSumVectorArray\_OpenMP(N\_Vector v, booleantype tf)
}
%%--------------------------------------
\sunmodfun{N\_VEnableScaleVectorArray\_OpenMP}
{
  This function enables (\id{SUNTRUE}) or disables (\id{SUNFALSE}) the scale
  operation for vector arrays in the OpenMP vector. The return value is \id{0} for
  success and \id{-1} if the input vector or its \id{ops} structure are \id{NULL}.
}
{
  int N\_VEnableScaleVectorArray\_OpenMP(N\_Vector v, booleantype tf)
}
%%--------------------------------------
\sunmodfun{N\_VEnableConstVectorArray\_OpenMP}
{
  This function enables (\id{SUNTRUE}) or disables (\id{SUNFALSE}) the const
  operation for vector arrays in the OpenMP vector. The return value is \id{0} for
  success and \id{-1} if the input vector or its \id{ops} structure are \id{NULL}.
}
{
  int N\_VEnableConstVectorArray\_OpenMP(N\_Vector v, booleantype tf)
}
%%--------------------------------------
\sunmodfun{N\_VEnableWrmsNormVectorArray\_OpenMP}
{
  This function enables (\id{SUNTRUE}) or disables (\id{SUNFALSE}) the WRMS norm
  operation for vector arrays in the OpenMP vector. The return value is \id{0} for
  success and \id{-1} if the input vector or its \id{ops} structure are \id{NULL}.
}
{
  int N\_VEnableWrmsNormVectorArray\_OpenMP(N\_Vector v, booleantype tf)
}
%%--------------------------------------
\sunmodfun{N\_VEnableWrmsNormMaskVectorArray\_OpenMP}
{
  This function enables (\id{SUNTRUE}) or disables (\id{SUNFALSE}) the masked WRMS
  norm operation for vector arrays in the OpenMP vector. The return value is
  \id{0} for success and \id{-1} if the input vector or its \id{ops} structure are
  \id{NULL}.
}
{
  int N\_VEnableWrmsNormMaskVectorArray\_OpenMP(N\_Vector v, booleantype tf)
}
%%--------------------------------------
\sunmodfun{N\_VEnableScaleAddMultiVectorArray\_OpenMP}
{
  This function enables (\id{SUNTRUE}) or disables (\id{SUNFALSE}) the scale and
  add a vector array to multiple vector arrays operation in the OpenMP vector. The
  return value is \id{0} for success and \id{-1} if the input vector or its
  \id{ops} structure are \id{NULL}.
}
{
  int N\_VEnableScaleAddMultiVectorArray\_OpenMP(N\_Vector v, booleantype tf)
}
%%--------------------------------------
\sunmodfun{N\_VEnableLinearCombinationVectorArray\_OpenMP}
{
  This function enables (\id{SUNTRUE}) or disables (\id{SUNFALSE}) the linear
  combination operation for vector arrays in the OpenMP vector. The return value
  is \id{0} for success and \id{-1} if the input vector or its \id{ops} structure
  are \id{NULL}.
}
{
  int N\_VEnableLinearCombinationVectorArray\_OpenMP(N\_Vector v,
  \newlinefill{int N\_VEnableLinearCombinationVectorArray\_OpenMP}
  booleantype tf)
}
%%
%%------------------------------------
%%
\paragraph{\bf Notes}

\begin{itemize}

\item
  When looping over the components of an \id{N\_Vector} \id{v}, it is
  more efficient to first obtain the component array via
  \id{v\_data = NV\_DATA\_OMP(v)} and then access \id{v\_data[i]} within the
  loop than it is to use \id{NV\_Ith\_OMP(v,i)} within the loop.

\item
  {\warn}\id{N\_VNewEmpty\_OpenMP}, \id{N\_VMake\_OpenMP},
  and \id{N\_VCloneVectorArrayEmpty\_OpenMP} set the field
  {\em own\_data} $=$ \id{SUNFALSE}.
  \id{N\_VDestroy\_OpenMP} and \id{N\_VDestroyVectorArray\_OpenMP}
  will not attempt to free the pointer {\em data} for any \id{N\_Vector} with
  {\em own\_data} set to \id{SUNFALSE}. In such a case, it is the user's responsibility to
  deallocate the {\em data} pointer.

\item
  {\warn}To maximize efficiency, vector operations in the {\nvecopenmp} implementation
  that have more than one \id{N\_Vector} argument do not check for
  consistent internal representation of these vectors. It is the user's
  responsibility to ensure that such routines are called with \id{N\_Vector}
  arguments that were all created with the same internal representations.

\end{itemize}


% ====================================================================
\subsection{NVECTOR\_OPENMP Fortran interfaces}
\label{ss:nvec_openmp_fortran}
% ====================================================================

The {\nvecopenmp} module provides a {\F} 2003 module as well as {\F} 77
style interface functions for use from {\F} applications.

\subsubsection*{FORTRAN 2003 interface module}
The \ID{nvector\_openmp\_mod} {\F} module defines interfaces to most
{\nvecopenmp} {\CC} functions using the intrinsic \id{iso\_c\_binding}
module which provides a standardized mechanism for interoperating with {\CC}. As
noted in the {\CC} function descriptions above, the interface functions are
named after the corresponding {\CC} function, but with a leading `F'. For
example, the function \id{N\_VNew\_OpenMP} is interfaced as
\id{FN\_VNew\_OpenMP}.

The {\F} 2003 {\nvecopenmp} interface module can be accessed with the \id{use}
statement, i.e. \id{use fnvector\_openmp\_mod}, and linking to the library
\id{libsundials\_fnvectoropenmp\_mod}.{\em lib} in addition to the {\CC} library.
For details on where the library and module file
\id{fnvector\_openmp\_mod.mod} are installed see Appendix \ref{c:install}.

\subsubsection*{FORTRAN 77 interface functions}
For solvers that include a {\F} 77 interface module, the {\nvecopenmp}
module also includes a {\F}-callable function
\id{FNVINITOMP(code, NEQ, NUMTHREADS, IER)}, to initialize this
module.  Here \id{code} is an input solver id
(1 for {\cvode}, 2 for {\ida}, 3 for {\kinsol}, 4 for {\arkode}); NEQ is
the problem size (declared so as to match C type \id{long int});
NUMTHREADS is the number of threads; and IER is an error return flag
equal 0 for success and -1 for failure.
