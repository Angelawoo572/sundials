% This is a shared SUNDIALS TEX file with description of
% the PETSc nvector wrapper implementation
%
The {\nvecpetsc} is an {\nvector} wrapper around the PETSc vector. It defines the 
{\em content} field of a \id{N\_Vector} to be a structure containing the global 
and local lengths of the vector, a pointer to the PETSc vector,
an {\mpi} communicator, and a boolean flag {\em own\_data} indicating ownership of 
the wrapped PETSc vector.
%%
\begin{verbatim} 
struct _N_VectorContent_petsc {
  long int local_length;
  long int global_length;
  booleantype own_data;
  Vec *pvec;
  MPI_Comm comm;
};
\end{verbatim}
%%
%%--------------------------------------------

The header file to be included when using this module is \id{nvector\_petsc.h}.

Note that the PETSc vector wrapper requires {\sundials} to be built with {\mpi} support.
The following six macros are provided to access the content of a {\nvecpetsc}
vector. The suffix \id{\_PTC} in the names denotes the PETSc wrapper 
version.
\begin{itemize}

\item 
  \ID{NV\_CONTENT\_PTC}

  This macro gives access to the contents of the PETSC
  vector \id{N\_Vector}.
  
  The assignment \id{v\_cont = NV\_CONTENT\_PTC(v)} sets       
  \id{v\_cont} to be a pointer to the \id{N\_Vector} content    
  structure of type \id{struct \_N\_VectorContent\_petsc}.
  
  Implementation:
  
  \verb|#define NV_CONTENT_PTC(v) ( (N_VectorContent_petsc)(v->content) )|
  
\item 
  \ID{NV\_OWN\_DATA\_PTC}, \ID{NV\_PVEC\_PTC}, 
  \ID{NV\_LOCLENGTH\_PTC}, \ID{NV\_GLOBLENGTH\_PTC}
  
  These macros give individual access to the parts of    
  the content of a PETSC \id{N\_Vector}.                        
  
  The assignment \id{v\_pvec = NV\_PVEC\_PTC(v)} sets the PETSc vector
  pointer \id{v\_pvec} to the address of the PETSc vector wrapped by
  the \id{N\_Vector} \id{v}.
  
  The assignment \id{v\_llen = NV\_LOCLENGTH\_PTC(v)} sets \id{v\_llen} to be     
  the length of the local part of \id{v}. 
  The call \id{NV\_LENGTH\_PTC(v) = llen\_v} sets      
  the local length of \id{v} to be \id{llen\_v}.
  
  The assignment \id{v\_glen = NV\_GLOBLENGTH\_PTC(v)} sets \id{v\_glen} to  
  be the global length of the vector \id{v}.                    
  The call \id{NV\_GLOBLENGTH\_PTC(v) = glen\_v} sets the global       
  length of \id{v} to be \id{glen\_v}.
  
  {\warn}Assignment of either the local or global lengths in the above
  manner does not affect the actual vector lengths stored in the PETSc
  vector \id{pvec}, only the integers stored in the {\nvecpetsc} object.
  
  Implementation:
  
  \verb|#define NV_OWN_DATA_PTC(v)   ( NV_CONTENT_PTC(v)->own_data )|

  \verb|#define NV_PVEC_PTC(v)       ( NV_CONTENT_PTC(v)->pvec )|

  \verb|#define NV_LOCLENGTH_PTC(v)  ( NV_CONTENT_PTC(v)->local_length )|

  \verb|#define NV_GLOBLENGTH_PTC(v) ( NV_CONTENT_PTC(v)->global_length )|
  
\item \ID{NV\_COMM\_PTC}

  This macro provides access to the {\mpi} communicator used by the {\nvecpetsc}
  vector.

  Implementation:

  \verb|#define NV_COMM_PTC(v) ( NV_CONTENT_PTC(v)->comm )|

\end{itemize}
%%
%%--------------------------------------------
%%
The {\nvecpetsc} module defines implementations of all vector operations listed 
in Table \ref{t:nvecops}, except for \verb|N_VGetArrayPointer| and 
\verb|N_VSetArrayPointer|. The names of vector operations are obtained from those in 
Table \ref{t:nvecops} by appending the suffix \id{\_petsc} (e.g. \id{N\_VDestroy\_petsc}).
The module {\nvecpetsc}  provides the following additional user-callable routines:
%%
%%
\begin{itemize}

%%--------------------------------------

\item  \ID{N\_VNew\_petsc}
  
  This function creates and allocates a new {\nvector} wrapper and PETSc 
  vector within.
 
  

\begin{verbatim}
N_Vector N_VNew_petsc(MPI_Comm comm, 
                      long int local_length, 
                      long int global_length);
\end{verbatim}
  
%%--------------------------------------

\item \ID{N\_VNewEmpty\_petsc}
 
  This function creates a new {\nvector} wrapper with the pointer to
  the wrapped PETSc vector set to (\id{NULL}).
 
  

\begin{verbatim}
N_Vector N_VNewEmpty_petsc(MPI_Comm comm, 
                           long int local_length, 
                           long int global_length);
\end{verbatim}

  
%%--------------------------------------

\item \ID{N\_VMake\_petsc}
  
  This function creates and allocates memory for a {\nvecpetsc}
  wrapper with a user-provided PETSc vector.
 
  

\begin{verbatim}
N_Vector N_VMake_petsc(Vec *pvec);
\end{verbatim}

%%--------------------------------------


\item \ID{N\_VCloneVectorArray\_petsc}
 
  This function creates (by cloning) an array of \id{count} {\nvecpetsc} vectors.
 
\begin{verbatim}
N_Vector *N_VCloneVectorArray_petsc(int count, N_Vector w);
\end{verbatim}

%%--------------------------------------

\item \ID{N\_VCloneVectorArrayEmpty\_petsc}
 
  This function creates (by cloning) an array of \id{count} {\nvecpetsc} vectors,
  each with pointers to PETSc vectors set to (\id{NULL}).
 
\begin{verbatim}
N_Vector *N_VCloneEmptyVectorArray_petsc(int count, N_Vector w);
\end{verbatim}

%%--------------------------------------

\item \ID{N\_VDestroyVectorArray\_petsc}
 
 This function frees memory allocated for the array of \id{count} variables of
 type \id{N\_Vector} created with \id{N\_VCloneVectorArray\_petsc} or with
 \id{N\_VCloneVectorArrayEmpty\_petsc}.
 

 \verb|void N_VDestroyVectorArray_petsc(N_Vector *vs, int count);|


%%--------------------------------------

\item \ID{N\_VPrint\_petsc}
  
  This function prints the content of the wrapped PETSc vector to stdout.
 
    
  \verb|void N_VPrint_petsc(N_Vector v);|


\end{itemize}
%%
%%------------------------------------
%%
\paragraph{\bf Notes} 
           
\begin{itemize}
                                        
\item
  {\warn}\id{N\_VNewEmpty\_petsc}, \id{N\_VMake\_petsc}, and
  \id{N\_VCloneVectorArrayEmpty\_petsc} set the field {\em own\_data} $=$ \id{FALSE}.   
  \id{N\_VDestroy\_petsc} and \id{N\_VDestroyVectorArray\_petsc}
  will not attempt to free the pointer {\em pvec} for any \id{N\_Vector} with
  {\em own\_data} set to \id{FALSE}. In such a case, it is the user's responsibility to
  deallocate the {\em pvec} pointer.

\item
  {\warn}To maximize efficiency, vector operations in the {\nvecpetsc} implementation
  that have more than one \id{N\_Vector} argument do not check for
  consistent internal representation of these vectors. It is the user's 
  responsibility to ensure that such routines are called with \id{N\_Vector}
  arguments that were all created with the same internal representations.

\end{itemize}

