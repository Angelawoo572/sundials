%% This is a shared SUNDIALS TEX file with a description of the
%% dense sunlinsol implementation
%%

The dense implementation of the {\sunlinsol} module provided with
{\sundials}, {\sunlinsoldense}, is designed to be used with the
corresponding {\sunmatdense} matrix type, and one of the serial or
shared-memory {\nvector} implementations ({\nvecs}, {\nvecopenmp} or
{\nvecpthreads}).  The {\sunlinsoldense} module defines the {\em
content} field of a \id{SUNLinearSolver} to be the following structure:
%%
\begin{verbatim} 
struct _SUNLinearSolverContent_Dense {
  sunindextype N;
  sunindextype *pivots;
  long int last_flag;
};
\end{verbatim}
%%
These entries of the \emph{content} field contain the following
information:
\begin{description}
  \item[N] - size of the linear system,
  \item[pivots] - index array for partial pivoting in LU factorization,
  \item[last\_flag] - last error return flag from internal function evaluations.
\end{description}

This solver is constructed to perform the following operations:
\begin{itemize}
\item The ``setup'' call performs a $LU$ factorization with
  partial (row) pivoting ($\mathcal O(N^3)$ cost), $PA=LU$, where $P$
  is a permutation matrix, $L$ is a lower triangular matrix with 1's
  on the diagonal, and $U$ is an upper triangular matrix.  This
  factorization is stored in-place on the input {\sunmatdense} object
  $A$, with pivoting information encoding $P$ stored in
  the \id{pivots} array.
\item The ``solve'' call performs pivoting and forward and
  backward substitution using the stored \id{pivots} array and the
  $LU$ factors held in the {\sunmatdense} object ($\mathcal O(N^2)$
  cost).
\end{itemize}

\noindent The header file to include when using this module 
is \id{sunlinsol/sunlinsol\_dense.h}. The {\sunlinsoldense} module
is accessible from all {\sundials} solvers \textit{without}
linking to the \\
\id{libsundials\_sunlinsoldense} module library. \\

%%
%%----------------------------------------------
%%

\noindent The {\sunlinsoldense} module defines dense implementations of all
``direct'' linear solver operations listed in
Table \ref{t:sunlinsolops}:
\begin{itemize}
\item \id{SUNLinSolGetType\_Dense}
\item \id{SUNLinSolInitialize\_Dense} -- this does nothing, since all
  consistency checks are performed at solver creation.
\item \id{SUNLinSolSetup\_Dense} -- this performs the $LU$ factorization.
\item \id{SUNLinSolSolve\_Dense} -- this uses the $LU$ factors
  and \id{pivots} array to perform the solve.
\item \id{SUNLinSolLastFlag\_Dense}
\item \id{SUNLinSolSpace\_Dense} -- this only returns information for
  the storage \emph{within} the solver object, i.e.~storage
  for \id{N}, \id{last\_flag}, and \id{pivots}.
\item \id{SUNLinSolFree\_Dense}
\end{itemize}
The module {\sunlinsoldense} provides the following additional
user-callable constructor routine: 
%%
\begin{itemize}

%%--------------------------------------

\item \ID{SUNDenseLinearSolver}

  This function creates and allocates memory for a dense \id{SUNLinearSolver}.
  Its arguments are an {\nvector} and {\sunmatrix}, that it uses to
  determine the linear system size and to assess compatibility with
  the linear solver implementation.

  This routine will perform consistency checks to ensure that it is
  called with consistent {\nvector} and {\sunmatrix} implementations.
  These are currently limited to the {\sunmatdense} matrix type and
  the {\nvecs}, {\nvecopenmp}, and {\nvecpthreads} vector types.  As
  additional compatible matrix and vector implementations are added to
  {\sundials}, these will be included within this compatibility check.

  If either \id{A} or \id{y} are incompatible then this routine will
  return \id{NULL}.

  \verb|SUNLinearSolver SUNDenseLinearSolver(N_Vector y, SUNMatrix A);|

\end{itemize}
%%
%%------------------------------------
%%
For solvers that include a Fortran interface module, the {\sunlinsoldense}
module also includes the Fortran-callable
function \id{FSUNDenseLinSolInit(code, ier)} to initialize
this {\sunlinsoldense} module for a given {\sundials} solver.
Here \id{code} is an integer input solver id (1 for {\cvode}, 2 for {\ida}, 3
for {\kinsol}, 4 for {\arkode}); \id{ier} is an error return flag 
equal to 0 for success and -1 for failure. Both \id{code} and \id{ier}
are declared to match C type \id{int}. This routine must be
called \emph{after} both the {\nvector} and {\sunmatrix} objects have
been initialized. Additionally, when using {\arkode} with a non-identity
mass matrix, the Fortran-callable
function \id{FSUNMassDenseLinSolInit(ier)} initializes this
{\sunlinsoldense} module for solving mass matrix linear systems.
