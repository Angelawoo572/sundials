\section{SUNLinearSolver Examples}\label{ss:sunlinsol_examples}

There are \id{SUNLinearSolver} examples that may be installed for each
implementation; these make use of the functions in \id{test\_sunlinsol.c}.
These example functions show simple usage of the \id{SUNLinearSolver} family
of functions.  The inputs to the examples depend on the linear solver type,
and are output to \texttt{stdout} if the example is run without the
appropriate number of command-line arguments.

\noindent The following is a list of the example functions in \id{test\_sunlinsol.c}:
\begin{itemize}
\item \id{Test\_SUNLinSolGetType}: Verifies the returned solver type against
  the value that should be returned.
\item \id{Test\_SUNLinSolInitialize}: Verifies that \id{SUNLinSolInitialize}
  can be called and returns successfully.
\item \id{Test\_SUNLinSolSetup}: Verifies that \id{SUNLinSolSetup} can
  be called and returns successfully.
\item \id{Test\_SUNLinSolSolve}: Given a {\sunmatrix} object $A$,
  {\nvector} objects $x$ and $b$ (where $Ax=b$) and a desired solution
  tolerance \texttt{tol}, this routine clones $x$ into a new vector $y$,
  calls \\ \noindent
  \id{SUNLinSolSolve} to fill $y$ as the solution to $Ay=b$ (to
  the input tolerance), verifies that each entry in $x$ and $y$
  match to within \texttt{10*tol}, and overwrites $x$ with $y$ prior
  to returning (in case the calling routine would like to investigate
  further).
\item \id{Test\_SUNLinSolSetATimes} (iterative solvers only): Verifies that
  \id{SUNLinSolSetATimes} can be called and returns successfully.
\item \id{Test\_SUNLinSolSetPreconditioner} (iterative solvers only):
  Verifies that \\ \noindent
  \id{SUNLinSolSetPreconditioner} can be called and
  returns successfully.
\item \id{Test\_SUNLinSolSetScalingVectors} (iterative solvers only):
  Verifies that \\ \noindent
  \id{SUNLinSolSetScalingVectors} can be called and
  returns successfully.
\item \id{Test\_SUNLinSolLastFlag}: Verifies that \id{SUNLinSolLastFlag} can
  be called, and outputs the result to \texttt{stdout}.
\item \id{Test\_SUNLinSolNumIters} (iterative solvers only): Verifies that
  \id{SUNLinSolNumIters} can be called, and outputs the result to
  \texttt{stdout}.
\item \id{Test\_SUNLinSolResNorm} (iterative solvers only): Verifies that
  \id{SUNLinSolResNorm} can be called, and that the result is
  non-negative.
\item \id{Test\_SUNLinSolResid} (iterative solvers only): Verifies that
  \id{SUNLinSolResid} can be called.
\item \id{Test\_SUNLinSolSpace} verifies that \id{SUNLinSolSpace} can be
  called, and outputs the results to \texttt{stdout}.
\end{itemize}
We'll note that these tests should be performed in a particular
order.  For either direct or iterative linear
solvers, \id{Test\_SUNLinSolInitialize} must be called
before \id{Test\_SUNLinSolSetup}, which must be called
before \id{Test\_SUNLinSolSolve}.  Additionally, for iterative linear
solvers \\ \noindent
\id{Test\_SUNLinSolSetATimes}, \id{Test\_SUNLinSolSetPreconditioner}
and \\ \noindent
\id{Test\_SUNLinSolSetScalingVectors} should be called
before \id{Test\_SUNLinSolInitialize};
similarly \id{Test\_SUNLinSolNumIters}, \id{Test\_SUNLinSolResNorm}
and \id{Test\_SUNLinSolResid} should be called
after \id{Test\_SUNLinSolSolve}.  These are called in the appropriate
order in all of the example problems.
