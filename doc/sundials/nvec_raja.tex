% This is a shared SUNDIALS TEX file with description of
% the CUDA nvector implementation
%
\section{The NVECTOR\_RAJA implementation}\label{ss:nvec_raja}

The {\nvecraja} module is an experimental {\nvector} implementation using the
\href{https://software.llnl.gov/RAJA/}{\raja} hardware abstraction layer.
In this implementation, {\raja}
allows for {\sundials} vector kernels to run on GPU devices. The module is intended for users
who are already familiar with {\raja} and GPU programming. Building this vector
module requires a C++11 compliant compiler and a CUDA software development toolkit.
Besides the {\cuda} backend, {\raja} has other backends such as serial, OpenMP,
and OpenACC. These backends are not used in this {\sundials} release.
Class \id{Vector} in namespace \id{sunrajavec} manages the vector data layout:
\begin{verbatim}
template <class T, class I>
class Vector {
   I size_;
   I mem_size_;
   T* h_vec_;
   T* d_vec_;
  ...
};
\end{verbatim}
The class members are: vector size (length), size of the vector data
memory block, the global vector size (length), a pointer to  the
vector data on the host, and a pointer to the vector data on the device.
The class \id{Vector} inherits from an empty structure
\begin{verbatim}
struct _N_VectorContent_Raja { };
\end{verbatim}
to interface the C++ class with the {\nvector} C code. When instantiated, the class
\id{Vector} will allocate memory on both the host and the device. Due to the rapid
progress of {\raja} development, we expect that the \id{sunrajavec::Vector}
class will change frequently in future {\sundials} releases. The code is
structured so that it can tolerate significant changes in the
\id{sunrajavec::Vector} class without requiring changes to the user API.

%%
%%--------------------------------------------

The header file to include when using this module is \id{nvector\_raja.h}.
The installed module library to link to are \id{libsundials\_nveccudaraja.\textit{lib}}.
The extension \id{\textit{.lib}} is typically \id{.so} for shared libraries and \id{.a}
for static libraries.

% ====================================================================
\subsection{NVECTOR\_RAJA functions}
\label{ss:nvec_raja_functions}
% ====================================================================

Unlike other native {\sundials} vector types, {\nvecraja} does not provide macros
to access its member variables. Instead, user should use the accessor functions:
%%--------------------------------------
%%--------------------------------------
\sunmodfun{N\_VGetHostArrayPointer\_Raja}
{
  This function returns a pointer to the vector data on the host.
}
{
  realtype *N\_VGetHostArrayPointer\_Raja(N\_Vector v)
}
%%--------------------------------------
\sunmodfun{N\_VGetDeviceArrayPointer\_Raja}
{
  This function returns a pointer to the vector data on the device.
}
{
  realtype *N\_VGetDeviceArrayPointer\_Raja(N\_Vector v)
}
%%--------------------------------------

The {\nvecraja} module defines the implementations of all vector operations listed
in Tables \ref{t:nvecops}, \ref{t:nvecfusedops}, \ref{t:nvecarrayops},
and \ref{t:nveclocalops}, except
for \id{N\_VDotProdMulti}, \id{N\_VWrmsNormVectorArray}, and \\ \noindent
\id{N\_VWrmsNormMaskVectorArray} as support for arrays of reduction vectors is not
yet supported in {\raja}. These function will be added to the {\nvecraja}
implementation in the future. Additionally the vector operations \id{N\_VGetArrayPointer} and
\id{N\_VSetArrayPointer} are not implemented by the {\raja} vector.
As such, this vector cannot be used with the {\sundials} Fortran interfaces,
nor with the {\sundials} direct solvers and preconditioners.
The {\nvecraja} module provides separate functions to access data on the host
and on the device. It also provides methods for copying data from the host to
the device and vice versa. Usage examples of {\nvecraja} are provided in
some example programs for {\cvode} \cite{cvode_ex}.

The names of vector operations are obtained from those in Tables \ref{t:nvecops},
\ref{t:nvecfusedops}, \ref{t:nvecarrayops}, and \ref{t:nveclocalops}
by appending the suffix \id{\_Raja} (e.g. \id{N\_VDestroy\_Raja}).
The module {\nvecraja}  provides the following additional user-callable routines:
%%--------------------------------------
\sunmodfun{N\_VNew\_Raja}
{
  This function creates and allocates memory for a {\cuda} \id{N\_Vector}.
  The vector data array is allocated on both the host and device.
}
{
  N\_Vector N\_VNew\_Raja(sunindextype length)
}
%%--------------------------------------
\sunmodfun{N\_VNewEmpty\_Raja}
{
  This function creates a new {\nvector} wrapper with the pointer to
  the wrapped {\raja} vector set to \id{NULL}. It is used by the
  \id{N\_VNew\_Raja}, \id{N\_VMake\_Raja}, and \id{N\_VClone\_Raja}
  implementations.
}
{
  N\_Vector N\_VNewEmpty\_Raja()
}
%%--------------------------------------
\sunmodfun{N\_VMake\_Raja}
{
  This function creates and allocates memory for an {\nvecraja}
  wrapper around a user-provided \id{sunrajavec::Vector} class.
  Its only argument is of type \newline
  \id{N\_VectorContent\_Raja}, which is the pointer to the class.
}
{
  N\_Vector N\_VMake\_Raja(N\_VectorContent\_Raja c)
}
%%--------------------------------------
\sunmodfun{N\_VCopyToDevice\_Raja}
{
 This function copies host vector data to the device.
}
{
 realtype *N\_VCopyToDevice\_Raja(N\_Vector v)
}
%%--------------------------------------
\sunmodfun{N\_VCopyFromDevice\_Raja}
{
  This function copies vector data from the device to the host.
}
{
  realtype *N\_VCopyFromDevice\_Raja(N\_Vector v)
}
%%--------------------------------------
\sunmodfun{N\_VPrint\_Raja}
{
  This function prints the content of a {\raja} vector to \id{stdout}.
}
{
  void N\_VPrint\_Raja(N\_Vector v)
}
%%--------------------------------------
\sunmodfun{N\_VPrintFile\_Raja}
{
  This function prints the content of a {\raja} vector to \id{outfile}.
}
{
  void N\_VPrintFile\_Raja(N\_Vector v, FILE *outfile)
}
%%--------------------------------------

By default all fused and vector array operations are disabled in the {\nvecraja}
module. The following additional user-callable routines are provided to
enable or disable fused and vector array operations for a specific vector. To
ensure consistency across vectors it is recommended to first create a vector
with \id{N\_VNew\_Raja}, enable/disable the desired operations for that vector
with the functions below, and create any additional vectors from that vector
using \id{N\_VClone}. This guarantees the new vectors will have the same
operations enabled/disabled as cloned vectors inherit the same enable/disable
options as the vector they are cloned from while vectors created with
\id{N\_VNew\_Raja} will have the default settings for the {\nvecraja} module.
%%--------------------------------------
\sunmodfun{N\_VEnableFusedOps\_Raja}
{
  This function enables (\id{SUNTRUE}) or disables (\id{SUNFALSE}) all fused and
  vector array operations in the {\raja} vector. The return value is \id{0} for
  success and \id{-1} if the input vector or its \id{ops} structure are \id{NULL}.
}
{
  int N\_VEnableFusedOps\_Raja(N\_Vector v, booleantype tf)
}
%%--------------------------------------
\sunmodfun{N\_VEnableLinearCombination\_Raja}
{
  This function enables (\id{SUNTRUE}) or disables (\id{SUNFALSE}) the linear
  combination fused operation in the {\raja} vector. The return value is \id{0} for
  success and \id{-1} if the input vector or its \id{ops} structure are \id{NULL}.
}
{
  int N\_VEnableLinearCombination\_Raja(N\_Vector v, booleantype tf)
}
%%--------------------------------------
\sunmodfun{N\_VEnableScaleAddMulti\_Raja}
{
  This function enables (\id{SUNTRUE}) or disables (\id{SUNFALSE}) the scale and
  add a vector to multiple vectors fused operation in the {\raja} vector. The
  return value is \id{0} for success and \id{-1} if the input vector or its
  \id{ops} structure are \id{NULL}.
}
{
  int N\_VEnableScaleAddMulti\_Raja(N\_Vector v, booleantype tf)
}
%%--------------------------------------
\sunmodfun{N\_VEnableLinearSumVectorArray\_Raja}
{
  This function enables (\id{SUNTRUE}) or disables (\id{SUNFALSE}) the linear sum
  operation for vector arrays in the {\raja} vector. The return value is \id{0} for
  success and \id{-1} if the input vector or its \id{ops} structure are \id{NULL}.
}
{
  int N\_VEnableLinearSumVectorArray\_Raja(N\_Vector v, booleantype tf)
}
%%--------------------------------------
\sunmodfun{N\_VEnableScaleVectorArray\_Raja}
{
  This function enables (\id{SUNTRUE}) or disables (\id{SUNFALSE}) the scale
  operation for vector arrays in the {\raja} vector. The return value is \id{0} for
  success and \id{-1} if the input vector or its \id{ops} structure are \id{NULL}.
}
{
  int N\_VEnableScaleVectorArray\_Raja(N\_Vector v, booleantype tf)
}
%%--------------------------------------
\sunmodfun{N\_VEnableConstVectorArray\_Raja}
{
  This function enables (\id{SUNTRUE}) or disables (\id{SUNFALSE}) the const
  operation for vector arrays in the {\raja} vector. The return value is \id{0} for
  success and \id{-1} if the input vector or its \id{ops} structure are \id{NULL}.
}
{
  int N\_VEnableConstVectorArray\_Raja(N\_Vector v, booleantype tf)
}
%%--------------------------------------
\sunmodfun{N\_VEnableScaleAddMultiVectorArray\_Raja}
{
  This function enables (\id{SUNTRUE}) or disables (\id{SUNFALSE}) the scale and
  add a vector array to multiple vector arrays operation in the {\raja} vector. The
  return value is \id{0} for success and \id{-1} if the input vector or its
  \id{ops} structure are \id{NULL}.
}
{
  int N\_VEnableScaleAddMultiVectorArray\_Raja(N\_Vector v, booleantype tf)
}
%%--------------------------------------
\sunmodfun{N\_VEnableLinearCombinationVectorArray\_Raja}
{
  This function enables (\id{SUNTRUE}) or disables (\id{SUNFALSE}) the linear
  combination operation for vector arrays in the {\raja} vector. The return value
  is \id{0} for success and \id{-1} if the input vector or its \id{ops} structure
  are \id{NULL}.
}
{
  int N\_VEnableLinearCombinationVectorArray\_Raja(N\_Vector v,
  \newlinefill{int N\_VEnableLinearCombinationVectorArray\_Raja}
  booleantype tf)
}
%%
%%------------------------------------
%%
\paragraph{\bf Notes}

\begin{itemize}

\item
  When there is a need to access components of an \id{N\_Vector\_Raja}, \id{v},
  it is recommeded to use functions \id{N\_VGetDeviceArrayPointer\_Raja} or
  \id{N\_VGetHostArrayPointer\_Raja}.

% \item
%   {\warn}Unlike in other {\nvector} implementations, vector data will always be
%   deleted when invoking \id{N\_VDestroy\_Raja} and \id{N\_VDestroyVectorArray\_Raja},
%   even when the vector is created using \id{N\_VMake\_Raja}. It is user's responsibility
%   to track memory allocations and deletions when using \id{N\_VMake\_Raja}.

\item
  {\warn}To maximize efficiency, vector operations in the {\nvecraja} implementation
  that have more than one \id{N\_Vector} argument do not check for
  consistent internal representations of these vectors. It is the user's
  responsibility to ensure that such routines are called with \id{N\_Vector}
  arguments that were all created with the same internal representations.

\end{itemize}
