% This is a shared SUNDIALS TEX file with description of
% the CUDA nvector implementation
%
The {\nvecraja} module is an experimental {\nvector} implementation using the
\href{https://software.llnl.gov/RAJA/}{\raja} hardware abstraction layer.
In this implementation, {\raja}
allows for {\sundials} vector kernels to run on GPU devices. The module is intended for users
who are already familiar with {\raja} and GPU programming. Building this vector
module requires a C++11 compliant compiler and a CUDA software development toolkit.
Besides the {\cuda} backend, {\raja} has other backends such as serial, OpenMP,
and OpenAC. These backends are not used in this {\sundials} release.
Class \id{Vector} in namespace \id{sunrajavec} manages the vector data layout:
\begin{verbatim}
template <class T, class I>
class Vector {
  I size_;
  I mem_size_;
  T* h_vec_;
  T* d_vec_;

  ...
};
\end{verbatim}
The class members are: vector size (length), size of the vector data memory block,
and pointers to vector data on the host and on the device. The class \id{Vector}
inherits from an empty structure
\begin{verbatim}
struct _N_VectorContent_Raja {
};
\end{verbatim}
to interface the C++ class with the {\nvector} C code. When instantiated, the class
\id{Vector} will allocate memory on both the host and the device. Due to the rapid
progress of {\raja} development, we expect that the \id{sunrajavec::Vector}
class will change frequently in future {\sundials} releases. The code is
structured so that it can tolerate significant changes in the
\id{sunrajavec::Vector} class without requiring changes to the user API.

%%
%%--------------------------------------------

The {\nvecraja} module can be utilized for single-node parallelism or in a distributed context with MPI.
The header file to include when using this module for single-node parallelism is \id{nvector\_raja.h}.
The header file to include when using this module in the distributed case is \id{nvector\_mpiraja.h}.
Note that only the {\nvecraja} constructor signature differs between the two header files.
The installed module libraries to link to are \id{libsundials\_nvecraja.\textit{lib}} in the single-node case
or \id{libsundials\_nvecmpicudaraja.\textit{lib}} in the distributed case. {\sundials} must be built with
MPI support if the distributed library is desired. The extension, \id{\em.lib}, is typically \id{.so} for shared libraries and \id{.a} for static libraries.

Unlike other native {\sundials} vector types, {\nvecraja} does not provide macros
to access its member variables.Instead, user should use standalone functions in
namespace \id{sunrajavec}.
\begin{itemize}

\item
  \ID{getDevData(N\_Vector v)}

  This function takes \id{N\_Vector} as an argument and returns raw pointer to vector
  data on the device (GPU). It is users responsibility to ensure correct vector is
  passed as the argument.

\item
  \ID{getHostData(N\_Vector v)}

  This function takes \id{N\_Vector} as an argument and returns raw pointer to vector
  data on the host (CPU memory). It is users responsibility to ensure correct vector is
  passed as the argument.

\item \ID{getSize(N\_Vector v)}

  Returns vector's local length.


\item \ID{getGlobalSize(N\_Vector v)}

  Returns vector's global length.


\item \ID{getMPICom(N\_Vector v)}

  Takes \id{N\_Vector} as an argument and returns sundials communicator of type
  \id{SUNDIALS\_Comm}.

\end{itemize}

%Note that {\nvecraja} requires {\sundials} to be built with {\mpi} support.

%%
%%--------------------------------------------
%%
The {\nvecraja} module defines the implementations of all vector operations listed
in Tables \ref{t:nvecops}, \ref{t:nvecfusedops}, and \ref{t:nvecarrayops}, except
for \id{N\_VDotProdMulti}, \id{N\_VWrmsNormVectorArray}, and \\ \noindent
\id{N\_VWrmsNormMaskVectorArray} as support for arrays of reduction vectors is not
yet supported in {\raja}. These function will be added to the {\nvecraja}
implementation in the future. Additionally the vector operations \id{N\_VGetArrayPointer} and
\id{N\_VSetArrayPointer} are not implemented by the {\raja} vector.
As such, this vector cannot be used with {\sundials} Fortran interfaces,
nor with {\sundials} direct solvers and preconditioners.
The {\nvecraja} module provides separate functions to access data on the host
and on the device. It also provides methods for copying data from the host to
the device and vice versa. Usage examples of {\nvecraja} are provided in
some example programs for {\cvode} \cite{cvode_ex}.

The names of vector operations are obtained from those in Tables \ref{t:nvecops},
\ref{t:nvecfusedops}, and \ref{t:nvecarrayops}, by appending the suffix \id{\_Raja}
(e.g. \id{N\_VDestroy\_Raja}).
The module {\nvecraja}  provides the following additional user-callable routines:
%%
%%
\begin{itemize}


%%--------------------------------------

\item \ID{N\_VNew\_Raja}

  \textit{Note: this function signature is defined in the header \id{nvector\_mpiraja.h}
    and should be used when using this module in a distributed context.}
  This function creates and allocates memory for a {\raja} \id{N\_Vector}.
  The memory is allocated on both host and device. Its arguments are local
  and global vector lengths, as well as the {\mpi} communicator. Use this 
  constructor with the \id{libsundials\_nvecmpicudaraja}.\text{lib} library.

\begin{verbatim}
N_Vector N_VNew_Raja(MPI_Comm comm,
                     sunindextype local_length,
                     sunindextype global_length);
\end{verbatim}


%%--------------------------------------

\item \ID{N\_VNew\_Raja}

  \textit{Note: this function signature is defined in the header \id{nvector\_raja.h}
    and should be used when using this module for single-node parallelism.}
  This function creates and allocates memory for a {\raja} \id{N\_Vector}
  on a single node. The memory is allocated on both host and device.
  Its only argument is vector length. Use this constructor with the 
  \id{libsundials\_nveccudaraja}.\text{lib} library.

\begin{verbatim}
N_Vector N_VNew_Raja(sunindextype length);
\end{verbatim}


%%--------------------------------------

\item \ID{N\_VNewEmpty\_Raja}

  This function creates a new {\nvector} wrapper with the pointer to
  the wrapped {\raja} vector set to (\id{NULL}). It is used by the
  \id{N\_VNew\_Raja}, \id{N\_VMake\_Raja}, and \id{N\_VClone\_Raja}
  implementations.

\begin{verbatim}
N_Vector N_VNewEmpty_Raja(sunindextype vec_length);
\end{verbatim}


%%--------------------------------------

\item \ID{N\_VMake\_Raja}

  This function creates and allocates memory for an {\nvecraja}
  wrapper around a user-provided \id{sunrajavec::Vector} class.
  Its only argument is of type \id{N\_VectorContent\_Raja}, which
  is the pointer to the class.

\begin{verbatim}
N_Vector N_VMake_Raja(N_VectorContent_Raja c);
\end{verbatim}

%%--------------------------------------


\item \ID{N\_VGetLength\_Raja}

 This function returns the length of the vector.

 \verb|sunindextype N_VGetLength_Raja(N_Vector v);|

%%--------------------------------------

\item \ID{N\_VGetHostArrayPointer\_Raja}

 This function returns a pointer to the vector data on the host.

 \verb|realtype *N_VGetHostArrayPointer_Raja(N_Vector v);|


%%--------------------------------------

\item \ID{N\_VGetDeviceArrayPointer\_Raja}

 This function returns a pointer to the vector data on the device.

 \verb|realtype *N_VGetDeviceArrayPointer_Raja(N_Vector v);|


%%--------------------------------------

\item \ID{N\_VCopyToDevice\_Raja}

 This function copies host vector data to the device.

 \verb|realtype *N_VCopyToDevice_Raja(N_Vector v);|


%%--------------------------------------

\item \ID{N\_VCopyFromDevice\_Raja}

 This function copies vector data from the device to the host.

 \verb|realtype *N_VCopyFromDevice_Raja(N_Vector v);|


%%--------------------------------------

\item \ID{N\_VPrint\_Raja}

  This function prints the content of a {\raja} vector to \id{stdout}.

  \verb|void N_VPrint_Raja(N_Vector v);|

%%--------------------------------------

\item \ID{N\_VPrintFile\_Raja}

  This function prints the content of a {\raja} vector to \id{outfile}.

  \verb|void N_VPrintFile_Raja(N_Vector v, FILE *outfile);|


\end{itemize}
%%
%%------------------------------------
%%
\paragraph{\bf Notes}

\begin{itemize}

\item
  When there is a need to access components of an \id{N\_Vector\_Raja}, \id{v},
  it is recommeded to use functions \id{N\_VGetDeviceArrayPointer\_Raja} or
  \id{N\_VGetHostArrayPointer\_Raja}.

% \item
%   {\warn}Unlike in other {\nvector} implementations, vector data will always be
%   deleted when invoking \id{N\_VDestroy\_Raja} and \id{N\_VDestroyVectorArray\_Raja},
%   even when the vector is created using \id{N\_VMake\_Raja}. It is user's responsibility
%   to track memory allocations and deletions when using \id{N\_VMake\_Raja}.

\item
  {\warn}To maximize efficiency, vector operations in the {\nvecraja} implementation
  that have more than one \id{N\_Vector} argument do not check for
  consistent internal representations of these vectors. It is the user's
  responsibility to ensure that such routines are called with \id{N\_Vector}
  arguments that were all created with the same internal representations.

\end{itemize}

