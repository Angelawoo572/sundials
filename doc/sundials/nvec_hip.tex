% This is a shared SUNDIALS TEX file with description of
% the HIP nvector implementation
%
\section{The NVECTOR\_HIP implementation}\label{ss:nvec_hip}

The {\nvechip} module is an {\nvector} implementation using the AMD ROCm {\hip}
library. The module allows for {\sundials} vector kernels to run on AMD or
NVIDIA GPU devices. It is intended for users who are already familiar with
{\hip} and GPU programming. Building this vector module requires the HIP-clang
compiler. The vector content layout is as follows:

\begin{verbatim}
struct _N_VectorContent_Hip
{
  sunindextype       length;
  booleantype        own_exec;
  booleantype        own_helper;
  SUNMemory          host_data;
  SUNMemory          device_data;
  SUNHipExecPolicy*  stream_exec_policy;
  SUNHipExecPolicy*  reduce_exec_policy;
  SUNMemoryHelper    mem_helper;
  void*              priv; /* 'private' data */
};

typedef struct _N_VectorContent_Hip *N_VectorContent_Hip;
\end{verbatim}

The content members are the vector length (size), a boolean flag that signals if
the vector owns the data (i.e. it is in charge of freeing the data), pointers to
vector data on the host and the device, pointers to \id{SUNHipExecPolicy}
implementations that control how the HIP kernels are launched for streaming and
reduction vector kernels, and a private data structure which holds additional members
that should not be accessed directly.

When instantiated with \id{N\_VNew\_Hip}, the underlying data will be allocated
memory on both the host and the device. Alternatively, a user can provide host
and device data arrays by using the \id{N\_VMake\_Hip} constructor. To use {\hip}
managed memory, the constructors \id{N\_VNewManaged\_Hip} and \newline
\id{N\_VMakeManaged\_Hip} are provided. Details on each of these constructors
are provided below.

To use the {\nvechip} module, the header file to include is \id{nvector\_hip.h},
and the library to link to is \id{libsundials\_nvechip.\textit{lib}}. The
extension \id{\textit{.lib}} is typically \id{.so} for shared libraries and \id{.a}
for static libraries.

% ====================================================================
\subsection{NVECTOR\_HIP functions}
\label{ss:nvec_hip_functions}
% ====================================================================

Unlike other native {\sundials} vector types, {\nvechip} does not provide macros
to access its member variables. Instead, user should use the accessor functions:
%%--------------------------------------
\sunmodfun{N\_VGetHostArrayPointer\_Hip}
{
  This function returns a pointer to the vector data on the host.
}
{
  realtype *N\_VGetHostArrayPointer\_Hip(N\_Vector v)
}
%%--------------------------------------
\sunmodfun{N\_VGetDeviceArrayPointer\_Hip}
{
  This function returns a pointer to the vector data on the device.
}
{
  realtype *N\_VGetDeviceArrayPointer\_Hip(N\_Vector v)
}
%%--------------------------------------
\sunmodfun{N\_VIsManagedMemory\_Hip}
{
  This function returns a boolean flag indicating if the vector
  data is allocated in managed memory or not.
}
{
  booleantype *N\_VIsManagedMemory\_Hip(N\_Vector v)
}
%%--------------------------------------------

The {\nvechip} module defines implementations of all vector operations listed
in Tables \ref{ss:nvecops}, \ref{ss:nvecfusedops}, \ref{ss:nvecarrayops}
and \ref{ss:nveclocalops}, except for \id{N\_VSetArrayPointer}.
The names of vector operations are obtained from those in Tables \ref{ss:nvecops},
\ref{ss:nvecfusedops}, \ref{ss:nvecarrayops}, and \ref{ss:nveclocalops}
by appending the suffix \id{\_Hip}
(e.g. \id{N\_VDestroy\_Hip}). The module {\nvechip} provides the following
functions:
%%--------------------------------------
\sunmodfun{N\_VNew\_Hip}
{
  This function creates an empty {\hip} \id{N\_Vector} with the data
  pointers set to \id{NULL}.
}
{
  N\_Vector N\_VNew\_Hip(sunindextype length)
}
%%--------------------------------------
\sunmodfun{N\_VNewManaged\_Hip}
{
  This function creates and allocates memory for a {\hip} \id{N\_Vector}.
  The vector data array is allocated in managed memory.
}
{
  N\_Vector N\_VNewManaged\_Hip(sunindextype length)
}
%%--------------------------------------
\sunmodfun{N\_VNewEmpty\_Hip}
{
  This function creates a new {\nvector} wrapper with the pointer to
  the wrapped {\hip} vector set to \id{NULL}. It is used by the
  \id{N\_VNew\_Hip}, \id{N\_VMake\_Hip}, and \id{N\_VClone\_Hip}
  implementations.
}
{
  N\_Vector N\_VNewEmpty\_Hip()
}
%%--------------------------------------
\sunmodfun{N\_VMake\_Hip}
{
  This function creates an {\nvechip} with user-supplied vector data arrays
  \id{h\_vdata} and \id{d\_vdata}. This function does not allocate memory for
  data itself.
}
{
  N\_Vector N\_VMake\_Hip(sunindextype length, realtype *h\_data, realtype *dev\_data)
}
%%--------------------------------------
\sunmodfun{N\_VMakeManaged\_Hip}
{
  This function creates an {\nvechip} with a user-supplied managed memory data
  array. This function does not allocate memory for data itself.
}
{
  N\_Vector N\_VMakeManaged\_Hip(sunindextype length, realtype *vdata)
}

The module {\nvechip} also provides the following user-callable routines:
%%--------------------------------------
\sunmodfun{N\_VSetKernelExecPolicy\_Hip}
{
  This function sets the execution policies which control the kernel parameters
  utilized when launching the streaming and reduction HIP kernels. By default
  the vector is setup to use the \id{SUNHipThreadDirectExecPolicy} and
  \id{SUNHipBlockReduceExecPolicy}. Any custom execution policy for reductions
  must ensure that the grid dimensions (number of thread blocks) is a multiple of
  the HIP warp size (64 when targeting AMD GPUs and 32 when targing NVIDIA GPUs).
  See section \ref{ss:sunhipexecpolicy} below for more
  information about the \id{SUNHipExecPolicy} class.

  \textit{Note: All vectors used in a single instance of a {\sundials} solver must
  use the same execution policy. It is \textbf{strongly recommended} that
  this function is called immediately after constructing the vector, and
  any subsequent vector be created by cloning to ensure consistent execution
  policies across vectors.}
}
{
  void N\_VSetKernelExecPolicy\_Hip(N\_Vector v,
  \newlinefill{void N\_VSetKernelExecPolicy\_Hip}
  SUNHipExecPolicy* stream\_exec\_policy,
  \newlinefill{void N\_VSetKernelExecPolicy\_Hip}
  SUNHipExecPolicy* reduce\_exec\_policy);
}
%%--------------------------------------
\sunmodfun{N\_VCopyToDevice\_Hip}
{
 This function copies host vector data to the device.
}
{
 void N\_VCopyToDevice\_Hip(N\_Vector v)
}
%%--------------------------------------
\sunmodfun{N\_VCopyFromDevice\_Hip}
{
 This function copies vector data from the device to the host.
}
{
 void N\_VCopyFromDevice\_Hip(N\_Vector v)
}
%%--------------------------------------
\sunmodfun{N\_VPrint\_Hip}
{
  This function prints the content of a {\hip} vector to \id{stdout}.
}
{
  void N\_VPrint\_Hip(N\_Vector v)
}
%%--------------------------------------
\sunmodfun{N\_VPrintFile\_Hip}
{
  This function prints the content of a {\hip} vector to \id{outfile}.
}
{
  void N\_VPrintFile\_Hip(N\_Vector v, FILE *outfile)
}
%%--------------------------------------

By default all fused and vector array operations are disabled in the {\nvechip}
module. The following additional user-callable routines are provided to
enable or disable fused and vector array operations for a specific vector. To
ensure consistency across vectors it is recommended to first create a vector
with \id{N\_VNew\_Hip}, enable/disable the desired operations for that vector
with the functions below, and create any additional vectors from that vector
using \id{N\_VClone}. This guarantees the new vectors will have the same
operations enabled/disabled as cloned vectors inherit the same enable/disable
options as the vector they are cloned from while vectors created with
\id{N\_VNew\_Hip} will have the default settings for the {\nvechip} module.
%%--------------------------------------
\sunmodfun{N\_VEnableFusedOps\_Hip}
{
  This function enables (\id{SUNTRUE}) or disables (\id{SUNFALSE}) all fused and
  vector array operations in the {\hip} vector. The return value is \id{0} for
  success and \id{-1} if the input vector or its \id{ops} structure are \id{NULL}.
}
{
  int N\_VEnableFusedOps\_Hip(N\_Vector v, booleantype tf)
}
%%--------------------------------------
\sunmodfun{N\_VEnableLinearCombination\_Hip}
{
  This function enables (\id{SUNTRUE}) or disables (\id{SUNFALSE}) the linear
  combination fused operation in the {\hip} vector. The return value is \id{0} for
  success and \id{-1} if the input vector or its \id{ops} structure are \id{NULL}.
}
{
  int N\_VEnableLinearCombination\_Hip(N\_Vector v, booleantype tf)
}
%%--------------------------------------
\sunmodfun{N\_VEnableScaleAddMulti\_Hip}
{
  This function enables (\id{SUNTRUE}) or disables (\id{SUNFALSE}) the scale and
  add a vector to multiple vectors fused operation in the {\hip} vector. The
  return value is \id{0} for success and \id{-1} if the input vector or its
  \id{ops} structure are \id{NULL}.
}
{
  int N\_VEnableScaleAddMulti\_Hip(N\_Vector v, booleantype tf)
}
%%--------------------------------------
\sunmodfun{N\_VEnableDotProdMulti\_Hip}
{
  This function enables (\id{SUNTRUE}) or disables (\id{SUNFALSE}) the multiple
  dot products fused operation in the {\hip} vector. The return value is \id{0}
  for success and \id{-1} if the input vector or its \id{ops} structure are
  \id{NULL}.
}
{
  int N\_VEnableDotProdMulti\_Hip(N\_Vector v, booleantype tf)
}
%%--------------------------------------
\sunmodfun{N\_VEnableLinearSumVectorArray\_Hip}
{
  This function enables (\id{SUNTRUE}) or disables (\id{SUNFALSE}) the linear sum
  operation for vector arrays in the {\hip} vector. The return value is \id{0} for
  success and \id{-1} if the input vector or its \id{ops} structure are \id{NULL}.
}
{
  int N\_VEnableLinearSumVectorArray\_Hip(N\_Vector v, booleantype tf)
}
%%--------------------------------------
\sunmodfun{N\_VEnableScaleVectorArray\_Hip}
{
  This function enables (\id{SUNTRUE}) or disables (\id{SUNFALSE}) the scale
  operation for vector arrays in the {\hip} vector. The return value is \id{0} for
  success and \id{-1} if the input vector or its \id{ops} structure are \id{NULL}.
}
{
  int N\_VEnableScaleVectorArray\_Hip(N\_Vector v, booleantype tf)
}
%%--------------------------------------
\sunmodfun{N\_VEnableConstVectorArray\_Hip}
{
  This function enables (\id{SUNTRUE}) or disables (\id{SUNFALSE}) the const
  operation for vector arrays in the {\hip} vector. The return value is \id{0} for
  success and \id{-1} if the input vector or its \id{ops} structure are \id{NULL}.
}
{
  int N\_VEnableConstVectorArray\_Hip(N\_Vector v, booleantype tf)
}
%%--------------------------------------
\sunmodfun{N\_VEnableWrmsNormVectorArray\_Hip}
{
  This function enables (\id{SUNTRUE}) or disables (\id{SUNFALSE}) the WRMS norm
  operation for vector arrays in the {\hip} vector. The return value is \id{0} for
  success and \id{-1} if the input vector or its \id{ops} structure are \id{NULL}.
}
{
  int N\_VEnableWrmsNormVectorArray\_Hip(N\_Vector v, booleantype tf)
}
%%--------------------------------------
\sunmodfun{N\_VEnableWrmsNormMaskVectorArray\_Hip}
{
  This function enables (\id{SUNTRUE}) or disables (\id{SUNFALSE}) the masked WRMS
  norm operation for vector arrays in the {\hip} vector. The return value is
  \id{0} for success and \id{-1} if the input vector or its \id{ops} structure are
  \id{NULL}.
}
{
  int N\_VEnableWrmsNormMaskVectorArray\_Hip(N\_Vector v, booleantype tf)
}
%%--------------------------------------
\sunmodfun{N\_VEnableScaleAddMultiVectorArray\_Hip}
{
  This function enables (\id{SUNTRUE}) or disables (\id{SUNFALSE}) the scale and
  add a vector array to multiple vector arrays operation in the {\hip} vector. The
  return value is \id{0} for success and \id{-1} if the input vector or its
  \id{ops} structure are \id{NULL}.
}
{
  int N\_VEnableScaleAddMultiVectorArray\_Hip(N\_Vector v, booleantype tf)
}
%%--------------------------------------
\sunmodfun{N\_VEnableLinearCombinationVectorArray\_Hip}
{
  This function enables (\id{SUNTRUE}) or disables (\id{SUNFALSE}) the linear
  combination operation for vector arrays in the {\hip} vector. The return value
  is \id{0} for success and \id{-1} if the input vector or its \id{ops} structure
  are \id{NULL}.
}
{
  int N\_VEnableLinearCombinationVectorArray\_Hip(N\_Vector v,
  \newlinefill{int N\_VEnableLinearCombinationVectorArray\_Hip}
  booleantype tf)
}
%%
%%------------------------------------
%%
\paragraph{\bf Notes}

\begin{itemize}

\item
  When there is a need to access components of an \id{N\_Vector\_Hip}, \id{v},
  it is recommended to use functions \id{N\_VGetDeviceArrayPointer\_Hip} or
  \id{N\_VGetHostArrayPointer\_Hip}. However, when using managed memory, the
  function \id{N\_VGetArrayPointer} may also be used.

\item
  {\warn}To maximize efficiency, vector operations in the {\nvechip} implementation
  that have more than one \id{N\_Vector} argument do not check for
  consistent internal representations of these vectors. It is the user's
  responsibility to ensure that such routines are called with \id{N\_Vector}
  arguments that were all created with the same internal representations.

\end{itemize}

%% Eventually we should move this section to a "Using <package> in GPU Environments" section
\subsection{The SUNHipExecPolicy Class}\label{ss:sunhipexecpolicy}

In order to provide maximum flexibility to users, the HIP kernel execution parameters used
by kernels within SUNDIALS are defined by objects of the \id{sundials::HipExecPolicy}
abstract class type (this class can be accessed in the global namespace as \id{SUNHipExecPolicy}).
Thus, users may provide custom execution policies that fit the needs of their problem. The
\id{sundials::HipExecPolicy} is defined in the header file \id{sundials\_hip\_policies.hpp},
and is as follows:

\begin{verbatim}
class HipExecPolicy
{
public:
  virtual size_t gridSize(size_t numWorkUnits = 0, size_t blockDim = 0) const = 0;
  virtual size_t blockSize(size_t numWorkUnits = 0, size_t gridDim = 0) const = 0;
  virtual hipStream_t stream() const = 0;
  virtual HipExecPolicy* clone() const = 0;
  virtual ~HipExecPolicy() {}
};
\end{verbatim}

To define a custom execution policy, a user simply needs to create a class that inherits from
the abstract class and implements the methods. The {\sundials} provided
\id{sundials::HipThreadDirectExecPolicy} (aka in the global namespace as
\id{SUNHipThreadDirectExecPolicy}) class is a good example of a what a custom execution policy
may look like:

\begin{verbatim}
class HipThreadDirectExecPolicy : public HipExecPolicy
{
public:
  HipThreadDirectExecPolicy(const size_t blockDim, const hipStream_t stream = 0)
    : blockDim_(blockDim), stream_(stream)
  {}

  HipThreadDirectExecPolicy(const HipThreadDirectExecPolicy& ex)
    : blockDim_(ex.blockDim_), stream_(ex.stream_)
  {}

  virtual size_t gridSize(size_t numWorkUnits = 0, size_t blockDim = 0) const
  {
    return (numWorkUnits + blockSize() - 1) / blockSize();
  }

  virtual size_t blockSize(size_t numWorkUnits = 0, size_t gridDim = 0) const
  {
    return blockDim_;
  }

  virtual hipStream_t stream() const
  {
    return stream_;
  }

  virtual HipExecPolicy* clone() const
  {
    return static_cast<HipExecPolicy*>(new HipThreadDirectExecPolicy(*this));
  }

private:
  const hipStream_t stream_;
  const size_t blockDim_;
};
\end{verbatim}

In total, {\sundials} provides 3 execution policies:

\begin{enumerate}
  \item \id{SUNHipThreadDirectExecPolicy(const size\_t blockDim, const hipStream\_t stream = 0)}
    maps each HIP thread to a work unit. The number of threads per block (blockDim) can be set
    to anything. The grid size will be calculated so that there are enough threads for one
    thread per element. If a HIP stream is provided, it will be used to execute the kernel.

  \item \id{SUNHipGridStrideExecPolicy(const size\_t blockDim, const size\_t gridDim, const hipStream\_t stream = 0)}
    is for kernels that use grid stride loops. The number of threads per block (blockDim)
    can be set to anything. The number of blocks (gridDim) can be set to anything. If a
    HIP stream is provided, it will be used to execute the kernel.

  \item \id{SUNHipBlockReduceExecPolicy(const size\_t blockDim, const size\_t gridDim, const hipStream\_t stream = 0)}
    is for kernels performing a reduction across individual thread blocks. The number of threads
    per block (blockDim) can be set to any valid multiple of the HIP warp size. The grid size
    (gridDim) can be set to any value greater than 0. If it is set to 0, then the grid size
    will be chosen so that there is enough threads for one thread per work unit. If a
    HIP stream is provided, it will be used to execute the kernel.
\end{enumerate}

For example, a policy that uses 128 threads per block and a user provided stream can be
created like so:

\begin{verbatim}
  hipStream_t stream;
  hipStreamCreate(&stream);
  SUNHipThreadDirectExecPolicy thread_direct(128, stream);
\end{verbatim}

These default policy objects can be reused for multiple {\sundials} data structures
since they do not hold any modifiable state information.
