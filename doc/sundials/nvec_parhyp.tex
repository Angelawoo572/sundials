% This is a shared SUNDIALS TEX file with description of
% the MPI parallel ParHyp hypre nvector implementation
%
The {\nvecph} implementation of the {\nvector} module provided with
{\sundials} is a wrapper around {\hypre}'s ParVector class. 
Most of the vector kernels simply call {\hypre} vector operations. 
The implementation defines the {\em content} field of \id{N\_Vector} to 
be a structure containing the global and local lengths of the vector, a 
pointer to an object of type \id{hypre\_ParVector}, an {\mpi} communicator, 
and a boolean flag {\em own\_parvector} indicating ownership of the
{\hypre} parallel vector object {\em x}.
%%
%%
\begin{verbatim}
struct _N_VectorContent_ParHyp {
  sunindextype local_length;
  sunindextype global_length;
  booleantype own_parvector;
  MPI_Comm comm;
  hypre_ParVector *x;
};
\end{verbatim}
%%
%%--------------------------------------------
The header file to include when using this module is \id{nvector\_parhyp.h}.
The installed module library to link to is
\id{libsundials\_nvecparhyp.\textit{lib}}
where \id{\em.lib} is typically \id{.so} for shared libraries and \id{.a}
for static libraries.

Unlike native {\sundials} vector types, {\nvecph} does not provide macros 
to access its member variables.
Note that {\nvecph} requires {\sundials} to be built with {\mpi} support.

%%
%%--------------------------------------------
%%
The {\nvecph} module defines implementations of all vector operations 
listed in Tables \ref{t:nvecops}, \ref{t:nvecfusedops}, and \ref{t:nvecarrayops}, except
for \id{N\_VSetArrayPointer} and \id{N\_VGetArrayPointer}, because accessing raw vector
data is handled by low-level {\hypre} functions.
As such, this vector is not available for use with {\sundials} Fortran interfaces.
When access to raw vector data is needed, one
should extract the {\hypre} vector first, and then use {\hypre}
methods to access the data. Usage examples of {\nvecph} are provided in
the \id{cvAdvDiff\_non\_ph.c} example program for {\cvode} \cite{cvode_ex}
and the \id{ark\_diurnal\_kry\_ph.c} example program for {\arkode} \cite{arkode_ex}.

The names of parhyp methods are obtained from those in Tables \ref{t:nvecops},
\ref{t:nvecfusedops}, and \ref{t:nvecarrayops}
by appending the suffix \id{\_ParHyp} (e.g. \id{N\_VDestroy\_ParHyp}).
The module {\nvecph} provides the following additional user-callable routines:
%%
%%
\begin{itemize}

%%--------------------------------------

\item \ID{N\_VNewEmpty\_ParHyp}
 
  This function creates a new parhyp \id{N\_Vector} with the pointer to the {\hypre} 
  vector set to \id{NULL}.


\begin{verbatim}
N_Vector N_VNewEmpty_ParHyp(MPI_Comm comm, 
                            sunindextype local_length, 
                            sunindextype global_length);
\end{verbatim}

  
%%--------------------------------------

\item \ID{N\_VMake\_ParHyp}
  
  This function creates an \verb|N_Vector| wrapper around an existing
  {\hypre} parallel vector. It does {\em not} allocate memory for \id{x} 
  itself.  

\begin{verbatim}
N_Vector N_VMake_ParHyp(hypre_ParVector *x);
\end{verbatim}

%%--------------------------------------

\item \ID{N\_VGetVector\_ParHyp}
  
  This function returns a pointer to the underlying {\hypre} vector.
 
    
  \verb|hypre_ParVector *N_VGetVector_ParHyp(N_Vector v);|


%%--------------------------------------

\item \ID{N\_VCloneVectorArray\_ParHyp}
 
  This function creates (by cloning) an array of \id{count} parallel vectors.
 
\begin{verbatim}
N_Vector *N_VCloneVectorArray_ParHyp(int count, N_Vector w);
\end{verbatim}

%%--------------------------------------

\item \ID{N\_VCloneVectorArrayEmpty\_ParHyp}
 
  This function creates (by cloning) an array of \id{count} parallel vectors,
  each with an empty (\id{NULL}) data array.
 
\begin{verbatim}
N_Vector *N_VCloneVectorArrayEmpty_ParHyp(int count, N_Vector w);
\end{verbatim}

%%--------------------------------------

\item \ID{N\_VDestroyVectorArray\_ParHyp}
 
 This function frees memory allocated for the array of \id{count}  variables of
 type \id{N\_Vector} created with \id{N\_VCloneVectorArray\_ParHyp} or with
 \id{N\_VCloneVectorArrayEmpty\_ParHyp}.
 

 \verb|void N_VDestroyVectorArray_ParHyp(N_Vector *vs, int count);|


%%--------------------------------------

\item \ID{N\_VPrint\_ParHyp}
  
  This function prints the local content of a parhyp vector to \id{stdout}.
    
  \verb|void N_VPrint_ParHyp(N_Vector v);|

%%--------------------------------------

\item \ID{N\_VPrintFile\_ParHyp}
  
  This function prints the local content of a parhyp vector to \id{outfile}.
    
  \verb|void N_VPrintFile_ParHyp(N_Vector v, FILE *outfile);|

\end{itemize}

%%
%%------------------------------------
%%
\paragraph{\bf Notes} 
           
\begin{itemize}
                                        
\item
  When there is a need to access components of an \id{N\_Vector\_ParHyp}, \id{v}, 
  it is recommended to extract the {\hypre} vector via       
  \id{x\_vec = N\_VGetVector\_ParHyp(v)} and then access components using 
  appropriate {\hypre} functions.        
                                                               
\item
  {\warn}\id{N\_VNewEmpty\_ParHyp}, \id{N\_VMake\_ParHyp}, 
  and \id{N\_VCloneVectorArrayEmpty\_ParHyp} set the field 
  {\em own\_parvector} to \id{SUNFALSE}. 
  \id{N\_VDestroy\_ParHyp} and \id{N\_VDestroyVectorArray\_ParHyp}
  will not attempt to delete an underlying {\hypre} vector for any \id{N\_Vector} 
  with {\em own\_parvector} set to \id{SUNFALSE}. In such a case, it is the 
  user's responsibility to delete the underlying vector.

\item
  {\warn}To maximize efficiency, vector operations in the {\nvecph} implementation
  that have more than one \id{N\_Vector} argument do not check for
  consistent internal representations of these vectors. It is the user's 
  responsibility to ensure that such routines are called with \id{N\_Vector}
  arguments that were all created with the same internal representations.

\end{itemize}

% For solvers that include a Fortran interface module, the {\nvecph} module
% also includes a Fortran-callable function
% \id{FNVINITPH(COMM, code, NLOCAL, NGLOBAL, IER)},
% to initialize this {\nvecph} module.  Here \id{COMM} is the MPI communicator,
% \id{code} is an input solver id (1 for {\cvode}, 2 for {\ida}, 3 for {\kinsol},
% 4 for {\arkode}); \id{NLOCAL} and \id{NGLOBAL} are the local and global
% vector sizes, respectively (declared so as to match C type \id{long int});
% and IER is an error return flag equal 0 for success and -1 for failure.
% 
% {\warn}Note: If the header file \id{sundials\_config.h} defines
% \id{SUNDIALS\_MPI\_COMM\_F2C} to be $1$ (meaning the {\mpi}
% implementation used to build {\sundials} includes the
% \id{MPI\_Comm\_f2c} function), then \id{COMM} can be any valid
% {\mpi} communicator. Otherwise, \id{MPI\_COMM\_WORLD} will be used, so
% just pass an integer value as a placeholder.
