% This is a shared SUNDIALS TEX file with description of
% the generic nvector abstraction
%
The {\sundials} solvers are written in a data-independent manner.
They all operate on generic vectors (of type \Id{N\_Vector}) through a set of
operations defined by the particular {\nvector} implementation.
Users can provide their own specific implementation of the {\nvector}
module, or use one of the implementations provided with {\sundials}.
The generic {\nvector} is described below and the implementations
provided with {\sundials} are described in the following sections.

% ====================================================================
\section{The NVECTOR API}
\label{s:nvector_api}
% ====================================================================

The generic {\nvector} API can be broken down into groups of functions:
the core vector operations, the fused vector operations, the vector array
operations, the local reduction operations, the exchange operations, and
finally some utility functions. All but the last group are defined by a
particular {\nvector} implementation. The utility functions are defined
by the generic {\nvector} itself.

% ====================================================================
\subsection{NVECTOR core functions}\label{ss:nvecops}

\ucfunctionf{N\_VGetVectorID}
{
  id = N\_VGetVectorID(w);
}
{
  Returns the vector type identifier for the vector \id{w}. It is used to determine the
  vector implementation type (e.g.~serial, parallel,\ldots) from the abstract
  \id{N\_Vector} interface.
}
{
  \begin{args}[w]
  \item[w] (\id{N\_Vector}) a {\nvector} object
  \end{args}
}
{
  This function returns an \id{N\_Vector\_ID}. Possible values are given in Table
  \ref{t:vectorIDs}.
}
{}

\ucfunctionf{N\_VClone}
{
  v = N\_VClone(w);
}
{
  Creates a new \id{N\_Vector} of the same type as an existing vector \id{w} and sets the
  {\em ops} field. It does not copy the vector, but rather allocates storage for the new vector.
}
{
  \begin{args}[w]
  \item[w] (\id{N\_Vector}) a {\nvector} object
  \end{args}
}
{
  This function returns an \id{N\_Vector} object. If an error occurs, then this
  routine will return \id{NULL}.
}
{}

\ucfunctionf{N\_VCloneEmpty}
{
  v = N\_VCloneEmpty(w);
}
{
  Creates a new \id{N\_Vector} of the same type as an existing vector \id{w} and sets the
  {\em ops} field. It does not allocate storage for data.
}
{
  \begin{args}[w]
  \item[w] (\id{N\_Vector}) a {\nvector} object
  \end{args}
}
{
  This function returns an \id{N\_Vector} object. If an error occurs, then this
  routine will return \id{NULL}.
}
{}

\ucfunctionf{N\_VDestroy}
{
  N\_VDestroy(v);
}
{
  Destroys the \id{N\_Vector} \id{v} and frees memory allocated for its
  internal data.
}
{
  \begin{args}[v]
  \item[v] (\id{N\_Vector}) a {\nvector} object to destroy
  \end{args}
}
{}
{}

\ucfunctionfl{N\_VSpace}
{
  N\_VSpace(v, \&lrw, \&liw);
}
{
  Returns storage requirements for one \id{N\_Vector}.
  \id{lrw} contains the number of realtype words and \id{liw}
  contains the number of integer words, This function is advisory
  only, for use in determining a user's total space requirements;
  it could be a dummy function in a user-supplied
  {\nvector} module if that information is not of interest.
}
{
  \begin{args}[v]
  \item[v] (\id{N\_Vector}) a {\nvector} object
  \item[lrw] (\id{sunindextype*}) out parameter containing the number of realtype words
  \item[liw] (\id{sunindextype*}) out parameter containing the number of integer words
  \end{args}
}
{}
{}
{
  integer(c\_long) :: lrw(1), liw(1)\\
  call FN\_VSpace\_Serial(v, lrw, liw)
}

\ucfunctionf{N\_VGetArrayPointer}
{
  vdata = N\_VGetArrayPointer(v);
}
{
  Returns a pointer to a \id{realtype} array from the \id{N\_Vector} \id{v}.
  Note that this assumes that the internal data in \id{N\_Vector} is
  a contiguous array of \id{realtype}. This routine is only used in the
  solver-specific interfaces to the dense and banded (serial) linear
  solvers, the sparse linear solvers (serial and threaded), and in the
  interfaces to the banded (serial) and band-block-diagonal (parallel)
  preconditioner modules provided with {\sundials}.
}
{
  \begin{args}[v]
  \item[v] (\id{N\_Vector}) a {\nvector} object
  \end{args}
}
{
  \id{realtype*}
}
{}

\ucfunctionf{N\_VSetArrayPointer}
{
  N\_VSetArrayPointer(vdata, v);
}
{
  Overwrites the pointer to the data in an \id{N\_Vector} with a given \id{realtype*}.
  Note that this assumes that the internal data in \id{N\_Vector} is a contiguous
  array of \id{realtype}. This routine is only used in the interfaces to the dense
  (serial) linear solver, hence need not exist in a user-supplied {\nvector} module
  for a parallel environment.
}
{
  \begin{args}[v]
  \item[v] (\id{N\_Vector}) a {\nvector} object
  \end{args}
}
{}
{}

\ucfunctionf{N\_VGetCommunicator}
{
  N\_VGetCommunicator(v);
}
{

  Returns a pointer to the \id{MPI\_Comm} object associated with the
  vector (if applicable). For MPI-unaware vector implementations, this
  should return \id{NULL}.
}
{
  \begin{args}[v]
  \item[v] (\id{N\_Vector}) a {\nvector} object
  \end{args}
}
{
  A \id{void *} pointer to the \id{MPI\_Comm} object if the vector is MPI-aware,
  otherwise \id{NULL}.
}
{}

\ucfunctionf{N\_VGetLength}
{
  N\_VGetLength(v);
}
{
  Returns the global length (number of `active' entries) in the
  {\nvector} \id{v}.  This value should be cumulative across all
  processes if the vector is used in a parallel environment.  If \id{v}
  contains additional storage, e.g., for parallel communication, those
  entries should not be included.
}
{
  \begin{args}[v]
  \item[v] (\id{N\_Vector}) a {\nvector} object
  \end{args}
}
{
  \id{sunindextype}
}
{}

\ucfunctionf{N\_VLinearSum}
{
  N\_VLinearSum(a, x, b, y, z);
}
{
  Performs the operation $z = a x + b y$, where $a$ and $b$ are \id{realtype}
  scalars and $x$ and $y$ are of type \id{N\_Vector}:
  $z_i = a x_i + b y_i, \: i=0,\ldots,n-1$.
}
{
  \begin{args}[a]
  \item[a] (\id{realtype}) constant that scales \id{x}
  \item[x] (\id{N\_Vector}) a {\nvector} object
  \item[b] (\id{realtype}) constant that scales \id{y}
  \item[y] (\id{N\_Vector}) a {\nvector} object
  \item[z] (\id{N\_Vector}) a {\nvector} object containing the result
  \end{args}
}
{
  The output vector \id{z} can be the same as either of the input vectors (\id{x} or \id{y}).
}
{}

\ucfunctionf{N\_VConst}
{
  N\_VConst(c, z);
}
{
  Sets all components of the \id{N\_Vector} \id{z} to \id{realtype} \id{c}:
  $z_i = c,\: i=0,\ldots,n-1$.
}
{
  \begin{args}[c]
  \item[c] (\id{realtype}) constant to set all components of \id{z} to
  \item[z] (\id{N\_Vector}) a {\nvector} object containing the result
  \end{args}
}
{}
{}

\ucfunctionf{N\_VProd}
{
  N\_VProd(x, y, z);
}
{
  Sets the \id{N\_Vector} \id{z} to be the component-wise product of the
  \id{N\_Vector} inputs \id{x} and \id{y}: $z_i = x_i y_i,\: i=0,\ldots,n-1$.
}
{
  \begin{args}[x]
  \item[x] (\id{N\_Vector}) a {\nvector} object
  \item[y] (\id{N\_Vector}) a {\nvector} object
  \item[z] (\id{N\_Vector}) a {\nvector} object containing the result
  \end{args}
}
{}
{}

\ucfunctionf{N\_VDiv}
{
  N\_VDiv(x, y, z);
}
{
  Sets the \id{N\_Vector} \id{z} to be the component-wise ratio of the
  \id{N\_Vector} inputs \id{x} and \id{y}:
  $z_i = x_i / y_i,\: i=0,\ldots,n-1$. The $y_i$ may not be tested
  for $0$ values. It should only be called with a \id{y} that is
  guaranteed to have all nonzero components.
}
{
  \begin{args}[x]
  \item[x] (\id{N\_Vector}) a {\nvector} object
  \item[y] (\id{N\_Vector}) a {\nvector} object
  \item[z] (\id{N\_Vector}) a {\nvector} object containing the result
  \end{args}
}
{}
{}

\ucfunctionf{N\_VScale}
{
  N\_VScale(c, x, z);
}
{
  Scales the \id{N\_Vector} \id{x} by the \id{realtype} scalar \id{c}
  and returns the result in \id{z}: $z_i = c x_i , \: i=0,\ldots,n-1$.
}
{
  \begin{args}[c]
  \item[c] (\id{realtype}) constant that scales the vector \id{x}
  \item[x] (\id{N\_Vector}) a {\nvector} object
  \item[z] (\id{N\_Vector}) a {\nvector} object containing the result
  \end{args}
}
{}
{}

\ucfunctionf{N\_VAbs}
{
  N\_VAbs(x, z);
}
{
  Sets the components of the \id{N\_Vector} \id{z} to be the absolute
  values of the components of the \id{N\_Vector} \id{x}:
  $z_i = | x_i | , \: i=0,\ldots,n-1$.
}
{
  \begin{args}[x]
  \item[x] (\id{N\_Vector}) a {\nvector} object
  \item[z] (\id{N\_Vector}) a {\nvector} object containing the result
  \end{args}
}
{}
{}

\ucfunctionf{N\_VInv}
{
  N\_VInv(x, z);
}
{

  Sets the components of the \id{N\_Vector} \id{z} to be the inverses
  of the components of the \id{N\_Vector} \id{x}:
  $z_i = 1.0 /  x_i  , \: i=0,\ldots,n-1$. This routine
  may not check for division by $0$. It should be called only with an
  \id{x} which is guaranteed to have all nonzero components.
}
{
  \begin{args}[x]
  \item[x] (\id{N\_Vector}) a {\nvector} object to
  \item[z] (\id{N\_Vector}) a {\nvector} object containing the result
  \end{args}
}
{}
{}

\ucfunctionf{N\_VAddConst}
{
  N\_VAddConst(x, b, z);
}
{
  Adds the \id{realtype} scalar \id{b} to all components of \id{x}
  and returns the result in the \id{N\_Vector} \id{z}:
  $z_i = x_i + b , \: i=0,\ldots,n-1$.
}
{
  \begin{args}[x]
  \item[x] (\id{N\_Vector}) a {\nvector} object
  \item[b] (\id{realtype}) constant added to all components of \id{x}
  \item[z] (\id{N\_Vector}) a {\nvector} object containing the result
  \end{args}
}
{}
{}

\ucfunctionf{N\_VDotProd}
{
  d = N\_VDotProd(x, y);
}
{
  Returns the value of the ordinary dot product of \id{x} and \id{y}:
  $d=\sum_{i=0}^{n-1} x_i y_i$.
}
{
  \begin{args}[x]
  \item[x] (\id{N\_Vector}) a {\nvector} object
    with \id{y}
  \item[y] (\id{N\_Vector}) a {\nvector} object
    with \id{x}
  \end{args}
}
{
  \id{realtype}
}
{}

\ucfunctionf{N\_VMaxNorm}
{
  m = N\_VMaxNorm(x);
}
{
  Returns the maximum norm of the \id{N\_Vector} \id{x}:
  $m = \max_{i} | x_i |$.
}
{
  \begin{args}[x]
  \item[x] (\id{N\_Vector}) a {\nvector} object
  \end{args}
}
{
  \id{realtype}
}
{}

\ucfunctionf{N\_VWrmsNorm}
{
  m = N\_VWrmsNorm(x, w)
}
{
  Returns the weighted root-mean-square norm of the \id{N\_Vector} \id{x} with
  \id{realtype} weight vector \id{w}:
  $m = \sqrt{\left( \sum_{i=0}^{n-1} (x_i w_i)^2 \right) / n}$.
}
{
  \begin{args}[x]
  \item[x] (\id{N\_Vector}) a {\nvector} object
  \item[w] (\id{N\_Vector}) a {\nvector} object containing weights
  \end{args}
}
{
  \id{realtype}
}
{}

\ucfunctionf{N\_VWrmsNormMask}
{
  m = N\_VWrmsNormMask(x, w, id);
}
{
  Returns the weighted root mean square norm of the \id{N\_Vector} \id{x} with
  \id{realtype} weight vector \id{w} built using only
  the elements of \id{x} corresponding to
  positive elements of the \id{N\_Vector} \id{id}:
  $m = \sqrt{\left( \sum_{i=0}^{n-1} (x_i w_i H(id_i))^2 \right) / n}$,
  where
  $
  H(\alpha) =
  \begin{cases}
  1 & \alpha > 0 \\
  0 & \alpha \leq 0
  \end{cases}
  $
}
{
  \begin{args}[x]
  \item[x] (\id{N\_Vector}) a {\nvector} object
  \item[w] (\id{N\_Vector}) a {\nvector} object containing weights
  \item[id] (\id{N\_Vector}) mask vector
  \end{args}
}
{
  \id{realtype}
}
{}

\ucfunctionf{N\_VMin}
{
  m = N\_VMin(x);
}
{
  Returns the smallest element of the \id{N\_Vector} \id{x}:
  $m = \min_i x_i $.
}
{
  \begin{args}[x]
  \item[x] (\id{N\_Vector}) a {\nvector} object
  \end{args}
}
{
  \id{realtype}
}
{}

\ucfunctionf{N\_VWL2Norm}
{
  m = N\_VWL2Norm(x, w);
}
{
  Returns the weighted Euclidean $\ell_2$ norm of the \id{N\_Vector} \id{x}
  with \id{realtype} weight vector \id{w}:
  $m = \sqrt{\sum_{i=0}^{n-1} (x_i w_i)^2}$.
}
{
  \begin{args}[x]
  \item[x] (\id{N\_Vector}) a {\nvector} object
  \item[w] (\id{N\_Vector}) a {\nvector} object containing weights
  \end{args}
}
{
  \id{realtype}
}
{}

\ucfunctionf{N\_VL1Norm}
{
  m = N\_VL1Norm(x);
}
{
  Returns the $\ell_1$ norm of the \id{N\_Vector} \id{x}:
  $m = \sum_{i=0}^{n-1} | x_i |$.
}
{
  \begin{args}[x]
  \item[x] (\id{N\_Vector}) a {\nvector} object to obtain the norm of
  \end{args}
}
{
  \id{realtype}
}
{}

\ucfunctionf{N\_VCompare}
{
  N\_VCompare(c, x, z);
}
{
  Compares the components of the \id{N\_Vector} \id{x} to the \id{realtype}
  scalar \id{c} and returns an \id{N\_Vector} \id{z} such that:
  $z_i = 1.0$ if $| x_i | \ge c$ and $z_i = 0.0$ otherwise.
}
{
  \begin{args}[c]
  \item[c] (\id{realtype}) constant that each component of \id{x} is compared to
  \item[x] (\id{N\_Vector}) a {\nvector} object
  \item[z] (\id{N\_Vector}) a {\nvector} object containing the result
  \end{args}
}
{}
{}

\ucfunctionf{N\_VInvTest}
{
  t = N\_VInvTest(x, z);
}
{
  Sets the components of the \id{N\_Vector} \id{z} to be the inverses
  of the components of the \id{N\_Vector} \id{x}, with prior testing
  for zero values: $z_i = 1.0 /  x_i  , \: i=0,\ldots,n-1$.
}
{
  \begin{args}[x]
  \item[x] (\id{N\_Vector}) a {\nvector} object
  \item[z] (\id{N\_Vector}) an output {\nvector} object
  \end{args}
}
{
  Returns a \id{booleantype} with value \id{SUNTRUE} if all components
  of \id{x} are nonzero (successful inversion) and returns
  \id{SUNFALSE} otherwise.
}
{}

\ucfunctionf{N\_VConstrMask}
{
  t = N\_VConstrMask(c, x, m);
}
{
  Performs the following constraint tests:
  $x_i > 0$ if $c_i=2$,
  $x_i \ge 0$ if $c_i=1$,
  $x_i \le 0$ if $c_i=-1$,
  $x_i < 0$ if $c_i=-2$.
  There is no constraint on $x_i$ if $c_i=0$. This routine returns a boolean
  assigned to \id{SUNFALSE} if any element failed the constraint test and
  assigned to \id{SUNTRUE} if all passed.  It also sets a mask vector \id{m},
  with elements equal to $1.0$ where the constraint test failed, and $0.0$
  where the test passed. This routine is used only for constraint checking.
}
{
  \begin{args}[c]
  \item[c] (\id{realtype}) scalar constraint value
  \item[x] (\id{N\_Vector}) a {\nvector} object
  \item[m] (\id{N\_Vector}) output mask vector
  \end{args}
}
{
  Returns a \id{booleantype} with value \id{SUNFALSE} if any element failed the
  constraint test, and \id{SUNTRUE} if all passed.
}
{}

\ucfunctionf{N\_VMinQuotient}
{
  minq = N\_VMinQuotient(num, denom);
}
{
  This routine returns the minimum of the quotients obtained
  by term-wise dividing \id{num}$_i$ by \id{denom}$_i$.
  A zero element in \id{denom} will be skipped.
  If no such quotients are found, then the large value
  \Id{BIG\_REAL} (defined in the header file \id{sundials\_types.h})
  is returned.
}
{
  \begin{args}[x]
  \item[num] (\id{N\_Vector}) a {\nvector} object used as the numerator
  \item[denom] (\id{N\_Vector}) a {\nvector} object used as the denominator
  \end{args}
}
{
  \id{realtype}
}
{}


% ====================================================================
\subsection{NVECTOR fused functions}\label{ss:nvecfusedops}

Fused and vector array operations are intended to increase data reuse, reduce
parallel communication on distributed memory systems, and lower the number of
kernel launches on systems with accelerators. If a particular {\nvector}
implementation defines a fused or vector array operation as \id{NULL}, the
generic {\nvector} module will automatically call standard vector operations as
necessary to complete the desired operation.  In all
{\sundials}-provided {\nvector} implementations, all fused and vector
array operations are disabled by default.  However, these
implementations provide additional user-callable functions to enable/disable
any or all of the fused and vector array operations. See the following sections
for the implementation specific functions to enable/disable operations.


\ucfunctionfl{N\_VLinearCombination}
{
  ier = N\_VLinearCombination(nv, c, X, z);
}
{
  This routine computes the linear combination of $n_v$ vectors with $n$
  elements:
  \begin{equation*}
    z_i = \sum_{j=0}^{n_v-1} c_j x_{j,i}, \quad i=0,\ldots,n-1,
  \end{equation*}
  where $c$ is an array of $n_v$ scalars, $X$ is an array of $n_v$ vectors,
  and $z$ is the output vector.
}
{
  \begin{args}[nv]
  \item[nv] (\id{int}) the number of vectors in the linear combination
  \item[c] (\id{realtype*}) an array of $n_v$ scalars used to scale
    the corresponding vector in \id{X}
  \item[X] (\id{N\_Vector*}) an array of $n_v$ {\nvector} objects
    to be scaled and combined
  \item[z] (\id{N\_Vector}) a {\nvector} object containing the result
  \end{args}
}
{
  Returns an \id{int} with value \id{0} for success and a non-zero value otherwise.
}
{
  If the output vector $z$ is one of the vectors in $X$, then it \textit{must} be
  the first vector in the vector array.
}
{
  real(c\_double) :: c(nv)\\
  type(c\_ptr), target :: X(nv)\\
  type(N\_Vector), pointer :: z\\
  ierr = FN\_VLinearCombination(nv, c, X, z)
}

\ucfunctionfl{N\_VScaleAddMulti}
{
  ier = N\_VScaleAddMulti(nv, c, x, Y, Z);
}
{
  This routine scales and adds one vector to $n_v$ vectors with $n$ elements:
  \begin{equation*}
    z_{j,i} = c_j x_i + y_{j,i}, \quad j=0,\ldots,n_v-1 \quad i=0,\ldots,n-1,
  \end{equation*}
  where $c$ is an array of $n_v$ scalars, $x$ is the vector to be scaled and
  added to each vector in the vector array of $n_v$ vectors $Y$, and $Z$ is a
  vector array of $n_v$ output vectors.
}
{
  \begin{args}[nv]
  \item[nv] (\id{int}) the number of scalars and vectors in \id{c}, \id{Y}, and \id{Z}
  \item[c] (\id{realtype*}) an array of $n_v$ scalars
  \item[x] (\id{N\_Vector}) a {\nvector} object to be scaled and added to each
    vector in \id{Y}
  \item[Y] (\id{N\_Vector*}) an array of $n_v$ {\nvector} objects where each vector
    $j$ will have the vector \id{x} scaled by \id{c\_j} added to it
  \item[Z] (\id{N\_Vector}) an output array of $n_v$ {\nvector} objects
  \end{args}
}
{
  Returns an \id{int} with value \id{0} for success and a non-zero value otherwise.
}
{}
{
  real(c\_double) :: c(nv)\\
  type(c\_ptr), target :: Y(nv), Z(nv)\\
  type(N\_Vector), pointer :: x\\
  ierr = FN\_VScaleAddMulti(nv, c, x, Y, Z)
}

\ucfunctionfl{N\_VDotProdMulti}
{
  ier = N\_VDotProdMulti(nv, x, Y, d);
}
{
  This routine computes the dot product of a vector with $n_v$ other vectors:
  \begin{equation*}
    d_j = \sum_{i=0}^{n-1} x_i y_{j,i}, \quad j=0,\ldots,n_v-1,
  \end{equation*}
  where $d$ is an array of $n_v$ scalars containing the dot products of the
  vector $x$ with each of the $n_v$ vectors in the vector array $Y$.
}
{
  \begin{args}[nv]
  \item[nv] (\id{int}) the number of vectors in \id{Y}
  \item[x] (\id{N\_Vector}) a {\nvector} object to be used in a dot product
    with each of the vectors in \id{Y}
  \item[Y] (\id{N\_Vector*}) an array of $n_v$ {\nvector} objects to use
    in a dot product with \id{x}
  \item[d] (\id{realtype*}) an output array of $n_v$ dot products
  \end{args}
}
{
  Returns an \id{int} with value \id{0} for success and a non-zero value otherwise.
}
{}
{
  real(c\_double) :: d(nv)\\
  type(c\_ptr), target :: Y(nv)\\
  type(N\_Vector), pointer :: x\\
  ierr = FN\_VDotProdMulti(nv, x, Y, d)
}


% ====================================================================
\subsection{NVECTOR vector array functions}\label{ss:nvecarrayops}


\ucfunctionf{N\_VLinearSumVectorArray}
{
  ier = N\_VLinearSumVectorArray(nv, a, X, b, Y, Z);
}
{
  This routine computes the linear sum of two vector arrays containing $n_v$
  vectors of $n$ elements:
  \begin{equation*}
    z_{j,i} = a x_{j,i} + b y_{j,i}, \quad i=0,\ldots,n-1 \quad j=0,\ldots,n_v-1,
  \end{equation*}
  where $a$ and $b$ are scalars and $X$, $Y$, and $Z$ are arrays of $n_v$ vectors.
}
{
  \begin{args}[nv]
  \item[nv] (\id{int}) the number of vectors in the vector arrays
  \item[a] (\id{realtype}) constant to scale each vector in \id{X} by
  \item[X] (\id{N\_Vector*}) an array of $n_v$ {\nvector} objects
  \item[Y] (\id{N\_Vector*}) an array of $n_v$ {\nvector} objects
  \item[Z] (\id{N\_Vector*}) an output array of $n_v$ {\nvector} objects
  \end{args}
}
{
  Returns an \id{int} with value \id{0} for success and a non-zero value otherwise.
}
{}

\ucfunctionf{N\_VScaleVectorArray}
{
  ier = N\_VScaleVectorArray(nv, c, X, Z);
}
{
  This routine scales each vector of $n$ elements in a vector array of $n_v$
  vectors by a potentially different constant:
  \begin{equation*}
    z_{j,i} = c_j x_{j,i}, \quad i=0,\ldots,n-1 \quad j=0,\ldots,n_v-1,
  \end{equation*}
  where $c$ is an array of $n_v$ scalars and $X$ and $Z$ are arrays of $n_v$
  vectors.
}
{
  \begin{args}[nv]
  \item[nv] (\id{int}) the number of vectors in the vector arrays
  \item[c] (\id{realtype}) constant to scale each vector in \id{X} by
  \item[X] (\id{N\_Vector*}) an array of $n_v$ {\nvector} objects
  \item[Z] (\id{N\_Vector*}) an output array of $n_v$ {\nvector} objects
  \end{args}
}
{
  Returns an \id{int} with value \id{0} for success and a non-zero value otherwise.
}
{}

\ucfunctionf{N\_VConstVectorArray}
{
  ier = N\_VConstVectorArray(nv, c, X);
}
{
  This routine sets each element in a vector of $n$ elements in a vector array of
  $n_v$ vectors to the same value:
  \begin{equation*}
    z_{j,i} = c, \quad i=0,\ldots,n-1 \quad j=0,\ldots,n_v-1,
  \end{equation*}
  where $c$ is a scalar and $X$ is an array of $n_v$ vectors.
}
{
  \begin{args}[nv]
  \item[nv] (\id{int}) the number of vectors in \id{X}
  \item[c] (\id{realtype}) constant to set every element in every
    vector of \id{X} to
  \item[X] (\id{N\_Vector*}) an array of $n_v$ {\nvector} objects
  \end{args}
}
{
  Returns an \id{int} with value \id{0} for success and a non-zero value otherwise.
}
{}

\ucfunctionf{N\_VWrmsNormVectorArray}
{
  ier = N\_VWrmsNormVectorArray(nv, X, W, m);
}
{
  This routine computes the weighted root mean square norm of $n_v$ vectors with
  $n$ elements:
  \begin{equation*}
    m_j = \left( \frac1n \sum_{i=0}^{n-1} \left(x_{j,i} w_{j,i}\right)^2\right)^{1/2}, \quad j=0,\ldots,n_v-1,
  \end{equation*}
  where $m$ contains the $n_v$ norms of the vectors in the vector array $X$ with
  corresponding weight vectors $W$.
}
{
  \begin{args}[nv]
  \item[nv] (\id{int}) the number of vectors in the vector arrays
  \item[X] (\id{N\_Vector*}) an array of $n_v$ {\nvector} objects
  \item[W] (\id{N\_Vector*}) an array of $n_v$ {\nvector} objects
  \item[m] (\id{realtype*}) an output array of $n_v$ norms
  \end{args}
}
{
  Returns an \id{int} with value \id{0} for success and a non-zero value otherwise.
}
{}

\ucfunctionf{N\_VWrmsNormMaskVectorArray}
{
  ier = N\_VWrmsNormMaskVectorArray(nv, X, W, id, m);
}
{
  This routine computes the masked weighted root mean square norm of $n_v$
  vectors with $n$ elements:
  \begin{equation*}
    m_j = \left( \frac1n \sum_{i=0}^{n-1} \left(x_{j,i} w_{j,i}
    H(id_i)\right)^2 \right)^{1/2}, \quad j=0,\ldots,n_v-1,
  \end{equation*}
  $H(id_i)=1$ for $id_i > 0$ and is zero otherwise, $m$ contains the $n_v$
  norms of the vectors in the vector array $X$ with corresponding weight
  vectors $W$ and mask vector $id$.
}
{
  \begin{args}[nv]
  \item[nv] (\id{int}) the number of vectors in the vector arrays
  \item[X] (\id{N\_Vector*}) an array of $n_v$ {\nvector} objects
  \item[W] (\id{N\_Vector*}) an array of $n_v$ {\nvector} objects
  \item[id] (\id{N\_Vector}) the mask vector
  \item[m] (\id{realtype*}) an output array of $n_v$ norms
  \end{args}
}
{
  Returns an \id{int} with value \id{0} for success and a non-zero value otherwise.
}
{}

\ucfunction{N\_VScaleAddMultiVectorArray}
{
  ier = N\_VScaleAddMultiVectorArray(nv, ns, c, X, YY, ZZ);
}
{
  This routine scales and adds a vector in a vector array of $n_v$ vectors to
  the corresponding vector in $n_s$ vector arrays:
  \begin{equation*}
    z_{k,j,i} = c_k x_{j,i} + y_{k,j,i}, \quad i=0,\ldots,n-1 \quad j=0,\ldots,nv-1, \quad k=0,\ldots,ns-1
  \end{equation*}
  where $c$ is an array of $n_s$ scalars, $X$ is a vector array of $n_v$ vectors
  to be scaled and added to the corresponding vector in each of the $n_s$ vector
  arrays in the array of vector arrays $YY$ and stored in the output array of vector
  arrays $ZZ$.
}
{
  \begin{args}[nv]
  \item[nv] (\id{int}) the number of vectors in the vector arrays
  \item[ns] (\id{int}) the number of scalars in \id{c} and vector arrays
    in \id{YY} and \id{ZZ}
  \item[c] (\id{realtype*}) an array of $n_s$ scalars
  \item[X] (\id{N\_Vector*}) an array of $n_v$ {\nvector} objects
  \item[YY] (\id{N\_Vector**}) an array of $n_s$ {\nvector} arrays
  \item[ZZ] (\id{N\_Vector**}) an output array of $n_s$ {\nvector}
    arrays
  \end{args}
}
{
  Returns an \id{int} with value \id{0} for success and a non-zero value otherwise.
}
{}

\ucfunction{N\_VLinearCombinationVectorArray}
{
  ier = N\_VLinearCombinationVectorArray(nv, ns, c, XX, Z);
}
{
  This routine computes the linear combination of $n_s$ vector arrays containing
  $n_v$ vectors with $n$ elements:
  \begin{equation*}
  z_{j,i} = \sum_{k=0}^{n_s-1} c_k x_{k,j,i}, \quad i=0,\ldots,n-1 \quad j=0,\ldots,n_v-1,
  \end{equation*}
  where $c$ is an array of $n_s$ scalars (type \id{realtype*}), $XX$
  (type \id{N\_Vector**}) is an array of $n_s$ vector arrays each containing $n_v$
  vectors to be summed into the output vector array of $n_v$ vectors $Z$ (type
  \id{N\_Vector*}). If the output vector array $Z$ is one of the vector arrays in
  $XX$, then it \textit{must} be the first vector array in $XX$.
}
{
 \begin{args}[nv]
  \item[nv] (\id{int}) the number of vectors in the vector arrays
  \item[ns] (\id{int}) the number of scalars in \id{c} and vector arrays
    in \id{YY} and \id{ZZ}
  \item[c] (\id{realtype*}) an array of $n_s$ scalars
  \item[XX] (\id{N\_Vector**}) an array of $n_s$ {\nvector} arrays
  \item[Z] (\id{N\_Vector*}) an output array {\nvector} objects
  \end{args}
}
{
  Returns an \id{int} with value \id{0} for success and a non-zero value otherwise.
}
{}

% ====================================================================
\subsection{NVECTOR local reduction functions}\label{ss:nveclocalops}

Local reduction operations are intended to reduce parallel
communication on distributed memory systems, particularly when
{\nvector} objects are combined together within a \\
{\nvecmpimanyvector} object (see Section \ref{ss:nvec_mpimanyvector}).  If a
particular {\nvector} implementation defines a local reduction
operation as \id{NULL}, the {\nvecmpimanyvector} module will
automatically call standard vector reduction operations as necessary
to complete the desired operation. All {\sundials}-provided {\nvector}
implementations include these local reduction operations, which may be
used as templates for user-defined {\nvector} implementations.


\ucfunctionf{N\_VDotProdLocal}
{
  d = N\_VDotProdLocal(x, y);
}
{
  This routine computes the MPI task-local portion of the ordinary dot product of \id{x} and \id{y}:
  \begin{equation*}
    d=\sum_{i=0}^{n_{local}-1} x_i y_i,
  \end{equation*}
  where $n_{local}$ corresponds
  to the number of components in the vector on this MPI task (or
  $n_{local}=n$ for MPI-unaware applications).
}
{
  \begin{args}[x]
  \item[x] (\id{N\_Vector}) a {\nvector} object
  \item[y] (\id{N\_Vector}) a {\nvector} object
  \end{args}
}
{
  \id{realtype}
}
{}

\ucfunctionf{N\_VMaxNormLocal}
{
  m = N\_VMaxNormLocal(x);
}
{
  This routine computes the MPI task-local portion of the maximum norm of the \id{N\_Vector} \id{x}:
  \begin{equation*}
    m = \max_{0\le i< n_{local}} | x_i |,
  \end{equation*}
  where $n_{local}$ corresponds
  to the number of components in the vector on this MPI task (or
  $n_{local}=n$ for MPI-unaware applications).
}
{
  \begin{args}[x]
  \item[x] (\id{N\_Vector}) a {\nvector} object
  \end{args}
}
{
  \id{realtype}
}
{}

\ucfunctionf{N\_VMinLocal}
{
  m = N\_VMinLocal(x);
}
{
  This routine computes the smallest element of the MPI task-local portion of
  the \id{N\_Vector} \id{x}:
  \begin{equation*}
    m = \min_{0\le i< n_{local}} x_i,
  \end{equation*}
  where $n_{local}$ corresponds
  to the number of components in the vector on this MPI task (or
  $n_{local}=n$ for MPI-unaware applications).
}
{
  \begin{args}[x]
  \item[x] (\id{N\_Vector}) a {\nvector} object
  \end{args}
}
{
  \id{realtype}
}
{}

\ucfunctionf{N\_VL1NormLocal}
{
  n = N\_VL1NormLocal(x);
}
{
  This routine computes the MPI task-local portion of the $\ell_1$ norm of the \id{N\_Vector} \id{x}:
  \begin{equation*}
  n = \sum_{i=0}^{n_{local}-1} | x_i |,
  \end{equation*}
  where $n_{local}$ corresponds
  to the number of components in the vector on this MPI task (or
  $n_{local}=n$ for MPI-unaware applications).
}
{
  \begin{args}[x]
  \item[x] (\id{N\_Vector}) a {\nvector} object
  \end{args}
}
{
  \id{realtype}
}
{}

\ucfunctionf{N\_VWSqrSumLocal}
{
  s = N\_VWSqrSumLocal(x,w);
}
{
  This routine computes the MPI task-local portion of the weighted
  squared sum of the \id{N\_Vector} \id{x} with weight vector \id{w}:
  \begin{equation*}
    s = \sum_{i=0}^{n_{local}-1} (x_i w_i)^2,
  \end{equation*}
  where $n_{local}$ corresponds
  to the number of components in the vector on this MPI task (or
  $n_{local}=n$ for MPI-unaware applications).
}
{
  \begin{args}[x]
  \item[x] (\id{N\_Vector}) a {\nvector} object
  \item[w] (\id{N\_Vector}) a {\nvector} object containing weights
  \end{args}
}
{
  \id{realtype}
}
{}

\ucfunctionf{N\_VWSqrSumMaskLocal}
{
  s = N\_VWSqrSumMaskLocal(x,w,id);
}
{
  This routine computes the MPI task-local portion of the weighted
  squared sum of the \id{N\_Vector} \id{x} with weight
  vector \id{w} built using only the elements of \id{x} corresponding to
  positive elements of the \id{N\_Vector} \id{id}:
  \begin{equation*}
    m = \sum_{i=0}^{n_{local}-1} (x_i w_i H(id_i))^2, \quad \text{where} \quad H(\alpha)
  = \begin{cases} 1 & \alpha > 0 \\ 0 & \alpha \leq 0 \end{cases}
  \end{equation*}
  and
  $n_{local}$ corresponds to the number of components in the vector on
  this MPI task (or $n_{local}=n$ for MPI-unaware applications).
}
{
  \begin{args}[x]
  \item[x] (\id{N\_Vector}) a {\nvector} object
  \item[w] (\id{N\_Vector}) a {\nvector} object containing weights
  \item[id] (\id{N\_Vector}) a {\nvector} object used as a mask
  \end{args}
}
{
  \id{realtype}
}
{}

\ucfunctionf{N\_VInvTestLocal}
{
  t = N\_VInvTestLocal(x, z);
}
{
  Sets the MPI task-local components of the \id{N\_Vector} \id{z} to
  be the inverses of the components of the \id{N\_Vector} \id{x}, with
  prior testing for zero values:
  \begin{equation*}
  z_i = 1.0 /  x_i  , \: i=0,\ldots,n_{local}-1,
  \end{equation*}
  where $n_{local}$
  corresponds to the number of components in the vector on this MPI task
  (or $n_{local}=n$ for MPI-unaware applications).
}
{
  \begin{args}[x]
  \item[x] (\id{N\_Vector}) a {\nvector} object
  \item[z] (\id{N\_Vector}) an output {\nvector} object
  \end{args}
}
{
  Returns a \id{booleantype} with the value \id{SUNTRUE} if all task-local
  components of \id{x} are nonzero (successful inversion) and with the
  value \id{SUNFALSE} otherwise.
}
{}

\ucfunctionf{N\_VConstrMaskLocal}
{
  t = N\_VConstrMaskLocal(c,x,m);
}
{
  Performs the following constraint tests:
  {\begin{align*}
  x_i > 0        & \quad \text{if} \quad c_i=2, \\
  x_i \ge 0      & \quad \text{if} \quad c_i=1, \\
  x_i \le 0      & \quad \text{if} \quad c_i=-1, \\
  x_i < 0        & \quad \text{if} \quad c_i=-2, \text{and} \\
  \text{no test} & \quad \text{if} \quad c_i=0,
  \end{align*}}%
  for all MPI task-local components of the vectors.
  It sets a mask vector \id{m}, with elements equal to $1.0$ where
  the constraint test failed, and $0.0$ where the test passed. This
  routine is used only for constraint checking.
}
{
  \begin{args}[c]
  \item[c] (\id{realtype}) scalar constraint value
  \item[x] (\id{N\_Vector}) a {\nvector} object
  \item[m] (\id{N\_Vector}) output mask vector
  \end{args}
}
{
  Returns a \id{booleantype} with the value \id{SUNFALSE} if any
  task-local element failed the constraint test and the value
  \id{SUNTRUE} if all passed.
}
{}

\ucfunctionf{N\_VMinQuotientLocal}
{
  minq = N\_VMinQuotientLocal(num,denom);
}
{
  This routine returns the minimum of the quotients obtained
  by term-wise dividing \id{num}$_i$ by \id{denom}$_i$, for all MPI
  task-local components of the vectors.  A zero element in \id{denom}
  will be skipped. If no such quotients are found, then the large value
  \Id{BIG\_REAL} (defined in the header file \id{sundials\_types.h})
  is returned.
}
{
  \begin{args}[denom]
  \item[num] (\id{N\_Vector}) a {\nvector} object used as the numerator
  \item[denom] (\id{N\_Vector}) a {\nvector} object used as the denominator
  \end{args}
}
{
  \id{realtype}
}
{}


% ====================================================================
\subsection{NVECTOR exchange operations}\label{ss:nvecexchangeops}

The following vector exchange operations are also \textit{optional} and are
intended only for use when interfacing with the XBraid library for
parallel-in-time integration. In that setting these operations are required but
are otherwise unused by SUNDIALS packages and may be set to \id{NULL}. For each
operation, we give the function signature, a description of the expected
behavior, and an example of the function usage.

\ucfunctionf{N\_VBufSize}
{
  flag = N\_VBufSize(N\_Vector x, sunindextype *size);
}
{
  This routine returns the buffer size need to exchange in the data in the
  vector \id{x} between computational nodes.
}
{
  \begin{args}[size]
  \item[x] (\id{N\_Vector}) a {\nvector} object
  \item[size] (\id{sunindextype*}) the size of the message buffer
  \end{args}
}
{
  Returns an \id{int} with value \id{0} for success and a non-zero value otherwise.
}
{}

\ucfunctionf{N\_VBufPack}
{
  flag = N\_VBufPack(N\_Vector x, void *buf);
}
{
  This routine fills the exchange buffer \id{buf} with the vector data in \id{x}.
}
{
  \begin{args}[buf]
  \item[x] (\id{N\_Vector}) a {\nvector} object
  \item[buf] (\id{sunindextype*}) the exchange buffer to pack
  \end{args}
}
{
  Returns an \id{int} with value \id{0} for success and a non-zero value otherwise.
}
{}

\ucfunctionf{N\_VBufUnpack}
{
  flag = N\_VBufUnpack(N\_Vector x, void *buf);
}
{
  This routine unpacks the data in the exchange buffer \id{buf} into the vector
  \id{x}.
}
{
  \begin{args}[buf]
  \item[x] (\id{N\_Vector}) a {\nvector} object
  \item[buf] (\id{sunindextype*}) the exchange buffer to unpack
  \end{args}
}
{
  Returns an \id{int} with value \id{0} for success and a non-zero value otherwise.
}
{}



% ====================================================================
\subsection{NVECTOR utility functions}\label{ss:nvecutils}

To aid in the creation of custom {\nvector} modules the generic {\nvector}
module provides three  utility functions \id{N\_VNewEmpty}, \id{N\_VCopyOps}
and \id{N\_VFreeEmpty}. When used in custom {\nvector} constructors and clone
routines these functions will ease the introduction of any new optional vector
operations to the {\nvector} API by ensuring only required operations need to
be set and all operations are copied when cloning a vector.

To aid the use of arrays of {\nvector} objects, the generic {\nvector} module
also provides the utility functions \ID{N\_VCloneVectorArray},
\ID{N\_VCloneVectorArrayEmpty}, and \ID{N\_VDestroyVectorArray}.


\ucfunctionf{N\_VNewEmpty}
{
  v = N\_VNewEmpty();
}
{
  The function \Id{N\_VNewEmpty} allocates a new generic {\nvector} object and
  initializes its content pointer and the function pointers in the operations
  structure to \id{NULL}.
}
{}
{
  This function returns an \id{N\_Vector} object. If an error occurs when
  allocating the object, then this routine will return \id{NULL}.
}
{}
{}

\ucfunctionf{N\_VCopyOps}
{
  retval = N\_VCopyOps(w, v);
}
{
  The function \Id{N\_VCopyOps} copies the function pointers in the \id{ops}
  structure of \id{w} into the \id{ops} structure of \id{v}.
}
{
  \begin{args}[w]
  \item[w] (\id{N\_Vector}) the vector to copy operations from
  \item[v] (\id{N\_Vector}) the vector to copy operations to
  \end{args}
}
{
  This returns \id{0} if successful and a non-zero value if either of the inputs
  are \id{NULL} or the \id{ops} structure of either input is \id{NULL}.
}
{}

\ucfunctionf{N\_VFreeEmpty}
{
  N\_VFreeEmpty(v);
}
{
  This routine frees the generic \id{N\_Vector} object, under the assumption that any
  implementation-specific data that was allocated within the underlying content structure
  has already been freed. It will additionally test whether the ops pointer is \id{NULL},
  and, if it is not, it will free it as well.
}
{
  \begin{args}[v]
  \item[v] (\id{N\_Vector})
  \end{args}
}
{}
{}

\ucfunction{N\_VCloneEmptyVectorArray}
{
  vecarray = N\_VCloneEmptyVectorArray(count, w);
}
{
  Creates an array of \id{count} variables of type \id{N\_Vector},
  each of the same type as the existing \id{N\_Vector} w. It achieves
  this by calling the implementation-specific \id{N\_VCloneEmpty} operation.
}
{
  \begin{args}[count]
  \item[count] (\id{int}) the size of the vector array
  \item[w] (\id{N\_Vector}) the vector to clone
  \end{args}
}
{
  Returns an array of \id{count} \id{N\_Vector} objects if successful, or
  \id{NULL} if an error occurred while cloning.
}
{}

\ucfunction{N\_VCloneVectorArray}
{
  vecarray = N\_VCloneVectorArray(count, w);
}
{
  Creates an array of \id{count} variables of type \id{N\_Vector},
  each of the same type as the existing \id{N\_Vector} w. It achieves
  this by calling the implementation-specific \id{N\_VClone} operation.
}
{
  \begin{args}[count]
  \item[count] (\id{int}) the size of the vector array
  \item[w] (\id{N\_Vector}) the vector to clone
  \end{args}
}
{
  Returns an array of \id{count} \id{N\_Vector} objects if successful, or
  \id{NULL} if an error occurred while cloning.
}
{}

\ucfunction{N\_VDestroyVectorArray}
{
   N\_VDestroyVectorArray(count, w);
}
{
  Destroys (frees) an array of variables of type \id{N\_Vector}. It
  depends on the implementation-specific \id{N\_VDestroy} operation.
}
{
  \begin{args}[count]
  \item[vs] (\id{N\_Vector*}) the array of vectors to destroy
  \item[count] (\id{int}) the size of the vector array
  \end{args}
}
{}
{}

\ucfunctionf{N\_VNewVectorArray}
{
  vecarray = N\_VNewVectorArray(count);
}
{
  Returns an empty \id{N\_Vector} array large enough to hold \id{count}
  \id{N\_Vector} objects. This function is primarily meant for users of
  the Fortran 2003 interface.
}
{
  \begin{args}[count]
  \item[count] (\id{int}) the size of the vector array
  \end{args}
}
{
  Returns a \id{N\_Vector*} if successful, Returns \id{NULL} if an error occurred.
}
{
  Users of the Fortran 2003 interface to the \id{N\_VManyVector} or
  \id{N\_VMPIManyVector} will need this to create an array to hold
  the subvectors. Note that this function does restrict the the max
  number of subvectors usable with the \id{N\_VManyVector} and
  \id{N\_VMPIManyVector} to the max size of an \id{int} despite the
  ManyVector implementations accepting a subvector count larger than
  this value.
}

\ucfunctionf{N\_VGetVecAtIndexVectorArray}
{
  v = N\_VGetVecAtIndexVectorArray(vecs, index);
}
{
  Returns the \id{N\_Vector} object stored in the vector array at the
  provided index. This function is primarily meant for users of the
  Fortran 2003 interface.
}
{
  \begin{args}[count]
  \item[vecs] (\id{N\_Vector}*) the array of vectors to index
  \item[index] (\id{int}) the index of the vector to return
  \end{args}
}
{
  Returns the \id{N\_Vector} object stored in the vector array at the
  provided index. Returns \id{NULL} if an error occurred.
}
{}

\ucfunctionf{N\_VSetVecAtIndexVectorArray}
{
  N\_VSetVecAtIndexVectorArray(vecs, index, v);
}
{
  Sets the \id{N\_Vector} object stored in the vector array at the
  provided index. This function is primarily meant for users of the
  Fortran 2003 interface.
}
{
  \begin{args}[count]
  \item[vecs] (\id{N\_Vector}*) the array of vectors to index
  \item[index] (\id{int}) the index of the vector to return
  \item[v] (\id{N\_Vector}) the vector to store at the index
  \end{args}
}
{}
{}


% ====================================================================
\subsection{NVECTOR identifiers}
\label{ss:nvecIDs}

Each {\nvector} implementation included in {\sundials} has a
unique identifier specified in enumeration and shown in Table \ref{t:vectorIDs}.

\begin{table}
\centering
\caption{Vector Identifications associated with vector kernels supplied with \id{\sundials}.}
\label{t:vectorIDs}
\medskip
\begin{tabular}{|l|l|c|}
\hline
{\bf Vector ID} & {\bf Vector type} & {\bf ID Value} \\
\hline
SUNDIALS\_NVEC\_SERIAL        & Serial                                        & 0 \\
SUNDIALS\_NVEC\_PARALLEL      & Distributed memory parallel (MPI)             & 1 \\
SUNDIALS\_NVEC\_OPENMP        & OpenMP shared memory parallel                 & 2 \\
SUNDIALS\_NVEC\_PTHREADS      & PThreads shared memory parallel               & 3 \\
SUNDIALS\_NVEC\_PARHYP        & {\hypre} ParHyp parallel vector               & 4 \\
SUNDIALS\_NVEC\_PETSC         & {\petsc} parallel vector                      & 5 \\
SUNDIALS\_NVEC\_CUDA          & {\cuda} parallel vector                       & 6 \\
SUNDIALS\_NVEC\_RAJA          & {\raja} parallel vector                       & 7 \\
SUNDIALS\_NVEC\_OPENMPDEV     & OpenMP parallel vector with device offloading & 8 \\
SUNDIALS\_NVEC\_TRILINOS      & {\trilinos} Tpetra vector                     & 9 \\
SUNDIALS\_NVEC\_MANYVECTOR    & ``ManyVector'' vector                         & 10 \\
SUNDIALS\_NVEC\_MPIMANYVECTOR & MPI-enabled ``ManyVector'' vector             & 11 \\
SUNDIALS\_NVEC\_MPIPLUSX      & MPI+X vector                                  & 12 \\
SUNDIALS\_NVEC\_CUSTOM        & User-provided custom vector                   & 13 \\
\hline
\end{tabular}
\end{table}


% ====================================================================
\subsection{The generic NVECTOR module implementation}
\label{ss:nvec_impl_details}

The generic \ID{N\_Vector} type is a pointer to a structure that has an
implementation-dependent {\em content} field containing the
description and actual data of the vector, and an {\em ops} field
pointing to a structure with generic vector operations.
The type \id{N\_Vector} is defined as
%%
%%
\begin{verbatim}
typedef struct _generic_N_Vector *N_Vector;

struct _generic_N_Vector {
    void *content;
    struct _generic_N_Vector_Ops *ops;
};
\end{verbatim}
%%
%%
The \id{\_generic\_N\_Vector\_Ops} structure is essentially a list of pointers to
the various actual vector operations, and is defined as
%%
\begin{verbatim}
struct _generic_N_Vector_Ops {
  N_Vector_ID  (*nvgetvectorid)(N_Vector);
  N_Vector     (*nvclone)(N_Vector);
  N_Vector     (*nvcloneempty)(N_Vector);
  void         (*nvdestroy)(N_Vector);
  void         (*nvspace)(N_Vector, sunindextype *, sunindextype *);
  realtype*    (*nvgetarraypointer)(N_Vector);
  void         (*nvsetarraypointer)(realtype *, N_Vector);
  void*        (*nvgetcommunicator)(N_Vector);
  sunindextype (*nvgetlength)(N_Vector);
  void         (*nvlinearsum)(realtype, N_Vector, realtype, N_Vector, N_Vector);
  void         (*nvconst)(realtype, N_Vector);
  void         (*nvprod)(N_Vector, N_Vector, N_Vector);
  void         (*nvdiv)(N_Vector, N_Vector, N_Vector);
  void         (*nvscale)(realtype, N_Vector, N_Vector);
  void         (*nvabs)(N_Vector, N_Vector);
  void         (*nvinv)(N_Vector, N_Vector);
  void         (*nvaddconst)(N_Vector, realtype, N_Vector);
  realtype     (*nvdotprod)(N_Vector, N_Vector);
  realtype     (*nvmaxnorm)(N_Vector);
  realtype     (*nvwrmsnorm)(N_Vector, N_Vector);
  realtype     (*nvwrmsnormmask)(N_Vector, N_Vector, N_Vector);
  realtype     (*nvmin)(N_Vector);
  realtype     (*nvwl2norm)(N_Vector, N_Vector);
  realtype     (*nvl1norm)(N_Vector);
  void         (*nvcompare)(realtype, N_Vector, N_Vector);
  booleantype  (*nvinvtest)(N_Vector, N_Vector);
  booleantype  (*nvconstrmask)(N_Vector, N_Vector, N_Vector);
  realtype     (*nvminquotient)(N_Vector, N_Vector);
  int          (*nvlinearcombination)(int, realtype*, N_Vector*, N_Vector);
  int          (*nvscaleaddmulti)(int, realtype*, N_Vector, N_Vector*, N_Vector*);
  int          (*nvdotprodmulti)(int, N_Vector, N_Vector*, realtype*);
  int          (*nvlinearsumvectorarray)(int, realtype, N_Vector*, realtype,
                                         N_Vector*, N_Vector*);
  int          (*nvscalevectorarray)(int, realtype*, N_Vector*, N_Vector*);
  int          (*nvconstvectorarray)(int, realtype, N_Vector*);
  int          (*nvwrmsnomrvectorarray)(int, N_Vector*, N_Vector*, realtype*);
  int          (*nvwrmsnomrmaskvectorarray)(int, N_Vector*, N_Vector*, N_Vector,
                                            realtype*);
  int          (*nvscaleaddmultivectorarray)(int, int, realtype*, N_Vector*,
                                             N_Vector**, N_Vector**);
  int          (*nvlinearcombinationvectorarray)(int, int, realtype*, N_Vector**,
                                                 N_Vector*);
  realtype     (*nvdotprodlocal)(N_Vector, N_Vector);
  realtype     (*nvmaxnormlocal)(N_Vector);
  realtype     (*nvminlocal)(N_Vector);
  realtype     (*nvl1normlocal)(N_Vector);
  booleantype  (*nvinvtestlocal)(N_Vector, N_Vector);
  booleantype  (*nvconstrmasklocal)(N_Vector, N_Vector, N_Vector);
  realtype     (*nvminquotientlocal)(N_Vector, N_Vector);
  realtype     (*nvwsqrsumlocal)(N_Vector, N_Vector);
  realtype     (*nvwsqrsummasklocal(N_Vector, N_Vector, N_Vector);
  int          (*nvbufsize)(N_Vector, sunindextype *);
  int          (*nvbufpack)(N_Vector, void*);
  int          (*nvbufunpack)(N_Vector, void*);
};
\end{verbatim}

The generic {\nvector} module defines and implements the vector operations
acting on an \id{N\_Vector}. These routines are nothing but wrappers for
the vector operations defined by a particular {\nvector} implementation,
which are accessed through the {\em ops} field of the \id{N\_Vector}
structure. To illustrate this point we show below the implementation of a
typical vector operation from the generic {\nvector} module, namely \id{N\_VScale},
which performs the scaling of a vector \id{x} by a scalar \id{c}:
%%
%%
\begin{verbatim}
void N_VScale(realtype c, N_Vector x, N_Vector z)
{
   z->ops->nvscale(c, x, z);
}
\end{verbatim}
%%
%%
Section \ref{ss:nvecops} defines a complete list of all standard vector operations
defined by the generic {\nvector} module. Sections \ref{ss:nvecfusedops},
\ref{ss:nvecarrayops} and \ref{ss:nveclocalops} list \textit{optional} fused,
vector array and local reduction operations, respectively.


The Fortran 2003 interface provides a \id{bind(C)} derived-type for the
\id{\_generic\_N\_Vector} and the \id{\_generic\_N\_Vector\_Ops} structures.
Their definition is given below.
%%
%%
\begin{verbatim}
 type, bind(C), public :: N_Vector
  type(C_PTR), public :: content
  type(C_PTR), public :: ops
 end type N_Vector

 type, bind(C), public :: N_Vector_Ops
  type(C_FUNPTR), public :: nvgetvectorid
  type(C_FUNPTR), public :: nvclone
  type(C_FUNPTR), public :: nvcloneempty
  type(C_FUNPTR), public :: nvdestroy
  type(C_FUNPTR), public :: nvspace
  type(C_FUNPTR), public :: nvgetarraypointer
  type(C_FUNPTR), public :: nvsetarraypointer
  type(C_FUNPTR), public :: nvgetcommunicator
  type(C_FUNPTR), public :: nvgetlength
  type(C_FUNPTR), public :: nvlinearsum
  type(C_FUNPTR), public :: nvconst
  type(C_FUNPTR), public :: nvprod
  type(C_FUNPTR), public :: nvdiv
  type(C_FUNPTR), public :: nvscale
  type(C_FUNPTR), public :: nvabs
  type(C_FUNPTR), public :: nvinv
  type(C_FUNPTR), public :: nvaddconst
  type(C_FUNPTR), public :: nvdotprod
  type(C_FUNPTR), public :: nvmaxnorm
  type(C_FUNPTR), public :: nvwrmsnorm
  type(C_FUNPTR), public :: nvwrmsnormmask
  type(C_FUNPTR), public :: nvmin
  type(C_FUNPTR), public :: nvwl2norm
  type(C_FUNPTR), public :: nvl1norm
  type(C_FUNPTR), public :: nvcompare
  type(C_FUNPTR), public :: nvinvtest
  type(C_FUNPTR), public :: nvconstrmask
  type(C_FUNPTR), public :: nvminquotient
  type(C_FUNPTR), public :: nvlinearcombination
  type(C_FUNPTR), public :: nvscaleaddmulti
  type(C_FUNPTR), public :: nvdotprodmulti
  type(C_FUNPTR), public :: nvlinearsumvectorarray
  type(C_FUNPTR), public :: nvscalevectorarray
  type(C_FUNPTR), public :: nvconstvectorarray
  type(C_FUNPTR), public :: nvwrmsnormvectorarray
  type(C_FUNPTR), public :: nvwrmsnormmaskvectorarray
  type(C_FUNPTR), public :: nvscaleaddmultivectorarray
  type(C_FUNPTR), public :: nvlinearcombinationvectorarray
  type(C_FUNPTR), public :: nvdotprodlocal
  type(C_FUNPTR), public :: nvmaxnormlocal
  type(C_FUNPTR), public :: nvminlocal
  type(C_FUNPTR), public :: nvl1normlocal
  type(C_FUNPTR), public :: nvinvtestlocal
  type(C_FUNPTR), public :: nvconstrmasklocal
  type(C_FUNPTR), public :: nvminquotientlocal
  type(C_FUNPTR), public :: nvwsqrsumlocal
  type(C_FUNPTR), public :: nvwsqrsummasklocal
  type(C_FUNPTR), public :: nvbufsize
  type(C_FUNPTR), public :: nvbufpack
  type(C_FUNPTR), public :: nvbufunpack
 end type N_Vector_Ops
\end{verbatim}

% =====================================================================
\subsection{Implementing a custom NVECTOR}
\label{ss:nvector_custom_implmentation}

A particular implementation of the {\nvector} module must:

\begin{itemize}
\item Specify the {\em content} field of \id{N\_Vector}.
\item Define and implement the vector operations.
  Note that the names of these routines should be unique to that implementation in order
  to permit using more than one {\nvector} module (each with different \id{N\_Vector}
  internal data representations) in the same code.
\item Define and implement user-callable constructor and destructor
  routines to create and free an \id{N\_Vector} with
  the new {\em content} field and with {\em ops} pointing to the
  new vector operations.
\item Optionally, define and implement additional user-callable routines
  acting on the newly defined \id{N\_Vector} (e.g., a routine to print
  the content for debugging purposes).
\item Optionally, provide accessor macros as needed for that particular implementation to
  be used to access different parts in the {\em content} field of the newly defined \id{N\_Vector}.
\end{itemize}

It is recommended that a user-supplied {\nvector} implementation returns the
\id{SUNDIALS\_NVEC\_CUSTOM} identifier from the \id{N\_VGetVectorID} function.

To aid in the creation of custom {\nvector} modules the generic {\nvector}
module provides two utility functions \id{N\_VNewEmpty} and \id{N\_VCopyOps}.
When used in custom {\nvector} constructors and clone routines these functions
will ease the introduction of any new optional vector operations to the
{\nvector} API by ensuring only required operations need to be set and all
operations are copied when cloning a vector.


\subsubsection{Support for complex-valued vectors}

While {\sundials} itself is written under an assumption of real-valued
data, it does provide limited support for complex-valued problems.
However, since none of the built-in {\nvector} modules supports
complex-valued data, users must provide a custom {\nvector}
implementation for this task.  Many of the {\nvector} routines
described in Sections \ref{ss:nvecops}-\ref{ss:nveclocalops} above
naturally extend to complex-valued vectors; however, some do not.  To
this end, we provide the following guidance:

\begin{itemize}
\item \id{N\_VMin} and \id{N\_VMinLocal} should return the minimum of
  all \emph{real} components of the vector, i.e.,  $m = \min_i
  \operatorname{real}(x_i) $.

\item \id{N\_VConst} (and similarly \id{N\_VConstVectorArray}) should
  set the real components of the vector to the input constant, and set
  all imaginary components to zero, i.e.,
  $z_i = c + 0 j,\: i=0,\ldots,n-1$.

\item \id{N\_VAddConst} should only update the real components of the
  vector with the input constant, leaving all imaginary components
  unchanged.

\item \id{N\_VWrmsNorm}, \id{N\_VWrmsNormMask}, \id{N\_VWSqrSumLocal}
  and \id{N\_VWSqrSumMaskLocal} should assume that all entries of the
  weight vector \id{w} and the mask vector \id{id} are real-valued.

\item \id{N\_VDotProd} should mathematically return a complex number
  for complex-valued vectors; as this is not possible with
  {\sundials}' current \id{realtype}, this routine should
  be set to \id{NULL} in the custom {\nvector} implementation.

\item \id{N\_VCompare}, \id{N\_VConstrMask}, \id{N\_VMinQuotient},
  \id{N\_VConstrMaskLocal} and \id{N\_VMinQuotientLocal}
  are ill-defined due to the lack of a clear ordering in the
  complex plane.  These routines should be set to \id{NULL}
  in the custom {\nvector} implementation.

\end{itemize}

While many {\sundials} solver modules may be utilized on
complex-valued data, others cannot.  Specifically, although both
{\sunnonlinsolnewton} and {\sunnonlinsolfixedpoint} may be used with
any of the IVP solvers ({\cvode}, {\cvodes}, {\ida}, {\idas} and
{\arkode}) for complex-valued problems, the Anderson-acceleration
feature {\sunnonlinsolfixedpoint} cannot be used due to its reliance
on \id{N\_VDotProd}.  By this same logic, the Anderson acceleration
feature within {\kinsol} also will not work with complex-valued
vectors.

Similarly, although each package's linear solver interface (e.g.,
{\cvls}) may be used on complex-valued problems, none of the built-in
{\sunmatrix} or {\sunlinsol} modules work.  Hence a complex-valued
user should provide a custom {\sunlinsol} (and optionally a custom
{\sunmatrix}) implementation for solving linear systems, and then
attach this module as normal to the package's linear solver
interface.

Finally, constraint-handling features of each package cannot be used
for complex-valued data, due to the issue of
ordering in the complex plane discussed above with
\id{N\_VCompare}, \id{N\_VConstrMask}, \id{N\_VMinQuotient},
\id{N\_VConstrMaskLocal} and \id{N\_VMinQuotientLocal}.

We provide a simple example of a complex-valued example problem,
including a custom complex-valued Fortran 2003 {\nvector} module, in the
files
\newline\noindent\id{examples/arkode/F2003\_custom/ark\_analytic\_complex\_f2003.f90},
\newline\noindent\id{examples/arkode/F2003\_custom/fnvector\_complex\_mod.f90}, and
\newline\noindent\id{examples/arkode/F2003\_custom/test\_fnvector\_complex\_mod.f90}.
