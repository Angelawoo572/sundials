% This is a shared SUNDIALS TEX file with description of
% the CUDA nvector implementation
%
\section{The NVECTOR\_CUDA implementation}\label{ss:nvec_cuda}

The {\nveccuda} module is an experimental {\nvector} implementation in the {\cuda} language.
The module allows for {\sundials} vector kernels to run on GPU devices. It is intended for users
who are already familiar with {\cuda} and GPU programming. Building this vector
module requires a CUDA compiler and, by extension, a C++ compiler. The class \id{Vector}
in the namespace \id{suncudavec} manages the vector data layout:

\begin{verbatim}
template <class T, class I>
class Vector {
  I size_;
  I mem_size_;
  I global_size_;
  T* h_vec_;
  T* d_vec_;
  ThreadPartitioning<T, I>* partStream_;
  ThreadPartitioning<T, I>* partReduce_;
  bool ownPartitioning_;
  bool ownData_;
  bool managed_mem_;
  SUNMPI_Comm comm_;
  ...
};
\end{verbatim}

The class members are vector size (length), size of the vector data memory
block, pointers to vector data on the host and the device, pointers
to \id{ThreadPartitioning} implementations that handle thread partitioning for
streaming and reduction vector kernels, a boolean flag that signals if the
vector owns the thread partitioning, a boolean flag that signals if the vector
owns the data, a boolean flag that signals if managed memory is used for the
data arrays, and the MPI communicator. The class \id{Vector} inherits from the
empty structure
\begin{verbatim}
struct _N_VectorContent_Cuda {};
\end{verbatim}
to interface the C++ class with the {\nvector} C code. Due to the rapid progress
of {\cuda} development, we expect that the \id{suncudavec::Vector} class will
change frequently in future {\sundials} releases. The code is structured so that
it can tolerate significant changes in the \id{suncudavec::Vector} class without
requiring changes to the user API.

When instantiated with \id{N\_VNew\_Cuda}, the class \id{Vector} will allocate
memory on both the host and the device. Alternatively, a user can provide host
and device data arrays by using the \id{N\_VMake\_Cuda} constructor. To use {\cuda}
managed memory, the constructors \id{N\_VNewManaged\_Cuda} and \newline
\id{N\_VMakeManaged\_Cuda} are provided. Details on each of these constructors
are provided below.

The {\nveccuda} module can be utilized for single-node parallelism or in a
distributed context with MPI. In the single-node case the header file to
include \id{nvector\_cuda.h} and the library to link to is
\id{libsundials\_nveccuda.\textit{lib}}. In the a distributed setting the header
file to include is \id{nvector\_mpicuda.h} and the library to link to is
\id{libsundials\_nvecmpicuda.\textit{lib}}. The extension, \id{\em.lib}, is
typically \id{.so} for shared libraries and \id{.a} for static libraries.
Only one of these libraries may be linked to when creating an executable
or library. {\sundials} must be built with MPI support if the distributed
library is desired. 


% ====================================================================
\subsection{NVECTOR\_CUDA functions}
\label{ss:nvec_cuda_functions}
% ====================================================================

Unlike other native {\sundials} vector types, {\nveccuda} does not provide macros
to access its member variables. Instead, user should use the accessor functions:
%%--------------------------------------
\sunmodfun{N\_VGetLength\_Cuda}
{
  This function returns the global length of the vector.
}
{
  sunindextype N\_VGetLength\_Cuda(N\_Vector v)
}
%%--------------------------------------
\sunmodfun{N\_VGetLocalLength\_Cuda}
{
  This function returns the local length of the vector.
  
  Note: This function is for use in a \textit{distributed} context and
  is defined in the header \id{nvector\_mpicuda.h} and the library
  to link to is \id{libsundials\_nvecmpicuda}.\text{lib}.
}
{
  sunindextype N\_VGetLocalLength\_Cuda(N\_Vector v)
}
%%--------------------------------------
\sunmodfun{N\_VGetHostArrayPointer\_Cuda}
{
  This function returns a pointer to the vector data on the host.
}
{
  realtype *N\_VGetHostArrayPointer\_Cuda(N\_Vector v)
}
%%--------------------------------------
\sunmodfun{N\_VGetDeviceArrayPointer\_Cuda}
{
  This function returns a pointer to the vector data on the device.
}
{
  realtype *N\_VGetDeviceArrayPointer\_Cuda(N\_Vector v)
}
%%--------------------------------------
\sunmodfun{N\_VGetMPIComm\_Cuda}
{
  This function returns the {\mpi} communicator for the vector.

  Note: This function is for use in a \textit{distributed} context
  and is defined in the header \id{nvector\_mpicuda.h} and the
  library to link to is \id{libsundials\_nvecmpicuda}.\text{lib}.
}
{
  MPI\_Comm N\_VGetMPIComm\_Cuda(N\_Vector v)
}
%%--------------------------------------
\sunmodfun{N\_VIsManagedMemory\_Cuda}
{
  This function returns a boolean flag indicating if the vector
  data is allocated in managed memory or not.
}
{
  booleantype *N\_VIsManagedMemory\_Cuda(N\_Vector v)
}
%%--------------------------------------------

The {\nveccuda} module defines implementations of all vector operations listed
in Tables \ref{t:nvecops}, \ref{t:nvecfusedops}, and \ref{t:nvecarrayops}, except
for \id{N\_VGetArrayPointer} and \id{N\_VSetArrayPointer}.
As such, this vector cannot be used with the {\sundials} Fortran interfaces,
nor with the {\sundials} direct solvers and preconditioners. Instead,
the {\nveccuda} module provides separate functions to access data on the host
and on the device. It also provides methods for copying from the host to
the device and vice versa. Usage examples of {\nveccuda} are provided in
some example programs for {\cvode} \cite{cvode_ex}.

The names of vector operations are obtained from those in Tables \ref{t:nvecops},
\ref{t:nvecfusedops}, and \ref{t:nvecarrayops} by appending the suffix \id{\_Cuda}
(e.g. \id{N\_VDestroy\_Cuda}). The module {\nveccuda} provides the following
functions:
%%--------------------------------------
\sunmodfuns{N\_VNew\_Cuda}
{
  This function creates and allocates memory for a {\cuda} \id{N\_Vector}.
  The vector data array is allocated on both the host and device.
}
{
  In the \textit{single-node} setting, the only input is the vector length. This
  constructor is defined in the header \id{nvector\_cuda.h} and the library to
  link to is \id{libsundials\_nveccuda}.\text{lib}.
}
{
  N\_Vector N\_VNew\_Cuda(sunindextype length)
}
{
  When used in a \textit{distributed} context with MPI, the arguments are the
  {\mpi} communicator, the local vector length, and the global vector length.
  This constructor is defined in the header \id{nvector\_mpicuda.h} and
  the library to link to is \id{libsundials\_nvecmpicuda}.\text{lib}.
}
{
  N\_Vector N\_VNew\_Cuda(MPI\_Comm comm, sunindextype local\_length,
  \newlinefill{N\_Vector N\_VNew\_Cuda}
  sunindextype global\_length)
}
%%--------------------------------------
\sunmodfuns{N\_VNewManaged\_Cuda}
{
  This function creates and allocates memory for a {\cuda} \id{N\_Vector}
  on a single node. The vector data array is allocated in managed memory.
}
{
  In the \textit{single-node} setting, the only input is the vector length. This
  constructor is defined in the header \id{nvector\_cuda.h} and the library to
  link to is \id{libsundials\_nveccuda}.\text{lib}.
}
{
  N\_Vector N\_VNewManaged\_Cuda(sunindextype length)
}
{
  When used in a \textit{distributed} context with MPI, the arguments are the
  {\mpi} communicator, the local vector lenght, and the global vector length.
  This constructor is defined in the header \id{nvector\_mpicuda.h} and
  the library to link to is \id{libsundials\_nvecmpicuda}.\text{lib}.
}
{
  N\_Vector N\_VNewManaged\_Cuda(MPI\_Comm comm, sunindextype local\_length,
  \newlinefill{N\_Vector N\_VNewManaged\_Cuda}
  sunindextype global\_length)
}
%%--------------------------------------
\sunmodfun{N\_VNewEmpty\_Cuda}
{
  This function creates a new {\nvector} wrapper with the pointer to
  the wrapped {\cuda} vector set to \id{NULL}. It is used by the
  \id{N\_VNew\_Cuda}, \id{N\_VMake\_Cuda}, and \id{N\_VClone\_Cuda}
  implementations.
}
{
  N\_Vector N\_VNewEmpty\_Cuda()
}
%%--------------------------------------
\sunmodfuns{N\_VMake\_Cuda}
{
  This function creates an {\nveccuda} with user-supplied vector data arrays
  \id{h\_vdata} and \id{d\_vdata}. This function does not allocate memory for
  data itself.
}
{
  In the \textit{single-node} setting, the inputs are the vector length, the
  host data array, and the device data. This constructor is defined in the
  header \id{nvector\_cuda.h} and the library to link to is
  \id{libsundials\_nveccuda}.\text{lib}.
}
{
  N\_Vector N\_VMake\_Cuda(sunindextype length, realtype *h\_vdata,
  \newlinefill{N\_Vector N\_VMake\_Cuda}
  realtype *d\_vdata)
}
{
  When used in a \textit{distributed} context with MPI, the arguments are the
  {\mpi} communicator, the local vector lenght, the global vector length, the
  host data array, and the device data array. This constructor is defined in the
  header \id{nvector\_mpicuda.h} and the library to link to is
  \id{libsundials\_nvecmpicuda}.\text{lib}.
}
{
  N\_Vector N\_VMake\_Cuda(MPI\_Comm comm, sunindextype local\_length,
  \newlinefill{N\_Vector N\_VMake\_Cuda}
  sunindextype global\_length, realtype *h\_vdata,
  \newlinefill{N\_Vector N\_VMake\_Cuda}
  realtype *d\_vdata)
}
%%--------------------------------------
\sunmodfuns{N\_VMakeManaged\_Cuda}
{
  This function creates an {\nveccuda} with a user-supplied managed memory data
  array. This function does not allocate memory for data itself.
}
{
  In the \textit{single-node} setting, the inputs are the vector length and the
  managed data array. This constructor is defined in the header
  \id{nvector\_cuda.h} and the library to link to is
  \id{libsundials\_nveccuda}.\text{lib}.
}
{
  N\_Vector N\_VMakeManaged\_Cuda(sunindextype length, realtype *vdata)
}
{
  When used in a \textit{distributed} context with MPI, the arguments are the
  {\mpi} communicator, the local vector lenght, the global vector length, the
  managed data array. This constructor is defined in the header
  \id{nvector\_mpicuda.h} and the library to link to is 
  \id{libsundials\_nvecmpicuda}.\text{lib}.
}
{
  N\_Vector N\_VMakeManaged\_Cuda(MPI\_Comm comm, sunindextype local\_length,
  \newlinefill{N\_Vector N\_VMakeManaged\_Cuda}
  sunindextype global\_length, realtype *vdata)
}


The module {\nveccuda} also provides the following user-callable routines:
%%--------------------------------------
\sunmodfun{N\_VSetCudaStream\_Cuda}
{
  This function sets the {\cuda} stream that all vector kernels will be launched on.
  By default an {\nveccuda} uses the default {\cuda} stream.\\

  \textit{Note: All vectors used in a single instance of a SUNDIALS solver must
  use the same {\cuda} stream, and the {\cuda} stream must be set prior to
  solver initialization. Additionally, if manually instantiating the stream and
  reduce \id{ThreadPartitioning} of a \id{suncudavec::Vector}, ensure that they
  use the same {\cuda} stream.}
}
{
  void N\_VSetCudaStream\_Cuda(N\_Vector v, cudaStream\_t *stream)
}
%%--------------------------------------
\sunmodfun{N\_VCopyToDevice\_Cuda}
{
 This function copies host vector data to the device.
}
{
 void N\_VCopyToDevice\_Cuda(N\_Vector v)
}
%%--------------------------------------
\sunmodfun{N\_VCopyFromDevice\_Cuda}
{
 This function copies vector data from the device to the host.
}
{
 void N\_VCopyFromDevice\_Cuda(N\_Vector v)
}
%%--------------------------------------
\sunmodfun{N\_VPrint\_Cuda}
{
  This function prints the content of a {\cuda} vector to \id{stdout}.
}
{
  void N\_VPrint\_Cuda(N\_Vector v)
}
%%--------------------------------------
\sunmodfun{N\_VPrintFile\_Cuda}
{
  This function prints the content of a {\cuda} vector to \id{outfile}.
}
{
  void N\_VPrintFile\_Cuda(N\_Vector v, FILE *outfile)
}
%%--------------------------------------

By default all fused and vector array operations are disabled in the {\nveccuda}
module. The following additional user-callable routines are provided to
enable or disable fused and vector array operations for a specific vector. To
ensure consistency across vectors it is recommended to first create a vector
with \id{N\_VNew\_Cuda}, enable/disable the desired operations for that vector
with the functions below, and create any additional vectors from that vector
using \id{N\_VClone}. This guarantees the new vectors will have the same
operations enabled/disabled as cloned vectors inherit the same enable/disable
options as the vector they are cloned from while vectors created with
\id{N\_VNew\_Cuda} will have the default settings for the {\nveccuda} module.
%%--------------------------------------
\sunmodfun{N\_VEnableFusedOps\_Cuda}
{
  This function enables (\id{SUNTRUE}) or disables (\id{SUNFALSE}) all fused and
  vector array operations in the {\cuda} vector. The return value is \id{0} for
  success and \id{-1} if the input vector or its \id{ops} structure are \id{NULL}.
}
{
  int N\_VEnableFusedOps\_Cuda(N\_Vector v, booleantype tf)
}
%%--------------------------------------
\sunmodfun{N\_VEnableLinearCombination\_Cuda}
{
  This function enables (\id{SUNTRUE}) or disables (\id{SUNFALSE}) the linear
  combination fused operation in the {\cuda} vector. The return value is \id{0} for
  success and \id{-1} if the input vector or its \id{ops} structure are \id{NULL}.
}
{
  int N\_VEnableLinearCombination\_Cuda(N\_Vector v, booleantype tf)
}
%%--------------------------------------
\sunmodfun{N\_VEnableScaleAddMulti\_Cuda}
{
  This function enables (\id{SUNTRUE}) or disables (\id{SUNFALSE}) the scale and
  add a vector to multiple vectors fused operation in the {\cuda} vector. The
  return value is \id{0} for success and \id{-1} if the input vector or its
  \id{ops} structure are \id{NULL}.
}
{
  int N\_VEnableScaleAddMulti\_Cuda(N\_Vector v, booleantype tf)
}
%%--------------------------------------
\sunmodfun{N\_VEnableDotProdMulti\_Cuda}
{
  This function enables (\id{SUNTRUE}) or disables (\id{SUNFALSE}) the multiple
  dot products fused operation in the {\cuda} vector. The return value is \id{0}
  for success and \id{-1} if the input vector or its \id{ops} structure are
  \id{NULL}.
}
{
  int N\_VEnableDotProdMulti\_Cuda(N\_Vector v, booleantype tf)
}
%%--------------------------------------
\sunmodfun{N\_VEnableLinearSumVectorArray\_Cuda}
{
  This function enables (\id{SUNTRUE}) or disables (\id{SUNFALSE}) the linear sum
  operation for vector arrays in the {\cuda} vector. The return value is \id{0} for
  success and \id{-1} if the input vector or its \id{ops} structure are \id{NULL}.
}
{
  int N\_VEnableLinearSumVectorArray\_Cuda(N\_Vector v, booleantype tf)
}
%%--------------------------------------
\sunmodfun{N\_VEnableScaleVectorArray\_Cuda}
{
  This function enables (\id{SUNTRUE}) or disables (\id{SUNFALSE}) the scale
  operation for vector arrays in the {\cuda} vector. The return value is \id{0} for
  success and \id{-1} if the input vector or its \id{ops} structure are \id{NULL}.
}
{
  int N\_VEnableScaleVectorArray\_Cuda(N\_Vector v, booleantype tf)
}
%%--------------------------------------
\sunmodfun{N\_VEnableConstVectorArray\_Cuda}
{
  This function enables (\id{SUNTRUE}) or disables (\id{SUNFALSE}) the const
  operation for vector arrays in the {\cuda} vector. The return value is \id{0} for
  success and \id{-1} if the input vector or its \id{ops} structure are \id{NULL}.
}
{
  int N\_VEnableConstVectorArray\_Cuda(N\_Vector v, booleantype tf)
}
%%--------------------------------------
\sunmodfun{N\_VEnableWrmsNormVectorArray\_Cuda}
{
  This function enables (\id{SUNTRUE}) or disables (\id{SUNFALSE}) the WRMS norm
  operation for vector arrays in the {\cuda} vector. The return value is \id{0} for
  success and \id{-1} if the input vector or its \id{ops} structure are \id{NULL}.
}
{
  int N\_VEnableWrmsNormVectorArray\_Cuda(N\_Vector v, booleantype tf)
}
%%--------------------------------------
\sunmodfun{N\_VEnableWrmsNormMaskVectorArray\_Cuda}
{
  This function enables (\id{SUNTRUE}) or disables (\id{SUNFALSE}) the masked WRMS
  norm operation for vector arrays in the {\cuda} vector. The return value is
  \id{0} for success and \id{-1} if the input vector or its \id{ops} structure are
  \id{NULL}.
}
{
  int N\_VEnableWrmsNormMaskVectorArray\_Cuda(N\_Vector v, booleantype tf)
}
%%--------------------------------------
\sunmodfun{N\_VEnableScaleAddMultiVectorArray\_Cuda}
{
  This function enables (\id{SUNTRUE}) or disables (\id{SUNFALSE}) the scale and
  add a vector array to multiple vector arrays operation in the {\cuda} vector. The
  return value is \id{0} for success and \id{-1} if the input vector or its
  \id{ops} structure are \id{NULL}.
}
{
  int N\_VEnableScaleAddMultiVectorArray\_Cuda(N\_Vector v, booleantype tf)
}
%%--------------------------------------
\sunmodfun{N\_VEnableLinearCombinationVectorArray\_Cuda}
{
  This function enables (\id{SUNTRUE}) or disables (\id{SUNFALSE}) the linear
  combination operation for vector arrays in the {\cuda} vector. The return value
  is \id{0} for success and \id{-1} if the input vector or its \id{ops} structure
  are \id{NULL}.
}
{
  int N\_VEnableLinearCombinationVectorArray\_Cuda(N\_Vector v,
  \newlinefill{int N\_VEnableLinearCombinationVectorArray\_Cuda}
  booleantype tf)
}
%%
%%------------------------------------
%%
\paragraph{\bf Notes}

\begin{itemize}

\item
  When there is a need to access components of an \id{N\_Vector\_Cuda}, \id{v},
  it is recommeded to use functions \id{N\_VGetDeviceArrayPointer\_Cuda} or
  \id{N\_VGetHostArrayPointer\_Cuda}.

\item
  {\warn}To maximize efficiency, vector operations in the {\nveccuda} implementation
  that have more than one \id{N\_Vector} argument do not check for
  consistent internal representations of these vectors. It is the user's
  responsibility to ensure that such routines are called with \id{N\_Vector}
  arguments that were all created with the same internal representations.

\end{itemize}

