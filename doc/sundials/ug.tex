\usepackage{epsfig, supertabular, makeidx}

\usepackage{amsmath, amssymb}

% Package to add Bibliography and Index to TOC
% but not the TOC itself :-)
\usepackage[nottoc]{tocbibind}

\usepackage{calc}
\usepackage{hangcaption}
\usepackage{subfigure}
\usepackage[usenames]{color}
\usepackage{epsfig}
\usepackage{pstricks}

% Needed for sideways
\usepackage{rotating}

% Packages used for output and source code listings
\usepackage{fancyvrb}
\usepackage{listings}

\usepackage{fancyheadings}

% External references
\usepackage{xr}

%----- Define some colors
\definecolor{gray}{rgb}{0.5,0.5,0.5} 

%----- Define the warning sign

\newcommand{\marginlabel}[1]
{\mbox{}\marginpar{\raggedleft\hspace{0pt}{\rule{0mm}{1mm}#1}}}

\newcommand{\warn}{%
\marginlabel{\pstribox[shadow=false,fillstyle=solid,fillcolor=yellow,trimode=*U]{\footnotesize !}}%%
}

%----- Page formatting
%% \paperwidth = (1in+\hoffset)*2 + \oddsidemargin + \textwidth + \evensidemargin

\setlength{\paperwidth}{8.5in}
\setlength{\textwidth}{6.1in}
\setlength{\hoffset}{0.0in}
\setlength{\oddsidemargin}{0.2in}
\setlength{\evensidemargin}{0.2in}
\setlength{\marginparsep}{0.25in}
\setlength{\marginparwidth}{0.5in}

\setlength{\paperheight}{11.0in}
\setlength{\textheight}{9.0in}
\setlength{\voffset}{-0.25in}
\setlength{\topmargin}{0.0in}
%% 
%% \setlength{\textheight}{9.0in}

%----- VERSIONS AND UCRL NUMBERS OF SUNDIALS CODES
\newcommand{\cvrelease}{v2.5.0}
\newcommand{\cvucrlug}{UCRL-SM-208108}
\newcommand{\cvucrlex}{UCRL-SM-208110}

\newcommand{\cvsrelease}{v2.5.0}
\newcommand{\cvsucrlug}{UCRL-SM-208111}
\newcommand{\cvsucrlex}{UCRL-SM-208115}

\newcommand{\idarelease}{v2.5.0}
\newcommand{\idaucrlug}{UCRL-SM-208112}
\newcommand{\idaucrlex}{UCRL-SM-208113}

\newcommand{\kinrelease}{v2.5.0}
\newcommand{\kinucrlug}{UCRL-SM-208116}
\newcommand{\kinucrlex}{UCRL-SM-208114}

%----- SUNDIALS MODULES
\newcommand{\sundials}{{\normalfont\scshape sundials}}
\newcommand{\shared}{{\normalfont\scshape shared}}
\newcommand{\nvector}{{\normalfont\scshape nvector}}
\newcommand{\fnvector}{{\normalfont\scshape fnvector}}
\newcommand{\nvecp}{{\normalfont\scshape nvector\_parallel}}
\newcommand{\nvecs}{{\normalfont\scshape nvector\_serial}}
\newcommand{\nvecspc}{{\normalfont\scshape nvector\_spcparallel}}
\newcommand{\cvode}{{\normalfont\scshape cvode}}
\newcommand{\pvode}{{\normalfont\scshape pvode}}
\newcommand{\cvodes}{{\normalfont\scshape cvodes}}
\newcommand{\ida}{{\normalfont\scshape ida}}
\newcommand{\idas}{{\normalfont\scshape idas}}
\newcommand{\kinsol}{{\normalfont\scshape kinsol}}
\newcommand{\sundialsTB}{{\normalfont\scshape sundialsTB}}

%----- OTHER PACKAGES
\newcommand{\vode}{{\normalfont\scshape vode}}
\newcommand{\vodpk}{{\normalfont\scshape vodpk}}
\newcommand{\lsode}{{\normalfont\scshape lsode}}
\newcommand{\daspk}{{\normalfont\scshape daspk}}
\newcommand{\daspkadjoint}{{\normalfont\scshape daspkadjoint}}

%----- CVODE and CVODES COMPONENTS
\newcommand{\cvdls}{{\normalfont\scshape cvdls}}
\newcommand{\cvdense}{{\normalfont\scshape cvdense}}
\newcommand{\cvband}{{\normalfont\scshape cvband}}
\newcommand{\cvdiag}{{\normalfont\scshape cvdiag}}
\newcommand{\cvspils}{{\normalfont\scshape cvspils}}
\newcommand{\cvspgmr}{{\normalfont\scshape cvspgmr}}
\newcommand{\cvspbcg}{{\normalfont\scshape cvspbcg}}
\newcommand{\cvsptfqmr}{{\normalfont\scshape cvsptfqmr}}
\newcommand{\cvsp}{{\normalfont\scshape cvsp}\id{***}}
\newcommand{\cvbandpre}{{\normalfont\scshape cvbandpre}}
\newcommand{\cvbbdpre}{{\normalfont\scshape cvbbdpre}}
\newcommand{\cvodea}{{\normalfont\scshape cvodea}}
\newcommand{\fcvode}{{\normalfont\scshape fcvode}}
\newcommand{\fcvbp}{{\normalfont\scshape fcvbp}}
\newcommand{\fcvbbd}{{\normalfont\scshape fcvbbd}}
\newcommand{\fcvroot}{{\normalfont\scshape fcvroot}}
\newcommand{\stald}{{\normalfont\scshape stald}}

%----- KINSOL COMPONENTS
\newcommand{\kinspils}{{\normalfont\scshape kinspils}}
\newcommand{\kinspgmr}{{\normalfont\scshape kinspgmr}}
\newcommand{\kinspbcg}{{\normalfont\scshape kinspbcg}}
\newcommand{\kinsptfqmr}{{\normalfont\scshape kinsptfqmr}}
\newcommand{\kinbbdpre}{{\normalfont\scshape kinbbdpre}}
\newcommand{\kinsp}{{\normalfont\scshape kinsp}\id{***}}
\newcommand{\kindense}{{\normalfont\scshape kindense}}
\newcommand{\kinband}{{\normalfont\scshape kinband}}
\newcommand{\fkinbbd}{{\normalfont\scshape fkinbbd}}
\newcommand{\fkindense}{{\normalfont\scshape fkindense}}
\newcommand{\fkinband}{{\normalfont\scshape fkinband}}
\newcommand{\fkinsol}{{\normalfont\scshape fkinsol}}

%----- IDA and IDAS COMPONENTS
\newcommand{\idadls}{{\normalfont\scshape idadls}}
\newcommand{\idadense}{{\normalfont\scshape idadense}}
\newcommand{\idaband}{{\normalfont\scshape idaband}}
\newcommand{\idaspils}{{\normalfont\scshape idaspils}}
\newcommand{\idaspgmr}{{\normalfont\scshape idaspgmr}}
\newcommand{\idaspbcg}{{\normalfont\scshape idaspbcg}}
\newcommand{\idasptfqmr}{{\normalfont\scshape idasptfqmr}}
\newcommand{\idasp}{{\normalfont\scshape idasp}\id{***}}
\newcommand{\idabbdpre}{{\normalfont\scshape idabbdpre}}
\newcommand{\idaa}{{\normalfont\scshape idaa}}
\newcommand{\fida}{{\normalfont\scshape fida}}
\newcommand{\fidaband}{{\normalfont\scshape fidaband}}
\newcommand{\fidadense}{{\normalfont\scshape fidadense}}
\newcommand{\fidabbd}{{\normalfont\scshape fidabbd}}
\newcommand{\fidaroot}{{\normalfont\scshape fidaroot}}

%----- SHARED COMPONENTS
\newcommand{\sundialsmath}{{\normalfont\scshape sundialsmath}}
\newcommand{\diag}{{\normalfont\scshape diag}}
\newcommand{\dense}{{\normalfont\scshape dense}}
\newcommand{\band}{{\normalfont\scshape band}}
\newcommand{\spgmr}{{\normalfont\scshape spgmr}}
\newcommand{\spbcg}{{\normalfont\scshape spbcg}}
\newcommand{\sptfqmr}{{\normalfont\scshape sptfqmr}}

%----- OS
\newcommand{\linux}{{\sc LINUX}}
\newcommand{\unix}{{\sc UNIX}}
\newcommand{\mpi}{{\sc MPI}}

%----- C and Fortran languages
\newcommand{\C}{{\sc C}}
\newcommand{\CPP}{{\sc C}\raisebox{0.2em}{\tiny ++}}
\newcommand{\F}{{\sc Fortran}}

%------ Serial or Parallel
\newcommand{\p}{[{\bf P}]}
\newcommand{\s}{[{\bf S}]}

%------ Appendix in text
\newcommand{\A}{App. }

%------ Index entries
\newcommand{\ID}[1]{{\tt #1}{\index{#1@\texttt{#1}}}}
%\newcommand{\ID}[1]{{\tt #1}{\index{#1@\texttt{#1}|textbf}}}
% NOTE: the above version confuses hyperref...
\newcommand{\Id}[1]{{\tt #1}{\index{#1@\texttt{#1}}}}
\newcommand{\id}[1]{{\tt #1}}

%%----- Shortcuts for math formulas
\newcommand{\mb}[1]{{\mbox{\scriptsize #1}}}

\newcommand{\dfdy}{\frac{\partial f}{\partial y}}
\newcommand{\dfdyI}{\partial f / \partial y}
\newcommand{\dfdpi}{\frac{\partial f}{\partial p_i}}
\newcommand{\dfdpiI}{\partial f / \partial p_i}

\newcommand{\dFdy}{\frac{\partial f}{\partial y}}
\newcommand{\dFdyI}{\partial f / \partial y}
\newcommand{\dFdyp}{\frac{\partial f}{\partial y'}}
\newcommand{\dFdypI}{\partial f / \partial y'}
\newcommand{\dFdpi}{\frac{\partial f}{\partial p_i}}
\newcommand{\dFdpiI}{\partial f / \partial p_i}

\newcommand{\rhomax}{\rho_{\max}}

\newcommand{\cm}{$\checkmark$}

%%------ Specify depths for section numbering and TOC

\setcounter{tocdepth}{3}
\setcounter{secnumdepth}{3}

%%-------

\newcommand{\includeCode}[1]
{
  \VerbatimInput[numbers=left,fontsize=\small]{#1}
}

\newcommand{\includeOutput}[2]
{
  \vspace{0.1in}
  \VerbatimInput[frame=single,
                 framesep=0.1in,
                 label={\tt #1} sample output,
                 fontsize=\footnotesize]{#2}
}

%%-------

\newcommand{\ugref}[1]{\S\ref{#1}}

%%-------

\newcommand{\clearemptydoublepage}{\newpage{\pagestyle{empty}\cleardoublepage}}

%%-------

\newcommand{\disclaimer}{%
\thispagestyle{empty}% no number of this page
\vglue5\baselineskip
\begin{center}
{\bf DISCLAIMER}
\end{center}
\noindent
This document was prepared as an account of work sponsored by an agency of the
United States Government.  Neither the United States Government nor the University
of California nor any of their employees, makes any warranty, express or implied,
or assumes any legal liability or responsibility for the accuracy, completeness,
or usefulness of any information, apparatus, product, or process disclosed, or
represents that its use would not infringe privately owned rights. Reference
herein to any specific commercial product, process, or service by trade name,
trademark, manufacturer, or otherwise, does not necessarily constitute or imply
its endorsement, recommendation, or favoring by the United States Government
or the University of California.  The views and opinions of authors expressed
herein do not necessarily state or reflect those of the United States Government
or the University of California, and shall not be used for advertising or
product endorsement purposes.

\vskip2\baselineskip
This research was supported under the auspices of the U.S. Department of Energy by
the University of California, Lawrence Livermore National Laboratory under
contract No.  W-7405-Eng-48.
\vfill
\begin{center}
Approved for public release; further dissemination unlimited
\end{center}
\clearpage
}


%%--------------------------------

\newcommand{\frontug}
{

  %% Start roman numbering
  \pagenumbering{roman}
  
  %% Title page
  \maketitle
  \disclaimer

  %% Contents, tables, and figures
  \tableofcontents
  \clearemptydoublepage
  \listoftables
  \clearemptydoublepage
  \listoffigures
  \clearemptydoublepage
  
  %% Start arabic numbering
  \pagenumbering{arabic}

  %% Define page style for the body of the document
  \lhead[\fancyplain{}{\bfseries\thepage}]%
        {\fancyplain{}{\bfseries\rightmark}}
  \rhead[\fancyplain{}{\bfseries\leftmark}]%
        {\fancyplain{}{\bfseries\thepage}}
  \cfoot{}
  \pagestyle{fancyplain}


  %% Some settings for code listings
  \lstset{
    language=C, fancyvrb=true, 
    basicstyle=\footnotesize\ttfamily, 
    commentstyle=\color{MidnightBlue},
    keywordstyle=\color{BrickRed},
    stringstyle=\color{OliveGreen},
    numbers=left, numberstyle=\tiny, numbersep=15pt}

}

%%--------------------------------
\newcommand{\frontex}
{

  \pagestyle{empty}
  \maketitle
  \disclaimer
  \tableofcontents

  %% Start arabic numbering
  \clearemptydoublepage
  %%\clearpage
  \pagestyle{plain}\pagenumbering{arabic}

  %% Some settings for code listings
  \lstset{
    fancyvrb=true, 
    basicstyle=\footnotesize\ttfamily, 
    commentstyle=\color{MidnightBlue},
    keywordstyle=\color{BrickRed},
    stringstyle=\color{OliveGreen},
    numbers=left, numberstyle=\tiny, numbersep=15pt}

}

%%----------------------------------
%% List of configure options
%%---------------------------------
\newenvironment{config}
{\begin{list}{}{
      \setlength{\leftmargin}{2em}
      \setlength{\rightmargin}{0em}
      \setlength{\topsep}{0.05in}
      \setlength{\itemindent}{-2em}
      \setlength{\itemsep}{0.05in}}}
  {\end{list}}
%%---------------------------------
%% Steps used in skeleton programs
%%---------------------------------
\newcounter{Stepsctr}
\newenvironment{Steps}
{\stepcounter{Stepsctr}
  \begin{list}{\arabic{Stepsctr}. }{
      \usecounter{Stepsctr}
      \setlength{\parsep}{0.5em}
      \setlength{\labelsep}{0em}
      \settowidth{\labelwidth}{99. }
      \setlength{\leftmargin}{\labelwidth+\labelsep}}}
  {\end{list}}
%%-----------------------------------------------------
%% Underlying list environemnt for function definitions
%%-----------------------------------------------------
\newenvironment{Ventry}[1][\quad]
{\begin{list}{}{
      \setlength{\rightmargin}{0em}
      \setlength{\topsep}{0.05in}
      \setlength{\itemsep}{0em}
      \setlength{\itemindent}{0em}
      \setlength{\labelsep}{0.5em}
      \renewcommand{\makelabel}[1]{##1\hfill}
      \settowidth{\labelwidth}{#1}
      \setlength{\leftmargin}{\labelwidth+\labelsep}}}
  {\end{list}}
%%----------------------------------
%% List of function arguments
%%---------------------------------
\newenvironment{args}[1][\quad]
{\begin{list}{}{
      \setlength{\rightmargin}{0em}
      \setlength{\topsep}{0em}
      \setlength{\itemsep}{0em}
      \setlength{\itemindent}{0em}
      \setlength{\labelsep}{0.5em}
      \renewcommand{\makelabel}[1]{\id{##1}\hfill}
      \settowidth{\labelwidth}{\id{#1}}
      \setlength{\leftmargin}{\labelwidth+\labelsep}}}
  {\end{list}}
%%---------------------------------
%% User-callable function
%%---------------------------------
\newcommand{\ucfunction}[6]{
  \noindent\paragraph{\fbox{\id{#1}}}
  \begin{Ventry}[Return value]
  \item[Call]{\id{#2}}
  \item[Description]{#3}
  \item[Arguments]{#4}
  \item[Return value]{#5}
  \addNotes{#6}
  \end{Ventry}
}
%%---------------------------------
%% User-supplied function
%%---------------------------------
\newcommand{\usfunction}[6]{
  \noindent\paragraph{\fbox{\id{#1}}}
  \begin{Ventry}[Return value]
  \item[Definition]{\id{\begin{tabular}[t]{@{}r@{}l@{}}#2\end{tabular}}}
  \item[Purpose]{#3}
  \item[Arguments]{#4}
  \item[Return value]{#5}
  \addNotes{#6}
  \end{Ventry}
}
%%---------------------------------
\makeatletter
\long\def\addNotes#1{\def\@tempa{#1}\ifx\@tempa\empty\else\item[Notes]{#1}\fi}
\makeatother



%%---------------------------------
%% Finally, use hyperref package to include links in the PDF
%% Note that since hyperref redefines a bunch of LaTeX commands,
%% to give it a fighting chance, it MUST be included last.

\usepackage[
letterpaper=true, 
dvips, 
ps2pdf, 
hyperindex=true, 
linktocpage=true,
colorlinks=true, 
linkcolor=blue,
citecolor=blue,
bookmarks=true]
{hyperref}

