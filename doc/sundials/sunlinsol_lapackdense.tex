%% This is a shared SUNDIALS TEX file with a description of the
%% lapackdense sunlinsol implementation
%%

The LAPACK dense implementation of the {\sunlinsol} module provided
with {\sundials}, {\sunlinsollapdense}, is designed to be used with the 
corresponding {\sunmatdense} matrix type, and one of the serial or
shared-memory {\nvector} implementations ({\nvecs}, {\nvecopenmp} or
{\nvecpthreads}).  The {\sunlinsollapdense} module defines the {\em
content} field of a \id{SUNLinearSolver} to be the following structure:
%%
\begin{verbatim} 
struct _SUNLinearSolverContent_Dense {
  sunindextype N;
  sunindextype *pivots;
  long int last_flag;
};
\end{verbatim}
%%
These entries of the \emph{content} field contain the following
information:
\begin{description}
  \item[N] - size of the linear system,
  \item[pivots] - index array for partial pivoting in LU factorization,
  \item[last\_flag] - last error return flag from internal function evaluations.
\end{description}

{\warn} The {\sunlinsollapdense} module is a {\sunlinsol} wrapper for
the LAPACK dense matrix factorization and solve routines, \id{*GETRF}
and \id{*GETRS}, where \id{*} is either \id{D} or \id{S}, depending on
whether {\sundials} was configured to have \id{realtype} set to
\id{double} or \id{single}, respectively (see Section \ref{s:types}).
Therfore in order to use the {\sunlinsollapdense} module it is assumed
that LAPACK has been installed on the system prior to installation of
{\sundials}, and that {\sundials} has been configured appropriately to
link with LAPACK (see Appendix \ref{c:install} for details).  
We note that since there do not exist 128-bit floating-point
factorization and solve routines in LAPACK, this interface cannot be
compiled when using \id{extended} precision for \id{realtype}.
Similarly, since there do not exist 128-bit integer LAPACK routines,
the {\sunlinsollapdense} module also cannot be compiled when using
\id{int64\_t} for the \id{sunindextype}.

This solver is constructed to perform the following operations:
\begin{itemize}
\item In the ``setup'' call, this performs a $LU$ factorization with
  partial (row) pivoting ($\mathcal O(N^3)$ cost), $PA=LU$, where $P$
  is a permutation matrix, $L$ is a lower triangular matrix with 1's
  on the diagonal, and $U$ is an upper triangular matrix.  This
  factorization is stored in-place on the input {\sunmatdense} object
  $A$, with pivoting information encoding $P$ stored in
  the \id{pivots} array.
\item In the ``solve'' call, this performs pivoting, forward and
  backward substitution using the stored \id{pivots} array and the
  $LU$ factors held in the {\sunmatdense} object ($\mathcal O(N^2)$
  cost).
\end{itemize}

\noindent The header file to be included when using this module 
is \id{sunlinsol/sunlinsol\_lapackdense.h}. \\
%%
%%----------------------------------------------
%%
The {\sunlinsollapdense} module defines dense implementations of all
``direct'' linear solver operations listed in
Table \ref{t:sunlinsolops}:
\begin{itemize}
\item \id{SUNLinSolGetType\_LapackDense}
\item \id{SUNLinSolInitialize\_LapackDense} -- this does nothing, since all
  consistency checks were performed at solver creation.
\item \id{SUNLinSolSetup\_LapackDense} -- this calls either
  \id{DGETRF} or \id{SGETRF} to perform the $LU$ factorization.
\item \id{SUNLinSolSolve\_LapackDense} -- this calls either
  \id{DGETRS} or \id{SGETRS} to use the $LU$ factors and \id{pivots}
  array to perform the solve.
\item \id{SUNLinSolLastFlag\_LapackDense}
\item \id{SUNLinSolSpace\_LapackDense} -- this only returns information for
  the storage \emph{within} the solver object, i.e.~storage
  for \id{N}, \id{last\_flag} and \id{pivots}.
\item \id{SUNLinSolFree\_LapackDense}
\end{itemize}
The module {\sunlinsollapdense} provides the following additional
user-callable routine: 
%%
\begin{itemize}

%%--------------------------------------

\item \ID{SUNLapackDense}

  This function creates and allocates memory for a LAPACK dense
  \id{SUNLinearSolver}.  Its arguments are an {\nvector} and
  {\sunmatrix}, that it uses to determine the linear system size and
  to assess compatibility with the linear solver implementation.

  This routine will perform consistency checks to ensure that it is
  called with consistent {\nvector} and {\sunmatrix} implementations.
  These are currently limited to the {\sunmatdense} matrix type, and
  the {\nvecs}, {\nvecopenmp} and {\nvecpthreads} vector types.  As
  additional compatible matrix and vector implementations are added to
  {\sundials}, these will be included within this compatibility check.

  If either \id{A} or \id{y} are incompatible then this routine will
  return \id{NULL}.

  \verb|SUNLinearSolver SUNLapackDense(N_Vector y, SUNMatrix A);|

\end{itemize}
%%
%%------------------------------------
%%
For solvers that include a Fortran interface module, the
{\sunlinsollapdense} module also includes the Fortran-callable
function \id{FSUNLapackDenseInit(code, ier)} to initialize
this {\sunlinsollapdense} module for a given {\sundials} solver.
Here \id{code} is an input solver id (1 for {\cvode}, 2 for {\ida}, 3
for {\kinsol}, 4 for {\arkode}); \id{ier} is an error return flag 
equal 0 for success and -1 for failure (declared so as to match C type
\id{int}).  This routine must be called \emph{after} both the
{\nvector} and {\sunmatrix} objects have been initialized.
Additionally, when using {\arkode} with non-identity mass matrix, the
Fortran-callable function \id{FSUNMassLapackDenseInit(ier)}  
initializes this {\sunlinsollapdense} module for solving mass matrix
linear systems.
