% This is a shared SUNDIALS TEX file with description of
% the generic dense linear solver

\section{The DENSE module}\label{ss:dense}

\index{generic linear solvers!DENSE@{\dense}}
Relative to the {\sundials} {\em source\_tree}, the files comprising the
{\dense} generic linear solver are as follows:
\begin{itemize}
\item header files (located in {\em source\_tree}\id{/shared/include})\\
  \id{sundials\_dense.h} \id{sundials\_smalldense.h} \\
  \id{sundials\_types.h} \id{sundials\_math.h}  \id{sundials\_config.h}
\item source files (located in {\em source\_tree}\id{/shared/source})\\
  \id{sundials\_dense.c} \id{sundials\_smalldense.c} \id{sundials\_math.c}
\end{itemize}
Only two of the preprocessing directives in the header file \id{sundials\_config.h} 
are relevant to the {\dense} package by itself (see \S\ref{ss:no_config} for details):
\begin{itemize}
\item (required) definition of the precision of the {\sundials} type \id{realtype}. 
  One of the following lines must be present:\\
  \id{\#define SUNDIALS\_DOUBLE\_PRECISION 1}\\
  \id{\#define SUNDIALS\_SINGLE\_PRECISION 1}\\
  \id{\#define SUNDIALS\_EXTENDED\_PRECISION 1}
\item (optional) use of generic math functions:
  \id{\#define SUNDIALS\_USE\_GENERIC\_MATH 1}
\end{itemize}
The \id{sundials\_types.h} header file defines the {\sundials}
\id{realtype} and \id{booleantype} types and the macro \id{RCONST}, while the 
\id{sundials\_math.h} header file is needed for the \id{ABS} macro and \id{RAbs} function.

The eight files listed above can be extracted from the {\sundials} {\em source\_tree} and
compiled by themselves into a {\dense} library or into a larger user code.

\subsection{Type DenseMat}
\index{DENSE@{\dense} generic linear solver!type \id{DenseMat}|(}
The type \ID{DenseMat} is defined to be a pointer to a structure    
with a size and a data field:
\begin{verbatim}
typedef struct {
  long int size;
  realtype  **data;
} *DenseMat;
\end{verbatim}
The {\em size} field indicates the number of columns (which is the same as the
number of rows) of a dense matrix, while the {\em data} field is a two 
dimensional array used for component storage. 
The elements of a dense matrix are stored columnwise (i.e columns are stored 
one on top of the other in memory). If \id{A} is of type \id{DenseMat}, 
then the (\id{i},\id{j})-th element of \id{A} 
(with $0 \le$ \id{i}, \id{j} $\le$ \id{size}$-1$) 
is given by the expression \id{(A->data)[j][i]} 
or by the expression \id{(A->data)[0][j*size+i]}. The macros below     
allow a user to efficiently access individual matrix           
elements without writing out explicit data structure           
references and without knowing too much about the underlying   
element storage. The only storage assumption needed is that    
elements are stored columnwise and that a pointer to the \id{j}-th   
column of elements can be obtained via the \id{DENSE\_COL} macro.    
Users should use these macros whenever possible.               
\index{DENSE@{\dense} generic linear solver!type \id{DenseMat}|)}

\subsection{Accessor Macros}
\index{DENSE@{\dense} generic linear solver!macros|(}
The following two macros are defined by the {\dense} module to provide
access to data in the \id{DenseMat} type:
\begin{itemize}
\item \ID{DENSE\_ELEM}
  \par Usage : \id{DENSE\_ELEM(A,i,j) = a\_ij;} or
  \id{a\_ij = DENSE\_ELEM(A,i,j);}
  \par \id{DENSE\_ELEM} references the (\id{i},\id{j})-th element of the $N \times N$
  \id{DenseMat} \id{A}, $0 \le$ \id{i}, \id{j} $\le N-1$.
  
\item \ID{DENSE\_COL}
  \par Usage : \id{col\_j = DENSE\_COL(A,j);}
  \par \id{DENSE\_COL} references the \id{j}-th column of the $N \times N$
  \id{DenseMat} \id{A}, $0 \le$ \id{j} $\le N-1$. The type of the expression          
  \id{DENSE\_COL(A,j)} is \id{realtype *} . After the assignment in the usage    
  above, \id{col\_j} may be treated as an array indexed from $0$ to $N-1$. 
  The (\id{i}, \id{j})-th element of \id{A} is referenced by \id{col\_j[i]}.  
\end{itemize}
\index{DENSE@{\dense} generic linear solver!macros|)}

\subsection{Functions}
\index{DENSE@{\dense} generic linear solver!functions!large matrix|(}
The following functions for \id{DenseMat} matrices are available
in the {\dense} package.  For full details, see the header file \id{sundials\_dense.h}.
\begin{itemize}
\item \id{DenseAllocMat}: allocation of a \id{DenseMat} matrix;
\item \id{DenseAllocPiv}: allocation of a pivot array for use
  with \id{DenseFactor}/\id{DenseBacksolve};
\item \id{DenseFactor}: LU factorization with partial pivoting;
\item \id{DenseBacksolve}: solution of $Ax = b$ using LU factorization;
\item \id{DenseZero}: load a matrix with zeros;
\item \id{DenseCopy}: copy one matrix to another;
\item \id{DenseScale}: scale a matrix by a scalar;
\item \id{DenseAddI}: increment a matrix by the identity matrix;
\item \id{DenseFreeMat}: free memory for a \id{DenseMat} matrix;
\item \id{DenseFreePiv}: free memory for a pivot array;
\item \id{DensePrint}: print a \id{DenseMat} matrix to standard output.
\end{itemize}
\index{DENSE@{\dense} generic linear solver!functions!large matrix|)}

\subsection{Small Dense Matrix Functions}
\index{DENSE@{\dense} generic linear solver!functions!small matrix|(}
The following functions for small dense matrices are available in the
{\dense} package:
%
\begin{itemize}

\item \ID{denalloc}
  \par \id{denalloc(n)} allocates storage for an  \id{n} by \id{n} dense matrix. 
  It returns a pointer to the newly allocated storage if            
  successful. If the memory request cannot be satisfied, then    
  \id{denalloc} returns \id{NULL}. The underlying type of the dense matrix 
  returned is \id{realtype**}. If we allocate a dense matrix \id{realtype** a} by 
  \id{a = denalloc(n)}, then \id{a[j][i]} references the (\id{i},\id{j})-th element   
  of the matrix \id{a}, $0 \le$ \id{i}, \id{j} $\le$ \id{n}$-1$, and \id{a[j]} 
  is a pointer to the first element in the \id{j}-th column of \id{a}. 
  The location \id{a[0]} contains a pointer to \id{n}$^2$ contiguous locations which contain   
  the elements of \id{a}.

\item \ID{denallocpiv}
  \par \id{denallocpiv(n)} allocates an array of \id{n} integers. 
  It returns a pointer to the first element in the array if successful. 
  It returns \id{NULL} if the memory request could not be satisfied.

\item \ID{gefa}
  \par \id{gefa(a,n,p)} factors the \id{n} by \id{n} dense matrix \id{a}. 
  It overwrites the elements of \id{a} with its LU factors and keeps track of the
  pivot rows chosen in the pivot array \id{p}.

  A successful LU factorization leaves the matrix \id{a} and the      
  pivot array \id{p} with the following information:                  
  \begin{enumerate}
  \item 
    \id{p[k]} contains the row number of the pivot element chosen   
    at the beginning of elimination step \id{k}, \id{k} $ = 0, 1, ..., $\id{n}$-1$.  
                                                                 
  \item 
    If the unique LU factorization of \id{a} is given by $Pa = LU$,   
    where $P$ is a permutation matrix, $L$ is a lower triangular   
    matrix with all $1$'s on the diagonal, and $U$ is an upper     
    triangular matrix, then the upper triangular part of \id{a}     
    (including its diagonal) contains $U$ and the strictly lower 
    triangular part of \id{a} contains the multipliers, $I-L$. 
                                                                  
    \id{gefa} returns 0 if successful. Otherwise it encountered a zero  
    diagonal element during the factorization. In this case it     
    returns the column index (numbered from one) at which it       
    encountered the zero.
    \end{enumerate}

\item \ID{gesl}
  \par \id{gesl(a,n,p,b)} solves the \id{n} by \id{n} linear system $ax = b$. 
  It assumes that \id{a} has been LU-factored and the pivot array \id{p} has  
  been set by a successful call to \id{gefa(a,n,p)}. The solution $x$   
  is written into the \id{b} array.

\item \ID{denzero}
  \par \id{denzero(a,n)} sets all the elements of the \id{n} by \id{n} dense matrix
  \id{a} to be $0.0$;

\item \ID{dencopy}
  \par \id{dencopy(a,b,n)} copies the \id{n} by \id{n} dense matrix \id{a} into the
  \id{n} by \id{n} dense matrix \id{b};

\item \ID{denscale}
  \par \id{denscale(c,a,n)} scales every element in the \id{n} by \id{n} dense
  matrix \id{a} by \id{c};

\item \ID{denaddI}
  \par \id{denaddI(a,n)} increments the \id{n} by \id{n} dense matrix \id{a} by the
  identity matrix;

\item \ID{denfreepiv}
  \par \id{denfreepiv(p)} frees the pivot array \id{p} allocated by \id{denallocpiv};

\item \ID{denfree}
  \par \id{denfree(a)} frees the dense matrix \id{a} allocated by \id{denalloc};

\item \ID{denprint}
  \par \id{denprint(a,n)} prints the \id{n} by \id{n} dense matrix \id{a} to standard     
  output as it would normally appear on paper. It is intended as 
  a debugging tool with small values of \id{n}. The elements are      
  printed using the \id{\%g} option. A blank line is printed before    
  and after the matrix. 

\end{itemize}
\index{DENSE@{\dense} generic linear solver!functions!small matrix|)}
