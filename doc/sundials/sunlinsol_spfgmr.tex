%% This is a shared SUNDIALS TEX file with a description of the
%% spfgmr sunlinsol implementation
%%

The {\spfgmr} (Scaled, Preconditioned, Flexible, Generalized Minimum
Residual \cite{Saa:93}) implementation of the {\sunlinsol} module
provided with {\sundials}, {\sunlinsolspfgmr}, is an iterative linear
solver that is designed to be compatible with any {\nvector}
implementation (serial, threaded, parallel, user-supplied) that
supports a minimal subset of operations (\id{N\_VClone},
\id{N\_VDotProd}, \id{N\_VScale}, \id{N\_VLinearSum}, \id{N\_VProd},
\id{N\_VConst}, \id{N\_VDiv} and \id{N\_VDestroy}).  Unlike the other
Krylov iterative linear solvers supplied with {\sundials}, FGMRES is
specifically designed to work with a changing preconditioner
(e.g.~from an iterative method).

The {\sunlinsolspfgmr} module defines the {\em content} field of a
\id{SUNLinearSolver} to be the following structure:
%%
\begin{verbatim} 
struct _SUNLinearSolverContent_SPFGMR {
  int maxl;
  int pretype;
  int gstype;
  int max_restarts;
  int numiters;
  realtype resnorm;
  long int last_flag;
  ATSetupFn ATSetup;
  ATimesFn ATimes;
  void* ATData;
  PSetupFn Psetup;
  PSolveFn Psolve;
  void* PData;
  N_Vector s1;
  N_Vector s2;
  N_Vector *V;
  N_Vector *Z;
  realtype **Hes;
  realtype *givens;
  N_Vector xcor;
  realtype *yg;
  N_Vector vtemp;
};
\end{verbatim}
%%
These entries of the \emph{content} field contain the following
information:
\begin{description}
  \item[maxl] - number of FGMRES basis vectors to use (default is 5)
  \item[pretype] - flag for use of preconditioning (default is none)
  \item[gstype] - flag for type of Gram-Schmidt orthogonalization
    (default is modified Gram-Schmidt)
  \item[max\_restarts] - number of FGMRES restarts to allow
    (default is 0) 
  \item[numiters] - number of iterations from most-recent solve
  \item[resnorm] - final linear residual norm from most-recent solve
  \item[last\_flag] - last error return flag from internal function
  \item[ATSetup] - function pointer to setup routine for \id{ATimes} data
  \item[ATimes] - function pointer to perform $Av$ product
  \item[ATData] - pointer to structure for \id{ATSetup}, \id{ATimes}
  \item[Psetup] - function pointer to preconditioner setup routine
  \item[Psolve] - function pointer to preconditioner solve routine
  \item[PData] - pointer to structure for \id{Psetup}, \id{Psolve}
  \item[s1, s2] - vector pointers for supplied scaling matrices
    (default are \id{NULL})
  \item[V] - the array of Krylov basis vectors
    $v_1, \ldots, v_{\text{\id{maxl}}+1}$, stored in \id{V[0]},
    \ldots, \id{V[maxl]}. Each $v_i$ is a vector of type {\nvector}.
  \item[Z] - the array of preconditioned Krylov basis vectors
    $z_1, \ldots, z_{\text{\id{maxl}}+1}$, stored in \id{Z[0]},
    \ldots, \id{Z[maxl]}. Each $z_i$ is a vector of type {\nvector}.
  \item[Hes] - the $(\text{\id{maxl}}+1)\times\text{\id{maxl}}$
    Hessenberg matrix. It is stored row-wise so that the (i,j)th
    element is given by \id{Hes[i][j]}. 
  \item[givens] - a length \id{2*maxl} array which represents the
    Givens rotation matrices that arise in the FGMRES algorithm. These
    matrices are $F_0, F_1, \ldots, F_j$, where
    $F_i = \begin{bmatrix}
      1 &        &   &     &      &   &        &   \\
        & \ddots &   &     &      &   &        &   \\
        &        & 1 &     &      &   &        &   \\
        &        &   & c_i & -s_i &   &        &   \\
        &        &   & s_i &  c_i &   &        &   \\
        &        &   &     &      & 1 &        &   \\
        &        &   &     &      &   & \ddots &   \\
        &        &   &     &      &   &        & 1\end{bmatrix}$
    are represented in the \id{givens} vector as \id{givens[0] =}
    $c_0$, \id{givens[1] = } $s_0$, \id{givens[2] = } $c_1$,
    \id{givens[3] = } $s_1$, \ldots \id{givens[2j] = } $c_j$,
    \id{givens[2j+1] = } $s_j$.
  \item[xcor] - a vector which holds the scaled, preconditioned
    correction to the initial guess 
  \item[yg] - a length \id{(maxl+1)} array of \id{realtype} values
    used to hold ``short'' vectors (e.g. $y$ and $g$).
  \item[vtemp] - temporary vector storage
\end{description}

This solver is constructed to perform the following operations:
\begin{itemize}
\item During construction, the \id{xcor} and \id{vtemp} arrays are
  cloned from a template {\nvector} that is input, and default solver
  parameters are set.
\item User-facing ``set'' routines may be called to modify default
  solver parameters.
\item Additional ``set'' routines are called by the {\sundials} solver
  that interfaces with {\sunlinsolspfgmr} to supply the \id{ATSetup},
  \id{ATimes}, \id{PSetup} and \id{Psolve} function pointers and
  \id{s1} and \id{s2} scaling vectors.
\item In the ``initialize'' call, the remaining solver data is
  allocated (\id{V}, \id{Hes}, \id{givens}, \id{yg} )
\item In the ``setup'' call, any non-\id{NULL} \id{ATSetup} and
  \id{PSetup} functions are called.  Typically, these are provided by
  the {\sundials} solvers themselves, that translate between the
  generic \id{ATSetup} and \id{PSetup} functions and the
  solver-specific routines (solver-supplied or user-supplied).
\item In the ``solve'' call the FGMRES iteration is performed.  This
  will include scaling, preconditioning and restarts if those options
  have been supplied.
\end{itemize}

\noindent The header file to be included when using this module 
is \id{sunlinsol/sunlinsol\_spfgmr.h}. \\
%%
%%----------------------------------------------
%%
The {\sunlinsolspfgmr} module defines implementations of all
``iterative'' linear solver operations listed in Table
\ref{t:sunlinsolops}:
\begin{itemize}
\item \id{SUNLinSolGetType\_SPFGMR}
\item \id{SUNLinSolInitialize\_SPFGMR}
\item \id{SUNLinSolSetATimes\_SPFGMR}
\item \id{SUNLinSolSetPreconditioner\_SPFGMR}
\item \id{SUNLinSolSetScalingVectors\_SPFGMR}
\item \id{SUNLinSolSetup\_SPFGMR}
\item \id{SUNLinSolSolve\_SPFGMR}
\item \id{SUNLinSolNumIters\_SPFGMR}
\item \id{SUNLinSolResNorm\_SPFGMR}
\item \id{SUNLinSolResid\_SPFGMR}
\item \id{SUNLinSolLastFlag\_SPFGMR}
\item \id{SUNLinSolSpace\_SPFGMR}
\item \id{SUNLinSolFree\_SPFGMR}
\end{itemize}
The module {\sunlinsolspfgmr} provides the following additional
user-callable routines: 
%%
\begin{itemize}

%%--------------------------------------

\item \ID{SUNSPFGMR}

  This function creates and allocates memory for a {\spfgmr}
  \id{SUNLinearSolver}.  Its arguments are an {\nvector}, a flag
  indicating to use preconditioning, and the number of Krylov basis
  vectors to use. 

  This routine will perform consistency checks to ensure that it is
  called with a consistent {\nvector} implementation (i.e.~that it
  supplies the requisite vector operations).  If \id{y} is
  incompatible then this routine will return \id{NULL}.

  A \id{maxl} argument that is $\le0$ will result in the default
  value (5).

  Since the FGMRES algorithm is designed to only support right
  preconditioning, then any of the \id{pretype}
  inputs \id{PREC\_LEFT} (1), \id{PREC\_RIGHT} (2), or \id{PREC\_BOTH}
  (3) will result in use of \id{PREC\_RIGHT};  any other integer input
  will result in the default (no preconditioning).

  \verb|SUNLinearSolver SUNSPFGMR(N_Vector y, int pretype, int maxl);|

%%--------------------------------------

\item \ID{SUNSPFGMRSetPrecType}

  This function updates the flag indicating use of preconditioning.
  Since the FGMRES algorithm is designed to only support right
  preconditioning, then any of the \id{pretype}
  inputs \id{PREC\_LEFT} (1), \id{PREC\_RIGHT} (2), or \id{PREC\_BOTH}
  (3) will result in use of \id{PREC\_RIGHT};  any other integer input
  will result in the default (no preconditioning).

  This routine will return with one of the error codes
  \id{SUNLS\_MEM\_NULL} (\id{S} is \id{NULL}) or \id{SUNLS\_SUCCESS}.
  
  \verb|int SUNSPFGMRSetPrecType(SUNLinearSolver S, int pretype);|

%%--------------------------------------

\item \ID{SUNSPFGMRSetGSType}

  This function sets the type of Gram-Schmidt orthogonalization to
  use.  Supported values are \id{MODIFIED\_GS} (1) and
  \id{CLASSICAL\_GS} (2).  Any other integer input will result in a
  failure, returning error code \id{SUNLS\_ILL\_INPUT}.

  This routine will return with one of the error codes
  \id{SUNLS\_ILL\_INPUT} (illegal \id{gstype}), \id{SUNLS\_MEM\_NULL}
  (\id{S} is \id{NULL}) or \id{SUNLS\_SUCCESS}.
  
  \verb|int SUNSPFGMRSetGSType(SUNLinearSolver S, int gstype);|


%%--------------------------------------

\item \ID{SUNSPFGMRSetMaxRestarts}

  This function sets the number of FGMRES restarts to 
  allow.  A negative input will result in the default of 0.

  This routine will return with one of the error codes
  \id{SUNLS\_MEM\_NULL} (\id{S} is \id{NULL}) or \id{SUNLS\_SUCCESS}.
  
  \verb|int SUNSPFGMRSetMaxRestarts(SUNLinearSolver S, int maxrs);|

\end{itemize}
%%
%%------------------------------------
%%
For solvers that include a Fortran interface module, the
{\sunlinsolspfgmr} module also includes the Fortran-callable
function \id{FSUNSPFGMRInit(code, pretype, maxl, ier)} to initialize
this {\sunlinsolspfgmr} module for a given {\sundials} solver.
Here \id{code} is an input solver id (1 for {\cvode}, 2 for {\ida}, 3
for {\kinsol}, 4 for {\arkode}); \id{pretype} and \id{maxl} are the
same as for the C function \ID{SUNSPFGMR}; \id{ier} is an error return
flag equal 0 for success and -1 for failure.  All of these input
arguments should be declared so as to match C type \id{int}).  This
routine must be called \emph{after} the {\nvector} object has been
initialized.  Additionally, when using {\arkode} with non-identity
mass matrix, the Fortran-callable
function \id{FSUNMassSPFGMRInit(pretype, maxl, ier)} initializes this 
{\sunlinsolspfgmr} module for solving mass matrix linear systems.

The \id{SUNSPFGMRSetPrecType}, \id{SUNSPFGMRSetGSType} and
\id{SUNSPFGMRSetMaxRestarts} routines also support Fortran interfaces
for the system and mass matrix solvers:
\begin{itemize}
\item \id{FSUNSPFGMRSetGSType(code, gstype, ier)} -- all arguments
  should be commensurate with a C \id{int}
\item \id{FSUNMassSPFGMRSetGSType(gstype, ier)}
\item \id{FSUNSPFGMRSetPrecType(code, pretype, ier)} -- all arguments
  should be commensurate with a C \id{int}
\item \id{FSUNMassSPFGMRSetPrecType(pretype, ier)}
\item \id{FSUNSPFGMRSetMaxRS(code, maxrs, ier)} -- all arguments
  should be commensurate with a C \id{int}
\item \id{FSUNMassSPFGMRSetMaxRS(maxrs, ier)}
\end{itemize}
