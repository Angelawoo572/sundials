%% This is a shared SUNDIALS TEX file with a description of the
%% diagonal sunlinsol implementation
%%

The diagonal implementation of the {\sunlinsol} module provided with
{\sundials}, {\sunlinsoldiag}, is designed to be used with the
corresponding {\sunmatdiag} matrix type (and any of the {\nvector}
implementations).  The {\sunlinsoldiag} module defines the
{\em content} field of a \id{SUNLinearSolver} to be the following
structure:
%%
\begin{verbatim} 
struct _SUNLinearSolverContent_Diagonal {
  long int last_flag;
};
\end{verbatim}
%%
The entry of the \emph{content} field consists of only a long integer
containing the last error return flag from {\sunlinsoldiag} functions.

This solver is constructed to perform the following operations:
\begin{itemize}
\item In the ``setup'' call, this inverts a diagonal matrix using the
  {\nvector} routine \id{N\_VInvTest}.  This inverse is stored
  in-place on the input {\sunmatdiag} object $A$.
\item In the ``solve'' call, this performs the diagonal solve using
  the matrix inverse by calling the {\nvector} routine \id{N\_VProd}.
\end{itemize}

\noindent The header file to be included when using this module 
is \id{sunlinsol/sunlinsol\_diagonal.h}. \\
%%
%%----------------------------------------------
%%
The {\sunlinsoldiag} module defines diagonal implementations of all
``direct'' linear solver operations listed in Table \ref{t:sunlinsolops}:
\begin{itemize}
\item \id{SUNLinSolGetType\_Diagonal}
\item \id{SUNLinSolInitialize\_Diagonal} -- this does nothing, since all
  consistency checks were performed at solver creation.
\item \id{SUNLinSolSetup\_Diagonal} -- this inverts the input diagonal
  matrix.
\item \id{SUNLinSolSolve\_Diagonal} -- this multiplies with the input
  diagonal matrix.
\item \id{SUNLinSolLastFlag\_Diagonal}
\item \id{SUNLinSolSpace\_Diagonal} -- this only returns information for
  the storage \emph{within} the solver object, i.e.~storage
  for \id{last\_flag}.
\item \id{SUNLinSolFree\_Diagonal}
\end{itemize}
The module {\sunlinsoldiag} provides the following additional
user-callable routine: 
%%
\begin{itemize}

%%--------------------------------------

\item \ID{SUNDiagonalLinearSolver}

  This function creates and allocates memory for a diagonal
  \id{SUNLinearSolver}.  Its arguments are an {\nvector} and
  {\sunmatrix}, that it uses to assess compatibility with the linear
  solver implementation.

  This routine will perform consistency checks to ensure that it is
  called with consistent {\nvector} and {\sunmatrix} implementations.
  Since a {\sunmatdiag} merely contains a {\nvector}, then this checks
  that both the input {\nvector} and the {\sunmatdiag} {\nvector}
  have the same vector type.  If these do not match, then this routine
  will return \id{NULL}.

  Additionally, this routine checks that both of the requisite
  {\nvector} operations, \id{N\_VInvTest} and \id{N\_VProd}, are
  supplied by the {\nvector} implementation.  If either are missing,
  then this routine will return \id{NULL}.

  \verb|SUNLinearSolver SUNDiagonalLinearSolver(N_Vector y, SUNMatrix A);|

\end{itemize}
%%
%%------------------------------------
%%
For solvers that include a Fortran interface module, the {\sunlinsoldiag}
module also includes the Fortran-callable
function \id{FSUNDiagLinSolInit(code, ier)} to initialize
this {\sunlinsoldiag} module for a given {\sundials} solver.
Here \id{code} is an input solver id (1 for {\cvode}, 2 for {\ida}, 3
for {\kinsol}, 4 for {\arkode}); \id{ier} is an error return flag 
equal 0 for success and -1 for failure (declared so as to match C type
\id{int}).  This routine must be called \emph{after} both the
{\nvector} and {\sunmatrix} objects have been initialized.
Additionally, when using {\arkode} with non-identity mass matrix, the
Fortran-callable function \id{FSUNMassDiagLinSolInit(ier)}  
initializes this {\sunlinsoldiag} module for solving mass matrix
linear systems.
