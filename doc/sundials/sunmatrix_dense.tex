%% This is a shared SUNDIALS TEX file with a description of the
%% dense sunmatrix implementation
%%
\section{The SUNMatrix\_Dense implementation}\label{ss:sunmat_dense}

The dense implementation of the {\sunmatrix} module provided with
{\sundials}, {\sunmatdense}, defines the {\em content} field
of \id{SUNMatrix} to be the following structure:
%%
\begin{verbatim}
struct _SUNMatrixContent_Dense {
  sunindextype M;
  sunindextype N;
  realtype *data;
  sunindextype ldata;
  realtype **cols;
};
\end{verbatim}
%%
These entries of the \emph{content} field contain the following
information:
\begin{args}[ldata]
  \item[M] - number of rows
  \item[N] - number of columns
  \item[data] - pointer to a contiguous block of \id{realtype} variables.
    The elements of the dense matrix are stored columnwise, i.e.~the
    (\id{i},\id{j})-th element of a dense {\sunmatrix} \id{A}
    (with $0 \le$ \id{i} $<$ \id{M} and $ 0 \le$ \id{j} $<$ \id{N})
    may be accessed via \id{data[j*M+i]}.
  \item[ldata] - length of the data array ($=$ \id{M}$\cdot$\id{N}).
  \item[cols] - array of pointers. \id{cols[j]} points to the first
    element of the j-th column of the matrix in the array \id{data}.
    The (\id{i},\id{j})-th element of a dense {\sunmatrix} \id{A}
    (with $0 \le$ \id{i} $<$ \id{M} and $ 0 \le$ \id{j} $<$ \id{N})
    may be accessed via \id{cols[j][i]}.
\end{args}

\noindent The header file to include when using this module
is \id{sunmatrix/sunmatrix\_dense.h}. The {\sunmatdense} module
is accessible from all {\sundials} solvers \textit{without}
linking to the \newline
\id{libsundials\_sunmatrixdense} module library.


% ====================================================================
\subsection{SUNMatrix\_Dense accessor macros}
\label{ss:sunmat_dense_macros}
% ====================================================================

The following macros are provided to access the
content of a {\sunmatdense} matrix. The prefix \id{SM\_} in the names
denotes that these macros are for \emph{SUNMatrix} implementations,
and the suffix \id{\_D} denotes that these are specific to
the \emph{dense} version.
%%
\begin{itemize}

\item \ID{SM\_CONTENT\_D}

  This macro gives access to the contents of the
  dense \id{SUNMatrix}.

  The assignment \id{A\_cont} $=$ \id{SM\_CONTENT\_D(A)} sets
  \id{A\_cont} to be a pointer to the dense \id{SUNMatrix} content
  structure.

  Implementation:

  \verb|#define SM_CONTENT_D(A)     ( (SUNMatrixContent_Dense)(A->content) )|

\item \ID{SM\_ROWS\_D}, \ID{SM\_COLUMNS\_D}, and \ID{SM\_LDATA\_D}

  These macros give individual access to various lengths relevant to the
  content of a dense \id{SUNMatrix}.

  These may be used either to retrieve or to set these values.  For
  example, the assignment \id{A\_rows = SM\_ROWS\_D(A)} sets \id{A\_rows} to be
  the number of rows in the matrix \id{A}.  Similarly, the
  assignment \id{SM\_COLUMNS\_D(A) = A\_cols} sets the number of
  columns in \id{A} to equal \id{A\_cols}.

  Implementation:

  \verb|#define SM_ROWS_D(A)        ( SM_CONTENT_D(A)->M )|

  \verb|#define SM_COLUMNS_D(A)     ( SM_CONTENT_D(A)->N )|

  \verb|#define SM_LDATA_D(A)       ( SM_CONTENT_D(A)->ldata )|

\item \ID{SM\_DATA\_D} and \ID{SM\_COLS\_D}

  These macros give access to the \id{data} and \id{cols} pointers for
  the matrix entries.

  The assignment \id{A\_data = SM\_DATA\_D(A)} sets \id{A\_data} to be
  a pointer to the first component of the data array for the dense
  \id{SUNMatrix} \id{A}.  The assignment \id{SM\_DATA\_D(A) = A\_data}
  sets the data array of \id{A} to be \id{A\_data} by storing the
  pointer \id{A\_data}.

  Similarly, the assignment \id{A\_cols = SM\_COLS\_D(A)} sets \id{A\_cols} to be
  a pointer to the array of column pointers for the dense \id{SUNMatrix} \id{A}.
  The assignment \id{SM\_COLS\_D(A) = A\_cols} sets the column pointer
  array of \id{A} to be \id{A\_cols} by storing the pointer \id{A\_cols}.

  Implementation:

  \verb|#define SM_DATA_D(A)        ( SM_CONTENT_D(A)->data )|

  \verb|#define SM_COLS_D(A)        ( SM_CONTENT_D(A)->cols )|


\item \ID{SM\_COLUMN\_D} and \ID{SM\_ELEMENT\_D}

  These macros give access to the individual columns and entries of
  the data array of a dense \id{SUNMatrix}.

  The assignment \id{col\_j = SM\_COLUMN\_D(A,j)} sets \id{col\_j} to be
  a pointer to the first entry of the \id{j}-th column of the $\id{M} \times \id{N}$
  dense matrix \id{A} (with $0 \le \id{j} < \id{N}$).  The type of the
  expression \id{SM\_COLUMN\_D(A,j)} is \id{realtype *}.  The pointer
  returned by the call \id{SM\_COLUMN\_D(A,j)} can be treated as
  an array which is indexed from $0$ to $\id{M}-1$.

  The assignments \id{SM\_ELEMENT\_D(A,i,j) = a\_ij} and \id{a\_ij =
  SM\_ELEMENT\_D(A,i,j)} reference the (\id{i},\id{j})-th element of the
  $\id{M} \times \id{N}$ dense matrix \id{A} (with $0 \le \id{i} < \id{M}$ and
  $0 \le \id{j} < \id{N}$).

  Implementation:

  \verb|#define SM_COLUMN_D(A,j)    ( (SM_CONTENT_D(A)->cols)[j] )|

  \verb|#define SM_ELEMENT_D(A,i,j) ( (SM_CONTENT_D(A)->cols)[j][i] )|

\end{itemize}


% ====================================================================
\subsection{SUNMatrix\_Dense functions}
\label{ss:sunmat_dense_functions}
% ====================================================================

The {\sunmatdense} module defines dense implementations of all matrix
operations listed in Section \ref{ss:sunmatrix_functions}. Their names are obtained
from those in Section \ref{ss:sunmatrix_functions} by appending the
suffix \id{\_Dense} (e.g. \id{SUNMatCopy\_Dense}).
All the standard matrix operations listed in Section \ref{ss:sunmatrix_functions} with the suffix
\id{\_Dense} appended are callable via the {\F} 2003 interface by prepending an
`F' (e.g. \id{FSUNMatCopy\_Dense}).

The module {\sunmatdense} provides the following additional
user-callable routines:
%%--------------------------------------
\sunmodfunf{SUNDenseMatrix}
{
  This constructor function creates and allocates memory for a dense \id{SUNMatrix}.
  Its arguments are the number of rows, \id{M}, and columns, \id{N}, for
  the dense matrix.
}
{
  SUNMatrix SUNDenseMatrix(sunindextype M, sunindextype N)
}
%%--------------------------------------
\sunmodfun{SUNDenseMatrix\_Print}
{
  This function prints the content of a dense \id{SUNMatrix} to the
  output stream specified by \id{outfile}.  Note: \id{stdout}
  or \id{stderr} may be used as arguments for \id{outfile} to print
  directly to standard output or standard error, respectively.
}
{
  void SUNDenseMatrix\_Print(SUNMatrix A, FILE* outfile)
}
%%--------------------------------------
\sunmodfunf{SUNDenseMatrix\_Rows}
{
  This function returns the number of rows in the dense \id{SUNMatrix}.
}
{
  sunindextype SUNDenseMatrix\_Rows(SUNMatrix A)
}
%%--------------------------------------
\sunmodfunf{SUNDenseMatrix\_Columns}
{
  This function returns the number of columns in the dense \id{SUNMatrix}.
}
{
  sunindextype SUNDenseMatrix\_Columns(SUNMatrix A)
}
%%--------------------------------------
\sunmodfunf{SUNDenseMatrix\_LData}
{
  This function returns the length of the data array for the dense \id{SUNMatrix}.
}
{
  sunindextype SUNDenseMatrix\_LData(SUNMatrix A)
}
%%--------------------------------------
\sunmodfunf{SUNDenseMatrix\_Data}
{
  This function returns a pointer to the data array for the dense \id{SUNMatrix}.
}
{
  realtype* SUNDenseMatrix\_Data(SUNMatrix A)
}
%%--------------------------------------
\sunmodfun{SUNDenseMatrix\_Cols}
{
  This function returns a pointer to the cols array for the dense \id{SUNMatrix}.
}
{
  realtype** SUNDenseMatrix\_Cols(SUNMatrix A)
}
%%--------------------------------------
\sunmodfunf{SUNDenseMatrix\_Column}
{
  This function returns a pointer to the first entry of the jth
  column of the dense \id{SUNMatrix}.  The resulting pointer should
  be indexed over the range $0$ to $\id{M}-1$.
}
{
  realtype* SUNDenseMatrix\_Column(SUNMatrix A, sunindextype j)
}
%%
%%------------------------------------
%%
\paragraph{\bf Notes}

\begin{itemize}

\item
  When looping over the components of a dense \id{SUNMatrix} \id{A},
  the most efficient approaches are to:
  \begin{itemize}
    \item First obtain the component array via \id{A\_data = SM\_DATA\_D(A)} or\\
    \id{A\_data = SUNDenseMatrix\_Data(A)} and then
    access \id{A\_data[i]} within the loop.

    \item First obtain the array of column pointers via \id{A\_cols = SM\_COLS\_D(A)} or\\
    \id{A\_cols = SUNDenseMatrix\_Cols(A)}, and then
    access \id{A\_cols[j][i]} within the loop.

    \item Within a loop over the columns, access the column pointer via\\
    \id{A\_colj = SUNDenseMatrix\_Column(A,j)} and then to access the
    entries within that column using \id{A\_colj[i]} within the loop.
  \end{itemize}
  All three of these are more efficient than
  using \id{SM\_ELEMENT\_D(A,i,j)} within a double loop.

\item
  {\warn} Within the \id{SUNMatMatvec\_Dense} routine, internal
  consistency checks are performed to ensure that the matrix is called
  with consistent {\nvector} implementations.  These are currently
  limited to: {\nvecs}, {\nvecopenmp}, and {\nvecpthreads}.  As additional
  compatible vector implementations are added to {\sundials}, these
  will be included within this compatibility check.

\end{itemize}



% ====================================================================
\subsection{SUNMatrix\_Dense Fortran interfaces}
\label{ss:sunmat_dense_fortran}
% ====================================================================

The {\sunmatdense} module provides a {\F} 2003 module as well as {\F} 77
style interface functions for use from {\F} applications.

\subsubsection*{FORTRAN 2003 interface module}
The \ID{fsunmatrix\_dense\_mod} {\F} module defines interfaces to most
{\sunmatdense} {\CC} functions using the intrinsic \id{iso\_c\_binding}
module which provides a standardized mechanism for interoperating with {\CC}. As
noted in the {\CC} function descriptions above, the interface functions are
named after the corresponding {\CC} function, but with a leading `F'. For
example, the function \id{SUNDenseMatrix} is interfaced as
\id{FSUNDenseMatrix}.

The {\F} 2003 {\sunmatdense} interface module can be accessed with the \id{use}
statement, i.e. \id{use fsunmatrix\_dense\_mod}, and linking to the library
\id{libsundials\_fsunmatrixdense\_mod}.{\em lib} in addition to the {\CC} library.
For details on where the library and module file
\id{fsunmatrix\_dense\_mod.mod} are installed see Appendix \ref{c:install}.
We note that the module is accessible from the {\F} 2003 {\sundials} integrators
\textit{without} separately linking to the
\id{libsundials\_fsunmatrixdense\_mod} library.

\subsubsection*{FORTRAN 77 interface functions}
For solvers that include a {\F} interface module, the {\sunmatdense}
module also includes the {\F}-callable
function \id{FSUNDenseMatInit(code, M, N, ier)} to initialize
this {\sunmatdense} module for a given {\sundials} solver.
Here \id{code} is an integer input solver id (1 for {\cvode}, 2 for {\ida}, 3
for {\kinsol}, 4 for {\arkode}); \id{M} and \id{N} are the
corresponding dense matrix construction arguments (declared to
match C type \id{long int}); and \id{ier} is an error return flag
equal to 0 for success and -1 for failure. Both \id{code} and \id{ier}
are declared to match C type \id{int}. Additionally, when using
{\arkode} with a non-identity mass matrix, the {\F}-callable
function \id{FSUNDenseMassMatInit(M, N, ier)} initializes this
{\sunmatdense} module for storing the mass matrix.
