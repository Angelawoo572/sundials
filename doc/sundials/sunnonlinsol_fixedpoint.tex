This section describes the {\sunnonlinsol} implementation of a fixed point
(functional) iteration with optional Anderson acceleration. To access the
{\sunnonlinsolfixedpoint} module, include the header file
\id{sunnonlinsol/sunnonlinsol\_fixedpoint.h}. We note that the
{\sunnonlinsolfixedpoint} module is accessible from {\sundials} integrators
\textit{without} separately linking to the\\
\noindent\id{libsundials\_sunnonlinsolfixedpoint} module library. 

% ====================================================================
\subsection{SUNNonlinearSolver\_FixedPoint description}
\label{ss:sunnonlinsolfixedpoint_math}
% ====================================================================

To find the solution to
\begin{equation}\label{e:fixed_point_sys}
  G(y) = y \, 
\end{equation}
given an initial guess $y^{(0)}$, the fixed point iteration computes a series of
approximate solutions
\begin{equation}\label{e:fixed_point_iteration}
  y^{(n+1)} = G(y^{(n)})
\end{equation}
where $n$ is the iteration index. The convergence of this iteration may be
accelerated using Anderson's method \cite{Anderson65, Walker-Ni09, Fang-Saad09,
LWWY11}. With Anderson acceleration using subspace size $m$, the series of
approximate solutions can be formulated as the linear combination
\begin{equation}\label{e:accelerated_fixed_point_iteration}
  y^{(n+1)} = \sum_{i=0}^{m_n} \alpha_i^{(n)} G(y^{(n-m_n+i)})
\end{equation}
where $m_n = \min\{m,n\}$ and the factors
\begin{equation}
\alpha^{(n)} =(\alpha_0^{(n)}, \ldots, \alpha_{m_n}^{(n)})
\end{equation}
solve the minimization problem $\min_\alpha  \| F_n \alpha^T \|_2$ under the
constraint that $\sum_{i=0}^{m_n} \alpha_i = 1$ where 
\begin{equation}
F_{n} = (f_{n-m_n}, \ldots, f_{n}) 
\end{equation}
with $f_i = G(y^{(i)}) - y^{(i)}$. Due to this constraint, in the limit of $m=0$
the accelerated fixed point iteration formula
\eqref{e:accelerated_fixed_point_iteration} simplifies to the standard
fixed point iteration \eqref{e:fixed_point_iteration}.

Following the recommendations made in \cite{Walker-Ni09}, the
{\sunnonlinsolfixedpoint} implementation computes the series of approximate
solutions as
\begin{equation}\label{e:accelerated_fixed_point_iteration_impl}
y^{(n+1)} = G(y^{(n)})-\sum_{i=0}^{m_n-1} \gamma_i^{(n)} \Delta g_{n-m_n+i}
\end{equation}
with $\Delta g_i = G(y^{(i+1)}) - G(y^{(i)})$ and where the factors
\begin{equation}
\gamma^{(n)} =(\gamma_0^{(n)}, \ldots, \gamma_{m_n-1}^{(n)})
\end{equation}
solve the unconstrained minimization problem
 $\min_\gamma \| f_n - \Delta F_n \gamma^T \|_2$ where 
\begin{equation}
\Delta F_{n} = (\Delta f_{n-m_n}, \ldots, \Delta f_{n-1}),
\end{equation}
with $\Delta f_i = f_{i+1} - f_i$. The least-squares problem is solved by
applying a QR factorization to $\Delta F_n = Q_n R_n$ and solving
 $R_n \gamma = Q_n^T f_n$.

The acceleration subspace size $m$ is required when constructing the
{\sunnonlinsolfixedpoint} object.  The default maximum number of
iterations and the stopping criteria for the fixed point iteration are
supplied by the {\sundials} integrator when {\sunnonlinsolfixedpoint}
is attached to it.  Both the maximum number of iterations and the
convergence test function may be modified by the user by calling
\id{SUNNonlinSolSetMaxIters} and \id{SUNNonlinSolSetConvTestFn}
functions after attaching the {\sunnonlinsolfixedpoint} object to the
integrator.

% ====================================================================
\subsection{SUNNonlinearSolver\_FixedPoint functions}
\label{ss:sunnonlinsolfixedpoint_functions}
% ====================================================================

The {\sunnonlinsolfixedpoint} module provides the following constructors
for creating a \newline
\id{SUNNonlinearSolver} object.
% --------------------------------------------------------------------
\ucfunctionf{SUNNonlinSol\_FixedPoint}
{
  NLS = SUNNonlinSol\_FixedPoint(y, m);
}
{
  The function \ID{SUNNonlinSol\_FixedPoint} creates a
  \id{SUNNonlinearSolver} object for use with {\sundials} integrators to
  solve nonlinear systems of the form $G(y) = y$.
}
{
  \begin{args}[y]
  \item[y] (\id{N\_Vector})
    a template for cloning vectors needed within the solver
  \item[m] (\id{int})
    the number of acceleration vectors to use
  \end{args}
}
{
  The return value \id{NLS} (of type \id{SUNNonlinearSolver}) will be
  a {\sunnonlinsol} object if the constructor exits successfully,
  otherwise \id{NLS} will be \id{NULL}.
}
{}
% --------------------------------------------------------------------
\ucfunctionf{SUNNonlinSol\_FixedPointSens}
{
  NLS = SUNNonlinSol\_FixedPointSens(count, y, m);
}
{
  The function \ID{SUNNonlinSol\_FixedPointSens} creates a
  \id{SUNNonlinearSolver} object for use with {\sundials} sensitivity enabled
  integrators ({\cvodes} and {\idas}) to solve nonlinear systems of the form
  $G(y) = y$. 
}
{
  \begin{args}[count]
  \item[count] (\id{int})
    the number of vectors in the nonlinear solve. When integrating a system
    containing \id{Ns} sensitivities the value of \id{count} is:
    \begin{itemize}
      \item \id{Ns+1} if using a \textit{simultaneous} corrector approach.
      \item \id{Ns} if using a \textit{staggered} corrector approach.
    \end{itemize}
  \item[y] (\id{N\_Vector})
    a template for cloning vectors needed within the solver.
  \item[m] (\id{int})
    the number of acceleration vectors to use.
  \end{args}
}
{
  The return value \id{NLS} (of type \id{SUNNonlinearSolver}) will be
  a {\sunnonlinsol} object if the constructor exits successfully,
  otherwise \id{NLS} will be \id{NULL}.
}
{}
% --------------------------------------------------------------------
Since the accelerated fixed point iteration
\eqref{e:fixed_point_iteration} does not require the setup or solution
of any linear systems, the {\sunnonlinsolfixedpoint} module implements
all of the functions defined in sections \ref{ss:sunnonlinsol_corefn} --
\ref{ss:sunnonlinsol_getfn} except for the \id{SUNNonlinSolSetup},
\id{SUNNonlinSolSetLSetupFn}, and \\ \noindent
\id{SUNNonlinSolSetLSolveFn} functions, that are set to \id{NULL}.
The {\sunnonlinsolfixedpoint} functions have the same names as those
defined by the generic {\sunnonlinsol} API with \id{\_FixedPoint}
appended to the function name.  Unless using the
{\sunnonlinsolfixedpoint} module as a standalone nonlinear solver the
generic functions defined in sections \ref{ss:sunnonlinsol_corefn} --
\ref{ss:sunnonlinsol_getfn} should be called in favor of the
{\sunnonlinsolfixedpoint}-specific implementations. 

The {\sunnonlinsolfixedpoint} module also defines the following additional
user-callable function.
% --------------------------------------------------------------------
\ucfunctionf{SUNNonlinSolGetSysFn\_FixedPoint}
{
  retval = SUNNonlinSolGetSysFn\_FixedPoint(NLS, SysFn);
}
{
  The function \ID{SUNNonlinSolGetSysFn\_FixedPoint} returns the fixed-point
  function that defines the nonlinear system.
}
{
  \begin{args}[SysFn]
  \item[NLS] (\id{SUNNonlinearSolver})
    a {\sunnonlinsol} object
  \item[SysFn] (\id{SUNNonlinSolSysFn*})
    the function defining the nonlinear system.
  \end{args}
}
{
  The return value \id{retval} (of type \id{int}) should be zero for a
  successful call, and a negative value for a failure.
}
{
  This function is intended for users that wish to evaluate the
  fixed-point function in a custom convergence test function for the
  {\sunnonlinsolfixedpoint} module. We note that {\sunnonlinsolfixedpoint}
  will not leverage the results from any user calls to \id{SysFn}.
}


% ====================================================================
\subsection{SUNNonlinearSolver\_FixedPoint Fortran interfaces}
\label{ss:sunnonlinsolfixedpoint_fortran}
% ====================================================================

The {\sunnonlinsolfixedpoint} module provides a {\F} 2003 module as well as
{\F} 77 style interface functions for use from {\F} applications.

\subsubsection*{FORTRAN 2003 interface module}
The \ID{fsunnonlinsol\_fixedpoint\_mod} {\F} module defines interfaces to all
{\sunnonlinsolfixedpoint} {\CC} functions using the intrinsic \id{iso\_c\_binding}
module which provides a standardized mechanism for interoperating with {\CC}. As
noted in the {\CC} function descriptions above, the interface functions are
named after the corresponding {\CC} function, but with a leading `F'. For
example, the function \id{SUNNonlinSol\_FixedPoint} is interfaced as
\id{FSUNNonlinSol\_FixedPoint}.

The {\F} 2003 {\sunnonlinsolfixedpoint} interface module can be accessed with the \id{use}
statement, i.e. \id{use fsunnonlinsol\_fixedpoint\_mod}, and linking to the library
\id{libsundials\_fsunnonlinsolfixedpoint\_mod}.{\em lib} in addition to the {\CC} library. For details on where the library and module file
\id{fsunnonlinsol\_fixedpoint\_mod.mod} are installed see Appendix \ref{c:install}.
We note that the module is accessible from the {\F} 2003 {\sundials} integrators
\textit{without} separately linking to the
\id{libsundials\_fsunnonlinsolfixedpoint\_mod} library.

\subsubsection*{FORTRAN 77 interface functions}
For {\sundials} integrators that include a {\F} 77 interface, the
{\sunnonlinsolfixedpoint} module also includes a Fortran-callable
function for creating a \id{SUNNonlinearSolver} object.
%
% --------------------------------------------------------------------
%
\ucfunction{FSUNFIXEDPOINTINIT}
{
  FSUNFIXEDPOINTINIT(code, m, ier);
}
{
  The function \ID{FSUNFIXEDPOINTINIT} can be called for Fortran programs
  to create a\\
  \id{SUNNonlinearSolver} object for use with {\sundials}
  integrators to solve nonlinear systems of the form $G(y) = y$.
}
{
  \begin{args}[code]
  \item[code] (\id{int*})
    is an integer input specifying the solver id (1 for {\cvode}, 2
    for {\ida}, and 4 for {\arkode}).
  \item[m] (\id{int*})
    is an integer input specifying the number of acceleration vectors.
  \end{args}
}
{
  \id{ier} is a return completion flag equal to \id{0} for a success
  return and \id{-1} otherwise. See printed message for details in case
  of failure.
}
{}


% ====================================================================
\subsection{SUNNonlinearSolver\_FixedPoint content}
\label{ss:sunnonlinsolfixedpoint_content}
% ====================================================================

The {\sunnonlinsolfixedpoint} module defines the {\textit{content} field of a
\id{SUNNonlinearSolver} as the following structure:
%%
%%
\begin{verbatim}
struct _SUNNonlinearSolverContent_FixedPoint {

  SUNNonlinSolSysFn      Sys;
  SUNNonlinSolConvTestFn CTest;

  int       m;
  int      *imap;
  realtype *R;
  realtype *gamma;
  realtype *cvals;
  N_Vector *df;
  N_Vector *dg;
  N_Vector *q;
  N_Vector *Xvecs;
  N_Vector  yprev;
  N_Vector  gy;
  N_Vector  fold;
  N_Vector  gold;
  N_Vector  delta;
  int       curiter;
  int       maxiters;
  long int  niters;
  long int  nconvfails;
};
\end{verbatim}
%%
%%
The following entries of the \emph{content} field are always
allocated:
\begin{args}[nconvfails]
  \item[Sys]        - function for evaluating the nonlinear system,
  \item[CTest]      - function for checking convergence of the fixed point iteration,
  \item[yprev]      - \id{N\_Vector} used to store previous fixed-point iterate,
  \item[gy]         - \id{N\_Vector} used to store $G(y)$ in fixed-point algorithm,
  \item[delta]      - \id{N\_Vector} used to store difference between successive fixed-point iterates,
  \item[curiter]    - the current number of iterations in the solve attempt,
  \item[maxiters]   - the maximum number of fixed-point iterations allowed in
                      a solve, and
  \item[niters]     - the total number of nonlinear iterations across all
                      solves.
  \item[nconvfails] - the total number of nonlinear convergence failures across
                      all solves.
  \item[m]          - number of acceleration vectors,
\end{args}
\vspace{1em}

\noindent If Anderson acceleration is requested (i.e., $m>0$ in the call to
\ID{SUNNonlinSol\_FixedPoint}), then the following items are also
allocated within the \emph{content} field:
\begin{args}[maxiters]
  \item[imap]  - index array used in acceleration algorithm (length \id{m})
  \item[R]     - small matrix used in acceleration algorithm (length \id{m*m})
  \item[gamma] - small vector used in acceleration algorithm (length \id{m})
  \item[cvals] - small vector used in acceleration algorithm (length \id{m+1})
  \item[df]    - array of \id{N\_Vectors} used in acceleration algorithm (length \id{m})
  \item[dg]    - array of \id{N\_Vectors} used in acceleration algorithm (length \id{m})
  \item[q]     - array of \id{N\_Vectors} used in acceleration algorithm (length \id{m})
  \item[Xvecs] - \id{N\_Vector} pointer array used in acceleration algorithm (length \id{m+1})
  \item[fold]  - \id{N\_Vector} used in acceleration algorithm
  \item[gold]  - \id{N\_Vector} used in acceleration algorithm
\end{args}
