\chapter{SUNDIALS Package Installation Procedure}\label{c:install}

The installation of any {\sundials} package is accomplished by installing the
{\sundials} suite as a whole, according to the instructions that
follow. The same procedure applies whether or not the downloaded
file contains one or all solvers in {\sundials}.

The {\sundials} suite (or individual solvers) are distributed as
compressed archives (\id{.tar.gz}). The name of the distribution
archive is of the form {\em solver}\id{-x.y.z.tar.gz}, where {\em solver}
is one of: \id{sundials}, \id{cvode}, \id{cvodes}, \id{arkode}, \id{ida},
\id{idas}, or \id{kinsol}, and \id{x.y.z} represents the version number
(of the {\sundials} suite or of the individual solver)
.
%%
To begin the installation, first uncompress and expand the sources, by issuing
\begin{verbatim}
   % tar xzf solver-x.y.z.tar.gz
\end{verbatim}
This will extract source files under a directory {\em solver}\id{-x.y.z}.

Starting with version \id{2.6.0} of {\sundials}, CMake is the only supported method
of installation.
The explanations on the installation procedure begins with a few common observations:
%%
\begin{itemize}

\item The remainder of this chapter will follow these conventions:
  \begin{description}
  \item[{\em srcdir}] 
    is the directory {\em solver}\id{-x.y.z} created above; i.e., the 
    directory containing the {\sundials} sources.
  \item[{\em builddir}]
    is the (temporary) directory under which {\sundials} is built.
  \item[{\em instdir}]
    is the directory under which the {\sundials} exported header files
    and libraries will be installed. Typically, header files are exported under a directory
    {\em instdir}\id{/include} while libraries are installed under {\em instdir}\id{/lib},
    with {\em instdir} specified at configuration time.
  \end{description}

\item For {\sundials} CMake-based installation, in-source builds are prohibited; in other words, the
  build directory {\em builddir} can {\bf not} be the same as {\em srcdir}
  and such an attempt will lead to an error. This
  prevents ``polluting'' the source tree and allows efficient builds
  for different configurations and/or options.

\item {\warn}The installation directory {\em instdir} can {\bf not} be the same as
  the source directory {\em srcdir}.

\item By default, only the libraries and header files are exported to the installation
  directory {\em instdir}.  If enabled by the user (with the
  appropriate toggle for CMake), the
  examples distributed with {\sundials} will be built together with
  the solver libraries but the installation step will result in
  exporting (by default in a subdirectory of the installation
  directory) the example sources and sample outputs together with
  automatically generated configuration files that reference the {\em
  installed} {\sundials} headers and libraries.  As such, these
  configuration files for the {\sundials} examples can be used as
  "templates" for your own problems. CMake installs \id{CMakeLists.txt} files and also
  (as an option available only under Unix/Linux) \id{Makefile} files. Note this
  installation approach also allows the option of building the
  {\sundials} examples without having to install them.  (This can be
  used as a sanity check for the freshly built libraries.)

\item Even if generation of shared libraries is enabled, only static libraries
  are created for the FCMIX modules.  (Because of the use of fixed names for
  the Fortran user-provided subroutines, FCMIX shared libraries would result in
  "undefined symbol" errors at link time.)

\end{itemize}


%%===============================================================================
\section{CMake-based installation}\label{s:cmake_inst}
%%===============================================================================

CMake-based installation provides a platform-independent build system. CMake can generate
Unix and Linux Makefiles, as well as KDevelop, Visual Studio, and 
(Apple) XCode project files from the same configuration file.
In addition, CMake also provides a GUI front end and which allows an interactive build and
installation process.

The {\sundials} build process requires CMake version \id{2.8.1} or
higher and a working compiler.  On Unix-like operating systems, it
also requires Make (and \id{curses}, including its development libraries,
for the GUI front end to CMake, \id{ccmake}), while on Windows it
requires Visual Studio.  While many Linux distributions offer CMake,
the version included is probably out of date.  Many new CMake
features have been added recently, and you should download the latest
version from {\tt http://www.cmake.org}.  Build instructions for CMake
(only necessary for Unix-like systems) can be found on the CMake website.
Once CMake is installed, Linux/Unix users will be able to use \id{ccmake},
while Windows users will be able to use \id{CMakeSetup}.

As previously noted, when using CMake to configure, build and install {\sundials}, it is always
required to use a separate build directory. While in-source builds are possible, they are
explicitly prohibited by the {\sundials} CMake scripts (one of the reasons being that, unlike
autotools, CMake does not provide a \id{make distclean} procedure and it is therefore
difficult to clean-up the source tree after an in-source build). By ensuring a separate build
directory, it is an easy task for the user to clean-up all traces of the build by simply removing
the build directory. CMake does generate a \id{make clean} which will remove files generated by the
compiler and linker.

\subsection{Configuring, building, and installing on Unix-like systems}

The default CMake configuration will build all included solvers and
associated examples and will build static and shared libraries. The
{\em installdir} defaults to \id{/usr/local} and can be changed by
setting the \id{CMAKE\_INSTALL\_PREFIX} variable.  Support for FORTRAN
and all other options are disabled.

CMake can be used from the command line with the \id{cmake} command, or from a \id{curses}-based GUI
by using the \id{ccmake} command. Examples for using both methods will be presented.
For the examples shown it is assumed that there is a top level {\sundials} directory
with appropriate source, build and install directories:

\begin{verbatim}
   % mkdir (...)sundials/instdir
   % mkdir (...)sundials/builddir
   % cd (...)sundials/builddir
\end{verbatim}

%%
%% Building with the GUI configuration
%%
\subsubsection*{Building with the GUI}

Using CMake with the GUI follows this general process:  
\begin{itemize}
\item Select and modify values, run configure (\id{c} key)
\item New values are denoted with an asterisk
\item To set a variable, move the cursor to the variable and press enter
  \begin{itemize}
  \item If it is a boolean (ON/OFF) it will toggle the value
  \item If it is string or file, it will allow editing of the string
  \item For file and directories, the \id{<tab>} key can be used to complete 
  \end{itemize}
\item Repeat until all values are set as desired and the generate option is available (\id{g} key)
\item Some variables (advanced variables) are not visible right away
\item To see advanced variables, toggle to advanced mode (\id{t} key)
\item To search for a variable press \id{/} key, and to repeat the search, press the
\id{n} key
\end{itemize}

To build the default configuration using the GUI, from the {\em builddir} enter
the ccmake command and point to the {\em srcdir}:

\begin{verbatim}
    % ccmake ../srcdir
\end{verbatim}

The default configuration screen is shown in Figure
\ref{f:ccmakedefault}. 
\begin{figure}[!ht]
{\centerline{\psfig{figure=ccmakedefault.eps,width=\textwidth}}}
\caption [Initial {\em ccmake} configuration screen]
{Default configuration screen. Note: Initial screen is empty.
To get this default configuration, press 'c' repeatedly (accepting default values denoted with asterisk)
until the 'g' option is available.}
\label{f:ccmakedefault}
\end{figure}

The default {\em instdir} for both {\sundials} and corresponding examples
can be changed by setting the \id{CMAKE\_INSTALL\_PREFIX} and
the \id{EXAMPLES\_INSTALL\_PATH} as shown in figure
\ref{f:ccmakeprefix}. 
\begin{figure}[!ht]
{\centerline{\psfig{figure=ccmakeprefix.eps,width=\textwidth}}}
\caption [Changing the {\em instdir}]
{Changing the {\em instdir} for {\sundials} and
corresponding {\id examples} }
\label{f:ccmakeprefix}
\end{figure}

Pressing the (\id{g} key) will generate makefiles including all dependencies
and all rules to build {\sundials} on this system. 
Back at the command prompt, you can now run:

\begin{verbatim}
  % make
\end{verbatim}

To install {\sundials} in the installation directory specified in the configuration, simply run:

\begin{verbatim}
  % make install
\end{verbatim}

%%
%% *** NOTE: The TestRunner will not be distributed at this time.
%% *** Thus the following is commented out from the documentation.
%% TestRunner
%%
%\subsubsection*{Testing Installation}
%The distribution of {\sundials} includes several examples corresponding to the solvers to be
%installed. Also included in the source bundle is a test script: \id{testRunner}, configured by CMake
%to test the included examples.
%To run the tests, enter:

%\begin{verbatim}
%  % make test
%\end{verbatim}
%The output of \id{testRunner} should look similar to the screens in figure
%\ref{f:testrunner}. The success of each test is based on a line-by-line comparison of expected output files, bundled with the source code, with
%the output of the newly compiled examples. The file compare does allow some differences in rounding for float values.\\\\
%NOTE: Some tests may {\em fail} due to differences in machine architecture, compiler versions, third party libraries etc.{\warn}

%\begin{figure}[!ht]
%{\centerline{\psfig{figure=testrunnertop.eps,width=\textwidth}}}
%\vspace{3 mm}
%{\centerline{\psfig{figure=testrunnerbot.eps,width=\textwidth}}}
%\caption [Running {\em testRunner}]
%{Invoking {\em testRunner} with {\id make test} to execute all configured
%{\id examples} }
%\label{f:testrunner}
%\end{figure}

 
%%
%% Building from the command line
%%
\subsubsection*{Building from the command line}

Using CMake from the command line is simply a matter of specifying CMake variable settings
with the {\id cmake} command.  The following will build the default configuration:  

\begin{verbatim}
   % cmake -DCMAKE_INSTALL_PREFIX=/home/myname/sundials/instdir \
   > -DEXAMPLES_INSTALL_PATH=/home/myname/sundials/instdir/examples \
   > ../srcdir
   % make
   % make install
\end{verbatim}


\subsection{Configuration options (Unix/Linux)}\label{ss:configuration_options_nix}

A complete list of all available options for a CMake-based {\sundials}
configuration is provide below. Note that the default values shown are for 
a typical configuration on a Linux system and are provided as illustration only.

\begin{description}
\item[\id{BUILD\_ARKODE}] - 
  Build the ARKODE library 
  \\
  Default: ON
\item[\id{BUILD\_CVODE}] - 
  Build the CVODE library 
  \\
  Default: ON
\item[\id{BUILD\_CVODES}] - 
  Build the CVODES library 
  \\
  Default: ON
\item[\id{BUILD\_IDA}] - 
   Build the IDA library 
  \\
   Default: ON
\item[\id{BUILD\_IDAS}] - 
  Build the IDAS library 
  \\
  Default: ON
\item[\id{BUILD\_KINSOL}] - 
  Build the KINSOL library 
  \\
  Default: ON
\item[\id{BUILD\_SHARED\_LIBS}] - 
  Build shared libraries
  \\
  Default: OFF 
\item[\id{BUILD\_STATIC\_LIBS}] - 
  Build static libraries
  \\
  Default: ON 
\item[\id{CMAKE\_BUILD\_TYPE}] -  
  Choose the type of build, options are: 
  None (CMAKE\_C\_FLAGS used) Debug Release RelWithDebInfo MinSizeRel
  \\
  Default:
\item[\id{CMAKE\_C\_COMPILER}] - 
  C compiler
  \\
  Default: /usr/bin/cc 
\item[\id{CMAKE\_C\_FLAGS}] -  
  Flags for C compiler
  \\
  Default:
\item[\id{CMAKE\_C\_FLAGS\_DEBUG}] -      
  Flags used by the compiler during debug builds
  \\
  Default: -g 
\item[\id{CMAKE\_C\_FLAGS\_MINSIZEREL}] -  
  Flags used by the compiler during release minsize builds
  \\
  Default: -Os -DNDEBUG 
\item[\id{CMAKE\_C\_FLAGS\_RELEASE}] -    
  Flags used by the compiler during release builds
  \\
  Default: -O3 -DNDEBUG 
\item[\id{CMAKE\_Fortran\_COMPILER}] - 
  Fortran compiler
  \\
  Default: /usr/bin/gfortran
  \\
  Note: Fortran support (and all related options) are triggered only if
  either Fortran-C support is enabled (\id{FCMIX\_ENABLE} is ON) or
  Blas/Lapack support is enabled (\id{LAPACK\_ENABLE} is ON).
\item[\id{CMAKE\_Fortran\_FLAGS}] - 
  Flags for Fortran compiler
  \\
  Default:
\item[\id{CMAKE\_Fortran\_FLAGS\_DEBUG}] - 
  Flags used by the compiler during debug builds
  \\
  Default:
\item[\id{CMAKE\_Fortran\_FLAGS\_MINSIZEREL}] - 
  Flags used by the compiler during release minsize builds
  \\
  Default:
\item[\id{CMAKE\_Fortran\_FLAGS\_RELEASE}] - 
  Flags used by the compiler during release builds
  \\
  Default:
\item[\id{CMAKE\_INSTALL\_PREFIX}] -   
  Install path prefix, prepended onto install directories
  \\
  Default: /usr/local 
  \\
  Note: The user must have write access to the location specified through
  this option. Exported {\sundials} header files and libraries will be 
  installed under subdirectories \id{include} and \id{lib} of 
  \id{CMAKE\_INSTALL\_PREFIX}, respectively.
\item[\id{EXAMPLES\_ENABLE}] -   
  Build the {\sundials} examples
  \\
  Default: ON
  \\
\item[\id{EXAMPLES\_INSTALL}] - 
  Install example files
  \\
  Default: ON
  \\
  Note: This option is triggered only if building example programs
  is enabled (\id{EXAMPLES\_ENABLE} ON). If the user requires
  installation of example programs then the sources and sample output files
  for all {\sundials} modules that are currently enabled will be exported to
  the directory specified by \id{EXAMPLES\_INSTALL\_PATH}. A CMake configuration
  script will also be automatically generated and exported to the same directory.
  Additionally, if the configuration is done under a Unix-like system, makefiles
  for the compilation of the example programs (using the installed {\sundials} libraries)
  will be automatically generated and exported to the directory
  specified by \id{EXAMPLES\_INSTALL\_PATH}.
\item[\id{EXAMPLES\_INSTALL\_PATH}] - 
  Output directory for installing example files
  \\
  Default: /usr/local/examples
  \\
  Note: The actual default value for this option will an \id{examples}
  subdirectory created under \id{CMAKE\_INSTALL\_PREFIX}.
\item[\id{FCMIX\_ENABLE}] - 
  Enable Fortran-C support   
  \\
  Default: OFF 
\item[\id{KLU\_ENABLE}] - 
  Enable KLU support
  \\
  Default: OFF 
\item[\id{LAPACK\_ENABLE}] -  
  Enable Lapack support
  \\
  Default: OFF
  \\
  Note: Setting this option to ON will trigger the two additional options
  see below.
\item[\id{LAPACK\_LIBRARIES}] - 
  Lapack (and Blas) libraries
  \\
  Default: /usr/lib/liblapack.so;/usr/lib/libblas.so
  \\
  Note: CMake will search for these libraries in your \id{LD\_LIBRARY\_PATH} prior
  to searching default system paths.
\item[\id{MPI\_ENABLE}] -  
  Enable MPI support
  \\
  Default: OFF 
  \\
  Note: Setting this option to ON will trigger several additional options
  related to MPI.
\item[\id{MPI\_MPICC}] - 
  \id{mpicc} program
  \\
  Default: 
\item[\id{MPI\_RUN\_COMMAND}] -  
  Specify run command for MPI  
  \\
  Default: mpirun 
  \\
  Note: This can either be set to \id{mpirun} for OpenMPI or \id{srun} if jobs are
  managed by \id{SLURM} - Simple Linux Utility for Resource Management as exists on
  LLNL's high performance computing clusters. 
\item[\id{MPI\_MPIF77}] - 
  \id{mpif77} program
  \\
  Default: 
  \\
  Note: This option is triggered only if using MPI compiler scripts
  (\id{MPI\_USE\_MPISCRIPTS} is ON) and Fortran-C support is enabled
  (\id{FCMIx\_ENABLE} is ON).
\item[\id{OPENMP\_ENABLE}] -  
  Enable OpenMP support
  \\
  Default: OFF 
  \\
  Turn on support for the OpenMP based nvector.
\item[\id{PTHREAD\_ENABLE}] -  
  Enable Pthreads support
  \\
  Default: OFF 
  \\
  Turn on support for the Pthreads based nvector.
\item[\id{SUNDIALS\_PRECISION}] -   
  Precision used in {\sundials}, options are: double, single or extended
  \\
  Default: double 
\item[\id{SUPERLUMT\_ENABLE}] - 
  Enable SUPERLU\_MT support   
  \\
  Default: OFF 
\item[\id{TESTRUNNER}] - 
  Location of \id{testRunner} script   
  \\
  Default: \id{srcdir}/\id{testRunner} 
\item[\id{USE\_GENERIC\_MATH}] -   
  Use generic (stdc) math libraries
  \\
  Default: ON 
\end{description}

%%===============================================================================

\subsection{Configuration examples}

The following examples will help demonstrate usage of the CMake configure options.

\noindent To configure {\sundials} using the default C and Fortran compilers,
and default \id{mpicc} and \id{mpif77} parallel compilers, 
enable compilation of examples, and install libraries, headers, and
example sources under subdirectories of
\id{/home/myname/sundials/}, use:

\begin{verbatim}
   % cmake \
   > -DCMAKE_INSTALL_PREFIX=/home/myname/sundials/instdir \
   > -DEXAMPLES_INSTALL_PATH=/home/myname/sundials/instdir/examples \
   > -DMPI_ENABLE=ON \
   > -DFCMIX_ENABLE=ON \
   > /home/myname/sundials/srcdir
   %
   % make install
   % 
\end{verbatim}

\noindent To disable installation of the examples, use:
\begin{verbatim}
   % cmake \
   > -DCMAKE_INSTALL_PREFIX=/home/myname/sundials/instdir \
   > -DEXAMPLES_INSTALL_PATH=/home/myname/sundials/instdir/examples \
   > -DMPI_ENABLE=ON \
   > -DFCMIX_ENABLE=ON \
   > -DEXAMPLES_INSTALL=OFF \
   > /home/myname/sundials/srcdir
   %
   % make install
   % 
\end{verbatim}

%%===============================================================================
\subsection{Working with external Libraries}

The {\sundials} Suite contains many options to enable implementation flexibility
when developing solutions. The following are some notes addressing specific configurations
when using the supported third party libraries.

\subsubsection*{Building with LAPACK and BLAS}
To enable LAPACK and BLAS libraries, set the \id{LAPACK\_ENABLE} option to \id{ON}.
If the directory containing the LAPACK and BLAS libraries is in the \id{LD\_LIBRARY\_PATH} environment
variable, CMake will set the \id{LAPACK\_LIBRARIES} variable accordingly, otherwise CMake
will attemp to find the LAPACK and BLAS libraries in standard system locations.
To explicitly tell CMake what libraries to use, the \id{LAPACK\_LIBRARIES} varible can be
set to the desired libraries.
\noindent Example: 
\begin{verbatim}
   % cmake \
   > -DCMAKE_INSTALL_PREFIX=/home/myname/sundials/instdir \
   > -DEXAMPLES_INSTALL_PATH=/home/myname/sundials/instdir/examples \
   > -DLAPACK_LIBRARIES=/mypath/lib/liblapack.so;/mypath/lib/libblas.so \
   > /home/myname/sundials/srcdir
   %
   % make install
   % 
\end{verbatim}


\subsubsection*{Building with KLU}
The KLU libraries are part of SuiteSparse, a suite of sparse matrix software,
available from the Texas A\&M University website: {\tt http://faculty.cse.tamu.edu/davis/suitesparse.html}.
{\sundials} has been tested with SuiteSparse version 4.2.1.
To enable KLU, set \id{KLU\_ENABLE} to \id{ON}, set \id{KLU\_INCLUDE\_DIR} to the \id{include}
path of the KLU installation and set \id{KLU\_LIBRARY\_DIR} to the \id{lib} path of the KLU installation.
The CMake configure will result in populating the following variables: \id{AMD\_LIBRARY},
\id{AMD\_LIBRARY\_DIR}, \id{BTF\_LIBRARY}, \id{BTF\_LIBRARY\_DIR},
\id{COLAMD\_LIBRARY}, \id{COLAMD\_LIBRARY\_DIR}, and
\newline\id{KLU\_LIBRARY}

\subsubsection*{Building with SuperLU\_MT}
The SuperLU\_MT libraries are available for download from the Lawrence Berkeley National Laboratory website:
{\tt http://crd-legacy.lbl.gov/$\sim$xiaoye/SuperLU/\#superlu\_mt}. 
{\sundials} has been tested with SuperLU\_MT version 2.4.
To enable SuperLU\_MT, set  \id{SUPERLUMT\_ENABLE} to \id{ON}, set \id{SUPERLUMT\_INCLUDE\_DIR}
to the \id{SRC} path of the SuperLU\_MT installation, and set the variable
\newline\id{SUPERLUMT\_LIBRARY\_DIR} to the \id{lib} path of the SuperLU\_MT installation.
At the same time, the variable
\id{SUPERLUMT\_THREAD\_TYPE} must be set to either \id{Pthread} or \id{OpenMP}.

\noindent Do not mix thread types when building {\sundials} solvers.
If threading is enabled for {\sundials} by having either \id{OPENMP\_ENABLE} or \id{PTHREAD\_ENABLE} set to \id{ON}
then SuperLU\_MT should be set to use the same threading type.{\warn}

%%===============================================================================
\section{Building and Running Examples}
%%===============================================================================
Each of the {\sundials} solvers is distributed with a set of examples demonstrating basic usage.
To build and install the examples, set both \id{EXAMPLES\_ENABLE} and \id{EXAMPLES\_INSTALL} to \id{ON}.
Specify the installation path for the examples with the variable \id{EXAMPLES\_INSTALL\_PATH}. CMake will generate
\id{CMakeLists.txt} configuration files (and \id{Makefile} files if on Linux/Unix) that reference the
{\em installed} {\sundials} headers and libraries.

\noindent Either the \id{CMakeLists.txt} file or the traditional \id{Makefile} may be used to build the examples
as well as serve as a template for creating user developed solutions.
To use the supplied \id{Makefile} simply run \id{make} to compile and generate the executables.
To use CMake, from within the installed example directory, run \id{cmake} (or \id{ccmake} to use the GUI)
followed by \id{make} to compile the example code.
Note that if CMake is used, it will overwrite the traditional \id{Makefile} with a new CMake generated \id{Makefile}.
\noindent The resulting output from running the examples can be compared with example output bundled
in the {\sundials} distribution.

\noindent NOTE: There will potentially be differences in the output due to machine architecture, compiler versions,
use of third party libraries etc.{\warn} 


%%===============================================================================
\section{Configuring, building, and installing  on Windows}\label{s:cmake_windows}
%%===============================================================================
Use \id{CMakeSetup} from the CMake install location.
Make sure to select the appropriate source and the build directory.
Also, make sure to pick the appropriate generator (on Visual Studio 6, 
pick the Visual Studio 6 generator). Some CMake versions will ask you 
to select the generator the first time you press Configure instead of 
having a drop-down menu in the main dialog. 

CMake will now create Visual Studio project files. You should now be able to 
open the {\sundials} project (or workspace) file. Make sure to select the appropriate 
build type (Debug, Release, ...). To build {\sundials}, simply build the \id{ALL\_BUILD}
target. To install {\sundials}, simply run the \id{INSTALL} target within the build system.


%%===============================================================================
\section{Installed libraries and exported header files}
%%===============================================================================

Using the CMake {\sundials} build system, the command
\begin{verbatim}
   % make install
\end{verbatim}
will install the libraries under {\em libdir} and the public header
files under {\em includedir}. The values for these directories are
{\em instdir}\id{/lib} and {\em instdir}\id{/include},
respectively. The location can be changed by setting the CMake variable \id{CMAKE\_INSTALL\_PREFIX}.
Although all installed libraries reside under {\em libdir}\id{/lib}, the public header files
are further organized into subdirectories under {\em includedir}\id{/include}.

The installed libraries and exported header files are listed for
reference in Tables \ref{t:sundials_files} and \ref{t:sundials_files2}.
The file extension .{\em lib}
is typically \id{.so} for shared libraries and \id{.a} for static libraries.
Note that, in the Tables, names are relative to {\em libdir}
for libraries and to {\em includedir} for header files.

A typical user program need not explicitly include any of the shared
{\sundials} header files from under the {\em includedir}\id{/include}\id{/sundials}
directory since they are explicitly included by the appropriate solver
header files ({\em e.g.}, \id{cvode\_dense.h} includes
\id{sundials\_dense.h}). However, it is both legal and safe to do so,
and would be useful, for example, if the functions declared in \id{sundials\_dense.h} 
are to be used in building a preconditioner.

\begin{table}
\centering
\caption{{\sundials} libraries and header files}\label{t:sundials_files}
\medskip
\begin{tabular}{|l|l|ll|} 
\hline %% --------------------------------------------------
{\shared} & Libraries    & n/a                               & \\ 
\cline{2-4}
          & Header files & sundials/sundials\_config.h       & sundials/sundials\_types.h    \\
          &              & sundials/sundials\_math.h         & \\
          &              & sundials/sundials\_nvector.h      & sundials/sundials\_fnvector.h \\
          &              & sundials/sundials\_direct.h       & sundials/sundials\_lapack.h   \\
          &              & sundials/sundials\_dense.h        & sundials/sundials\_band.h     \\
          &              & sundials/sundials\_sparse.h       & \\
          &              & sundials/sundials\_iterative.h    & sundials/sundials\_spgmr.h    \\
          &              & sundials/sundials\_spbcgs.h       & sundials/sundials\_sptfqmr.h  \\
          &              & sundials/sundials\_pcg.h          & sundials/sundials\_spfgmr.h   \\
\hline %% --------------------------------------------------
{\nvecs}  & Libraries    & libsundials\_nvecserial.{\em lib} & libsundials\_fnvecserial.a    \\ 
\cline{2-4}
          & Header files & nvector/nvector\_serial.h         & \\ 
\hline %% --------------------------------------------------
{\nvecp}  & Libraries    & libsundials\_nvecparallel.{\em lib} & libsundials\_fnvecparallel.a \\
\cline{2-4}
          & Header files & nvector/nvector\_parallel.h       & \\ 
\hline %% --------------------------------------------------
{\nvecopenmp}  & Libraries    & libsundials\_nvecopenmp.{\em lib} & libsundials\_fnvecopenmp.a \\ 
\cline{2-4}
          & Header files & nvector/nvector\_openmp.h         & \\ 
\hline %% --------------------------------------------------
{\nvecpthreads}  & Libraries    & libsundials\_nvecpthreads.{\em lib} & libsundials\_fnvecpthreads.a \\ 
\cline{2-4}
          & Header files & nvector/nvector\_pthreads.h         & \\ 
\hline %% --------------------------------------------------
{\cvode}  & Libraries    & libsundials\_cvode.{\em lib}      & libsundials\_fcvode.a \\
\cline{2-4}
          & Header files & cvode/cvode.h                     & cvode/cvode\_impl.h   \\
          &              & cvode/cvode\_direct.h             & cvode/cvode\_lapack.h \\
          &              & cvode/cvode\_dense.h              & cvode/cvode\_band.h   \\
          &              & cvode/cvode\_diag.h               & \\
          &              & cvode/cvode\_sparse.h             & cvode/cvode\_klu.h    \\
          &              & cvode/cvode\_superlumt.h          & \\
          &              & cvode/cvode\_spils.h              & cvode/cvode\_spgmr.h  \\
          &              & cvode/cvode\_sptfqmr.h            & cvode/cvode\_spbcgs.h \\
          &              & cvode/cvode\_bandpre.h            & cvode/cvode\_bbdpre.h \\
\hline %% --------------------------------------------------
{\cvodes} & Libraries    & libsundials\_cvodes.{\em lib}     & \\
\cline{2-4}
          & Header files & cvodes/cvodes.h                     & cvodes/cvodes\_impl.h   \\
          &              & cvodes/cvodes\_direct.h             & cvodes/cvodes\_lapack.h \\          
          &              & cvodes/cvodes\_dense.h              & cvodes/cvodes\_band.h   \\
          &              & cvodes/cvodes\_diag.h               & \\
          &              & cvodes/cvodes\_sparse.h             & cvodes/cvodes\_klu.h    \\
          &              & cvodes/cvodes\_superlumt.h          & \\
          &              & cvodes/cvodes\_spils.h              & cvodes/cvodes\_spgmr.h  \\
          &              & cvodes/cvodes\_sptfqmr.h            & cvodes/cvodes\_spbcgs.h \\
          &              & cvodes/cvodes\_bandpre.h            & cvodes/cvodes\_bbdpre.h \\
\hline %% --------------------------------------------------
{\arkode} & Libraries    & libsundials\_arkode.{\em lib}  & libsundials\_farkode.a \\
\cline{2-4}
          & Header files & arkode/arkode.h                     & arkode/arkode\_impl.h   \\
          &              & arkode/arkode\_direct.h             & arkode/arkode\_lapack.h \\
          &              & arkode/arkode\_dense.h              & arkode/arkode\_band.h   \\
          &              & arkode/arkode\_sparse.h             & arkode/arkode\_klu.h    \\
          &              & arkode/arkode\_superlumt.h          & \\
          &              & arkode/arkode\_spils.h              & arkode/arkode\_spgmr.h  \\
          &              & arkode/arkode\_sptfqmr.h            & arkode/arkode\_spbcgs.h \\
          &              & arkode/arkode\_pcg.h                & arkode/arkode\_spfgmr.h \\
          &              & arkode/arkode\_bandpre.h            & arkode/arkode\_bbdpre.h \\
\hline %% --------------------------------------------------
\end{tabular}
\end{table}


\begin{table}
\centering
\caption{{\sundials} libraries and header files (cont.)}\label{t:sundials_files2}
\medskip
\begin{tabular}{|l|l|ll|} 
\hline %% --------------------------------------------------
{\ida}    & Libraries    & libsundials\_ida.{\em lib}        & libsundials\_fida.a \\
\cline{2-4}
          & Header files & ida/ida.h                         & ida/ida\_impl.h     \\
          &              & ida/ida\_direct.h                 & ida/ida\_lapack.h   \\
          &              & ida/ida\_dense.h                  & ida/ida\_band.h     \\
          &              & ida/ida\_sparse.h                 & ida/ida\_klu.h      \\
          &              & ida/ida\_superlumt.h              & \\
          &              & ida/ida\_spils.h                  & ida/ida\_spgmr.h    \\
          &              & ida/ida\_spbcgs.h                 & ida/ida\_sptfqmr.h  \\
          &              & ida/ida\_bbdpre.h                 & \\
\hline %% --------------------------------------------------
{\idas}    & Libraries    & libsundials\_idas.{\em lib}      & \\
\cline{2-4}
          & Header files & idas/idas.h                         & idas/idas\_impl.h     \\
          &              & idas/idas\_direct.h                 & idas/idas\_lapack.h   \\
          &              & idas/idas\_dense.h                  & idas/idas\_band.h     \\
          &              & idas/idas\_sparse.h                 & idas/idas\_klu.h      \\
          &              & idas/idas\_superlumt.h              & \\
          &              & idas/idas\_spils.h                  & idas/idas\_spgmr.h    \\
          &              & idas/idas\_spbcgs.h                 & idas/idas\_sptfqmr.h  \\
          &              & idas/idas\_bbdpre.h                 & \\
\hline %% --------------------------------------------------
{\kinsol} & Libraries    & libsundials\_kinsol.{\em lib}     & libsundials\_fkinsol.a \\
\cline{2-4}
          & Header files & kinsol/kinsol.h                         & kinsol/kinsol\_impl.h     \\
          &              & kinsol/kinsol\_direct.h                 & kinsol/kinsol\_lapack.h   \\
          &              & kinsol/kinsol\_dense.h                  & kinsol/kinsol\_band.h     \\
          &              & kinsol/kinsol\_sparse.h                 & kinsol/kinsol\_klu.h      \\
          &              & kinsol/kinsol\_superlumt.h              & \\
          &              & kinsol/kinsol\_spils.h                  & kinsol/kinsol\_spgmr.h    \\
          &              & kinsol/kinsol\_spbcgs.h                 & kinsol/kinsol\_sptfqmr.h  \\
          &              & kinsol/kinsol\_bbdpre.h                 & kinsol/kinsol\_spfgmr.h   \\
\hline %% --------------------------------------------------
\end{tabular}
\end{table}
