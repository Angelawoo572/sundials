%===================================================================================
\chapter{Description of the SUNLinearSolver module}\label{s:sunlinsol}
%===================================================================================
\index{SUNLinearSolver@\texttt{SUNLinearSolver} module}
% This is a shared SUNDIALS TEX file with description of
% the generic sunlinsol abstraction
%
For problems that involve the solution of linear systems of equations,
the {\sundials} packages operate using generic linear solver modules
defined through the {\sunlinsol} API.  This allows {\sundials}
packages to utilize any valid {\sunlinsol} implementation that provides
a set of required functions.  These functions can be divided into
three categories.  The first are the core linear solver functions.  The
second group consists of ``set'' routines to supply the linear solver object
with functions provided by the {\sundials} package, or for modification
of solver parameters.  The last group consists of ``get'' routines for
retrieving artifacts (statistics, residual vectors, etc.) from the
linear solver.  All of these functions are defined in the header file
\Id{sundials/sundials\_linearsolver.h}.

The implementations provided with {\sundials} work in coordination
with the {\sundials} generic {\nvector} and {\sunmatrix} modules to
provide a set of compatible data structures and solvers for the
solution of linear systems using direct or matrix-free iterative
methods. Moreover, advanced users can provide a customized
\Id{SUNLineaerSolver} implementation to any {\sundials} package,
particularly in cases where they provide their own {\nvector} and/or
{\sunmatrix} modules.

Historically, the {\sundials} packages have been designed to specifically
leverage the use of either \emph{direct linear solvers} or matrix-free,
\emph{scaled, preconditioned, iterative linear solvers}.  However,
matrix-based iterative linear solvers are supported.

The iterative linear solvers packaged with {\sundials} leverage
scaling and preconditioning, as applicable, to balance error between
solution components and to accelerate convergence of the linear
solver.  To this end, instead of solving the  linear system $Ax = b$
directly, these apply the underlying iterative algorithm to the
transformed system
\begin{equation}
  \label{eq:transformed_linear_system}
  \tilde{A} \tilde{x} = \tilde{b}
\end{equation}
where
\begin{align}
  \notag
  \tilde{A} &= S_1 P_1^{-1} A P_2^{-1} S_2^{-1},\\
  \label{eq:transformed_linear_system_components}
  \tilde{b} &= S_1 P_1^{-1} b,\\
  \notag
  \tilde{x} &= S_2 P_2 x,
\end{align}
and where
\begin{itemize}
\item $P_1$ is the left preconditioner,
\item $P_2$ is the right preconditioner,
\item $S_1$ is a diagonal matrix of scale factors for $P_1^{-1} b$,
\item $S_2$ is a diagonal matrix of scale factors for $P_2 x$.
\end{itemize}
{\sundials} packages request that iterative linear solvers stop
based on the 2-norm of the scaled preconditioned residual meeting a
prescribed tolerance
\[
  \left\| \tilde{b} - \tilde{A} \tilde{x} \right\|_2  <  \text{tol}.
\]

When provided an iterative {\sunlinsol} implementation that does not
support the scaling matrices $S_1$ and $S_2$, {\sundials}'
packages will adjust the value of $\text{tol}$ accordingly.  In
this case, they instead request that iterative linear solvers stop
based on the criteria
\[
   \left\| P_1^{-1} b - P_1^{-1} A x \right\|_2  <  \text{tol}.
\]
We note that the corresponding adjustments to $\text{tol}$ in
this case are non-optimal, in that they cannot balance error between
specific entries of the solution $x$, only the aggregate error
in the overall solution vector.

We further note that not all of the {\sundials}-provided iterative
linear solvers support the full range of the above options (e.g.,
separate left/right preconditioning), and that some of the {\sundials}
packages only utilize a subset of these options.  Further details on
these exceptions are described in the documentation for each
{\sunlinsol} implementation, or for each {\sundials} package.

%---------------------------------------------------------------------------
\subsection{\id{SUNLinearSolver} core functions}\label{ss:sunlinsol_CoreFn}

The core linear solver functions consist of four required routines to get
the linear solver type \\ \noindent (\Id{SUNLinSolGetType}), initialize
the linear solver object once all solver-specific options have been
set (\Id{SUNLinSolInitialize}), set up the linear solver object
to utilize an updated matrix $A$ \\ \noindent (\Id{SUNLinSolSetup}),
and solve the linear system $Ax=b$ (\Id{SUNLinSolSolve}).
The remaining routine for destruction of the linear solver object
(\Id{SUNLinSolFree}) is optional.

% --------------------------------------------------------------------
\ucfunction{SUNLinSolGetType}
{
  type = SUNLinSolGetType(LS);
}
{
  The \textit{required} function \Id{SUNLinSolGetType} returns the
  type identifier for the linear solver \id{LS}. It is used to
  determine the solver type (direct or iterative) from
  the abstract \id{SUNLinearSolver} interface.  This is used to assess
  compatibility with {\sundials}-provided linear solver interfaces.
}
{
  \begin{args}[LS]
  \item[LS] (\id{SUNLinearSolver})
    a {\sunlinsol} object.
  \end{args}
}
{
  The return value \id{type} (of type \id{int}) will be one of the
  following:
  \begin{args}[SUNLINEARSOLVER\_ITERATIVE]
  \item[\Id{SUNNONLINEARSOLVER\_DIRECT}]
    \id{0}, the {\sunlinsol} module uses direct methods to solve the
    linear system.
  \item[\Id{SUNNONLINEARSOLVER\_ITERATIVE}]
    \id{1}, the {\sunlinsol} module iteratively solves the linear
    system, stopping when the linear residual is within a prescribed
    tolerance.
  \end{args}
}
{
}
% --------------------------------------------------------------------
\ucfunction{SUNLinSolInitialize}
{
  retval = SUNLinSolInitialize(LS);
}
{
  The \textit{required} function \Id{SUNLinSolInitialize} performs
  linear solver initialization (assumes that all solver-specific
  options have been set).
}
{
  \begin{args}[LS]
  \item[LS] (\id{SUNLinearSolver})
    a {\sunlinsol} object.
  \end{args}
}
{
  This should return zero for a
  successful call, and a negative value for a failure, ideally
  returning one of the generic error codes listed in Table
  \ref{t:sunlinsolerr}.
}
{
}
% --------------------------------------------------------------------
\ucfunction{SUNLinSolSetup}
{
  retval = SUNLinSolSetup(LS, A);
}
{
  The \textit{required} function \Id{SUNLinSolSetup} performs
  any linear solver setup needed, based on an updated system
  {\sunmatrix} \id{A}.  This may be called frequently (e.g. with a full
  Newton method) or infrequently (for a modified Newton method), based
  on the type of integrator and/or nonlinear solver requesting the
  solves.
}
{
  \begin{args}[LS]
  \item[LS] (\id{SUNLinearSolver})
    a {\sunlinsol} object.
  \item[A] (\id{SUNMatrix})
    a {\sunmatrix} object.
  \end{args}
}
{
  This should return zero for a successful call, a positive
  value for a recoverable failure and a negative value for an
  unrecoverable failure, ideally returning one of the generic error
  codes listed in Table \ref{t:sunlinsolerr}.
}
{
}
% --------------------------------------------------------------------
\ucfunction{SUNLinSolSolve}
{
  retval = SUNLinSolSolve(LS, A, x, b, tol);
}
{
  The \textit{required} function \Id{SUNLinSolSolve} solves a linear system $Ax = b$.
}
{
  \begin{args}[LS]
  \item[LS] (\id{SUNLinearSolver})
    a {\sunlinsol} object.
  \item[A] (\id{SUNMatrix})
    a {\sunmatrix} object.
  \item[x] (\id{N\_Vector})
    a {\nvector} object.
  \item[b] (\id{N\_Vector})
    a {\nvector} object.
  \item[tol] (\id{realtype})
    the desired linear solver tolerance.
  \end{args}
}
{
  This should return zero for
  a successful call, a positive value for a recoverable failure and a
  negative value for an unrecoverable failure, ideally returning one
  of the generic error codes listed in Table \ref{t:sunlinsolerr}.
}
{
  {\bf Direct solvers:} can ignore the  *tol* argument.

  {\bf Matrix-free solvers:} can ignore the {\sunmatrix} input \id{A}
  since a \id{NULL} argument will be passed (these should instead rely
  on the matrix-vector product function supplied through the routine
  \id{SUNLinSolSetATimes}.

  {\bf Iterative solvers:} These should attempt to solve to the
  specified tolerance \id{tol} in a weighted 2-norm.  If the solver
  does not support scaling then it should just use a 2-norm.
}
% --------------------------------------------------------------------
\ucfunction{SUNLinSolFree}
{
  retval = SUNLinSolFree(LS);
}
{
  The \textit{optional} function \Id{SUNLinSolFree} frees memory allocated by the linear solver.
}
{
  \begin{args}[LS]
  \item[LS] (\id{SUNLinearSolver})
    a {\sunlinsol} object.
  \end{args}
}
{
  This should return zero for a successful call and a negative value
  for a failure.
}
{
}
% --------------------------------------------------------------------


%---------------------------------------------------------------------------
\subsection{\id{SUNLinearSolver} set functions}\label{ss:sunlinsol_SetFn}

The following set functions are used to supply linear solver modules with
functions defined by the {\sundials} packages and to modify solver
parameters.  Only the routine for setting the matrix-vector product
routine is required, and that is only for matrix-free linear solver
modules.  Otherwise, all other set functions are optional.  {\sunlinsol}
implementations that do not provide the functionality for any optional
routine should leave the corresponding function pointer \id{NULL}
instead of supplying a dummy routine.

% --------------------------------------------------------------------
\ucfunction{SUNLinSolSetATimes}
{
  retval = SUNLinSolSetATimes(LS, A\_data, ATimes);
}
{
  The function \Id{SUNLinSolSetATimes} is required for matrix-free
  linear solvers; otherwise it is optional.

  This routine provides an \id{ATimesFn} function pointer, as well as
  a \id{void *} pointer to a data structure used by this routine, to a
  linear solver object.  {\sundials} packages will call this function
  to set the matrix-vector product function to either a
  solver-provided difference-quotient via vector operations or a
  user-supplied solver-specific routine.

}
{
  \begin{args}[ATimes]
  \item[LS] (\id{SUNLinearSolver})
    a {\sunlinsol} object.
  \item[A\_data] (\id{void*})
    data structure passed to \id{ATimes}.
  \item[ATimes] (\id{ATimesFn})
    function pointer implementing the matrix-vector product routine.
  \end{args}
}
{
  This routine should return zero for a successful call, and a
  negative value for a failure, ideally returning one of the generic
  error codes listed in Table \ref{t:sunlinsolerr}.
}
{
}
% --------------------------------------------------------------------
\ucfunction{SUNLinSolSetPreconditioner}
{
  retval = SUNLinSolSetPreconditioner(LS, Pdata, Pset, Psol);
}
{
  The \emph{optional} function \Id{SUNLinSolSetPreconditioner}
  provides \id{PSetupFn} and \id{PSolveFn} function pointers that
  implement the preconditioner solves $P_1^{-1}$ and $P_2^{-1}$ from
  equations
  \eqref{eq:transformed_linear_system}-\eqref{eq:transformed_linear_system_components}.
  This routine will be called by a {\sundials} package, which will
  provide translation between the generic \id{Pset} and \id{Psol}
  calls and the package- or user-supplied routines.
}
{
  \begin{args}[Pdata]
  \item[LS] (\id{SUNLinearSolver})
    a {\sunlinsol} object.
  \item[Pdata] (\id{void*})
    data structure passed to both \id{Pset} and \id{Psol}.
  \item[Pset] (\id{PSetupFn})
    function pointer implementing the preconditioner setup.
  \item[Psol] (\id{PSolveFn})
    function pointer implementing the preconditioner solve.
  \end{args}
}
{
  This routine should return zero for a successful call, and a
  negative value for a failure, ideally returning one of the generic
  error codes listed in Table \ref{t:sunlinsolerr}.
}
{
}
% --------------------------------------------------------------------
\ucfunction{SUNLinSolSetScalingVectors}
{
  retval = SUNLinSolSetScalingVectors(LS, s1, s2);
}
{
  The \emph{optional} function \Id{SUNLinSolSetScalingVectors}
  provides left/right scaling vectors for the linear system
  solve.  Here, \id{s1} and \id{s2} are {\nvector} of positive scale factors
  containing the diagonal of the matrices $S_1$ and $S_2$ from
  equations
  \eqref{eq:transformed_linear_system}-\eqref{eq:transformed_linear_system_components},
  respectively.
  Neither of these vectors need to be tested for positivity, and a \id{NULL}
  argument for either indicates that the corresponding scaling matrix
  is the identity.
}
{
  \begin{args}[LS]
  \item[LS] (\id{SUNLinearSolver})
    a {\sunlinsol} object.
  \item[s1] (\id{N\_Vector})
    diagonal of the matrix $S_1$
  \item[s2] (\id{N\_Vector})
    diagonal of the matrix $S_2$
  \end{args}
}
{
  This routine should return zero for a successful call, and a
  negative value for a failure, ideally returning one of the generic
  error codes listed in Table \ref{t:sunlinsolerr}.
}
{
}
% --------------------------------------------------------------------


%---------------------------------------------------------------------------
\subsection{\id{SUNLinearSolver} get functions}\label{ss:sunlinsol_GetFn}

The following get functions allow SUNDIALS packages to retrieve
results from the linear solve.  All routines are optional.

% --------------------------------------------------------------------
\ucfunction{SUNLinSolNumIters}
{
  its = SUNLinSolNumIters(LS);
}
{
  The \emph{optional} function \Id{SUNLinSolNumIters}
  should return the number of linear iterations performed in
  the last `solve' call.
}
{
  \begin{args}[LS]
  \item[LS] (\id{SUNLinearSolver})
    a {\sunlinsol} object.
  \end{args}
}
{
  \id{int} containing the number of iterations
}
{
}
% --------------------------------------------------------------------
\ucfunction{SUNLinSolResNorm}
{
  rnorm = SUNLinSolResNorm(LS);
}
{
  The \emph{optional} function \Id{SUNLinSolResNorm}
  should return the final residual norm from the last
  `solve' call.
}
{
  \begin{args}[LS]
  \item[LS] (\id{SUNLinearSolver})
    a {\sunlinsol} object.
  \end{args}
}
{
  \id{realtype} containing the final residual norm
}
{
}
% --------------------------------------------------------------------
\ucfunction{SUNLinSolResid}
{
  rvec = SUNLinSolResid(LS);
}
{
   If an iterative method computes the preconditioned initial residual
   and returns with a successful solve without performing any
   iterations (i.e. either the initial guess or the preconditioner is
   sufficiently accurate), then this \emph{optional} routine may be
   called by the SUNDIALS package.  This routine should return the
   {\nvector} containing the preconditioned initial residual vector
}
{
  \begin{args}[LS]
  \item[LS] (\id{SUNLinearSolver})
    a {\sunlinsol} object.
  \end{args}
}
{
  \id{N\_Vector} containing the final residual vector
}
{
  Since \id{N\_Vector} is actually a pointer, and the results
  are not modified, this routine should \emph{not} require additional
  memory allocation.  If the {\sunlinsol} object does not retain a
  vector for this purpose, then this function pointer should be left
  \id{NULL} in the implementation.
}
% --------------------------------------------------------------------
\ucfunction{SUNLinSolLastFlag}
{
  lflag = SUNLinSolLastFlag(LS);
}
{
  The \emph{optional} function \Id{SUNLinSolLastFlag}
  should return the last error flag encountered within the
  linear solver. This is not called by the {\sundials} packages
  directly; it allows the user to investigate linear solver issues
  after a failed solve.
}
{
  \begin{args}[LS]
  \item[LS] (\id{SUNLinearSolver})
    a {\sunlinsol} object.
  \end{args}
}
{
  \id{long int} containing the most recent error flag
}
{
}
% --------------------------------------------------------------------
\ucfunction{SUNLinSolSpace}
{
  retval = SUNLinSolSpace(LS, \&lrw, \&liw);
}
{
  The \emph{optional} function \Id{SUNLinSolSpace}
  should return the storage requirements for the linear
  solver \id{LS}.
}
{
  \begin{args}[lrw]
  \item[LS] (\id{SUNLinearSolver})
    a {\sunlinsol} object.
  \item[lrw] (\id{long int*})
    the number of realtype words stored by the linear solver.
  \item[liw] (\id{long int*})
    the number of integer words stored by the linear solver.
  \end{args}
}
{
  This should return zero for a successful call, and a negative value
  for a failure, ideally returning one of the generic error codes
  listed in Table \ref{t:sunlinsolerr}.
}
{
  This function is advisory only, for use in determining a user's
  total space requirements.
}
% --------------------------------------------------------------------



%---------------------------------------------------------------------------
\subsection{Functions provided by {\sundials} packages}\label{ss:sunlinsol_SUNSuppliedFn}

To interface with the {\sunlinsol} modules, the {\sundials} packages
supply a variety of routines for evaluating the matrix-vector product,
and setting up and applying the preconditioner.  These
package-provided routines translate between the user-supplied ODE, DAE
or nonlinear systems and the generic interfaces to the linear systems
of equations that result in their solution.  The types for functions
provided to a {\sunlinsol} module are defined in the header
file \id{sundials/sundials\_iterative.h}, and are described below.



% --------------------------------------------------------------------
\usfunction{ATimesFn}
{
  typedef int (*ATimesFn)(void *A\_data, N\_Vector v, N\_Vector z);
}
{
  These functions compute the action of a matrix on a vector,
  performing the operation $z = Av$.  Memory for \id{z} should already be
  allocted prior to calling this function.  The vector \id{v} should
  be left unchanged.
}
{
  \begin{args}
  \item[A\_data]
    is a pointer to client data, the same as that supplied to \id{SUNLinSolSetATimes}.
  \item[v]
    is the input vector to multiply.
  \item[z]
    is the output vector computed.
  \end{args}
}
{
  This routine should return 0 if successful and a
  non-zero value if unsuccessful.
}
{
}
% --------------------------------------------------------------------
\usfunction{PSetupFn}
{
  typedef int (*PSetupFn)(void *P\_data)
}
{
  These functions set up any requisite problem data in preparation
  for calls to the corresponding \id{PSolveFn}.
}
{
  \begin{args}
  \item[P\_data]
    is a pointer to client data, the same pointer as that supplied to the routine
    \id{SUNLinSolSetPreconditioner}.
  \end{args}
}
{
  This routine should return 0 if successful and a non-zero value if unsuccessful.
}
{
}
% --------------------------------------------------------------------
\usfunction{PSolveFn}
{
  typedef int (*PSolveFn)(&void *P\_data, N\_Vector r, N\_Vector z, \\
                          &realtype tol, int lr)
}
{
  These functions solve the preconditioner equation $Pz = r$
  for the vector $z$.  Memory for \id{z} should already be
  allocted prior to calling this function.  The
  parameter \id{P\_data} is a pointer to any information about $P$
  which the function needs in order to do its job (set up by the
  corresponding \id{PSetupFn}. The parameter \id{lr} is input, and
  indicates whether $P$ is to be taken as the left preconditioner or
  the right preconditioner: \id{lr} = 1 for left and \id{lr} = 2 for
  right.  If preconditioning is on one side only, \id{lr} can be
  ignored.  If the preconditioner is iterative, then it should strive
  to solve the preconditioner equation so that
  \[
      \| Pz - r \|_{\text{wrms}} < tol
  \]
  where the weight vector for the WRMS norm may be accessed from the
  main package memory structure.  The vector \id{r} should not be
  modified by the \id{PSolveFn}.
}
{
  \begin{args}
  \item[P\_data]
    is a pointer to client data, the same pointer as that supplied to the routine \id{SUNLinSolSetPreconditioner}.
  \item[r]
    is the right-hand side vector for the preconditioner system
  \item[z]
    is the solution vector for the preconditioner system
  \item[tol]
    is the desired tolerance for an iterative preconditioner
  \item[lr]
    is flag indicating whether the routine should perform left (1) or
    right (2) preconditioning.
  \end{args}
}
{
  This routine should return 0 if successful and a non-zero value if
  unsuccessful.  On a failure, a negative return value indicates an
  unrecoverable condition, while a positive value indicates a
  recoverable one, in which the calling routine may reattempt the
  solution after updating preconditioner data.
}
{
}
% --------------------------------------------------------------------



%---------------------------------------------------------------------------
\subsection{\id{SUNLinearSolver} return codes}\label{ss:sunlinsol_ErrorCodes}


The functions provided to {\sunlinsol} modules by each {\sundials}
package, and functions within the {\sundials}-provided {\sunlinsol}
implementations utilize a common set of return codes, shown in the
Table \ref{t:sunlinsolerr}.  These adhere to a
common pattern: 0 indicates success, a postitive value corresponds to
a recoverable failure, and a negative value indicates a
non-recoverable failure.  Aside from this pattern, the actual values
of each error code are primarily to provide additional information to
the user in case of a linear solver failure.

\newlength{\ColumnOne}
\settowidth{\ColumnOne}{\id{SUNLS\_PACKAGE\_FAIL\_UNREC}}
\newlength{\ColumnTwo}
\settowidth{\ColumnTwo}{\id{Value}}
\newlength{\ColumnThree}
\setlength{\ColumnThree}{\textwidth}
\addtolength{\ColumnThree}{-0.5in}
\addtolength{\ColumnThree}{-\ColumnOne}
\addtolength{\ColumnThree}{-\ColumnTwo}

\tablecaption{Description of the \id{SUNLinearSolver} error codes}\label{t:sunlinsolerr}
\tablehead{\hline {\rule{0mm}{5mm}}{\bf Name} & {\bf Value} & {\bf Description} \\[3mm] \hline\hline}
\tabletail{\hline \multicolumn{3}{|r|}{\small\slshape continued on next page} \\ \hline}
\begin{xtabular}{|p{\ColumnOne}|p{\ColumnTwo}|p{\ColumnThree}|}
%%
\id{SUNLS\_SUCCESS} & \id{0} & successful call or converged solve
\\[1mm]
%%
\id{SUNLS\_MEM\_NULL} & \id{-1} & the memory argument to the function is \id{NULL}
\\[1mm]
%%
\id{SUNLS\_ILL\_INPUT} & \id{-2} & an illegal input has been provided to the function
\\[1mm]
%%
\id{SUNLS\_MEM\_FAIL} & \id{-3} & failed memory access or allocation
\\[1mm]
%%
\id{SUNLS\_ATIMES\_FAIL\_UNREC} & \id{-4} & an unrecoverable failure
  occurred in the \id{ATimes} routine
\\[1mm]
%%
\id{SUNLS\_PSET\_FAIL\_UNREC} & \id{-5} & an unrecoverable failure
  occurred in the \id{Pset} routine
\\[1mm]
%%
\id{SUNLS\_PSOLVE\_FAIL\_UNREC} & \id{-6} & an unrecoverable failure
  occurred in the \id{Psolve} routine
\\[1mm]
%%
\id{SUNLS\_PACKAGE\_FAIL\_UNREC} & \id{-7} & an unrecoverable failure
  occurred in an external linear solver package
\\[1mm]
%%
\id{SUNLS\_GS\_FAIL} & \id{-8} & a failure occurred during
  Gram-Schmidt orthogonalization ({\sunlinsolspgmr}/{\sunlinsolspfgmr})
\\[1mm]
%%
\id{SUNLS\_QRSOL\_FAIL} & \id{-9} & a singular $R$ matrix was
  encountered in a QR factorization ({\sunlinsolspgmr}/{\sunlinsolspfgmr})
\\[1mm]
%%
\id{SUNLS\_RES\_REDUCED} & \id{1} &  an iterative solver reduced the
  residual, but did not converge to the desired tolerance
\\[1mm]
%%
\id{SUNLS\_CONV\_FAIL} & \id{2} &  an iterative solver did not
converge (and the residual was not reduced)
\\[1mm]
%%
\id{SUNLS\_ATIMES\_FAIL\_REC} & \id{3} & a recoverable failure occurred
  in the \id{ATimes} routine
\\[1mm]
%%
\id{SUNLS\_PSET\_FAIL\_REC} & \id{4} & a recoverable failure occurred
  in the \id{Pset} routine
\\[1mm]
%%
\id{SUNLS\_PSOLVE\_FAIL\_REC} & \id{5} & a recoverable failure occurred
  in the \id{Psolve} routine
\\[1mm]
%%
\id{SUNLS\_PACKAGE\_FAIL\_REC} & \id{6} &  a recoverable failure
  occurred in an external linear solver package
\\[1mm]
%%
\id{SUNLS\_QRFACT\_FAIL} & \id{7} & a singular matrix was encountered
  during a QR factorization ({\sunlinsolspgmr}/{\sunlinsolspfgmr})
\\[1mm]
%%
\id{SUNLS\_LUFACT\_FAIL} & \id{8} & a singular matrix was encountered
  during a LU factorization ({\sunlinsoldense}/{\sunlinsolband})
\\
\end{xtabular}
\bigskip


%---------------------------------------------------------------------------
\subsection{The generic \id{SUNLinearSolver} module}\label{ss:sunlinsol_Generic}

SUNDIALS packages interact with specific {\sunlinsol} implementations
through the generic {\sunlinsol} module on which all other {\sunlinsol}
iplementations are built.  The \id{SUNLinearSolver} type is a pointer
to a structure containing an implementation-dependent \emph{content} field,
and an \emph{ops} field.  The type \Id{SUNLinearSolver} is defined as
%%
%%
\begin{verbatim}
typedef struct _generic_SUNLinearSolver *SUNLinearSolver;

struct _generic_SUNLinearSolver {
  void *content;
  struct _generic_SUNLinearSolver_Ops *ops;
};
\end{verbatim}
%%
%%
where the \id{\_generic\_SUNLinearSolver\_Ops} structure is a list of
pointers to the various actual linear solver operations provided by a
specific implementation.  The \id{\_generic\_SUNLinearSolver\_Ops}
structure is defined as
%%
\begin{verbatim}
struct _generic_SUNLinearSolver_Ops {
  SUNLinearSolver_Type (*gettype)(SUNLinearSolver);
  int                  (*setatimes)(SUNLinearSolver, void*, ATimesFn);
  int                  (*setpreconditioner)(SUNLinearSolver, void*,
                                            PSetupFn, PSolveFn);
  int                  (*setscalingvectors)(SUNLinearSolver,
                                            N_Vector, N_Vector);
  int                  (*initialize)(SUNLinearSolver);
  int                  (*setup)(SUNLinearSolver, SUNMatrix);
  int                  (*solve)(SUNLinearSolver, SUNMatrix, N_Vector,
                                N_Vector, realtype);
  int                  (*numiters)(SUNLinearSolver);
  realtype             (*resnorm)(SUNLinearSolver);
  long int             (*lastflag)(SUNLinearSolver);
  int                  (*space)(SUNLinearSolver, long int*, long int*);
  N_Vector             (*resid)(SUNLinearSolver);
  int                  (*free)(SUNLinearSolver);
};
\end{verbatim}

The generic {\sunlinsol} module defines and implements the linear
solver operations defined in Sections
\ref{ss:sunlinsol_CoreFn}-\ref{ss:sunlinsol_GetFn}.  These routines
are in fact only wrappers to the linear solver operations
defined by a particular {\sunlinsol} implementation, which are
accessed through the {\em ops} field of the \id{SUNLinearSolver}
structure. To illustrate this point we show below the implementation
of a typical linear solver operation from the generic {\sunlinsol}
module, namely \id{SUNLinSolInitialize}, which initializes a
{\sunlinsol} object for use after it has been created and configured,
and returns a flag denoting a successful/failed operation:
%%
%%
\begin{verbatim}
int SUNLinSolInitialize(SUNLinearSolver S)
{
  return ((int) S->ops->initialize(S));
}
\end{verbatim}
%%
%%


\section{Compatibility of \id{SUNLinearSolver} modules}\label{ss:sunlinsol_compatibility}


We note that not all {\sunlinsol} types are compatible with all
{\sunmatrix} and {\nvector} types provided with {\sundials}.  In Table
\ref{t:linsol-matrix} we show the matrix-based linear solvers
available as {\sunlinsol} modules, and the compatible matrix
implementations.  Recall that Table \ref{t:solver-vector} shows the
compatibility between all {\sunlinsol} modules and vector
implementations.

\tablecaption{{\sundials} matrix-based linear solvers and matrix
              implementations that can be used for each.}\label{t:linsol-matrix}
\tablehead{\hline \multicolumn{1}{|p{2cm}|}{{Linear Solver Interface}} &
                  \multicolumn{1}{p{1.3cm}|}{{Dense Matrix}} &
                  \multicolumn{1}{p{1.3cm}|}{{Banded Matrix}} &
                  \multicolumn{1}{p{1.4cm}|}{{Sparse Matrix}} &
                  \multicolumn{1}{p{1.4cm}|}{{User Supplied}} \\ \hline}
\tabletail{\hline \multicolumn{5}{|r|}{\small\slshape continued on next page} \\ \hline}
\begin{center}
\begin{xtabular}{|l|c|c|c|c|}
%   Linear Solver &  Dense   & Banded & Sparse & User     \\
%   Interface     &          &        &        & Supplied \\
    Dense         &  \cm     &        &        & \cm      \\
    Band          &          & \cm    &        & \cm      \\
    LapackDense   &  \cm     &        &        & \cm      \\
    LapackBand    &          & \cm    &        & \cm      \\
    \klu          &          &        &  \cm   & \cm      \\
    \superlumt    &          &        &  \cm   & \cm      \\
    User supplied &  \cm     & \cm    &  \cm   & \cm      \\
    \hline
\end{xtabular}
\end{center}
\bigskip



\section{Implementing a custom \id{SUNLinearSolver} module}\label{ss:sunlinsol_custom}

A particular implementation of the {\sunlinsol} module must:
\begin{itemize}
\item Specify the {\em content} field of the \id{SUNLinearSolver} object.
\item Define and implement a minimal subset of the linear solver
  operations. See the documentation for each {\sundials} linear solver
  interface to determine which {\sunlinsol} operations they require.

  Note that the names of these routines should be unique to that
  implementation in order to permit using more than one {\sunlinsol}
  module (each with different \id{SUNLinearSolver} internal data
  representations) in the same code.
\item Define and implement user-callable constructor and destructor
  routines to create and free a \id{SUNLinearSolver} with
  the new {\em content} field and with {\em ops} pointing to the
  new linear solver operations.
\end{itemize}

We note that the function pointers for all unsupported optional
routines should be set to \id{NULL} in the \emph{ops} structure.  This
allows the {\sundials} package that is using the {\sunlinsol} object
to know that the associated functionality is not supported.

Additionally, a {\sunlinsol} implementation \emph{may} do the
following:
\begin{itemize}
\item Define and implement additional user-callable ``set'' routines
  acting on the \id{SUNLinearSolver}, e.g., for setting various
  configuration options to tune the linear solver to a
  particular problem.
\item Provide additional user-callable ``get'' routines acting on the
  \id{SUNLinearSolver} object, e.g., for returning various solve
  statistics.
\end{itemize}





%---------------------------------------------------------------------------
\section{The SUNLinearSolver\_Dense implementation}\label{ss:sunlinsol_dense}
%% This is a shared SUNDIALS TEX file with a description of the
%% dense sunlinsol implementation
%%

The dense implementation of the {\sunlinsol} module provided with
{\sundials}, {\sunlinsoldense}, is designed to be used with the
corresponding {\sunmatdense} matrix type, and one of the serial or
shared-memory {\nvector} implementations ({\nvecs}, {\nvecopenmp} or
{\nvecpthreads}).  The {\sunlinsoldense} module defines the {\em
content} field of a \id{SUNLinearSolver} to be the following structure:
%%
\begin{verbatim} 
struct _SUNLinearSolverContent_Dense {
  sunindextype N;
  sunindextype *pivots;
  long int last_flag;
};
\end{verbatim}
%%
These entries of the \emph{content} field contain the following
information:
\begin{description}
  \item[N] - size of the linear system,
  \item[pivots] - index array for partial pivoting in LU factorization,
  \item[last\_flag] - last error return flag from internal function evaluations.
\end{description}

This solver is constructed to perform the following operations:
\begin{itemize}
\item In the ``setup'' call, this performs a $LU$ factorization with
  partial (row) pivoting ($\mathcal O(N^3)$ cost), $PA=LU$, where $P$
  is a permutation matrix, $L$ is a lower triangular matrix with 1's
  on the diagonal, and $U$ is an upper triangular matrix.  This
  factorization is stored in-place on the input {\sunmatdense} object
  $A$, with pivoting information encoding $P$ stored in
  the \id{pivots} array.
\item In the ``solve'' call, this performs pivoting, forward and
  backward substitution using the stored \id{pivots} array and the
  $LU$ factors held in the {\sunmatdense} object ($\mathcal O(N^2)$
  cost).
\end{itemize}

\noindent The header file to be included when using this module 
is \id{sunlinsol/sunlinsol\_dense.h}. \\
%%
%%----------------------------------------------
%%
The {\sunlinsoldense} module defines dense implementations of all
``direct'' linear solver operations listed in
Table \ref{t:sunlinsolops}:
\begin{itemize}
\item \id{SUNLinSolGetType\_Dense}
\item \id{SUNLinSolInitialize\_Dense} -- this does nothing, since all
  consistency checks were performed at solver creation.
\item \id{SUNLinSolSetup\_Dense} -- this performs the $LU$ factorization.
\item \id{SUNLinSolSolve\_Dense} -- this uses the $LU$ factors
  and \id{pivots} array to perform the solve.
\item \id{SUNLinSolLastFlag\_Dense}
\item \id{SUNLinSolSpace\_Dense} -- this only returns information for
  the storage \emph{within} the solver object, i.e.~storage
  for \id{N}, \id{last\_flag} and \id{pivots}.
\item \id{SUNLinSolFree\_Dense}
\end{itemize}
The module {\sunlinsoldense} provides the following additional
user-callable routine: 
%%
\begin{itemize}

%%--------------------------------------

\item \ID{SUNDenseLinearSolver}

  This function creates and allocates memory for a dense \id{SUNLinearSolver}.
  Its arguments are an {\nvector} and {\sunmatrix}, that it uses to
  determine the linear system size and to assess compatibility with
  the linear solver implementation.

  This routine will perform consistency checks to ensure that it is
  called with consistent {\nvector} and {\sunmatrix} implementations.
  These are currently limited to the {\sunmatdense} matrix type, and
  the {\nvecs}, {\nvecopenmp} and {\nvecpthreads} vector types.  As
  additional compatible matrix and vector implementations are added to
  {\sundials}, these will be included within this compatibility check.

  If either \id{A} or \id{y} are incompatible then this routine will
  return \id{NULL}.

  \verb|SUNLinearSolver SUNDenseLinearSolver(N_Vector y, SUNMatrix A);|

\end{itemize}
%%
%%------------------------------------
%%
For solvers that include a Fortran interface module, the {\sunlinsoldense}
module also includes the Fortran-callable
function \id{FSUNDenseLinSolInit(code, ier)} to initialize
this {\sunlinsoldense} module for a given {\sundials} solver.
Here \id{code} is an input solver id (1 for {\cvode}, 2 for {\ida}, 3
for {\kinsol}, 4 for {\arkode}); \id{ier} is an error return flag 
equal 0 for success and -1 for failure (declared so as to match C type
\id{int}).  This routine must be called \emph{after} both the
{\nvector} and {\sunmatrix} objects have been initialized.
Additionally, when using {\arkode} with non-identity mass matrix, the
Fortran-callable function \id{FSUNMassDenseLinSolInit(ier)}  
initializes this {\sunlinsoldense} module for solving mass matrix
linear systems.


%---------------------------------------------------------------------------
\section{The SUNLinearSolver\_Band implementation}\label{ss:sunlinsol_band}
%% This is a shared SUNDIALS TEX file with a description of the
%% band sunlinsol implementation
%%

The band implementation of the {\sunlinsol} module provided with
{\sundials}, {\sunlinsolband}, is designed to be used with the
corresponding {\sunmatband} matrix type, and one of the serial or
shared-memory {\nvector} implementations ({\nvecs}, {\nvecopenmp} or
{\nvecpthreads}).


%---------------------------------------------------------------------------
\subsection{{\sunlinsolband} usage}\label{ss:sunlinsol_band_usage}

The header file to include when using this module is
\id{sunlinsol/sunlinsol\_band.h}. The {\sunlinsolband} module 
is accessible from all {\sundials} solvers \textit{without}
linking to the \\ \noindent
\id{libsundials\_sunlinsolband} module library.

The module {\sunlinsolband} provides the following user-callable constructor routine: 
%%
% --------------------------------------------------------------------
\ucfunction{SUNLinSol\_Band}
{
  LS = SUNLinSol\_Band(y, A);
}
{
  The function \ID{SUNLinSol\_Band} creates and allocates memory for
  a band \id{SUNLinearSolver} object.
}
{
  \begin{args}[y]
  \item[y] (\id{N\_Vector})
    a template for cloning vectors needed within the solver
  \item[A] (\id{SUNMatrix})
    a {\sunmatband} matrix template for cloning matrices needed
    within the solver 
  \end{args}
}
{
  This returns a \id{SUNLinearSolver} object.  If either \id{A} or
  \id{y} are incompatible then this routine will return \id{NULL}.
}
{
  This routine will perform consistency checks to ensure that it is
  called with consistent {\nvector} and {\sunmatrix} implementations.
  These are currently limited to the {\sunmatdense} matrix type and
  the {\nvecs}, {\nvecopenmp}, and {\nvecpthreads} vector types.  As
  additional compatible matrix and vector implementations are added to
  {\sundials}, these will be included within this compatibility check.

  Additionally, this routine will verify that the input matrix \id{A}
  is allocated with appropriate upper bandwidth storage for the $LU$
  factorization.
}
% --------------------------------------------------------------------
%%
For backwards compatibility, we also provide the wrapper functions:
\begin{itemize}

\item \ID{SUNBandLinearSolver}

  Wrapper function for \ID{SUNLinSol\_Band}, with identical input and
  output arguments.

\end{itemize}
%%
%%------------------------------------
%%
For solvers that include a Fortran interface module, the {\sunlinsolband}
module also includes a Fortran-callable function for creating a
\id{SUNLinearSolver} object.
\ucfunction{FSUNBANDLINSOLINIT}
{
  FSUNBANDLINSOLINIT(code, ier)
}
{
  The function \ID{FSUNBANDLINSOLINIT} can be called for Fortran programs
  to create a band \id{SUNLinearSolver} object.
}
{
  \begin{args}[code]
  \item[code] (\id{int*})
    is an integer input specifying the solver id (1 for {\cvode}, 2
    for {\ida}, 3 for {\kinsol}, and 4 for {\arkode}).
  \end{args}
}
{
  \id{ier} is a return completion flag equal to \id{0} for a success
  return and \id{-1} otherwise. See printed message for details in case
  of failure.
}
{
  This routine must be
  called \emph{after} both the {\nvector} and {\sunmatrix} objects have
  been initialized.
}
Additionally, when using {\arkode} with a non-identity
mass matrix, the {\sunlinsolband} module includes a Fortran-callable
function for creating a \id{SUNLinearSolver} mass matrix solver
object.
\ucfunction{FSUNMASSBANDLINSOLINIT}
{
  FSUNMASSBANDLINSOLINIT(ier)
}
{
  The function \ID{FSUNMASSBANDLINSOLINIT} can be called for Fortran programs
  to create a band \id{SUNLinearSolver} object for mass matrix linear
  systems.
}
{
}
{
  \id{ier} is a \id{int} return completion flag equal to \id{0} for a success
  return and \id{-1} otherwise. See printed message for details in case
  of failure.
}
{
  This routine must be
  called \emph{after} both the {\nvector} and {\sunmatrix} mass-matrix
  objects have been initialized.
}

%---------------------------------------------------------------------------
\subsection{{\sunlinsolband} description}\label{ss:sunlinsol_band_description}



The {\sunlinsolband} module defines the {\em
content} field of a \id{SUNLinearSolver} to be the following structure:
%%
\begin{verbatim} 
struct _SUNLinearSolverContent_Band {
  sunindextype N;
  sunindextype *pivots;
  long int last_flag;
};
\end{verbatim}
%%
These entries of the \emph{content} field contain the following
information:
\begin{description}
  \item[N] - size of the linear system,
  \item[pivots] - index array for partial pivoting in LU factorization,
  \item[last\_flag] - last error return flag from internal function evaluations.
\end{description}

This solver is constructed to perform the following operations:
\begin{itemize}
\item The ``setup'' call performs a $LU$ factorization with
  partial (row) pivoting, $PA=LU$, where $P$ is a permutation matrix,
  $L$ is a lower triangular matrix with 1's on the diagonal, and $U$
  is an upper triangular matrix.  This factorization is stored
  in-place on the input {\sunmatband} object $A$, with pivoting
  information encoding $P$ stored in the \id{pivots} array.
\item The ``solve'' call performs pivoting and forward and
  backward substitution using the stored \id{pivots} array and the
  $LU$ factors held in the {\sunmatband} object.
\item
  {\warn} $A$ must be allocated to accommodate the increase in upper
  bandwidth that occurs during factorization.  More precisely, if $A$
  is a band matrix with upper bandwidth \id{mu} and lower bandwidth
  \id{ml}, then the upper triangular factor $U$ can have upper
  bandwidth as big as \id{smu = MIN(N-1,mu+ml)}. The lower triangular
  factor $L$ has lower bandwidth \id{ml}.
\end{itemize}


%%
%%----------------------------------------------
%%

\noindent The {\sunlinsolband} module defines band implementations of all
``direct'' linear solver operations listed in Sections
\ref{ss:sunlinsol_CoreFn}-\ref{ss:sunlinsol_GetFn}:
\begin{itemize}
\item \id{SUNLinSolGetType\_Band}
\item \id{SUNLinSolInitialize\_Band} -- this does nothing, since all
  consistency checks are performed at solver creation.
\item \id{SUNLinSolSetup\_Band} -- this performs the $LU$ factorization.
\item \id{SUNLinSolSolve\_Band} -- this uses the $LU$ factors
  and \id{pivots} array to perform the solve.
\item \id{SUNLinSolLastFlag\_Band}
\item \id{SUNLinSolSpace\_Band} -- this only returns information for
  the storage \emph{within} the solver object, i.e.~storage
  for \id{N}, \id{last\_flag}, and \id{pivots}.
\item \id{SUNLinSolFree\_Band}
\end{itemize}


%---------------------------------------------------------------------------
\section{The SUNLinearSolver\_LapackDense implementation}\label{ss:sunlinsol_lapdense}
% ====================================================================
\section{The SUNLinearSolver\_LapackDense implementation}
\label{ss:sunlinsol_lapdense}
% ====================================================================

This section describes the {\sunlinsol} implementation for solving dense linear
systems with LAPACK. The {\sunlinsollapdense} module is designed to be used with the
corresponding {\sunmatdense} matrix type, and one of the serial or
shared-memory {\nvector} implementations ({\nvecs}, {\nvecopenmp}, or
{\nvecpthreads}).

To access the {\sunlinsollapdense} module, include the header file \newline
\id{sunlinsol/sunlinsol\_lapackdense.h}. The installed module library to link
to is \newline
\id{libsundials\_sunlinsollapackdense.\textit{lib}} where \id{\em.lib}
is typically \id{.so} for shared libraries and \id{.a} for static libraries.

The {\sunlinsollapdense} module is a {\sunlinsol} wrapper for
the LAPACK dense matrix factorization and solve routines, \id{*GETRF}
and \id{*GETRS}, where \id{*} is either \id{D} or \id{S}, depending on
whether {\sundials} was configured to have \id{realtype} set to
\id{double} or \id{single}, respectively (see Section \ref{s:types}).
In order to use the {\sunlinsollapdense} module it is assumed
that LAPACK has been installed on the system prior to installation of
{\sundials}, and that {\sundials} has been configured appropriately to
link with LAPACK (see Appendix \ref{c:install} for details).  
We note that since there do not exist 128-bit floating-point
factorization and solve routines in LAPACK, this interface cannot be
compiled when using \id{extended} precision for \id{realtype}.
Similarly, since there do not exist 64-bit integer LAPACK routines,
the {\sunlinsollapdense} module also cannot be compiled when using
64-bit integers for the \id{sunindextype}. {\warn}


% ====================================================================
\subsection{SUNLinearSolver\_LapackDense description}
\label{ss:sunlinsol_lapdense_description}
% ====================================================================

This solver is constructed to perform the following operations:
\begin{itemize}
\item The ``setup'' call performs a $LU$ factorization with
  partial (row) pivoting ($\mathcal O(N^3)$ cost), $PA=LU$, where $P$
  is a permutation matrix, $L$ is a lower triangular matrix with 1's
  on the diagonal, and $U$ is an upper triangular matrix.  This
  factorization is stored in-place on the input {\sunmatdense} object
  $A$, with pivoting information encoding $P$ stored in
  the \id{pivots} array.
\item The ``solve'' call performs pivoting and forward and
  backward substitution using the stored \id{pivots} array and the
  $LU$ factors held in the {\sunmatdense} object ($\mathcal O(N^2)$
  cost).
\end{itemize}


% ====================================================================
\subsection{SUNLinearSolver\_LapackDense functions}
\label{ss:sunlinsol_lapdense_functions}
% ====================================================================

The {\sunlinsollapdense} module provides the following user-callable constructor
for creating a \newline \id{SUNLinearSolver} object.
%
% --------------------------------------------------------------------
%
\ucfunctiond{SUNLinSol\_LapackDense}
{
  LS = SUNLinSol\_LapackDense(y, A);
}
{
  The function \ID{SUNLinSol\_LapackDense} creates and allocates memory for
  a LAPACK-based, dense \id{SUNLinearSolver} object.
}
{
  \begin{args}[y]
  \item[y] (\id{N\_Vector})
    a template for cloning vectors needed within the solver
  \item[A] (\id{SUNMatrix})
    a {\sunmatdense} matrix template for cloning matrices needed
    within the solver 
  \end{args}
}
{
  This returns a \id{SUNLinearSolver} object.  If either \id{A} or
  \id{y} are incompatible then this routine will return \id{NULL}.
}
{
  This routine will perform consistency checks to ensure that it is
  called with consistent {\nvector} and {\sunmatrix} implementations.
  These are currently limited to the {\sunmatdense} matrix type and
  the {\nvecs}, {\nvecopenmp}, and {\nvecpthreads} vector types.  As
  additional compatible matrix and vector implementations are added to
  {\sundials}, these will be included within this compatibility check.
}
{SUNLapackDense}
%
% --------------------------------------------------------------------
%
The {\sunlinsollapdense} module defines dense implementations of all
``direct'' linear solver operations listed in Sections
\ref{ss:sunlinsol_CoreFn} -- \ref{ss:sunlinsol_GetFn}:
\begin{itemize}
\item \id{SUNLinSolGetType\_LapackDense}
\item \id{SUNLinSolInitialize\_LapackDense} -- this does nothing, since all
  consistency checks are performed at solver creation.
\item \id{SUNLinSolSetup\_LapackDense} -- this calls either
  \id{DGETRF} or \id{SGETRF} to perform the $LU$ factorization.
\item \id{SUNLinSolSolve\_LapackDense} -- this calls either
  \id{DGETRS} or \id{SGETRS} to use the $LU$ factors and \id{pivots}
  array to perform the solve.
\item \id{SUNLinSolLastFlag\_LapackDense}
\item \id{SUNLinSolSpace\_LapackDense} -- this only returns information for
  the storage \emph{within} the solver object, i.e.~storage
  for \id{N}, \id{last\_flag}, and \id{pivots}.
\item \id{SUNLinSolFree\_LapackDense}
\end{itemize}


% ====================================================================
\subsection{SUNLinearSolver\_LapackDense Fortran interfaces}
\label{ss:sunlinsol_lapdense_fortran}
% ====================================================================

For solvers that include a {\F} 77 interface module, the {\sunlinsollapdense}
module also includes a Fortran-callable function for creating a
\id{SUNLinearSolver} object.
%
% --------------------------------------------------------------------
%
\ucfunction{FSUNLAPACKDENSEINIT}
{
  FSUNLAPACKDENSEINIT(code, ier)
}
{
  The function \ID{FSUNLAPACKDENSEINIT} can be called for Fortran programs
  to create a LAPACK-based dense \id{SUNLinearSolver} object.
}
{
  \begin{args}[code]
  \item[code] (\id{int*})
    is an integer input specifying the solver id (1 for {\cvode}, 2
    for {\ida}, 3 for {\kinsol}, and 4 for {\arkode}).
  \end{args}
}
{
  \id{ier} is a return completion flag equal to \id{0} for a success
  return and \id{-1} otherwise. See printed message for details in case
  of failure.
}
{
  This routine must be
  called \emph{after} both the {\nvector} and {\sunmatrix} objects have
  been initialized.
}
Additionally, when using {\arkode} with a non-identity
mass matrix, the {\sunlinsollapdense} module includes a Fortran-callable
function for creating a \id{SUNLinearSolver} mass matrix solver
object.
%
% --------------------------------------------------------------------
%
\ucfunction{FSUNMASSLAPACKDENSEINIT}
{
  FSUNMASSLAPACKDENSEINIT(ier)
}
{
  The function \ID{FSUNMASSLAPACKDENSEINIT} can be called for Fortran programs
  to create a LAPACK-based, dense \id{SUNLinearSolver} object for mass matrix linear
  systems.
}
{}
{
  \id{ier} is a \id{int} return completion flag equal to \id{0} for a success
  return and \id{-1} otherwise. See printed message for details in case
  of failure.
}
{
  This routine must be
  called \emph{after} both the {\nvector} and {\sunmatrix} mass-matrix
  objects have been initialized.
}


% ====================================================================
\subsection{SUNLinearSolver\_LapackDense content}
\label{ss:sunlinsol_lapdense_content}
% ====================================================================

The {\sunlinsollapdense} module defines the \textit{content} field of a
\id{SUNLinearSolver} as the following structure:
%%
\begin{verbatim} 
struct _SUNLinearSolverContent_Dense {
  sunindextype N;
  sunindextype *pivots;
  long int last_flag;
};
\end{verbatim}
%%
These entries of the \emph{content} field contain the following
information:
\begin{args}[last\_flag]
  \item[N] - size of the linear system,
  \item[pivots] - index array for partial pivoting in LU factorization,
  \item[last\_flag] - last error return flag from internal function evaluations.
\end{args}



%---------------------------------------------------------------------------
\section{The SUNLinearSolver\_LapackBand implementation}\label{ss:sunlinsol_lapband}
% ====================================================================
\section{The SUNLinearSolver\_LapackBand implementation}
\label{ss:sunlinsol_lapband}
% ====================================================================

This section describes the {\sunlinsol} implementation for solving banded linear
systems with LAPACK. The {\sunlinsollapband} module is designed to be used with the
corresponding {\sunmatband} matrix type, and one of the serial or
shared-memory {\nvector} implementations ({\nvecs}, {\nvecopenmp}, or
{\nvecpthreads}).

To access the {\sunlinsollapband} module, include the header file \newline
\id{sunlinsol/sunlinsol\_lapackband.h}. The installed module library to link
to is \newline
\id{libsundials\_sunlinsollapackband.\textit{lib}} where \id{\em.lib}
is typically \id{.so} for shared libraries and \id{.a} for static libraries.

The {\sunlinsollapband} module is a {\sunlinsol} wrapper for
the LAPACK band matrix factorization and solve routines, \id{*GBTRF}
and \id{*GBTRS}, where \id{*} is either \id{D} or \id{S}, depending on
whether {\sundials} was configured to have \id{realtype} set to
\id{double} or \id{single}, respectively (see Section \ref{s:types}).
In order to use the {\sunlinsollapband} module it is assumed
that LAPACK has been installed on the system prior to installation of
{\sundials}, and that {\sundials} has been configured appropriately to
link with LAPACK (see Appendix \ref{c:install} for details).  
We note that since there do not exist 128-bit floating-point
factorization and solve routines in LAPACK, this interface cannot be
compiled when using \id{extended} precision for \id{realtype}.
Similarly, since there do not exist 64-bit integer LAPACK routines,
the {\sunlinsollapband} module also cannot be compiled when using
64-bit integers for the \id{sunindextype}. {\warn}


% ====================================================================
\subsection{SUNLinearSolver\_LapackBand description}
\label{ss:sunlinsol_lapband_description}
% ====================================================================

This solver is constructed to perform the following operations:
\begin{itemize}
\item The ``setup'' call performs a $LU$ factorization with
  partial (row) pivoting, $PA=LU$, where $P$ is a permutation matrix,
  $L$ is a lower triangular matrix with 1's on the diagonal, and $U$
  is an upper triangular matrix.  This factorization is stored
  in-place on the input {\sunmatband} object $A$, with pivoting
  information encoding $P$ stored in the \id{pivots} array.
\item The ``solve'' call performs pivoting and forward and
  backward substitution using the stored \id{pivots} array and the
  $LU$ factors held in the {\sunmatband} object.
\item
  $A$ must be allocated to accommodate the increase in upper
  bandwidth that occurs during factorization.  More precisely, if $A$
  is a band matrix with upper bandwidth \id{mu} and lower bandwidth
  \id{ml}, then the upper triangular factor $U$ can have upper
  bandwidth as big as \id{smu = MIN(N-1,mu+ml)}. The lower triangular
  factor $L$ has lower bandwidth \id{ml}. {\warn}
\end{itemize}


% ====================================================================
\subsection{SUNLinearSolver\_LapackBand functions}
\label{ss:sunlinsol_lapband_functions}
% ====================================================================

The {\sunlinsollapband} module provides the following user-callable constructor
for creating a \newline \id{SUNLinearSolver} object.
%
% --------------------------------------------------------------------
%
\ucfunctiond{SUNLinSol\_LapackBand}
{
  LS = SUNLinSol\_LapackBand(y, A);
}
{
  The function \ID{SUNLinSol\_LapackBand} creates and allocates memory for
  a LAPACK-based, band \id{SUNLinearSolver} object.
}
{
  \begin{args}[y]
  \item[y] (\id{N\_Vector})
    a template for cloning vectors needed within the solver
  \item[A] (\id{SUNMatrix})
    a {\sunmatband} matrix template for cloning matrices needed
    within the solver 
  \end{args}
}
{
  This returns a \id{SUNLinearSolver} object.  If either \id{A} or
  \id{y} are incompatible then this routine will return \id{NULL}.
}
{
  This routine will perform consistency checks to ensure that it is
  called with consistent {\nvector} and {\sunmatrix} implementations.
  These are currently limited to the {\sunmatband} matrix type and
  the {\nvecs}, {\nvecopenmp}, and {\nvecpthreads} vector types.  As
  additional compatible matrix and vector implementations are added to
  {\sundials}, these will be included within this compatibility check.
  
  Additionally, this routine will verify that the input matrix \id{A}
  is allocated with appropriate upper bandwidth storage for the $LU$
  factorization.
}
{SUNLapackBand}
%
% --------------------------------------------------------------------
%
The {\sunlinsollapband} module defines band implementations of all
``direct'' linear solver operations listed in Sections
\ref{ss:sunlinsol_CoreFn} -- \ref{ss:sunlinsol_GetFn}:
\begin{itemize}
\item \id{SUNLinSolGetType\_LapackBand}
\item \id{SUNLinSolInitialize\_LapackBand} -- this does nothing, since all
  consistency checks are performed at solver creation.
\item \id{SUNLinSolSetup\_LapackBand} -- this calls either
  \id{DGBTRF} or \id{SGBTRF} to perform the $LU$ factorization.
\item \id{SUNLinSolSolve\_LapackBand} -- this calls either
  \id{DGBTRS} or \id{SGBTRS} to use the $LU$ factors and \id{pivots}
  array to perform the solve.
\item \id{SUNLinSolLastFlag\_LapackBand}
\item \id{SUNLinSolSpace\_LapackBand} -- this only returns information for
  the storage \emph{within} the solver object, i.e.~storage
  for \id{N}, \id{last\_flag}, and \id{pivots}.
\item \id{SUNLinSolFree\_LapackBand}
\end{itemize}


% ====================================================================
\subsection{SUNLinearSolver\_LapackBand Fortran interfaces}
\label{ss:sunlinsol_lapband_fortran}
% ====================================================================

For solvers that include a {\F} 77 interface module, the {\sunlinsollapband}
module also includes a Fortran-callable function for creating a
\id{SUNLinearSolver} object.
%
% --------------------------------------------------------------------
%
\ucfunction{FSUNLAPACKDENSEINIT}
{
  FSUNLAPACKBANDINIT(code, ier)
}
{
  The function \ID{FSUNLAPACKBANDINIT} can be called for Fortran programs
  to create a LAPACK-based band \id{SUNLinearSolver} object.
}
{
  \begin{args}[code]
  \item[code] (\id{int*})
    is an integer input specifying the solver id (1 for {\cvode}, 2
    for {\ida}, 3 for {\kinsol}, and 4 for {\arkode}).
  \end{args}
}
{
  \id{ier} is a return completion flag equal to \id{0} for a success
  return and \id{-1} otherwise. See printed message for details in case
  of failure.
}
{
  This routine must be
  called \emph{after} both the {\nvector} and {\sunmatrix} objects have
  been initialized.
}
Additionally, when using {\arkode} with a non-identity
mass matrix, the {\sunlinsollapband} module includes a Fortran-callable
function for creating a \id{SUNLinearSolver} mass matrix solver
object.
%
% --------------------------------------------------------------------
%
\ucfunction{FSUNMASSLAPACKBANDINIT}
{
  FSUNMASSLAPACKBANDINIT(ier)
}
{
  The function \ID{FSUNMASSLAPACKBANDINIT} can be called for Fortran programs
  to create a LAPACK-based, band \id{SUNLinearSolver} object for mass matrix linear
  systems.
}
{}
{
  \id{ier} is a \id{int} return completion flag equal to \id{0} for a success
  return and \id{-1} otherwise. See printed message for details in case
  of failure.
}
{
  This routine must be
  called \emph{after} both the {\nvector} and {\sunmatrix} mass-matrix
  objects have been initialized.
}


% ====================================================================
\subsection{SUNLinearSolver\_LapackBand content}
\label{ss:sunlinsol_lapband_content}
% ====================================================================

The {\sunlinsollapband} module defines the \textit{content} field of a
\id{SUNLinearSolver} as the following structure:
%%
\begin{verbatim} 
struct _SUNLinearSolverContent_Band {
  sunindextype N;
  sunindextype *pivots;
  long int last_flag;
};
\end{verbatim}
%%
These entries of the \emph{content} field contain the following
information:
\begin{args}[last\_flag]
  \item[N] - size of the linear system,
  \item[pivots] - index array for partial pivoting in LU factorization,
  \item[last\_flag] - last error return flag from internal function evaluations.
\end{args}



%---------------------------------------------------------------------------
\section{The SUNLinearSolver\_KLU implementation}\label{ss:sunlinsol_klu}
%% This is a shared SUNDIALS TEX file with a description of the
%% klu sunlinsol implementation
%%

The {\klu} implementation of the {\sunlinsol} module provided with
{\sundials}, {\sunlinsolklu}, is designed to be used with the
corresponding {\sunmatsparse} matrix type, and one of the serial or
shared-memory {\nvector} implementations ({\nvecs}, {\nvecopenmp}, or 
{\nvecpthreads}).  The {\sunlinsolklu} module defines the {\em
content} field of a \id{SUNLinearSolver} to be the following structure:
%%
\begin{verbatim} 
struct _SUNLinearSolverContent_KLU {
  long int         last_flag;
  int              first_factorize;
  sun_klu_symbolic *symbolic;
  sun_klu_numeric  *numeric;
  sun_klu_common   common;
  sunindextype     (*klu_solver)(sun_klu_symbolic*, sun_klu_numeric*,
                                 sunindextype, sunindextype,
                                 double*, sun_klu_common*);
};
\end{verbatim}
%%
These entries of the \emph{content} field contain the following
information:
\begin{description}
  \item[last\_flag] - last error return flag from internal function evaluations,
  \item[first\_factorize] - flag indicating whether the factorization
    has ever been performed, 
  \item[Symbolic] - {\klu} storage structure for symbolic factorization components,
  \item[Numeric] - {\klu} storage structure for numeric factorization components,
  \item[Common] - storage structure for common {\klu} solver components,
  \item[klu\_solver] -- pointer to the appropriate {\klu} solver function
    (depending on whether it is using a CSR or CSC sparse matrix).
\end{description}

{\warn} The {\sunlinsolklu} module is a {\sunlinsol} wrapper for
the {\klu} sparse matrix factorization and solver library written by Tim
Davis \cite{KLU_site,DaPa:10}.  In order to use the
{\sunlinsolklu} interface to {\klu}, it is assumed that {\klu} has
been installed on the system prior to installation of {\sundials}, and
that {\sundials} has been configured appropriately to link with {\klu}
(see Appendix \ref{c:install} for details).  Additionally, this
wrapper only supports double-precision calculations, and therefore
cannot be compiled if {\sundials} is configured to have \id{realtype}
set to either \id{extended} or \id{single} (see Section \ref{s:types}).
Since the {\klu} library supports both 32-bit and 64-bit integers, this
interface will be compiled for either of the available \id{sunindextype} options.

The {\klu} library has a symbolic factorization routine that computes
the permutation of the linear system matrix to block triangular form
and the permutations that will pre-order the diagonal blocks (the only
ones that need to be factored) to reduce fill-in (using AMD, COLAMD,
CHOLAMD, natural, or an ordering given by the user).  Of these
ordering choices, the default value in the {\sunlinsolklu} 
module is the COLAMD ordering.

{\klu} breaks the factorization into two separate parts.  The first is
a symbolic factorization and the second is a numeric factorization
that returns the factored matrix along with final pivot information.   
{\klu} also has a refactor routine that can be called instead of the numeric 
factorization.  This routine will reuse the pivot information.  This routine 
also returns diagnostic information that a user can examine to determine if 
numerical stability is being lost and a full numerical factorization should 
be done instead of the refactor.

Since the linear systems that arise within the context of {\sundials}
calculations will typically have identical sparsity patterns, the
{\sunlinsolklu} module is constructed to perform the
following operations:
\begin{itemize}
\item The first time that the ``setup'' routine is called, it
  performs the symbolic factorization, followed by an initial
  numerical factorization.  
\item On subsequent calls to the ``setup'' routine, it calls the
  appropriate {\klu} ``refactor'' routine, followed by estimates of
  the numerical conditioning using the relevant ``rcond'', and if
  necessary ``condest'', routine(s).  If these estimates of the
  condition number are larger than $\varepsilon^{-2/3}$ (where
  $\varepsilon$ is the double-precision unit roundoff), then a new
  factorization is performed.
\item The module includes the routine \id{SUNKLUReInit}, that 
  can be called by the user to force a full refactorization at the
  next ``setup'' call. 
\item The ``solve'' call performs pivoting and forward and
  backward substitution using the stored {\klu} data structures.  We
  note that in this solve {\klu} operates on the native data arrays
  for the right-hand side and solution vectors, without requiring
  costly data copies.
\end{itemize}


\noindent The header file to be included when using this module 
is \id{sunlinsol/sunlinsol\_klu.h}. \\
%%
%%----------------------------------------------
%%
The {\sunlinsolklu} module defines implementations of all
``direct'' linear solver operations listed in
Table \ref{t:sunlinsolops}:
\begin{itemize}
\item \id{SUNLinSolGetType\_KLU}
\item \id{SUNLinSolInitialize\_KLU} -- this sets the
  \id{first\_factorize} flag to 1, forcing both symbolic and numerical
  factorizations on the subsequent ``setup'' call.
\item \id{SUNLinSolSetup\_KLU} -- this performs either a $LU$
  factorization or refactorization of the input matrix.
\item \id{SUNLinSolSolve\_KLU} -- this calls the appropriate {\klu}
  solve routine to utilize the $LU$ factors to solve the linear
  system. 
\item \id{SUNLinSolLastFlag\_KLU}
\item \id{SUNLinSolSpace\_KLU} -- this only returns information for
  the storage within the solver \emph{interface}, i.e.~storage for the
  integers \id{last\_flag} and \id{first\_factorize}.  For additional
  space requirements, see the {\klu} documentation.
\item \id{SUNLinSolFree\_KLU}
\end{itemize}
The module {\sunlinsolklu} provides the following additional
user-callable routines: 
%%
\begin{itemize}

%%--------------------------------------

\item \ID{SUNKLU}

  This constructor function creates and allocates memory for a {\sunlinsolklu}
  object.  Its arguments are an {\nvector} and {\sunmatrix}, that it
  uses to determine the linear system size and to assess compatibility
  with the linear solver implementation. 

  This routine will perform consistency checks to ensure that it is
  called with consistent {\nvector} and {\sunmatrix} implementations.
  These are currently limited to the {\sunmatsparse} matrix type
  (using either CSR or CSC storage formats) and the {\nvecs},
  {\nvecopenmp}, and {\nvecpthreads} vector types.  As additional
  compatible matrix and vector implementations are added to
  {\sundials}, these will be included within this compatibility
  check. 

  If either \id{A} or \id{y} are incompatible then this routine will
  return \id{NULL}.

  \verb|SUNLinearSolver SUNKLU(N_Vector y, SUNMatrix A);|

%%--------------------------------------

\item \ID{SUNKLUReInit}

  This function reinitializes memory and flags for a new factorization
  (symbolic and numeric) to be conducted at the next solver setup
  call.  This routine is useful in the cases where the number of
  nonzeroes has changed or if the structure of the linear system has
  changed which would require a new symbolic (and numeric
  factorization). 

  The \id{reinit\_type} argument governs the level of
  reinitialization.  The allowed values are: 
  \begin{itemize}
  \item[1] The Jacobian matrix will be destroyed and a new one will be
    allocated based on the \id{nnz} value passed to this call.  New
    symbolic and numeric factorizations will be completed at the next
    solver setup.
  \item[2] Only symbolic and numeric factorizations will be completed.
    It is assumed that the Jacobian size has not exceeded the size of
    \id{nnz} given in the sparse matrix provided to the original
    constructor routine (or the previous \id{SUNKLUReInit} call). 
  \end{itemize}
  
  This routine assumes no other changes to solver use are necessary.

  The return values from this function are \id{SUNLS\_MEM\_NULL}
  (either \id{S} or \id{A} are \id{NULL}), \id{SUNLS\_ILL\_INPUT}
  (\id{A} does not have type \id{SUNMATRIX\_SPARSE} or
  \id{reinit\_type} is invalid), \id{SUNLS\_MEM\_FAIL} (reallocation
  of the sparse matrix failed) or \id{SUNLS\_SUCCESS}.
  
\begin{verbatim}
int SUNKLUReInit(SUNLinearSolver S, SUNMatrix A, 
                 sunindextype nnz, int reinit_type);
\end{verbatim}


%%--------------------------------------

\item \ID{SUNKLUSetOrdering}

  This function sets the ordering used by {\klu} for reducing fill in
  the linear solve.  Options for \id{ordering\_choice} are:
  \begin{itemize}
  \item[0] AMD,
  \item[1] COLAMD, and
  \item[2] the natural ordering.
  \end{itemize}
  The default is 1 for COLAMD.

  The return values from this function are \id{SUNLS\_MEM\_NULL}
  (\id{S} is \id{NULL}), \id{SUNLS\_ILL\_INPUT}
  (invalid \id{ordering\_choice}), or \id{SUNLS\_SUCCESS}.
  
  \verb|int SUNKLUSetOrdering(SUNLinearSolver S, int ordering_choice);|

\end{itemize}
%%
%%------------------------------------
%%
For solvers that include a Fortran interface module, the
{\sunlinsolklu} module also includes the Fortran-callable
function \id{FSUNKLUInit(code, ier)} to initialize this
{\sunlinsolklu} module for a given {\sundials} solver.  Here \id{code}
is an integer input solver id (1 for {\cvode}, 2 for {\ida}, 3 for {\kinsol},
4 for {\arkode}); \id{ier} is an error return flag equal to 0 for success
and -1 for failure. Both \id{code} and \id{ier}
are declared to match C type \id{int}. This
routine must be called \emph{after} both the {\nvector} and
{\sunmatrix} objects have been initialized.  Additionally, when using
{\arkode} with a non-identity mass matrix, the Fortran-callable function
\id{FSUNMassKLUInit(ier)} initializes this {\sunlinsolklu} module for
solving mass matrix linear systems.

The \id{SUNKLUReInit} and \ID{SUNKLUSetOrdering} routines also support
Fortran interfaces for the system and mass matrix solvers:
\begin{itemize}
\item \id{FSUNKLUReInit(code, NNZ, reinit\_type, ier)} -- \id{NNZ}
  should be commensurate with a C \id{long int} and \id{reinit\_type}
  should be commensurate with a C \id{int}
\item \id{FSUNMassKLUReInit(NNZ, reinit\_type, ier)}
\item \id{FSUNKLUSetOrdering(code, ordering, ier)} -- \id{ordering}
  should be commensurate with a C \id{int}
\item \id{FSUNMassKLUSetOrdering(ordering, ier)}
\end{itemize}


%---------------------------------------------------------------------------
\section{The SUNLinearSolver\_SuperLUMT implementation}\label{ss:sunlinsol_superlumt}
% ====================================================================
\section{The SUNLinearSolver\_SuperLUMT implementation}
\label{ss:sunlinsol_superlumt}
% ====================================================================

This section describes the {\sunlinsol} implementation for solving sparse linear
systems with SuperLU\_MT. The {\superlumt} module is designed to be used with the
corresponding {\sunmatsparse} matrix type, and one of the serial or
shared-memory {\nvector} implementations ({\nvecs}, {\nvecopenmp}, or 
{\nvecpthreads}). While these are compatible, it is not recommended
to use a threaded vector module with {\sunlinsolslumt} unless it is
the {\nvecopenmp} module and the {\superlumt} library has also been
compiled with OpenMP.

The header file to include when using this module 
is \id{sunlinsol/sunlinsol\_superlumt.h}. The installed module
library to link to is
\id{libsundials\_sunlinsolsuperlumt.\textit{lib}}
where \id{\em.lib} is typically \id{.so} for shared libraries and
\id{.a} for static libraries.

The {\sunlinsolslumt} module is a {\sunlinsol} wrapper for
the {\superlumt} sparse matrix factorization and solver library
written by X. Sherry Li \cite{SuperLUMT_site,Li:05,DGL:99}.  The
package performs matrix factorization using threads to enhance
efficiency in shared memory parallel environments.  It should be noted
that threads are only used in the factorization step.  In
order to use the {\sunlinsolslumt} interface to {\superlumt}, it is
assumed that {\superlumt} has been installed on the system prior to
installation of {\sundials}, and that {\sundials} has been configured
appropriately to link with {\superlumt} (see Appendix \ref{c:install}
for details).  Additionally, this wrapper only supports single- and
double-precision calculations, and therefore cannot be compiled if
{\sundials} is configured to have \id{realtype} set to \id{extended}
(see Section \ref{s:types}).  Moreover, since the {\superlumt} library
may be installed to support either 32-bit or 64-bit integers, it is
assumed that the {\superlumt} library is installed using the same
integer precision as the {\sundials} \id{sunindextype} option. {\warn}

% ====================================================================
\subsection{SUNLinearSolver\_SuperLUMT description}
\label{ss:sunlinsol_slumt_usage}
% ====================================================================

The {\superlumt} library has a symbolic factorization routine that
computes the permutation of the linear system matrix to reduce fill-in
on subsequent $LU$ factorizations (using COLAMD, minimal degree
ordering on $A^T*A$, minimal degree ordering on $A^T+A$, or natural
ordering).  Of these ordering choices, the default value in the
{\sunlinsolslumt} module is the COLAMD ordering. 

Since the linear systems that arise within the context of {\sundials}
calculations will typically have identical sparsity patterns, the
{\sunlinsolslumt} module is constructed to perform the
following operations:
\begin{itemize}
\item The first time that the ``setup'' routine is called, it
  performs the symbolic factorization, followed by an initial
  numerical factorization.  
\item On subsequent calls to the ``setup'' routine, it skips the
  symbolic factorization, and only refactors the input matrix.
\item The ``solve'' call performs pivoting and forward and
  backward substitution using the stored {\superlumt} data
  structures.  We note that in this solve {\superlumt} operates on the
  native data arrays for the right-hand side and solution vectors,
  without requiring costly data copies.
\end{itemize}


% ====================================================================
\subsection{SUNLinearSolver\_SuperLUMT functions}
\label{ss:sunlinsol_slumt_functions}
% ====================================================================

The module {\sunlinsolslumt} provides the following user-callable constructor
for creating a \newline \id{SUNLinearSolver} object.
%
% --------------------------------------------------------------------
%
\ucfunctiond{SUNLinSol\_SuperLUMT}
{
  LS = SUNLinSol\_SuperLUMT(y, A, num\_threads);
}
{
  The function \ID{SUNLinSol\_SuperLUMT} creates and allocates memory for
  a SuperLU\_MT-based \id{SUNLinearSolver} object.
}
{
  \begin{args}[num\_threads]
  \item[y] (\id{N\_Vector})
    a template for cloning vectors needed within the solver
  \item[A] (\id{SUNMatrix})
    a {\sunmatsparse} matrix template for cloning matrices needed
    within the solver 
  \item[num\_threads] (\id{int})
    desired number of threads (OpenMP or Pthreads, depending on how
    {\superlumt} was installed) to use during the factorization steps
  \end{args}
}
{
  This returns a \id{SUNLinearSolver} object.  If either \id{A} or
  \id{y} are incompatible then this routine will return \id{NULL}.
}
{
  This routine analyzes the input matrix and vector to determine the
  linear system size and to assess compatibility with the {\superlumt}
  library.

  This routine will perform consistency checks to ensure that it is
  called with consistent {\nvector} and {\sunmatrix} implementations.
  These are currently limited to the {\sunmatsparse} matrix type
  (using either CSR or CSC storage formats) and the {\nvecs},
  {\nvecopenmp}, and {\nvecpthreads} vector types.  As additional
  compatible matrix and vector implementations are added to
  {\sundials}, these will be included within this compatibility
  check.

  The \id{num\_threads} argument is not checked and is passed directly
  to {\superlumt} routines.
}
{SUNSuperLUMT}
%
% --------------------------------------------------------------------
%
\noindent The {\sunlinsolslumt} module defines implementations of all
``direct'' linear solver operations listed in Sections
\ref{ss:sunlinsol_CoreFn} -- \ref{ss:sunlinsol_GetFn}:
\begin{itemize}
\item \id{SUNLinSolGetType\_SuperLUMT}
\item \id{SUNLinSolInitialize\_SuperLUMT} -- this sets the
  \id{first\_factorize} flag to 1 and resets the internal {\superlumt}
  statistics variables.
\item \id{SUNLinSolSetup\_SuperLUMT} -- this performs either a $LU$
  factorization or refactorization of the input matrix.
\item \id{SUNLinSolSolve\_SuperLUMT} -- this calls the appropriate
  {\superlumt} solve routine to utilize the $LU$ factors to solve the
  linear system. 
\item \id{SUNLinSolLastFlag\_SuperLUMT}
\item \id{SUNLinSolSpace\_SuperLUMT} -- this only returns information for
  the storage within the solver \emph{interface}, i.e.~storage for the
  integers \id{last\_flag} and \id{first\_factorize}.  For additional
  space requirements, see the {\superlumt} documentation.
\item \id{SUNLinSolFree\_SuperLUMT}
\end{itemize}

The {\sunlinsolslumt} module also defines the following additional
user-callable function.
%
% --------------------------------------------------------------------
%
\ucfunctiond{SUNLinSol\_SuperLUMTSetOrdering}
{
  retval = SUNLinSol\_SuperLUMTSetOrdering(LS, ordering);
}
{
  This function sets the ordering used by {\superlumt} for reducing fill in
  the linear solve.
}
{
  \begin{args}[ordering]
  \item[LS] (\id{SUNLinearSolver})
    the {\sunlinsolslumt} object
  \item[ordering] (\id{int})
    a flag indicating the ordering algorithm to use, the options are:
    \begin{itemize}
    \item[0] natural ordering
    \item[1] minimal degree ordering on $A^TA$
    \item[2] minimal degree ordering on $A^T+A$
    \item[3] COLAMD ordering for unsymmetric matrices
    \end{itemize}
    The default is 3 for COLAMD.
  \end{args}
}
{
  The return values from this function are \id{SUNLS\_MEM\_NULL}
  (\id{S} is \id{NULL}), \newline \id{SUNLS\_ILL\_INPUT}
  (invalid ordering choice), or \id{SUNLS\_SUCCESS}.
}
{}
{SUNSuperLUMTSetOrdering}

% ====================================================================
\subsection{SUNLinearSolver\_SuperLUMT Fortran interfaces}
\label{ss:sunlinsol_slumt_fortran}
% ====================================================================

For solvers that include a Fortran interface module, the
{\sunlinsolslumt} module also includes a Fortran-callable function
for creating a \id{SUNLinearSolver} object.
%
% --------------------------------------------------------------------
%
\ucfunction{FSUNSUPERLUMTINIT}
{
  FSUNSUPERLUMTINIT(code, num\_threads, ier)
}
{
  The function \ID{FSUNSUPERLUMTINIT} can be called for Fortran programs
  to create a {\sunlinsolklu} object.
}
{
  \begin{args}[num\_threads]
  \item[code] (\id{int*})
    is an integer input specifying the solver id (1 for {\cvode}, 2
    for {\ida}, 3 for {\kinsol}, and 4 for {\arkode}).
  \item[num\_threads] (\id{int*})
    desired number of threads (OpenMP or Pthreads, depending on how
    {\superlumt} was installed) to use during the factorization steps
  \end{args}
}
{
  \id{ier} is a return completion flag equal to \id{0} for a success
  return and \id{-1} otherwise. See printed message for details in case
  of failure.
}
{
  This routine must be
  called \emph{after} both the {\nvector} and {\sunmatrix} objects have
  been initialized.
}
Additionally, when using {\arkode} with a non-identity
mass matrix, the {\sunlinsolslumt} module includes a Fortran-callable
function for creating a \id{SUNLinearSolver} mass matrix solver
object.
%
% --------------------------------------------------------------------
%
\ucfunction{FSUNMASSSUPERLUMTINIT}
{
  FSUNMASSSUPERLUMTINIT(num\_threads, ier)
}
{
  The function \ID{FSUNMASSSUPERLUMTINIT} can be called for Fortran programs
  to create a SuperLU\_MT-based \id{SUNLinearSolver} object for mass matrix linear
  systems.
}
{
  \begin{args}[num\_threads]
  \item[num\_threads] (\id{int*})
    desired number of threads (OpenMP or Pthreads, depending on how
    {\superlumt} was installed) to use during the factorization steps.
  \end{args}
}
{
  \id{ier} is a \id{int} return completion flag equal to \id{0} for a success
  return and \id{-1} otherwise. See printed message for details in case
  of failure.
}
{
  This routine must be
  called \emph{after} both the {\nvector} and {\sunmatrix} mass-matrix
  objects have been initialized.
}
The \ID{SUNLinSol\_SuperLUMTSetOrdering} routine also supports Fortran
interfaces for the system and mass matrix solvers:
%
% --------------------------------------------------------------------
%
\ucfunction{FSUNSUPERLUMTSETORDERING}
{
  FSUNSUPERLUMTSETORDERING(code, ordering, ier)
}
{
  The function \ID{FSUNSUPERLUMTSETORDERING} can be called for Fortran programs
  to update the ordering algorithm in a {\sunlinsolslumt} object.
}
{
  \begin{args}[ordering]
  \item[code] (\id{int*})
    is an integer input specifying the solver id (1 for {\cvode}, 2
    for {\ida}, 3 for {\kinsol}, and 4 for {\arkode}).
  \item[ordering] (\id{int*})
    a flag indicating the ordering algorithm, options are:
    \begin{itemize}
    \item[0] natural ordering
    \item[1] minimal degree ordering on $A^TA$
    \item[2] minimal degree ordering on $A^T+A$
    \item[3] COLAMD ordering for unsymmetric matrices
    \end{itemize}
    The default is 3 for COLAMD.
  \end{args}
}
{
  \id{ier} is a \id{int} return completion flag equal to \id{0} for a success
  return and \id{-1} otherwise. See printed message for details in case
  of failure.
}
{
  See \id{SUNLinSol\_SuperLUMTSetOrdering} for complete further
  documentation of this routine. 
}
%
% --------------------------------------------------------------------
%
\ucfunction{FSUNMASSUPERLUMTSETORDERING}
{
  FSUNMASSUPERLUMTSETORDERING(ordering, ier)
}
{
  The function \ID{FSUNMASSUPERLUMTSETORDERING} can be called for Fortran
  programs to update the ordering algorithm in a {\sunlinsolslumt}
  object for mass matrix linear systems.
}
{
  \begin{args}[ordering]
  \item[ordering] (\id{int*})
    a flag indicating the ordering algorithm, options are:
    \begin{itemize}
    \item[0] natural ordering
    \item[1] minimal degree ordering on $A^TA$
    \item[2] minimal degree ordering on $A^T+A$
    \item[3] COLAMD ordering for unsymmetric matrices
    \end{itemize}
    The default is 3 for COLAMD.
  \end{args}
}
{
  \id{ier} is a \id{int} return completion flag equal to \id{0} for a success
  return and \id{-1} otherwise. See printed message for details in case
  of failure.
}
{
  See \id{SUNLinSol\_SuperLUMTSetOrdering} for complete further
  documentation of this routine. 
}


% ====================================================================
\subsection{SUNLinearSolver\_SuperLUMT content}
\label{ss:sunlinsol_slumt_content}
% ====================================================================

The {\sunlinsolslumt} module defines the \textit{content} field of a
\id{SUNLinearSolver} as the following structure:
%%
\begin{verbatim} 
struct _SUNLinearSolverContent_SuperLUMT {
  long int     last_flag;
  int          first_factorize;
  SuperMatrix  *A, *AC, *L, *U, *B;
  Gstat_t      *Gstat;
  sunindextype *perm_r, *perm_c;
  sunindextype N;
  int          num_threads;
  realtype     diag_pivot_thresh; 
  int          ordering;
  superlumt_options_t *options;
};
\end{verbatim}
%%
These entries of the \emph{content} field contain the following
information:
\begin{args}[diag\_pivot\_thresh]
  \item[last\_flag] - last error return flag from internal function evaluations,
  \item[first\_factorize] - flag indicating whether the factorization
    has ever been performed, 
  \item[A, AC, L, U, B] - \id{SuperMatrix} pointers used in solve,
  \item[Gstat] - \id{GStat\_t} object used in solve,
  \item[perm\_r, perm\_c] - permutation arrays used in solve,
  \item[N] - size of the linear system,
  \item[num\_threads] - number of OpenMP/Pthreads threads to use,
  \item[diag\_pivot\_thresh] - threshold on diagonal pivoting,
  \item[ordering] - flag for which reordering algorithm to use,
  \item[options] - pointer to {\superlumt} options structure.
\end{args}



%---------------------------------------------------------------------------
\section{The SUNLinearSolver\_SPGMR implementation}\label{ss:sunlinsol_spgmr}
%% This is a shared SUNDIALS TEX file with a description of the
%% spgmr sunlinsol implementation
%%

The {\spgmr} (Scaled, Preconditioned, Generalized Minimum
Residual \cite{SaSc:86}) implementation of the {\sunlinsol} module
provided with {\sundials}, {\sunlinsolspgmr}, is an iterative linear
solver that is designed to be compatible with any {\nvector}
implementation (serial, threaded, parallel, user-supplied) that
supports a minimal subset of operations (\id{N\_VClone}, 
\id{N\_VDotProd}, \id{N\_VScale}, \id{N\_VLinearSum}, \id{N\_VProd},
\id{N\_VConst}, \id{N\_VDiv} and \id{N\_VDestroy}).  

The {\sunlinsolspgmr} module defines the {\em content} field of a
\id{SUNLinearSolver} to be the following structure:
%%
\begin{verbatim} 
struct _SUNLinearSolverContent_SPGMR {
  int maxl;
  int pretype;
  int gstype;
  int max_restarts;
  int numiters;
  realtype resnorm;
  long int last_flag;
  ATimesFn ATimes;
  void* ATData;
  PSetupFn Psetup;
  PSolveFn Psolve;
  void* PData;
  N_Vector s1;
  N_Vector s2;
  N_Vector *V;
  realtype **Hes;
  realtype *givens;
  N_Vector xcor;
  realtype *yg;
  N_Vector vtemp;
};
\end{verbatim}
%%
These entries of the \emph{content} field contain the following
information:
\begin{description}
  \item[maxl] - number of GMRES basis vectors to use (default is 5)
  \item[pretype] - flag for type of preconditioning to employ
    (default is none)
  \item[gstype] - flag for type of Gram-Schmidt orthogonalization
    (default is modified Gram-Schmidt)
  \item[max\_restarts] - number of GMRES restarts to allow
    (default is 0) 
  \item[numiters] - number of iterations from most-recent solve
  \item[resnorm] - final linear residual norm from most-recent solve
  \item[last\_flag] - last error return flag from internal function
  \item[ATimes] - function pointer to perform $Av$ product
  \item[ATData] - pointer to structure for \id{ATimes}
  \item[Psetup] - function pointer to preconditioner setup routine
  \item[Psolve] - function pointer to preconditioner solve routine
  \item[PData] - pointer to structure for \id{Psetup}, \id{Psolve}
  \item[s1, s2] - vector pointers for supplied scaling matrices
    (default are \id{NULL})
  \item[V] - the array of Krylov basis vectors
    $v_1, \ldots, v_{\text{\id{maxl}}+1}$, stored in \id{V[0]},
    \ldots, \id{V[maxl]}. Each $v_i$ is a vector of type {\nvector}.
  \item[Hes] - the $(\text{\id{maxl}}+1)\times\text{\id{maxl}}$
    Hessenberg matrix. It is stored row-wise so that the (i,j)th
    element is given by \id{Hes[i][j]}. 
  \item[givens] - a length \id{2*maxl} array which represents the
    Givens rotation matrices that arise in the GMRES algorithm. These
    matrices are $F_0, F_1, \ldots, F_j$, where
    $F_i = \begin{bmatrix}
      1 &        &   &     &      &   &        &   \\
        & \ddots &   &     &      &   &        &   \\
        &        & 1 &     &      &   &        &   \\
        &        &   & c_i & -s_i &   &        &   \\
        &        &   & s_i &  c_i &   &        &   \\
        &        &   &     &      & 1 &        &   \\
        &        &   &     &      &   & \ddots &   \\
        &        &   &     &      &   &        & 1\end{bmatrix}$
    are represented in the \id{givens} vector as \id{givens[0] =}
    $c_0$, \id{givens[1] = } $s_0$, \id{givens[2] = } $c_1$,
    \id{givens[3] = } $s_1$, \ldots \id{givens[2j] = } $c_j$,
    \id{givens[2j+1] = } $s_j$.
  \item[xcor] - a vector which holds the scaled, preconditioned
    correction to the initial guess 
  \item[yg] - a length \id{(maxl+1)} array of \id{realtype} values
    used to hold ``short'' vectors (e.g. $y$ and $g$).
  \item[vtemp] - temporary vector storage
\end{description}

This solver is constructed to perform the following operations:
\begin{itemize}
\item During construction, the \id{xcor} and \id{vtemp} arrays are
  cloned from a template {\nvector} that is input, and default solver
  parameters are set.
\item User-facing ``set'' routines may be called to modify default
  solver parameters.
\item Additional ``set'' routines are called by the {\sundials} solver
  that interfaces with {\sunlinsolspgmr} to supply the 
  \id{ATimes}, \id{PSetup} and \id{Psolve} function pointers and
  \id{s1} and \id{s2} scaling vectors.
\item In the ``initialize'' call, the remaining solver data is
  allocated (\id{V}, \id{Hes}, \id{givens}, \id{yg} )
\item In the ``setup'' call, any non-\id{NULL} 
  \id{PSetup} function is called.  Typically, this is provided by
  the {\sundials} solver itself, that translates between the
  generic \id{PSetup} function and the
  solver-specific routine (solver-supplied or user-supplied).
\item In the ``solve'' call the GMRES iteration is performed.  This
  will include scaling, preconditioning and restarts if those options
  have been supplied.
\end{itemize}

\noindent The header file to be included when using this module 
is \id{sunlinsol/sunlinsol\_spgmr.h}. \\
%%
%%----------------------------------------------
%%
The {\sunlinsolspgmr} module defines implementations of all
``iterative'' linear solver operations listed in Table
\ref{t:sunlinsolops}:
\begin{itemize}
\item \id{SUNLinSolGetType\_SPGMR}
\item \id{SUNLinSolInitialize\_SPGMR}
\item \id{SUNLinSolSetATimes\_SPGMR}
\item \id{SUNLinSolSetPreconditioner\_SPGMR}
\item \id{SUNLinSolSetScalingVectors\_SPGMR}
\item \id{SUNLinSolSetup\_SPGMR}
\item \id{SUNLinSolSolve\_SPGMR}
\item \id{SUNLinSolNumIters\_SPGMR}
\item \id{SUNLinSolResNorm\_SPGMR}
\item \id{SUNLinSolResid\_SPGMR}
\item \id{SUNLinSolLastFlag\_SPGMR}
\item \id{SUNLinSolSpace\_SPGMR}
\item \id{SUNLinSolFree\_SPGMR}
\end{itemize}
The module {\sunlinsolspgmr} provides the following additional
user-callable routines: 
%%
\begin{itemize}

%%--------------------------------------

\item \ID{SUNSPGMR}

  This function creates and allocates memory for a {\spgmr}
  \id{SUNLinearSolver}.  Its arguments are an {\nvector}, the desired
  type of preconditioning, and the number of Krylov basis vectors to use.

  This routine will perform consistency checks to ensure that it is
  called with a consistent {\nvector} implementation (i.e.~that it
  supplies the requisite vector operations).  If \id{y} is
  incompatible then this routine will return \id{NULL}.

  A \id{maxl} argument that is $\le0$ will result in the default
  value (5).

  Allowable inputs for \id{pretype} are \id{PREC\_NONE} (0),
  \id{PREC\_LEFT} (1), \id{PREC\_RIGHT} (2) and \id{PREC\_BOTH} (3);
  any other integer input will result in the default (no
  preconditioning).  We note that some {\sundials} solvers are
  designed to only work with right preconditioning ({\kinsol}, {\ida},
  {\idas}).  While it is possible to configure a {\sunlinsolspgmr}
  object to use \id{PREC\_LEFT} or \id{PREC\_BOTH} with these solvers,
  this use mode is not supported and may result in inferior
  performance.

  \verb|SUNLinearSolver SUNSPGMR(N_Vector y, int pretype, int maxl);|

%%--------------------------------------

\item \ID{SUNSPGMRSetPrecType}

  This function updates the type of preconditioning to use.  Supported
  values are \id{PREC\_NONE} (0), \id{PREC\_LEFT} (1),
  \id{PREC\_RIGHT} (2) and \id{PREC\_BOTH} (3).  

  This routine will return with one of the error codes
  \id{SUNLS\_ILL\_INPUT} (illegal \id{pretype}), \id{SUNLS\_MEM\_NULL}
  (\id{S} is \id{NULL}) or \id{SUNLS\_SUCCESS}.
  
  \verb|int SUNSPGMRSetPrecType(SUNLinearSolver S, int pretype);|

%%--------------------------------------

\item \ID{SUNSPGMRSetGSType}

  This function sets the type of Gram-Schmidt orthogonalization to
  use.  Supported values are \id{MODIFIED\_GS} (1) and
  \id{CLASSICAL\_GS} (2).  Any other integer input will result in a
  failure, returning error code \id{SUNLS\_ILL\_INPUT}.

  This routine will return with one of the error codes
  \id{SUNLS\_ILL\_INPUT} (illegal \id{gstype}), \id{SUNLS\_MEM\_NULL}
  (\id{S} is \id{NULL}) or \id{SUNLS\_SUCCESS}.
  
  \verb|int SUNSPGMRSetGSType(SUNLinearSolver S, int gstype);|


%%--------------------------------------

\item \ID{SUNSPGMRSetMaxRestarts}

  This function sets the number of GMRES restarts to 
  allow.  A negative input will result in the default of 0.

  This routine will return with one of the error codes
  \id{SUNLS\_MEM\_NULL} (\id{S} is \id{NULL}) or \id{SUNLS\_SUCCESS}.
  
  \verb|int SUNSPGMRSetMaxRestarts(SUNLinearSolver S, int maxrs);|

\end{itemize}
%%
%%------------------------------------
%%
For solvers that include a Fortran interface module, the
{\sunlinsolspgmr} module also includes the Fortran-callable
function \id{FSUNSPGMRInit(code, pretype, maxl, ier)} to initialize
this {\sunlinsolspgmr} module for a given {\sundials} solver.
Here \id{code} is an input solver id (1 for {\cvode}, 2 for {\ida}, 3
for {\kinsol}, 4 for {\arkode}); \id{pretype} and \id{maxl} are the
same as for the C function \ID{SUNSPGMR}; \id{ier} is an error return
flag equal 0 for success and -1 for failure.  All of these input
arguments should be declared so as to match C type \id{int}).  This
routine must be called \emph{after} the {\nvector} object has been
initialized.  Additionally, when using {\arkode} with non-identity
mass matrix, the Fortran-callable
function \id{FSUNMassSPGMRInit(pretype, maxl, ier)} initializes this 
{\sunlinsolspgmr} module for solving mass matrix linear systems.

The \id{SUNSPGMRSetPrecType}, \id{SUNSPGMRSetGSType} and
\id{SUNSPGMRSetMaxRestarts} routines also support Fortran interfaces
for the system and mass matrix solvers:
\begin{itemize}
\item \id{FSUNSPGMRSetGSType(code, gstype, ier)} -- all arguments
  should be commensurate with a C \id{int}
\item \id{FSUNMassSPGMRSetGSType(gstype, ier)}
\item \id{FSUNSPGMRSetPrecType(code, pretype, ier)} -- all arguments
  should be commensurate with a C \id{int}
\item \id{FSUNMassSPGMRSetPrecType(pretype, ier)}
\item \id{FSUNSPGMRSetMaxRS(code, maxrs, ier)} -- all arguments
  should be commensurate with a C \id{int}
\item \id{FSUNMassSPGMRSetMaxRS(maxrs, ier)}
\end{itemize}


%---------------------------------------------------------------------------
\section{The SUNLinearSolver\_SPFGMR implementation}\label{ss:sunlinsol_spfgmr}
%% This is a shared SUNDIALS TEX file with a description of the
%% spfgmr sunlinsol implementation
%%

The {\spfgmr} (Scaled, Preconditioned, Flexible, Generalized Minimum
Residual \cite{Saa:93}) implementation of the {\sunlinsol} module
provided with {\sundials}, {\sunlinsolspfgmr}, is an iterative linear
solver that is designed to be compatible with any {\nvector}
implementation (serial, threaded, parallel, and user-supplied) that
supports a minimal subset of operations (\id{N\_VClone},
\id{N\_VDotProd}, \id{N\_VScale}, \id{N\_VLinearSum}, \id{N\_VProd},
\id{N\_VConst}, \id{N\_VDiv}, and \id{N\_VDestroy}).  Unlike the other
Krylov iterative linear solvers supplied with {\sundials}, FGMRES is
specifically designed to work with a changing preconditioner
(e.g.~from an iterative method).

The {\sunlinsolspfgmr} module defines the {\em content} field of a
\id{SUNLinearSolver} to be the following structure:
%%
\begin{verbatim} 
struct _SUNLinearSolverContent_SPFGMR {
  int maxl;
  int pretype;
  int gstype;
  int max_restarts;
  int numiters;
  realtype resnorm;
  long int last_flag;
  ATimesFn ATimes;
  void* ATData;
  PSetupFn Psetup;
  PSolveFn Psolve;
  void* PData;
  N_Vector s1;
  N_Vector s2;
  N_Vector *V;
  N_Vector *Z;
  realtype **Hes;
  realtype *givens;
  N_Vector xcor;
  realtype *yg;
  N_Vector vtemp;
};
\end{verbatim}
%%
These entries of the \emph{content} field contain the following
information:
\begin{description}
  \item[maxl] - number of FGMRES basis vectors to use (default is 5),
  \item[pretype] - flag for use of preconditioning (default is none),
  \item[gstype] - flag for type of Gram-Schmidt orthogonalization
    (default is modified Gram-Schmidt),
  \item[max\_restarts] - number of FGMRES restarts to allow
    (default is 0),
  \item[numiters] - number of iterations from the most-recent solve,
  \item[resnorm] - final linear residual norm from the most-recent solve,
  \item[last\_flag] - last error return flag from an internal function,
  \item[ATimes] - function pointer to perform $Av$ product,
  \item[ATData] - pointer to structure for \id{ATimes},
  \item[Psetup] - function pointer to preconditioner setup routine,
  \item[Psolve] - function pointer to preconditioner solve routine,
  \item[PData] - pointer to structure for \id{Psetup} and \id{Psolve},
  \item[s1, s2] - vector pointers for supplied scaling matrices
    (default is \id{NULL}),
  \item[V] - the array of Krylov basis vectors
    $v_1, \ldots, v_{\text{\id{maxl}}+1}$, stored in \id{V[0]},
    \ldots, \id{V[maxl]}. Each $v_i$ is a vector of type {\nvector}.,
  \item[Z] - the array of preconditioned Krylov basis vectors
    $z_1, \ldots, z_{\text{\id{maxl}}+1}$, stored in \id{Z[0]},
    \ldots, \id{Z[maxl]}. Each $z_i$ is a vector of type {\nvector}.,
  \item[Hes] - the $(\text{\id{maxl}}+1)\times\text{\id{maxl}}$
    Hessenberg matrix. It is stored row-wise so that the (i,j)th
    element is given by \id{Hes[i][j]}.,
  \item[givens] - a length \id{2*maxl} array which represents the
    Givens rotation matrices that arise in the FGMRES algorithm. These
    matrices are $F_0, F_1, \ldots, F_j$, where
    $F_i = \begin{bmatrix}
      1 &        &   &     &      &   &        &   \\
        & \ddots &   &     &      &   &        &   \\
        &        & 1 &     &      &   &        &   \\
        &        &   & c_i & -s_i &   &        &   \\
        &        &   & s_i &  c_i &   &        &   \\
        &        &   &     &      & 1 &        &   \\
        &        &   &     &      &   & \ddots &   \\
        &        &   &     &      &   &        & 1\end{bmatrix}$,
    are represented in the \id{givens} vector as \id{givens[0] =}
    $c_0$, \id{givens[1] = } $s_0$, \id{givens[2] = } $c_1$,
    \id{givens[3] = } $s_1$, \ldots \id{givens[2j] = } $c_j$,
    \id{givens[2j+1] = } $s_j$.,
  \item[xcor] - a vector which holds the scaled, preconditioned
    correction to the initial guess,
  \item[yg] - a length \id{(maxl+1)} array of \id{realtype} values
    used to hold ``short'' vectors (e.g. $y$ and $g$),
  \item[vtemp] - temporary vector storage.
\end{description}

This solver is constructed to perform the following operations:
\begin{itemize}
\item During construction, the \id{xcor} and \id{vtemp} arrays are
  cloned from a template {\nvector} that is input, and default solver
  parameters are set.
\item User-facing ``set'' routines may be called to modify default
  solver parameters.
\item Additional ``set'' routines are called by the {\sundials} solver
  that interfaces with {\sunlinsolspfgmr} to supply the 
  \id{ATimes}, \id{PSetup}, and \id{Psolve} function pointers and
  \id{s1} and \id{s2} scaling vectors.
\item In the ``initialize'' call, the remaining solver data is
  allocated (\id{V}, \id{Hes}, \id{givens}, and \id{yg} )
\item In the ``setup'' call, any non-\id{NULL}
  \id{PSetup} function is called.  Typically, this is provided by
  the {\sundials} solver itself, that translates between the
  generic \id{PSetup} function and the
  solver-specific routine (solver-supplied or user-supplied).
\item In the ``solve'' call, the FGMRES iteration is performed.  This
  will include scaling, preconditioning, and restarts if those options
  have been supplied.
\end{itemize}

\noindent The header file to include when using this module 
is \id{sunlinsol/sunlinsol\_spfgmr.h}. The {\sunlinsolspfgmr} module
is accessible from all {\sundials} solvers \textit{without}
linking to the \\
\id{libsundials\_sunlinsolspfgmr} module library. \\

%%
%%----------------------------------------------
%%

\noindent The {\sunlinsolspfgmr} module defines implementations of all
``iterative'' linear solver operations listed in Table
\ref{t:sunlinsolops}:
\begin{itemize}
\item \id{SUNLinSolGetType\_SPFGMR}
\item \id{SUNLinSolInitialize\_SPFGMR}
\item \id{SUNLinSolSetATimes\_SPFGMR}
\item \id{SUNLinSolSetPreconditioner\_SPFGMR}
\item \id{SUNLinSolSetScalingVectors\_SPFGMR}
\item \id{SUNLinSolSetup\_SPFGMR}
\item \id{SUNLinSolSolve\_SPFGMR}
\item \id{SUNLinSolNumIters\_SPFGMR}
\item \id{SUNLinSolResNorm\_SPFGMR}
\item \id{SUNLinSolResid\_SPFGMR}
\item \id{SUNLinSolLastFlag\_SPFGMR}
\item \id{SUNLinSolSpace\_SPFGMR}
\item \id{SUNLinSolFree\_SPFGMR}
\end{itemize}
The module {\sunlinsolspfgmr} provides the following additional
user-callable routines: 
%%
\begin{itemize}

%%--------------------------------------

\item \ID{SUNSPFGMR}

  This constructor function creates and allocates memory for a {\spfgmr}
  \id{SUNLinearSolver}.  Its arguments are an {\nvector}, a flag
  indicating to use preconditioning, and the number of Krylov basis
  vectors to use. 

  This routine will perform consistency checks to ensure that it is
  called with a consistent {\nvector} implementation (i.e.~that it
  supplies the requisite vector operations).  If \id{y} is
  incompatible, then this routine will return \id{NULL}.

  A \id{maxl} argument that is $\le0$ will result in the default
  value (5).

  Since the FGMRES algorithm is designed to only support right
  preconditioning, then any of the \id{pretype}
  inputs \id{PREC\_LEFT} (1), \id{PREC\_RIGHT} (2), or \id{PREC\_BOTH}
  (3) will result in use of \id{PREC\_RIGHT};  any other integer input
  will result in the default (no preconditioning).  We note that some
  SUNDIALS solvers are designed to only work with left preconditioning
  ({\ida} and {\idas}). While it is possible to use a
  right-preconditioned {\sunlinsolspfgmr} object for these packages,
  this use mode is not supported and may result in inferior
  performance.

  \verb|SUNLinearSolver SUNSPFGMR(N_Vector y, int pretype, int maxl);|

%%--------------------------------------

\item \ID{SUNSPFGMRSetPrecType}

  This function updates the flag indicating use of preconditioning.
  Since the FGMRES algorithm is designed to only support right
  preconditioning, then any of the \id{pretype}
  inputs \id{PREC\_LEFT} (1), \id{PREC\_RIGHT} (2), or \id{PREC\_BOTH}
  (3) will result in use of \id{PREC\_RIGHT};  any other integer input
  will result in the default (no preconditioning).

  This routine will return with one of the error codes
  \id{SUNLS\_MEM\_NULL} (\id{S} is \id{NULL}) or \id{SUNLS\_SUCCESS}.
  
  \verb|int SUNSPFGMRSetPrecType(SUNLinearSolver S, int pretype);|

%%--------------------------------------

\item \ID{SUNSPFGMRSetGSType}

  This function sets the type of Gram-Schmidt orthogonalization to
  use.  Supported values are \id{MODIFIED\_GS} (1) and
  \id{CLASSICAL\_GS} (2).  Any other integer input will result in a
  failure, returning error code \id{SUNLS\_ILL\_INPUT}.

  This routine will return with one of the error codes
  \id{SUNLS\_ILL\_INPUT} (illegal \id{gstype}), \\ \noindent
  \id{SUNLS\_MEM\_NULL} (\id{S} is \id{NULL}), or \id{SUNLS\_SUCCESS}.
  
  \verb|int SUNSPFGMRSetGSType(SUNLinearSolver S, int gstype);|


%%--------------------------------------

\item \ID{SUNSPFGMRSetMaxRestarts}

  This function sets the number of FGMRES restarts to 
  allow.  A negative input will result in the default of 0.

  This routine will return with one of the error codes
  \id{SUNLS\_MEM\_NULL} (\id{S} is \id{NULL}) or \id{SUNLS\_SUCCESS}.
  
  \verb|int SUNSPFGMRSetMaxRestarts(SUNLinearSolver S, int maxrs);|

\end{itemize}
%%
%%------------------------------------
%%
For solvers that include a Fortran interface module, the
{\sunlinsolspfgmr} module also includes the Fortran-callable
function \id{FSUNSPFGMRInit(code, pretype, maxl, ier)} to initialize
this {\sunlinsolspfgmr} module for a given {\sundials} solver.
Here \id{code} is an integer input solver id (1 for {\cvode}, 2 for {\ida}, 3
for {\kinsol}, 4 for {\arkode}); \id{pretype} and \id{maxl} are the
same as for the C function \ID{SUNSPFGMR}; \id{ier} is an error return
flag equal to 0 for success and -1 for failure.  All of these input
arguments should be declared so as to match C type \id{int}.  This
routine must be called \emph{after} the {\nvector} object has been
initialized.  Additionally, when using {\arkode} with a non-identity
mass matrix, the Fortran-callable
function \id{FSUNMassSPFGMRInit(pretype, maxl, ier)} initializes this 
{\sunlinsolspfgmr} module for solving mass matrix linear systems.

The \id{SUNSPFGMRSetPrecType}, \id{SUNSPFGMRSetGSType}, and
\id{SUNSPFGMRSetMaxRestarts} routines also support Fortran interfaces
for the system and mass matrix solvers (all arguments should be
commensurate with a C \id{int}): 
\begin{itemize}
\item \id{FSUNSPFGMRSetGSType(code, gstype, ier)}
\item \id{FSUNMassSPFGMRSetGSType(gstype, ier)}
\item \id{FSUNSPFGMRSetPrecType(code, pretype, ier)}
\item \id{FSUNMassSPFGMRSetPrecType(pretype, ier)}
\item \id{FSUNSPFGMRSetMaxRS(code, maxrs, ier)}
\item \id{FSUNMassSPFGMRSetMaxRS(maxrs, ier)}
\end{itemize}


%---------------------------------------------------------------------------
\section{The SUNLinearSolver\_SPBCGS implementation}\label{ss:sunlinsol_spbcgs}
% ====================================================================
\section{The SUNLinearSolver\_SPBCGS implementation}
\label{ss:sunlinsol_spbcgs}
% ====================================================================

This section describes the {\sunlinsol} implementation of the {\spbcgs}
(Scaled, Preconditioned, Bi-Conjugate Gradient, Stabilized \cite{Van:92})
iterative linear solver. The {\sunlinsolspbcgs} module is designed to be
compatible with any {\nvector} implementation that supports a minimal subset
of operations (\id{N\_VClone}, \id{N\_VDotProd}, \id{N\_VScale},
\id{N\_VLinearSum}, \id{N\_VProd}, \id{N\_VDiv}, and
\id{N\_VDestroy}). Unlike the {\spgmr} and {\spfgmr} algorithms, {\spbcgs}
requires a fixed amount of memory that does not increase with the number of
allowed iterations.

To access the {\sunlinsolspbcgs} module, include the header file
\id{sunlinsol/sunlinsol\_spbcgs.h}. We note that the {\sunlinsolspbcgs} module is
accessible from {\sundials} packages \textit{without} separately linking to
the \id{libsundials\_sunlinsolspbcgs} module library.


% ====================================================================
\subsection{SUNLinearSolver\_SPBCGS description}
\label{ss:sunlinsol_spbcgs_description}
% ====================================================================

This solver is constructed to perform the following operations:
\begin{itemize}
\item During construction all {\nvector} solver data is allocated,
  with vectors cloned from a template {\nvector} that is input, and
  default solver parameters are set.
\item User-facing ``set'' routines may be called to modify default
  solver parameters.
\item Additional ``set'' routines are called by the {\sundials} solver
  that interfaces with {\sunlinsolspbcgs} to supply the
  \id{ATimes}, \id{PSetup}, and \id{Psolve} function pointers and
  \id{s1} and \id{s2} scaling vectors.
\item In the ``initialize'' call, the solver parameters are checked
  for validity.
\item In the ``setup'' call, any non-\id{NULL}
  \id{PSetup} function is called.  Typically, this is provided by
  the {\sundials} solver itself, that translates between the
  generic \id{PSetup} function and the
  solver-specific routine (solver-supplied or user-supplied).
\item In the ``solve'' call the {\spbcgs} iteration is performed.  This
  will include scaling and preconditioning if those options have been
  supplied.
\end{itemize}


% ====================================================================
\subsection{SUNLinearSolver\_SPBCGS functions}
\label{ss:sunlinsol_spbcgs_functions}
% ====================================================================

The {\sunlinsolspbcgs} module provides the following user-callable constructor
for creating a \newline \id{SUNLinearSolver} object.
%
% --------------------------------------------------------------------
%
\ucfunctiondf{SUNLinSol\_SPBCGS}
{
  LS = SUNLinSol\_SPBCGS(y, pretype, maxl);
}
{
  The function \ID{SUNLinSol\_SPBCGS} creates and allocates memory for
  a {\spbcgs} \newline \id{SUNLinearSolver} object.
}
{
  \begin{args}[pretype]
  \item[y] (\id{N\_Vector})
    a template for cloning vectors needed within the solver
  \item[pretype] (\id{int})
    flag indicating the desired type of preconditioning, allowed
    values are:
    \begin{itemize}
    \item \id{PREC\_NONE} (0)
    \item \id{PREC\_LEFT} (1)
    \item \id{PREC\_RIGHT} (2)
    \item \id{PREC\_BOTH} (3)
    \end{itemize}
    Any other integer input will result in the default (no
    preconditioning).
  \item[maxl] (\id{int})
    the number of linear iterations to allow. Values $\le0$ will
    result in the default value (5).
  \end{args}
}
{
  This returns a \id{SUNLinearSolver} object.  If either \id{y} is
  incompatible then this routine will return \id{NULL}.
}
{
  This routine will perform consistency checks to ensure that it is
  called with a consistent {\nvector} implementation (i.e.~that it
  supplies the requisite vector operations).  If \id{y} is
  incompatible, then this routine will return \id{NULL}.

  We note that some {\sundials} solvers are designed to only work
  with left preconditioning ({\ida} and {\idas}) and others with only
  right preconditioning ({\kinsol}). While it is possible to configure
  a {\sunlinsolspbcgs} object to use any of the preconditioning options
  with these solvers, this use mode is not supported and may result in
  inferior performance.
}
{SUNSPBCGS}
%
% --------------------------------------------------------------------
%
The {\sunlinsolspbcgs} module defines implementations of all
``iterative'' linear solver operations listed in Sections
\ref{ss:sunlinsol_CoreFn} -- \ref{ss:sunlinsol_GetFn}:
\begin{itemize}
\item \id{SUNLinSolGetType\_SPBCGS}
\item \id{SUNLinSolInitialize\_SPBCGS}
\item \id{SUNLinSolSetATimes\_SPBCGS}
\item \id{SUNLinSolSetPreconditioner\_SPBCGS}
\item \id{SUNLinSolSetScalingVectors\_SPBCGS}
\item \id{SUNLinSolSetup\_SPBCGS}
\item \id{SUNLinSolSolve\_SPBCGS}
\item \id{SUNLinSolNumIters\_SPBCGS}
\item \id{SUNLinSolResNorm\_SPBCGS}
\item \id{SUNLinSolResid\_SPBCGS}
\item \id{SUNLinSolLastFlag\_SPBCGS}
\item \id{SUNLinSolSpace\_SPBCGS}
\item \id{SUNLinSolFree\_SPBCGS}
\end{itemize}
All of the listed operations are callable via the {\F} 2003 interface module
by prepending an `F' to the function name.

The {\sunlinsolspbcgs} module also defines the following additional
user-callable functions.
%
% --------------------------------------------------------------------
%
\ucfunctiondf{SUNLinSol\_SPBCGSSetPrecType}
{
  retval = SUNLinSol\_SPBCGSSetPrecType(LS, pretype);
}
{
  The function \ID{SUNLinSol\_SPBCGSSetPrecType} updates the type of
  preconditioning to use in the {\sunlinsolspbcgs} object.
}
{
  \begin{args}[pretype]
  \item[LS] (\id{SUNLinearSolver})
    the {\sunlinsolspbcgs} object to update
  \item[pretype] (\id{int})
    flag indicating the desired type of preconditioning, allowed
    values match those discussed in \id{SUNLinSol\_SPBCGS}.
  \end{args}
}
{
  This routine will return with one of the error codes
  \id{SUNLS\_ILL\_INPUT} (illegal \id{pretype}), \id{SUNLS\_MEM\_NULL}
  (\id{S} is \id{NULL}) or \id{SUNLS\_SUCCESS}.
}
{}
{SUNSPBCGSSetPrecType}
%
% --------------------------------------------------------------------
%
\ucfunctiondf{SUNLinSol\_SPBCGSSetMaxl}
{
  retval = SUNLinSol\_SPBCGSSetMaxl(LS, maxl);
}
{
  The function \ID{SUNLinSol\_SPBCGSSetMaxl} updates the number of
  linear solver iterations to allow.
}
{
  \begin{args}[maxl]
  \item[LS] (\id{SUNLinearSolver})
    the {\sunlinsolspbcgs} object to update
  \item[maxl] (\id{int})
    flag indicating the number of iterations to allow. Values $\le0$
    will result in the default value (5).
  \end{args}
}
{
  This routine will return with one of the error codes
  \id{SUNLS\_MEM\_NULL} (\id{S} is \id{NULL}) or \id{SUNLS\_SUCCESS}.
}
{}
{SUNSPBCGSSetMaxl}
%
% --------------------------------------------------------------------
%
\ucfunctionf{SUNLinSolSetInfoFile\_SPBCGS}
{
  retval = SUNLinSolSetInfoFile\_SPBCGS(LS, info\_file);
}
{
  The function \ID{SUNLinSolSetInfoFile\_SPBCGS} sets the
  output file where all informative (non-error) messages should be directed.
}
{
  \begin{args}[info\_file]
    \item[LS] (\id{SUNLinearSolver})
      a {\sunnonlinsol} object
    \item[info\_file] (\id{FILE*}) pointer to output file (\id{stdout} by default);
      a \id{NULL} input will disable output
  \end{args}
}
{
  The return value is
  \begin{itemize}
    \item \id{SUNLS\_SUCCESS} if successful
    \item \id{SUNLS\_MEM\_NULL} if the SUNLinearSolver memory was \id{NULL}
    \item \id{SUNLS\_ILL\_INPUT} if {\sundials} was not built with monitoring enabled
  \end{itemize}
}
{
  This function is intended for users that wish to monitor the linear
  solver progress. By default, the file pointer is set to \id{stdout}.

  \textbf{{\sundials} must be built with the CMake option
  \id{SUNDIALS\_BUILD\_WITH\_MONITORING}, to utilize this function.}
  See section \ref{ss:configuration_options_nix} for more information.
}
%
% --------------------------------------------------------------------
%
\ucfunctionf{SUNLinSolSetPrintLevel\_SPBCGS}
{
  retval = SUNLinSolSetPrintLevel\_SPBCGS(NLS, print\_level);
}
{
  The function \ID{SUNLinSolSetPrintLevel\_SPBCGS} specifies the level
  of verbosity of the output.
}
{
  \begin{args}[print\_level]
  \item[LS] (\id{SUNLinearSolver})
    a {\sunnonlinsol} object
  \item[print\_level] (\id{int}) flag indicating level of verbosity;
    must be one of:
    \begin{itemize}
      \item 0, no information is printed (default)
      \item 1, for each linear iteration the residual norm is printed
    \end{itemize}
  \end{args}
}
{
  The return value is
  \begin{itemize}
    \item \id{SUNLS\_SUCCESS} if successful
    \item \id{SUNLS\_MEM\_NULL} if the SUNLinearSolver memory was \id{NULL}
    \item \id{SUNLS\_ILL\_INPUT} if {\sundials} was not built with monitoring enabled,
      or the print level value was invalid
  \end{itemize}
}
{
  This function is intended for users that wish to monitor the linear
  solver progress. By default, the print level is 0.

  \textbf{{\sundials} must be built with the CMake option
  \id{SUNDIALS\_BUILD\_WITH\_MONITORING}, to utilize this function.}
  See section \ref{ss:configuration_options_nix} for more information.
}


% ====================================================================
\subsection{SUNLinearSolver\_SPBCGS Fortran interfaces}
\label{ss:sunlinsol_spbcgs_fortran}
% ====================================================================

The {\sunlinsolspbcgs} module provides a {\F} 2003 module as well as {\F} 77
style interface functions for use from {\F} applications.

\subsubsection*{FORTRAN 2003 interface module}
The \ID{fsunlinsol\_spbcgs\_mod} {\F} module defines interfaces to all
{\sunlinsolspbcgs} {\CC} functions using the intrinsic \id{iso\_c\_binding}
module which provides a standardized mechanism for interoperating with {\CC}. As
noted in the {\CC} function descriptions above, the interface functions are
named after the corresponding {\CC} function, but with a leading `F'. For
example, the function \id{SUNLinSol\_SPBCGS} is interfaced as
\id{FSUNLinSol\_SPBCGS}.

The {\F} 2003 {\sunlinsolspbcgs} interface module can be accessed with the \id{use}
statement, i.e. \id{use fsunlinsol\_spbcgs\_mod}, and linking to the library
\id{libsundials\_fsunlinsolspbcgs\_mod}.{\em lib} in addition to the {\CC} library.
For details on where the library and module file \newline
\id{fsunlinsol\_spbcgs\_mod.mod} are installed see Appendix \ref{c:install}.
We note that the module is accessible from the {\F} 2003 {\sundials} integrators
\textit{without} separately linking to the \newline
\id{libsundials\_fsunlinsolspbcgs\_mod} library.

\subsubsection*{FORTRAN 77 interface functions}
For solvers that include a {\F} 77 interface module, the
{\sunlinsolspbcgs} module also includes a Fortran-callable function
for creating a \id{SUNLinearSolver} object.
%
% --------------------------------------------------------------------
%
\ucfunction{FSUNSPBCGSINIT}
{
  FSUNSPBCGSINIT(code, pretype, maxl, ier)
}
{
  The function \ID{FSUNSPBCGSINIT} can be called for Fortran programs
  to create a {\sunlinsolspbcgs} object.
}
{
  \begin{args}[pretype]
  \item[code] (\id{int*})
    is an integer input specifying the solver id (1 for {\cvode}, 2
    for {\ida}, 3 for {\kinsol}, and 4 for {\arkode}).
  \item[pretype] (\id{int*})
    flag indicating desired preconditioning type
  \item[maxl] (\id{int*})
    flag indicating number of iterations to allow
  \end{args}
}
{
  \id{ier} is a return completion flag equal to \id{0} for a success
  return and \id{-1} otherwise. See printed message for details in case
  of failure.
}
{
  This routine must be called \emph{after} the {\nvector} object has
  been initialized.

  Allowable values for \id{pretype} and \id{maxl} are the same as for
  the {\CC} function \newline \ID{SUNLinSol\_SPBCGS}.
}
Additionally, when using {\arkode} with a non-identity
mass matrix, the {\sunlinsolspbcgs} module includes a Fortran-callable
function for creating a \id{SUNLinearSolver} mass matrix solver
object.
%
% --------------------------------------------------------------------
%
\ucfunction{FSUNMASSSPBCGSINIT}
{
  FSUNMASSSPBCGSINIT(pretype, maxl, ier)
}
{
  The function \ID{FSUNMASSSPBCGSINIT} can be called for Fortran programs
  to create a {\sunlinsolspbcgs} object for mass matrix linear systems.
}
{
  \begin{args}[pretype]
  \item[pretype] (\id{int*})
    flag indicating desired preconditioning type
  \item[maxl] (\id{int*})
    flag indicating number of iterations to allow
  \end{args}
}
{
  \id{ier} is a \id{int} return completion flag equal to \id{0} for a success
  return and \id{-1} otherwise. See printed message for details in case
  of failure.
}
{
  This routine must be called \emph{after} the {\nvector} object has
  been initialized.

  Allowable values for \id{pretype} and \id{maxl} are the same as for
  the {\CC} function \newline \ID{SUNLinSol\_SPBCGS}.
}
%
% --------------------------------------------------------------------
%
The \id{SUNLinSol\_SPBCGSSetPrecType} and
\id{SUNLinSol\_SPBCGSSetMaxl} routines also support Fortran interfaces
for the system and mass matrix solvers.
%
% --------------------------------------------------------------------
%
\ucfunction{FSUNSPBCGSSETPRECTYPE}
{
  FSUNSPBCGSSETPRECTYPE(code, pretype, ier)
}
{
  The function \ID{FSUNSPBCGSSETPRECTYPE} can be called for Fortran
  programs to change the type of preconditioning to use.
}
{
  \begin{args}[pretype]
  \item[code] (\id{int*})
    is an integer input specifying the solver id (1 for {\cvode}, 2
    for {\ida}, 3 for {\kinsol}, and 4 for {\arkode}).
  \item[pretype] (\id{int*})
    flag indicating the type of preconditioning to use.
  \end{args}
}
{
  \id{ier} is a \id{int} return completion flag equal to \id{0} for a success
  return and \id{-1} otherwise. See printed message for details in case
  of failure.
}
{
  See \id{SUNLinSol\_SPBCGSSetPrecType} for complete further documentation of
  this routine.
}
%
% --------------------------------------------------------------------
%
\ucfunction{FSUNMASSSPBCGSSETPRECTYPE}
{
  FSUNMASSSPBCGSSETPRECTYPE(pretype, ier)
}
{
  The function \ID{FSUNMASSSPBCGSSETPRECTYPE} can be called for Fortran
  programs to change the type of preconditioning for mass matrix
  linear systems.
}
{
  The arguments are identical to \id{FSUNSPBCGSSETPRECTYPE} above, except that
  \id{code} is not needed since mass matrix linear systems only arise
  in {\arkode}.
}
{
  \id{ier} is a \id{int} return completion flag equal to \id{0} for a success
  return and \id{-1} otherwise. See printed message for details in case
  of failure.
}
{
  See \id{SUNLinSol\_SPBCGSSetPrecType} for complete further documentation of
  this routine.
}
%
% --------------------------------------------------------------------
%
\ucfunction{FSUNSPBCGSSETMAXL}
{
  FSUNSPBCGSSETMAXL(code, maxl, ier)
}
{
  The function \ID{FSUNSPBCGSSETMAXL} can be called for Fortran
  programs to change the maximum number of iterations to allow.
}
{
  \begin{args}[maxl]
  \item[code] (\id{int*})
    is an integer input specifying the solver id (1 for {\cvode}, 2
    for {\ida}, 3 for {\kinsol}, and 4 for {\arkode}).
  \item[maxl] (\id{int*})
    the number of iterations to allow.
  \end{args}
}
{
  \id{ier} is a \id{int} return completion flag equal to \id{0} for a success
  return and \id{-1} otherwise. See printed message for details in case
  of failure.
}
{
  See \id{SUNLinSol\_SPBCGSSetMaxl} for complete further
  documentation of this routine.
}
%
% --------------------------------------------------------------------
%
\ucfunction{FSUNMASSSPBCGSSETMAXL}
{
  FSUNMASSSPBCGSSETMAXL(maxl, ier)
}
{
  The function \ID{FSUNMASSSPBCGSSETMAXL} can be called for Fortran
  programs to change the type of preconditioning for mass matrix
  linear systems.
}
{
  The arguments are identical to \id{FSUNSPBCGSSETMAXL} above, except that
  \id{code} is not needed since mass matrix linear systems only arise
  in {\arkode}.
}
{
  \id{ier} is a \id{int} return completion flag equal to \id{0} for a success
  return and \id{-1} otherwise. See printed message for details in case
  of failure.
}
{
  See \id{SUNLinSol\_SPBCGSSetMaxl} for complete further
  documentation of this routine.
}
%
% --------------------------------------------------------------------
%

% ====================================================================
\subsection{SUNLinearSolver\_SPBCGS content}
\label{ss:sunlinsol_spbcgs_content}
% ====================================================================

The {\sunlinsolspbcgs} module defines the \textit{content} field of a
\id{SUNLinearSolver} as the following structure:
%%
\begin{verbatim}
struct _SUNLinearSolverContent_SPBCGS {
  int maxl;
  int pretype;
  int numiters;
  realtype resnorm;
  int last_flag;
  ATimesFn ATimes;
  void* ATData;
  PSetupFn Psetup;
  PSolveFn Psolve;
  void* PData;
  N_Vector s1;
  N_Vector s2;
  N_Vector r;
  N_Vector r_star;
  N_Vector p;
  N_Vector q;
  N_Vector u;
  N_Vector Ap;
  N_Vector vtemp;
  int      print_level;
  FILE*    info_file;
};
\end{verbatim}
%%
These entries of the \emph{content} field contain the following
information:
\begin{args}[print\_level]
  \item[maxl] - number of {\spbcgs} iterations to allow (default is 5),
  \item[pretype] - flag for type of preconditioning to employ
    (default is none),
  \item[numiters] - number of iterations from the most-recent solve,
  \item[resnorm] - final linear residual norm from the most-recent solve,
  \item[last\_flag] - last error return flag from an internal function,
  \item[ATimes] - function pointer to perform $Av$ product,
  \item[ATData] - pointer to structure for \id{ATimes},
  \item[Psetup] - function pointer to preconditioner setup routine,
  \item[Psolve] - function pointer to preconditioner solve routine,
  \item[PData] - pointer to structure for \id{Psetup} and \id{Psolve},
  \item[s1, s2] - vector pointers for supplied scaling matrices
    (default is \id{NULL}),
  \item[r] - a {\nvector} which holds the current scaled,
    preconditioned linear system residual,
  \item[r\_star] - a {\nvector} which holds the initial scaled,
    preconditioned linear system residual,
  \item[p, q, u, Ap, vtemp] - {\nvector}s used for workspace by the
    {\spbcgs} algorithm.
  \item[print\_level] - controls the amount of information to be printed to the info file
  \item[info\_file]   - the file where all informative (non-error) messages will be directed
\end{args}


%---------------------------------------------------------------------------
\section{The SUNLinearSolver\_SPTFQMR implementation}\label{ss:sunlinsol_sptfqmr}
% ====================================================================
\section{The SUNLinearSolver\_SPTFQMR implementation}
\label{ss:sunlinsol_sptfqmr}
% ====================================================================

This section describes the {\sunlinsol} implementation of the {\sptfqmr}
(Scaled, Preconditioned, \newline Transpose-Free Quasi-Minimum Residual \cite{Fre:93})
iterative linear solver. The {\sunlinsolsptfqmr} module is designed to be
compatible with any {\nvector} implementation that supports a minimal subset
of operations (\id{N\_VClone}, \id{N\_VDotProd}, \id{N\_VScale},
\id{N\_VLinearSum}, \id{N\_VProd}, \id{N\_VConst}, \id{N\_VDiv}, and
\id{N\_VDestroy}). Unlike the {\spgmr} and {\spfgmr} algorithms, {\sptfqmr}
requires a fixed amount of memory that does not increase with the number of
allowed iterations.

To access the {\sunlinsolsptfqmr} module, include the header file \newline
\id{sunlinsol/sunlinsol\_sptfqmr.h}. We note that the {\sunlinsolsptfqmr} module is
accessible from {\sundials} packages \textit{without} separately linking to
the \id{libsundials\_sunlinsolsptfqmr} module library.


% ====================================================================
\subsection{SUNLinearSolver\_SPTFQMR description}
\label{ss:sunlinsol_sptfqmr_description}
% ====================================================================

This solver is constructed to perform the following operations:
\begin{itemize}
\item During construction all {\nvector} solver data is allocated,
  with vectors cloned from a template {\nvector} that is input, and
  default solver parameters are set.
\item User-facing ``set'' routines may be called to modify default
  solver parameters.
\item Additional ``set'' routines are called by the {\sundials} solver
  that interfaces with \newline {\sunlinsolsptfqmr} to supply the
  \id{ATimes}, \id{PSetup}, and \id{Psolve} function pointers and
  \id{s1} and \id{s2} scaling vectors.
\item In the ``initialize'' call, the solver parameters are checked
  for validity.
\item In the ``setup'' call, any non-\id{NULL}
  \id{PSetup} function is called.  Typically, this is provided by
  the {\sundials} solver itself, that translates between the
  generic \id{PSetup} function and the
  solver-specific routine (solver-supplied or user-supplied).
\item In the ``solve'' call the TFQMR iteration is performed.  This
  will include scaling and preconditioning if those options have been
  supplied.
\end{itemize}


% ====================================================================
\subsection{SUNLinearSolver\_SPTFQMR functions}
\label{ss:sunlinsol_sptfqmr_functions}
% ====================================================================

The {\sunlinsolsptfqmr} module provides the following user-callable constructor
for creating a \newline \id{SUNLinearSolver} object.
%
% --------------------------------------------------------------------
%
\ucfunctiondf{SUNLinSol\_SPTFQMR}
{
  LS = SUNLinSol\_SPTFQMR(y, pretype, maxl);
}
{
  The function \ID{SUNLinSol\_SPTFQMR} creates and allocates memory for
  a {\sptfqmr} \newline \id{SUNLinearSolver} object.
}
{
  \begin{args}[pretype]
  \item[y] (\id{N\_Vector})
    a template for cloning vectors needed within the solver
  \item[pretype] (\id{int})
    flag indicating the desired type of preconditioning, allowed
    values are:
    \begin{itemize}
    \item \id{PREC\_NONE} (0)
    \item \id{PREC\_LEFT} (1)
    \item \id{PREC\_RIGHT} (2)
    \item \id{PREC\_BOTH} (3)
    \end{itemize}
    Any other integer input will result in the default (no
    preconditioning).
  \item[maxl] (\id{int})
    the number of linear iterations to allow. Values $\le0$ will
    result in the default value (5).
  \end{args}
}
{
  This returns a \id{SUNLinearSolver} object.  If either \id{y} is
  incompatible then this routine will return \id{NULL}.
}
{
  This routine will perform consistency checks to ensure that it is
  called with a consistent {\nvector} implementation (i.e.~that it
  supplies the requisite vector operations).  If \id{y} is
  incompatible, then this routine will return \id{NULL}.

  We note that some {\sundials} solvers are designed to only work
  with left preconditioning ({\ida} and {\idas}) and others with only
  right preconditioning ({\kinsol}). While it is possible to configure
  a {\sunlinsolsptfqmr} object to use any of the preconditioning options
  with these solvers, this use mode is not supported and may result in
  inferior performance.
}
{SUNSPTFQMR}
%
% --------------------------------------------------------------------
%
The {\sunlinsolsptfqmr} module defines implementations of all
``iterative'' linear solver operations listed in Sections
\ref{ss:sunlinsol_CoreFn} -- \ref{ss:sunlinsol_GetFn}:
\begin{itemize}
\item \id{SUNLinSolGetType\_SPTFQMR}
\item \id{SUNLinSolInitialize\_SPTFQMR}
\item \id{SUNLinSolSetATimes\_SPTFQMR}
\item \id{SUNLinSolSetPreconditioner\_SPTFQMR}
\item \id{SUNLinSolSetScalingVectors\_SPTFQMR}
\item \id{SUNLinSolSetup\_SPTFQMR}
\item \id{SUNLinSolSolve\_SPTFQMR}
\item \id{SUNLinSolNumIters\_SPTFQMR}
\item \id{SUNLinSolResNorm\_SPTFQMR}
\item \id{SUNLinSolResid\_SPTFQMR}
\item \id{SUNLinSolLastFlag\_SPTFQMR}
\item \id{SUNLinSolSpace\_SPTFQMR}
\item \id{SUNLinSolFree\_SPTFQMR}
\end{itemize}
All of the listed operations are callable via the {\F} 2003 interface module
by prepending an `F' to the function name.

The {\sunlinsolsptfqmr} module also defines the following additional
user-callable functions.
%
% --------------------------------------------------------------------
%
\ucfunctiondf{SUNLinSol\_SPTFQMRSetPrecType}
{
  retval = SUNLinSol\_SPTFQMRSetPrecType(LS, pretype);
}
{
  The function \ID{SUNLinSol\_SPTFQMRSetPrecType} updates the type of
  preconditioning to use in the {\sunlinsolsptfqmr} object.
}
{
  \begin{args}[pretype]
  \item[LS] (\id{SUNLinearSolver})
    the {\sunlinsolsptfqmr} object to update
  \item[pretype] (\id{int})
    flag indicating the desired type of preconditioning, allowed
    values match those discussed in \id{SUNLinSol\_SPTFQMR}.
  \end{args}
}
{
  This routine will return with one of the error codes
  \id{SUNLS\_ILL\_INPUT} (illegal \id{pretype}), \id{SUNLS\_MEM\_NULL}
  (\id{S} is \id{NULL}) or \id{SUNLS\_SUCCESS}.
}
{}
{SUNSPTFQMRSetPrecType}
%
% --------------------------------------------------------------------
%
\ucfunctionf{SUNLinSol\_SPTFQMRSetMaxl}
{
  retval = SUNLinSol\_SPTFQMRSetMaxl(LS, maxl);
}
{
  The function \ID{SUNLinSol\_SPTFQMRSetMaxl} updates the number of
  linear solver iterations to allow.
}
{
  \begin{args}[maxl]
  \item[LS] (\id{SUNLinearSolver})
    the {\sunlinsolsptfqmr} object to update
  \item[maxl] (\id{int})
    flag indicating the number of iterations to allow; values $\le0$
    will result in the default value (5)
  \end{args}
}
{
  This routine will return with one of the error codes
  \id{SUNLS\_MEM\_NULL} (\id{S} is \id{NULL}) or \id{SUNLS\_SUCCESS}.
}
{}
{SUNSPTFQMRSetMaxl}


% ====================================================================
\subsection{SUNLinearSolver\_SPTFQMR Fortran interfaces}
\label{ss:sunlinsol_sptfqmr_fortran}
% ====================================================================

The {\sunlinsolspfgmr} module provides a {\F} 2003 module as well as {\F} 77
style interface functions for use from {\F} applications.

\subsubsection*{FORTRAN 2003 interface module}
The \ID{fsunlinsol\_sptfqmr\_mod} {\F} module defines interfaces to all
{\sunlinsolspfgmr} {\CC} functions using the intrinsic \id{iso\_c\_binding}
module which provides a standardized mechanism for interoperating with {\CC}. As
noted in the {\CC} function descriptions above, the interface functions are
named after the corresponding {\CC} function, but with a leading `F'. For
example, the function \id{SUNLinSol\_SPTFQMR} is interfaced as
\id{FSUNLinSol\_SPTFQMR}.

The {\F} 2003 {\sunlinsolspfgmr} interface module can be accessed with the \id{use}
statement, i.e. \id{use fsunlinsol\_sptfqmr\_mod}, and linking to the library
\id{libsundials\_fsunlinsolsptfqmr\_mod}.{\em lib} in addition to the {\CC} library.
For details on where the library and module file \newline
\id{fsunlinsol\_sptfqmr\_mod.mod} are installed see Appendix \ref{c:install}.
We note that the module is accessible from the {\F} 2003 {\sundials} integrators
\textit{without} separately linking to the \newline
\id{libsundials\_fsunlinsolsptfqmr\_mod} library.

\subsubsection*{FORTRAN 77 interface functions}
For solvers that include a {\F} 77 interface module, the
{\sunlinsolsptfqmr} module also includes a Fortran-callable function
for creating a \id{SUNLinearSolver} object.
%
% --------------------------------------------------------------------
%
\ucfunction{FSUNSPTFQMRINIT}
{
  FSUNSPTFQMRINIT(code, pretype, maxl, ier)
}
{
  The function \ID{FSUNSPTFQMRINIT} can be called for Fortran programs
  to create a {\sunlinsolsptfqmr} object.
}
{
  \begin{args}[pretype]
  \item[code] (\id{int*})
    is an integer input specifying the solver id (1 for {\cvode}, 2
    for {\ida}, 3 for {\kinsol}, and 4 for {\arkode}).
  \item[pretype] (\id{int*})
    flag indicating desired preconditioning type
  \item[maxl] (\id{int*})
    flag indicating number of iterations to allow
  \end{args}
}
{
  \id{ier} is a return completion flag equal to \id{0} for a success
  return and \id{-1} otherwise. See printed message for details in case
  of failure.
}
{
  This routine must be called \emph{after} the {\nvector} object has
  been initialized.

  Allowable values for \id{pretype} and \id{maxl} are the same as for
  the {\CC} function \newline \ID{SUNLinSol\_SPTFQMR}.
}
Additionally, when using {\arkode} with a non-identity
mass matrix, the {\sunlinsolsptfqmr} module includes a Fortran-callable
function for creating a \id{SUNLinearSolver} mass matrix solver
object.
%
% --------------------------------------------------------------------
%
\ucfunction{FSUNMASSSPTFQMRINIT}
{
  FSUNMASSSPTFQMRINIT(pretype, maxl, ier)
}
{
  The function \ID{FSUNMASSSPTFQMRINIT} can be called for Fortran programs
  to create a {\sunlinsolsptfqmr} object for mass matrix linear systems.
}
{
  \begin{args}[pretype]
  \item[pretype] (\id{int*})
    flag indicating desired preconditioning type
  \item[maxl] (\id{int*})
    flag indicating number of iterations to allow
  \end{args}
}
{
  \id{ier} is a \id{int} return completion flag equal to \id{0} for a success
  return and \id{-1} otherwise. See printed message for details in case
  of failure.
}
{
  This routine must be called \emph{after} the {\nvector} object has
  been initialized.

  Allowable values for \id{pretype} and \id{maxl} are the same as for
  the {\CC} function \newline \ID{SUNLinSol\_SPTFQMR}.
}
%
% --------------------------------------------------------------------
%
The \id{SUNLinSol\_SPTFQMRSetPrecType} and
\id{SUNLinSol\_SPTFQMRSetMaxl} routines also support Fortran
interfaces for the system and mass matrix solvers.
%
% --------------------------------------------------------------------
%
\ucfunction{FSUNSPTFQMRSETPRECTYPE}
{
  FSUNSPTFQMRSETPRECTYPE(code, pretype, ier)
}
{
  The function \ID{FSUNSPTFQMRSETPRECTYPE} can be called for Fortran
  programs to change the type of preconditioning to use.
}
{
  \begin{args}[pretype]
  \item[code] (\id{int*})
    is an integer input specifying the solver id (1 for {\cvode}, 2
    for {\ida}, 3 for {\kinsol}, and 4 for {\arkode}).
  \item[pretype] (\id{int*})
    flag indicating the type of preconditioning to use.
  \end{args}
}
{
  \id{ier} is a \id{int} return completion flag equal to \id{0} for a success
  return and \id{-1} otherwise. See printed message for details in case
  of failure.
}
{
  See \id{SUNLinSol\_SPTFQMRSetPrecType} for complete further documentation of
  this routine.
}
%
% --------------------------------------------------------------------
%
\ucfunction{FSUNMASSSPTFQMRSETPRECTYPE}
{
  FSUNMASSSPTFQMRSETPRECTYPE(pretype, ier)
}
{
  The function \ID{FSUNMASSSPTFQMRSETPRECTYPE} can be called for Fortran
  programs to change the type of preconditioning for mass matrix
  linear systems.
}
{
  The arguments are identical to \id{FSUNSPTFQMRSETPRECTYPE} above,
  except that \id{code} is not needed since mass matrix linear systems
  only arise in {\arkode}.
}
{
  \id{ier} is a \id{int} return completion flag equal to \id{0} for a success
  return and \id{-1} otherwise. See printed message for details in case
  of failure.
}
{
  See \id{SUNLinSol\_SPTFQMRSetPrecType} for complete further documentation of
  this routine.
}
%
% --------------------------------------------------------------------
%
\ucfunction{FSUNSPTFQMRSETMAXL}
{
  FSUNSPTFQMRSETMAXL(code, maxl, ier)
}
{
  The function \ID{FSUNSPTFQMRSETMAXL} can be called for Fortran
  programs to change the maximum number of iterations to allow.
}
{
  \begin{args}[maxl]
  \item[code] (\id{int*})
    is an integer input specifying the solver id (1 for {\cvode}, 2
    for {\ida}, 3 for {\kinsol}, and 4 for {\arkode}).
  \item[maxl] (\id{int*})
    the number of iterations to allow.
  \end{args}
}
{
  \id{ier} is a \id{int} return completion flag equal to \id{0} for a success
  return and \id{-1} otherwise. See printed message for details in case
  of failure.
}
{
  See \id{SUNLinSol\_SPTFQMRSetMaxl} for complete further
  documentation of this routine.
}
%
% --------------------------------------------------------------------
%
\ucfunction{FSUNMASSSPTFQMRSETMAXL}
{
  FSUNMASSSPTFQMRSETMAXL(maxl, ier)
}
{
  The function \ID{FSUNMASSSPTFQMRSETMAXL} can be called for Fortran
  programs to change the type of preconditioning for mass matrix
  linear systems.
}
{
  The arguments are identical to \id{FSUNSPTFQMRSETMAXL} above, except that
  \id{code} is not needed since mass matrix linear systems only arise
  in {\arkode}.
}
{
  \id{ier} is a \id{int} return completion flag equal to \id{0} for a success
  return and \id{-1} otherwise. See printed message for details in case
  of failure.
}
{
  See \id{SUNLinSol\_SPTFQMRSetMaxl} for complete further
  documentation of this routine.
}
%
% --------------------------------------------------------------------
%

% ====================================================================
\subsection{SUNLinearSolver\_SPTFQMR content}
\label{ss:sunlinsol_sptfqmr_content}
% ====================================================================

The {\sunlinsolsptfqmr} module defines the \textit{content} field of a
\id{SUNLinearSolver} as the following structure:
%%
\begin{verbatim}
struct _SUNLinearSolverContent_SPTFQMR {
  int maxl;
  int pretype;
  int numiters;
  realtype resnorm;
  long int last_flag;
  ATimesFn ATimes;
  void* ATData;
  PSetupFn Psetup;
  PSolveFn Psolve;
  void* PData;
  N_Vector s1;
  N_Vector s2;
  N_Vector r_star;
  N_Vector q;
  N_Vector d;
  N_Vector v;
  N_Vector p;
  N_Vector *r;
  N_Vector u;
  N_Vector vtemp1;
  N_Vector vtemp2;
  N_Vector vtemp3;
};
\end{verbatim}
%%
These entries of the \emph{content} field contain the following
information:
\begin{args}[last\_flag]
  \item[maxl] - number of TFQMR iterations to allow (default is 5),
  \item[pretype] - flag for type of preconditioning to employ
    (default is none),
  \item[numiters] - number of iterations from the most-recent solve,
  \item[resnorm] - final linear residual norm from the most-recent solve,
  \item[last\_flag] - last error return flag from an internal function,
  \item[ATimes] - function pointer to perform $Av$ product,
  \item[ATData] - pointer to structure for \id{ATimes},
  \item[Psetup] - function pointer to preconditioner setup routine,
  \item[Psolve] - function pointer to preconditioner solve routine,
  \item[PData] - pointer to structure for \id{Psetup} and \id{Psolve},
  \item[s1, s2] - vector pointers for supplied scaling matrices
    (default is \id{NULL}),
  \item[r\_star] - a {\nvector} which holds the initial scaled,
    preconditioned linear system residual,
  \item[q, d, v, p, u] - {\nvector}s used for workspace by the SPTFQMR
    algorithm,
  \item [r] - array of two {\nvector}s used for workspace within the
    SPTFQMR algorithm,
  \item[vtemp1, vtemp2, vtemp3] - temporary vector storage.
\end{args}



%---------------------------------------------------------------------------
\section{The SUNLinearSolver\_PCG implementation}\label{ss:sunlinsol_pcg}
%% This is a shared SUNDIALS TEX file with a description of the
%% pcg sunlinsol implementation
%%

The {\pcg} (Preconditioned Conjugate Gradient \cite{HeSt:52})
implementation of the {\sunlinsol} module provided with {\sundials},
{\sunlinsolpcg}, is an iterative linear solver that is designed to be
compatible with any {\nvector} implementation (serial, threaded,
parallel, user-supplied) that supports a minimal subset of operations
(\id{N\_VClone}, \id{N\_VDotProd}, \id{N\_VScale}, \id{N\_VLinearSum},
\id{N\_VProd} and \id{N\_VDestroy}).  Unlike the {\spgmr} and {\spfgmr}
algorithms, {\pcg} requires a fixed amount of memory that does not
scale with the number of allowed iterations.

Unlike all of the other iterative linear solvers supplied with
{\sundials}, {\pcg} should only be used on \emph{symmetric} linear
systems (e.g.~mass matrix linear systems encountered in
{\arkode}). As a result, the explanation of the role of scaling and
preconditioning matrices given in general must be modified in this
scenario.  The {\pcg} algorithm solves a linear system $Ax = b$ where  
$A$ is a symmetric ($A^T=A$), real-valued matrix.  Preconditioning is
allowed, and is applied in a symmetric fashion on both the right and
left.  Scaling is also allowed and is applied symmetrically.  We
denote the preconditioner and scaling matrices as follows:
\begin{itemize}
\item $P$ is the preconditioner (assumed symmetric),
\item $S$ is a diagonal matrix of scale factors.
\end{itemize}
The matrices $A$ and $P$ are not required explicitly; only routines
that provide $A$ and $P^{-1}$ as operators are required.  The diagonal
of the matrix $S$ is held in a single {\nvector}, supplied by the user
of this module.

In this notation, {\pcg} applies the underlying CG algorithm to the
equivalent transformed system 
\begin{equation}
  \label{eq:transformed_linear_systemPCG}
  \tilde{A} \tilde{x} = \tilde{b}
\end{equation}
where
\begin{align}
  \notag
  \tilde{A} &= S P^{-1} A P^{-1} S,\\
  \label{eq:transformed_linear_system_componentsPCG}
  \tilde{b} &= S P^{-1} b,\\
  \notag
  \tilde{x} &= S^{-1} P x.
\end{align} 
The scaling matrix must be chosen so that the vectors $SP^{-1}b$ and
$S^{-1}Px$ have dimensionless components.

The stopping test for the PCG iterations is on the L2 norm of the
scaled preconditioned residual:
\begin{align*}
  &\| \tilde{b} - \tilde{A} \tilde{x} \|_2  <  \delta\\
  \Leftrightarrow\quad &\\
  &\| S P^{-1} b - S P^{-1} A x \|_2  <  \delta\\
  \Leftrightarrow\quad &\\
  &\| P^{-1} b - P^{-1} A x \|_S  <  \delta
\end{align*}
where $\| v \|_S = \sqrt{v^T S^T S v}$, with an input tolerance $\delta$.

The {\sunlinsolpcg} module defines the {\em content} field of a
\id{SUNLinearSolver} to be the following structure:
%%
\begin{verbatim} 
struct _SUNLinearSolverContent_PCG {
  int maxl;
  int pretype;
  int numiters;
  realtype resnorm;
  long int last_flag;
  ATSetupFn ATSetup;
  ATimesFn ATimes;
  void* ATData;
  PSetupFn Psetup;
  PSolveFn Psolve;
  void* PData;
  N_Vector s;
  N_Vector r;
  N_Vector p;
  N_Vector z;
  N_Vector Ap;
};
\end{verbatim}
%%
These entries of the \emph{content} field contain the following
information:
\begin{description}
  \item[maxl] - number of {\pcg} iterations to allow (default is 5)
  \item[pretype] - flag for use of preconditioning (default is none)
  \item[numiters] - number of iterations from most-recent solve
  \item[resnorm] - final linear residual norm from most-recent solve
  \item[last\_flag] - last error return flag from internal function
  \item[ATSetup] - function pointer to setup routine for \id{ATimes} data
  \item[ATimes] - function pointer to perform $Av$ product
  \item[ATData] - pointer to structure for \id{ATSetup}, \id{ATimes}
  \item[Psetup] - function pointer to preconditioner setup routine
  \item[Psolve] - function pointer to preconditioner solve routine
  \item[PData] - pointer to structure for \id{Psetup}, \id{Psolve}
  \item[s] - vector pointer for supplied scaling matrix
    (default is \id{NULL})
  \item[r] - a {\nvector} which holds the preconditioned linear system
    residual
  \item[p, z, Ap] - {\nvector}s used for workspace by the
    {\pcg} algorithm. 
\end{description}

This solver is constructed to perform the following operations:
\begin{itemize}
\item During construction all {\nvector} solver data is allocated,
  with vectors cloned from a template {\nvector} that is input, and
  default solver parameters are set.
\item User-facing ``set'' routines may be called to modify default
  solver parameters.
\item Additional ``set'' routines are called by the {\sundials} solver
  that interfaces with {\sunlinsolpcg} to supply the \id{ATSetup},
  \id{ATimes}, \id{PSetup} and \id{Psolve} function pointers and
  \id{s} scaling vector.
\item In the ``initialize'' call, the solver parameters are checked
  for validity.
\item In the ``setup'' call, any non-\id{NULL} \id{ATSetup} and
  \id{PSetup} functions are called.  Typically, these are provided by
  the {\sundials} solvers themselves, that translate between the
  generic \id{ATSetup} and \id{PSetup} functions and the
  solver-specific routines (solver-supplied or user-supplied).
\item In the ``solve'' call the {\pcg} iteration is performed.  This
  will include scaling and preconditioning if those options have been
  supplied.
\end{itemize}

\noindent The header file to be included when using this module 
is \id{sunlinsol/sunlinsol\_pcg.h}. \\
%%
%%----------------------------------------------
%%
The {\sunlinsolpcg} module defines implementations of all
``iterative'' linear solver operations listed in Table
\ref{t:sunlinsolops}:
\begin{itemize}
\item \id{SUNLinSolGetType\_PCG}
\item \id{SUNLinSolInitialize\_PCG}
\item \id{SUNLinSolSetATimes\_PCG}
\item \id{SUNLinSolSetPreconditioner\_PCG}
\item \id{SUNLinSolSetScalingVectors\_PCG} -- since {\pcg} only
  supports symmetric scaling, the second {\nvector} argument to this
  function is ignored
\item \id{SUNLinSolSetup\_PCG}
\item \id{SUNLinSolSolve\_PCG}
\item \id{SUNLinSolNumIters\_PCG}
\item \id{SUNLinSolResNorm\_PCG}
\item \id{SUNLinSolResid\_PCG}
\item \id{SUNLinSolLastFlag\_PCG}
\item \id{SUNLinSolSpace\_PCG}
\item \id{SUNLinSolFree\_PCG}
\end{itemize}
The module {\sunlinsolpcg} provides the following additional
user-callable routines: 
%%
\begin{itemize}

%%--------------------------------------

\item \ID{SUNPCG}

  This function creates and allocates memory for a {\pcg}
  \id{SUNLinearSolver}.  Its arguments are an {\nvector}, a flag
  indicating to use preconditioning, and the number of linear
  iterations to allow. 

  This routine will perform consistency checks to ensure that it is
  called with a consistent {\nvector} implementation (i.e.~that it
  supplies the requisite vector operations).  If \id{y} is
  incompatible then this routine will return \id{NULL}.

  A \id{maxl} argument that is $\le0$ will result in the default
  value (5).

  Since the {\pcg} algorithm is designed to only support symmetric
  preconditioning, then any of the \id{pretype} inputs \id{PREC\_LEFT}
  (1), \id{PREC\_RIGHT} (2), or \id{PREC\_BOTH} (3) will result in use
  of the symmetric preconditioner;  any other integer input will
  result in the default (no preconditioning).

  \verb|SUNLinearSolver SUNPCG(N_Vector y, int pretype, int maxl);|

%%--------------------------------------

\item \ID{SUNPCGSetPrecType}

  This function updates the flag indicating use of preconditioning.
  As above, any one of the input values, \id{PREC\_LEFT} (1),
  \id{PREC\_RIGHT} (2) and \id{PREC\_BOTH} (3) will enable
  preconditioning; \id{PREC\_NONE} (0) disables preconditioning.

  This routine will return with one of the error codes
  \id{SUNLS\_ILL\_INPUT} (illegal \id{pretype}), \id{SUNLS\_MEM\_NULL}
  (\id{S} is \id{NULL}) or \id{SUNLS\_SUCCESS}.
  
  \verb|int SUNPCGSetPrecType(SUNLinearSolver S, int pretype);|

%%--------------------------------------

\item \ID{SUNPCGSetMaxl}

  This function updates the number of linear solver iterations to
  allow. 

  A \id{maxl} argument that is $\le0$ will result in the default
  value (5).

  This routine will return with one of the error codes
  \id{SUNLS\_MEM\_NULL} (\id{S} is \id{NULL}) or \id{SUNLS\_SUCCESS}.
  
  \verb|int SUNPCGSetMaxl(SUNLinearSolver S, int maxl);|

\end{itemize}
%%
%%------------------------------------
%%
For solvers that include a Fortran interface module, the
{\sunlinsolpcg} module also includes the Fortran-callable
function \id{FSUNPCGInit(code, pretype, maxl, ier)} to initialize
this {\sunlinsolpcg} module for a given {\sundials} solver.
Here \id{code} is an input solver id (1 for {\cvode}, 2 for {\ida}, 3
for {\kinsol}, 4 for {\arkode}); \id{pretype} and \id{maxl} are the
same as for the C function \ID{SUNPCG}; \id{ier} is an error return
flag equal 0 for success and -1 for failure.  All of these input
arguments should be declared so as to match C type \id{int}).  This
routine must be called \emph{after} the {\nvector} object has been
initialized.  Additionally, when using {\arkode} with non-identity
mass matrix, the Fortran-callable function 
\id{FSUNMassPCGInit(pretype, maxl, ier)} initializes this
{\sunlinsolpcg} module for solving mass matrix linear systems.

The \id{SUNPCGSetPrecType} and \id{SUNPCGSetMaxl} routines also
support Fortran interfaces for the system and mass matrix solvers:
\begin{itemize}
\item \id{FSUNPCGSetPrecType(code, pretype, ier)} -- all arguments
  should be commensurate with a C \id{int}
\item \id{FSUNMassPCGSetPrecType(pretype, ier)}
\item \id{FSUNPCGSetMaxl(code, maxl, ier)} -- all arguments
  should be commensurate with a C \id{int}
\item \id{FSUNMassPCGSetMaxl(maxl, ier)}
\end{itemize}


%---------------------------------------------------------------------------

\section{SUNLinearSolver Examples}\label{ss:sunlinsol_examples}

There are \id{SUNLinearSolver} examples that may be installed for each
implementation; these make use of the functions in \id{test\_sunlinsol.c}.
These example functions show simple usage of the \id{SUNLinearSolver} family
of functions.  The inputs to the examples depend on the linear solver type,
and are output to \texttt{stdout} if the example is run without the
appropriate number of command-line arguments.

\noindent The following is a list of the example functions in \id{test\_sunlinsol.c}:
\begin{itemize}
\item \id{Test\_SUNLinSolGetType}: Verifies the returned solver type against
  the value that should be returned.
\item \id{Test\_SUNLinSolInitialize}: Verifies that \id{SUNLinSolInitialize}
  can be called and returns successfully.
\item \id{Test\_SUNLinSolSetup}: Verifies that \id{SUNLinSolSetup} can
  be called and returns successfully.
\item \id{Test\_SUNLinSolSolve}: Given a {\sunmatrix} object $A$,
  {\nvector} objects $x$ and $b$ (where $Ax=b$) and a desired solution
  tolerance \texttt{tol}, this routine clones $x$ into a new vector $y$,
  calls \\ \noindent
  \id{SUNLinSolSolve} to fill $y$ as the solution to $Ay=b$ (to
  the input tolerance), verifies that each entry in $x$ and $y$
  match to within \texttt{10*tol}, and overwrites $x$ with $y$ prior
  to returning (in case the calling routine would like to investigate
  further).
\item \id{Test\_SUNLinSolSetATimes} (iterative solvers only): Verifies that
  \id{SUNLinSolSetATimes} can be called and returns successfully.
\item \id{Test\_SUNLinSolSetPreconditioner} (iterative solvers only):
  Verifies that \\ \noindent
  \id{SUNLinSolSetPreconditioner} can be called and
  returns successfully.
\item \id{Test\_SUNLinSolSetScalingVectors} (iterative solvers only):
  Verifies that \\ \noindent
  \id{SUNLinSolSetScalingVectors} can be called and
  returns successfully.
\item \id{Test\_SUNLinSolLastFlag}: Verifies that \id{SUNLinSolLastFlag} can
  be called, and outputs the result to \texttt{stdout}.
\item \id{Test\_SUNLinSolNumIters} (iterative solvers only): Verifies that
  \id{SUNLinSolNumIters} can be called, and outputs the result to
  \texttt{stdout}.
\item \id{Test\_SUNLinSolResNorm} (iterative solvers only): Verifies that
  \id{SUNLinSolResNorm} can be called, and that the result is
  non-negative.
\item \id{Test\_SUNLinSolResid} (iterative solvers only): Verifies that
  \id{SUNLinSolResid} can be called.
\item \id{Test\_SUNLinSolSpace} verifies that \id{SUNLinSolSpace} can be
  called, and outputs the results to \texttt{stdout}.
\end{itemize}
We'll note that these tests should be performed in a particular
order.  For either direct or iterative linear
solvers, \id{Test\_SUNLinSolInitialize} must be called
before \id{Test\_SUNLinSolSetup}, which must be called
before \id{Test\_SUNLinSolSolve}.  Additionally, for iterative linear
solvers \\ \noindent
\id{Test\_SUNLinSolSetATimes}, \id{Test\_SUNLinSolSetPreconditioner}
and \\ \noindent
\id{Test\_SUNLinSolSetScalingVectors} should be called
before \id{Test\_SUNLinSolInitialize};
similarly \id{Test\_SUNLinSolNumIters}, \id{Test\_SUNLinSolResNorm}
and \id{Test\_SUNLinSolResid} should be called
after \id{Test\_SUNLinSolSolve}.  These are called in the appropriate
order in all of the example problems.



%---------------------------------------------------------------------------
\section{IDA SUNLinearSolver interface}
\label{s:sunlinsol_interface}
%---------------------------------------------------------------------------

Table \ref{t:sunlinsoluse} below lists the {\sunlinsol} module linear solver
functions used within the {\idals} interface. As with the {\sunmatrix} module, we
emphasize that the {\ida} user does not need to know detailed usage of linear
solver functions by the {\ida} code modules in order to use {\ida}. The
information is presented as an implementation detail for the interested reader.

The linear solver functions listed below are marked with \cm to
indicate that they are required, or with $\dagger$ to indicate that
they are only called if they are non-\id{NULL} in the {\sunlinsol}
implementation that is being used. Note:
\begin{enumerate}
\item Although {\idals} does not call \id{SUNLinSolLastFlag}
  directly, this routine is available for users to query linear solver
  issues directly.
\item Although {\idals} does not call \id{SUNLinSolFree}
  directly, this routine should be available for users to call when
  cleaning up from a simulation.
\end{enumerate}

\begin{table}[htb]
\centering
\caption{List of linear solver function usage in the {\idals} interface}\label{t:sunlinsoluse}
\medskip
\begin{tabular}{|r|c|c|c|} \hline
                                                    & 
\begin{sideways}{DIRECT}             \end{sideways} & 
\begin{sideways}{ITERATIVE}          \end{sideways} & 
\begin{sideways}{MATRIX\_ITERATIVE}  \end{sideways} \\ \hline\hline
%                                  DIRECT       ITER    & MAT-ITER  
\id{SUNLinSolGetType}           &    \cm    &    \cm    &    \cm    \\ \hline
\id{SUNLinSolSetATimes}         & $\dagger$ &    \cm    & $\dagger$ \\ \hline
\id{SUNLinSolSetPreconditioner} & $\dagger$ & $\dagger$ & $\dagger$ \\ \hline
\id{SUNLinSolSetScalingVectors} & $\dagger$ & $\dagger$ & $\dagger$ \\ \hline
\id{SUNLinSolInitialize}        &    \cm    &    \cm    &    \cm    \\ \hline
\id{SUNLinSolSetup}             &    \cm    &    \cm    &    \cm    \\ \hline
\id{SUNLinSolSolve}             &    \cm    &    \cm    &    \cm    \\ \hline
\id{SUNLinSolNumIters}          &           &    \cm    &    \cm    \\ \hline
\id{SUNLinSolResid}             &           &    \cm    &    \cm    \\ \hline
$^1$\id{SUNLinSolLastFlag}      &           &           &           \\ \hline
$^2$\id{SUNLinSolFree}          &           &           &           \\ \hline
\id{SUNLinSolSpace}             & $\dagger$ & $\dagger$ & $\dagger$ \\ \hline
\end{tabular}
\end{table}

Since there are a wide range of potential {\sunlinsol} use cases, the following
subsections describe some details of the {\idals} interface, in the case that
interested users wish to develop custom {\sunlinsol} modules.

%---------------------------------------------------------------------------
\subsection{Lagged matrix information}
\label{ss:sunlinsol_lagged_matrix}
%---------------------------------------------------------------------------

If the {\sunlinsol} object self-identifies as having type
\id{SUNLINEARSOLVER\_DIRECT} or \\ \noindent
\id{SUNLINEARSOLVER\_MATRIX\_ITERATIVE}, then the {\sunlinsol} object solves a
linear system \emph{defined} by a {\sunmatrix} object. {\idals} will update the
matrix information infrequently according to the strategies outlined in
\S\ref{ss:ivp_sol}. To this end, we differentiate between the \emph{desired}
linear system $Mx=b$ and the \emph{actual} linear system $\bar{M}\bar{x} = b$.
Since {\idals} updates the {\sunmatrix} object infrequently, it is
likely that $\alpha\ne\bar{\alpha}$, and in turn $M\ne\bar{M}$.
Therefore, after calling the {\sunlinsol}-provided \id{SUNLinSolSolve}
routine, we test whether $\alpha / \bar{\alpha} \ne 1$, and if this is
the case we scale the solution $\bar{x}$ to correct the linear system
solution $x$ via
\begin{equation}
  \label{eq:rescaling}
  x = \frac{2}{1 + \alpha / \bar{\alpha}} \bar{x}.
\end{equation}
The motivation for this selection of the scaling factor $c = 2/(1 +
\alpha/\bar{\alpha})$ is discussed in detail in \cite{BBH:89,Hin:00}.
In short, if we consider a stationary iteration for the linear system
as consisting of a solve with $\bar{M}$ followed by scaling by $c$,
then for a linear constant-coefficient problem, the error in the
solution vector will be reduced at each iteration by the error matrix
$E = I - c \bar{M}^{-1} M$, with a convergence rate given by the
spectral radius of $E$.  Assuming that stiff systems have a spectrum
spread widely over the left half-plane, $c$ is chosen to minimize the
magnitude of the eigenvalues of $E$.


%---------------------------------------------------------------------------
\subsection{Iterative linear solver tolerance}
\label{ss:sunlinsol_iterative_tolerance}
%---------------------------------------------------------------------------

If the {\sunlinsol} object self-identifies as having type
\id{SUNLINEARSOLVER\_ITERATIVE} or \newline
\id{SUNLINEARSOLVER\_MATRIX\_ITERATIVE} then {\idals} will set the input
tolerance \id{delta} as described in \S\ref{ss:ivp_sol}. However, if the
iterative linear solver does not support scaling matrices (i.e., the \newline
\id{SUNLinSolSetScalingVectors} routine is \id{NULL}), then {\idals} will attempt 
to adjust the linear solver tolerance to account for this lack of functionality.
To this end, the following assumptions are made:
\begin{enumerate}
\item All solution components have similar magnitude; hence the error
  weight vector $W$ used in the WRMS norm (see \S\ref{ss:ivp_sol})
  should satisfy the assumption 
  \[
    W_i \approx W_{mean},\quad \text{for}\quad i=0,\ldots,n-1.
  \]
\item The {\sunlinsol} object uses a standard 2-norm to measure
  convergence.
\end{enumerate}

Since {\ida} uses identical left and right scaling matrices,
$S_1 = S_2 = S = \operatorname{diag}(W)$, then the linear
solver convergence requirement is converted as follows
(using the notation from equations
\eqref{eq:transformed_linear_system}-\eqref{eq:transformed_linear_system_components}):
\begin{align*}
  &\left\| \tilde{b} - \tilde{A} \tilde{x} \right\|_2  <  \text{tol}\\
  \Leftrightarrow \quad & \left\| S P_1^{-1} b - S P_1^{-1} A x \right\|_2  <  \text{tol}\\
  \Leftrightarrow \quad & \sum_{i=0}^{n-1} \left[W_i \left(P_1^{-1} (b - A x)\right)_i\right]^2  <  \text{tol}^2\\
  \Leftrightarrow \quad & W_{mean}^2 \sum_{i=0}^{n-1} \left[\left(P_1^{-1} (b - A x)\right)_i\right]^2  <  \text{tol}^2\\
  \Leftrightarrow \quad & \sum_{i=0}^{n-1} \left[\left(P_1^{-1} (b - A x)\right)_i\right]^2  <  \left(\frac{\text{tol}}{W_{mean}}\right)^2\\
  \Leftrightarrow \quad & \left\| P_1^{-1} (b - A x)\right\|_2  <  \frac{\text{tol}}{W_{mean}}
\end{align*}
Therefore the tolerance scaling factor
\[
  W_{mean} = \|W\|_2 / \sqrt{n}
\]
is computed and the scaled tolerance \id{delta}$= \text{tol} / W_{mean}$ is
supplied to the {\sunlinsol} object.

%---------------------------------------------------------------------------
% sunlinsol module sections
%---------------------------------------------------------------------------

%% This is a shared SUNDIALS TEX file with a description of the
%% dense sunlinsol implementation
%%

The dense implementation of the {\sunlinsol} module provided with
{\sundials}, {\sunlinsoldense}, is designed to be used with the
corresponding {\sunmatdense} matrix type, and one of the serial or
shared-memory {\nvector} implementations ({\nvecs}, {\nvecopenmp} or
{\nvecpthreads}).  The {\sunlinsoldense} module defines the {\em
content} field of a \id{SUNLinearSolver} to be the following structure:
%%
\begin{verbatim} 
struct _SUNLinearSolverContent_Dense {
  sunindextype N;
  sunindextype *pivots;
  long int last_flag;
};
\end{verbatim}
%%
These entries of the \emph{content} field contain the following
information:
\begin{description}
  \item[N] - size of the linear system,
  \item[pivots] - index array for partial pivoting in LU factorization,
  \item[last\_flag] - last error return flag from internal function evaluations.
\end{description}

This solver is constructed to perform the following operations:
\begin{itemize}
\item In the ``setup'' call, this performs a $LU$ factorization with
  partial (row) pivoting ($\mathcal O(N^3)$ cost), $PA=LU$, where $P$
  is a permutation matrix, $L$ is a lower triangular matrix with 1's
  on the diagonal, and $U$ is an upper triangular matrix.  This
  factorization is stored in-place on the input {\sunmatdense} object
  $A$, with pivoting information encoding $P$ stored in
  the \id{pivots} array.
\item In the ``solve'' call, this performs pivoting, forward and
  backward substitution using the stored \id{pivots} array and the
  $LU$ factors held in the {\sunmatdense} object ($\mathcal O(N^2)$
  cost).
\end{itemize}

\noindent The header file to be included when using this module 
is \id{sunlinsol/sunlinsol\_dense.h}. \\
%%
%%----------------------------------------------
%%
The {\sunlinsoldense} module defines dense implementations of all
``direct'' linear solver operations listed in
Table \ref{t:sunlinsolops}:
\begin{itemize}
\item \id{SUNLinSolGetType\_Dense}
\item \id{SUNLinSolInitialize\_Dense} -- this does nothing, since all
  consistency checks were performed at solver creation.
\item \id{SUNLinSolSetup\_Dense} -- this performs the $LU$ factorization.
\item \id{SUNLinSolSolve\_Dense} -- this uses the $LU$ factors
  and \id{pivots} array to perform the solve.
\item \id{SUNLinSolLastFlag\_Dense}
\item \id{SUNLinSolSpace\_Dense} -- this only returns information for
  the storage \emph{within} the solver object, i.e.~storage
  for \id{N}, \id{last\_flag} and \id{pivots}.
\item \id{SUNLinSolFree\_Dense}
\end{itemize}
The module {\sunlinsoldense} provides the following additional
user-callable routine: 
%%
\begin{itemize}

%%--------------------------------------

\item \ID{SUNDenseLinearSolver}

  This function creates and allocates memory for a dense \id{SUNLinearSolver}.
  Its arguments are an {\nvector} and {\sunmatrix}, that it uses to
  determine the linear system size and to assess compatibility with
  the linear solver implementation.

  This routine will perform consistency checks to ensure that it is
  called with consistent {\nvector} and {\sunmatrix} implementations.
  These are currently limited to the {\sunmatdense} matrix type, and
  the {\nvecs}, {\nvecopenmp} and {\nvecpthreads} vector types.  As
  additional compatible matrix and vector implementations are added to
  {\sundials}, these will be included within this compatibility check.

  If either \id{A} or \id{y} are incompatible then this routine will
  return \id{NULL}.

  \verb|SUNLinearSolver SUNDenseLinearSolver(N_Vector y, SUNMatrix A);|

\end{itemize}
%%
%%------------------------------------
%%
For solvers that include a Fortran interface module, the {\sunlinsoldense}
module also includes the Fortran-callable
function \id{FSUNDenseLinSolInit(code, ier)} to initialize
this {\sunlinsoldense} module for a given {\sundials} solver.
Here \id{code} is an input solver id (1 for {\cvode}, 2 for {\ida}, 3
for {\kinsol}, 4 for {\arkode}); \id{ier} is an error return flag 
equal 0 for success and -1 for failure (declared so as to match C type
\id{int}).  This routine must be called \emph{after} both the
{\nvector} and {\sunmatrix} objects have been initialized.
Additionally, when using {\arkode} with non-identity mass matrix, the
Fortran-callable function \id{FSUNMassDenseLinSolInit(ier)}  
initializes this {\sunlinsoldense} module for solving mass matrix
linear systems.

%% This is a shared SUNDIALS TEX file with a description of the
%% band sunlinsol implementation
%%

The band implementation of the {\sunlinsol} module provided with
{\sundials}, {\sunlinsolband}, is designed to be used with the
corresponding {\sunmatband} matrix type, and one of the serial or
shared-memory {\nvector} implementations ({\nvecs}, {\nvecopenmp} or
{\nvecpthreads}).


%---------------------------------------------------------------------------
\subsection{{\sunlinsolband} usage}\label{ss:sunlinsol_band_usage}

The header file to include when using this module is
\id{sunlinsol/sunlinsol\_band.h}. The {\sunlinsolband} module 
is accessible from all {\sundials} solvers \textit{without}
linking to the \\ \noindent
\id{libsundials\_sunlinsolband} module library.

The module {\sunlinsolband} provides the following user-callable constructor routine: 
%%
% --------------------------------------------------------------------
\ucfunction{SUNLinSol\_Band}
{
  LS = SUNLinSol\_Band(y, A);
}
{
  The function \ID{SUNLinSol\_Band} creates and allocates memory for
  a band \id{SUNLinearSolver} object.
}
{
  \begin{args}[y]
  \item[y] (\id{N\_Vector})
    a template for cloning vectors needed within the solver
  \item[A] (\id{SUNMatrix})
    a {\sunmatband} matrix template for cloning matrices needed
    within the solver 
  \end{args}
}
{
  This returns a \id{SUNLinearSolver} object.  If either \id{A} or
  \id{y} are incompatible then this routine will return \id{NULL}.
}
{
  This routine will perform consistency checks to ensure that it is
  called with consistent {\nvector} and {\sunmatrix} implementations.
  These are currently limited to the {\sunmatdense} matrix type and
  the {\nvecs}, {\nvecopenmp}, and {\nvecpthreads} vector types.  As
  additional compatible matrix and vector implementations are added to
  {\sundials}, these will be included within this compatibility check.

  Additionally, this routine will verify that the input matrix \id{A}
  is allocated with appropriate upper bandwidth storage for the $LU$
  factorization.
}
% --------------------------------------------------------------------
%%
For backwards compatibility, we also provide the wrapper functions:
\begin{itemize}

\item \ID{SUNBandLinearSolver}

  Wrapper function for \ID{SUNLinSol\_Band}, with identical input and
  output arguments.

\end{itemize}
%%
%%------------------------------------
%%
For solvers that include a Fortran interface module, the {\sunlinsolband}
module also includes a Fortran-callable function for creating a
\id{SUNLinearSolver} object.
\ucfunction{FSUNBANDLINSOLINIT}
{
  FSUNBANDLINSOLINIT(code, ier)
}
{
  The function \ID{FSUNBANDLINSOLINIT} can be called for Fortran programs
  to create a band \id{SUNLinearSolver} object.
}
{
  \begin{args}[code]
  \item[code] (\id{int*})
    is an integer input specifying the solver id (1 for {\cvode}, 2
    for {\ida}, 3 for {\kinsol}, and 4 for {\arkode}).
  \end{args}
}
{
  \id{ier} is a return completion flag equal to \id{0} for a success
  return and \id{-1} otherwise. See printed message for details in case
  of failure.
}
{
  This routine must be
  called \emph{after} both the {\nvector} and {\sunmatrix} objects have
  been initialized.
}
Additionally, when using {\arkode} with a non-identity
mass matrix, the {\sunlinsolband} module includes a Fortran-callable
function for creating a \id{SUNLinearSolver} mass matrix solver
object.
\ucfunction{FSUNMASSBANDLINSOLINIT}
{
  FSUNMASSBANDLINSOLINIT(ier)
}
{
  The function \ID{FSUNMASSBANDLINSOLINIT} can be called for Fortran programs
  to create a band \id{SUNLinearSolver} object for mass matrix linear
  systems.
}
{
}
{
  \id{ier} is a \id{int} return completion flag equal to \id{0} for a success
  return and \id{-1} otherwise. See printed message for details in case
  of failure.
}
{
  This routine must be
  called \emph{after} both the {\nvector} and {\sunmatrix} mass-matrix
  objects have been initialized.
}

%---------------------------------------------------------------------------
\subsection{{\sunlinsolband} description}\label{ss:sunlinsol_band_description}



The {\sunlinsolband} module defines the {\em
content} field of a \id{SUNLinearSolver} to be the following structure:
%%
\begin{verbatim} 
struct _SUNLinearSolverContent_Band {
  sunindextype N;
  sunindextype *pivots;
  long int last_flag;
};
\end{verbatim}
%%
These entries of the \emph{content} field contain the following
information:
\begin{description}
  \item[N] - size of the linear system,
  \item[pivots] - index array for partial pivoting in LU factorization,
  \item[last\_flag] - last error return flag from internal function evaluations.
\end{description}

This solver is constructed to perform the following operations:
\begin{itemize}
\item The ``setup'' call performs a $LU$ factorization with
  partial (row) pivoting, $PA=LU$, where $P$ is a permutation matrix,
  $L$ is a lower triangular matrix with 1's on the diagonal, and $U$
  is an upper triangular matrix.  This factorization is stored
  in-place on the input {\sunmatband} object $A$, with pivoting
  information encoding $P$ stored in the \id{pivots} array.
\item The ``solve'' call performs pivoting and forward and
  backward substitution using the stored \id{pivots} array and the
  $LU$ factors held in the {\sunmatband} object.
\item
  {\warn} $A$ must be allocated to accommodate the increase in upper
  bandwidth that occurs during factorization.  More precisely, if $A$
  is a band matrix with upper bandwidth \id{mu} and lower bandwidth
  \id{ml}, then the upper triangular factor $U$ can have upper
  bandwidth as big as \id{smu = MIN(N-1,mu+ml)}. The lower triangular
  factor $L$ has lower bandwidth \id{ml}.
\end{itemize}


%%
%%----------------------------------------------
%%

\noindent The {\sunlinsolband} module defines band implementations of all
``direct'' linear solver operations listed in Sections
\ref{ss:sunlinsol_CoreFn}-\ref{ss:sunlinsol_GetFn}:
\begin{itemize}
\item \id{SUNLinSolGetType\_Band}
\item \id{SUNLinSolInitialize\_Band} -- this does nothing, since all
  consistency checks are performed at solver creation.
\item \id{SUNLinSolSetup\_Band} -- this performs the $LU$ factorization.
\item \id{SUNLinSolSolve\_Band} -- this uses the $LU$ factors
  and \id{pivots} array to perform the solve.
\item \id{SUNLinSolLastFlag\_Band}
\item \id{SUNLinSolSpace\_Band} -- this only returns information for
  the storage \emph{within} the solver object, i.e.~storage
  for \id{N}, \id{last\_flag}, and \id{pivots}.
\item \id{SUNLinSolFree\_Band}
\end{itemize}

% ====================================================================
\section{The SUNLinearSolver\_LapackDense implementation}
\label{ss:sunlinsol_lapdense}
% ====================================================================

This section describes the {\sunlinsol} implementation for solving dense linear
systems with LAPACK. The {\sunlinsollapdense} module is designed to be used with the
corresponding {\sunmatdense} matrix type, and one of the serial or
shared-memory {\nvector} implementations ({\nvecs}, {\nvecopenmp}, or
{\nvecpthreads}).

To access the {\sunlinsollapdense} module, include the header file \newline
\id{sunlinsol/sunlinsol\_lapackdense.h}. The installed module library to link
to is \newline
\id{libsundials\_sunlinsollapackdense.\textit{lib}} where \id{\em.lib}
is typically \id{.so} for shared libraries and \id{.a} for static libraries.

The {\sunlinsollapdense} module is a {\sunlinsol} wrapper for
the LAPACK dense matrix factorization and solve routines, \id{*GETRF}
and \id{*GETRS}, where \id{*} is either \id{D} or \id{S}, depending on
whether {\sundials} was configured to have \id{realtype} set to
\id{double} or \id{single}, respectively (see Section \ref{s:types}).
In order to use the {\sunlinsollapdense} module it is assumed
that LAPACK has been installed on the system prior to installation of
{\sundials}, and that {\sundials} has been configured appropriately to
link with LAPACK (see Appendix \ref{c:install} for details).  
We note that since there do not exist 128-bit floating-point
factorization and solve routines in LAPACK, this interface cannot be
compiled when using \id{extended} precision for \id{realtype}.
Similarly, since there do not exist 64-bit integer LAPACK routines,
the {\sunlinsollapdense} module also cannot be compiled when using
64-bit integers for the \id{sunindextype}. {\warn}


% ====================================================================
\subsection{SUNLinearSolver\_LapackDense description}
\label{ss:sunlinsol_lapdense_description}
% ====================================================================

This solver is constructed to perform the following operations:
\begin{itemize}
\item The ``setup'' call performs a $LU$ factorization with
  partial (row) pivoting ($\mathcal O(N^3)$ cost), $PA=LU$, where $P$
  is a permutation matrix, $L$ is a lower triangular matrix with 1's
  on the diagonal, and $U$ is an upper triangular matrix.  This
  factorization is stored in-place on the input {\sunmatdense} object
  $A$, with pivoting information encoding $P$ stored in
  the \id{pivots} array.
\item The ``solve'' call performs pivoting and forward and
  backward substitution using the stored \id{pivots} array and the
  $LU$ factors held in the {\sunmatdense} object ($\mathcal O(N^2)$
  cost).
\end{itemize}


% ====================================================================
\subsection{SUNLinearSolver\_LapackDense functions}
\label{ss:sunlinsol_lapdense_functions}
% ====================================================================

The {\sunlinsollapdense} module provides the following user-callable constructor
for creating a \newline \id{SUNLinearSolver} object.
%
% --------------------------------------------------------------------
%
\ucfunctiond{SUNLinSol\_LapackDense}
{
  LS = SUNLinSol\_LapackDense(y, A);
}
{
  The function \ID{SUNLinSol\_LapackDense} creates and allocates memory for
  a LAPACK-based, dense \id{SUNLinearSolver} object.
}
{
  \begin{args}[y]
  \item[y] (\id{N\_Vector})
    a template for cloning vectors needed within the solver
  \item[A] (\id{SUNMatrix})
    a {\sunmatdense} matrix template for cloning matrices needed
    within the solver 
  \end{args}
}
{
  This returns a \id{SUNLinearSolver} object.  If either \id{A} or
  \id{y} are incompatible then this routine will return \id{NULL}.
}
{
  This routine will perform consistency checks to ensure that it is
  called with consistent {\nvector} and {\sunmatrix} implementations.
  These are currently limited to the {\sunmatdense} matrix type and
  the {\nvecs}, {\nvecopenmp}, and {\nvecpthreads} vector types.  As
  additional compatible matrix and vector implementations are added to
  {\sundials}, these will be included within this compatibility check.
}
{SUNLapackDense}
%
% --------------------------------------------------------------------
%
The {\sunlinsollapdense} module defines dense implementations of all
``direct'' linear solver operations listed in Sections
\ref{ss:sunlinsol_CoreFn} -- \ref{ss:sunlinsol_GetFn}:
\begin{itemize}
\item \id{SUNLinSolGetType\_LapackDense}
\item \id{SUNLinSolInitialize\_LapackDense} -- this does nothing, since all
  consistency checks are performed at solver creation.
\item \id{SUNLinSolSetup\_LapackDense} -- this calls either
  \id{DGETRF} or \id{SGETRF} to perform the $LU$ factorization.
\item \id{SUNLinSolSolve\_LapackDense} -- this calls either
  \id{DGETRS} or \id{SGETRS} to use the $LU$ factors and \id{pivots}
  array to perform the solve.
\item \id{SUNLinSolLastFlag\_LapackDense}
\item \id{SUNLinSolSpace\_LapackDense} -- this only returns information for
  the storage \emph{within} the solver object, i.e.~storage
  for \id{N}, \id{last\_flag}, and \id{pivots}.
\item \id{SUNLinSolFree\_LapackDense}
\end{itemize}


% ====================================================================
\subsection{SUNLinearSolver\_LapackDense Fortran interfaces}
\label{ss:sunlinsol_lapdense_fortran}
% ====================================================================

For solvers that include a {\F} 77 interface module, the {\sunlinsollapdense}
module also includes a Fortran-callable function for creating a
\id{SUNLinearSolver} object.
%
% --------------------------------------------------------------------
%
\ucfunction{FSUNLAPACKDENSEINIT}
{
  FSUNLAPACKDENSEINIT(code, ier)
}
{
  The function \ID{FSUNLAPACKDENSEINIT} can be called for Fortran programs
  to create a LAPACK-based dense \id{SUNLinearSolver} object.
}
{
  \begin{args}[code]
  \item[code] (\id{int*})
    is an integer input specifying the solver id (1 for {\cvode}, 2
    for {\ida}, 3 for {\kinsol}, and 4 for {\arkode}).
  \end{args}
}
{
  \id{ier} is a return completion flag equal to \id{0} for a success
  return and \id{-1} otherwise. See printed message for details in case
  of failure.
}
{
  This routine must be
  called \emph{after} both the {\nvector} and {\sunmatrix} objects have
  been initialized.
}
Additionally, when using {\arkode} with a non-identity
mass matrix, the {\sunlinsollapdense} module includes a Fortran-callable
function for creating a \id{SUNLinearSolver} mass matrix solver
object.
%
% --------------------------------------------------------------------
%
\ucfunction{FSUNMASSLAPACKDENSEINIT}
{
  FSUNMASSLAPACKDENSEINIT(ier)
}
{
  The function \ID{FSUNMASSLAPACKDENSEINIT} can be called for Fortran programs
  to create a LAPACK-based, dense \id{SUNLinearSolver} object for mass matrix linear
  systems.
}
{}
{
  \id{ier} is a \id{int} return completion flag equal to \id{0} for a success
  return and \id{-1} otherwise. See printed message for details in case
  of failure.
}
{
  This routine must be
  called \emph{after} both the {\nvector} and {\sunmatrix} mass-matrix
  objects have been initialized.
}


% ====================================================================
\subsection{SUNLinearSolver\_LapackDense content}
\label{ss:sunlinsol_lapdense_content}
% ====================================================================

The {\sunlinsollapdense} module defines the \textit{content} field of a
\id{SUNLinearSolver} as the following structure:
%%
\begin{verbatim} 
struct _SUNLinearSolverContent_Dense {
  sunindextype N;
  sunindextype *pivots;
  long int last_flag;
};
\end{verbatim}
%%
These entries of the \emph{content} field contain the following
information:
\begin{args}[last\_flag]
  \item[N] - size of the linear system,
  \item[pivots] - index array for partial pivoting in LU factorization,
  \item[last\_flag] - last error return flag from internal function evaluations.
\end{args}


% ====================================================================
\section{The SUNLinearSolver\_LapackBand implementation}
\label{ss:sunlinsol_lapband}
% ====================================================================

This section describes the {\sunlinsol} implementation for solving banded linear
systems with LAPACK. The {\sunlinsollapband} module is designed to be used with the
corresponding {\sunmatband} matrix type, and one of the serial or
shared-memory {\nvector} implementations ({\nvecs}, {\nvecopenmp}, or
{\nvecpthreads}).

To access the {\sunlinsollapband} module, include the header file \newline
\id{sunlinsol/sunlinsol\_lapackband.h}. The installed module library to link
to is \newline
\id{libsundials\_sunlinsollapackband.\textit{lib}} where \id{\em.lib}
is typically \id{.so} for shared libraries and \id{.a} for static libraries.

The {\sunlinsollapband} module is a {\sunlinsol} wrapper for
the LAPACK band matrix factorization and solve routines, \id{*GBTRF}
and \id{*GBTRS}, where \id{*} is either \id{D} or \id{S}, depending on
whether {\sundials} was configured to have \id{realtype} set to
\id{double} or \id{single}, respectively (see Section \ref{s:types}).
In order to use the {\sunlinsollapband} module it is assumed
that LAPACK has been installed on the system prior to installation of
{\sundials}, and that {\sundials} has been configured appropriately to
link with LAPACK (see Appendix \ref{c:install} for details).  
We note that since there do not exist 128-bit floating-point
factorization and solve routines in LAPACK, this interface cannot be
compiled when using \id{extended} precision for \id{realtype}.
Similarly, since there do not exist 64-bit integer LAPACK routines,
the {\sunlinsollapband} module also cannot be compiled when using
64-bit integers for the \id{sunindextype}. {\warn}


% ====================================================================
\subsection{SUNLinearSolver\_LapackBand description}
\label{ss:sunlinsol_lapband_description}
% ====================================================================

This solver is constructed to perform the following operations:
\begin{itemize}
\item The ``setup'' call performs a $LU$ factorization with
  partial (row) pivoting, $PA=LU$, where $P$ is a permutation matrix,
  $L$ is a lower triangular matrix with 1's on the diagonal, and $U$
  is an upper triangular matrix.  This factorization is stored
  in-place on the input {\sunmatband} object $A$, with pivoting
  information encoding $P$ stored in the \id{pivots} array.
\item The ``solve'' call performs pivoting and forward and
  backward substitution using the stored \id{pivots} array and the
  $LU$ factors held in the {\sunmatband} object.
\item
  $A$ must be allocated to accommodate the increase in upper
  bandwidth that occurs during factorization.  More precisely, if $A$
  is a band matrix with upper bandwidth \id{mu} and lower bandwidth
  \id{ml}, then the upper triangular factor $U$ can have upper
  bandwidth as big as \id{smu = MIN(N-1,mu+ml)}. The lower triangular
  factor $L$ has lower bandwidth \id{ml}. {\warn}
\end{itemize}


% ====================================================================
\subsection{SUNLinearSolver\_LapackBand functions}
\label{ss:sunlinsol_lapband_functions}
% ====================================================================

The {\sunlinsollapband} module provides the following user-callable constructor
for creating a \newline \id{SUNLinearSolver} object.
%
% --------------------------------------------------------------------
%
\ucfunctiond{SUNLinSol\_LapackBand}
{
  LS = SUNLinSol\_LapackBand(y, A);
}
{
  The function \ID{SUNLinSol\_LapackBand} creates and allocates memory for
  a LAPACK-based, band \id{SUNLinearSolver} object.
}
{
  \begin{args}[y]
  \item[y] (\id{N\_Vector})
    a template for cloning vectors needed within the solver
  \item[A] (\id{SUNMatrix})
    a {\sunmatband} matrix template for cloning matrices needed
    within the solver 
  \end{args}
}
{
  This returns a \id{SUNLinearSolver} object.  If either \id{A} or
  \id{y} are incompatible then this routine will return \id{NULL}.
}
{
  This routine will perform consistency checks to ensure that it is
  called with consistent {\nvector} and {\sunmatrix} implementations.
  These are currently limited to the {\sunmatband} matrix type and
  the {\nvecs}, {\nvecopenmp}, and {\nvecpthreads} vector types.  As
  additional compatible matrix and vector implementations are added to
  {\sundials}, these will be included within this compatibility check.
  
  Additionally, this routine will verify that the input matrix \id{A}
  is allocated with appropriate upper bandwidth storage for the $LU$
  factorization.
}
{SUNLapackBand}
%
% --------------------------------------------------------------------
%
The {\sunlinsollapband} module defines band implementations of all
``direct'' linear solver operations listed in Sections
\ref{ss:sunlinsol_CoreFn} -- \ref{ss:sunlinsol_GetFn}:
\begin{itemize}
\item \id{SUNLinSolGetType\_LapackBand}
\item \id{SUNLinSolInitialize\_LapackBand} -- this does nothing, since all
  consistency checks are performed at solver creation.
\item \id{SUNLinSolSetup\_LapackBand} -- this calls either
  \id{DGBTRF} or \id{SGBTRF} to perform the $LU$ factorization.
\item \id{SUNLinSolSolve\_LapackBand} -- this calls either
  \id{DGBTRS} or \id{SGBTRS} to use the $LU$ factors and \id{pivots}
  array to perform the solve.
\item \id{SUNLinSolLastFlag\_LapackBand}
\item \id{SUNLinSolSpace\_LapackBand} -- this only returns information for
  the storage \emph{within} the solver object, i.e.~storage
  for \id{N}, \id{last\_flag}, and \id{pivots}.
\item \id{SUNLinSolFree\_LapackBand}
\end{itemize}


% ====================================================================
\subsection{SUNLinearSolver\_LapackBand Fortran interfaces}
\label{ss:sunlinsol_lapband_fortran}
% ====================================================================

For solvers that include a {\F} 77 interface module, the {\sunlinsollapband}
module also includes a Fortran-callable function for creating a
\id{SUNLinearSolver} object.
%
% --------------------------------------------------------------------
%
\ucfunction{FSUNLAPACKDENSEINIT}
{
  FSUNLAPACKBANDINIT(code, ier)
}
{
  The function \ID{FSUNLAPACKBANDINIT} can be called for Fortran programs
  to create a LAPACK-based band \id{SUNLinearSolver} object.
}
{
  \begin{args}[code]
  \item[code] (\id{int*})
    is an integer input specifying the solver id (1 for {\cvode}, 2
    for {\ida}, 3 for {\kinsol}, and 4 for {\arkode}).
  \end{args}
}
{
  \id{ier} is a return completion flag equal to \id{0} for a success
  return and \id{-1} otherwise. See printed message for details in case
  of failure.
}
{
  This routine must be
  called \emph{after} both the {\nvector} and {\sunmatrix} objects have
  been initialized.
}
Additionally, when using {\arkode} with a non-identity
mass matrix, the {\sunlinsollapband} module includes a Fortran-callable
function for creating a \id{SUNLinearSolver} mass matrix solver
object.
%
% --------------------------------------------------------------------
%
\ucfunction{FSUNMASSLAPACKBANDINIT}
{
  FSUNMASSLAPACKBANDINIT(ier)
}
{
  The function \ID{FSUNMASSLAPACKBANDINIT} can be called for Fortran programs
  to create a LAPACK-based, band \id{SUNLinearSolver} object for mass matrix linear
  systems.
}
{}
{
  \id{ier} is a \id{int} return completion flag equal to \id{0} for a success
  return and \id{-1} otherwise. See printed message for details in case
  of failure.
}
{
  This routine must be
  called \emph{after} both the {\nvector} and {\sunmatrix} mass-matrix
  objects have been initialized.
}


% ====================================================================
\subsection{SUNLinearSolver\_LapackBand content}
\label{ss:sunlinsol_lapband_content}
% ====================================================================

The {\sunlinsollapband} module defines the \textit{content} field of a
\id{SUNLinearSolver} as the following structure:
%%
\begin{verbatim} 
struct _SUNLinearSolverContent_Band {
  sunindextype N;
  sunindextype *pivots;
  long int last_flag;
};
\end{verbatim}
%%
These entries of the \emph{content} field contain the following
information:
\begin{args}[last\_flag]
  \item[N] - size of the linear system,
  \item[pivots] - index array for partial pivoting in LU factorization,
  \item[last\_flag] - last error return flag from internal function evaluations.
\end{args}


%% This is a shared SUNDIALS TEX file with a description of the
%% klu sunlinsol implementation
%%

The {\klu} implementation of the {\sunlinsol} module provided with
{\sundials}, {\sunlinsolklu}, is designed to be used with the
corresponding {\sunmatsparse} matrix type, and one of the serial or
shared-memory {\nvector} implementations ({\nvecs}, {\nvecopenmp}, or 
{\nvecpthreads}).  The {\sunlinsolklu} module defines the {\em
content} field of a \id{SUNLinearSolver} to be the following structure:
%%
\begin{verbatim} 
struct _SUNLinearSolverContent_KLU {
  long int         last_flag;
  int              first_factorize;
  sun_klu_symbolic *symbolic;
  sun_klu_numeric  *numeric;
  sun_klu_common   common;
  sunindextype     (*klu_solver)(sun_klu_symbolic*, sun_klu_numeric*,
                                 sunindextype, sunindextype,
                                 double*, sun_klu_common*);
};
\end{verbatim}
%%
These entries of the \emph{content} field contain the following
information:
\begin{description}
  \item[last\_flag] - last error return flag from internal function evaluations,
  \item[first\_factorize] - flag indicating whether the factorization
    has ever been performed, 
  \item[Symbolic] - {\klu} storage structure for symbolic factorization components,
  \item[Numeric] - {\klu} storage structure for numeric factorization components,
  \item[Common] - storage structure for common {\klu} solver components,
  \item[klu\_solver] -- pointer to the appropriate {\klu} solver function
    (depending on whether it is using a CSR or CSC sparse matrix).
\end{description}

{\warn} The {\sunlinsolklu} module is a {\sunlinsol} wrapper for
the {\klu} sparse matrix factorization and solver library written by Tim
Davis \cite{KLU_site,DaPa:10}.  In order to use the
{\sunlinsolklu} interface to {\klu}, it is assumed that {\klu} has
been installed on the system prior to installation of {\sundials}, and
that {\sundials} has been configured appropriately to link with {\klu}
(see Appendix \ref{c:install} for details).  Additionally, this
wrapper only supports double-precision calculations, and therefore
cannot be compiled if {\sundials} is configured to have \id{realtype}
set to either \id{extended} or \id{single} (see Section \ref{s:types}).
Since the {\klu} library supports both 32-bit and 64-bit integers, this
interface will be compiled for either of the available \id{sunindextype} options.

The {\klu} library has a symbolic factorization routine that computes
the permutation of the linear system matrix to block triangular form
and the permutations that will pre-order the diagonal blocks (the only
ones that need to be factored) to reduce fill-in (using AMD, COLAMD,
CHOLAMD, natural, or an ordering given by the user).  Of these
ordering choices, the default value in the {\sunlinsolklu} 
module is the COLAMD ordering.

{\klu} breaks the factorization into two separate parts.  The first is
a symbolic factorization and the second is a numeric factorization
that returns the factored matrix along with final pivot information.   
{\klu} also has a refactor routine that can be called instead of the numeric 
factorization.  This routine will reuse the pivot information.  This routine 
also returns diagnostic information that a user can examine to determine if 
numerical stability is being lost and a full numerical factorization should 
be done instead of the refactor.

Since the linear systems that arise within the context of {\sundials}
calculations will typically have identical sparsity patterns, the
{\sunlinsolklu} module is constructed to perform the
following operations:
\begin{itemize}
\item The first time that the ``setup'' routine is called, it
  performs the symbolic factorization, followed by an initial
  numerical factorization.  
\item On subsequent calls to the ``setup'' routine, it calls the
  appropriate {\klu} ``refactor'' routine, followed by estimates of
  the numerical conditioning using the relevant ``rcond'', and if
  necessary ``condest'', routine(s).  If these estimates of the
  condition number are larger than $\varepsilon^{-2/3}$ (where
  $\varepsilon$ is the double-precision unit roundoff), then a new
  factorization is performed.
\item The module includes the routine \id{SUNKLUReInit}, that 
  can be called by the user to force a full refactorization at the
  next ``setup'' call. 
\item The ``solve'' call performs pivoting and forward and
  backward substitution using the stored {\klu} data structures.  We
  note that in this solve {\klu} operates on the native data arrays
  for the right-hand side and solution vectors, without requiring
  costly data copies.
\end{itemize}


\noindent The header file to be included when using this module 
is \id{sunlinsol/sunlinsol\_klu.h}. \\
%%
%%----------------------------------------------
%%
The {\sunlinsolklu} module defines implementations of all
``direct'' linear solver operations listed in
Table \ref{t:sunlinsolops}:
\begin{itemize}
\item \id{SUNLinSolGetType\_KLU}
\item \id{SUNLinSolInitialize\_KLU} -- this sets the
  \id{first\_factorize} flag to 1, forcing both symbolic and numerical
  factorizations on the subsequent ``setup'' call.
\item \id{SUNLinSolSetup\_KLU} -- this performs either a $LU$
  factorization or refactorization of the input matrix.
\item \id{SUNLinSolSolve\_KLU} -- this calls the appropriate {\klu}
  solve routine to utilize the $LU$ factors to solve the linear
  system. 
\item \id{SUNLinSolLastFlag\_KLU}
\item \id{SUNLinSolSpace\_KLU} -- this only returns information for
  the storage within the solver \emph{interface}, i.e.~storage for the
  integers \id{last\_flag} and \id{first\_factorize}.  For additional
  space requirements, see the {\klu} documentation.
\item \id{SUNLinSolFree\_KLU}
\end{itemize}
The module {\sunlinsolklu} provides the following additional
user-callable routines: 
%%
\begin{itemize}

%%--------------------------------------

\item \ID{SUNKLU}

  This constructor function creates and allocates memory for a {\sunlinsolklu}
  object.  Its arguments are an {\nvector} and {\sunmatrix}, that it
  uses to determine the linear system size and to assess compatibility
  with the linear solver implementation. 

  This routine will perform consistency checks to ensure that it is
  called with consistent {\nvector} and {\sunmatrix} implementations.
  These are currently limited to the {\sunmatsparse} matrix type
  (using either CSR or CSC storage formats) and the {\nvecs},
  {\nvecopenmp}, and {\nvecpthreads} vector types.  As additional
  compatible matrix and vector implementations are added to
  {\sundials}, these will be included within this compatibility
  check. 

  If either \id{A} or \id{y} are incompatible then this routine will
  return \id{NULL}.

  \verb|SUNLinearSolver SUNKLU(N_Vector y, SUNMatrix A);|

%%--------------------------------------

\item \ID{SUNKLUReInit}

  This function reinitializes memory and flags for a new factorization
  (symbolic and numeric) to be conducted at the next solver setup
  call.  This routine is useful in the cases where the number of
  nonzeroes has changed or if the structure of the linear system has
  changed which would require a new symbolic (and numeric
  factorization). 

  The \id{reinit\_type} argument governs the level of
  reinitialization.  The allowed values are: 
  \begin{itemize}
  \item[1] The Jacobian matrix will be destroyed and a new one will be
    allocated based on the \id{nnz} value passed to this call.  New
    symbolic and numeric factorizations will be completed at the next
    solver setup.
  \item[2] Only symbolic and numeric factorizations will be completed.
    It is assumed that the Jacobian size has not exceeded the size of
    \id{nnz} given in the sparse matrix provided to the original
    constructor routine (or the previous \id{SUNKLUReInit} call). 
  \end{itemize}
  
  This routine assumes no other changes to solver use are necessary.

  The return values from this function are \id{SUNLS\_MEM\_NULL}
  (either \id{S} or \id{A} are \id{NULL}), \id{SUNLS\_ILL\_INPUT}
  (\id{A} does not have type \id{SUNMATRIX\_SPARSE} or
  \id{reinit\_type} is invalid), \id{SUNLS\_MEM\_FAIL} (reallocation
  of the sparse matrix failed) or \id{SUNLS\_SUCCESS}.
  
\begin{verbatim}
int SUNKLUReInit(SUNLinearSolver S, SUNMatrix A, 
                 sunindextype nnz, int reinit_type);
\end{verbatim}


%%--------------------------------------

\item \ID{SUNKLUSetOrdering}

  This function sets the ordering used by {\klu} for reducing fill in
  the linear solve.  Options for \id{ordering\_choice} are:
  \begin{itemize}
  \item[0] AMD,
  \item[1] COLAMD, and
  \item[2] the natural ordering.
  \end{itemize}
  The default is 1 for COLAMD.

  The return values from this function are \id{SUNLS\_MEM\_NULL}
  (\id{S} is \id{NULL}), \id{SUNLS\_ILL\_INPUT}
  (invalid \id{ordering\_choice}), or \id{SUNLS\_SUCCESS}.
  
  \verb|int SUNKLUSetOrdering(SUNLinearSolver S, int ordering_choice);|

\end{itemize}
%%
%%------------------------------------
%%
For solvers that include a Fortran interface module, the
{\sunlinsolklu} module also includes the Fortran-callable
function \id{FSUNKLUInit(code, ier)} to initialize this
{\sunlinsolklu} module for a given {\sundials} solver.  Here \id{code}
is an integer input solver id (1 for {\cvode}, 2 for {\ida}, 3 for {\kinsol},
4 for {\arkode}); \id{ier} is an error return flag equal to 0 for success
and -1 for failure. Both \id{code} and \id{ier}
are declared to match C type \id{int}. This
routine must be called \emph{after} both the {\nvector} and
{\sunmatrix} objects have been initialized.  Additionally, when using
{\arkode} with a non-identity mass matrix, the Fortran-callable function
\id{FSUNMassKLUInit(ier)} initializes this {\sunlinsolklu} module for
solving mass matrix linear systems.

The \id{SUNKLUReInit} and \ID{SUNKLUSetOrdering} routines also support
Fortran interfaces for the system and mass matrix solvers:
\begin{itemize}
\item \id{FSUNKLUReInit(code, NNZ, reinit\_type, ier)} -- \id{NNZ}
  should be commensurate with a C \id{long int} and \id{reinit\_type}
  should be commensurate with a C \id{int}
\item \id{FSUNMassKLUReInit(NNZ, reinit\_type, ier)}
\item \id{FSUNKLUSetOrdering(code, ordering, ier)} -- \id{ordering}
  should be commensurate with a C \id{int}
\item \id{FSUNMassKLUSetOrdering(ordering, ier)}
\end{itemize}

%% This is a shared SUNDIALS TEX file with a description of the
%% superludist sunlinsol implementation
%%
\section{The SUNLinearSolver\_SuperLUDIST implementation}\label{ss:sunlinsol_sludist}

The {\superludist} implementation of the {\sunlinsol} module provided with
{\sundials},\\
\noindent{\sunlinsolsludist}, is designed to be used with the
corresponding {\sunmatslunrloc} matrix type, and one of the serial, threaded
or parallel {\nvector} implementations ({\nvecs}, {\nvecopenmp}, {\nvecpthreads},
{\nvecp}, or {\nvecph}).

The header file to include when using this module
is \id{sunlinsol/sunlinsol\_superludist.h}. The installed module
library to link to is
\id{libsundials\_sunlinsolsuperludist.\textit{lib}}
where \id{\em.lib} is typically \id{.so} for shared libraries and
\id{.a} for static libraries.


% --------------------------------------------------------------------
\subsection{SUNLinearSolver\_SuperLUDIST description}\label{ss:sunlinsol_sludist_description}

The {\sunlinsolsludist} module is a {\sunlinsol} adapter for the
{\superludist} sparse matrix factorization and solver library written by
X. Sherry Li \cite{SuperLUDIST_site,GDL:07,LD:03,SLUUG:99}.
The package uses a SPMD parallel programming model and multithreading
to enhance efficiency in distributed-memory parallel environments with
multicore nodes and possibly GPU accelerators. It uses {\mpi} for communication,
{\openmp} for threading, and {\cuda} for GPU support. In order to use the
{\sunlinsolsludist} interface to {\superludist}, it is assumed that {\superludist}
has been installed on the system prior to installation of {\sundials}, and
that {\sundials} has been configured appropriately to link with {\superludist}
(see Appendix \ref{c:install} for details). Additionally, the adapter only
supports double-precision calculations, and therefore cannot be compiled if {\sundials}
is configured to use single or extended precision. Moreover, since the {\superludist}
library may be installed to support either 32-bit or 64-bit integers,
it is assumed that the {\superludist} library is installed using the same
integer size as {\sundials}.

The {\superludist} library provides many options to control how a linear
system will be solved. These options may be set by a user on an instance
of the \id{superlu\_dist\_options\_t} struct, and then it may be provided
as an argument to the {\sunlinsolsludist} constructor. The {\sunlinsolsludist}
module will respect all options set except for \id{Fact} -- this option is
necessarily modified by the {\sunlinsolsludist} module in the setup and solve routines.

Since the linear systems that arise within the context of {\sundials}
calculations will typically have identical sparsity patterns, the
{\sunlinsolsludist} module is constructed to perform the
following operations:
\begin{itemize}
\item The first time that the ``setup'' routine is called, it
  sets the {\superludist} option \id{Fact} to \id{DOFACT} so that a subsequent
  call to the ``solve'' routine will perform a symbolic factorization,
  followed by an initial numerical factorization before continuing
  to solve the system.
\item On subsequent calls to the ``setup'' routine, it sets the
  {\superludist} option \id{Fact} to \id{SamePattern} so that
  a subsequent call to ``solve'' will perform factorization assuming
  the same sparsity pattern as prior, i.e. it will reuse the column
  permutation vector.
\item If ``setup'' is called prior to the ``solve'' routine, then the ``solve''
  routine will perform a symbolic factorization, followed by an initial
  numerical factorization before continuing to the sparse triangular
  solves, and, potentially, iterative refinement. If ``setup'' is not
  called prior, ``solve'' will skip to the triangular solve step. We
  note that in this solve {\superludist} operates on the native data arrays
  for the right-hand side and solution vectors, without requiring costly data copies.
\end{itemize}


\subsection{SUNLinearSolver\_SuperLUDIST functions}\label{ss:sunlinsol_sludist_functions}

The {\sunlinsolsludist} module defines implementations of all
``direct'' linear solver operations listed in Sections
\ref{ss:sunlinsol_CoreFn}-\ref{ss:sunlinsol_GetFn}:
\begin{itemize}
\item \id{SUNLinSolGetType\_SuperLUDIST}
\item \id{SUNLinSolInitialize\_SuperLUDIST} -- this sets the
  \id{first\_factorize} flag to 1 and resets the internal {\superludist}
  statistics variables.
\item \id{SUNLinSolSetup\_SuperLUDIST} -- this sets the appropriate
  {\superludist} options so that a subsequent solve will perform a
  symbolic and numerical factorization before proceeding with the
  triangular solves
\item \id{SUNLinSolSolve\_SuperLUDIST} -- this calls the {\superludist}
  solve routine to perform factorization (if the setup routine
  was called prior) and then use the $LU$ factors to solve the
  linear system.
\item \id{SUNLinSolLastFlag\_SuperLUDIST}
\item \id{SUNLinSolSpace\_SuperLUDIST} -- this only returns information for
  the storage within the solver \emph{interface}, i.e.~storage for the
  integers \id{last\_flag} and \id{first\_factorize}.  For additional
  space requirements, see the {\superludist} documentation.
\item \id{SUNLinSolFree\_SuperLUDIST}
\end{itemize}

In addition, the module {\sunlinsolsludist} provides the following user-callable routines:
%%
% --------------------------------------------------------------------
\ucfunction{SUNLinSol\_SuperLUDIST}
{
  LS = SUNLinSol\_SuperLUDIST(y, A, grid, lu, scaleperm, solve, stat, options);
}
{
  The function \ID{SUNLinSol\_SuperLUDIST} creates and allocates memory for a
  {\sunlinsolsludist} object.
}
{
  \begin{args}[options]
  \item[y] (\id{N\_Vector})
    a template for cloning vectors needed within the solver
  \item[A] (\id{SUNMatrix})
    a {\sunmatslunrloc} matrix template for cloning matrices needed
    within the solver
  \item[grid] (\id{gridinfo\_t*})
  \item[lu] (\id{LUstruct\_t*})
  \item[scaleperm] (\id{ScalePermstruct\_t*})
  \item[solve] (\id{SOLVEstruct\_t*})
  \item[stat] (\id{SuperLUStat\_t*})
  \item[options] (\id{superlu\_dist\_options\_t*})
  \end{args}
}
{
  This returns a \id{SUNLinearSolver} object.  If either \id{A} or
  \id{y} are incompatible then this routine will return \id{NULL}.
}
{
  This routine analyzes the input matrix and vector to determine the
  linear system size and to assess compatibility with the {\superludist}
  library.

  This routine will perform consistency checks to ensure that it is
  called with consistent {\nvector} and {\sunmatrix} implementations.
  These are currently limited to the {\sunmatslunrloc} matrix type
  and the {\nvecs}, {\nvecp}, {\nvecph}, {\nvecopenmp}, and {\nvecpthreads}
  vector types. As additional compatible matrix and vector implementations
  are added to {\sundials}, these will be included within this compatibility
  check.

  The \id{grid}, \id{lu}, \id{scaleperm}, \id{solve}, and \id{options} arguments
  are not checked and are passed directly to {\superludist} routines.

  Some struct members of the \id{options} argument are modified internally
  by the {\sunlinsolsludist} solver. Specifically the member \id{Fact},
  is modified in the setup and solve routines.
}

% --------------------------------------------------------------------
\ucfunction{SUNLinSol\_SuperLUDIST\_GetBerr}
{
  realtype berr = SUNLinSol\_SuperLUDIST\_GetBerr(LS);
}
{
  The function \ID{SUNLinSol\_SuperLUDIST\_GetBerr} returns the componentwise
  relative backward error of the computed solution.
}
{
  \begin{args}[LS]
  \item[LS] (\id{SUNLinearSolver})
    the {\sunlinsolsludist} object
  \end{args}
}
{
  \id{realtype}
}
{
}

% --------------------------------------------------------------------
\ucfunction{SUNLinSol\_SuperLUDIST\_GetGridinfo}
{
  gridinfo\_t *grid = SUNLinSol\_SuperLUDIST\_GetGridinfo(LS);
}
{
  The function \ID{SUNLinSol\_SuperLUDIST\_GetGridinfo} returns the
  {\superludist} structure that contains the 2D process grid.
}
{
  \begin{args}[LS]
  \item[LS] (\id{SUNLinearSolver})
    the {\sunlinsolsludist} object
  \end{args}
}
{
  \id{gridinfo\_t*}
}
{
}

% --------------------------------------------------------------------
\ucfunction{SUNLinSol\_SuperLUDIST\_GetLUstruct}
{
  LUstruct\_t *lu = SUNLinSol\_SuperLUDIST\_GetLUstruct(LS);
}
{
  The function \ID{SUNLinSol\_SuperLUDIST\_GetLUstruct} returns the
  {\superludist} structure that contains the distributed $L$ and $U$ factors.
}
{
  \begin{args}[LS]
  \item[LS] (\id{SUNLinearSolver})
    the {\sunlinsolsludist} object
  \end{args}
}
{
  \id{LUstruct\_t*}
}
{
}

% --------------------------------------------------------------------
\ucfunction{SUNLinSol\_SuperLUDIST\_GetSuperLUOptions}
{
  superlu\_dist\_options\_t *opts = SUNLinSol\_SuperLUDIST\_GetSuperLUOptions(LS);
}
{
  The function \ID{SUNLinSol\_SuperLUDIST\_GetSuperLUOptions} returns the
  {\superludist} structure that contains the options which control how
  the linear system is factorized and solved.
}
{
  \begin{args}[LS]
  \item[LS] (\id{SUNLinearSolver})
    the {\sunlinsolsludist} object
  \end{args}
}
{
  \id{superlu\_dist\_options\_t*}
}
{
}

% --------------------------------------------------------------------
\ucfunction{SUNLinSol\_SuperLUDIST\_GetScalePermstruct}
{
  ScalePermstruct\_t *sp = SUNLinSol\_SuperLUDIST\_GetScalePermstruct(LS);
}
{
  The function \ID{SUNLinSol\_SuperLUDIST\_GetScalePermstruct} returns the
  {\superludist} structure that contains the vectors that describe the
  transformations done to the matrix, $A$.
}
{
  \begin{args}[LS]
  \item[LS] (\id{SUNLinearSolver})
    the {\sunlinsolsludist} object
  \end{args}
}
{
  \id{ScalePermstruct\_t*}
}
{
}

% --------------------------------------------------------------------
\ucfunction{SUNLinSol\_SuperLUDIST\_GetSOLVEstruct}
{
  SOLVEstruct\_t *solve = SUNLinSol\_SuperLUDIST\_GetSOLVEstruct(LS);
}
{
  The function \ID{SUNLinSol\_SuperLUDIST\_GetSOLVEstruct} returns the
  {\superludist} structure that contains information for communication
  during the solution phase.
}
{
  \begin{args}[LS]
  \item[LS] (\id{SUNLinearSolver})
    the {\sunlinsolsludist} object
  \end{args}
}
{
  \id{SOLVEstruct\_t*}
}
{
}

% --------------------------------------------------------------------
\ucfunction{SUNLinSol\_SuperLUDIST\_GetSuperLUStat}
{
  SuperLUStat\_t *stat = SUNLinSol\_SuperLUDIST\_GetSuperLUStat(LS);
}
{
  The function \ID{SUNLinSol\_SuperLUDIST\_GetSuperLUStat} returns the
  {\superludist} structure that stores information about runtime and
  flop count.
}
{
  \begin{args}[LS]
  \item[LS] (\id{SUNLinearSolver})
    the {\sunlinsolsludist} object
  \end{args}
}
{
  \id{SuperLUStat\_t*}
}
{
}


% --------------------------------------------------------------------
\subsection{SUNLinearSolver\_SuperLUDIST content}\label{ss:sunlinsol_sludist_content}

The {\sunlinsolsludist} module defines the {\em
content} field of a \id{SUNLinearSolver} to be the following structure:
%%
\begin{verbatim}
struct _SUNLinearSolverContent_SuperLUDIST {
  booleantype             first_factorize;
  int                     last_flag;
  realtype                berr;
  gridinfo_t              *grid;
  LUstruct_t              *lu;
  superlu_dist_options_t  *options;
  ScalePermstruct_t       *scaleperm;
  SOLVEstruct_t           *solve;
  SuperLUStat_t           *stat;
  sunindextype            N;
};
\end{verbatim}
%%
These entries of the \emph{content} field contain the following
information:
\begin{description}
  \item[first\_factorize] - flag indicating whether the factorization
    has ever been performed,
  \item[last\_flag] - last error return flag from calls to internal routines,
  \item[berr] - the componentwise relative backward error of the computed solution,
  \item[grid] - pointer to the {\superludist} structure that stores the 2D process grid,
  \item[lu] - pointer to the {\superludist} structure that stores the distributed $L$
    and $U$ factors,
  \item[options] - pointer to {\superludist} options structure,
  \item[scaleperm] - pointer to the {\superludist} structure that stores vectors describing
    the transformations done to the matrix, $A$,
  \item[solve] - pointer to the {\superludist} solve structure,
  \item[stat] - pointer to the {\superludist} structure that stores information about runtime
    and flop count,
  \item[N] - the number of equations in the system
\end{description}

% ====================================================================
\section{The SUNLinearSolver\_SuperLUMT implementation}
\label{ss:sunlinsol_superlumt}
% ====================================================================

This section describes the {\sunlinsol} implementation for solving sparse linear
systems with SuperLU\_MT. The {\superlumt} module is designed to be used with the
corresponding {\sunmatsparse} matrix type, and one of the serial or
shared-memory {\nvector} implementations ({\nvecs}, {\nvecopenmp}, or 
{\nvecpthreads}). While these are compatible, it is not recommended
to use a threaded vector module with {\sunlinsolslumt} unless it is
the {\nvecopenmp} module and the {\superlumt} library has also been
compiled with OpenMP.

The header file to include when using this module 
is \id{sunlinsol/sunlinsol\_superlumt.h}. The installed module
library to link to is
\id{libsundials\_sunlinsolsuperlumt.\textit{lib}}
where \id{\em.lib} is typically \id{.so} for shared libraries and
\id{.a} for static libraries.

The {\sunlinsolslumt} module is a {\sunlinsol} wrapper for
the {\superlumt} sparse matrix factorization and solver library
written by X. Sherry Li \cite{SuperLUMT_site,Li:05,DGL:99}.  The
package performs matrix factorization using threads to enhance
efficiency in shared memory parallel environments.  It should be noted
that threads are only used in the factorization step.  In
order to use the {\sunlinsolslumt} interface to {\superlumt}, it is
assumed that {\superlumt} has been installed on the system prior to
installation of {\sundials}, and that {\sundials} has been configured
appropriately to link with {\superlumt} (see Appendix \ref{c:install}
for details).  Additionally, this wrapper only supports single- and
double-precision calculations, and therefore cannot be compiled if
{\sundials} is configured to have \id{realtype} set to \id{extended}
(see Section \ref{s:types}).  Moreover, since the {\superlumt} library
may be installed to support either 32-bit or 64-bit integers, it is
assumed that the {\superlumt} library is installed using the same
integer precision as the {\sundials} \id{sunindextype} option. {\warn}

% ====================================================================
\subsection{SUNLinearSolver\_SuperLUMT description}
\label{ss:sunlinsol_slumt_usage}
% ====================================================================

The {\superlumt} library has a symbolic factorization routine that
computes the permutation of the linear system matrix to reduce fill-in
on subsequent $LU$ factorizations (using COLAMD, minimal degree
ordering on $A^T*A$, minimal degree ordering on $A^T+A$, or natural
ordering).  Of these ordering choices, the default value in the
{\sunlinsolslumt} module is the COLAMD ordering. 

Since the linear systems that arise within the context of {\sundials}
calculations will typically have identical sparsity patterns, the
{\sunlinsolslumt} module is constructed to perform the
following operations:
\begin{itemize}
\item The first time that the ``setup'' routine is called, it
  performs the symbolic factorization, followed by an initial
  numerical factorization.  
\item On subsequent calls to the ``setup'' routine, it skips the
  symbolic factorization, and only refactors the input matrix.
\item The ``solve'' call performs pivoting and forward and
  backward substitution using the stored {\superlumt} data
  structures.  We note that in this solve {\superlumt} operates on the
  native data arrays for the right-hand side and solution vectors,
  without requiring costly data copies.
\end{itemize}


% ====================================================================
\subsection{SUNLinearSolver\_SuperLUMT functions}
\label{ss:sunlinsol_slumt_functions}
% ====================================================================

The module {\sunlinsolslumt} provides the following user-callable constructor
for creating a \newline \id{SUNLinearSolver} object.
%
% --------------------------------------------------------------------
%
\ucfunctiond{SUNLinSol\_SuperLUMT}
{
  LS = SUNLinSol\_SuperLUMT(y, A, num\_threads);
}
{
  The function \ID{SUNLinSol\_SuperLUMT} creates and allocates memory for
  a SuperLU\_MT-based \id{SUNLinearSolver} object.
}
{
  \begin{args}[num\_threads]
  \item[y] (\id{N\_Vector})
    a template for cloning vectors needed within the solver
  \item[A] (\id{SUNMatrix})
    a {\sunmatsparse} matrix template for cloning matrices needed
    within the solver 
  \item[num\_threads] (\id{int})
    desired number of threads (OpenMP or Pthreads, depending on how
    {\superlumt} was installed) to use during the factorization steps
  \end{args}
}
{
  This returns a \id{SUNLinearSolver} object.  If either \id{A} or
  \id{y} are incompatible then this routine will return \id{NULL}.
}
{
  This routine analyzes the input matrix and vector to determine the
  linear system size and to assess compatibility with the {\superlumt}
  library.

  This routine will perform consistency checks to ensure that it is
  called with consistent {\nvector} and {\sunmatrix} implementations.
  These are currently limited to the {\sunmatsparse} matrix type
  (using either CSR or CSC storage formats) and the {\nvecs},
  {\nvecopenmp}, and {\nvecpthreads} vector types.  As additional
  compatible matrix and vector implementations are added to
  {\sundials}, these will be included within this compatibility
  check.

  The \id{num\_threads} argument is not checked and is passed directly
  to {\superlumt} routines.
}
{SUNSuperLUMT}
%
% --------------------------------------------------------------------
%
\noindent The {\sunlinsolslumt} module defines implementations of all
``direct'' linear solver operations listed in Sections
\ref{ss:sunlinsol_CoreFn} -- \ref{ss:sunlinsol_GetFn}:
\begin{itemize}
\item \id{SUNLinSolGetType\_SuperLUMT}
\item \id{SUNLinSolInitialize\_SuperLUMT} -- this sets the
  \id{first\_factorize} flag to 1 and resets the internal {\superlumt}
  statistics variables.
\item \id{SUNLinSolSetup\_SuperLUMT} -- this performs either a $LU$
  factorization or refactorization of the input matrix.
\item \id{SUNLinSolSolve\_SuperLUMT} -- this calls the appropriate
  {\superlumt} solve routine to utilize the $LU$ factors to solve the
  linear system. 
\item \id{SUNLinSolLastFlag\_SuperLUMT}
\item \id{SUNLinSolSpace\_SuperLUMT} -- this only returns information for
  the storage within the solver \emph{interface}, i.e.~storage for the
  integers \id{last\_flag} and \id{first\_factorize}.  For additional
  space requirements, see the {\superlumt} documentation.
\item \id{SUNLinSolFree\_SuperLUMT}
\end{itemize}

The {\sunlinsolslumt} module also defines the following additional
user-callable function.
%
% --------------------------------------------------------------------
%
\ucfunctiond{SUNLinSol\_SuperLUMTSetOrdering}
{
  retval = SUNLinSol\_SuperLUMTSetOrdering(LS, ordering);
}
{
  This function sets the ordering used by {\superlumt} for reducing fill in
  the linear solve.
}
{
  \begin{args}[ordering]
  \item[LS] (\id{SUNLinearSolver})
    the {\sunlinsolslumt} object
  \item[ordering] (\id{int})
    a flag indicating the ordering algorithm to use, the options are:
    \begin{itemize}
    \item[0] natural ordering
    \item[1] minimal degree ordering on $A^TA$
    \item[2] minimal degree ordering on $A^T+A$
    \item[3] COLAMD ordering for unsymmetric matrices
    \end{itemize}
    The default is 3 for COLAMD.
  \end{args}
}
{
  The return values from this function are \id{SUNLS\_MEM\_NULL}
  (\id{S} is \id{NULL}), \newline \id{SUNLS\_ILL\_INPUT}
  (invalid ordering choice), or \id{SUNLS\_SUCCESS}.
}
{}
{SUNSuperLUMTSetOrdering}

% ====================================================================
\subsection{SUNLinearSolver\_SuperLUMT Fortran interfaces}
\label{ss:sunlinsol_slumt_fortran}
% ====================================================================

For solvers that include a Fortran interface module, the
{\sunlinsolslumt} module also includes a Fortran-callable function
for creating a \id{SUNLinearSolver} object.
%
% --------------------------------------------------------------------
%
\ucfunction{FSUNSUPERLUMTINIT}
{
  FSUNSUPERLUMTINIT(code, num\_threads, ier)
}
{
  The function \ID{FSUNSUPERLUMTINIT} can be called for Fortran programs
  to create a {\sunlinsolklu} object.
}
{
  \begin{args}[num\_threads]
  \item[code] (\id{int*})
    is an integer input specifying the solver id (1 for {\cvode}, 2
    for {\ida}, 3 for {\kinsol}, and 4 for {\arkode}).
  \item[num\_threads] (\id{int*})
    desired number of threads (OpenMP or Pthreads, depending on how
    {\superlumt} was installed) to use during the factorization steps
  \end{args}
}
{
  \id{ier} is a return completion flag equal to \id{0} for a success
  return and \id{-1} otherwise. See printed message for details in case
  of failure.
}
{
  This routine must be
  called \emph{after} both the {\nvector} and {\sunmatrix} objects have
  been initialized.
}
Additionally, when using {\arkode} with a non-identity
mass matrix, the {\sunlinsolslumt} module includes a Fortran-callable
function for creating a \id{SUNLinearSolver} mass matrix solver
object.
%
% --------------------------------------------------------------------
%
\ucfunction{FSUNMASSSUPERLUMTINIT}
{
  FSUNMASSSUPERLUMTINIT(num\_threads, ier)
}
{
  The function \ID{FSUNMASSSUPERLUMTINIT} can be called for Fortran programs
  to create a SuperLU\_MT-based \id{SUNLinearSolver} object for mass matrix linear
  systems.
}
{
  \begin{args}[num\_threads]
  \item[num\_threads] (\id{int*})
    desired number of threads (OpenMP or Pthreads, depending on how
    {\superlumt} was installed) to use during the factorization steps.
  \end{args}
}
{
  \id{ier} is a \id{int} return completion flag equal to \id{0} for a success
  return and \id{-1} otherwise. See printed message for details in case
  of failure.
}
{
  This routine must be
  called \emph{after} both the {\nvector} and {\sunmatrix} mass-matrix
  objects have been initialized.
}
The \ID{SUNLinSol\_SuperLUMTSetOrdering} routine also supports Fortran
interfaces for the system and mass matrix solvers:
%
% --------------------------------------------------------------------
%
\ucfunction{FSUNSUPERLUMTSETORDERING}
{
  FSUNSUPERLUMTSETORDERING(code, ordering, ier)
}
{
  The function \ID{FSUNSUPERLUMTSETORDERING} can be called for Fortran programs
  to update the ordering algorithm in a {\sunlinsolslumt} object.
}
{
  \begin{args}[ordering]
  \item[code] (\id{int*})
    is an integer input specifying the solver id (1 for {\cvode}, 2
    for {\ida}, 3 for {\kinsol}, and 4 for {\arkode}).
  \item[ordering] (\id{int*})
    a flag indicating the ordering algorithm, options are:
    \begin{itemize}
    \item[0] natural ordering
    \item[1] minimal degree ordering on $A^TA$
    \item[2] minimal degree ordering on $A^T+A$
    \item[3] COLAMD ordering for unsymmetric matrices
    \end{itemize}
    The default is 3 for COLAMD.
  \end{args}
}
{
  \id{ier} is a \id{int} return completion flag equal to \id{0} for a success
  return and \id{-1} otherwise. See printed message for details in case
  of failure.
}
{
  See \id{SUNLinSol\_SuperLUMTSetOrdering} for complete further
  documentation of this routine. 
}
%
% --------------------------------------------------------------------
%
\ucfunction{FSUNMASSUPERLUMTSETORDERING}
{
  FSUNMASSUPERLUMTSETORDERING(ordering, ier)
}
{
  The function \ID{FSUNMASSUPERLUMTSETORDERING} can be called for Fortran
  programs to update the ordering algorithm in a {\sunlinsolslumt}
  object for mass matrix linear systems.
}
{
  \begin{args}[ordering]
  \item[ordering] (\id{int*})
    a flag indicating the ordering algorithm, options are:
    \begin{itemize}
    \item[0] natural ordering
    \item[1] minimal degree ordering on $A^TA$
    \item[2] minimal degree ordering on $A^T+A$
    \item[3] COLAMD ordering for unsymmetric matrices
    \end{itemize}
    The default is 3 for COLAMD.
  \end{args}
}
{
  \id{ier} is a \id{int} return completion flag equal to \id{0} for a success
  return and \id{-1} otherwise. See printed message for details in case
  of failure.
}
{
  See \id{SUNLinSol\_SuperLUMTSetOrdering} for complete further
  documentation of this routine. 
}


% ====================================================================
\subsection{SUNLinearSolver\_SuperLUMT content}
\label{ss:sunlinsol_slumt_content}
% ====================================================================

The {\sunlinsolslumt} module defines the \textit{content} field of a
\id{SUNLinearSolver} as the following structure:
%%
\begin{verbatim} 
struct _SUNLinearSolverContent_SuperLUMT {
  long int     last_flag;
  int          first_factorize;
  SuperMatrix  *A, *AC, *L, *U, *B;
  Gstat_t      *Gstat;
  sunindextype *perm_r, *perm_c;
  sunindextype N;
  int          num_threads;
  realtype     diag_pivot_thresh; 
  int          ordering;
  superlumt_options_t *options;
};
\end{verbatim}
%%
These entries of the \emph{content} field contain the following
information:
\begin{args}[diag\_pivot\_thresh]
  \item[last\_flag] - last error return flag from internal function evaluations,
  \item[first\_factorize] - flag indicating whether the factorization
    has ever been performed, 
  \item[A, AC, L, U, B] - \id{SuperMatrix} pointers used in solve,
  \item[Gstat] - \id{GStat\_t} object used in solve,
  \item[perm\_r, perm\_c] - permutation arrays used in solve,
  \item[N] - size of the linear system,
  \item[num\_threads] - number of OpenMP/Pthreads threads to use,
  \item[diag\_pivot\_thresh] - threshold on diagonal pivoting,
  \item[ordering] - flag for which reordering algorithm to use,
  \item[options] - pointer to {\superlumt} options structure.
\end{args}


%% This is a shared SUNDIALS TEX file with a description of the
%% cusolversp batch QR sunlinsol implementation
%%
\section{The SUNLinearSolver\_cuSolverSp\_batchQR implementation}\label{ss:sunlinsol_cuspbqr}

The \id{SUNLinearSolver\_cuSolverSp\_batchQR} implementation of the {\sunlinsol} API is
designed to be used with the \id{SUNMATRIX\_CUSPARSE} matrix, and the {\nveccuda} vector.
The header file to include when using this module is \id{sunlinsol/sunlinsol\_cusolversp\_batchqr.h}.
The installed library to link to is \id{libsundials\_sunlinsolcusolversp.\textit{lib}}
where \id{\em.lib} is typically \id{.so} for shared libraries and \id{.a} for static libraries.
\newline
\newline
{\warn}The \id{SUNLinearSolver\_cuSolverSp\_batchQR} module is experimental and subject to change.

% --------------------------------------------------------------------
\subsection{SUNLinearSolver\_cuSolverSp\_batchQR description}\label{ss:sunlinsol_cuspbqr_description}

The \id{SUNLinearSolver\_cuSolverSp\_batchQR} implementation provides an interface to
the batched sparse QR factorization method provided by the NVIDIA cuSOLVER library
\cite{cuSOLVER_site}. The module is designed for solving block diagonal linear systems
of the form
\begin{equation*}
  \begin{bmatrix}
    \mathbf{A_1} & 0 & \cdots & 0\\
    0 & \mathbf{A_2} & \cdots & 0\\
    \vdots & \vdots & \ddots & \vdots\\
    0 & 0 & \cdots & \mathbf{A_n}\\
  \end{bmatrix}
  x_j
  =
  b_j
\end{equation*}
where all block matrices $\mathbf{A_j}$ share the same sparsisty pattern. The matrix
must be the \id{SUNMATRIX\_CUSPAESE} module.


% % --------------------------------------------------------------------
\subsection{SUNLinearSolver\_cuSolverSp\_batchQR functions}\label{ss:sunlinsol_cuspbqr_functions}

The \id{SUNLinearSolver\_cuSolverSp\_batchQR} module defines implementations of
all ``direct'' linear solver operations listed in
Sections \ref{ss:sunlinsol_CoreFn}-\ref{ss:sunlinsol_GetFn}:
%%
%%
\begin{itemize}
\item \id{SUNLinSolGetType\_cuSolverSp\_batchQR}
\item \id{SUNLinSolInitialize\_cuSolverSp\_batchQR} -- this sets the
  \id{first\_factorize} flag to 1
\item \id{SUNLinSolSetup\_cuSolverSp\_batchQR} -- this always copies the
  relevant {\sunmatsparse} data to the GPU; if this is the first setup
  it will perform symbolic analysis on the system
\item \id{SUNLinSolSolve\_cuSolverSp\_batchQR} -- this calls the
  \id{cusolverSpXcsrqrsvBatched} routine to perform factorization
\item \id{SUNLinSolLastFlag\_cuSolverSp\_batchQR}
\item \id{SUNLinSolFree\_cuSolverSp\_batchQR}
\end{itemize}
%%
%%
In addition, the module provides the following user-callable routines:
%%
%%
\ucfunction{SUNLinSol\_cuSolverSp\_batchQR}
{
  LS = SUNLinSol\_cuSolverSp\_batchQR(y, A, cusol);
}
{
  The function \ID{SUNLinSol\_cuSolverSp\_batchQR} creates and allocates memory for a
  {\sunlinsol} object.
}
{
  \begin{args}[subsys\_size]
  \item[y] (\id{N\_Vector})
    a {\nveccuda} vector for checking compatibility with the solver
  \item[A] (\id{SUNMatrix})
    a {\sunmatsparse} matrix for checking compatibility with the solver
  \item[cusol] (\id{cusolverHandle\_t}) a valid cuSOLVER handle
  \end{args}
}
{
  This returns a \id{SUNLinearSolver} object.  If either \id{A} or
  \id{y} are incompatible then this routine will return \id{NULL}.
}
{
  This routine analyzes the input matrix and vector to determine the
  linear system size and to assess compatibility with the solver.

  This routine will perform consistency checks to ensure that it is
  called with consistent {\nvector} and {\sunmatrix} implementations.
  These are currently limited to the \id{SUNMAT\_CUSPARSE} matrix type
  and the {\nveccuda} vector type. As additional compatible matrix and
  vector implementations are added to {\sundials}, these will be included
  within this compatibility check.
}
%
%
\ucfunction{SUNLinSol\_cuSolverSp\_batchQR\_GetDescription}
{
  SUNLinSol\_cuSolverSp\_batchQR\_GetDescription(LS, \&desc);
}
{
  The function \ID{SUNLinSol\_cuSolverSp\_batchQR\_GetDescription}
  accesses the string description of the object (empty by default).
}
{
  \begin{args}[options]
  \item[LS] (\id{SUNLinearSolver})
    a \id{SUNLinSol\_cuSolverSp\_batchQR} object
  \item[desc] (\id{char **})
    the string description of the linear solver
  \end{args}
}
{ None }
{}
%%
%%
\ucfunction{SUNLinSol\_cuSolverSp\_batchQR\_SetDescription}
{
  SUNLinSol\_cuSolverSp\_batchQR\_SetDescription(LS, desc);
}
{
  The function \ID{SUNLinSol\_cuSolverSp\_batchQR\_SetDescription}
  sets the string description of the object (empty by default).
}
{
  \begin{args}[options]
  \item[LS] (\id{SUNLinearSolver})
    a \id{SUNLinSol\_cuSolverSp\_batchQR} object
  \item[desc] (\id{const char *})
    the string description of the linear solver
  \end{args}
}
{ None }
{}
%%
%%
\ucfunction{SUNLinSol\_cuSolverSp\_batchQR\_GetDeviceSpace}
{
  SUNLinSol\_cuSolverSp\_batchQR\_GetDeviceSpace(LS, cuSolverInternal, cuSolverWorkspace);
}
{
  The function \id{SUNLinSol\_cuSolverSp\_batchQR\_GetDeviceSpace}
  returns the cuSOLVER batch QR method internal buffer size, in bytes,
  in the argument \id{cuSolverInternal} and the cuSOLVER
  batch QR workspace buffer size, in bytes, in the agrument
  \id{cuSolverWorkspace}. The size of the internal buffer is
  proportional to the number of matrix blocks while the size
  of the workspace is almost independent of the number of blocks.
}
{
  \begin{args}[options]
  \item[LS] (\id{SUNLinearSolver})
    a \id{SUNLinSol\_cuSolverSp\_batchQR} object
  \item[cuSolverInternal] (\id{size\_t *})
    output -- the size of the cuSOLVER internal buffer in bytes
  \item[cuSolverWorkspace] (\id{size\_t *})
    output -- the size of the cuSOLVER workspace buffer in bytes
  \end{args}
}
{ None }
{}

% --------------------------------------------------------------------
\subsection{SUNLinearSolver\_cuSolverSp\_batchQR content}\label{ss:sunlinsol_cuspbqr_content}

The \id{SUNLinearSolver\_cuSolverSp\_batchQR} module defines the
{\em content} field of a \id{SUNLinearSolver} to be the following structure:
%%
\begin{verbatim}
struct _SUNLinearSolverContent_cuSolverSp_batchQR {
  int                last_flag;          /* last return flag                                     */
  booleantype        first_factorize;    /* is this the first factorization?                     */
  size_t             internal_size;      /* size of cusolver internal buffer for Q and R         */
  size_t             workspace_size;     /* size of cusolver memory block for num. factorization */
  cusolverSpHandle_t cusolver_handle;    /* cuSolverSp context                                   */
  csrqrInfo_t        info;               /* opaque cusolver data structure                       */
  void*              workspace;          /* memory block used by cusolver                        */
  const char*        desc;               /* description of this linear solver                    */
};
\end{verbatim}

%% This is a shared SUNDIALS TEX file with a description of the
%% spgmr sunlinsol implementation
%%

The {\spgmr} (Scaled, Preconditioned, Generalized Minimum
Residual \cite{SaSc:86}) implementation of the {\sunlinsol} module
provided with {\sundials}, {\sunlinsolspgmr}, is an iterative linear
solver that is designed to be compatible with any {\nvector}
implementation (serial, threaded, parallel, user-supplied) that
supports a minimal subset of operations (\id{N\_VClone}, 
\id{N\_VDotProd}, \id{N\_VScale}, \id{N\_VLinearSum}, \id{N\_VProd},
\id{N\_VConst}, \id{N\_VDiv} and \id{N\_VDestroy}).  

The {\sunlinsolspgmr} module defines the {\em content} field of a
\id{SUNLinearSolver} to be the following structure:
%%
\begin{verbatim} 
struct _SUNLinearSolverContent_SPGMR {
  int maxl;
  int pretype;
  int gstype;
  int max_restarts;
  int numiters;
  realtype resnorm;
  long int last_flag;
  ATimesFn ATimes;
  void* ATData;
  PSetupFn Psetup;
  PSolveFn Psolve;
  void* PData;
  N_Vector s1;
  N_Vector s2;
  N_Vector *V;
  realtype **Hes;
  realtype *givens;
  N_Vector xcor;
  realtype *yg;
  N_Vector vtemp;
};
\end{verbatim}
%%
These entries of the \emph{content} field contain the following
information:
\begin{description}
  \item[maxl] - number of GMRES basis vectors to use (default is 5)
  \item[pretype] - flag for type of preconditioning to employ
    (default is none)
  \item[gstype] - flag for type of Gram-Schmidt orthogonalization
    (default is modified Gram-Schmidt)
  \item[max\_restarts] - number of GMRES restarts to allow
    (default is 0) 
  \item[numiters] - number of iterations from most-recent solve
  \item[resnorm] - final linear residual norm from most-recent solve
  \item[last\_flag] - last error return flag from internal function
  \item[ATimes] - function pointer to perform $Av$ product
  \item[ATData] - pointer to structure for \id{ATimes}
  \item[Psetup] - function pointer to preconditioner setup routine
  \item[Psolve] - function pointer to preconditioner solve routine
  \item[PData] - pointer to structure for \id{Psetup}, \id{Psolve}
  \item[s1, s2] - vector pointers for supplied scaling matrices
    (default are \id{NULL})
  \item[V] - the array of Krylov basis vectors
    $v_1, \ldots, v_{\text{\id{maxl}}+1}$, stored in \id{V[0]},
    \ldots, \id{V[maxl]}. Each $v_i$ is a vector of type {\nvector}.
  \item[Hes] - the $(\text{\id{maxl}}+1)\times\text{\id{maxl}}$
    Hessenberg matrix. It is stored row-wise so that the (i,j)th
    element is given by \id{Hes[i][j]}. 
  \item[givens] - a length \id{2*maxl} array which represents the
    Givens rotation matrices that arise in the GMRES algorithm. These
    matrices are $F_0, F_1, \ldots, F_j$, where
    $F_i = \begin{bmatrix}
      1 &        &   &     &      &   &        &   \\
        & \ddots &   &     &      &   &        &   \\
        &        & 1 &     &      &   &        &   \\
        &        &   & c_i & -s_i &   &        &   \\
        &        &   & s_i &  c_i &   &        &   \\
        &        &   &     &      & 1 &        &   \\
        &        &   &     &      &   & \ddots &   \\
        &        &   &     &      &   &        & 1\end{bmatrix}$
    are represented in the \id{givens} vector as \id{givens[0] =}
    $c_0$, \id{givens[1] = } $s_0$, \id{givens[2] = } $c_1$,
    \id{givens[3] = } $s_1$, \ldots \id{givens[2j] = } $c_j$,
    \id{givens[2j+1] = } $s_j$.
  \item[xcor] - a vector which holds the scaled, preconditioned
    correction to the initial guess 
  \item[yg] - a length \id{(maxl+1)} array of \id{realtype} values
    used to hold ``short'' vectors (e.g. $y$ and $g$).
  \item[vtemp] - temporary vector storage
\end{description}

This solver is constructed to perform the following operations:
\begin{itemize}
\item During construction, the \id{xcor} and \id{vtemp} arrays are
  cloned from a template {\nvector} that is input, and default solver
  parameters are set.
\item User-facing ``set'' routines may be called to modify default
  solver parameters.
\item Additional ``set'' routines are called by the {\sundials} solver
  that interfaces with {\sunlinsolspgmr} to supply the 
  \id{ATimes}, \id{PSetup} and \id{Psolve} function pointers and
  \id{s1} and \id{s2} scaling vectors.
\item In the ``initialize'' call, the remaining solver data is
  allocated (\id{V}, \id{Hes}, \id{givens}, \id{yg} )
\item In the ``setup'' call, any non-\id{NULL} 
  \id{PSetup} function is called.  Typically, this is provided by
  the {\sundials} solver itself, that translates between the
  generic \id{PSetup} function and the
  solver-specific routine (solver-supplied or user-supplied).
\item In the ``solve'' call the GMRES iteration is performed.  This
  will include scaling, preconditioning and restarts if those options
  have been supplied.
\end{itemize}

\noindent The header file to be included when using this module 
is \id{sunlinsol/sunlinsol\_spgmr.h}. \\
%%
%%----------------------------------------------
%%
The {\sunlinsolspgmr} module defines implementations of all
``iterative'' linear solver operations listed in Table
\ref{t:sunlinsolops}:
\begin{itemize}
\item \id{SUNLinSolGetType\_SPGMR}
\item \id{SUNLinSolInitialize\_SPGMR}
\item \id{SUNLinSolSetATimes\_SPGMR}
\item \id{SUNLinSolSetPreconditioner\_SPGMR}
\item \id{SUNLinSolSetScalingVectors\_SPGMR}
\item \id{SUNLinSolSetup\_SPGMR}
\item \id{SUNLinSolSolve\_SPGMR}
\item \id{SUNLinSolNumIters\_SPGMR}
\item \id{SUNLinSolResNorm\_SPGMR}
\item \id{SUNLinSolResid\_SPGMR}
\item \id{SUNLinSolLastFlag\_SPGMR}
\item \id{SUNLinSolSpace\_SPGMR}
\item \id{SUNLinSolFree\_SPGMR}
\end{itemize}
The module {\sunlinsolspgmr} provides the following additional
user-callable routines: 
%%
\begin{itemize}

%%--------------------------------------

\item \ID{SUNSPGMR}

  This function creates and allocates memory for a {\spgmr}
  \id{SUNLinearSolver}.  Its arguments are an {\nvector}, the desired
  type of preconditioning, and the number of Krylov basis vectors to use.

  This routine will perform consistency checks to ensure that it is
  called with a consistent {\nvector} implementation (i.e.~that it
  supplies the requisite vector operations).  If \id{y} is
  incompatible then this routine will return \id{NULL}.

  A \id{maxl} argument that is $\le0$ will result in the default
  value (5).

  Allowable inputs for \id{pretype} are \id{PREC\_NONE} (0),
  \id{PREC\_LEFT} (1), \id{PREC\_RIGHT} (2) and \id{PREC\_BOTH} (3);
  any other integer input will result in the default (no
  preconditioning).  We note that some {\sundials} solvers are
  designed to only work with right preconditioning ({\kinsol}, {\ida},
  {\idas}).  While it is possible to configure a {\sunlinsolspgmr}
  object to use \id{PREC\_LEFT} or \id{PREC\_BOTH} with these solvers,
  this use mode is not supported and may result in inferior
  performance.

  \verb|SUNLinearSolver SUNSPGMR(N_Vector y, int pretype, int maxl);|

%%--------------------------------------

\item \ID{SUNSPGMRSetPrecType}

  This function updates the type of preconditioning to use.  Supported
  values are \id{PREC\_NONE} (0), \id{PREC\_LEFT} (1),
  \id{PREC\_RIGHT} (2) and \id{PREC\_BOTH} (3).  

  This routine will return with one of the error codes
  \id{SUNLS\_ILL\_INPUT} (illegal \id{pretype}), \id{SUNLS\_MEM\_NULL}
  (\id{S} is \id{NULL}) or \id{SUNLS\_SUCCESS}.
  
  \verb|int SUNSPGMRSetPrecType(SUNLinearSolver S, int pretype);|

%%--------------------------------------

\item \ID{SUNSPGMRSetGSType}

  This function sets the type of Gram-Schmidt orthogonalization to
  use.  Supported values are \id{MODIFIED\_GS} (1) and
  \id{CLASSICAL\_GS} (2).  Any other integer input will result in a
  failure, returning error code \id{SUNLS\_ILL\_INPUT}.

  This routine will return with one of the error codes
  \id{SUNLS\_ILL\_INPUT} (illegal \id{gstype}), \id{SUNLS\_MEM\_NULL}
  (\id{S} is \id{NULL}) or \id{SUNLS\_SUCCESS}.
  
  \verb|int SUNSPGMRSetGSType(SUNLinearSolver S, int gstype);|


%%--------------------------------------

\item \ID{SUNSPGMRSetMaxRestarts}

  This function sets the number of GMRES restarts to 
  allow.  A negative input will result in the default of 0.

  This routine will return with one of the error codes
  \id{SUNLS\_MEM\_NULL} (\id{S} is \id{NULL}) or \id{SUNLS\_SUCCESS}.
  
  \verb|int SUNSPGMRSetMaxRestarts(SUNLinearSolver S, int maxrs);|

\end{itemize}
%%
%%------------------------------------
%%
For solvers that include a Fortran interface module, the
{\sunlinsolspgmr} module also includes the Fortran-callable
function \id{FSUNSPGMRInit(code, pretype, maxl, ier)} to initialize
this {\sunlinsolspgmr} module for a given {\sundials} solver.
Here \id{code} is an input solver id (1 for {\cvode}, 2 for {\ida}, 3
for {\kinsol}, 4 for {\arkode}); \id{pretype} and \id{maxl} are the
same as for the C function \ID{SUNSPGMR}; \id{ier} is an error return
flag equal 0 for success and -1 for failure.  All of these input
arguments should be declared so as to match C type \id{int}).  This
routine must be called \emph{after} the {\nvector} object has been
initialized.  Additionally, when using {\arkode} with non-identity
mass matrix, the Fortran-callable
function \id{FSUNMassSPGMRInit(pretype, maxl, ier)} initializes this 
{\sunlinsolspgmr} module for solving mass matrix linear systems.

The \id{SUNSPGMRSetPrecType}, \id{SUNSPGMRSetGSType} and
\id{SUNSPGMRSetMaxRestarts} routines also support Fortran interfaces
for the system and mass matrix solvers:
\begin{itemize}
\item \id{FSUNSPGMRSetGSType(code, gstype, ier)} -- all arguments
  should be commensurate with a C \id{int}
\item \id{FSUNMassSPGMRSetGSType(gstype, ier)}
\item \id{FSUNSPGMRSetPrecType(code, pretype, ier)} -- all arguments
  should be commensurate with a C \id{int}
\item \id{FSUNMassSPGMRSetPrecType(pretype, ier)}
\item \id{FSUNSPGMRSetMaxRS(code, maxrs, ier)} -- all arguments
  should be commensurate with a C \id{int}
\item \id{FSUNMassSPGMRSetMaxRS(maxrs, ier)}
\end{itemize}

%% This is a shared SUNDIALS TEX file with a description of the
%% spfgmr sunlinsol implementation
%%

The {\spfgmr} (Scaled, Preconditioned, Flexible, Generalized Minimum
Residual \cite{Saa:93}) implementation of the {\sunlinsol} module
provided with {\sundials}, {\sunlinsolspfgmr}, is an iterative linear
solver that is designed to be compatible with any {\nvector}
implementation (serial, threaded, parallel, and user-supplied) that
supports a minimal subset of operations (\id{N\_VClone},
\id{N\_VDotProd}, \id{N\_VScale}, \id{N\_VLinearSum}, \id{N\_VProd},
\id{N\_VConst}, \id{N\_VDiv}, and \id{N\_VDestroy}).  Unlike the other
Krylov iterative linear solvers supplied with {\sundials}, FGMRES is
specifically designed to work with a changing preconditioner
(e.g.~from an iterative method).

The {\sunlinsolspfgmr} module defines the {\em content} field of a
\id{SUNLinearSolver} to be the following structure:
%%
\begin{verbatim} 
struct _SUNLinearSolverContent_SPFGMR {
  int maxl;
  int pretype;
  int gstype;
  int max_restarts;
  int numiters;
  realtype resnorm;
  long int last_flag;
  ATimesFn ATimes;
  void* ATData;
  PSetupFn Psetup;
  PSolveFn Psolve;
  void* PData;
  N_Vector s1;
  N_Vector s2;
  N_Vector *V;
  N_Vector *Z;
  realtype **Hes;
  realtype *givens;
  N_Vector xcor;
  realtype *yg;
  N_Vector vtemp;
};
\end{verbatim}
%%
These entries of the \emph{content} field contain the following
information:
\begin{description}
  \item[maxl] - number of FGMRES basis vectors to use (default is 5),
  \item[pretype] - flag for use of preconditioning (default is none),
  \item[gstype] - flag for type of Gram-Schmidt orthogonalization
    (default is modified Gram-Schmidt),
  \item[max\_restarts] - number of FGMRES restarts to allow
    (default is 0),
  \item[numiters] - number of iterations from the most-recent solve,
  \item[resnorm] - final linear residual norm from the most-recent solve,
  \item[last\_flag] - last error return flag from an internal function,
  \item[ATimes] - function pointer to perform $Av$ product,
  \item[ATData] - pointer to structure for \id{ATimes},
  \item[Psetup] - function pointer to preconditioner setup routine,
  \item[Psolve] - function pointer to preconditioner solve routine,
  \item[PData] - pointer to structure for \id{Psetup} and \id{Psolve},
  \item[s1, s2] - vector pointers for supplied scaling matrices
    (default is \id{NULL}),
  \item[V] - the array of Krylov basis vectors
    $v_1, \ldots, v_{\text{\id{maxl}}+1}$, stored in \id{V[0]},
    \ldots, \id{V[maxl]}. Each $v_i$ is a vector of type {\nvector}.,
  \item[Z] - the array of preconditioned Krylov basis vectors
    $z_1, \ldots, z_{\text{\id{maxl}}+1}$, stored in \id{Z[0]},
    \ldots, \id{Z[maxl]}. Each $z_i$ is a vector of type {\nvector}.,
  \item[Hes] - the $(\text{\id{maxl}}+1)\times\text{\id{maxl}}$
    Hessenberg matrix. It is stored row-wise so that the (i,j)th
    element is given by \id{Hes[i][j]}.,
  \item[givens] - a length \id{2*maxl} array which represents the
    Givens rotation matrices that arise in the FGMRES algorithm. These
    matrices are $F_0, F_1, \ldots, F_j$, where
    $F_i = \begin{bmatrix}
      1 &        &   &     &      &   &        &   \\
        & \ddots &   &     &      &   &        &   \\
        &        & 1 &     &      &   &        &   \\
        &        &   & c_i & -s_i &   &        &   \\
        &        &   & s_i &  c_i &   &        &   \\
        &        &   &     &      & 1 &        &   \\
        &        &   &     &      &   & \ddots &   \\
        &        &   &     &      &   &        & 1\end{bmatrix}$,
    are represented in the \id{givens} vector as \id{givens[0] =}
    $c_0$, \id{givens[1] = } $s_0$, \id{givens[2] = } $c_1$,
    \id{givens[3] = } $s_1$, \ldots \id{givens[2j] = } $c_j$,
    \id{givens[2j+1] = } $s_j$.,
  \item[xcor] - a vector which holds the scaled, preconditioned
    correction to the initial guess,
  \item[yg] - a length \id{(maxl+1)} array of \id{realtype} values
    used to hold ``short'' vectors (e.g. $y$ and $g$),
  \item[vtemp] - temporary vector storage.
\end{description}

This solver is constructed to perform the following operations:
\begin{itemize}
\item During construction, the \id{xcor} and \id{vtemp} arrays are
  cloned from a template {\nvector} that is input, and default solver
  parameters are set.
\item User-facing ``set'' routines may be called to modify default
  solver parameters.
\item Additional ``set'' routines are called by the {\sundials} solver
  that interfaces with {\sunlinsolspfgmr} to supply the 
  \id{ATimes}, \id{PSetup}, and \id{Psolve} function pointers and
  \id{s1} and \id{s2} scaling vectors.
\item In the ``initialize'' call, the remaining solver data is
  allocated (\id{V}, \id{Hes}, \id{givens}, and \id{yg} )
\item In the ``setup'' call, any non-\id{NULL}
  \id{PSetup} function is called.  Typically, this is provided by
  the {\sundials} solver itself, that translates between the
  generic \id{PSetup} function and the
  solver-specific routine (solver-supplied or user-supplied).
\item In the ``solve'' call, the FGMRES iteration is performed.  This
  will include scaling, preconditioning, and restarts if those options
  have been supplied.
\end{itemize}

\noindent The header file to include when using this module 
is \id{sunlinsol/sunlinsol\_spfgmr.h}. The {\sunlinsolspfgmr} module
is accessible from all {\sundials} solvers \textit{without}
linking to the \\
\id{libsundials\_sunlinsolspfgmr} module library. \\

%%
%%----------------------------------------------
%%

\noindent The {\sunlinsolspfgmr} module defines implementations of all
``iterative'' linear solver operations listed in Table
\ref{t:sunlinsolops}:
\begin{itemize}
\item \id{SUNLinSolGetType\_SPFGMR}
\item \id{SUNLinSolInitialize\_SPFGMR}
\item \id{SUNLinSolSetATimes\_SPFGMR}
\item \id{SUNLinSolSetPreconditioner\_SPFGMR}
\item \id{SUNLinSolSetScalingVectors\_SPFGMR}
\item \id{SUNLinSolSetup\_SPFGMR}
\item \id{SUNLinSolSolve\_SPFGMR}
\item \id{SUNLinSolNumIters\_SPFGMR}
\item \id{SUNLinSolResNorm\_SPFGMR}
\item \id{SUNLinSolResid\_SPFGMR}
\item \id{SUNLinSolLastFlag\_SPFGMR}
\item \id{SUNLinSolSpace\_SPFGMR}
\item \id{SUNLinSolFree\_SPFGMR}
\end{itemize}
The module {\sunlinsolspfgmr} provides the following additional
user-callable routines: 
%%
\begin{itemize}

%%--------------------------------------

\item \ID{SUNSPFGMR}

  This constructor function creates and allocates memory for a {\spfgmr}
  \id{SUNLinearSolver}.  Its arguments are an {\nvector}, a flag
  indicating to use preconditioning, and the number of Krylov basis
  vectors to use. 

  This routine will perform consistency checks to ensure that it is
  called with a consistent {\nvector} implementation (i.e.~that it
  supplies the requisite vector operations).  If \id{y} is
  incompatible, then this routine will return \id{NULL}.

  A \id{maxl} argument that is $\le0$ will result in the default
  value (5).

  Since the FGMRES algorithm is designed to only support right
  preconditioning, then any of the \id{pretype}
  inputs \id{PREC\_LEFT} (1), \id{PREC\_RIGHT} (2), or \id{PREC\_BOTH}
  (3) will result in use of \id{PREC\_RIGHT};  any other integer input
  will result in the default (no preconditioning).  We note that some
  SUNDIALS solvers are designed to only work with left preconditioning
  ({\ida} and {\idas}). While it is possible to use a
  right-preconditioned {\sunlinsolspfgmr} object for these packages,
  this use mode is not supported and may result in inferior
  performance.

  \verb|SUNLinearSolver SUNSPFGMR(N_Vector y, int pretype, int maxl);|

%%--------------------------------------

\item \ID{SUNSPFGMRSetPrecType}

  This function updates the flag indicating use of preconditioning.
  Since the FGMRES algorithm is designed to only support right
  preconditioning, then any of the \id{pretype}
  inputs \id{PREC\_LEFT} (1), \id{PREC\_RIGHT} (2), or \id{PREC\_BOTH}
  (3) will result in use of \id{PREC\_RIGHT};  any other integer input
  will result in the default (no preconditioning).

  This routine will return with one of the error codes
  \id{SUNLS\_MEM\_NULL} (\id{S} is \id{NULL}) or \id{SUNLS\_SUCCESS}.
  
  \verb|int SUNSPFGMRSetPrecType(SUNLinearSolver S, int pretype);|

%%--------------------------------------

\item \ID{SUNSPFGMRSetGSType}

  This function sets the type of Gram-Schmidt orthogonalization to
  use.  Supported values are \id{MODIFIED\_GS} (1) and
  \id{CLASSICAL\_GS} (2).  Any other integer input will result in a
  failure, returning error code \id{SUNLS\_ILL\_INPUT}.

  This routine will return with one of the error codes
  \id{SUNLS\_ILL\_INPUT} (illegal \id{gstype}), \\ \noindent
  \id{SUNLS\_MEM\_NULL} (\id{S} is \id{NULL}), or \id{SUNLS\_SUCCESS}.
  
  \verb|int SUNSPFGMRSetGSType(SUNLinearSolver S, int gstype);|


%%--------------------------------------

\item \ID{SUNSPFGMRSetMaxRestarts}

  This function sets the number of FGMRES restarts to 
  allow.  A negative input will result in the default of 0.

  This routine will return with one of the error codes
  \id{SUNLS\_MEM\_NULL} (\id{S} is \id{NULL}) or \id{SUNLS\_SUCCESS}.
  
  \verb|int SUNSPFGMRSetMaxRestarts(SUNLinearSolver S, int maxrs);|

\end{itemize}
%%
%%------------------------------------
%%
For solvers that include a Fortran interface module, the
{\sunlinsolspfgmr} module also includes the Fortran-callable
function \id{FSUNSPFGMRInit(code, pretype, maxl, ier)} to initialize
this {\sunlinsolspfgmr} module for a given {\sundials} solver.
Here \id{code} is an integer input solver id (1 for {\cvode}, 2 for {\ida}, 3
for {\kinsol}, 4 for {\arkode}); \id{pretype} and \id{maxl} are the
same as for the C function \ID{SUNSPFGMR}; \id{ier} is an error return
flag equal to 0 for success and -1 for failure.  All of these input
arguments should be declared so as to match C type \id{int}.  This
routine must be called \emph{after} the {\nvector} object has been
initialized.  Additionally, when using {\arkode} with a non-identity
mass matrix, the Fortran-callable
function \id{FSUNMassSPFGMRInit(pretype, maxl, ier)} initializes this 
{\sunlinsolspfgmr} module for solving mass matrix linear systems.

The \id{SUNSPFGMRSetPrecType}, \id{SUNSPFGMRSetGSType}, and
\id{SUNSPFGMRSetMaxRestarts} routines also support Fortran interfaces
for the system and mass matrix solvers (all arguments should be
commensurate with a C \id{int}): 
\begin{itemize}
\item \id{FSUNSPFGMRSetGSType(code, gstype, ier)}
\item \id{FSUNMassSPFGMRSetGSType(gstype, ier)}
\item \id{FSUNSPFGMRSetPrecType(code, pretype, ier)}
\item \id{FSUNMassSPFGMRSetPrecType(pretype, ier)}
\item \id{FSUNSPFGMRSetMaxRS(code, maxrs, ier)}
\item \id{FSUNMassSPFGMRSetMaxRS(maxrs, ier)}
\end{itemize}

% ====================================================================
\section{The SUNLinearSolver\_SPBCGS implementation}
\label{ss:sunlinsol_spbcgs}
% ====================================================================

This section describes the {\sunlinsol} implementation of the {\spbcgs}
(Scaled, Preconditioned, Bi-Conjugate Gradient, Stabilized \cite{Van:92})
iterative linear solver. The {\sunlinsolspbcgs} module is designed to be
compatible with any {\nvector} implementation that supports a minimal subset
of operations (\id{N\_VClone}, \id{N\_VDotProd}, \id{N\_VScale},
\id{N\_VLinearSum}, \id{N\_VProd}, \id{N\_VDiv}, and
\id{N\_VDestroy}). Unlike the {\spgmr} and {\spfgmr} algorithms, {\spbcgs}
requires a fixed amount of memory that does not increase with the number of
allowed iterations.

To access the {\sunlinsolspbcgs} module, include the header file
\id{sunlinsol/sunlinsol\_spbcgs.h}. We note that the {\sunlinsolspbcgs} module is
accessible from {\sundials} packages \textit{without} separately linking to
the \id{libsundials\_sunlinsolspbcgs} module library.


% ====================================================================
\subsection{SUNLinearSolver\_SPBCGS description}
\label{ss:sunlinsol_spbcgs_description}
% ====================================================================

This solver is constructed to perform the following operations:
\begin{itemize}
\item During construction all {\nvector} solver data is allocated,
  with vectors cloned from a template {\nvector} that is input, and
  default solver parameters are set.
\item User-facing ``set'' routines may be called to modify default
  solver parameters.
\item Additional ``set'' routines are called by the {\sundials} solver
  that interfaces with {\sunlinsolspbcgs} to supply the
  \id{ATimes}, \id{PSetup}, and \id{Psolve} function pointers and
  \id{s1} and \id{s2} scaling vectors.
\item In the ``initialize'' call, the solver parameters are checked
  for validity.
\item In the ``setup'' call, any non-\id{NULL}
  \id{PSetup} function is called.  Typically, this is provided by
  the {\sundials} solver itself, that translates between the
  generic \id{PSetup} function and the
  solver-specific routine (solver-supplied or user-supplied).
\item In the ``solve'' call the {\spbcgs} iteration is performed.  This
  will include scaling and preconditioning if those options have been
  supplied.
\end{itemize}


% ====================================================================
\subsection{SUNLinearSolver\_SPBCGS functions}
\label{ss:sunlinsol_spbcgs_functions}
% ====================================================================

The {\sunlinsolspbcgs} module provides the following user-callable constructor
for creating a \newline \id{SUNLinearSolver} object.
%
% --------------------------------------------------------------------
%
\ucfunctiondf{SUNLinSol\_SPBCGS}
{
  LS = SUNLinSol\_SPBCGS(y, pretype, maxl);
}
{
  The function \ID{SUNLinSol\_SPBCGS} creates and allocates memory for
  a {\spbcgs} \newline \id{SUNLinearSolver} object.
}
{
  \begin{args}[pretype]
  \item[y] (\id{N\_Vector})
    a template for cloning vectors needed within the solver
  \item[pretype] (\id{int})
    flag indicating the desired type of preconditioning, allowed
    values are:
    \begin{itemize}
    \item \id{PREC\_NONE} (0)
    \item \id{PREC\_LEFT} (1)
    \item \id{PREC\_RIGHT} (2)
    \item \id{PREC\_BOTH} (3)
    \end{itemize}
    Any other integer input will result in the default (no
    preconditioning).
  \item[maxl] (\id{int})
    the number of linear iterations to allow. Values $\le0$ will
    result in the default value (5).
  \end{args}
}
{
  This returns a \id{SUNLinearSolver} object.  If either \id{y} is
  incompatible then this routine will return \id{NULL}.
}
{
  This routine will perform consistency checks to ensure that it is
  called with a consistent {\nvector} implementation (i.e.~that it
  supplies the requisite vector operations).  If \id{y} is
  incompatible, then this routine will return \id{NULL}.

  We note that some {\sundials} solvers are designed to only work
  with left preconditioning ({\ida} and {\idas}) and others with only
  right preconditioning ({\kinsol}). While it is possible to configure
  a {\sunlinsolspbcgs} object to use any of the preconditioning options
  with these solvers, this use mode is not supported and may result in
  inferior performance.
}
{SUNSPBCGS}
%
% --------------------------------------------------------------------
%
The {\sunlinsolspbcgs} module defines implementations of all
``iterative'' linear solver operations listed in Sections
\ref{ss:sunlinsol_CoreFn} -- \ref{ss:sunlinsol_GetFn}:
\begin{itemize}
\item \id{SUNLinSolGetType\_SPBCGS}
\item \id{SUNLinSolInitialize\_SPBCGS}
\item \id{SUNLinSolSetATimes\_SPBCGS}
\item \id{SUNLinSolSetPreconditioner\_SPBCGS}
\item \id{SUNLinSolSetScalingVectors\_SPBCGS}
\item \id{SUNLinSolSetup\_SPBCGS}
\item \id{SUNLinSolSolve\_SPBCGS}
\item \id{SUNLinSolNumIters\_SPBCGS}
\item \id{SUNLinSolResNorm\_SPBCGS}
\item \id{SUNLinSolResid\_SPBCGS}
\item \id{SUNLinSolLastFlag\_SPBCGS}
\item \id{SUNLinSolSpace\_SPBCGS}
\item \id{SUNLinSolFree\_SPBCGS}
\end{itemize}
All of the listed operations are callable via the {\F} 2003 interface module
by prepending an `F' to the function name.

The {\sunlinsolspbcgs} module also defines the following additional
user-callable functions.
%
% --------------------------------------------------------------------
%
\ucfunctiondf{SUNLinSol\_SPBCGSSetPrecType}
{
  retval = SUNLinSol\_SPBCGSSetPrecType(LS, pretype);
}
{
  The function \ID{SUNLinSol\_SPBCGSSetPrecType} updates the type of
  preconditioning to use in the {\sunlinsolspbcgs} object.
}
{
  \begin{args}[pretype]
  \item[LS] (\id{SUNLinearSolver})
    the {\sunlinsolspbcgs} object to update
  \item[pretype] (\id{int})
    flag indicating the desired type of preconditioning, allowed
    values match those discussed in \id{SUNLinSol\_SPBCGS}.
  \end{args}
}
{
  This routine will return with one of the error codes
  \id{SUNLS\_ILL\_INPUT} (illegal \id{pretype}), \id{SUNLS\_MEM\_NULL}
  (\id{S} is \id{NULL}) or \id{SUNLS\_SUCCESS}.
}
{}
{SUNSPBCGSSetPrecType}
%
% --------------------------------------------------------------------
%
\ucfunctiondf{SUNLinSol\_SPBCGSSetMaxl}
{
  retval = SUNLinSol\_SPBCGSSetMaxl(LS, maxl);
}
{
  The function \ID{SUNLinSol\_SPBCGSSetMaxl} updates the number of
  linear solver iterations to allow.
}
{
  \begin{args}[maxl]
  \item[LS] (\id{SUNLinearSolver})
    the {\sunlinsolspbcgs} object to update
  \item[maxl] (\id{int})
    flag indicating the number of iterations to allow. Values $\le0$
    will result in the default value (5).
  \end{args}
}
{
  This routine will return with one of the error codes
  \id{SUNLS\_MEM\_NULL} (\id{S} is \id{NULL}) or \id{SUNLS\_SUCCESS}.
}
{}
{SUNSPBCGSSetMaxl}
%
% --------------------------------------------------------------------
%
\ucfunctionf{SUNLinSolSetInfoFile\_SPBCGS}
{
  retval = SUNLinSolSetInfoFile\_SPBCGS(LS, info\_file);
}
{
  The function \ID{SUNLinSolSetInfoFile\_SPBCGS} sets the
  output file where all informative (non-error) messages should be directed.
}
{
  \begin{args}[info\_file]
    \item[LS] (\id{SUNLinearSolver})
      a {\sunnonlinsol} object
    \item[info\_file] (\id{FILE*}) pointer to output file (\id{stdout} by default);
      a \id{NULL} input will disable output
  \end{args}
}
{
  The return value is
  \begin{itemize}
    \item \id{SUNLS\_SUCCESS} if successful
    \item \id{SUNLS\_MEM\_NULL} if the SUNLinearSolver memory was \id{NULL}
    \item \id{SUNLS\_ILL\_INPUT} if {\sundials} was not built with monitoring enabled
  \end{itemize}
}
{
  This function is intended for users that wish to monitor the linear
  solver progress. By default, the file pointer is set to \id{stdout}.

  \textbf{{\sundials} must be built with the CMake option
  \id{SUNDIALS\_BUILD\_WITH\_MONITORING}, to utilize this function.}
  See section \ref{ss:configuration_options_nix} for more information.
}
%
% --------------------------------------------------------------------
%
\ucfunctionf{SUNLinSolSetPrintLevel\_SPBCGS}
{
  retval = SUNLinSolSetPrintLevel\_SPBCGS(NLS, print\_level);
}
{
  The function \ID{SUNLinSolSetPrintLevel\_SPBCGS} specifies the level
  of verbosity of the output.
}
{
  \begin{args}[print\_level]
  \item[LS] (\id{SUNLinearSolver})
    a {\sunnonlinsol} object
  \item[print\_level] (\id{int}) flag indicating level of verbosity;
    must be one of:
    \begin{itemize}
      \item 0, no information is printed (default)
      \item 1, for each linear iteration the residual norm is printed
    \end{itemize}
  \end{args}
}
{
  The return value is
  \begin{itemize}
    \item \id{SUNLS\_SUCCESS} if successful
    \item \id{SUNLS\_MEM\_NULL} if the SUNLinearSolver memory was \id{NULL}
    \item \id{SUNLS\_ILL\_INPUT} if {\sundials} was not built with monitoring enabled,
      or the print level value was invalid
  \end{itemize}
}
{
  This function is intended for users that wish to monitor the linear
  solver progress. By default, the print level is 0.

  \textbf{{\sundials} must be built with the CMake option
  \id{SUNDIALS\_BUILD\_WITH\_MONITORING}, to utilize this function.}
  See section \ref{ss:configuration_options_nix} for more information.
}


% ====================================================================
\subsection{SUNLinearSolver\_SPBCGS Fortran interfaces}
\label{ss:sunlinsol_spbcgs_fortran}
% ====================================================================

The {\sunlinsolspbcgs} module provides a {\F} 2003 module as well as {\F} 77
style interface functions for use from {\F} applications.

\subsubsection*{FORTRAN 2003 interface module}
The \ID{fsunlinsol\_spbcgs\_mod} {\F} module defines interfaces to all
{\sunlinsolspbcgs} {\CC} functions using the intrinsic \id{iso\_c\_binding}
module which provides a standardized mechanism for interoperating with {\CC}. As
noted in the {\CC} function descriptions above, the interface functions are
named after the corresponding {\CC} function, but with a leading `F'. For
example, the function \id{SUNLinSol\_SPBCGS} is interfaced as
\id{FSUNLinSol\_SPBCGS}.

The {\F} 2003 {\sunlinsolspbcgs} interface module can be accessed with the \id{use}
statement, i.e. \id{use fsunlinsol\_spbcgs\_mod}, and linking to the library
\id{libsundials\_fsunlinsolspbcgs\_mod}.{\em lib} in addition to the {\CC} library.
For details on where the library and module file \newline
\id{fsunlinsol\_spbcgs\_mod.mod} are installed see Appendix \ref{c:install}.
We note that the module is accessible from the {\F} 2003 {\sundials} integrators
\textit{without} separately linking to the \newline
\id{libsundials\_fsunlinsolspbcgs\_mod} library.

\subsubsection*{FORTRAN 77 interface functions}
For solvers that include a {\F} 77 interface module, the
{\sunlinsolspbcgs} module also includes a Fortran-callable function
for creating a \id{SUNLinearSolver} object.
%
% --------------------------------------------------------------------
%
\ucfunction{FSUNSPBCGSINIT}
{
  FSUNSPBCGSINIT(code, pretype, maxl, ier)
}
{
  The function \ID{FSUNSPBCGSINIT} can be called for Fortran programs
  to create a {\sunlinsolspbcgs} object.
}
{
  \begin{args}[pretype]
  \item[code] (\id{int*})
    is an integer input specifying the solver id (1 for {\cvode}, 2
    for {\ida}, 3 for {\kinsol}, and 4 for {\arkode}).
  \item[pretype] (\id{int*})
    flag indicating desired preconditioning type
  \item[maxl] (\id{int*})
    flag indicating number of iterations to allow
  \end{args}
}
{
  \id{ier} is a return completion flag equal to \id{0} for a success
  return and \id{-1} otherwise. See printed message for details in case
  of failure.
}
{
  This routine must be called \emph{after} the {\nvector} object has
  been initialized.

  Allowable values for \id{pretype} and \id{maxl} are the same as for
  the {\CC} function \newline \ID{SUNLinSol\_SPBCGS}.
}
Additionally, when using {\arkode} with a non-identity
mass matrix, the {\sunlinsolspbcgs} module includes a Fortran-callable
function for creating a \id{SUNLinearSolver} mass matrix solver
object.
%
% --------------------------------------------------------------------
%
\ucfunction{FSUNMASSSPBCGSINIT}
{
  FSUNMASSSPBCGSINIT(pretype, maxl, ier)
}
{
  The function \ID{FSUNMASSSPBCGSINIT} can be called for Fortran programs
  to create a {\sunlinsolspbcgs} object for mass matrix linear systems.
}
{
  \begin{args}[pretype]
  \item[pretype] (\id{int*})
    flag indicating desired preconditioning type
  \item[maxl] (\id{int*})
    flag indicating number of iterations to allow
  \end{args}
}
{
  \id{ier} is a \id{int} return completion flag equal to \id{0} for a success
  return and \id{-1} otherwise. See printed message for details in case
  of failure.
}
{
  This routine must be called \emph{after} the {\nvector} object has
  been initialized.

  Allowable values for \id{pretype} and \id{maxl} are the same as for
  the {\CC} function \newline \ID{SUNLinSol\_SPBCGS}.
}
%
% --------------------------------------------------------------------
%
The \id{SUNLinSol\_SPBCGSSetPrecType} and
\id{SUNLinSol\_SPBCGSSetMaxl} routines also support Fortran interfaces
for the system and mass matrix solvers.
%
% --------------------------------------------------------------------
%
\ucfunction{FSUNSPBCGSSETPRECTYPE}
{
  FSUNSPBCGSSETPRECTYPE(code, pretype, ier)
}
{
  The function \ID{FSUNSPBCGSSETPRECTYPE} can be called for Fortran
  programs to change the type of preconditioning to use.
}
{
  \begin{args}[pretype]
  \item[code] (\id{int*})
    is an integer input specifying the solver id (1 for {\cvode}, 2
    for {\ida}, 3 for {\kinsol}, and 4 for {\arkode}).
  \item[pretype] (\id{int*})
    flag indicating the type of preconditioning to use.
  \end{args}
}
{
  \id{ier} is a \id{int} return completion flag equal to \id{0} for a success
  return and \id{-1} otherwise. See printed message for details in case
  of failure.
}
{
  See \id{SUNLinSol\_SPBCGSSetPrecType} for complete further documentation of
  this routine.
}
%
% --------------------------------------------------------------------
%
\ucfunction{FSUNMASSSPBCGSSETPRECTYPE}
{
  FSUNMASSSPBCGSSETPRECTYPE(pretype, ier)
}
{
  The function \ID{FSUNMASSSPBCGSSETPRECTYPE} can be called for Fortran
  programs to change the type of preconditioning for mass matrix
  linear systems.
}
{
  The arguments are identical to \id{FSUNSPBCGSSETPRECTYPE} above, except that
  \id{code} is not needed since mass matrix linear systems only arise
  in {\arkode}.
}
{
  \id{ier} is a \id{int} return completion flag equal to \id{0} for a success
  return and \id{-1} otherwise. See printed message for details in case
  of failure.
}
{
  See \id{SUNLinSol\_SPBCGSSetPrecType} for complete further documentation of
  this routine.
}
%
% --------------------------------------------------------------------
%
\ucfunction{FSUNSPBCGSSETMAXL}
{
  FSUNSPBCGSSETMAXL(code, maxl, ier)
}
{
  The function \ID{FSUNSPBCGSSETMAXL} can be called for Fortran
  programs to change the maximum number of iterations to allow.
}
{
  \begin{args}[maxl]
  \item[code] (\id{int*})
    is an integer input specifying the solver id (1 for {\cvode}, 2
    for {\ida}, 3 for {\kinsol}, and 4 for {\arkode}).
  \item[maxl] (\id{int*})
    the number of iterations to allow.
  \end{args}
}
{
  \id{ier} is a \id{int} return completion flag equal to \id{0} for a success
  return and \id{-1} otherwise. See printed message for details in case
  of failure.
}
{
  See \id{SUNLinSol\_SPBCGSSetMaxl} for complete further
  documentation of this routine.
}
%
% --------------------------------------------------------------------
%
\ucfunction{FSUNMASSSPBCGSSETMAXL}
{
  FSUNMASSSPBCGSSETMAXL(maxl, ier)
}
{
  The function \ID{FSUNMASSSPBCGSSETMAXL} can be called for Fortran
  programs to change the type of preconditioning for mass matrix
  linear systems.
}
{
  The arguments are identical to \id{FSUNSPBCGSSETMAXL} above, except that
  \id{code} is not needed since mass matrix linear systems only arise
  in {\arkode}.
}
{
  \id{ier} is a \id{int} return completion flag equal to \id{0} for a success
  return and \id{-1} otherwise. See printed message for details in case
  of failure.
}
{
  See \id{SUNLinSol\_SPBCGSSetMaxl} for complete further
  documentation of this routine.
}
%
% --------------------------------------------------------------------
%

% ====================================================================
\subsection{SUNLinearSolver\_SPBCGS content}
\label{ss:sunlinsol_spbcgs_content}
% ====================================================================

The {\sunlinsolspbcgs} module defines the \textit{content} field of a
\id{SUNLinearSolver} as the following structure:
%%
\begin{verbatim}
struct _SUNLinearSolverContent_SPBCGS {
  int maxl;
  int pretype;
  int numiters;
  realtype resnorm;
  int last_flag;
  ATimesFn ATimes;
  void* ATData;
  PSetupFn Psetup;
  PSolveFn Psolve;
  void* PData;
  N_Vector s1;
  N_Vector s2;
  N_Vector r;
  N_Vector r_star;
  N_Vector p;
  N_Vector q;
  N_Vector u;
  N_Vector Ap;
  N_Vector vtemp;
  int      print_level;
  FILE*    info_file;
};
\end{verbatim}
%%
These entries of the \emph{content} field contain the following
information:
\begin{args}[print\_level]
  \item[maxl] - number of {\spbcgs} iterations to allow (default is 5),
  \item[pretype] - flag for type of preconditioning to employ
    (default is none),
  \item[numiters] - number of iterations from the most-recent solve,
  \item[resnorm] - final linear residual norm from the most-recent solve,
  \item[last\_flag] - last error return flag from an internal function,
  \item[ATimes] - function pointer to perform $Av$ product,
  \item[ATData] - pointer to structure for \id{ATimes},
  \item[Psetup] - function pointer to preconditioner setup routine,
  \item[Psolve] - function pointer to preconditioner solve routine,
  \item[PData] - pointer to structure for \id{Psetup} and \id{Psolve},
  \item[s1, s2] - vector pointers for supplied scaling matrices
    (default is \id{NULL}),
  \item[r] - a {\nvector} which holds the current scaled,
    preconditioned linear system residual,
  \item[r\_star] - a {\nvector} which holds the initial scaled,
    preconditioned linear system residual,
  \item[p, q, u, Ap, vtemp] - {\nvector}s used for workspace by the
    {\spbcgs} algorithm.
  \item[print\_level] - controls the amount of information to be printed to the info file
  \item[info\_file]   - the file where all informative (non-error) messages will be directed
\end{args}

% ====================================================================
\section{The SUNLinearSolver\_SPTFQMR implementation}
\label{ss:sunlinsol_sptfqmr}
% ====================================================================

This section describes the {\sunlinsol} implementation of the {\sptfqmr}
(Scaled, Preconditioned, \newline Transpose-Free Quasi-Minimum Residual \cite{Fre:93})
iterative linear solver. The {\sunlinsolsptfqmr} module is designed to be
compatible with any {\nvector} implementation that supports a minimal subset
of operations (\id{N\_VClone}, \id{N\_VDotProd}, \id{N\_VScale},
\id{N\_VLinearSum}, \id{N\_VProd}, \id{N\_VConst}, \id{N\_VDiv}, and
\id{N\_VDestroy}). Unlike the {\spgmr} and {\spfgmr} algorithms, {\sptfqmr}
requires a fixed amount of memory that does not increase with the number of
allowed iterations.

To access the {\sunlinsolsptfqmr} module, include the header file \newline
\id{sunlinsol/sunlinsol\_sptfqmr.h}. We note that the {\sunlinsolsptfqmr} module is
accessible from {\sundials} packages \textit{without} separately linking to
the \id{libsundials\_sunlinsolsptfqmr} module library.


% ====================================================================
\subsection{SUNLinearSolver\_SPTFQMR description}
\label{ss:sunlinsol_sptfqmr_description}
% ====================================================================

This solver is constructed to perform the following operations:
\begin{itemize}
\item During construction all {\nvector} solver data is allocated,
  with vectors cloned from a template {\nvector} that is input, and
  default solver parameters are set.
\item User-facing ``set'' routines may be called to modify default
  solver parameters.
\item Additional ``set'' routines are called by the {\sundials} solver
  that interfaces with \newline {\sunlinsolsptfqmr} to supply the
  \id{ATimes}, \id{PSetup}, and \id{Psolve} function pointers and
  \id{s1} and \id{s2} scaling vectors.
\item In the ``initialize'' call, the solver parameters are checked
  for validity.
\item In the ``setup'' call, any non-\id{NULL}
  \id{PSetup} function is called.  Typically, this is provided by
  the {\sundials} solver itself, that translates between the
  generic \id{PSetup} function and the
  solver-specific routine (solver-supplied or user-supplied).
\item In the ``solve'' call the TFQMR iteration is performed.  This
  will include scaling and preconditioning if those options have been
  supplied.
\end{itemize}


% ====================================================================
\subsection{SUNLinearSolver\_SPTFQMR functions}
\label{ss:sunlinsol_sptfqmr_functions}
% ====================================================================

The {\sunlinsolsptfqmr} module provides the following user-callable constructor
for creating a \newline \id{SUNLinearSolver} object.
%
% --------------------------------------------------------------------
%
\ucfunctiondf{SUNLinSol\_SPTFQMR}
{
  LS = SUNLinSol\_SPTFQMR(y, pretype, maxl);
}
{
  The function \ID{SUNLinSol\_SPTFQMR} creates and allocates memory for
  a {\sptfqmr} \newline \id{SUNLinearSolver} object.
}
{
  \begin{args}[pretype]
  \item[y] (\id{N\_Vector})
    a template for cloning vectors needed within the solver
  \item[pretype] (\id{int})
    flag indicating the desired type of preconditioning, allowed
    values are:
    \begin{itemize}
    \item \id{PREC\_NONE} (0)
    \item \id{PREC\_LEFT} (1)
    \item \id{PREC\_RIGHT} (2)
    \item \id{PREC\_BOTH} (3)
    \end{itemize}
    Any other integer input will result in the default (no
    preconditioning).
  \item[maxl] (\id{int})
    the number of linear iterations to allow. Values $\le0$ will
    result in the default value (5).
  \end{args}
}
{
  This returns a \id{SUNLinearSolver} object.  If either \id{y} is
  incompatible then this routine will return \id{NULL}.
}
{
  This routine will perform consistency checks to ensure that it is
  called with a consistent {\nvector} implementation (i.e.~that it
  supplies the requisite vector operations).  If \id{y} is
  incompatible, then this routine will return \id{NULL}.

  We note that some {\sundials} solvers are designed to only work
  with left preconditioning ({\ida} and {\idas}) and others with only
  right preconditioning ({\kinsol}). While it is possible to configure
  a {\sunlinsolsptfqmr} object to use any of the preconditioning options
  with these solvers, this use mode is not supported and may result in
  inferior performance.
}
{SUNSPTFQMR}
%
% --------------------------------------------------------------------
%
The {\sunlinsolsptfqmr} module defines implementations of all
``iterative'' linear solver operations listed in Sections
\ref{ss:sunlinsol_CoreFn} -- \ref{ss:sunlinsol_GetFn}:
\begin{itemize}
\item \id{SUNLinSolGetType\_SPTFQMR}
\item \id{SUNLinSolInitialize\_SPTFQMR}
\item \id{SUNLinSolSetATimes\_SPTFQMR}
\item \id{SUNLinSolSetPreconditioner\_SPTFQMR}
\item \id{SUNLinSolSetScalingVectors\_SPTFQMR}
\item \id{SUNLinSolSetup\_SPTFQMR}
\item \id{SUNLinSolSolve\_SPTFQMR}
\item \id{SUNLinSolNumIters\_SPTFQMR}
\item \id{SUNLinSolResNorm\_SPTFQMR}
\item \id{SUNLinSolResid\_SPTFQMR}
\item \id{SUNLinSolLastFlag\_SPTFQMR}
\item \id{SUNLinSolSpace\_SPTFQMR}
\item \id{SUNLinSolFree\_SPTFQMR}
\end{itemize}
All of the listed operations are callable via the {\F} 2003 interface module
by prepending an `F' to the function name.

The {\sunlinsolsptfqmr} module also defines the following additional
user-callable functions.
%
% --------------------------------------------------------------------
%
\ucfunctiondf{SUNLinSol\_SPTFQMRSetPrecType}
{
  retval = SUNLinSol\_SPTFQMRSetPrecType(LS, pretype);
}
{
  The function \ID{SUNLinSol\_SPTFQMRSetPrecType} updates the type of
  preconditioning to use in the {\sunlinsolsptfqmr} object.
}
{
  \begin{args}[pretype]
  \item[LS] (\id{SUNLinearSolver})
    the {\sunlinsolsptfqmr} object to update
  \item[pretype] (\id{int})
    flag indicating the desired type of preconditioning, allowed
    values match those discussed in \id{SUNLinSol\_SPTFQMR}.
  \end{args}
}
{
  This routine will return with one of the error codes
  \id{SUNLS\_ILL\_INPUT} (illegal \id{pretype}), \id{SUNLS\_MEM\_NULL}
  (\id{S} is \id{NULL}) or \id{SUNLS\_SUCCESS}.
}
{}
{SUNSPTFQMRSetPrecType}
%
% --------------------------------------------------------------------
%
\ucfunctionf{SUNLinSol\_SPTFQMRSetMaxl}
{
  retval = SUNLinSol\_SPTFQMRSetMaxl(LS, maxl);
}
{
  The function \ID{SUNLinSol\_SPTFQMRSetMaxl} updates the number of
  linear solver iterations to allow.
}
{
  \begin{args}[maxl]
  \item[LS] (\id{SUNLinearSolver})
    the {\sunlinsolsptfqmr} object to update
  \item[maxl] (\id{int})
    flag indicating the number of iterations to allow; values $\le0$
    will result in the default value (5)
  \end{args}
}
{
  This routine will return with one of the error codes
  \id{SUNLS\_MEM\_NULL} (\id{S} is \id{NULL}) or \id{SUNLS\_SUCCESS}.
}
{}
{SUNSPTFQMRSetMaxl}


% ====================================================================
\subsection{SUNLinearSolver\_SPTFQMR Fortran interfaces}
\label{ss:sunlinsol_sptfqmr_fortran}
% ====================================================================

The {\sunlinsolspfgmr} module provides a {\F} 2003 module as well as {\F} 77
style interface functions for use from {\F} applications.

\subsubsection*{FORTRAN 2003 interface module}
The \ID{fsunlinsol\_sptfqmr\_mod} {\F} module defines interfaces to all
{\sunlinsolspfgmr} {\CC} functions using the intrinsic \id{iso\_c\_binding}
module which provides a standardized mechanism for interoperating with {\CC}. As
noted in the {\CC} function descriptions above, the interface functions are
named after the corresponding {\CC} function, but with a leading `F'. For
example, the function \id{SUNLinSol\_SPTFQMR} is interfaced as
\id{FSUNLinSol\_SPTFQMR}.

The {\F} 2003 {\sunlinsolspfgmr} interface module can be accessed with the \id{use}
statement, i.e. \id{use fsunlinsol\_sptfqmr\_mod}, and linking to the library
\id{libsundials\_fsunlinsolsptfqmr\_mod}.{\em lib} in addition to the {\CC} library.
For details on where the library and module file \newline
\id{fsunlinsol\_sptfqmr\_mod.mod} are installed see Appendix \ref{c:install}.
We note that the module is accessible from the {\F} 2003 {\sundials} integrators
\textit{without} separately linking to the \newline
\id{libsundials\_fsunlinsolsptfqmr\_mod} library.

\subsubsection*{FORTRAN 77 interface functions}
For solvers that include a {\F} 77 interface module, the
{\sunlinsolsptfqmr} module also includes a Fortran-callable function
for creating a \id{SUNLinearSolver} object.
%
% --------------------------------------------------------------------
%
\ucfunction{FSUNSPTFQMRINIT}
{
  FSUNSPTFQMRINIT(code, pretype, maxl, ier)
}
{
  The function \ID{FSUNSPTFQMRINIT} can be called for Fortran programs
  to create a {\sunlinsolsptfqmr} object.
}
{
  \begin{args}[pretype]
  \item[code] (\id{int*})
    is an integer input specifying the solver id (1 for {\cvode}, 2
    for {\ida}, 3 for {\kinsol}, and 4 for {\arkode}).
  \item[pretype] (\id{int*})
    flag indicating desired preconditioning type
  \item[maxl] (\id{int*})
    flag indicating number of iterations to allow
  \end{args}
}
{
  \id{ier} is a return completion flag equal to \id{0} for a success
  return and \id{-1} otherwise. See printed message for details in case
  of failure.
}
{
  This routine must be called \emph{after} the {\nvector} object has
  been initialized.

  Allowable values for \id{pretype} and \id{maxl} are the same as for
  the {\CC} function \newline \ID{SUNLinSol\_SPTFQMR}.
}
Additionally, when using {\arkode} with a non-identity
mass matrix, the {\sunlinsolsptfqmr} module includes a Fortran-callable
function for creating a \id{SUNLinearSolver} mass matrix solver
object.
%
% --------------------------------------------------------------------
%
\ucfunction{FSUNMASSSPTFQMRINIT}
{
  FSUNMASSSPTFQMRINIT(pretype, maxl, ier)
}
{
  The function \ID{FSUNMASSSPTFQMRINIT} can be called for Fortran programs
  to create a {\sunlinsolsptfqmr} object for mass matrix linear systems.
}
{
  \begin{args}[pretype]
  \item[pretype] (\id{int*})
    flag indicating desired preconditioning type
  \item[maxl] (\id{int*})
    flag indicating number of iterations to allow
  \end{args}
}
{
  \id{ier} is a \id{int} return completion flag equal to \id{0} for a success
  return and \id{-1} otherwise. See printed message for details in case
  of failure.
}
{
  This routine must be called \emph{after} the {\nvector} object has
  been initialized.

  Allowable values for \id{pretype} and \id{maxl} are the same as for
  the {\CC} function \newline \ID{SUNLinSol\_SPTFQMR}.
}
%
% --------------------------------------------------------------------
%
The \id{SUNLinSol\_SPTFQMRSetPrecType} and
\id{SUNLinSol\_SPTFQMRSetMaxl} routines also support Fortran
interfaces for the system and mass matrix solvers.
%
% --------------------------------------------------------------------
%
\ucfunction{FSUNSPTFQMRSETPRECTYPE}
{
  FSUNSPTFQMRSETPRECTYPE(code, pretype, ier)
}
{
  The function \ID{FSUNSPTFQMRSETPRECTYPE} can be called for Fortran
  programs to change the type of preconditioning to use.
}
{
  \begin{args}[pretype]
  \item[code] (\id{int*})
    is an integer input specifying the solver id (1 for {\cvode}, 2
    for {\ida}, 3 for {\kinsol}, and 4 for {\arkode}).
  \item[pretype] (\id{int*})
    flag indicating the type of preconditioning to use.
  \end{args}
}
{
  \id{ier} is a \id{int} return completion flag equal to \id{0} for a success
  return and \id{-1} otherwise. See printed message for details in case
  of failure.
}
{
  See \id{SUNLinSol\_SPTFQMRSetPrecType} for complete further documentation of
  this routine.
}
%
% --------------------------------------------------------------------
%
\ucfunction{FSUNMASSSPTFQMRSETPRECTYPE}
{
  FSUNMASSSPTFQMRSETPRECTYPE(pretype, ier)
}
{
  The function \ID{FSUNMASSSPTFQMRSETPRECTYPE} can be called for Fortran
  programs to change the type of preconditioning for mass matrix
  linear systems.
}
{
  The arguments are identical to \id{FSUNSPTFQMRSETPRECTYPE} above,
  except that \id{code} is not needed since mass matrix linear systems
  only arise in {\arkode}.
}
{
  \id{ier} is a \id{int} return completion flag equal to \id{0} for a success
  return and \id{-1} otherwise. See printed message for details in case
  of failure.
}
{
  See \id{SUNLinSol\_SPTFQMRSetPrecType} for complete further documentation of
  this routine.
}
%
% --------------------------------------------------------------------
%
\ucfunction{FSUNSPTFQMRSETMAXL}
{
  FSUNSPTFQMRSETMAXL(code, maxl, ier)
}
{
  The function \ID{FSUNSPTFQMRSETMAXL} can be called for Fortran
  programs to change the maximum number of iterations to allow.
}
{
  \begin{args}[maxl]
  \item[code] (\id{int*})
    is an integer input specifying the solver id (1 for {\cvode}, 2
    for {\ida}, 3 for {\kinsol}, and 4 for {\arkode}).
  \item[maxl] (\id{int*})
    the number of iterations to allow.
  \end{args}
}
{
  \id{ier} is a \id{int} return completion flag equal to \id{0} for a success
  return and \id{-1} otherwise. See printed message for details in case
  of failure.
}
{
  See \id{SUNLinSol\_SPTFQMRSetMaxl} for complete further
  documentation of this routine.
}
%
% --------------------------------------------------------------------
%
\ucfunction{FSUNMASSSPTFQMRSETMAXL}
{
  FSUNMASSSPTFQMRSETMAXL(maxl, ier)
}
{
  The function \ID{FSUNMASSSPTFQMRSETMAXL} can be called for Fortran
  programs to change the type of preconditioning for mass matrix
  linear systems.
}
{
  The arguments are identical to \id{FSUNSPTFQMRSETMAXL} above, except that
  \id{code} is not needed since mass matrix linear systems only arise
  in {\arkode}.
}
{
  \id{ier} is a \id{int} return completion flag equal to \id{0} for a success
  return and \id{-1} otherwise. See printed message for details in case
  of failure.
}
{
  See \id{SUNLinSol\_SPTFQMRSetMaxl} for complete further
  documentation of this routine.
}
%
% --------------------------------------------------------------------
%

% ====================================================================
\subsection{SUNLinearSolver\_SPTFQMR content}
\label{ss:sunlinsol_sptfqmr_content}
% ====================================================================

The {\sunlinsolsptfqmr} module defines the \textit{content} field of a
\id{SUNLinearSolver} as the following structure:
%%
\begin{verbatim}
struct _SUNLinearSolverContent_SPTFQMR {
  int maxl;
  int pretype;
  int numiters;
  realtype resnorm;
  long int last_flag;
  ATimesFn ATimes;
  void* ATData;
  PSetupFn Psetup;
  PSolveFn Psolve;
  void* PData;
  N_Vector s1;
  N_Vector s2;
  N_Vector r_star;
  N_Vector q;
  N_Vector d;
  N_Vector v;
  N_Vector p;
  N_Vector *r;
  N_Vector u;
  N_Vector vtemp1;
  N_Vector vtemp2;
  N_Vector vtemp3;
};
\end{verbatim}
%%
These entries of the \emph{content} field contain the following
information:
\begin{args}[last\_flag]
  \item[maxl] - number of TFQMR iterations to allow (default is 5),
  \item[pretype] - flag for type of preconditioning to employ
    (default is none),
  \item[numiters] - number of iterations from the most-recent solve,
  \item[resnorm] - final linear residual norm from the most-recent solve,
  \item[last\_flag] - last error return flag from an internal function,
  \item[ATimes] - function pointer to perform $Av$ product,
  \item[ATData] - pointer to structure for \id{ATimes},
  \item[Psetup] - function pointer to preconditioner setup routine,
  \item[Psolve] - function pointer to preconditioner solve routine,
  \item[PData] - pointer to structure for \id{Psetup} and \id{Psolve},
  \item[s1, s2] - vector pointers for supplied scaling matrices
    (default is \id{NULL}),
  \item[r\_star] - a {\nvector} which holds the initial scaled,
    preconditioned linear system residual,
  \item[q, d, v, p, u] - {\nvector}s used for workspace by the SPTFQMR
    algorithm,
  \item [r] - array of two {\nvector}s used for workspace within the
    SPTFQMR algorithm,
  \item[vtemp1, vtemp2, vtemp3] - temporary vector storage.
\end{args}


%% This is a shared SUNDIALS TEX file with a description of the
%% pcg sunlinsol implementation
%%

The {\pcg} (Preconditioned Conjugate Gradient \cite{HeSt:52})
implementation of the {\sunlinsol} module provided with {\sundials},
{\sunlinsolpcg}, is an iterative linear solver that is designed to be
compatible with any {\nvector} implementation (serial, threaded,
parallel, user-supplied) that supports a minimal subset of operations
(\id{N\_VClone}, \id{N\_VDotProd}, \id{N\_VScale}, \id{N\_VLinearSum},
\id{N\_VProd} and \id{N\_VDestroy}).  Unlike the {\spgmr} and {\spfgmr}
algorithms, {\pcg} requires a fixed amount of memory that does not
scale with the number of allowed iterations.

Unlike all of the other iterative linear solvers supplied with
{\sundials}, {\pcg} should only be used on \emph{symmetric} linear
systems (e.g.~mass matrix linear systems encountered in
{\arkode}). As a result, the explanation of the role of scaling and
preconditioning matrices given in general must be modified in this
scenario.  The {\pcg} algorithm solves a linear system $Ax = b$ where  
$A$ is a symmetric ($A^T=A$), real-valued matrix.  Preconditioning is
allowed, and is applied in a symmetric fashion on both the right and
left.  Scaling is also allowed and is applied symmetrically.  We
denote the preconditioner and scaling matrices as follows:
\begin{itemize}
\item $P$ is the preconditioner (assumed symmetric),
\item $S$ is a diagonal matrix of scale factors.
\end{itemize}
The matrices $A$ and $P$ are not required explicitly; only routines
that provide $A$ and $P^{-1}$ as operators are required.  The diagonal
of the matrix $S$ is held in a single {\nvector}, supplied by the user
of this module.

In this notation, {\pcg} applies the underlying CG algorithm to the
equivalent transformed system 
\begin{equation}
  \label{eq:transformed_linear_systemPCG}
  \tilde{A} \tilde{x} = \tilde{b}
\end{equation}
where
\begin{align}
  \notag
  \tilde{A} &= S P^{-1} A P^{-1} S,\\
  \label{eq:transformed_linear_system_componentsPCG}
  \tilde{b} &= S P^{-1} b,\\
  \notag
  \tilde{x} &= S^{-1} P x.
\end{align} 
The scaling matrix must be chosen so that the vectors $SP^{-1}b$ and
$S^{-1}Px$ have dimensionless components.

The stopping test for the PCG iterations is on the L2 norm of the
scaled preconditioned residual:
\begin{align*}
  &\| \tilde{b} - \tilde{A} \tilde{x} \|_2  <  \delta\\
  \Leftrightarrow\quad &\\
  &\| S P^{-1} b - S P^{-1} A x \|_2  <  \delta\\
  \Leftrightarrow\quad &\\
  &\| P^{-1} b - P^{-1} A x \|_S  <  \delta
\end{align*}
where $\| v \|_S = \sqrt{v^T S^T S v}$, with an input tolerance $\delta$.

The {\sunlinsolpcg} module defines the {\em content} field of a
\id{SUNLinearSolver} to be the following structure:
%%
\begin{verbatim} 
struct _SUNLinearSolverContent_PCG {
  int maxl;
  int pretype;
  int numiters;
  realtype resnorm;
  long int last_flag;
  ATSetupFn ATSetup;
  ATimesFn ATimes;
  void* ATData;
  PSetupFn Psetup;
  PSolveFn Psolve;
  void* PData;
  N_Vector s;
  N_Vector r;
  N_Vector p;
  N_Vector z;
  N_Vector Ap;
};
\end{verbatim}
%%
These entries of the \emph{content} field contain the following
information:
\begin{description}
  \item[maxl] - number of {\pcg} iterations to allow (default is 5)
  \item[pretype] - flag for use of preconditioning (default is none)
  \item[numiters] - number of iterations from most-recent solve
  \item[resnorm] - final linear residual norm from most-recent solve
  \item[last\_flag] - last error return flag from internal function
  \item[ATSetup] - function pointer to setup routine for \id{ATimes} data
  \item[ATimes] - function pointer to perform $Av$ product
  \item[ATData] - pointer to structure for \id{ATSetup}, \id{ATimes}
  \item[Psetup] - function pointer to preconditioner setup routine
  \item[Psolve] - function pointer to preconditioner solve routine
  \item[PData] - pointer to structure for \id{Psetup}, \id{Psolve}
  \item[s] - vector pointer for supplied scaling matrix
    (default is \id{NULL})
  \item[r] - a {\nvector} which holds the preconditioned linear system
    residual
  \item[p, z, Ap] - {\nvector}s used for workspace by the
    {\pcg} algorithm. 
\end{description}

This solver is constructed to perform the following operations:
\begin{itemize}
\item During construction all {\nvector} solver data is allocated,
  with vectors cloned from a template {\nvector} that is input, and
  default solver parameters are set.
\item User-facing ``set'' routines may be called to modify default
  solver parameters.
\item Additional ``set'' routines are called by the {\sundials} solver
  that interfaces with {\sunlinsolpcg} to supply the \id{ATSetup},
  \id{ATimes}, \id{PSetup} and \id{Psolve} function pointers and
  \id{s} scaling vector.
\item In the ``initialize'' call, the solver parameters are checked
  for validity.
\item In the ``setup'' call, any non-\id{NULL} \id{ATSetup} and
  \id{PSetup} functions are called.  Typically, these are provided by
  the {\sundials} solvers themselves, that translate between the
  generic \id{ATSetup} and \id{PSetup} functions and the
  solver-specific routines (solver-supplied or user-supplied).
\item In the ``solve'' call the {\pcg} iteration is performed.  This
  will include scaling and preconditioning if those options have been
  supplied.
\end{itemize}

\noindent The header file to be included when using this module 
is \id{sunlinsol/sunlinsol\_pcg.h}. \\
%%
%%----------------------------------------------
%%
The {\sunlinsolpcg} module defines implementations of all
``iterative'' linear solver operations listed in Table
\ref{t:sunlinsolops}:
\begin{itemize}
\item \id{SUNLinSolGetType\_PCG}
\item \id{SUNLinSolInitialize\_PCG}
\item \id{SUNLinSolSetATimes\_PCG}
\item \id{SUNLinSolSetPreconditioner\_PCG}
\item \id{SUNLinSolSetScalingVectors\_PCG} -- since {\pcg} only
  supports symmetric scaling, the second {\nvector} argument to this
  function is ignored
\item \id{SUNLinSolSetup\_PCG}
\item \id{SUNLinSolSolve\_PCG}
\item \id{SUNLinSolNumIters\_PCG}
\item \id{SUNLinSolResNorm\_PCG}
\item \id{SUNLinSolResid\_PCG}
\item \id{SUNLinSolLastFlag\_PCG}
\item \id{SUNLinSolSpace\_PCG}
\item \id{SUNLinSolFree\_PCG}
\end{itemize}
The module {\sunlinsolpcg} provides the following additional
user-callable routines: 
%%
\begin{itemize}

%%--------------------------------------

\item \ID{SUNPCG}

  This function creates and allocates memory for a {\pcg}
  \id{SUNLinearSolver}.  Its arguments are an {\nvector}, a flag
  indicating to use preconditioning, and the number of linear
  iterations to allow. 

  This routine will perform consistency checks to ensure that it is
  called with a consistent {\nvector} implementation (i.e.~that it
  supplies the requisite vector operations).  If \id{y} is
  incompatible then this routine will return \id{NULL}.

  A \id{maxl} argument that is $\le0$ will result in the default
  value (5).

  Since the {\pcg} algorithm is designed to only support symmetric
  preconditioning, then any of the \id{pretype} inputs \id{PREC\_LEFT}
  (1), \id{PREC\_RIGHT} (2), or \id{PREC\_BOTH} (3) will result in use
  of the symmetric preconditioner;  any other integer input will
  result in the default (no preconditioning).

  \verb|SUNLinearSolver SUNPCG(N_Vector y, int pretype, int maxl);|

%%--------------------------------------

\item \ID{SUNPCGSetPrecType}

  This function updates the flag indicating use of preconditioning.
  As above, any one of the input values, \id{PREC\_LEFT} (1),
  \id{PREC\_RIGHT} (2) and \id{PREC\_BOTH} (3) will enable
  preconditioning; \id{PREC\_NONE} (0) disables preconditioning.

  This routine will return with one of the error codes
  \id{SUNLS\_ILL\_INPUT} (illegal \id{pretype}), \id{SUNLS\_MEM\_NULL}
  (\id{S} is \id{NULL}) or \id{SUNLS\_SUCCESS}.
  
  \verb|int SUNPCGSetPrecType(SUNLinearSolver S, int pretype);|

%%--------------------------------------

\item \ID{SUNPCGSetMaxl}

  This function updates the number of linear solver iterations to
  allow. 

  A \id{maxl} argument that is $\le0$ will result in the default
  value (5).

  This routine will return with one of the error codes
  \id{SUNLS\_MEM\_NULL} (\id{S} is \id{NULL}) or \id{SUNLS\_SUCCESS}.
  
  \verb|int SUNPCGSetMaxl(SUNLinearSolver S, int maxl);|

\end{itemize}
%%
%%------------------------------------
%%
For solvers that include a Fortran interface module, the
{\sunlinsolpcg} module also includes the Fortran-callable
function \id{FSUNPCGInit(code, pretype, maxl, ier)} to initialize
this {\sunlinsolpcg} module for a given {\sundials} solver.
Here \id{code} is an input solver id (1 for {\cvode}, 2 for {\ida}, 3
for {\kinsol}, 4 for {\arkode}); \id{pretype} and \id{maxl} are the
same as for the C function \ID{SUNPCG}; \id{ier} is an error return
flag equal 0 for success and -1 for failure.  All of these input
arguments should be declared so as to match C type \id{int}).  This
routine must be called \emph{after} the {\nvector} object has been
initialized.  Additionally, when using {\arkode} with non-identity
mass matrix, the Fortran-callable function 
\id{FSUNMassPCGInit(pretype, maxl, ier)} initializes this
{\sunlinsolpcg} module for solving mass matrix linear systems.

The \id{SUNPCGSetPrecType} and \id{SUNPCGSetMaxl} routines also
support Fortran interfaces for the system and mass matrix solvers:
\begin{itemize}
\item \id{FSUNPCGSetPrecType(code, pretype, ier)} -- all arguments
  should be commensurate with a C \id{int}
\item \id{FSUNMassPCGSetPrecType(pretype, ier)}
\item \id{FSUNPCGSetMaxl(code, maxl, ier)} -- all arguments
  should be commensurate with a C \id{int}
\item \id{FSUNMassPCGSetMaxl(maxl, ier)}
\end{itemize}

\section{SUNLinearSolver Examples}\label{ss:sunlinsol_examples}

There are \id{SUNLinearSolver} examples that may be installed for each
implementation; these make use of the functions in \id{test\_sunlinsol.c}.
These example functions show simple usage of the \id{SUNLinearSolver} family
of functions.  The inputs to the examples depend on the linear solver type,
and are output to \texttt{stdout} if the example is run without the
appropriate number of command-line arguments.

\noindent The following is a list of the example functions in \id{test\_sunlinsol.c}:
\begin{itemize}
\item \id{Test\_SUNLinSolGetType}: Verifies the returned solver type against
  the value that should be returned.
\item \id{Test\_SUNLinSolInitialize}: Verifies that \id{SUNLinSolInitialize}
  can be called and returns successfully.
\item \id{Test\_SUNLinSolSetup}: Verifies that \id{SUNLinSolSetup} can
  be called and returns successfully.
\item \id{Test\_SUNLinSolSolve}: Given a {\sunmatrix} object $A$,
  {\nvector} objects $x$ and $b$ (where $Ax=b$) and a desired solution
  tolerance \texttt{tol}, this routine clones $x$ into a new vector $y$,
  calls \\ \noindent
  \id{SUNLinSolSolve} to fill $y$ as the solution to $Ay=b$ (to
  the input tolerance), verifies that each entry in $x$ and $y$
  match to within \texttt{10*tol}, and overwrites $x$ with $y$ prior
  to returning (in case the calling routine would like to investigate
  further).
\item \id{Test\_SUNLinSolSetATimes} (iterative solvers only): Verifies that
  \id{SUNLinSolSetATimes} can be called and returns successfully.
\item \id{Test\_SUNLinSolSetPreconditioner} (iterative solvers only):
  Verifies that \\ \noindent
  \id{SUNLinSolSetPreconditioner} can be called and
  returns successfully.
\item \id{Test\_SUNLinSolSetScalingVectors} (iterative solvers only):
  Verifies that \\ \noindent
  \id{SUNLinSolSetScalingVectors} can be called and
  returns successfully.
\item \id{Test\_SUNLinSolLastFlag}: Verifies that \id{SUNLinSolLastFlag} can
  be called, and outputs the result to \texttt{stdout}.
\item \id{Test\_SUNLinSolNumIters} (iterative solvers only): Verifies that
  \id{SUNLinSolNumIters} can be called, and outputs the result to
  \texttt{stdout}.
\item \id{Test\_SUNLinSolResNorm} (iterative solvers only): Verifies that
  \id{SUNLinSolResNorm} can be called, and that the result is
  non-negative.
\item \id{Test\_SUNLinSolResid} (iterative solvers only): Verifies that
  \id{SUNLinSolResid} can be called.
\item \id{Test\_SUNLinSolSpace} verifies that \id{SUNLinSolSpace} can be
  called, and outputs the results to \texttt{stdout}.
\end{itemize}
We'll note that these tests should be performed in a particular
order.  For either direct or iterative linear
solvers, \id{Test\_SUNLinSolInitialize} must be called
before \id{Test\_SUNLinSolSetup}, which must be called
before \id{Test\_SUNLinSolSolve}.  Additionally, for iterative linear
solvers \\ \noindent
\id{Test\_SUNLinSolSetATimes}, \id{Test\_SUNLinSolSetPreconditioner}
and \\ \noindent
\id{Test\_SUNLinSolSetScalingVectors} should be called
before \id{Test\_SUNLinSolInitialize};
similarly \id{Test\_SUNLinSolNumIters}, \id{Test\_SUNLinSolResNorm}
and \id{Test\_SUNLinSolResid} should be called
after \id{Test\_SUNLinSolSolve}.  These are called in the appropriate
order in all of the example problems.

