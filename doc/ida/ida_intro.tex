%===================================================================================
\chapter{Introduction}\label{s:intro}
%===================================================================================

{\ida} is part of a software family called {\sundials}: 
SUite of Nonlinear and DIfferential/ALgebraic equation Solvers~\cite{HBGLSSW:05}. 
This suite consists of {\cvode}, {\kinsol}, and {\ida}, and variants of these
with sensitivity analysis capabilities.

{\ida} is a general purpose solver for the initial value problem (IVP) for
systems of differential-algebraic equations (DAEs).  The name IDA
stands for Implicit Differential-Algebraic solver.  {\ida} is based on
{\daspk}~\cite{BHP:94,BHP:98}, but is written in ANSI-standard C
rather than {\F}77.  Its most notable features are that,
(1) in the solution of the underlying nonlinear system at each time
step, it offers a choice of Newton/direct methods and a choice of
Inexact Newton/Krylov (iterative) methods; and
(2) it is written in a {\em data-independent} manner in that it acts
on generic vectors without any assumptions on the underlying organization
of the data.
Thus {\ida} shares significant modules previously
written within CASC at LLNL to support the ordinary differential
equation (ODE) solvers {\cvode}~\cite{cvode_ug,CoHi:96} and 
{\pvode}~\cite{ByHi:98,ByHi:99}, and also the nonlinear system solver 
{\kinsol}~\cite{kinsol_ug}.

The Newton/Krylov methods in {\ida} are:
the GMRES (Generalized Minimal RESidual)~\cite{SaSc:86},
Bi-CGStab (Bi-Conjugate Gradient Stabilized)~\cite{Van:92}, and
TFQMR (Transpose-Free Quasi-Minimal Residual) linear iterative 
methods~\cite{Fre:93}.  As Krylov methods, these require almost no 
matrix storage for solving the Newton equations as compared to direct
methods. However, the algorithms allow for a user-supplied preconditioner
matrix, and for most problems preconditioning is essential for an
efficient solution.

For very large DAE systems, the Krylov methods are preferable over
direct linear solver methods, and are often the only feasible choice.
Among the three Krylov methods in {\ida}, we recommend GMRES as the
best overall choice.  However, users are encouraged to compare all
three, especially if encountering convergence failures with GMRES.
Bi-CGFStab and TFQMR have an advantage in storage requirements, in
that the number of workspace vectors they require is fixed, while that
number for GMRES depends on the desired Krylov subspace size.

\index{IDA@{\ida}!motivation for writing in C|(}
There are several motivations for choosing the {\C} language for {\ida}.
First, a general movement away from {\F} and toward {\C} in scientific
computing is apparent.  Second, the pointer, structure, and dynamic
memory allocation features in {\C} are extremely useful in software of
this complexity, with the great variety of method options offered.
Finally, we prefer {\C} over {\CPP} for {\ida} because of the wider
availability of {\C} compilers, the potentially greater efficiency of {\C},
and the greater ease of interfacing the solver to applications written
in extended {\F}.
\index{IDA@{\ida}!motivation for writing in C|)}

\section{Changes from previous versions}

\subsection*{Changes in v2.7.0}

One significant design change was made with this release: The problem
size and its relatives, bandwidth parameters, related internal indices,
pivot arrays, and the optional output \id{lsflag} have all been
changed from type \id{int} to type \id{long int}, except for the
problem size and bandwidths in user calls to routines specifying
BLAS/LAPACK routines for the dense/band linear solvers.  The function
\id{NewIntArray} is replaced by a pair \id{NewIntArray}/\id{NewLintArray},
for \id{int} and \id{long int} arrays, respectively.

A large number of minor errors have been fixed.  Among these are the following:
After the solver memory is created, it is set to zero before being filled.
To be consistent with IDAS, IDA uses the function \id{IDAGetDky} for
optional output retrieval.
In each linear solver interface function, the linear solver memory is
freed on an error return, and the \id{**Free} function now includes a
line setting to NULL the main memory pointer to the linear solver memory.
A memory leak was fixed in two of the \id{IDASp***Free} functions.
In the installation files, we modified the treatment of the macro
SUNDIALS\_USE\_GENERIC\_MATH, so that the parameter GENERIC\_MATH\_LIB is
either defined (with no value) or not defined.

\subsection*{Changes in v2.6.0}

Two new features were added in this release: (a) a new linear solver module,
based on Blas and Lapack for both dense and banded matrices, and (b) option
to specify which direction of zero-crossing is to be monitored while performing
rootfinding. 

The user interface has been further refined. Some of the API changes involve:
(a) a reorganization of all linear solver modules into two families (besides 
the already present family of scaled preconditioned iterative linear solvers,
the direct solvers, including the new Lapack-based ones, were also organized 
into a {\em direct} family); (b) maintaining a single pointer to user data,
optionally specified through a \id{Set}-type function; (c) a general 
streamlining of the band-block-diagonal preconditioner module distributed 
with the solver.

\subsection*{Changes in v2.5.0}

The main changes in this release involve a rearrangement of the entire 
{\sundials} source tree (see \S\ref{ss:sun_org}). At the user interface 
level, the main impact is in the mechanism of including {\sundials} header
files which must now include the relative path (e.g. \id{\#include <cvode/cvode.h>}).
Additional changes were made to the build system: all exported header files are
now installed in separate subdirectories of the installation {\em include} directory.

A bug was fixed in the internal difference-quotient dense and banded
Jacobian approximations, related to the estimation of the perturbation
(which could have led to a failure of the linear solver when zero
components with sufficiently small absolute tolerances were present).

The user interface to the consistent initial conditions calculations was modified. 
The \id{IDACalcIC} arguments \id{t0}, \id{yy0}, and \id{yp0} were removed and 
a new function, \id{IDAGetconsistentIC} is provided (see \S\ref{ss:idacalcic} and 
\S\ref{sss:optout_iccalc} for details).

The functions in the generic dense linear solver (\id{sundials\_dense} and
\id{sundials\_smalldense}) were modified to work for rectangular $m \times n$
matrices ($m \le n$), while the factorization and solution functions were
renamed to \id{DenseGETRF}/\id{denGETRF} and \id{DenseGETRS}/\id{denGETRS}, 
respectively.
The factorization and solution functions in the generic band linear solver were 
renamed \id{BandGBTRF} and \id{BandGBTRS}, respectively.

\subsection*{Changes in v2.4.0}

{\fida}, a {\F}-{\C} interface module, was added (for details see Chapter
\ref{s:fcmix}).

{\idaspbcg} and {\idasptfqmr} modules have been added to interface with the
Scaled Preconditioned Bi-CGstab ({\spbcg}) and Scaled Preconditioned
Transpose-Free Quasi-Minimal Residual ({\sptfqmr}) linear solver modules,
respectively (for details see Chapter \ref{s:simulation}).
At the same time, function type names for Scaled Preconditioned Iterative
Linear Solvers were added for the user-supplied Jacobian-times-vector and
preconditioner setup and solve functions.

The rootfinding feature was added, whereby the roots of a set of given
functions may be computed during the integration of the DAE system.

A user-callable routine was added to access the estimated local error
vector.

The deallocation functions now take as arguments the address of the respective 
memory block pointer.

To reduce the possibility of conflicts, the names of all header files have
been changed by adding unique prefixes (\id{ida\_} and \id{sundials\_}).
When using the default installation procedure, the header files are exported
under various subdirectories of the target \id{include} directory. For more
details see Appendix \ref{c:install}.

\subsection*{Changes in v2.3.0}

The user interface has been further refined. Several functions used
for setting optional inputs were combined into a single one.
An optional user-supplied routine for setting the error weight
vector was added.
Additionally, to resolve potential variable scope issues, all
SUNDIALS solvers release user data right after its use. The build
systems has been further improved to make it more robust.

\subsection*{Changes in v2.2.2}

Minor corrections and improvements were made to the build system.
A new chapter in the User Guide was added --- with constants that
appear in the user interface.

\subsection*{Changes in v2.2.1}

The changes in this minor {\sundials} release affect only the build system.

\subsection*{Changes in v2.2.0}

The major changes from the previous version involve a redesign of the
user interface across the entire {\sundials} suite. We have eliminated the
mechanism of providing optional inputs and extracting optional statistics 
from the solver through the \id{iopt} and \id{ropt} arrays. Instead,
{\ida} now provides a set of routines (with prefix \id{IDASet})
to change the default values for various quantities controlling the
solver and a set of extraction routines (with prefix \id{IDAGet})
to extract statistics after return from the main solver routine.
Similarly, each linear solver module provides its own set of {\id{Set}-}
and {\id{Get}-type} routines. For more details see \S\ref{ss:optional_input}
and \S\ref{ss:optional_output}.

Additionally, the interfaces to several user-supplied routines
(such as those providing Jacobians and preconditioner information) 
were simplified by reducing the number
of arguments. The same information that was previously accessible
through such arguments can now be obtained through {\id{Get}-type}
functions.

Installation of {\ida} (and all of {\sundials}) has been completely 
redesigned and is now based on configure scripts.


\section{Reading this User Guide}\label{ss:reading}

The structure of this document is as follows:
\begin{itemize}
\item
  In Chapter \ref{s:math}, we give short descriptions of the numerical 
  methods implemented by {\ida} for the solution of initial value problems
  for systems of DAEs, along with short descriptions of preconditioning
  (\S\ref{s:preconditioning}) and rootfinding (\S\ref{ss:rootfinding}).
\item
  The following chapter describes the structure of the {\sundials} suite
  of solvers (\S\ref{ss:sun_org}) and the software organization of the {\ida}
  solver (\S\ref{ss:ida_org}). 
\item
  Chapter \ref{s:simulation} is the main usage document for {\ida} for
  {\C} applications.  It includes a complete description of the user interface
  for the integration of DAE initial value problems.
\item
  In Chapter \ref{s:fcmix}, we describe {\fida}, an interface module
  for the use of {\ida} with {\F} applications.
\item
  Chapter \ref{s:nvector} gives a brief overview of the generic {\nvector} module 
  shared among the various components of {\sundials}, as well as details on the two
  {\nvector} implementations provided with {\sundials}: a serial implementation
  (\S\ref{ss:nvec_ser}) and a parallel MPI implementation (\S\ref{ss:nvec_par}).
\item
  Chapter \ref{s:new_linsolv} describes the interfaces to the linear solver
  modules, so that a user can provide his/her own such module.
\item
  Chapter \ref{s:gen_linsolv} describes in detail the generic linear solvers shared 
  by all {\sundials} solvers.
\item
  Finally, in the appendices, we provide detailed instructions for the installation
  of {\ida}, within the structure of {\sundials} (Appendix \ref{c:install}), as well
  as a list of all the constants used for input to and output from {\ida} functions
  (Appendix \ref{c:constants}).
\end{itemize}

Finally, the reader should be aware of the following notational conventions
in this user guide:  program listings and identifiers (such as \id{IDAInit}) 
within textual explanations appear in typewriter type style; 
fields in {\C} structures (such as {\em content}) appear in italics;
and packages or modules, such as {\idadense}, are written in all capitals. 
Usage and installation instructions that constitute important warnings
are marked with a triangular symbol {\warn} in the margin.

\paragraph{Acknowledgments.}
We wish to acknowledge the contributions to previous versions of the
{\ida} code and user guide of Allan G. Taylor.

