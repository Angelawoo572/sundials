\subsection{KINSOL}

KINSOL solves nonlinear algebraic systems, which we write as
\[ F(u) = 0,~~ F:R^N \rightarrow R^N, \]
given an initial guess $u_0$.  It is a rewrite in C of the Fortran
code NKSOL of Brown and Saad \cite{NKSOL}.

The method used by KINSOL is Inexact Newton iteration.  The Newton
correction equation $J \Delta u_n = -F(u_n)$ is solved only
approximately, using a preconditioned Krylov method.  In this case,
the solver is SPGMR = Scaled Preconditioned GMRES, with restarts
allowed (optionally).  Preconditioning is done only on the right, so
that GMRES is applied to the linear systems $(JP^{-1})(P\Delta) = -F$.

The Krylov (GMRES) iteration requires matrix-vector products $J(u)v$,
which are approximated in a matrix-free manner by difference quotients:
\[ J(u)v \approx \frac{F(u+\sigma v) - F(u)}{\sigma} ~. \]
However, there is an option for a user-supplied routine to compute
these products.

KINSOL offers a choice of two Newton strategies: One is the basic
Inexact Newton scheme outlined above.  The second is Inexact Newton
augmented with a Linesearch/Backtrack algorithm to enhance global
convergence.  (NKSOL also had a Dogleg Method, but KINSOL does not.)

As a user option, KINSOL permits the application of inequality
constraints -- either $u^i > 0$ or $u^i < 0$.  Either constraint, or
no constraint, may be imposed on each component.

KINSOL controls errors at three different levels:

1. First, the Newton iteration must pass a stopping test,
\[ \|D_F F(u_n)\| < ftol ~, \]
where $D_F$ is an input scaling vector applied to $F$, and $ftol$ is
an input scalar tolerance.

2. The Krylov iteration must pass a stopping test,
\[ \|J \Delta_k + F\|_S < \eta_k \|F\|_S ~, \]
in which the norms are scaled using $D_S$.  There are three choices
for the ``forcing term'' $\eta_k$:
\vspace*{-.19in}
\begin{itemize}
\item two choices of Eisenstat and Walker \cite{EiWa96}; the default
is ``Choice 1''.
\item  $\eta_k$ = constant
\end{itemize}

3. The size of the steps $\Delta_k$ in $u$ is controlled with the help
of a user-supplied scaling vector $D_u$ = scaling for $u$.  This
scaling is also used in the choice of $\sigma$.


