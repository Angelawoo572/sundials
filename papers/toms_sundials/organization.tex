\section{Code Organization}

The writing of CVODE from the Fortran solvers VODE and VODPK involved a
complete redesign and reorganization.  This was done partly to exploit
features of C not present in Fortran, partly to achieve a more
object-oriented design, partly to maximize the reuse of code modules, and
partly to facilitate the extension from a serial to a parallel
implementation.

The process of modularization has continued with the collection of CVODE,
KINSOL, and IDA into SUNDIALS. The SUNDIALS distribution now contains common
modules in a shared directory. Additionally, compilation of SUNDIALS is now
independent of any prior specification of a particular NVECTOR
implementation, facilitating the use of binary libraries. The current
NVECTOR structure also allows use of multiple implementations within the
same code, as may be required to meet user needs.

The features of the design of CVODE include the following:
\begin{itemize}
\item Memory allocation is heavily used.
\item The linear solver modules are separate from the core integrator,
so that the latter is independent of the method for solving linear
systems.
\item Each linear solver has generic solver, which is independent of
the ODE contest, together with an interface to the CVODE core
integrator module.
\item The vector kernels (linear sums, dot products, norms, etc.) on
$N-$vectors are isolated in a separate NVECTOR module.
\end{itemize}

The following Fig. \ref{fig-cvorg} shows the overall structure of SUNDIALS
solvers, with the various separate modules.  The separation of the linear
solvers from the core integrator allows for easy addition of other linear
solver algorithms in the future.  At the bottom level are the NVECTOR
module, default serial and parallel implementations, and two small modules,
SUNDIALSTYPS and SUNDIALSMATH dealing with data types and elementary
mathematical operations.

\begin{figure}[l]
\centerline{\psfig{figure=cvorg.eps,height=6.5in}}
\vspace{0.8in}
\caption{Overall structure of the SUNDIALS  package.}
\label{fig-cvorg}
\end{figure}

A number of user-supplied routines for the SUNDIALS solvers are not shown in
\ref{fig-cvorg}.  These routines must include a routine for the evaluation
of $f$ (CVODE) or $F$ (KINSOL and IDA). The user provided routines may
include, depending on the options chosen, routines for Jacobian evaluation
(direct cases) or Jacobian-vector products (Krylov case), and the setup and
solution routines for the Krylov preconditioner.

The evolution of SUNDIALS has been directed keeping the entire set of
solvers in mind.  Thus CVODE, KINSOL and IDA all share many of the same
modules, as can be seen in \ref{fig-cvorg}.

The KINSOL solver, which includes only one linear system solver at present,
sharing with CVODE and IDA the GMRES based SPGMR module, the NVECTOR module,
and the lower level modules SUNDIALSTYPS and SUNDIALSMATH. The KINSOL user
must supply a routine for $F$ evaluation, and may (optionally) supply
routines for $Jv$ products, and for the preconditioner.

The structure and organization of IDA and CVODE are very similar. They
shares all of the generic dense, band, and SPGMR linear system solvers, the
NVECTOR module, and the modules SUNDIALSTYPS and SUNDIALSMATH.

The IDA user must supply a routine for the evaluation of $F$, and may
(optionally) include routines for Jacobian evaluation, and routines
for the preconditioner.

\subsection{Shared Modules - NVECTOR}

The NVECTOR module is shared among all the SUNDIALS solvers.
Actually, as supplied with the complete package, this module
includes three submodules:
\begin{itemize}
\item Generic NVECTOR
\item NVECTOR\_SERIAL
\item NVECTOR\_PARALLEL
\end{itemize}

The Generic NVECTOR module defines:
\begin{itemize}
\item an NVECTOR  structure specification {\tt NV\_Spec}
\item a data-independent {\tt N\_Vector} type
\item (in {\tt NV\_Spec}) a set of operations
\item a set of kernels, which are wrappers around the actual kernels
accessed through the operation set
\end{itemize}

Each NVECTOR\_*** implementation supplied with the package (or any
NVECTOR module defined by the user) defines:
\begin{itemize}
\item a content field of {\tt NV\_Spec}
\item a content field of {\tt N\_Vector}
\item a set of implemented vector kernels
\item a function to construct {\tt NV\_Spec} and fill the list of operations
\end{itemize}

More specifically, in NVECTOR\_PARALLEL, the types and kernels are
defined so as to operate on distributed vectors, with all N-vectors
distributed the same way.  The type {\tt NV\_Spec} includes
\begin{itemize}
  \item the local vector length
  \item the global vector length
  \item the MPI communicator
\end{itemize}
and the {\tt N\_Vector} content field includes
\begin{itemize}
  \item the local vector length
  \item the global vector length
  \item the local data array
\end{itemize}

If neither package NVECTOR implementation is suitable, the user can provide
one or more NVECTOR implementations.  For example, it might (and has been)
more practical to substitute a more complex data structure.


