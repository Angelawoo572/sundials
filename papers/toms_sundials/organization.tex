\section{Code Organization}
\label{s:organization}

The writing of CVODE from the Fortran solvers VODE and VODPK initiated a
complete redesign and reorganization of the existing LLNL  solver coding.
This was done to: exploit features of C not present in Fortran;
achieve a more object-oriented design; maximize the reuse of code
modules; and, facilitate the extension from a serial to a parallel
implementation.

The features of the design of CVODE include the following:
\begin{itemize}
\item Memory allocation is heavily used;
\item The linear solver modules are separate from the core integrator,
so that the latter is independent of the method for solving linear
systems;
\item Each linear solver module contains  a generic solver, which is independent 
of the ODE context, together with an interface to the CVODE core
integrator module; and, 
\item The vector operations  (linear sums, dot products, norms, etc.) on
$N-$vectors are isolated in a separate ``NVECTOR'' module.
\end{itemize}

The process of modularization has continued with the development of CVODE,
KINSOL, and IDA. The SUNDIALS distribution now contains a number of common
modules in a shared directory. Additionally, compilation of SUNDIALS is now
independent of any prior specification of a particular NVECTOR
implementation, facilitating the use of binary libraries. The current
NVECTOR structure also allows the use of multiple implementations within the
same code, as may be required to meet user needs.

Figure \ref{fig-sunorg} shows the overall structure of SUNDIALS, with the
various separate modules. The evolution of SUNDIALS has been directed toward
keeping the entire set of solvers in mind. Thus, CVODE, KINSOL and IDA share
much in their organization and have a number of common modules.  The
separation of the linear solvers from the core integrators allows for easy
addition of linear solvers not currently included in SUNDIALS. At the bottom
level is the NVECTOR module, providing key vector operations such as
creation, destruction, summation, and dot-products on potentially
distributed data vectors. Serial and parallel NVECTOR implementations are
included with SUNDIALS, but a user can substitute his/her own implementation
as useful. Two small modules defining several data types and elementary
mathematical operations are also included.

\begin{figure}[tp]
\centerline{\psfig{figure=sunorg.eps,width=\textwidth}}
\caption{Overall structure of the SUNDIALS  package.}
\label{fig-sunorg}
\end{figure}

A number of necessary and optional user-supplied routines for the SUNDIALS
solvers are not shown in \mbox{Figure \ref{fig-sunorg}}. The user must
provide a routine for the evaluation of $f$ (CVODE) or $F$ (KINSOL and
IDA). The user-provided routines may include, depending on the options
chosen, routines for Jacobian evaluation (direct cases) or Jacobian-vector
products (Krylov case), and routines for the setup and solution of Krylov
preconditioners.


\subsection{Shared Modules - Linear Solvers}

As can be seen in \mbox{Figure \ref{fig-sunorg}}, three linear solver
packages are currently included with SUNDIALS: a direct dense matrix solver
(DENSE), a direct band solver (BAND), and an iterative Krylov solver
(SPGMR). These are stand-alone packages in their own right.

The shared linear solvers are accessed from SUNDIALS via
solver-specific wrappers. 
Thus, SPGMR is accessed via CVSPGMR, IDASPGMR, and
KINSPGMR, for CVODE (and CVODES), IDA, and KINSOL, respectively. For the
DENSE solver, the wrappers are CVDENSE and IDADENSE for CVODE/CVODES and
IDA, respectively. Similar wrappers for BAND are CVBAND and IDABAND. KINSOL,
because of its design specific to large problems, does not interface with
the direct solvers.

\subsection{Shared Modules - NVECTOR}

A generic NVECTOR implementation is used within SUNDIALS to
operate on N-vectors. This generic implementation defines an NVECTOR
structure specification, a data-independent NVECTOR type, a set of abstract
vector operations, and a set of wrappers for accessing the actual vector
operations of the implementation under which an NVECTOR was created. Because
details of vector operations are thus encapsulated within each specific
NVECTOR implementation, SUNDIALS solvers are now independent of a specific
implementation. This allows the solvers to be precompiled as binary
libraries and allows more than one NVECTOR implementation to be used within
a single program.

A particular NVECTOR implementation, such as the serial and parallel 
implementations included with SUNDIALS or a user-provided implementation,
must provide certain functionalities. Each implementation must provide a
specification of the data needed to generate a new NVECTOR and a menu to
routines for operations on N-vectors, including, for example, creation,
destruction, summation, element-by-element inversion, and dot product. Each
NVECTOR includes data and descriptive fields as well as a pointer to the
specification for the implementation to which it belongs.

If neither the serial nor parallel package NVECTOR implementation is
suitable, the user can provide one or more NVECTOR implementations.  For
example, it might (and has been) more practical to substitute a more complex
data structure in a parallel implementation.
