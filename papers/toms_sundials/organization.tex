\section{Code Organization}

The writing of CVODE from the Fortran solvers VODE and VODPK involved
a complete redesign and reorganization.  This was done partly to
exploit features of C not present in Fortran, partly to achieve a more
object-oriented design, partly to maximize the reuse of code modules,
and partly to facilitate the extension from a serial to a parallel
implementation.

The features of the design of CVODE include the following:
\vspace*{-.19in}
\begin{itemize}
\item Memory allocation is heavily used.
\item The linear solver modules are separate from the core integrator,
so that the latter is independent of the method for solving linear
systems.
\item Each linear solver has generic solver, which is independent of
the ODE contest, together with an interface to the CVODE core
integrator module.
\item The vector kernels (linear sums, dot products, norms, etc.) on
$N-$vectors are isolated in a separate NVECTOR module.
\end{itemize}

The following Fig. \ref{fig-cvorg} shows the overall structure of
CVODE, with the various separate modules.  The separation of the
linear solvers from the core integrator allows for easy addition of
other linear solver algorithms in the future.  At the bottom level are
the NVECTOR module, and two small modules, SUNDIALSTYPS and
SUNDIALSMATH dealing with data types and elementary mathematical
operations.

\begin{figure}[p]
\centerline{\psfig{figure=cvorg.eps,height=5.5in}}
\vspace{0.8in}
\caption{Overall structure of the CVODE package.}
\label{fig-cvorg}
\end{figure}

Not shown in the figure are the user-supplied routines for CVODE.
These must include a routine for the evaluation of $f$, and may
(depending on the options chosen) include routines for Jacobian
evaluation (direct cases) or Jacobian-vector products (Krylov case),
and routines for the Krylov preconditioner.

The design of CVODE was also done with the addition of the other
solvers in mind.  Thus KINSOL and IDA, the other two basic members of
the SUNDIALS family, all share many of the same modules.  In fact
their organization charts very much resemble that of CVODE.

The KINSOL solver, which includes only one linear system solver at
present, shares with CVODE the SPGMR module, the NVECTOR module, and
the lower level modules SUNDIALSTYPS and SUNDIALSMATH.

The KINSOL user must supply a routine for $F$ evaluation, and may
(optionally) supply routines for $Jv$ products, and for the
preconditioner.

The structure and organization of IDA resembles that of CVODE even
more.  It shares with CVODE all of the generic dense, band, SPGMR
linear system solvers, the NVECTOR module, and the modules
SUNDIALSTYPS and SUNDIALSMATH.

The IDA user must supply a routine for the evaluation of $F$, and may
(optionally) include routines for Jacobian evaluation, and routines
for the preconditioner.

\subsection{Shared Modules - NVECTOR}

The NVECTOR module is shared among all the SUNDIALS solvers.
Actually, as supplied with the complete package, this module
includes three submodules:
\vspace*{-.19in}
\begin{itemize}
\item Generic NVECTOR
\item NVECTOR\_SERIAL
\item NVECTOR\_PARALLEL
\end{itemize}

The Generic NVECTOR module defines:
\vspace*{-.19in}
\begin{itemize}
\item a machine environment structure {\tt M\_Env}
\item a data-independent {\tt N\_Vector} type
\item (in {\tt M\_Env}) a set of operations
\item a set of kernels, which are wrappers around the actual kernels
accessed through the operation set
\end{itemize}

Each NVECTOR\_*** implementation supplied with the package (or any
NVECTOR module defined by the user) defines:
\vspace*{-.19in}
\begin{itemize}
\item a content field of {\tt M\_Env}
\item a content field of {\tt N\_Vector}
\item a set of implemented vector kernels
\item a function to construct {\tt M\_Env} and fill the list of operations
\end{itemize}

More specifically, in NVECTOR\_PARALLEL, the types and kernels are
defined so as to operate on distributed vectors, with all N-vectors
distributed the same way.  The type {\tt M\_Env} includes
\vspace*{-.19in}
\begin{itemize}
  \item the local vector length
  \item the global vector length
  \item the MPI communicator
\end{itemize}
and the {\tt N\_Vector} content field includes
\vspace*{-.19in}
\begin{itemize}
  \item the local vector length
  \item the global vector length
  \item the local data array
\end{itemize}

If neither package NVECTOR implementation is suitable, the user can
provide one.  For example, it might (and has been) more practical to
substitute a more complex data structure.


