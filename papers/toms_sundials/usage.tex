\section{Usage}

Because of space limitations, we will not describe in complete detail
the user interface for any of the SUNDIALS solvers here.  These are
given in the documentation that accompanies each solver.  However, in
order to convey the basic features of these interfaces, we give here a
summary of the usage of the parallel version of CVODE.  The usage
of the other solvers is quite similar.

For users accustomed to Fortan ODE solvers, using CVODE is quite
different, in that one calls several different routines for the
different parts of the solution process.  The following is a template
for the usage of parallel CVODE:

\begin{itemize}

\item Set local and global vector lengths, and specify active
processors (MPI communicator)

\item {\tt machEnv =  M\_EnvInit\_Parallel(comm, Nlocal, N, ...)}:
initialize NVECTOR\_- \newline PARALLEL

\item Set initial values of $t$ and $y$ (type {\tt N\_Vector})

\item {\tt mem = CVodeMalloc(...)}: initialize CVODE (provide {\tt f},
method options, tolerances, and allocate internal memory)

\item {\tt CVSpgmr(...)}: if using Newton iteration, specify SPGMR
(as linear solver) and routines for preconditioner setup and solve

\item {\tt for (tout=...) ier = CVode(...)}: integrate system to
$t =$ {\tt tout}

\item {\tt CVodeFree}: free CVODE memory

\item {\tt M\_EnvFree\_Parallel(machEnv)}: free NVECTOR\_PARALLEL memory

\end{itemize}

In addition to the name of the user function defining $f$ and the
initial conditions, the call to {\tt CVodeMalloc} also specifies the
ODE method (Adams or BDF), Newton iteration or functional iteration,
and tolerances.  CVODE controls numerical errors by way of two input
tolerances:
\vspace*{-.19in}
\begin{itemize}
\item {\tt rtol} = scalar relative tolerance
\item {\tt atol} = absolute tolerance = scalar or vector
\end{itemize}
The resulting error weights ~~{\tt rtol}$|y^i| + ${\tt atol}$^i~~$ are
used to scale all vectors in the control of the local integration
error, and are also used to scale the GMRES algoritm.

The call to the integrator {\tt CVode} includes an option to specify
either the normal mode (take steps until passing {\tt tout} and
interpolate) or one-step mode (take one step and return).


\section{Fortran Usage}

For two of the SUNDIALS solvers -- CVODE and KINSOL, users with
Fortran applications are accommodated.  This is done with a set of
interface routines that connect the C solver with the user's Fortran
routines.

The cross-language calls go in both directions:
\newline \hspace*{.5in} user's Fortran Main $\longrightarrow$
interfaces $\longrightarrow$ solver routines, and
\newline \hspace*{.5in} Solver routines $\longrightarrow$ interfaces
$\longrightarrow$ user's {\tt f} routine etc.

In order to achieve portability for these interfaces, all of the
Fortran user-supplied routines have fixed names.

These interfaces are provided as separate modules, called FCVODE and
FKINSOL, for CVODE and KINSOL, respectively.  In each case, small
examples programs are provided.


