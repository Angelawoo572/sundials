\section{Conclusions}\label{s:conclusions}

{\sf As discussed, mention all the cool applications here, to support
the claim that SUNDIALS is the best thing since sliced bread}

The following are brief descriptions of some of the major applications
of the SUNDIALS solvers that we have been involved with.

\begin{itemize}

\item Parallel CVODE is being used in a 3D tokamak turbulence model in
LLNL's Magnetic Fusion Energy Division.  A typical run has 7 unknowns
on a $64 \times 64 \times 40$ mesh, with up to 60 processors~\cite{RXH:02}.

\item A sensitivity analysis version of CVODE has been used to analyze
radiation diffusion problems and their sensitivity to the parameters that
characterize material opacity~\cite{LWG:03}, \cite{LHB:00}.

\item KINSOL with a HYPRE Multigrid preconditioner is being applied within
LLNL/CASC to solve a nonlinear Richards equation for pressures in
porous media flows.  Fully scalable solution performance obtained on
up to 225 processors of ASCI Blue.  SensKINSOL used to quantify
uncertainty in these groundwater problems.

\item CVODE, KINSOL, IDA, with MG preconditioner, are being used to
solve 3D neutral particle transport problems within LLNL/CASC.
Scalable performance obtained on up to 5800 processors on ASCI Red.

\item SensPVODE, SensKINSOL, and SensIDA have been used to determine
solution sensitivities of neutral particle transport applications at
LLNL w.r.t. various material properties, for solution uncertainty
quantification.

\item IDA and SensIDA are being used in a cloud and aerosol microphysics
model at LLNL to study cloud formation processes, in study of model
parameter sensitivity.

\end{itemize}

{\sf Make some concluding remarks and comments on future development 
directions.}



