\section{Introduction}

{\sf need for time integrators and solvers, need for robust
software that can be added to existing codes, introduce SUNDIALS
and its basic design principles (restrict to minimal info. from
user, let user supply data structures underneath, allow swap of
linear solvers), introduce need for sensitivity and ease of adding
it to existing codes for these problem classes. }

LLNL has a long history of research and development in ordinary
differential equation (ODE) methods and software, and closely related
areas, with emphasis on applications to partial differential equations
(PDEs).  Among the popular Fortran solvers written at LLNL are the
following:
\vspace*{-.19in}
\begin{itemize}

\item VODE: a solver for ODE initial value problems for stiff/nonstiff
systems, with direct solution of linear systems, by Brown, Byrne, and
Hindmarsh \cite{VODE}.

\item VODPK: a variant of VODE with preconditioned Krylov (GMRES
iteration) solution of the linear systems in place of direct methods,
by Brown, Byrne, and Hindmarsh \cite{VODPK}.

\item NKSOL: a Newton-Krylov (GMRES) solver for nonlinear algebraic
systems, by Brown and Saad \cite{NKSOL}.

\item DASPK: a solver for differential-algebraic equation (DAE)
systems (a variant of DASSL) with both direct and preconditioned
Krylov solution methods for the linear systems, by Brown, Hindmarsh,
and Petzold \cite{DASPK}.

\end{itemize}

In recent years, there has been special interest in two kinds of
extensions of this software.  One is the extension to parallel
solution of large problems, especially on massively parallel machines.
The other is extensions to do sensitivity analysis, which centers
on the calculation of sensitivity of solution with respect to model
parameters.

Starting in 1993, the push to solve large systems in parallel
motivated work to write or rewrite solvers in C.  The first result of
that effort was CVODE.  This was a rewrite in ANSI standard C of the
VODE and VODPK solvers combined, for serial machines
\cite{CVUdoc,CVODE}.  The next result of this effort was PVODE, a
parallel extension of CVODE \cite{PVUdoc,PVODE}.

Similar rewrites of NKSOL and DASPK were then done, using the same
general design as CVODE and PVODE.  The resulting solvers are called
KINSOL and IDA, respectively.

More recently, we have changed the naming of these codes, in order
to be consistent throughout the family.  Specifically, there is
one solver, CVODE, in two versions -- serial and parallel.  Thus
we refer to the parallel version of CVODE, rather than PVODE.


