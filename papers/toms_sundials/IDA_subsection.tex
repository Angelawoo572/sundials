\subsection{IDA}\label{ss:IDA}

% What does IDA do?
%------------------
The IDA code is a C implementation of a previous code, DASPK, a DAE
system solver written in Fortran by Petzold, Brown, and Hindmarsh
\cite{BHP:94,BCP:96}.  IDA solves the initial value problem for 
a DAE system of the general form
\begin{equation}\label{e:DAE}
  F(t,y,{\dot y}) = 0 \, ,
  \quad y(t_0) = y_0 \, ,~ {\dot y}(t_0) = {\dot y}_0 \, ,
\end{equation}
where $y$, ${\dot y}$, and $F$ are vectors in ${\bf R}^N$, $t$ is the independent
variable, ${\dot y} = dy/dt$, 
and initial conditions $y(t_0) = y_0$, ${\dot y}(t_0) = {\dot y}_0$ 
are given.  (Often $t$ is time, but it certainly need not be.)

% Initial condition calculation
%------------------------------
Prior to integrating a DAE initial-value problem, an important requirement 
is that the pair of vectors $y_0$ and ${\dot y_0}$ are both initialized to
satisfy the DAE residual $F(t_0,y_0, {\dot y}_0) = 0$.
For a class of problems that includes so-called
semi-explicit index-one systems, IDA provides a routine that computes
consistent initial conditions from a user's initial guess~\cite{BHP:98}.  
For this, the user must identify subvectors of $y$
(not necessarily contiguous), denoted $y_d$ and $y_a$, which are its
differential and algebraic parts, respectively, such that $F$ depends
on ${\dot y}_d$ but not on any components of ${\dot y}_a$.  The assumption that
the system is ``index-one'' means that for a given $t$ and $y_d$, the
system $F(t,y,{\dot y}) = 0$ defines $y_a$ uniquely.  In this case, a solver
within IDA computes $y_a$ and ${\dot y}_d$ at $t = t_0$, given $y_d$ and an
initial guess for $y_a$.  A second available option with this solver
also computes all of $y(t_0)$ given ${\dot y}(t_0)$; this is intended mainly
for quasi-steady state problems, where ${\dot y}(t_0) = 0$ is given.
In both cases, IDA solves the system $F(t_0,y_0, {\dot y}_0) = 0$ for the
unknown components of $y_0$ and ${\dot y}_0$, using Newton iteration
augmented with a linesearch global strategy.  In doing this, it makes
use of the existing machinery that is to be used for solving the
linear systems during the integration, in combination with certain
tricks involving the stepsize (which is set artificially for this
calculation).
For problems that do not fall into either of these categories, the
user is responsible for passing consistent values, or risk failure in
the numerical integration.

% Integration method and nonlinear system
%----------------------------------------
The integration method in IDA is variable-order, variable-coefficient
BDF, in fixed-leading-coefficient form~\cite{BCP:96}.
The method order ranges from 1 to 5, with the BDF of order $q$
given by the multistep formula
\begin{equation}\label{e:BDF}
  \sum_{i=0}^q \alpha_{n,i}y_{n-i} = h_n {\dot y}_n \, ,
\end{equation}
where $y_n$ and ${\dot y}_n$ are the computed approximations to $y(t_n)$
and ${\dot y}(t_n)$, respectively, and the stepsize is $h_n = t_n - t_{n-1}$.  
The coefficients $\alpha_{n,i}$ are uniquely determined by the order
$q$, and the history of the stepsizes.  The application of the BDF
(\ref{e:BDF}) to the DAE system (\ref{e:DAE}) results in a nonlinear
algebraic system to be solved at each step:
\begin{equation}\label{e:DAE_nls}
  G(y_n) \equiv 
  F \left( t_n , \, y_n , \, 
    h_n^{-1} \sum_{i=0}^q \alpha_{n,i}y_{n-i} \right) = 0 \, .
\end{equation}
%
Regardless of the method options, the solution of the nonlinear system
(\ref{e:DAE_nls}) is accomplished with some form of Newton iteration.
This leads to a linear system for each Newton correction, of the form
\begin{equation}\label{e:DAE_Newtoncorr}
  J [y_{n(m+1)} - y_{n(m)}] = -G(y_{n(m)})  \, , 
\end{equation}
where $y_{n(m)}$ is the $m$-th approximation to $y_n$. 
%
Here $J$ is some approximation to the system Jacobian
\begin{equation}\label{e:DAE_Jacobian}
  J = \frac{\partial G}{\partial y}
  = \frac{\partial F}{\partial y} + 
  \alpha\frac{\partial F}{\partial {\dot y}} \, ,
\end{equation}
where $\alpha = \alpha_{n,0}/h_n$.  The scalar $\alpha$ changes 
whenever the stepsize or method order changes.
%
The linear systems are solved by one of three methods:
\begin{itemize}
\item direct dense solve (serial version only),
\item direct banded solve (serial version only), or
\item SPGMR = Scaled Preconditioned GMRES, with restarts allowed.
\end{itemize}
For the SPGMR case, preconditioning is allowed only on the
left\footnote{Left preconditioning is required to make the norm of
the linear residual in the Newton iteration meaningful; in general,
$\| J \Delta y + G \|$ is meaningless, since the weights used in 
the WRMS-norm correspond to $y$.}, 
so that GMRES is applied to systems $(P^{-1}J)\Delta y = -P^{-1}G$. 

% Newton iteration
%-----------------
In the cases of a direct linear solver (dense or banded), the nonlinear 
iteration (\ref{e:DAE_Newtoncorr}) is a Modified Newton iteration, in
that the Jacobian $J$ is fixed (and usually out of date), with
a coefficient $\bar\alpha$ in place of $\alpha$ in $J$. When using
SPGMR as the linear solver, the iteration is an Inexact Newton iteration,
using the current Jacobian (through matrix-free products $Jv$), in 
which the linear residual $J\Delta y + G$ is nonzero but controlled.
The Jacobian matrix $J$ (direct cases) or preconditioner matrix $P$ 
(SPGMR case) is updated when:
\begin{itemize}
\item starting the problem,
\item the value $\bar\alpha$ at the last update is such that
  $\alpha / {\bar\alpha} < 3/5$ or $\alpha / {\bar\alpha} > 5/3$, or
\item a non-fatal convergence failure occured with an out-of-date $J$ or $P$.
\end{itemize}
The above strategy balances the high cost of frequent matrix evaluations
and preprocessing with the slow convergence due to infrequent updates.
To reduce storage costs, on an update, Jacobian information is always
reevaluated from scratch.

% Newton convergence test
%------------------------
Unlike the CVODE/CVODES case, the stopping test for the Newton iteration
in IDA insures that the iteration error $y_n - y_{n(m)}$ is small relative
to $y$ itself. For this, we estimate the linear convergence rate at all 
iterations $m>1$ as
\begin{equation*}
R = \left( \frac{\delta_m}{\delta_1} \right)^{\frac{1}{m-1}} \, , 
\end{equation*}
where the $\delta_m = y_{n(m)} - y_{n(m-1)}$ is the correction at
iteration $m=1,2,\ldots$. The Newton iteration is halted if $R>0.9$.
The convergence test at the $m$-th iteration is then
\begin{equation}\label{e:DAE_nls_test}
S \| \delta_m \| < 0.33 \, ,
\end{equation}
where $S = R/(R-1)$ whenever $m>1$ and $R\le 0.9$. The user has the
option of changing the constant in the convergence test from its default 
value of $0.33$.
%
The quantity $S$ is set to $S=20$ initially and whenever $J$ or $P$ is
updated, and is reset to $S=100$ on a step with $\alpha \neq \bar\alpha$.
Note that at $m=1$, the convergence test (\ref{e:DAE_nls_test}) uses an old 
value for $S$. Therefore, at the first Newton iteration, we make an additional
test and stop the iteration if $\|\delta_1\| < 0.33 \cdot 10^{-4}$
(since such a $\delta_1$ is probably just noise and therefore not appropriate 
for use in evaluating $R$).
%
We allow only a small number (default value 4) of Newton iterations.
If convergence fails with $J$ or $P$ current, 
we are forced to reduce the stepsize $h_n$, and we replace $h_n$ by $h_n/4$.
The integration is halted after a preset number (default value 10)
of convergence failures. Both the maximum allowable Newton iterations
and the maximum nonlinear convergence failures can be changed by the user
from their default values.

When SPGMR is used to solve the linear system, to minimize the effect of
linear iteration errors on the nonlinear and local integration error controls,
we require the preconditioned linear residual to be small relative to the 
allowed error in the Newton iteration; i.e., 
$\| P^{-1}(Jx+G) \| < 0.05 \cdot 0.33$.
The safety factor $0.05$ can be changed by the user.

% Jacobian DQ approximations
%---------------------------
In the direct cases, the Jacobian $J$ defined in (\ref{e:DAE_Jacobian}) 
can be either supplied by the user or have IDA compute one internally 
by difference quotients. In the latter case, we use the approximation
\begin{gather*}
  J_{ij} = [F^i(t,y+\sigma_j e_j,{\dot y}+\alpha\sigma_j e_j) - 
  F^i(t,y,{\dot y})]/\sigma_j \, , \text{ with}\\
  \sigma_j = \sqrt{U} \max \left\{ 
    |y^j|, |h{\dot y}^j|,W_j 
  \right\} \mbox{sign}(h {\dot y}^j)
\end{gather*}
where $U$ is the unit roundoff, $h$ is the current stepsize, and $W_j$ is 
the error weight for $y^j$ set in terms of input tolerances. 
%
In the SPGMR case, if a routine for $Jv$ is not supplied, such products are
approximated with
\begin{equation*}
Jv = [F(t,y+\sigma v,{\dot y}+\alpha\sigma v) - F(t,y,{\dot y})]/\sigma \, ,
\end{equation*}
where the increment $\sigma$ is proportional to the square root of the problem 
dimension $N$ (the default value of $1.0$ for the proportionality constant can 
be changed by the user).

% Error control
%--------------
During the course of integrating the system, IDA computes an estimate
of the local truncation error LTE at the $n$-th time step, and
requires this to satisfy the inequality
\begin{equation*}
  \| \mbox{LTE} \|_{\mbox{\scriptsize WRMS}} \leq 1 \, .               
\end{equation*}
Asymptotically, LTE varies as $h^{q+1}$ at stepsize $h$ and order $q$, as
does the predictor-corrector difference $\Delta_n \equiv y_n-y_{n(0)}$.  
Thus there is a constant $C$ such that
\[ \mbox{LTE} = C \Delta_n + O(h^{q+2}) \, , \]
and so the norm of LTE is estimated as $|C| \cdot \|\Delta_n\|$.
In addition, IDA requires that the error in the associated polynomial
interpolant over the current step be bounded by 1 in norm.  The
leading term of the norm of this error is bounded by
$\bar{C} \|\Delta_n\|$ for another constant $\bar{C}$.  Thus the local
error test in IDA is
\begin{equation}\label{lerrtest}
   \max\{ |C|, \bar{C} \} \|\Delta_n\| \leq 1 \, .
\end{equation}
A user option is available by which the algebraic components of the
error vector are omitted from the test (\ref{lerrtest}), if these have
been so identified.

% Step/order selection
%---------------------
In IDA, the local error test is tightly coupled with the logic for
selecting the stepsize and order.  First, there is an initial phase
that is treated specially; for the first few steps, the stepsize is
doubled and the order raised (from its initial value of 1), until (a)
the local error test (\ref{lerrtest}) fails, (b) the order is reduced
(by the rules given below), or (c) the order reaches 5 (the maximum).
For step and order selection on the general step, IDA uses a different
set of local error estimates, based on the asymptotic behavior of the
local error in the case of fixed stepsizes.  At each of the orders
$q'$ equal to $q$, $q-1$ (if $q > 1$), $q-2$ (if $q > 2$), or $q+1$
(if $q < 5$), there are constants $C(q')$ such that the norm of the
local truncation error at order $q'$ satisfies
\[ \mbox{ELTE}(q') = C(q') \| \phi(q'+1) \| + O(h^{q'+2}) \, , \]
where $\phi(k)$ is a modified divided difference of order $k$ that is
retained by IDA (and behaves asymptotically as $h^k$).
While the ELTE values are used to select stepsizes, the choice of
order in IDA is based on the requirement that the scaled derivative
norms, $\|h^k y^{(k)}\|$, are monotonically decreasing with $k$ for
$k$ near $q$.  These norms are again estimated using the $\phi(k)$,
and in fact 
\[ \|h^{q'+1} y^{(q'+1)}\| \approx T(q') \equiv (q'+1) \mbox{ELTE}(q') \, . \]

The step/order selection begins with a test for monotonicity that is
made even {\em before} the local error test is performed.  Namely,
the order is reset to $q' = q-1$ if (a) $q > 2$ and
$\max\{T(q-1),T(q-2)\} \leq T(q)$, or (b) $q=2$ and $T(1)\leq T(2)/2$;
otherwise $q' = q$.  Next the local error test (\ref{lerrtest}) is
performed, and if it fails, the step is redone at order $q\leftarrow q'$
and a new stepsize $h'$.  The latter is based on the $h^{q+1}$
asymptotic behavior of $\mbox{ELTE}(q)$, and, with a safety factor, is
given by
\[ \eta = h'/h = 0.9/[2 \, \mbox{ELTE}(q)]^{1/(q+1)} \, . \]
The value of $\eta$ is adjusted so that $0.25 \leq \eta \leq 0.9$
before setting $h \leftarrow h' = \eta h$.  If the local error test
fails a second time, IDA uses $\eta = 0.25$, and on the third
and subsequent failures it uses $q = 1$ and $\eta = 0.25$.  After
10 failures, IDA returns with a give-up message.

As soon as the local error test has passed, the step and order for the
next step may be adjusted.  No such change is made if $q' = q-1$ from
the prior test, if $q = 5$, or if $q$ was increased on the previous
step.  Otherwise, if the last $q+1$ steps were taken at a constant
order $q < 5$ and a constant stepsize, IDA considers raising the order
to $q+1$.  The logic is as follows: If $q = 1$, then reset $q = 2$ if
$T(2) < T(1)/2$.  If $q > 1$ then 
\begin{itemize}
\item reset $q \leftarrow q-1$ if $T(q-1) \leq \min\{T(q),T(q+1)\}$;
\item else reset $q \leftarrow q+1$ if $T(q+1) < T(q)$;
\item leave $q$ unchanged if $T(q-1) > T(q) \leq T(q+1)$.
\end{itemize}
In any case, the new stepsize $h'$ is set much as before:
\[ \eta = h'/h = 1/[2 \, \mbox{ELTE}(q)]^{1/(q+1)} \, . \]
The value of $\eta$ is adjusted such that (a) if $\eta > 2$, $\eta$ is
reset to 2; (b) if $\eta \leq 1$, keep $0.5 \leq \eta \leq 0.9$; and
(c) if $1 < \eta < 2$ use $\eta = 1$.  Finally $h$ is reset to
$h' = \eta h$.  Thus we do not increase the stepsize unless it can
be doubled.  See \cite{BCP:96} for details.  

% Additional constraints on y components
%---------------------------------------
IDA permits the user to impose optional inequality constraints on individual 
components of the solution vector $y$. Any of the following four constraints 
can be imposed: $y^i > 0$, $y^i < 0$, $y^i \geq 0$, or $y^i \leq 0$. 
The constraint satisfaction is tested after a successful nonlinear system solution. 
If any constraint fails, we declare a convergence failure of the Newton iteration 
and reduce the stepsize. Rather than cutting the stepsize by some arbitrary factor, 
IDA estimates a new stepsize $h'$ using a linear approximation of the components 
in $y$ that failed the constraint test (including a safety factor of $0.9$ to 
cover the strict inequality case). These additional constraints are also imposed
during the calculation of consistent initial conditions.

Normally, IDA takes steps until a user-defined output value $t =
t_{\mbox{\scriptsize out}}$ is overtaken, and then computes
$y(t_{\mbox{\scriptsize out}})$ by interpolation.  However, a
``one-step'' mode option is available, where control returns to the
calling program after each step.  There are also options to force IDA
not to integrate past a given stopping point $t = t_{\mbox{\scriptsize
stop}}$.

