SUNDIALS is a suite of advanced computational codes for solving
large-scale problems that can be modeled as a system of nonlinear
algebraic equations, or as initial-value problems in ordinary
differential or differential-algebraic equations. The basic versions
of these codes are called KINSOL, CVODE, and IDA, respectively. The
codes are written in ANSI standard C and are suitable for either serial
or parallel machine environments.  Common and notable features of
these codes include: inexact Newton-Krylov methods for solving
large-scale nonlinear systems; linear multistep methods for
time-dependent problems; a highly modular structure to allow
incorporation of different preconditioning and/or linear solver
methods; and clear interfaces allowing for users to provide their 
own data structures underneath the solvers.  We
describe the current capabilities of the codes, along with some of the
algorithms and heuristics used to achieve efficiency and robustness.
We also describe how the codes stem from previous and widely used
Fortran 77 solvers, and how the codes have been augmented with forward
and adjoint methods for carrying out first-order sensitivity analysis
with respect to model parameters or initial conditions.
