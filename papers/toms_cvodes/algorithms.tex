\section{Algorithms}\label{s:algorithms}

CVODES solves initial value problems (IVP) for systems of ODEs. 
Such problems can be stated as
\begin{equation}\label{e:ivp}
\dot{y} = f(t,\,y) \, , \quad y(t_0) = y_0 \, ,
\end{equation}
where $y \in {\bf R}^N$ and $\dot{y}\,=dy/dt$.
That is, (\ref{e:ivp}) represents a system of $N$ ordinary
differential equations and their initial conditions at some $t_0$. The
dependent variable is $y$ and the independent variable is $t$. The
independent variable need not appear explicitly in the vector valued
function $f$.

Additionally, if (\ref{e:ivp}) depends (through its right-hand side and/or its initial
conditions) on some parameters $p \in {\bf R}^{N_p}$, i.e.
\begin{equation}\label{e:ivp_p}
\dot{y}  = f(t,\,y,\,p) \, , \quad y(t_0)  = y_0(p) \, ,
\end{equation}
CVODES can also compute first order derivative information, performing either
{\em forward sensitivity analysis} or {\em adjoint sensitivity analysis}.
In the first case, CVODES computes the sensitivities of the solution with respect to the 
parameters $p$, while in the second case, CVODES computes the gradient of a 
{\em derived function} with respect to the parameters $p$.

In the remaining of this section we describe the algorithms implemented in CVODES,
with emphasis on sensitivity analysis. In Section~\ref{ss:integration} we give only a 
brief overview of the ODE integration algorithm to introduce some of the
quantities needed in the sequel. 
Since CVODES shares the main integration engine with CVODE, the interested reader is
directed to~\cite{HBGLSSW:03}.

%-----------------------------------------------------------------------------------

\subsection{ODE Integration}\label{ss:integration}

The IVP is solved by one of two numerical methods. These are the
backward differentiation formula (BDF) and the Adams-Moulton formula. 
Both are implemented in a variable-stepsize, variable-order form. The BDF
uses a fixed-leading-coefficient form. These formulas can both be
represented by a linear multistep formula 
\begin{equation}\label{e:lmm}
\sum_{i=0}^{K_1}\alpha_{n,i}y_{n-i} + h_n\sum_{i=0}^{K_2}\beta_{n,i} 
\dot{y}_{n-i}=0
\end{equation}
where the $N$-vector $y_n$ is the computed approximation to $y(t_n)$,
the exact solution of (\ref{e:ivp}) at $t_n$. The stepsize is
$h_n=t_n-t_{n-1}$.  The coefficients $\alpha_{n,i}$ and $\beta_{n,i}$
are uniquely determined by the particular integration formula, the
history of the stepsize, and the normalization $\alpha_{n,0}=-1$. The
Adams-Moulton formula is recommended for nonstiff ODEs and is
represented by (\ref{e:lmm}) with $K_1=1$ and $K_2=q-1$. The order
of this formula is $q$ and its values range from 1 through 12. For
stiff ODEs, BDF should be selected and is represented by 
(\ref{e:lmm}) with $K_1=q$ and $K_2=0$. For BDF, the order $q$ may
take on values from 1 through 5. In the case of either formula, the
integration begins with $q=1$, and after that $q$ varies automatically
and dynamically.

For either BDF or the Adams formula, $\dot{y}_n$ denotes
$f(t_n,\,y_n)$. That is, (\ref{e:lmm}) is an implicit formula, and 
the nonlinear equation 
\begin{equation}\label{e:nonlinear}
  \begin{split}
    G(y_n) &\equiv  y_n-h_n\beta_{n,0}f(t_n,\,y_n) - a_n=0   \\
    a_n &= \sum_{i>0}(\alpha_{n,i}y_{n-i}+h_n\beta_{n,i}\dot{y}_{n-i}) 
  \end{split}
\end{equation}
must be solved for $y_{n}$ at each time step. For nonstiff problems,
a functional (or fixpoint) iteration is normally used which does not
require the solution of a linear system of equations. For stiff
problems, a Newton iteration is used and for each iteration an
underlying linear system must be solved. This linear system of
equations has the form
\begin{equation}\label{e:Newton}
M[y_{n(m+1)}-y_{n(m)}]=-G(y_{n(m)}) \, ,
\end{equation}
where $y_{n(m)}$ is the $m$th approximation to $y_n$, and $M$
approximates $\partial G/ \partial y$:
\begin{equation} \label{e:N_Matrix}
M \approx I-\gamma J, ~~~~ J = \frac{\partial f}{\partial y}, ~~~~
    \gamma = h_n\beta_{n,0} ~.
\end{equation}
At present, aside from a diagonal Jacobian approximation, the other
options implemented in CVODES for solving the linear systems
(\ref{e:Newton}) are:
(a) a direct method with dense treatment of the Jacobian,
(b) a direct method with band treatment of the Jacobian, and
(c) an iterative method SPGMR (scaled, preconditioned
GMRES) \cite{BrHi:89}, which is a Krylov subspace method. In most
cases, performance of SPGMR is improved by user-supplied
preconditioners. The user may precondition the system on the left, on
the right, on both the left and right, or use no preconditioner.
In most cases of interest to the CVODES user, the technique of
integration will involve BDF and the Newton method coupled with one of the 
linear solver modules.

The integrator computes an estimate $E_{n}$ of the local error at each time
step, and strives to satisfy the following inequality
\begin{equation*}
\left\| E_n\right\|_{WRMS} < 1 ~,
\end{equation*}
where $\|\cdot\|_{WRMS}$ is the weighted root-mean-square norm~\cite{BCP:96}
defined in terms of the user-defined relative and absolute tolerances. 
Since these tolerances define the allowed error per step, they should be 
chosen conservatively. Experience indicates that a conservative choice yields 
a more economical solution than error tolerances that are too large.
The error control mechanism in CVODES varies the stepsize and order
in an attempt to take minimum number of steps while satisfying the local
error test. 

CVODES also incorporates an algorithm for special treatment of
quadratures depending on the solution $y$ of the (\ref{e:ivp}) or
(\ref{e:ivp_p}). Evaluation of integrals of the form
$G = \int_{t_0}^{t_f} g(t,y,p) dt$ can be done efficiently using the
underlying linear multistep method interpolant polynomials by
appending to (\ref{e:ivp_p}) an additional ODE
\begin{equation}\label{e:quad_eqns}
\dot\phi = g(t,y,p) \, , \quad \phi(t_0) = 0 \, ,
\end{equation}
in which case $G = \phi(t_f)$. In the context of an implicit ODE
integrator, since the right-hand side of (\ref{e:quad_eqns}) does not
depend on $\phi$, such equations need not participate in the solution of
the nonlinear system~(\ref{e:nonlinear}). CVODES allows the user to
identify these equations separately from those in~(\ref{e:ivp_p}) and
provides the option of including or excluding $\phi$ from the error
control algorithm.
%
The main reason for including this option in CVODES was the need for
efficient quadrature computation in the context of adjoint sensitivity 
analysis (see Section~\ref{ss:adj_sensitivity}).

A complete description of the CVODES integration algortihm, including 
the nonlinear solver convergence, error control mechanism, and heuristics 
related to stopping criteria and finite-difference parameter selection, is
given in~\cite{HBGLSSW:03}.

%-----------------------------------------------------------------------------------

\subsection{Forward Sensitivity Analysis}\label{ss:fwd_sensitivity}

Typically, the governing equations of complex, large-scale models
depend on various parameters,  through the right-hand side vector 
and/or through the vector of initial conditions, as in (\ref{e:ivp_p}).
In addition to numerically solving the ODEs, it may be desirable to
determine the sensitivity of the results with respect to the model
parameters. 
Such sensitivity information can be used to estimate which
parameters are most influential in affecting the behavior of the
simulation or to evaluate optimization gradients (in the setting of dynamic
optimization, parameter estimation, optimal control, etc.).

The {\em solution sensitivity} with respect to the model parameter
$p_i$ is defined as the vector 
$s_i (t) = {\partial y(t)}/{\partial p_i}$
and satisfies the following {\em forward sensitivity equations}
(or in short {\em sensitivity equations}):
\begin{equation}\label{e:sens_eqns}
\dot{s_i}  = \frac{\partial f}{\partial y} s_i + \frac{\partial f}{\partial p_i} \, ,
\quad s_i(t_0)  = \frac{\partial y_{0}(p)}{\partial p_i} \, ,
\end{equation}
which are obtained by applying the chain rule of differentiation to the original
ODEs (\ref{e:ivp_p}). 

When performing forward sensitivity analysis, CVODES carries out the time integration 
of the combined system, (\ref{e:ivp_p}) and (\ref{e:sens_eqns}), by viewing it as an ODE
system of size $N(N_s+1)$, where $N_s$ represents a subset of model parameters $p_i$, 
with respect to which sensitivities are desired ($N_s \le N_p$). 
However, major efficiency improvements can be obtained by taking advantage of the special 
form of the sensitivity equations as linearizations of the original ODEs. 
In particular, for stiff systems, in which case CVODES employs a Newton iteration, 
the original ODE system and all sensitivity systems share the same Jacobian matrix, 
and therefore the same iteration matrix $M$ in (\ref{e:N_Matrix}).

The sensitivity equations are solved with the same linear multistep formula that
was selected for the original ODEs and, if Newton iteration was selected, the
same linear solver is used in the correction phase for both state and sensitivity 
variables. In addition, CVODES offers the option of including
({\em full error control}) or excluding
({\em partial error control}) the sensitivity variables from the local 
error test.

\subsubsection{Forward sensitivity methods}
In what follows we briefly describe three methods that have been proposed for the 
solution of the combined ODE and sensitivity system for the vector
${\hat y} = [y, s_1, \ldots , s_{N_s}]$.
Due to its inefficiency, especially for large-scale problems, the first approach 
is not implemented in CVODES.

\begin{itemize}

\item[{\em Staggered Direct.}]
  In this approach \cite{CaSt:85}, the nonlinear system (\ref{e:nonlinear}) is first 
  solved and, once an acceptable numerical solution is obtained, the sensitivity 
  variables at the new step are found by directly solving (\ref{e:sens_eqns}) 
  after the BDF discretization is used to eliminate ${\dot s}_i$. 
  Although the system matrix of the above linear system is based on the exact same 
  information as the matrix $M$ in (\ref{e:N_Matrix}), it must be updated and factored 
  at every step of the integration as $M$ is updated only ocasionally. 
  The computational cost associated with these matrix updates and factrorizations 
  makes this method unattractive when compared with the methods described below and 
  is therefore not implemented in CVODES.
  
\item[{\em Simultaneous Corrector.}] 
  In this method \cite{MaPe:97}, the BDF discretization is applied simultaneously
  to both the original equations (\ref{e:ivp_p}) and the sensitivity systems
  (\ref{e:sens_eqns}) resulting in the following nonlinear system 
  \begin{equation*}
    {\hat G}({\hat y}_n) \equiv  
    {\hat y}_n - h_n\beta_{n,0} {\hat f}(t_n,\,{\hat y}_n) - {\hat a}_n = 0 \, ,
  \end{equation*}
  where
  ${\hat f} = [ f(t,y,p), \ldots , (\dfdyI)(t,y,p) s_i + (\dfdpiI)(t,y,p) , \ldots ]$
  and ${\hat a}_n$ are the terms in the BDF discretization that depend on the
  solution at previous integration steps.
  This combined nonlinear system can be solved as in (\ref{e:Newton}) using
  a modified Newton method by solving the corrector equation
  \begin{equation}\label{e:Newton_sim}
    {\hat M}[{\hat y}_{n(m+1)}-{\hat y}_{n(m)}]=-{\hat G}({\hat y}_{n(m)})
  \end{equation}
  at each iteration, where 
  \begin{equation*}
    {\hat M} = 
    \begin{bmatrix}
      M              &        &        &        &   \\
      \gamma J_1     & M      &        &        &   \\
      \gamma J_2     & 0      & M      &        &   \\
      \vdots         & \vdots & \ddots & \ddots &   \\
      \gamma J_{N_s} & 0      & \ldots & 0      & M 
    \end{bmatrix} \, ,
  \end{equation*}
  $M$ is defined as in (\ref{e:N_Matrix}), and 
  $J_i = ({\partial}/{\partial y})\left[ (\dfdyI) s_i + (\dfdpiI) \right]$.
  It can be shown that a 2-step quadratic convergence can be attained by only
  using the block-diagonal portion of ${\hat M}$ in the corrector equation
  (\ref{e:Newton_sim}). This results in a decoupling that allows the reuse of 
  $M$ without additional matrix factorizations. However, the products
  $(\dfdyI)s_i$ as well as the vectors $\dfdpiI$ must still be reevaluated at 
  each step of the iterative process (\ref{e:Newton_sim}) to update the 
  sensitivity portions of the residual ${\hat G}$.
  
\item[{\em Staggered corrector.}] In this approach \cite{FTB:97}, as in the staggered direct method,
  the nonlinear system (\ref{e:nonlinear}) is solved first using the Newton iteration
  (\ref{e:Newton}). Then, a separate Newton iteration is used to solve the
  sensitivity system (\ref{e:sens_eqns}):
  \begin{multline}\label{e:stgr_iterations}
    M [s_{i , n(m+1)} - s_{i , n(m)}]= \\
    s_{i, n(m)} - 
    \gamma \left( \dfdy (t_n , y_n, p) s_{i , n(m)} + \dfdpi (t_n , y_n , p) \right)
    -a_{i,n} \, ,
  \end{multline}
  where $a_{i,n} = \sum_{j>0}(\alpha_{n,j}s_{i , n-j}+h_n\beta_{n,j}\dot{s}_{i , n-j})$.
  In other words, a modified-Newton iteration is used to solve a linear system.
  In this approach, the vectors $\dfdpiI$ need be updated only once per integration step, 
  after the state correction phase (\ref{e:Newton}) has converged. Note also that 
  Jacobian-related data can be reused at all iterations (\ref{e:stgr_iterations})
  to evaluate the products $(\dfdyI) s_i$.
\end{itemize}  

CVODES implements the simultaneous corrector method and two flavors of the 
staggered corrector method which differ only if the sensitivity variables are
included in the error control test.
In the {\em full error control} case, 
the first variant of the staggered corrector method requires the convergence of 
the iterations (\ref{e:stgr_iterations}) for all $N_s$ sensitivity sytems and then 
performs the error test on the sensitivity variables. The second variant of the method
will perform the error test for each sensitivity vector $s_i,\,i=1,2,\ldots,N_s$
individually, as they pass the convergence test. Differences in performance
between the two variants may therefore be noticed whenever one of the sensitivity 
vectors $s_i$ fails a convergence or error test. 

An important observation is that the staggered corrector method, combined with 
the SPGMR linear solver effectively results in a staggered direct method. 
Indeed, SPGMR requires only the action of the matrix $M$ on a vector and
this can be provided with the current Jacobian information. Therefore, the
modified Newton procedure (\ref{e:stgr_iterations}) will theoretically converge 
after one iteration.

\subsubsection{Selection of the absolute tolerances for sensitivity variables}
If the sensitivities are considered in the error test, CVODES provides an 
automated estimation of absolute tolerances for the sensitivity variables 
based on the absolute tolerance for the corresponding state variable.
The relative tolerance for sensitivity variables is set to be the same as for 
the state variables. The selection of absolute tolerances for the sensitivity 
variables is based on the observation that the sensitivity vector $s_i$ will have 
units of $[y]/[p_i]$.
With this, the absolute tolerance for the $j$-th component of the sensitivity
vector $s_i$ is set to ${atol_j}/{|{\bar p}_i|}$,
where $atol$ are the absolute tolerances for the state variables and $\bar p$
is a vector of scaling factors that are dimensionally consistent with
the model parameters $p$ and give indication of their order of magnitude.
This choice of relative and absolute tolerances is equivalent 
to requiring that the weighted root-mean-square norm of the sensitivity 
vector $s_i$ with weights based on $s_i$ is the same as the
weighted root-mean-square norm of the vector of scaled sensitivities 
${\bar s}_i = |{\bar p}_i| s_i$ with weights based on the state variables
(the scaled sensitivities ${\bar s}_i$ being dimensionally consistent with the
state variables).

\subsubsection{Evaluation of the sensitivity right-hand side}
There are several methods for evaluating the right-hand side of the sensitivity 
systems (\ref{e:sens_eqns}): analytic evaluation, automatic differentiation, 
complex-step approximation, finite differences (or directional derivatives).
CVODES provides all the software hooks for implementing interfaces to
automatic differentiation or complex-step approximation and future versions
will provide these capabilities.
At the present time, besides the option for analytical sensitivity right hand 
sides (user-provided), CVODES can evaluate these quantities using various
finite difference-based approximations to evaluate the terms $(\dfdyI) s_i$ 
and $(\dfdpiI)$, or using directional derivatives to evaluate
$\left[ (\dfdyI) s_i + (\dfdpiI) \right]$.
As is typical for finite differences, the proper choice of perturbations is a 
delicate matter. CVODES takes into account several problem-related features;
the relative ODE error tolerance $rtol$, the machine unit roundoff $U$,
the scale factor ${\bar p}_i$, and the weighted root-mean-square norm of the 
sensitivity vector $s_i$.

Using central finite differences asd an example, the two terms 
$({\partial f}/{\partial y}) s_i$ 
and ${\partial f}/{\partial p_i}$ in the right-hand side of (\ref{e:sens_eqns}) 
can be evaluated separately:
\begin{gather}
  \frac{\partial f}{\partial y} s_i \approx \frac{f(t, y+\sigma_y s_i, p)-
    f(t, y-\sigma_y s_i, p)}{2\,\sigma_y} \, , \label{e:fd2} \\
  \frac{\partial f}{\partial p_i} \approx \frac{f(t,y,p + \sigma_i e_i)-
    f(t,y,p - \sigma_i e_i)}{2\,\sigma_i} \, , \tag{\ref{e:fd2}'} \\
  \sigma_i = |{\bar p}_i| \sqrt{\max(rtol, U)} \, , \quad
  \sigma_y = \frac{1}{\max(1/\sigma_i, \|s_i\|_{WRMS}/|{\bar p}_i|)} \, , \nonumber
\end{gather}
simultaneously:
\begin{gather}
  \frac{\partial f}{\partial y} s_i + \frac{\partial f}{\partial p_i} \approx
  \frac{f(t, y+\sigma s_i, p + \sigma e_i) -
    f(t, y-\sigma s_i, p - \sigma e_i)}{2\,\sigma} \, , \label{e:dd2} \\
  \sigma = \min(\sigma_i, \sigma_y) \, , \nonumber
\end{gather}
or adaptively switching between (\ref{e:fd2})+(\ref{e:fd2}') and (\ref{e:dd2}), 
depending on the relative size of the estimated finite difference 
increments $\sigma_i$ and $\sigma_y$.

%-----------------------------------------------------------------------------------

\subsection{Adjoint Sensitivity Analysis}\label{ss:adj_sensitivity}

In the {\em forward sensitivity approach} described in the previous
section, obtaining sensitivities with respect to $N_s$ parameters is roughly
equivalent to solving an ODE system of size $(1+N_s) N$. This can become 
prohibitively expensive, especially for large-scale problems, if sensitivities
with respect to many parameters are desired.
In this situation, the {\em adjoint sensitivity method} is a very
attractive alternative, provided that we do not need the solution sensitivities
$s_i$, but rather the gradients with respect to model parameters of a relatively 
few derived functionals of the solution. In other words, if $y(t)$ is the solution
of (\ref{e:ivp_p}), we wish to evaluate the gradient ${dG}/{dp}$ of
\begin{equation}\label{e:G}
G(p) = \int_{t_0}^{t_f} g(t, y, p) dt \, ,
\end{equation}
or, alternatively, the gradient ${dg}/{dp}$ of the function $g(t, x, p)$ 
at time $t_f$. The function $g$ must be smooth enough that $\partial g / \partial y$ 
and $\partial g / partial p$ exist and are bounded. 
In what follows, we only provide the final results for the gradients of both $G$ and 
$g(t_f)$. For details on the derivation see~\cite{CLPS:03}.
%
The gradient of $G$ with respect to $p$ is nothing but
\begin{equation}\label{e:dGdp}
  \frac{dG}{dp} = \lambda^T(t_0) s(t_0) + 
  \int_{t_0}^{t_f} \left( \frac{\partial g}{\partial p} + 
    \lambda^T \frac{\partial f}{\partial p} \right) dt,
\end{equation}
where $\lambda$ is solution of
\begin{equation}\label{e:adj_eqns}
{\dot \lambda} = -\left( \dfdy \right)^T \lambda - 
\left( \frac{\partial g}{\partial y} \right)^T \, ,
\quad \lambda(t_f) = 0
\end{equation}
and $s(t_0) = dy_0/dp$.
%
The gradient of $g(t_f,y,p)$ with respect to $p$ can be then obtained
by using the Leibnitz differentiation rule. Indeed, from (\ref{e:G}),
$({dg}/{dp})(t_f) = {d}/{dt_f}({dG}/{dp})$
and therefore, taking into account that $dG/dp$ in (\ref{e:dGdp}) depends on $t_f$
both through the upper integration limit and through $\lambda$ and that $\lambda(t_f) = 0$, 
\begin{equation}\label{e:dgdp}
  \frac{dg}{dp}(t_f) = 
  \frac{\partial g}{\partial p}(t_f) +
  \mu^T(t_0) s(t_0) + 
  \int_{t_0}^{t_f} \mu^T \frac{\partial f}{\partial p} dt \, ,
\end{equation}
where $\mu$ is the sensitivity of $\lambda$ with respect to the final integration 
limit and thus satisfies the following equation, obtained by taking the total derivative
with respect to $t_f$ of (\ref{e:adj_eqns}):
\begin{equation}\label{e:adj1_eqns}
{\dot \mu} = -\left( \dfdy \right)^T \mu \, ,
\quad \mu(t_f) = \left( \frac{\partial g}{\partial y}(t_f) \right)^T \, .
\end{equation}
The final condition on $\mu(t_f)$ follows from 
$(\partial\lambda/\partial t) + (\partial\lambda/\partial t_f) = 0$ at $t_f$, and
therefore, $\mu(t_f) = -{\dot\lambda}(t_f)$. 

The first thing to notice about the adjoint system (\ref{e:adj_eqns}) is that there is 
no explicit specification of the parameters $p$; this implies that, once the solution
$\lambda$ is found, the formula (\ref{e:dGdp}) can then be used to find the gradient
of $G$ with respect to any of the parameters $p$. The same holds true for the system
(\ref{e:adj1_eqns}) and the formula (\ref{e:dgdp}) for gradients of $g(t_f,y,p)$. 
The second important remark is that the adjoint systems are terminal value problems 
which depend on the solution $y(t)$ of the original IVP (\ref{e:ivp_p}). 
Therefore, a procedure is needed for providing the states $y$ obtained during a forward 
integration phase of (\ref{e:ivp_p}) to CVODES during the backward integration phase 
of (\ref{e:adj_eqns}) or (\ref{e:adj1_eqns}). 
The approach adopted in CVODES, based on {\em check-pointing} is described next.

During the backward integration, the evaluation of the right hand side 
of the adjoint system requires, at the current time, the states $y$ which
were computed in the forward integration phase.
Since CVODES implements variable-stepsize integration formulas,
it is unlikely that the states will be available at the desired time and
therefore some form of interpolation is needed. The CVODES implementation
being also variable-order, it is possible that during the forward
integration phase the order may be reduced as low as 1st order,
which means that there may be points in time where only $y$ and ${\dot y}$
are available. Therefore, CVODES employs a cubic Hermite interpolation
algorithm. However, especially for large-scale problems and long integration
intervals, the number and size of the vectors $y$ and ${\dot y}$ that would 
need to be stored make this approach computationally intractable. 

CVODES settles for a compromise between storage space and execution time by
implementing a so-called {\em check-pointing scheme}. At the cost of
at most one additional forward integration, this approach offers the best possible 
estimate of memory requirements for adjoint sensitivity analysis. To begin with,
based on the problem size $N$ and the available memory, the user decides on 
the number $N_d$ of data pairs $y$-${\dot y}$ that can be kept in memory for 
the purpose of interpolation. Then, during the first forward integration stage, 
every $N_d$ integration steps a check point is formed by saving enough information
(either in memory or on disk if needed) to allow for a hot restart, that is a restart
which will exactly reproduce the forward integration. In order to avoid storing
Jacobian-related data at each check point, a reevaluation of the iteration matrix
is forced before each check point. At the end of this stage, we are left with $N_c$ 
check points, including one at $t_0$.
During the backward integration stage, the adjoint variables are integrated
from $t_f$ to $t_0$ going from one check point to the previous one.
The backward integration from check point $i+1$ to check point $i$ is preceeded
by a forward integration from $i$ to $i+1$ during which $N_d$ data pairs 
$y$-${\dot y}$ are generated and stored in memory for interpolation.
%
This procedure is illustrated in Fig.~\ref{f:ckpnt}.
%
\begin{figure}
\centerline{\psfig{figure=ckpnt.eps,width=4in}}
\caption {Illustration of the check-pointing algorithm for generation of 
  the forward solution during the integration of the adjoint system.}
\label{f:ckpnt}
\end{figure}

This approach transfers the uncertainty in the number of integration
steps in the forward integration phase to uncertainty in the final number of check 
points. However, $N_c$ is much smaller than the number of steps taken during
the forward integration and there is no major penalty for writting and then reading
check point data to/from a temporary file.
%
Note that, at the end of the first forward integration stage, data pairs 
$y$-${\dot y}$ are available from the last check point to the end of the integration 
interval. If no check points are necessary, i.e $N_d$ is larger than the 
number of integration steps taken in the solution of (\ref{e:ivp_p}),
the total cost of an adjoint sensitivity computation can be as low as one forward
plus one backward integration.
%
In addition, CVODES provides the capability of reusing a set of check points
for multiple backward integrations, thus allowing for efficient computation of
gradients of several functionals (\ref{e:G}).

Finally, we note that the adjoint sensitivity module in CVODES provides the 
infrastructure to integrate backwards in time any ODE terminal value problem
dependent on the solution of the IVP (\ref{e:ivp_p}), including
adjoint systems (\ref{e:adj_eqns}) or (\ref{e:adj1_eqns}), as well as any other
quadrature ODEs that may be needed in evaluating the integrals in (\ref{e:dGdp}) 
or (\ref{e:dgdp}). In particular, for ODE systems arising from semi-discretization
of time-dependent PDEs, this feature allows for integration of either the 
discretized adjoint PDE system or the adjoint of the discretized PDE.
