% This is a shared SUNDIALS TEX file with description of
% the generic band linear solver
%
\subsection{Type BandMat}
\index{generic linear solvers!BAND@{\band}}
\index{BAND@{\band} generic linear solver!type \id{BandMat}|(}
The type \ID{BandMat} is the type of a large band matrix A (possibly distributed). 
It is defined to be a pointer to a structure defined by:
\begin{verbatim}
typedef struct {
  long int size;
  long int mu, ml, smu;
  realtype **data;
} *BandMat;
\end{verbatim}
The fields in the above structure are:
\begin{itemize}

\item {\em size} is the number of columns (which is the same as the number of  rows);
                                                                  
\item {\em mu} is the upper half-bandwidth, $0 \le$ {\em mu} $\le$ {\em size}$-1$;

\item {\em ml} is the lower half-bandwidth, $0 \le$ {\em ml} $\le$ {\em size}$-1$;

\item {\em smu} is the storage upper half-bandwidth, 
  {\em mu} $\le$ {\em smu} $\le$ {\em size}$-1$.      
  The \id{BandFactor} routine writes the LU factors           
  into the storage for A. The upper triangular factor U, 
  however, may have an upper half-bandwidth as big as         
  $\min(${\em size}$-1$,{\em mu}$+${\em ml}$)$
  because of partial pivoting. The {\em smu} field holds the upper 
  half-bandwidth allocated for A.       
  
\item {\em data} is a two dimensional array used for component storage.    
  The elements of a band matrix of type \id{BandMat} are      
  stored columnwise (i.e. columns are stored one on top  
  of the other in memory). Only elements within the      
  specified half-bandwidths are stored.                       
                                                                 
  If we number rows and columns in the band matrix starting      
  from $0$, then                                                   
  \begin{itemize}
  \item {\em data[0]} is a pointer to 
    ({\em smu}+{\em ml}+1)*{\em size} contiguous locations   
    which hold the elements within the band of A        
    
  \item {\em data[j]} is a pointer to the uppermost element within the band  
    in the j-th column. This pointer may be treated as   
    an array indexed from {\em smu}$-${\em mu} (to access the         
    uppermost element within the band in the j-th        
    column) to {\em smu}$+${\em ml} (to access the lowest element     
    within the band in the j-th column). Indices from $0$ 
    to {\em smu}$-${\em mu}$-1$ give access to extra storage elements   
    required by \id{BandFactor}.                          
    
  \item {\em data[j][i-j+smu]} is the $(i,j)$-th element, 
    $j-${\em mu} $\le i \le j+${\em ml}.    
  \end{itemize}

\end{itemize}               
                                                  
The macros below allow a user to access individual matrix      
elements without writing out explicit data structure           
references and without knowing too much about the underlying   
element storage. The only storage assumption needed is that    
elements are stored columnwise and that a pointer into the \id{j}-th 
column of elements can be obtained via the \id{BAND\_COL} macro.
Users should use these macros whenever possible.                                      

See Figure \ref{f:bandmat} for a diagram of the \id{BandMat} type.

\begin{figure}
\centerline{\psfig{figure=bandmat.eps,width=4.5 in}}
\caption[Diagram of the storage for a matrix of type \id{BandMat}]
  {Diagram of the storage for a band matrix of type \id{BandMat}. Here \id{A} is an
  $N \times N$ band matrix of type \id{BandMat} with upper and lower half-bandwidths \id{mu}
  and \id{ml}, respectively. The rows and columns of \id{A} are numbered from $0$ to $N-1$
  and the ($i,j$)-th element of \id{A} is denoted \id{A(i,j)}. The greyed out areas of
  the underlying component storage are used by the \id{BandFactor} and
  \id{BandBacksolve} routines.}\label{f:bandmat}
\end{figure}

 
\index{BAND@{\band} generic linear solver!type \id{BandMat}|)}

\subsection{Accessor Macros}
\index{BAND@{\band} generic linear solver!macros|(}
The following three macros are defined by the {\band} module to provide
access to data in the \id{BandMat} type:
\begin{itemize}
\item \ID{BAND\_ELEM}
  \par Usage : \id{BAND\_ELEM(A,i,j) = a\_ij;} or \id{a\_ij = BAND\_ELEM(A,i,j);}
  \par \id{BAND\_ELEM} references the (\id{i},\id{j})-th element of the
  $N \times N$ band matrix \id{A}, where $0 \le$ \id{i}, \id{j} $\le N-1$.
  The location (\id{i},\id{j}) should further satisfy 
  \id{j}$-$\id{(A->mu)} $\le$ \id{i} $\le$ \id{j}$+$\id{(A->ml)}.
\item \ID{BAND\_COL}
  \par Usage : \id{col\_j = BAND\_COL(A,j);}
  \par \id{BAND\_COL} references the diagonal element of the \id{j}-th
  column of the $N \times N$ band matrix \id{A}, $0 \le$ \id{j} $\le N-1$.
  The type of the expression \id{BAND\_COL(A,j)} is \id{realtype *}. 
  The pointer returned by the call \id{BAND\_COL(A,j)} can be treated as 
  an array which is indexed from $-$\id{(A->mu)} to \id{(A->ml)}.
\item \ID{BAND\_COL\_ELEM}
  \par Usage : \id{BAND\_COL\_ELEM(col\_j,i,j) = a\_ij;} or
  \id{a\_ij = BAND\_COL\_ELEM(col\_j,i,j);}
  \par This macro references the (\id{i},\id{j})-th entry of the band matrix \id{A}
  when used in conjunction with \id{BAND\_COL} to reference the \id{j}-th column through
  \id{col\_j}. The index (\id{i},\id{j}) should satisfy 
  \id{j}$-$\id{(A->mu)} $\le$ \id{i} $\le$ \id{j}$+$\id{(A->ml)}.
\end{itemize}
\index{BAND@{\band} generic linear solver!macros|)}

\subsection{Functions}
\index{BAND@{\band} generic linear solver!functions|(}
The following functions for \id{BandMat} matrices are available
in the {\band} package.  For full details, see the header file \id{sundials\_band.h}.
\begin{itemize}
\item \id{BandAllocMat}: allocation of a \id{BandMat} matrix;
\item \id{BandAllocPiv}: allocation of a pivot array for use
      with \id{BandFactor}/\id{BandBacksolve};
\item \id{BandFactor}: LU factorization with partial pivoting;
\item \id{BandBacksolve}: solution of $Ax = b$ using LU factorization;
\item \id{BandZero}: load a matrix with zeros;
\item \id{BandCopy}: copy one matrix to another;
\item \id{BandScale}: scale a matrix by a scalar;
\item \id{BandAddI}: increment a matrix by the identity matrix;
\item \id{BandFreeMat}: free memory for a \id{BandMat} matrix;
\item \id{BandFreePiv}: free memory for a pivot array;
\item \id{BandPrint}: print a \id{BandMat} matrix to standard output.
\end{itemize}
\index{BAND@{\band} generic linear solver!functions|)}
