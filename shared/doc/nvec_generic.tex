% This is a shared SUNDIALS TEX file with description of
% the generic nvector abstraction
%
The {\sundials} solvers are written in a data-independent manner. 
They all operate on generic vectors (of type \Id{N\_Vector}) through a set of
operations defined by the particular {\nvector} implementation.


The generic \ID{N\_Vector} type is a pointer to a structure that has an 
implementation-dependent {\em content} field containing the 
description and actual data of the vector and an {\em ops} field 
pointing to a structure with generic vector operations.
The type \id{N\_Vector} is defined as
%%
%%
\begin{verbatim}
typedef struct _generic_N_Vector *N_Vector;

struct _generic_N_Vector {
    void *content;
    struct _generic_N_Vector_Ops *ops;
};
\end{verbatim}
%%
%%
The \id{\_generic\_N\_Vector\_Ops} structure is essentially a list of pointers to
the various actual vector operations, and is defined as
%%
\begin{verbatim}
struct _generic_N_Vector_Ops {
  N_Vector    (*nvclone)(N_Vector);
  void        (*nvdestroy)(N_Vector);
  void        (*nvspace)(N_Vector, long int *, long int *);
  realtype*   (*nvgetarraypointer)(N_Vector);
  void        (*nvsetarraypointer)(realtype *, N_Vector);
  void        (*nvlinearsum)(realtype, N_Vector, realtype, N_Vector, N_Vector); 
  void        (*nvconst)(realtype, N_Vector);
  void        (*nvprod)(N_Vector, N_Vector, N_Vector);
  void        (*nvdiv)(N_Vector, N_Vector, N_Vector);
  void        (*nvscale)(realtype, N_Vector, N_Vector);
  void        (*nvabs)(N_Vector, N_Vector);
  void        (*nvinv)(N_Vector, N_Vector);
  void        (*nvaddconst)(N_Vector, realtype, N_Vector);
  realtype    (*nvdotprod)(N_Vector, N_Vector);
  realtype    (*nvmaxnorm)(N_Vector);
  realtype    (*nvwrmsnorm)(N_Vector, N_Vector);
  realtype    (*nvwrmsnormmask)(N_Vector, N_Vector, N_Vector);
  realtype    (*nvmin)(N_Vector);
  realtype    (*nvwl2norm)(N_Vector, N_Vector);
  realtype    (*nvl1norm)(N_Vector);
  void        (*nvcompare)(realtype, N_Vector, N_Vector);
  booleantype (*nvinvtest)(N_Vector, N_Vector);
  booleantype (*nvconstrmask)(N_Vector, N_Vector, N_Vector);
  realtype    (*nvminquotient)(N_Vector, N_Vector);
};
\end{verbatim}




The generic {\nvector} module also defines and implements the vector operations 
acting on \id{N\_Vector}.
These routines are nothing but wrappers around the vector operations defined by
a particular {\nvector} implementation, which are accessed through the {\em ops}
field of the \id{N\_Vector} structure. To illustrate this point we
show below the implementation of a typical vector operation from the
generic {\nvector} module, namely \id{N\_VScale}, which performs the scaling of a
vector \id{x} by a scalar \id{c}:
%%
%%
\begin{verbatim}
void N_VScale(realtype c, N_Vector x, N_Vector z) 
{
   z->ops->nvscale(c, x, z);
}
\end{verbatim}
%%
%%
Table \ref{t:nvecops} contains a complete list of all vector operations defined
by the generic {\nvector} module.



Finally, note that the generic {\nvector} module defines a function \ID{N\_VCloneVectorArray}
which creates (by cloning) an array of \id{count} \id{N\_Vector} of the same type as an
existing \id{N\_Vector}. Its prototype is
\begin{verbatim}
N_Vector *N_VCloneVectorArray(int count, N_Vector w);
\end{verbatim}
and its definition is based on the implementation-specific \id{N\_VClone} operation.
Such an array \id{vs} of \id{N\_Vector} can be destroyed by calling \ID{N\_VDestroyVectorArray}
whose prototype is
\begin{verbatim}
void N_VDestroyVectorArray(N_Vector *vs, int count);
\end{verbatim}
and whose definition is based on the implementation-specific \id{N\_VDestroy} operation.




A particular implementation of the {\nvector} module must:
\begin{itemize}
\item Specify the {\em content} field of \id{N\_Vector}.
\item Define and implement the vector operations. 
  Note that the names of these routines should be unique to that implementation in order 
  to permit using \id{N\_Vector}'s with different 
  internal representations in the same code.
\item Define and implement user-callable constructor and destructor
  routines to create and free an \id{N\_Vector} with
  the new {\em content} field and with {\em ops} pointing to the
  new vector kernels.
\item Optionally, define and implement additional user-callable routines
  acting on the newly defined \id{N\_Vector} (e.g., a routne to print
  the content for debugging purposes).
\item Optionally, provide accessor macros as needed for that particular implementation to 
  be used to access different parts in the {\em content} field of the newly defined \id{N\_Vector}.
\end{itemize}



%---------------------------------------------------------------------------
% Table of vector kernels
%---------------------------------------------------------------------------
\newpage

\newlength{\colone}
\settowidth{\colone}{\id{N\_VGetArrayPointer}}
\newlength{\coltwo}
\setlength{\coltwo}{\textwidth}
\addtolength{\coltwo}{-0.5in}
\addtolength{\coltwo}{-\colone}

\tablecaption{Description of the NVECTOR kernels}\label{t:nvecops}
\tablefirsthead{\hline {\bf Name} & {\bf Usage and Description} \\ \hline\hline}
\tablehead{\hline \multicolumn{2}{|l|}{\small\slshape continued from last page} \\
           \hline {\bf Name} & {\bf Usage and  Description} \\ \hline\hline}
\tabletail{\hline \multicolumn{2}{|r|}{\small\slshape continued on next page} \\ \hline}
\tablelasttail{\hline}
\begin{supertabular}{|p{\colone}|p{\coltwo}|}
&\\
\id{N\_VClone} & \id{v = N\_Vclone(w);} \\ 
& Creates a new \id{N\_Vector} of the same type as an existing vector and sets the
{\em ops} field.
It does not copy the vector, but rather allocates storage for the new vector.
\\
&\\
\id{N\_VDestroy} & \id{N\_VDestroy(v);} \\
& Destroys the \id{N\_Vector} \id{v} and frees memory allocated for its
internal data.
\\
&\\
\id{N\_VSpace} & \id{N\_VSpace(nvSpec, \&lrw, \&liw);} \\
& Returns storage requirements for one \id{N\_Vector}.
\id{lrw} contains the number of realtype words and \id{liw}
contains the number of integer words.
\\
&\\
\id{N\_VGetArrayPointer} & \id{vdata = N\_VGetArrayPointer(v);} \\
& Returns a pointer to a \id{realtype} array from the \id{N\_Vector} \id{v}.
Note that this assumes that the internal data in \id{N\_Vector} is
a contiguous array of \id{realtype}.
This routine is only used in the solver-specific interfaces to the 
dense and banded linear solvers, as well as the interfaces to 
the banded preconditioners provided with {\sundials}.
\\
&\\
\id{N\_VSetArrayPointer} & \id{N\_VSetArrayPointer(vdata, v);} \\
& Overwrites the data in an \id{N\_Vector} with a given array to \id{realtype}.
Note that this assumes that the internal data in \id{N\_Vector} is
a contiguous array of \id{realtype}.
This routine is only used in the interfaces to the dense linear
solver.
\\
&\\
\id{N\_VLinearSum} & \id{N\_VLinearSum(a, x, b, y, z);} \\
& Performs the operation $z = a x + b y$, where $a$ and $b$ are scalars
and $x$ and $y$ are \id{N\_Vector}'s:
$z_i = a x_i + b y_i, \: i=0,\ldots,n-1$.
\\
&\\
\id{N\_VConst} & \id{N\_VConst(c, z);} \\
& Sets all components of the \id{N\_Vector} \id{z} to \id{c}:
$z_i = c,\: i=0,\ldots,n-1$.
\\
&\\
\id{N\_VProd} & \id{N\_VProd(x, y, z);} \\
& Sets the \id{N\_Vector} \id{z} to be the component-wise product of the
\id{N\_Vector}'s \id{x} and \id{y}:
$z_i = x_i y_i,\: i=0,\ldots,n-1$.
\\
&\\
\id{N\_VDiv} & \id{N\_VDiv(x, y, z);} \\
& Sets the \id{N\_Vector} \id{z} to be the component-wise ratio of the
\id{N\_Vector}'s \id{x} and \id{y}:
$z_i = x_i / y_i,\: i=0,\ldots,n-1$. The $y_i$ may not be tested 
for $0$ values. It should only be called with an \id{x} that is
guaranteed to have all nonzero components.
\\
&\\
\id{N\_VScale} & \id{N\_VScale(c, x, z);} \\
& Scales the \id{N\_Vector} \id{x} by the scalar \id{c} and returns
the result in \id{z}:
$z_i = c x_i , \: i=0,\ldots,n-1$.
\\
&\\
\id{N\_VAbs} & \id{N\_VAbs(x, y);} \\
& Sets the components of the \id{N\_Vector} \id{y} to be the absolute
values of the components of the \id{N\_Vector} \id{x}:
$y_i = | x_i | , \: i=0,\ldots,n-1$.
\\
&\\
\id{N\_VInv} & \id{N\_VInv(x, z);} \\
& Sets the components of the \id{N\_Vector} \id{z} to be the inverses
of the components of the \id{N\_Vector} \id{x}:
$z_i = 1.0 /  x_i  , \: i=0,\ldots,n-1$. This routine
may not check for division by $0$. It should be called only with an 
\id{x} which is guaranteed to have all non-zero components.
\\
&\\
\id{N\_VAddConst} & \id{N\_VAddConst(x, b, z);} \\
& Adds the scalar \id{b} to all components of \id{x} and returns the
result in the \id{N\_Vector} \id{z}:
$z_i = x_i + b , \: i=0,\ldots,n-1$.
\\
&\\
\id{N\_VDotProd} & \id{d = N\_VDotProd(x, y);} \\
& Returns the value of the ordinary dot product of \id{x} and \id{y}:
$d=\sum_{i=0}^{n-1} x_i y_i$.
\\
&\\
\id{N\_VMaxNorm} & \id{m = N\_VMaxNorm(x);} \\
& Returns the maximum norm of the \id{N\_Vector} \id{x}:
$m = \max_{i} | x_i |$.
\\
&\\
\id{N\_VWrmsNorm} & \id{m = N\_VWrmsNorm(x, w)} \\
& Returns the weighted root mean square norm of the \id{N\_Vector} \id{x} with
weight vector \id{w}:
$m = \sqrt{\left( \sum_{i=0}^{n-1} (x_i w_i)^2 \right) / n}$.
\\
&\\
\id{N\_VWrmsNormMask} & \id{m = N\_VWrmsNormMask(x, w, id);} \\
& Returns the weighted root mean square norm of the \id{N\_Vector} \id{x} with
weight vector \id{w} built using only the elements of \id{x} corresponding to
nonzero elements of the \id{N\_Vector} \id{id}:\\
&$m = \sqrt{\left( \sum_{i=0}^{n-1} (x_i w_i \text{sign}(id_i))^2 \right) / n}$.
\\
&\\
\id{N\_VMin} & \id{m = N\_VMin(x);} \\
& Returns the smallest element of the \id{N\_Vector} \id{x}:
$m = \min_i x_i $.
\\
&\\
\id{N\_VWL2Norm} & \id{m = N\_VWL2Norm(x, w);} \\
& Returns the weighted Euclidean $\ell_2$ norm of the \id{N\_Vector} \id{x}
with weight vector \id{w}: 
$m = \sqrt{\sum_{i=0}^{n-1} (x_i w_i)^2}$.
\\
&\\
\id{N\_VL1Norm} & \id{m = N\_VL1Norm(x);} \\
& Returns the $\ell_1$ norm of the \id{N\_Vector} \id{x}:
$m = \sum_{i=0}^{n-1} | x_i |$.
\\
&\\
\id{N\_VCompare} & \id{N\_VCompare(c, x, z);} \\
& Compares the components of the \id{N\_Vector} \id{x} to the scalar
\id{c} and returns an \id{N\_Vector} \id{z} such that:
$z_i = 1.0$ if $| x_i | \ge c$ and $z_i = 0.0$ otherwise.
\\
&\\
\id{N\_VInvTest} & \id{t = N\_VInvTest(x, z);} \\
& Sets the components of the \id{N\_Vector} \id{z} to be the inverses
of the components of the \id{N\_Vector} \id{x}, with prior testing
for zero values:
$z_i = 1.0 /  x_i  , \: i=0,\ldots,n-1$.
This routine returns \id{TRUE} if all components of \id{x} are
non-zero (successful inversion) and returns \id{FALSE} otherwise.  
\\
&\\
\id{N\_VConstrMask} & \id{t = N\_VConstrMask(c, x, m);} \\
& Performs the following constraint tests:
$x_i > 0$ if $c_i=2$,
$x_i \ge 0$ if $c_i=1$,
$x_i \le 0$ if $c_i=-1$,
$x_i < 0$ if $c_i=-2$.
There is no constraint on $x_i$ if $c_i=0$.
This routine returns \id{FALSE} if any element failed
the constraint test, \id{TRUE} if all passed.  It also sets a
mask vector \id{m}, with elements equal to $1.0$ where the constraint 
test failed, and $0.0$ where the test passed.
This routine is used only for constraint checking.
\\
&\\
\id{N\_VMinQuotient} & \id{minq = N\_VMinQuotient(num, denom);} \\
& This routine returns the minimum of the quotients obtained   
by term-wise dividing \id{num}$_i$ by \id{denom}$_i$. 
A zero element in \id{denom} will be skipped. 
If no such quotients are found, then the large value 
\Id{BIG\_REAL} (defined in the header file \id{sundialstypes.h})
is returned. 
\\
&\\
\end{supertabular}
\bigskip
