% This is a shared SUNDIALS TEX file with description of
% types used in llntyps.h
%
\section{Description}

The \ID{sundialstypes.h} file contains the definitions of the types \id{realtype} and
\id{integertype}. {\sundials} solvers use the type \id{realtype} for all floating point data and
the type \id{integertype} for all problem size-related data such as the actual
problem size, the bandwidths in the {\band} solver, and the integers
stored in the length $N$ pivot arrays in both the {\dense} and {\band} solvers.
These types make it easy to solve problems of virtually any size
using single or double precision arithmetic. The type \id{realtype} can be
\id{double} or \id{float} and the type \id{integertype} can be \id{int} or
\id{long int}. The default settings are \id{double} and \id{long int}.

\section{Changing type \id{realtype}}

The user can change the precision of the {\sundials} solvers arithmetic from double to single
by changing in \id{sundialstypes.h} the definitions
\begin{verbatim}
#define SUNDIALS_FLOAT  0
#define SUNDIALS_DOUBLE 1
\end{verbatim}
to
\begin{verbatim}
#define SUNDIALS_FLOAT  1
#define SUNDIALS_DOUBLE 0
\end{verbatim}
which will also result in changing the typedef \id{typedef double realtype;} to 
\id{typedef float realtype;}
These macro definitions are used to enable \id{sundiastypes.h} to branch on
the setting of \id{realtype} at compile time. 

Changing from double precision to single precision arithmetic also
requires minor changes in the implementation file \id{sundialsmath.c} for
the {\sundialsmath} module which {\sundials} solvers use. The \id{RPowerR} and
\id{RSqrt} functions compute a real number raised to a real power and the
square root of a number, respectively. The default implementation of
these routines calls standard {\C} math library functions which do double
precision arithmetic. These implementations should be changed to call
single precision routines which are available on the user's machine.

Within {\sundials}, real constants are set by way of a macro called
\Id{RCONST}.  It is this macro that needs the ability to branch on the
setting for \id{realtype}.  In ANSI {\C}, a floating point constant with no
suffix is stored as a \id{double}.  Placing the suffix ``F'' at the
end of a floating point constant makes it a \id{float}. For example,
\begin{verbatim}
#define A 1.0
#define B 1.0F
\end{verbatim}
defines \id{A} to be a \id{double} constant 1.0 and \id{B} to be a
\id{float} constant $1.0$. The macro call \id{RCONST(1.0)}
expands to \id{1.0} if \id{realtype} is \id{double} and it expands to
\id{1.0F} if \id{realtype} is \id{float}. {\sundials} uses the \id{RCONST} macro for
all its floating point constants. 

A user program which uses the type \id{realtype} and the \id{RCONST} macro
to handle floating point constants is precision-independent except for
any calls to single or double precision standard math library
functions.  (Our demonstration programs use \id{realtype} but not
\id{RCONST}.)  Users can, however, use the type \id{double} or
\id{float} in their code (assuming the typedef for \id{realtype} matches
this choice).  Thus, a previously existing piece of ANSI {\C} code can use
{\sundials} without modifying the codes to use \id{realtype}.

\section{Changing type \id{integertype}}

{\sundials} uses the type \id{integertype} for all quantities related to problem
size.  On some machines the size of an \id{int} and a \id{long int}
are the same, but this is not always the case. 
If \id{int} is sufficiently large on a given machine, and the user wishes
to make \id{int} the \id{integertype} type, change the macro definitions in 
\id{sundialstypes.h}
\begin{verbatim}
#define SUNDIALS_INT      0
#define SUNDIALS_LONG_INT 1
\end{verbatim}
to
\begin{verbatim}
#define SUNDIALS_INT      1
#define SUNDIALS_LONG_INT 0
\end{verbatim}
which will also result in changing the typedef \id{typedef long int integertype;} 
to \id{typedef int integertype;}

In terms of the problem size $N$, and the bandwidths \id{ml} and \id{mu} 
in the case of the band module {\band}, the largest integer that must be 
accommodated by the \id{integertype} type is $N + $ \id{ml} + \id{mu} in the 
band case, and $N$ in all other cases. The user can use the type \id{int} 
or \id{long int} in his/her code instead of \id{integertype} (assuming the 
typedef for \id{integertype} matches this choice).