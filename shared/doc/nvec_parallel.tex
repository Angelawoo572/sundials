% This is a shared SUNDIALS TEX file with description of
% the parallel nvector implementation
%
The parallel implementation of the {\nvector} module provided with {\sundials},
{\nvecp}, defines the {\em content} 
field of \id{N\_Vector} to be a structure containing the global and local lengths 
of the vector, a pointer to the beginning of a contiguous local data array,
an {\mpi} communicator, an a boolean flag {\em own\_data} indicating ownership of 
the data array {\em data}.
%%
\begin{verbatim} 
struct _N_VectorContent_Parallel {
  long int local_length;
  long int global_length;
  booleantype own_data;
  realtype *data;
  MPI_Comm comm;
};
\end{verbatim}
%%
%%--------------------------------------------
%%
The following seven macros are provided to access the content of a {\nvecp}
vector. The suffix \id{\_P} in the names denotes parallel version.
\begin{itemize}

\item 
  \ID{NV\_CONTENT\_P}

  This macro gives access to the contents of the parallel
  vector \id{N\_Vector}.
  
  The assignment \id{v\_cont = NV\_CONTENT\_P(v)} sets       
  \id{v\_cont} to be a pointer to the \id{N\_Vector} content    
  structure of type \id{struct \_N\_VectorParallelContent}.
  
  Implementation:
  
  \verb|#define NV_CONTENT_P(v) ( (N_VectorContent_Parallel)(v->content) )|
  
\item 
  \ID{NV\_OWN\_DATA\_P}, \ID{NV\_DATA\_P}, 
  \ID{NV\_LOCLENGTH\_P}, \ID{NV\_GLOBLENGTH\_P}
  
  These macros give individual access to the parts of    
  the content of a parallel \id{N\_Vector}.                        
  
  The assignment \id{v\_data = NV\_DATA\_P(v)} sets \id{v\_data} to be     
  a pointer to the first component of the local data for the \id{N\_Vector} \id{v}. 
  The assignment \id{NV\_DATA\_P(v) = v\_data} sets the component array of 
  \id{v} to be \id{v\_data} by storing the pointer \id{v\_data}.                   
  
  The assignment \id{v\_llen = NV\_LOCLENGTH\_P(v)} sets \id{v\_llen} to be     
  the length of the local part of \id{v}. 
  The call \id{NV\_LENGTH\_P(v) = llen\_v} sets      
  the local length of \id{v} to be \id{llen\_v}.
  
  The assignment \id{v\_glen = NV\_GLOBLENGTH\_P(v)} sets \id{v\_glen} to  
  be the global length of the vector \id{v}.                    
  The call \id{NV\_GLOBLENGTH\_P(v) = glen\_v} sets the global       
  length of \id{v} to be \id{glen\_v}.
  
  Implementation:
  
  \verb|#define NV_OWN_DATA_P(v)   ( NV_CONTENT_P(v)->own_data )|

  \verb|#define NV_DATA_P(v)       ( NV_CONTENT_P(v)->data )|

  \verb|#define NV_LOCLENGTH_P(v)  ( NV_CONTENT_P(v)->local_length )|

  \verb|#define NV_GLOBLENGTH_P(v) ( NV_CONTENT_P(v)->global_length )|
  
\item \ID{NV\_COMM\_P}

  This macro provides access to the {\mpi} communicator used by the {\nvecp}
  vectors.

  Implementation:

  \verb|#define NV_COMM_P(v) ( NV_CONTENT_P(v)->comm )|

\item \ID{NV\_Ith\_P}

  This macro gives access to the individual components of the local data
  array of an \id{N\_Vector}.

  The assignment \id{r = NV\_Ith\_P(v,i)} sets \id{r} to be the value of 
  the \id{i}-th component of the local part of \id{v}. 
  The assignment \id{NV\_Ith\_P(v,i) = r}   
  sets the value of the \id{i}-th component of the local part of \id{v} 
  to be \id{r}.        
  
  Here $i$ ranges from $0$ to $n-1$, where $n$ is the local length.
      
  Implementation:

  \verb|#define NV_Ith_P(v,i) ( NV_DATA_P(v)[i] )|

\end{itemize}
%%
%%--------------------------------------------
%%
The {\nvecp} module defines parallel implementations of all vector operations listed 
in Table \ref{t:nvecops} and provides the following user-callable routines:
%%
%%
\begin{itemize}

%%--------------------------------------

\item  \ID{N\_VNew\_Parallel}
  
  This function creates and allocates memory for a parallel vector.
 
  

\begin{verbatim}
N_Vector N_VNew_Parallel(MPI_Comm comm, 
                         long int local_length, 
                         long int global_length);
\end{verbatim}
  
%%--------------------------------------

\item \ID{N\_VNewEmpty\_Parallel}
 
  This function creates a new parallel \id{N\_Vector} with an empty (\id{NULL}) data array.
 
  

\begin{verbatim}
N_Vector N_VNewEmpty_Parallel(MPI_Comm comm, 
                              long int local_length, 
                              long int global_length);
\end{verbatim}

  
%%--------------------------------------

\item \ID{N\_VCloneEmpty\_Parallel}
 
  This function creates a new parallel \id{N\_Vector} with an empty (\id{NULL}) data array
  by using an existing \id{N\_Vector} as a template.
 
  

  \verb|N_Vector N_VCloneEmpty_Parallel(N_Vector w);|
  
%%--------------------------------------

\item \ID{N\_VMake\_Parallel}
  
  This function creates and allocates memory for a parallel vector
  with user-provided data array.
 
  

\begin{verbatim}
N_Vector N_VMake_Parallel(MPI_Comm comm, 
                          long int local_length,
                          long int global_length,
                          realtype *v_data);
\end{verbatim}

%%--------------------------------------

\item \ID{N\_VNewVectorArray\_Parallel}
 
  This function creates an array of \id{count} parallel vectors.
 
\begin{verbatim}
N_Vector *N_VNewVectorArray_Parallel(int count, 
                                     MPI_Comm comm, 
                                     long int local_length,
                                     long int global_length);
\end{verbatim}

%%--------------------------------------

\item \ID{N\_VNewVectorArrayEmpty\_Parallel}
 
  This function creates an array of \id{count} parallel vectors,
  each with an empty (\id{NULL}) data array.
 
\begin{verbatim}
N_Vector *N_VNewVectorArrayEmpty_Parallel(int count, 
                                          MPI_Comm comm, 
                                          long int local_length,
                                          long int global_length);
\end{verbatim}

%%--------------------------------------

\item \ID{N\_VDestroyVectorArray\_Parallel}
 
 This function frees memory allocated for the array of \id{count} variables of type \id{N\_Vector}
 created with \id{N\_VNewVectorArray\_Parallel} or with \id{N\_VNewVectorArrayEmpty\_Parallel}.

 

 \verb|void N_VDestroyVectorArray_Parallel(N_Vector *vs, int count);|


%%--------------------------------------

\item \ID{N\_VPrint\_Parallel}
  
  This function prints the content of a parallel vector to stdout.
 
  
  
  \verb|void N_VPrint_Parallel(N_Vector v);|


\end{itemize}
%%
%%------------------------------------
%%
\paragraph{\bf Notes} 
           
\begin{itemize}
                                        
\item
  When looping over the components of an \id{N\_Vector} \id{v}, it is     
  more efficient to first obtain the local component array via       
  \id{v\_data = NV\_DATA\_P(v)} and then access \id{v\_data[i]} within the     
  loop than it is to use \id{NV\_Ith\_P(v,i)} within the loop.        
                                                               
\item
  {\warn} The {\nvecp} constructor functions \id{N\_VNewEmpty\_Parallel}, \id{N\_VMake\_Parallel}, 
  \id{N\_VCloneEmpty\_Parallel}, and \id{N\_VNewVectorArrayEmpty\_Parallel}
  set the field {\em own\_data} $=$ \id{FALSE}. 
  The functions \id{N\_VDestroy\_Parallel} and \id{N\_VDestroyVectorArray\_Parallel}
  will not attempt to free the pointer {\em data} for any \id{N\_Vector} with
  {\em own\_data} set to \id{FALSE}. In such a case, it is the user's responsibility to
  deallocate the {\em data} pointer.

\item
  {\warn} To maximize efficiency, vector operations in the {\nvecp} implementation
  that have more than one \id{N\_Vector} argument do not check for
  consistent internal representation of these vectors. It is the user's 
  responsability to ensure that such routines are called with \id{N\_Vector}
  arguments that were all created with the same internal representations.

\end{itemize}

