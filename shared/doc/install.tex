Generally speaking, the installation procedure outlined in
\S\ref{ss:install_steps} below will work on commodity {\linux}/{\unix} systems
without modification. Users are still encouraged, however, to carefully read
the entire chapter before attempting to install the {\sundials} suite, in case
non-default choices are desired for compilers, compilation options, or the like.
In lieu of reading the option list below, the user may invoke the configuration
script with the help flag to view a complete listing of available options, which
may be done by issuing 
\begin{verbatim}
   % ./configure --help 
\end{verbatim}
from within the \id{sundials} directory.

In the descriptions below, {\em build\_tree} refers to the directory under
which the user wants to build and/or install the {\sundials} package. By
default, the {\sundials} libraries and header files are installed under the
subdirectories {\em build\_tree}\id{/lib} and {\em build\_tree}\id{/include},
respectively. Also, {\em source\_tree} refers to the directory where the
{\sundials} source code is located. The chosen {\em build\_tree} may be
different from the {\em source\_tree}, thus allowing for multiple installations
of the {\sundials} suite with different configuration options.

Concerning the installation procedure outlined below, after invoking the
\id{tar} command with the appropriate options, the contents of the
{\sundials} archive (or the {\em source\_tree}) will be extracted to a
directory named \id{sundials}. Since the name of the extracted directory
is not version-specific it is recommended that the user refrain from
extracting the archive to a directory containing a previous version/release
of the {\sundials} suite. If the user is only upgrading and the previous 
installation of {\sundials} is not needed, then the user may remove the
previous installation by issuing 
\begin{verbatim}
   % rm -rf sundials
\end{verbatim}
from a shell command prompt.

Even though the installation procedure given below presupposes that the user
will use the default vector modules supplied with the distribution, using the
{\sundials} suite with a user-supplied vector module normally will not require
any changes to the build procedure.  

%%===============================================================================

\section{Installation steps}\label{ss:install_steps}

To install the {\sundials} suite, given a downloaded file named
{\em sundials\_file}\id{.tar.gz}, issue the following commands from
a shell command prompt, while within the directory where {\em source\_tree}
is to be located.  The names of installed libraries and header files
are listed in Table \ref{t:sundials_files} for reference.  (For brevity,
the corresponding \id{.c} files are not listed.)  Regarding the file
extension .{\em lib} appearing in Table \ref{t:sundials_files}, shared libraries
generally have an extension of \id{.so} and static libraries have an extension of
\id{.a}.  (See {\em Options for library support} for additional details.)
\begin{enumerate}
\item \id{gunzip} {\em sundials\_file}\id{.tar.gz}
\item \id{tar -xf} {\em sundials\_file}\id{.tar}\hspace{2em} [creates \id{sundials} directory]
\item \id{cd} {\em build\_tree}
\item {\em path\_to\_source\_tree}\id{/configure} {\em options}\hspace{2em} [options can be absent]
\item \id{make}
\item \id{make install}
\item \id{make examples}
\item If system storage space conservation is a priority, then issue \\
\verb+   % make clean+ \\
and/or \\
\verb+   % make examples_clean+ \\
from a shell command prompt to remove unneeded object files.
\end{enumerate}

%% >>>>

\newlength{\colA}
\settowidth{\colA}{{\nvecspc}}
\newlength{\colB}
\settowidth{\colB}{\id{libsundials\_nvecspcparallel.{\em lib}}}
\newlength{\colC}
\settowidth{\colC}{\id{nvector\_spcparallel.h}}

\tablecaption{SUNDIALS libraries and header files}
\label{t:sundials_files}
\tablefirsthead{\hline {\bf Module} & {\bf Libraries} & {\bf Header files} \\}
\tablehead{\hline \multicolumn{3}{|l|}{\small\slshape continued from last page} \\
           \hline {\bf Module} & {\bf Libraries} & {\bf Header files} \\ \hline}
\tabletail{\hline \multicolumn{3}{|r|}{\small\slshape continued on next page} \\ \hline}
\tablelasttail{\hline}

\begin{supertabular}{|p{\colA}|p{\colB}|p{\colC}|}
\hline
%% <<<<
%%
%%\begin{table}
%%\centering
%%\caption{SUNDIALS libraries and header files}
%%\label{t:sundials_files}
%%\medskip
%%\begin{tabular}{|l|l|l|}\hline
%%{\bf Module} & {\bf Libraries} & {\bf Header files} \\
%%\hline
{\shared} &  \id{libsundials\_shared.{\em lib}} & \id{sundialstypes.h}    \\
          &                                       & \id{sundialsmath.h}     \\
          &                                       & \id{sundials\_config.h} \\
          &                                       & \id{dense.h}            \\
          &                                       & \id{smalldense.h}       \\
          &                                       & \id{band.h}             \\
          &                                       & \id{spgmr.h}            \\
          &                                       & \id{spbcg.h}            \\
          &                                       & \id{sptfqmr.h}          \\
          &                                       & \id{iterative.h}        \\
          &                                       & \id{nvector.h}          \\
          &                                       & \id{fnvector.h}         \\
\hline
{\nvecs}  & \id{libsundials\_nvecserial.{\em lib}} & \id{nvector\_serial.h}  \\
          & \id{libsundials\_fnvecserial.a}          &                         \\
\hline
{\nvecp}  & \id{libsundials\_nvecparallel.{\em lib}} & \id{nvector\_parallel.h} \\
          & \id{libsundials\_fnvecparallel.a}        &                            \\
\hline
{\nvecspc}  & \id{libsundials\_nvecspcparallel.{\em lib}} & \id{nvector\_spcparallel.h} \\
            & \id{libsundials\_fnvecspcparallel.a}          &                             \\
\hline
{\cvode} & \id{libsundials\_cvode.{\em lib}} & \id{cvode.h}     \\
         & \id{libsundials\_fcvode.a}          & \id{cvdense.h}   \\
         &                                     & \id{cvband.h}    \\
         &                                     & \id{cvdiag.h}    \\
         &                                     & \id{cvspils.h}   \\
         &                                     & \id{cvspgmr.h}   \\
         &                                     & \id{cvspbcg.h}   \\
         &                                     & \id{cvsptfqmr.h} \\
         &                                     & \id{cvbandpre.h} \\
         &                                     & \id{cvbbdpre.h}  \\
\hline
{\cvodes} & \id{libsundials\_cvodes.{\em lib}} & \id{cvodes.h}    \\
          &                                      & \id{cvodea.h}    \\
          &                                      & \id{cvdense.h}   \\
          &                                      & \id{cvband.h}    \\
          &                                      & \id{cvdiag.h}    \\
          &                                      & \id{cvspils.h}   \\
          &                                      & \id{cvspgmr.h}   \\
          &                                      & \id{cvspbcg.h}   \\
          &                                      & \id{cvsptfmqr.h} \\
          &                                      & \id{cvbandpre.h} \\
          &                                      & \id{cvbbdpre.h}  \\
\hline
{\ida} & \id{libsundials\_ida.{\em lib}}  & \id{ida.h}        \\
       & \id{libsundials\_fida.a}           & \id{idadense.h}   \\
       &                                    & \id{idaband.h}    \\
       &                                    & \id{idaspils.h}   \\
       &                                    & \id{idaspgmr.h}   \\
       &                                    & \id{idaspbcg.h}   \\
       &                                    & \id{idasptfqmr.h} \\
       &                                    & \id{idabbdpre.h}  \\
\hline
{\kinsol} & \id{libsundials\_kinsol.{\em lib}} & \id{kinsol.h}     \\
          & \id{libsundials\_fkinsol.a}          & \id{kinspils.h}   \\
          &                                      & \id{kinspgmr.h}   \\
          &                                      & \id{kinspbcg.h}   \\
          &                                      & \id{kinsptfqmr.h} \\
          &                                      & \id{kinbbdpre.h}  \\
          &                                      & \id{kindense.h}   \\
%%\hline
%%\end{tabular}
%%
%%\end{table}
\end{supertabular}

%%===============================================================================

\section{Configuration options}\label{ss:configuration_options}

The installation procedure given above will generally work without modification;
however, if the system includes multiple {\mpi} implementations, then certain
configure script-related options may be used to indicate which {\mpi}
implementation should be used. Also, if the user wants to use non-default
language compilers, then, again, the necessary shell environment variables must
be appropriately redefined.
%%
The remainder of this section provides explanations of available configure script
options.


\subsection*{General options}

%%
%% General options
%%

\begin{config}
  
\item \id{--prefix=PREFIX}
  
  Location for architecture-independent files.
  
  Default: \id{PREFIX=}{\em build\_tree}
  
\item \id{--includedir=DIR}
  
  Alternate location for installation of header files.
  
  Default: \id{DIR=PREFIX/include}
  
\item \id{--libdir=DIR}
  
  Alternate location for installation of libraries.
  
  Default: \id{DIR=PREFIX/lib}

\item \id{--disable-examples}
  
  All available example programs are automatically built unless this option is
  given. The example executables are stored under the following subdirectories
  of the associated solver: 
  
  \begin{config}
  \item {\em build\_tree}/{\em solver}/\id{examples\_ser} : serial {\C} examples
  \item {\em build\_tree}/{\em solver}/\id{examples\_par} : parallel {\C} examples ({\mpi}-enabled)
  \item {\em build\_tree}/{\em solver}/\id{fcmix}/\id{examples\_ser} : serial {\F} examples
  \item {\em build\_tree}/{\em solver}/\id{fcmix}/\id{examples\_par} : parallel {\F} examples ({\mpi}-enabled)
  \end{config}
  
  {\em Note}: Some of these subdirectories may not exist depending upon the
  solver and/or the configuration options given.
  
\item \id{--disable-}{\em solver}

  Although each existing solver module is built by default, support for a
  given solver can be explicitly disabled using this option. 
  The valid values for {\em solver} are: \id{cvode}, \id{cvodes}, 
  \id{ida}, and \id{kinsol}.
  
\item \id{--with-cppflags=ARG}

  Specify additional {\C} preprocessor flags 
  (e.g., \id{ARG=-I<include\_dir>} if necessary header files are located in nonstandard locations).

\item \id{--with-cflags=ARG}

  Specify additional {\C} compilation flags.

\item \id{--with-ldflags=ARG}

  Specify additional linker flags 
  (e.g., \id{ARG=-L<lib\_dir>} if required libraries are located in nonstandard locations).

\item \id{--with-libs=ARG}

  Specify additional libraries to be used 
  (e.g., \id{ARG=-l<foo>} to link with the library named \id{libfoo.a} or \id{libfoo.so}).

\item \id{--with-precision=ARG}

  By default, {\sundials} will define a real number (internally referred to as
  \id{realtype}) to be a double-precision floating-point numeric data type (\id{double}
  {\C}-type); however, this option may be used to build {\sundials} with \id{realtype}
  alternatively defined as a single-precision floating-point numeric data type
  (\id{float} {\C}-type) if \id{ARG=single}, or as a \id{long double} {\C}-type
  if \\ \id{ARG=extended}.

  Default: \id{ARG=double}

\end{config}

%%
%% Fortran support
%%

\subsection*{Options for Fortran support}

\begin{config}

\item \id{--disable-f77}

  Using this option will disable all {\F} support. The {\fcvode}, {\fkinsol} and
  {\fnvector} modules will not be built regardless of availability.

\item \id{--with-fflags=ARG}

  Specify additional {\F} compilation flags.

\end{config}

\noindent The configuration script will attempt to automatically determine the
function name mangling scheme required by the specified {\F} compiler, but the
following two options may be used to override the default behavior.

\begin{config}

\item \id{--with-f77underscore=ARG}

  This option pertains to the {\fkinsol}, {\fcvode} and {\fnvector} {\F}-{\C}
  interface modules and is used to specify the number of underscores to append
  to function names so {\F} routines can properly link with the associated
  {\sundials} libraries. Valid values for \id{ARG} are: \id{none}, \id{one}
  and \id{two}.

  Default: \id{ARG=one}

\item \id{--with-f77case=ARG}

  Use this option to specify whether the external names of the {\fkinsol},
  {\fcvode} and {\fnvector} {\F}-{\C} interface functions should be lowercase
  or uppercase so {\F} routines can properly link with the associated {\sundials}
  libraries. Valid values for \id{ARG} are: \id{lower} and \id{upper}.

  Default: \id{ARG=lower}

\end{config}


%%
%% Parallel options
%%

\subsection*{Options for MPI support}

\noindent The following configuration options are only applicable to the parallel {\sundials} packages:

\begin{config}
  
\item \id{--disable-mpi}

  Using this option will completely disable {\mpi} support.

\item \id{--with-mpicc=ARG}
\item \id{--with-mpif77=ARG}

  By default, the configuration utility script will use the {\mpi} compiler
  scripts named \id{mpicc} and \id{mpif77} to compile the parallelized
  {\sundials} subroutines; however, for reasons of compatibility, different
  executable names may be specified via the above options. Also, \id{ARG=no}
  can be used to disable the use of {\mpi} compiler scripts, thus causing
  the serial {\C} and {\F} compilers to be used to compile the parallelized
  {\sundials} functions and examples.

\item \id{--with-mpi-root=MPIDIR}

  This option may be used to specify which {\mpi} implementation should be used.
  The {\sundials} configuration script will automatically check under the
  subdirectories \id{MPIDIR/include} and \id{MPIDIR/lib} for the necessary
  header files and libraries. The subdirectory \id{MPIDIR/bin} will also be
  searched for the {\C} and {\F} {\mpi} compiler scripts, unless the user uses
  \id{--with-mpicc=no} or \id{--with-mpif77=no}.

\item \id{--with-mpi-incdir=INCDIR}
\item \id{--with-mpi-libdir=LIBDIR}
\item \id{--with-mpi-libs=LIBS}

  These options may be used if the user would prefer not to use a preexisting
  {\mpi} compiler script, but instead would rather use a serial complier and
  provide the flags necessary to compile the {\mpi}-aware subroutines in
  {\sundials}.

  Often an {\mpi} implementation will have unique library names and so it may
  be necessary to specify the appropriate libraries to use (e.g.,
  \id{LIBS=-lmpich}).

  Default: \id{INCDIR=MPIDIR/include} and \id{LIBDIR=MPIDIR/lib}

\item \id{--with-mpi-flags=ARG}

  Specify additional {\mpi}-specific flags.

\end{config}


\subsection*{Options for library support}
%%
%% Shared or Static Libraries
%%

\noindent By default, only static libraries are built, but the following option
may be used to build shared libraries on supported platforms.

\begin{config}

\item \id{--enable-shared}

  Using this particular option will result in both static and shared versions
  of the available {\sundials} libraries being built if the system supports
  shared libraries. To build only shared libraries also specify \id{--disable-static}.

  {\em Note}: The {\fcvode} and {\fkinsol} libraries can only be built as static
  libraries because they contain references to externally defined symbols, namely
  user-supplied {\F} subroutines.  Although the {\F} interfaces to the serial and
  parallel implementations of the supplied {\nvector} module do not contain any
  unresolvable external symbols, the libraries are still built as static libraries
  for the purpose of consistency.

\end{config}

\subsection*{Options for cross-compilation}

%%
%% Cross-compilation
%%

\noindent If the {\sundials} suite will be cross-compiled (meaning the build
procedure will not be completed on the actual destination system, but rather
on an alternate system with a different architecture) then the following two
options should be used:

\begin{config}

\item \id{--build=BUILD}

  This particular option is used to specify the canonical system/platform name
  for the build system.

\item \id{--host=HOST}

  If cross-compiling, then the user must use this option to specify the canonical
  system/platform name for the destination system.

\end{config}

\subsection*{Environment variables}

%%
%% Environment variables
%%

\noindent The following environment variables can be locally (re)defined for use 
during the configuration of {\sundials}. See the next section for illustrations of these.

\begin{config}

\item \id{CC}

\item \id{F77}

  Since the configuration script uses the first {\C} and {\F} compilers found in
  the current executable search path, then each relevant shell variable (\id{CC}
  and \id{F77}) must be locally (re)defined in order to use a different compiler. 
  For example, to use \id{xcc} (executable name of chosen compiler) as the {\C}
  language compiler, use \id{CC=xcc} in the configure step.

\item \id{CFLAGS}

\item \id{FFLAGS}

  Use these environment variables to override the default {\C} and {\F}
  compilation flags.

\end{config}


%%===============================================================================

\section{Configuration examples}

The following examples are meant to help demonstrate proper usage of the configure options:

\begin{verbatim}
% configure CC=gcc F77=g77 --with-cflags=-g3 --with-fflags=-g3 \
            --with-mpicc=/usr/apps/mpich/1.2.4/bin/mpicc \ 
            --with-mpif77=/usr/apps/mpich/1.2.4/bin/mpif77
\end{verbatim}

\noindent The above example builds {\sundials} using \id{gcc} as the serial {\C}
compiler, \id{g77} as the serial {\F} compiler, \id{mpicc} as the parallel {\C}
compiler, \id{mpif77} as the parallel {\F} compiler, and appends the \id{-g3}
compilaton flag to the list of default flags.

\begin{verbatim}
% configure CC=gcc --disable-examples --with-mpicc=no \
            --with-mpi-root=/usr/apps/mpich/1.2.4 \
            --with-mpi-libs=-lmpich
\end{verbatim}

\noindent This example again builds {\sundials} using \id{gcc} as the serial
{\C} compiler, but the \id{--with-mpicc=no} option explicitly disables the use
of the corresponding {\mpi} compiler script. In addition, since the 
\id{--with-mpi-root} option is given, the compilation flags
\id{-I/usr/apps/mpich/1.2.4/include} and \id{-L/usr/apps/mpich/1.2.4/lib} are passed
to \id{gcc} when compiling the {\mpi}-enabled functions. 
The \id{--disable-examples} option disables the examples (which means
a {\F} compiler is not required).
The \id{--with-mpi-libs} option is required so that the configure
script can check if \id{gcc} can link with the appropriate {\mpi}
library.
