%%===================================================================================
\chapter{Using IDA}\label{s:simulation}
%%===================================================================================

This chapter is concerned with the use of {\ida} for the integration of DAEs.
The following sections treat the header files, the layout of the user's main
program, description of the {\ida} user-callable functions, and user-supplied functions. 
The listings of the sample programs in the companion document \cite{ida2.2.0_ex} 
may also be helpful. Those codes are intended to serve as templates and 
are included in the {\ida} package.

The user should be aware that not all linear solver modules are compatible 
with all {\nvector} implementations. 
\index{IDA@{\ida} linear solvers!NVECTOR@{\nvector} compatibility}
For example, {\nvecp} is not compatible with the direct dense or direct band 
linear solvers since these linear solver modules need to form the system Jacobian.
The {\idadense} and {\idaband} modules can only be used with {\nvecs}. 
The preconditioner module {\idabbdpre} can only be used with {\nvecp}. 

%%===================================================================================
\section{Data Types}\label{s:types}
%%===================================================================================
% This is a shared SUNDIALS TEX file with description of
% types used in llntyps.h
%
\index{portability}
The \ID{sundials\_types.h} file contains the definition of the type \ID{realtype},
which is used by the {\sundials} solvers for all floating-point data, the definition 
of the integer type \ID{sunindextype}, which is used for vector and matrix indices,
and \ID{booleantype}, which is used for certain logic operations within {\sundials}.


\subsection{Floating point types}

The type \id{realtype} can be \id{float}, \id{double}, or \id{long double}, with
the default being \id{double}.
The user can change the precision of the {\sundials} solvers arithmetic at the
configuration stage (see \S\ref{ss:configuration_options_nix}).

Additionally, based on the current precision, \id{sundials\_types.h} defines 
\Id{BIG\_REAL} to be the largest value representable as a \id{realtype},
\Id{SMALL\_REAL} to be the smallest value representable as a \id{realtype}, and
\Id{UNIT\_ROUNDOFF} to be the difference between $1.0$ and the minimum \id{realtype}
greater than $1.0$.

Within {\sundials}, real constants are set by way of a macro called
\Id{RCONST}.  It is this macro that needs the ability to branch on the
definition \id{realtype}.  In ANSI {\CC}, a floating-point constant with no
suffix is stored as a \id{double}.  Placing the suffix ``F'' at the
end of a floating point constant makes it a \id{float}, whereas using the suffix
``L'' makes it a \id{long double}.  For example,
\begin{verbatim}
#define A 1.0
#define B 1.0F
#define C 1.0L
\end{verbatim}
defines \id{A} to be a \id{double} constant equal to $1.0$, \id{B} to be a
\id{float} constant equal to $1.0$, and \id{C} to be a \id{long double} constant
equal to $1.0$.  The macro call \id{RCONST(1.0)} automatically expands to \id{1.0}
if \id{realtype} is \id{double}, to \id{1.0F} if \id{realtype} is \id{float},
or to \id{1.0L} if \id{realtype} is \id{long double}.  {\sundials} uses the
\id{RCONST} macro internally to declare all of its floating-point constants. 

A user program which uses the type \id{realtype} and the \id{RCONST} macro
to handle floating-point constants is precision-independent except for
any calls to precision-specific standard math library
functions.  (Our example programs use both \id{realtype} and
\id{RCONST}.)  Users can, however, use the type \id{double}, \id{float}, or
\id{long double} in their code (assuming that this usage is consistent
with the typedef for \id{realtype}).  Thus, a previously existing
piece of ANSI {\CC} code can use {\sundials} without modifying the code
to use \id{realtype}, so long as the {\sundials} libraries use the
correct precision (for details see \S\ref{ss:configuration_options_nix}).


\subsection{Integer types used for vector and matrix indices}

The type \id{sunindextype} can be either a 64- or 32-bit \emph{signed} integer.
The default is the portable \id{int64\_t} type, and the user can change it
to \id{int32\_t} at the configuration stage. The configuration system
will detect if the compiler does not support portable types, and will
replace \id{int64\_t} and \id{int32\_t} with \id{long long} and \id{int},
respectively, to ensure use of the desired sizes on Linux, Mac OS X and Windows
platforms. {\sundials} currently does not support \emph{unsigned} integer types 
for vector and matrix indices, although these could be added in the future if there 
is sufficient demand.


%%===================================================================================
\section{Header files}\label{ss:header_sim}
%%===================================================================================
\index{header files}
The calling program must include several header files so that various macros
and data types can be used. The header file that is always required is:
%%
\begin{itemize}
\item  \Id{ida.h}, 
  the header file for {\ida}, which defines the several
  types and various constants, and includes function prototypes.
\end{itemize}
%%
Note that \id{ida.h} includes \Id{sundialstypes.h}, 
which defines the types \id{realtype} and \id{booleantype}
and the constants \id{FALSE} and \id{TRUE}.

The calling program must also include an {\nvector} implementation header file
(see Chapter \ref{s:nvector} for details).
For the two {\nvector} implementations that are included in the {\ida} package,
the corresponding header files are:
%%
\begin{itemize}
\item \Id{nvector\_serial.h}, 
  which defines the serial implementation {\nvecs};
\item \Id{nvector\_parallel.h}, 
  which defines the parallel MPI implementation, {\nvecp}.
\end{itemize}
%%
Note that both these files include in turn the header file \Id{nvector.h} which 
defines the abstract \Id{N\_Vector} type. 

Finally, a linear solver module header file is required. 
\index{IDA@{\ida} linear solvers!header files}
The header files corresponding to the various linear solver options in {\ida} are:
%%
\begin{itemize}
\item \Id{idadense.h}, 
  which is used with the dense direct linear solver in 
  the context of {\ida}. This in turn includes a header file (\id{dense.h})
  which defines the \Id{DenseMat} type and corresponding accessor macros; 
\item \Id{idaband.h}, 
  which is used with the band direct linear solver in the
  context of {\ida}. This in turn includes a header file (\id{band.h})
  which defines the \Id{BandMat} type and corresponding accessor macros;
\item \Id{idaspgmr.h}, 
  which is used with the Krylov solver {\spgmr} in the
  context of {\ida}. This in turn includes a header file (\id{iterative.h})
  which enumerates the kind of preconditioning and the choices for the
  Gram-Schmidt process.
\end{itemize}

%%===================================================================================
\section{A skeleton of the user's main program}\label{ss:skeleton_sim}
%%===================================================================================

%A high-level view of the combined user program and {\ida} package is
%shown in Figure~\ref{f:sim_overview}.
%%
%\begin{figure}
%\centerline{\psfig{figure=idasim.eps,width=\textwidth}}
%\caption {Diagram of the user program and 
%  {\ida} package for integration of DAE}\label{f:sim_overview}
%\end{figure}
%%
The following is a skeleton of the user's main program (or calling
program) for the integration of a DAE IVP. Some steps are independent of the {\nvector}
implementation used; where this is not the case, usage specifications are given for the
two implementations provided with {\ida}: steps marked with {\p} correspond to 
{\nvecp}, while steps marked with {\s} correspond to {\nvecs}.
%%
%%
%%
\index{User main program!IDA@{\ida} usage}
\begin{Steps}
  
\item 
  {\bf {\p} Initialize MPI}

  Call \id{MPI\_Init(\&argc, \&argv);} to initialize MPI if used by
  the user's program, aside from the internal use in {\nvecp}.  
  Here \id{argc} and \id{argv} are the command line argument 
  counter and array received by \id{main}.
  
\item
  {\bf Set problem dimensions}

  {\s} Set \id{N}, the problem size $N$.

  {\p} Set \id{Nlocal}, the local vector length (the sub-vector
  length for this processor); \id{N}, the global vector length (the
  problem size $N$, and the sum of all the values of \id{Nlocal});
  and the active set of processors.
    
\item
  {\bf Set vector of initial values}
 
  To set the vectors \id{y0} and \id{yp0} to initial values for $y$ and $y'$, 
  use functions defined by a particular {\nvector} implementation. 
  If a \id{realtype} array  \id{ydata}
  already exists, containing the initial values of $y$, make the call:

  {\s} \id{y0 = NV\_Make\_Serial(N, ydata);}

  {\p} \id{y0 = NV\_Make\_Parallel(comm, Nlocal, N, ydata);}

  Otherwise, make the call:

  {\s} \id{y0 = NV\_New\_Serial(N);}

  {\p} \id{y0 = NV\_New\_Parallel(comm, Nlocal, N);}

  and load initial values into the structure defined by:

  {\s} \id{NV\_DATA\_S(y0)}

  {\p} \id{NV\_DATA\_P(y0)}

  Here \id{comm} is the MPI communicator, set in one of two ways: 
  If a proper subset of active processors is to be used, \id{comm} 
  must be set by suitable MPI calls. Otherwise, to specify that all 
  processors are to be used, \id{comm} must be \id{MPI\_COMM\_WORLD}.
  
  The initial conditions for $y'$ are set similarly.

\item\label{i:ida_create} 
  {\bf Create {\ida} object}

  Call \id{ida\_mem = }\id{IDACreate}\id{(...);} 
  to create the {\ida} memory block.
  \id{IDACreate} returns a pointer to the {\ida} memory structure.
  See \S\ref{sss:idamalloc} for details.

\item
  {\bf Set optional inputs}

  Call \id{IDASet*} functions to change from their default values any
  optional inputs that control the behavior of {\ida}.
  See \S\ref{ss:optional_input} for details.

\item\label{i:ida_malloc} 
  {\bf Allocate internal memory}

  Call \id{IDAMalloc}\id{(...);} 
  to provide required problem specifications,
  allocate internal memory for {\ida}, 
  and initialize {\ida}.
  \id{IDAMalloc} returns an error flag to indicate success or an illegal argument value.
  See \S\ref{sss:idamalloc} for details.
  
\item\label{i:lin_solver} 
  {\bf Attach linear solver module}

  Initialize the linear solver module
  with one of the following calls (for details see \S\ref{sss:lin_solv_init}):

  {\s} \id{flag = }\Id{IDADense}\id{(...);}

  {\s} \id{flag = }\Id{IDABand}\id{(...);}

  \id{flag = }\Id{IDASpgmr}\id{(...);}
  
\item
  {\bf Set linear solver optional inputs}

  Call \id{IDA*Set*} functions from the selected linear solver module to
  change optional inputs specific to that linear solver.
  See \S\ref{ss:optional_input} for details.

\item 
  {\bf Correct initial values}

  Optionally, call \id{IDACalcIC} to correct the initial values
  \id{y0} and \id{yp0}.

\item
  {\bf Advance solution in time}

  For each point at which output is desired, call
  \id{flag = }\Id{IDASolve}\id{(ida\_mem, tout, \&tret, yret, ypret, itask);}
  Set \id{itask} to specify the return mode.
  The vector \id{yret} (which can be the same as
  the vector \id{y0} above) will contain $y(t)$,
  while the vector \id{ypret} will contain $y^\prime(t)$.
  See \S\ref{sss:idasolve} for details.
  
\item
  {\bf Get optional outputs}

  Call \id{IDA*Get*} functions to obtain optional output.
  See \S\ref{ss:optional_output} for details.

\item
  {\bf Deallocate memory for solution vector}

  Upon completion of the integration, deallocate memory for the vectors \id{yret}
  and \id{ypret} by calling the destructor function defined by the {\nvector} 
  implementation:

  {\s} \id{NV\_Destroy\_Serial(yret);}

  {\p} \id{NV\_Destroy\_Parallel(yret);}
  
  and similarly for \id{ypret}.
  
\item
  {\bf Free solver memory}

  \Id{IDAFree}\id{(ida\_mem);} to free the memory allocated for {\ida}.
  
\item 
  {\bf {\p} Finalize MPI}
  
  Call \id{MPI\_Finalize();} to terminate MPI.

\end{Steps}

%%===================================================================================
\section{User-callable functions}
\label{ss:ida_fct_sim}
%%===================================================================================

This section describes the {\ida} functions that are called by the user to set up 
and solve a DAE. Some of these are required. However, starting with \S\ref{ss:optional_input},
the functions listed involve optional inputs/outputs or restarting, and those paragraphs can 
be skipped for a casual use of {\ida}. In any case, refer to \S\ref{ss:skeleton_sim} for
the correct order of these calls.

\subsection{IDA initialization and deallocation functions}
\label{sss:idamalloc}
%%
The following three functions must be called in the order listed. The last one is to be 
called only after the DAE solution is complete, as it frees the {\ida} memory block
created and allocated by the first two calls.
%%
\ucfunction{IDACreate}
{
  ida\_mem = IDACreate();
}
{
  The function \ID{IDACreate} instantiates an {\ida} solver object.
}
{
  \id{IDACreate} has no arguments.
}
{
  If successful, \id{IDACreate} returns a pointer to the newly created 
  {\ida} memory block (of type \id{void *}).
  If an error occurred, \id{IDACreate} prints an error message to \id{stderr}
  and returns \id{NULL}.
}
{}
%%
%%
\ucfunction{IDAMalloc}
{
flag = IDAMalloc(ida\_mem, res, t0, y0, yp0, itol, reltol, abstol);
}
{
  The function \ID{IDAMalloc} provides required problem and solution specifications, 
  allocates internal memory, and initializes {\ida}.
}
{
  \begin{args}[ida\_mem]
  \item[ida\_mem] (\id{void *})
    pointer to the {\ida} memory block returned by \id{IDACreate}.
  \item[res] (\Id{IDAResFn})
    is the {\C} function which computes $F$ in the DAE. This function has the form 
    \id{res(t, yy, yp, resval, res\_data)} (for full details see \S\ref{ss:user_fct_sim}).
  \item[t0] (\id{realtype})
    is the initial value of $t$.
  \item[y0] (\id{N\_Vector})
    is the initial value of $y$. 
  \item[yp0] (\id{N\_Vector})
    is the initial value of $y^\prime$. 
  \item[itol] (\id{int}) 
    is either \Id{IDA\_SS} or \Id{IDA\_SV}, where \Id{itol}$=$\id{IDA\_SS} indicates scalar relative error 
    tolerance and scalar absolute error tolerance, while \id{itol}$=$\id{IDA\_SV} indicates scalar
    relative error tolerance and vector absolute error tolerance. 
    The latter choice is important when the absolute error tolerance needs to
    be different for each component of the DAE. 
  \item[reltol] (\id{realtype *})
    \index{tolerances}
    is a pointer to the relative error tolerance.
  \item[abstol] (\id{void *})
    is a pointer to the absolute error tolerance.
  \end{args}
}
{
  The return flag \id{flag} (of type \id{int}) will be one of the following:
  \begin{args}[IDA\_ILL\_INPUT]
  \item[\Id{IDA\_SUCCESS}]
    The call to \id{IDAMalloc} was successful.
  \item[\Id{IDA\_MEM\_NULL}] 
    The {\ida} memory block was not initialized through a previous call to \id{IDACreate}.
  \item[\Id{IDA\_MEM\_FAIL}] 
    A memory allocation request has failed.
  \item[\Id{IDA\_ILL\_INPUT}] 
    An input argument to \id{IDAMalloc} has an illegal value.
  \end{args}
}
{
  If an error occurred, \id{IDAMalloc} also prints an error message to the
  file specified by the optional input \id{errfp}.
}
%%
%%
\ucfunction{IDAFree}
{
  IDAFree(ida\_mem);
}
{
  The function \ID{IDAFree} frees the pointer allocated by
  a previous call to \id{IDAMalloc}.
}
{
  The argument is the pointer to the {\ida} memory block (of type \id{void *}).
}
{
  The function \id{IDAFree} has no return value.
}
{}
%%
%%===================================================================================
%%
\subsection{Linear solver specification functions}\label{sss:lin_solv_init}

As previously explained, Newton iteration requires the solution of
linear systems of the form (\ref{e:DAE_Newtoncorr}).  There are three {\ida} linear
solvers currently available for this task: {\idadense}, {\idaband}, 
and {\idaspgmr}.  The first two are direct solvers and derive their name
from the type of approximation used for the Jacobian 
$J = \partial{F}/\partial{y} + c_j \partial{F}/\partial{y^\prime}$.
{\idadense} and {\idaband} work with
dense and banded approximations to $J$, respectively.  The
third {\ida} linear solver, {\idaspgmr}, is an iterative solver.  
The {\spgmr} in the name indicates that it uses a scaled preconditioned
GMRES method.

\index{IDA@{\ida} linear solvers!selecting one|(} 
To specify a {\ida} linear solver, after the call to \id{IDACreate}
but before any calls to \id{IDASolve}, the user's program must call one
of the functions \Id{IDADense}, \Id{IDABand}, \Id{IDASpgmr},
as documented below. The first argument passed to these functions is the {\ida}
memory pointer returned by \id{IDACreate}.  A call to one of these
functions links the main {\ida} integrator to a linear solver and
allows the user to specify parameters which are specific to a
particular solver, such as the bandwidths in the {\idaband} case.
%%
The use of each of the linear solvers involves certain constants and possibly 
some macros, that are likely to be needed in the user code.  These are
available in the corresponding header file associated with the linear
solver, as specified below.
\index{IDA@{\ida} linear solvers!selecting one|)}

\index{IDA@{\ida} linear solvers!built on generic solvers|(} 
In each case the linear
solver module used by {\ida} is actually built on top of a generic
linear system solver, which may be of interest in itself.  These
generic solvers, denoted {\dense}, {\band}, and {\spgmr}, are described
separately in \S\ref{s:gen_linsolv}.
\index{IDA@{\ida} linear solvers!built on generic solvers|)}
%%
%%
%%
\index{IDA@{\ida} linear solvers!IDADENSE@{\idadense}}
\index{IDADENSE@{\idadense} linear solver!selection of}
\index{IDADENSE@{\idadense} linear solver!NVECTOR@{\nvector} compatibility}
\ucfunction{IDADense}
{
  flag = IDADense(ida\_mem, N);
}
{
  The function \ID{IDADense} selects the {\idadense} linear solver. 

  The user's main function must include the \id{idadense.h} header file.
}
{
  \begin{args}[ida\_mem]
  \item[ida\_mem] (\id{void *})
    pointer to the {\ida} memory block.
  \item[N] (\id{long int})
    problem dimension.
  \end{args}
}
{
  The return value \id{flag} (of type \id{int}) is one of
  \begin{args}[IDADENSE\_ILL\_INPUT]
  \item[\Id{IDADENSE\_SUCCESS}] 
    The {\idadense} initialization was successful.
  \item[\Id{IDADENSE\_MEM\_NULL}]
    The \id{ida\_mem} pointer is \id{NULL}.
  \item[\Id{IDADENSE\_ILL\_INPUT}]
    The {\idadense} solver is not compatible with the current {\nvector} module.
  \item[\Id{IDADENSE\_MEM\_FAIL}]
    A memory allocation request failed.
  \end{args}
}
{
  The {\idadense} linear solver may not be compatible with a particular
  implementation of the {\nvector} module. 
  Of the two {\nvector} modules provided by {\sundials}, only {\nvecs} is 
  compatible, while {\nvecp} is not.
}
%%
%%
%%
\index{IDA@{\ida} linear solvers!IDABAND@{\idaband}}
\index{IDABAND@{\idaband} linear solver!selection of}
\index{IDABAND@{\idaband} linear solver!NVECTOR@{\nvector} compatibility}
\index{half-bandwidths}
\ucfunction{IDABand}
{
  flag = IDABand(ida\_mem, N, mupper, mlower);
}
{
  The function \ID{IDABand} selects the {\idaband} linear solver. 

  The user's main function must include the \id{idaband.h} header file.
}
{
  \begin{args}[ida\_mem]
  \item[ida\_mem] (\id{void *})
    pointer to the {\ida} memory block.
  \item[N] (\id{long int})
    problem dimension.
  \item[mupper] (\id{long int})
    upper half-bandwidth of the problem Jacobian (or of the approximation of it).
  \item[mlower] (\id{long int})
    lower half-bandwidth of the problem Jacobian (or of the approximation of it).
  \end{args}
}
{
  The return value \id{flag} (of type \id{int}) is one of
  \begin{args}[IDABAND\_ILL\_INPUT]
  \item[\Id{IDABAND\_SUCCESS}] 
    The {\idaband} initialization was successful.
  \item[\Id{IDABAND\_MEM\_NULL}]
    The \id{ida\_mem} pointer is \id{NULL}.
  \item[\Id{IDABAND\_ILL\_INPUT}]
    The {\idaband} solver is not compatible with the current {\nvector} module, or
    one of the Jacobian half-bandwidths is outside its valid range ($0 \ldots$ \id{N}$-1$).
  \item[\Id{IDABAND\_MEM\_FAIL}]
    A memory allocation request failed.
  \end{args}
}
{
  The {\idaband} linear solver may not be compatible with a particular
  implementation of the {\nvector} module. Of the two {\nvector} modules 
  provided by {\sundials}, only {\nvecs} is compatible, while {\nvecp} is not.
  The half-bandwidths are to be set so that the nonzero locations $(i,j)$ in the
  banded (approximate) Jacobian satisfy $-$\id{mlower} $\leq j-i \leq$ \id{mupper}.
}
%%
%%
%%
\index{IDA@{\ida} linear solvers!IDASPGMR@{\idaspgmr}}
\index{IDASPGMR@{\idaspgmr} linear solver!selection of} 
\ucfunction{IDASpgmr}
{
  flag = IDASpgmr(ida\_mem, maxl);
}
{
  The function \ID{IDASpgmr} selects the {\idaspgmr} linear solver. 

  The user's main function must include the \id{idaspgmr.h} header file.
}
{
  \begin{args}[ida\_mem]
  \item[ida\_mem] (\id{void *})
    pointer to the {\ida} memory block.
  \item[maxl] (\id{int})
    \index{maxl@\texttt{maxl}}
    maximum dimension of the Krylov subspace to be used. Pass $0$ to use the 
    default value \id{IDA\_SPGMR\_MAXL}$=5$.
  \end{args}
}
{
  The return value \id{flag} (of type \id{int}) is one of
  \begin{args}[IDASPGMR\_MEM\_NULL]
  \item[\Id{IDASPGMR\_SUCCESS}] 
    The {\idaspgmr} initialization was successful.
  \item[\Id{IDASPGMR\_MEM\_NULL}]
    The \id{ida\_mem} pointer is \id{NULL}.
  \item[\Id{IDASPGMR\_MEM\_FAIL}]
    A memory allocation request failed.
  \end{args}
}
{}

%%===================================================================================

\subsection{Initial condition calculation function}\label{ss:idacalcic}

\id{IDACalcIC} calculates corrected initial conditions for the DAE 
system for a class of index-one problems of semi-implicit form.
It uses Newton iteration combined with a linesearch algorithm.
Calling \id{IDACalcIC} is optional. It is only necessary when the  
initial conditions do not solve the given system;  i.e., if   
\id{y0} and \id{yp0} are known to satisfy $F(t_0, y_0, {\dot y}_0) = 0$, 
then a call to \id{IDACalcIC} is {\em not} necessary.

A call to \id{IDACalcIC} must be preceded by successful calls to  
\id{IDACreate} and \id{IDAMalloc}, and by a successful call to the 
linear system solver specification function.  
In addition, \id{IDACalcIC} assumes that the vectors \id{y0} and \id{yp0},
passed to \id{IDAMalloc}, and (if relevant) \id{id} and \id{constraints},
set through \id{IDASetId} and \id{IDASetConstraints}, respectively
(see \S\ref{ss:optional_input}), remain unaltered since that call.

The call to \id{IDACalcIC} should precede the call(s) to \id{IDASolve}  
for the given problem.           

%
\ucfunction{IDACalcIC}
{
  flag = IDACalcIC(ida\_mem, icopt, tout1);
}
{
  The function \ID{IDACalcIC} corrects the initial values \id{y0}, \id{yp0}.
}
{
  \begin{args}[ida\_mem]

  \item[ida\_mem] (\id{void *})
    pointer to the {\ida} memory block.

  \item[icopt] (\id{int})
    is the option of IDACalcIC to be used.
    
    \id{icopt}$=$\ID{CALC\_YA\_YDP\_INIT} directs \id{IDACalcIC} to compute 
    the algebraic components of $y$ and differential components of ${\dot y}$, 
    given the differential components of $y$.  
    This option requires that the \id{N\_Vector} \id{id} was set through 
    \id{IDASetId}, specifying the differential and algebraic components.  
    
    \id{icopt}$=$\ID{CALC\_Y\_INIT} directs \id{IDACalcIC} to compute all  
    components of $y$, given ${\dot y}$.  \id{id} is not required.

  \item[tout1] (\id{realtype})
    is the first value of $t$ at which a solution will be requested (from \id{IDASolve}).
    This is needed here to determine the direction of integration and rough scale  
    in the independent variable $t$.   

  \end{args}
}
{
  The return value \id{flag} (of type \id{int}) will be one of the following:

  \begin{args}[IDA\_LINESEARCH\_FAIL]

  \item[\Id{IDA\_SUCCESS}]
    \id{IDASolve} succeeded.

  \item[\Id{IDA\_MEM\_NULL}]
    The argument \id{ida\_mem} was \id{NULL}.             
               
  \item[\Id{IDA\_NO\_MALLOC}]
    The allocation function \id{IDAMalloc} has not been called.
                                                  
  \item[\Id{IDA\_ILL\_INPUT}]
    One of the input arguments was illegal.    
                                                                 
  \item[\Id{IDA\_LSETUP\_FAIL}] 
    The linear solver's setup function failed in an unrecoverable manner.

  \item[\Id{IDA\_LINIT\_FAIL}]
    The linear solver's initialization function failed.   
                                                                 
  \item[\Id{IDA\_LSOLVE\_FAIL}] 
    The linear solver's solve function failed in an unrecoverable manner.

  \item[\Id{IDA\_BAD\_EWT}]
    Some component of the error weight vector is zero (illegal), either for 
    the input value of \id{y0} or a corrected value.          
                                                                 
  \item[\Id{IDA\_FIRST\_RES\_FAIL}]
    The user's residual function returned a recoverable error flag on the first call,
    but \id{IDACalcIC} was unable to recover.       
                                                                 
  \item[\Id{IDA\_RES\_FAIL}]
    The user's residual function returned a nonrecoverable error flag.

  \item[\Id{IDA\_NO\_RECOVERY}]
    The user's residual function, or the linear solver's setup or solve function had a      
    recoverable error, but \id{IDACalcIC} was unable to recover.
                                                                 
  \item[\Id{IDA\_CONSTR\_FAIL}]
    \id{IDACalcIC} was unable to find a solution    
    satisfying the inequality constraints.     
                                                                 
  \item[\Id{IDA\_LINESEARCH\_FAIL}]
    The linesearch algorithm failed to find a solution with a step larger than 
    \id{steptol} in weighted RMS norm.                      
                                                                 
  \item[\Id{IDA\_CONV\_FAIL}]
    \id{IDACalcIC} failed to get convergence of the Newton iterations.

  \end{args} 
}
{
  All failure return values are negative and therefore a test \id{flag}$< 0$
  will trap all \id{IDACalcIC} failures.
}


%%===================================================================================

\subsection{IDA solver function}\label{sss:idasolve}
%
This is the central step in the solution process - the call to perform the integration 
of the DAE.
%
\ucfunction{IDASolve}
{
  flag = IDASolve(ida\_mem, tout, tret, yret, ypret, itask);
}
{
  The function \ID{IDASolve} integrates the DAE over an interval in $t$.
}
{
  \begin{args}[ida\_mem]
  \item[ida\_mem] (\id{void *})
    pointer to the {\ida} memory block.
  \item[tout] (\id{realtype})
    the next time at which a computed solution is desired.
  \item[tret] (\id{realtype *})
    the time reached by the solver.
  \item[yret] (\id{N\_Vector})
    the computed solution vector $y$.
  \item[ypret] (\id{N\_Vector})
    the computed solution vector $y^\prime$.
  \item[itask] (\id{int})
    \index{itask@\texttt{itask}}
    a flag indicating the job of the solver for the next user step. 
    The \Id{IDA\_NORMAL} task is to have the solver take internal steps until   
    it has reached or just passed the user specified \id{tout}
    parameter. The solver then interpolates in order to   
    return approximate values of $y($\id{tout}$)$ and $y^\prime($\id{tout}$)$. 
    The \Id{IDA\_ONE\_STEP} option tells the solver to just take one internal step  
    and return the solution at the point reached by that step. 
    The \Id{IDA\_NORMAL\_TSTOP} and \Id{IDA\_ONE\_STEP\_TSTOP} modes are     
    similar to \id{IDA\_NORMAL} and \id{IDA\_ONE\_STEP}, respectively, except    
    that the integration never proceeds past the value      
    \id{tstop} (specified through the function \id{IDASetStopTime}).
  \end{args}
}
{
  On return, \id{IDASolve} returns vectors \id{yret} and \id{ypret} and a corresponding 
  independent variable value $t=$\id{*tret}, such that (\id{yret}, \id{ypret}) are 
  the computed values of ($y(t)$, $y^\prime(t)$).

  In \id{NORMAL} mode with no errors, \id{*tret} will be equal to \id{tout} 
  and \id{yret} = $y($\id{tout}$)$, \id{ypret} = $y^\prime($\id{tout}$)$.

  The return value \id{flag} (of type \id{int}) will be one of the following:
  \begin{args}[IDA\_TOO\_MUCH\_WORK]
  \item[\Id{IDA\_SUCCESS}]
    \id{IDASolve} succeeded.
  \item[\Id{IDA\_TSTOP\_RETURN}]
    \id{IDASolve} succeeded by reaching the stop point specified through
    the optional input function \id{IDASetStopTime} (see \S\ref{ss:optional_input}).
  \item[\Id{IDA\_MEM\_NULL}]
    The \id{ida\_mem} argument was \id{NULL}.
  \item[\Id{IDA\_ILL\_INPUT}]
    One of the inputs to \id{IDASolve} is illegal. This includes the situation when a 
    component of the error weight vectors becomes negative during internal 
    time-stepping. The \id{IDA\_ILL\_INPUT} flag will also be returned if the linear 
    solver function initialization (called by the user after calling 
    \id{IDACreate}) failed to set the linear solver-specific \id{lsolve} field
    in \id{ida\_mem}. 
    In any case, the user should see the printed error message for more details.
  \item[\Id{IDA\_TOO\_MUCH\_WORK}] 
    The solver took \id{mxstep} internal steps but could not reach tout. 
    The default value for \id{mxstep} is \id{MXSTEP\_DEFAULT = 500}.
  \item[\Id{IDA\_TOO\_MUCH\_ACC}] 
    The solver could not satisfy the accuracy demanded by the user for some 
    internal step.
  \item[\Id{IDA\_ERR\_FAIL}]
    Error test failures occurred too many times (\id{MXNEF = 10}) during one 
    internal time step or occurred with $|h| = h_{min}$.
  \item[\Id{IDA\_CONV\_FAIL}] 
    Convergence test failures occurred too many times (\id{MXNCF = 10}) during 
    one internal time step or occurred with $|h| = h_{min}$.             
  \item[\Id{IDA\_LINIT\_FAIL}]
    The linear solver's initialization function failed.   
  \item[\Id{IDA\_LSETUP\_FAIL}] 
    The linear solver's setup function failed in an unrecoverable manner.
  \item[\Id{IDA\_LSOLVE\_FAIL}] 
    The linear solver's solve function failed in an unrecoverable manner.
  \item[\Id{IDA\_CONSTR\_FAIL}]
    The inequality constraints were violated and the solver was unable
    to recover.
  \item[\Id{IDA\_REP\_RES\_ERR}]
    The user's residual function repeatedly returned a recoverable error
    flag, but the solver was unable to recover.
  \item[\Id{IDA\_RES\_FAIL}]
    The user's residual function returned a nonrecoverable error flag.
  \end{args} 
}
{
  The vector \id{yret} can occupy the same space as the \id{y0} vector of 
  initial conditions that was passed to \id{IDAMalloc}, while the
  vector \id{ypret} can occupy the same space as the \id{yp0}.

  In the \id{IDA\_ONE\_STEP} mode, \id{tout} is used on the first call only, 
  to get the direction and rough scale of the independent variable.

  All failure return values are negative and therefore a test \id{flag}$< 0$
  will trap all \id{IDASolve} failures.
}

%%===================================================================================

\subsection{Optional input functions}\label{ss:optional_input}

{\ida} provides an extensive list of functions that can be used to change
from their default values various optional input parameters that control the
behavior of the {\ida} solver. 
Table \ref{t:optional_input} lists all optional input functions in {\ida} which 
are then described in detail in the remainder of this section.
For the most casual use of {\ida}, the reader can skip to \S\ref{ss:user_fct_sim}.

We note that, on error return, all these functions also print an error message to 
\id{stderr} (or to the file pointed to by \id{errfp} if already specified).
\index{error message}
We also note that all error return values are negative, so a test \id{flag}$<0$
will catch any error.

\begin{table}
\centering
\caption{Optional inputs for {\ida}, {\idadense}, {\idaband}, and {\idaspgmr}}
\label{t:optional_input}
\medskip
\begin{tabular}{|l|l|l|}\hline
{\bf Optional input} & {\bf Function name} & {\bf Default} \\
\hline
\multicolumn{3}{|c|}{\bf IDA main solver} \\
\hline
Pointer to an error file & \id{IDASetErrFile} & \id{stderr}  \\
Data for residual function & \id{IDASetRdata} & \id{NULL} \\
Maximum order for BDF method & \id{IDASetMaxOrd} & 5 \\
Maximum no. of internal steps before $t_{\mbox{\scriptsize out}}$ & \id{IDASetMaxNumSteps} & 500 \\
Initial step size & \id{IDASetInitStep} & estimated \\
Maximum absolute step size & \id{IDASetMaxStep} & $\infty$ \\
Value of $t_{stop}$ & \id{IDASetStopTime} & $\infty$ \\
Maximum no. of error test failures & \id{IDAMaxErrTestFails} & 10 \\
Maximum no. of nonlinear iterations & \id{IDASetMaxNonlinIters} & 4 \\
Maximum no. of convergence failures & \id{IDASetMaxConvFails} & 10 \\
Maximum no. of error test failures & \id{IDASetMaxErrTestFails} & 7 \\
Coeff. in the nonlinear convergence test & \id{IDASetNonlinConvCoef} & 0.33 \\
Suppress alg. vars. from error test & \id{IDASetSuppressAlg} & \id{FALSE} \\
Variable types (differential/algebraic) & \id{IDASetId} & \id{NULL} \\
Inequality constraints on solution & \id{IDASetConstraints} & \id{NULL} \\
Integration tolerances & \id{IDASetTolerances} & none \\
\hline
\multicolumn{3}{|c|}{\bf IDA initial conditions calculation} \\
\hline
Coeff. in the nonlinear convergence test & \id{IDASetNonlinConvCoefIC} & 0.0033 \\
Maximum no. of steps & \id{IDASetMaxNumStepsIC} & 5 \\
Maximum no. of Jacobian/precond. evals. & \id{IDASetMaxNumJacsIC} & 4 \\
Maximum no. of Newton iterations & \id{IDASetMaxNumItersIC} & 10 \\
Turn off linesearch & \id{IDASetLineSearchOffIC} & \id{FALSE} \\
Lower bound on Newton step & \id{IDASetStepToleranceIC} & $(2/3) \text{uround}$ \\ 
\hline
\multicolumn{3}{|c|}{\bf IDADENSE linear solver} \\
\hline
Dense Jacobian function & \id{IDADenseSetJacFn} & internal DQ \\
Data for Jacobian function & \id{IDADenseSetJacData} & NULL \\
\hline
\multicolumn{3}{|c|}{\bf IDABAND linear solver} \\
\hline
Band Jacobian function & \id{IDABandSetJacFn} & internal DQ \\
Data for Jacobian function & \id{IDABandSetJacData} & NULL \\
\hline
\multicolumn{3}{|c|}{\bf IDASPGMR linear solver} \\
\hline
Preconditioner solve function & \id{IDASpgmrSetPrecSolveFn} & NULL \\
Preconditioner setup function & \id{IDASpgmrSetPrecSetupFn} & NULL \\
Data for preconditioner functions & \id{IDASpgmrSetPrecData} & NULL \\
Jacobian times vector function & \id{IDASpgmrSetJacTimesVecFn} & NULL \\
Data for Jacobian times vector function &\id{IDASpgmrSetJacData} & NULL \\ 
Type of Gram-Schmidt orthogonalization & \id{IDASpgmrSetGSType} & classical GS \\
Maximum no. of restarts & \id{IDASpgmrSetMaxRestarts} & 5 \\
Factor in linear convergence test & \id{IDASpgmrSetEpsLin} & 0.05 \\
Factor in DQ increment calculation & \id{IDASpgmrSetIncrementFactor} & 1.0 \\
\hline
\end{tabular}
\end{table}

\subsubsection{Main solver optional input functions}
\index{optional input!solver|(}
The calls listed here can be executed in any order. However, if \id{IDASetErrFile} 
is to be called, that call should be first, in order to take effect for any later 
error message.

\index{error message}
\ucfunction{IDASetErrFile}
{
flag = IDASetErrFile(ida\_mem, errfp);
}
{
  The function \ID{IDASetErrFile} specifies the pointer to the file
  where all {\ida} messages should be directed.
}
{
  \begin{args}[ida\_mem]
  \item[ida\_mem] (\id{void *})
    pointer to the {\ida} memory block.
  \item[errfp] (\id{FILE *})
    pointer to output file.
  \end{args}
}
{
  The return value \id{flag} (of type \id{int}) is one of
  \begin{args}[IDA\_MEM\_NULL]
  \item[\Id{IDA\_SUCCESS}] 
    The optional value has been successfully set.
  \item[\Id{IDA\_MEM\_NULL}]
    The \id{ida\_mem} pointer is \id{NULL}.
  \end{args}
}
{
  The default value for \id{errfp} is \id{stderr}.

  Passing a value \id{NULL} disables all future error message output
  (except for the case in which the {\ida} memory pointer is \id{NULL}).
}
%%
%%
\ucfunction{IDASetRdata}
{
  flag = IDASetRdata(ida\_mem, res\_data);
}
{
  The function \ID{IDASetRdata} specifies the user data block \Id{res\_data}
  and attaches it to the main {\ida} memory block.
}
{
  \begin{args}[res\_data]
  \item[ida\_mem] (\id{void *})
    pointer to the {\ida} memory block.
  \item[res\_data] (\id{void *})
    pointer to the user data.
  \end{args}
}
{
  The return value \id{flag} (of type \id{int}) is one of
  \begin{args}[IDA\_MEM\_NULL]
  \item[\Id{IDA\_SUCCESS}] 
    The optional value has been successfully set.
  \item[\Id{IDA\_MEM\_NULL}]
    The \id{ida\_mem} pointer is \id{NULL}.
  \end{args}
}
{
  If \id{res\_data} is not specified, a \id{NULL} pointer is
  passed to all user functions that have it as an argument.
}
%%
%%
\ucfunction{IDASetMaxOrd}
{
flag = IDASetMaxOrder(ida\_mem, maxord);
}
{
  The function \ID{IDASetMaxOrder} specifies the maximum order of the 
  linear multistep method.
}
{
  \begin{args}[ida\_mem]
  \item[ida\_mem] (\id{void *})
    pointer to the {\ida} memory block.
  \item[maxord] (\id{int})
    value of the maximum method order.
  \end{args}
}
{
  The return value \id{flag} (of type \id{int}) is one of
  \begin{args}[IDA\_ILL\_INPUT]
  \item[\Id{IDA\_SUCCESS}] 
    The optional value has been successfully set.
  \item[\Id{IDA\_MEM\_NULL}]
    The \id{ida\_mem} pointer is \id{NULL}.
  \item[\Id{IDA\_ILL\_INPUT}]
    The specified value \id{maxord} is negative, or larger than 
    its previous value.
  \end{args}
}
{
  The default value is $5$.
  Since \id{maxord} affects the memory requirements
  for the internal {\ida} memory block, its value
  can not be increased past its previous value.
}
%%
%%
\ucfunction{IDASetMaxNumSteps}
{
flag = IDASetMaxNumSteps(ida\_mem, mxsteps);
}
{
  The function \ID{IDASetMaxNumSteps} specifies the maximum number
  of steps to be taken by the solver in its attempt to reach 
  the final time.
}
{
  \begin{args}[ida\_mem]
  \item[ida\_mem] (\id{void *})
    pointer to the {\ida} memory block.
  \item[mxsteps] (\id{long int})
    maximum allowed number of steps.
  \end{args}
}
{
  The return value \id{flag} (of type \id{int}) is one of
  \begin{args}[IDA\_ILL\_INPUT]
  \item[\Id{IDA\_SUCCESS}] 
    The optional value has been successfully set.
  \item[\Id{IDA\_MEM\_NULL}]
    The \id{ida\_mem} pointer is \id{NULL}.
  \item[\Id{IDA\_ILL\_INPUT}]
    \id{mxsteps} is non-positive.
  \end{args}
}
{
  The default value is $500$.
}
%%
%%
\ucfunction{IDASetInitStep}
{
flag = IDASetInitStep(ida\_mem, hin);
}
{
  The function \ID{IDASetInitStep} specifies the initial step size.
}
{
  \begin{args}[ida\_mem]
  \item[ida\_mem] (\id{void *})
    pointer to the {\ida} memory block.
  \item[hin] (\id{realtype})
    value of the initial step size.
  \end{args}
}
{
  The return value \id{flag} (of type \id{int}) is one of
  \begin{args}[IDA\_MEM\_NULL]
  \item[\Id{IDA\_SUCCESS}] 
    The optional value has been successfully set.
  \item[\Id{IDA\_MEM\_NULL}]
    The \id{ida\_mem} pointer is \id{NULL}.
  \end{args}
}
{
  By default, {\ida} estimates the initial step as solution of 
  $\| 0.5 h^2 \ddot y \|_{\mbox{\scriptsize WRMS}} = 1$,
  where $\ddot y$ is an estimated second derivative of the solution at the
  initial time.
}
%%
%%
\index{step size bounds|(}
\ucfunction{IDASetMinStep}
{
flag = IDASetMinStep(ida\_mem, hmin);
}
{
  The function \ID{IDASetMinStep} specifies the minimum absolute
  value of the step size.
}
{
  \begin{args}[ida\_mem]
  \item[ida\_mem] (\id{void *})
    pointer to the {\ida} memory block.
  \item[hmin] (\id{realtype})
    minimum absolute value of the step size.
  \end{args}
}
{
  The return value \id{flag} (of type \id{int}) is one of
  \begin{args}[IDA\_ILL\_INPUT]
  \item[\Id{IDA\_SUCCESS}] 
    The optional value has been successfully set.
  \item[\Id{IDA\_MEM\_NULL}]
    The \id{ida\_mem} pointer is \id{NULL}.
  \item[\Id{IDA\_ILL\_INPUT}]
    Either \id{hmin} is not positive or it is larger than the maximum allowable step.
  \end{args}
}
{
  The default value is $0.0$.
}
%%
%%
\ucfunction{IDASetMaxStep}
{
flag = IDASetMaxStep(ida\_mem, hmax);
}
{
  The function \ID{IDASetMaxStep} specifies the maximum absolute
  value of the step size.
}
{
  \begin{args}[ida\_mem]
  \item[ida\_mem] (\id{void *})
    pointer to the {\ida} memory block.
  \item[hmax] (\id{realtype})
    maximum absolute value of the step size.
  \end{args}
}
{
  The return value \id{flag} (of type \id{int}) is one of
  \begin{args}[IDA\_ILL\_INPUT]
  \item[\Id{IDA\_SUCCESS}] 
    The optional value has been successfully set.
  \item[\Id{IDA\_MEM\_NULL}]
    The \id{ida\_mem} pointer is \id{NULL}.
  \item[\Id{IDA\_ILL\_INPUT}]
    Either \id{hmax} is not positive or it is smaller than the minimum allowable step.
  \end{args}
}
{
  The default value is $\infty$.
}
\index{step size bounds|)}
%%
%%
\ucfunction{IDASetStopTime}
{
flag = IDASetStopTime(ida\_mem, tstop);
}
{
  The function \ID{IDASetStopTime} specifies the value of the
  independent variable $t$ past which the solution is not to proceed.
}
{
  \begin{args}[ida\_mem]
  \item[ida\_mem] (\id{void *})
    pointer to the {\ida} memory block.
  \item[tstop] (\id{realtype})
    value of the independent variable past which the solution should
    not proceed.
  \end{args}
}
{
  The return value \id{flag} (of type \id{int}) is one of
  \begin{args}[IDA\_MEM\_NULL]
  \item[\Id{IDA\_SUCCESS}] 
    The optional value has been successfully set.
  \item[\Id{IDA\_MEM\_NULL}]
    The \id{ida\_mem} pointer is \id{NULL}.
  \end{args}
}
{
  The default value is $\infty$.
}
%%
%%
\ucfunction{IDASetMaxErrTestFails}
{
flag = IDASetMaxErrTestFails(ida\_mem, maxnef);
}
{
  The function \ID{IDASetMaxErrTestFails} specifies the
  maximum number of error test failures in attempting one step.
}
{
  \begin{args}[ida\_mem]
  \item[ida\_mem] (\id{void *})
    pointer to the {\ida} memory block.
  \item[maxnef] (\id{int})
    maximum number of error test failures allowed on one step.
  \end{args}
}
{
  The return value \id{flag} (of type \id{int}) is one of
  \begin{args}[IDA\_MEM\_NULL]
  \item[\Id{IDA\_SUCCESS}] 
    The optional value has been successfully set.
  \item[\Id{IDA\_MEM\_NULL}]
    The \id{ida\_mem} pointer is \id{NULL}.
  \end{args}
}
{
  The default value is $7$.
}
%%
%%
\ucfunction{IDASetMaxNonlinIters}
{
flag = IDASetMaxNonlinIters(ida\_mem, maxcor);
}
{
  The function \ID{IDASetNonlinIters} specifies the maximum
  number of nonlinear solver iterations at one step.
}
{
  \begin{args}[ida\_mem]
  \item[ida\_mem] (\id{void *})
    pointer to the {\ida} memory block.
  \item[maxcor] (\id{int})
    maximum number of nonlinear solver iterations allowed on one step.
  \end{args}
}
{
  The return value \id{flag} (of type \id{int}) is one of
  \begin{args}[IDA\_MEM\_NULL]
  \item[\Id{IDA\_SUCCESS}] 
    The optional value has been successfully set.
  \item[\Id{IDA\_MEM\_NULL}]
    The \id{ida\_mem} pointer is \id{NULL}.
  \end{args}
}
{
  The default value is $3$.
}
%%
%%
\ucfunction{IDASetMaxConvFails}
{
flag = IDASetMaxConvFails(ida\_mem, maxncf);
}
{
  The function \ID{IDASetMaxConvFails} specifies the
  maximum number of nonlinear solver convergence failures at one step.
}
{
  \begin{args}[ida\_mem]
  \item[ida\_mem] (\id{void *})
    pointer to the {\ida} memory block.
  \item[maxncf] (\id{int})
    maximum number of allowable nonlinear solver convergence failures
    on one step.
  \end{args}
}
{
  The return value \id{flag} (of type \id{int}) is one of
  \begin{args}[IDA\_MEM\_NULL]
  \item[\Id{IDA\_SUCCESS}] 
    The optional value has been successfully set.
  \item[\Id{IDA\_MEM\_NULL}]
    The \id{ida\_mem} pointer is \id{NULL}.
  \end{args}
}
{
  The default value is $10$.
}
%%
%%
\ucfunction{IDASetNonlinConvCoef}
{
flag = IDASetNonlinConvCoef(ida\_mem, nlscoef);
}
{
  The function \ID{IDASetNonlinConvCoef} specifies the safety factor
  in the nonlinear convergence test (see \S\ref{s:math}).
}
{
  \begin{args}[ida\_mem]
  \item[ida\_mem] (\id{void *})
    pointer to the {\ida} memory block.
  \item[nlscoef] (\id{realtype})
    coefficient in nonlinear convergence test.
  \end{args}
}
{
  The return value \id{flag} (of type \id{int}) is one of
  \begin{args}[IDA\_MEM\_NULL]
  \item[\Id{IDA\_SUCCESS}] 
    The optional value has been successfully set.
  \item[\Id{IDA\_MEM\_NULL}]
    The \id{ida\_mem} pointer is \id{NULL}.
  \end{args}
}
{
  The default value is $0.1$.
}
%%
%%
\ucfunction{IDASetSuppressAlg}
{
flag = IDASetSuppressAlg(ida\_mem, suppressalg);
}
{
  The function \ID{IDASetNonlinConvCoef} indicates whether or not to
  suppress algebraic variables in the local error test.
}
{
  \begin{args}[suppressalg]
  \item[ida\_mem] (\id{void *})
    pointer to the {\ida} memory block.
  \item[suppresslag] (\id{booleantype})
    indicates whether to suppress (\id{TRUE}) or not
    (\id{FALSE}) the algebraic variables in the local error test.
  \end{args}
}
{
  The return value \id{flag} (of type \id{int}) is one of
  \begin{args}[IDA\_MEM\_NULL]
  \item[\Id{IDA\_SUCCESS}] 
    The optional value has been successfully set.
  \item[\Id{IDA\_MEM\_NULL}]
    The \id{ida\_mem} pointer is \id{NULL}.
  \end{args}
}
{
  The default value is \id{FALSE}.

  If \id{suppresslag}$=$\id{TRUE} is selected, then the \id{id} vector
  must be set (through \id{IDASetId}) to specify the algebraic components.
}
%%
%%
\ucfunction{IDASetId}
{
flag = IDASetId(ida\_mem, id);
}
{
  The function \ID{IDASetId} specifies algebraic/differential
  components in the $y$ vector.
}
{
  \begin{args}[ida\_mem]
  \item[ida\_mem] (\id{void *})
    pointer to the {\ida} memory block.
  \item[id] (\id{N\_Vector})
    state vector. A value of $1.0$ indicates an algebraic variable.
  \end{args}
}
{
  The return value \id{flag} (of type \id{int}) is one of
  \begin{args}[IDA\_MEM\_NULL]
  \item[\Id{IDA\_SUCCESS}] 
    The optional value has been successfully set.
  \item[\Id{IDA\_MEM\_NULL}]
    The \id{ida\_mem} pointer is \id{NULL}.
  \end{args}
}
{
  The vector \id{id} is required if the algebraic variables are to be 
  suppressed from the local error test (see \id{IDASetSuppressAlg}) or
  if \id{IDACalcIC} is to be called with \id{icopt} $=$ \id{CALC\_YA\_YDP\_INIT}
  (see \S\ref{ss:idacalcic}).
}
%%
%%
\ucfunction{IDASetConstraints}
{
flag = IDASetConstraints(ida\_mem, constraints);
}
{
  The function \ID{IDASetConstraints} specifies a vector defining
  inequality constraints for each component of the solution vector $y$.
}
{
  \begin{args}[ida\_mem]
  \item[ida\_mem] (\id{void *})
    pointer to the {\ida} memory block.
  \item[constraints] (\id{N\_Vector})
    vector of constraint flags. If \id{constraints[i]} is
    \begin{itemize}    
    \item[$0.0$] then no constraint is imposed on $y_i$. 
    \item[$1.0$] then $y_i$ will be constrained to be $y_i > 0.0$.
    \item[$-1.0$] then $y_i$ will be constrained to be $y_i < 0.0$.
    \item[$2.0$] then $y_i$ will be constrained to be $y_i \ge 0.0$.
    \item[$-2.0$] then $y_i$ will be constrained to be $y_i \le 0.0$.
    \end{itemize}
  \end{args}
}
{
  The return value \id{flag} (of type \id{int}) is one of
  \begin{args}[IDA\_MEM\_NULL]
  \item[\Id{IDA\_SUCCESS}] 
    The optional value has been successfuly set.
  \item[\Id{IDA\_MEM\_NULL}]
    The \id{ida\_mem} pointer is \id{NULL}.
  \end{args}
}
{
  The presence of a non-\id{NULL} constraints vector that is not $0.0$ in
  all components will cause constraint checking to be performed.
}
%%
\ucfunction{IDASetTolerances}
{
flag = IDASetTolerances(ida\_mem, itol, reltol, abstol);
}
{
  The function \ID{IDASetTolerances} resets the integration tolerances.
}
{
  \begin{args}[ida\_mem]
  \item[ida\_mem] (\id{void *})
    pointer to the {\cvodes} memory block.
  \item[itol] (\id{int}) 
    is either \ID{IDA\_SS} or \ID{IDA\_SV}, where \ID{itol}$=$\id{IDa\_SS} indicates scalar
    relative error tolerance and scalar absolute error tolerance, while
    \id{itol}$=$\id{IDA\_SV} indicates scalar relative error tolerance and vector
    absolute error tolerance.  The latter choice is important when the absolute
    error tolerance needs to be different for each component of the DAE. 
  \item[reltol] (\id{realtype *})
    \index{tolerances}
    is a pointer to the relative error tolerance.
  \item[abstol] (\id{void *})
    is a pointer to the absolute error tolerance.
  \end{args}
}
{
  The return value \id{flag} (of type \id{int}) is one of
  \begin{args}[IDA\_ILL\_INPUT]
  \item[\Id{IDA\_SUCCESS}] 
    The tolerances have been successfully set.
  \item[\Id{IDA\_MEM\_NULL}]
    The \id{ida\_mem} pointer is \id{NULL}.
  \item[\Id{IDA\_ILL\_INPUT}]
    An input argument has an illegal value.
  \end{args}
}
{
  The integration tolerances are initially specified in the call
  to \id{IDAMalloc} (see \S\ref{sss:idamalloc}). This function call is
  needed only if the tolerances are being changed from their values between
  succesive calls to \id{IDASolve}.
}
%%
\index{optional input!solver|)}
%%
%%
%%===================================================================================
%%
\subsubsection{Linear solver optional input functions}
%%
The linear solver modules, with one exception, allow for various optional 
inputs, which are described here. The diagonal linear solver module has no
optional inputs.
%%
\noindent\paragraph{\bf Dense Linear solver.}
\index{optional input!dense linear solver|(}
\index{IDADENSE@{\idadense} linear solver!optional input|(}
The \index{IDADENSE@{\idadense} linear solver!Jacobian approximation used by}
{\idadense} solver needs a function to compute a dense approximation to
the Jacobian matrix $J(t,y)$.  This function must be of type \id{IDADenseJacFn}. 
The user can supply his/her own dense Jacobian function, or use the default 
difference quotient function \Id{IDADenseDQJac} 
\index{Jacobian approximation function!dense!difference quotient}
that comes with the {\idadense} solver.
To specify a user-supplied Jacobian function \id{djac} and associated user 
data \id{jac\_data}, {\idadense} provides the functions \id{IDADenseSetJacFn}
and \id{IDADenseSetJacData}, respectively.
The {\idadense} solver passes the pointer it receives through \id{IDADenseSetJacData} 
to its dense Jacobian function. This allows the user to
create an arbitrary structure with relevant problem data and access it
during the execution of the user-supplied Jacobian function, without
using global data in the program.  The pointer \id{jac\_data} may be
identical to \id{res\_data}, if the latter was specified through \id{IDASetRdata}.
%%
\index{Jacobian approximation function!dense!user-supplied}
\ucfunction{IDADenseSetJacFn}
{
  flag = IDADenseSetJacFn(ida\_mem, djac);
}
{
  The function \ID{IDADenseSetJacFn} specifies the dense Jacobian
  approximation function to be used.
}
{
  \begin{args}[ida\_mem]
  \item[ida\_mem] (\id{void *})
    pointer to the {\ida} memory block.
  \item[djac] (\id{IDADenseJacFn})
    user-defined dense Jacobian approximation function.
  \end{args}
}
{
  The return value \id{flag} (of type \id{int}) is one of
  \begin{args}[IDADENSE\_LMEM\_NULL]
  \item[\Id{IDADENSE\_SUCCESS}] 
    The optional value has been successfuly set.
  \item[\Id{IDADENSE\_MEM\_NULL}]
    The \id{ida\_mem} pointer is \id{NULL}.
  \item[\Id{IDADENSE\_LMEM\_NULL}]
    The {\idadense} linear solver has not been initialized.
  \end{args}
}
{
  By default, {\idadense} uses the difference quotient function \id{IDADenseDQJac}.
  If \id{NULL} is passed to \id{djac}, this default function is used.

  The function type \id{IDADenseJacFn} is described in \S\ref{ss:djacFn}.
}
%%
\ucfunction{IDADenseSetJacData}
{
  flag = IDADenseSetJacData(ida\_mem, jac\_data);
}
{
  The function \ID{IDADenseSetJacData} specifies the data structure
  to be passed to the user supplied dense Jacobian approximation 
  function each time it is called.
}
{
  \begin{args}[jac\_data]
  \item[ida\_mem] (\id{void *})
    pointer to the {\ida} memory block.
  \item[jac\_data] (\id{void *})
    pointer to the user-defined data structure.
  \end{args}
}
{
  The return value \id{flag} (of type \id{int}) is one of
  \begin{args}[IDADENSE\_LMEM\_NULL]
  \item[\Id{IDADENSE\_SUCCESS}] 
    The optional value has been successfuly set.
  \item[\Id{IDADENSE\_MEM\_NULL}]
    The \id{ida\_mem} pointer is \id{NULL}.
  \item[\Id{IDADENSE\_LMEM\_NULL}]
    The {\idadense} linear solver has not been initialized.
  \end{args}
}
{}
\index{IDADENSE@{\idadense} linear solver!optional input|)}
\index{optional input!dense linear solver|)}
%%
%%---------------------------------------------------------
%%
\noindent\paragraph{\bf Band Linear solver.}
\index{optional input!band linear solver|(}
\index{IDABAND@{\idaband} linear solver!optional input|(}
The \index{IDABAND@{\idaband} linear solver!Jacobian approximation used by}
{\idadense} solver needs a function to compute a banded approximation to
the Jacobian matrix $J(t,y)$.  This function must be of type \id{IDABandJacFn}. 
The user can supply his/her own banded Jacobian approximation function, 
or use the default difference quotient function \Id{IDABandDQJac} 
\index{Jacobian approximation function!band!difference quotient}
that comes with the {\idaband} solver.
To specify a user-supplied Jacobian function \id{bjac} and associated user 
data \id{jac\_data}, {\idaband} provides the functions \id{IDABandSetJacFn}
and \id{IDABandSetJacData}, respectively.
The {\idaband} solver passes the pointer it receives through \id{IDABandSetJacData} 
to its banded Jacobian approximation function. This allows the user to
create an arbitrary structure with relevant problem data and access it
during the execution of the user-supplied Jacobian function, without
using global data in the program.  The pointer \id{jac\_data} may be
identical to \id{res\_data}, if the latter was specified through \id{IDASetRdata}.
%%
\index{Jacobian approximation function!band!user-supplied}
\ucfunction{IDABandSetJacFn}
{
  flag = IDABandSetJacFn(ida\_mem, bjac);
}
{
  The function \ID{IDABandSetJacFn} specifies the banded Jacobian
  approximation function to be used.
}
{
  \begin{args}[ida\_mem]
  \item[ida\_mem] (\id{void *})
    pointer to the {\ida} memory block.
  \item[bjac] (\id{IDABandJacFn})
    user-defined banded Jacobian approximation function.
  \end{args}
}
{
  The return value \id{flag} (of type \id{int}) is one of
  \begin{args}[IDABAND\_LMEM\_NULL]
  \item[\Id{IDABAND\_SUCCESS}] 
    The optional value has been successfuly set.
  \item[\Id{IDABAND\_MEM\_NULL}]
    The \id{ida\_mem} pointer is \id{NULL}.
  \item[\Id{IDABAND\_LMEM\_NULL}]
    The {\idaband} linear solver has not been initialized.
  \end{args}
}
{
  By default, {\idaband} uses the difference quotient function \id{IDABandDQJac}.
  If \id{NULL} is passed to \id{bjac}, this default function is used.

  The function type \id{IDABandJacFn} is described in \S\ref{ss:bjacFn}.
}
%%
\ucfunction{IDABandSetJacData}
{
  flag = IDABandSetJacData(ida\_mem, jac\_data);
}
{
  The function \ID{IDABandSetJacData} specifies the data structure
  to be passed to the user supplied banded Jacobian approximation 
  function each time it is called.
}
{
  \begin{args}[jac\_data]
  \item[ida\_mem] (\id{void *})
    pointer to the {\ida} memory block.
  \item[jac\_data] (\id{void *})
    pointer to the user-defined data structure.
  \end{args}
}
{
  The return value \id{flag} (of type \id{int}) is one of
  \begin{args}[IDABAND\_LMEM\_NULL]
  \item[\Id{IDABAND\_SUCCESS}] 
    The optional value has been successfuly set.
  \item[\Id{IDABAND\_MEM\_NULL}]
    The \id{ida\_mem} pointer is \id{NULL}.
  \item[\Id{IDABAND\_LMEM\_NULL}]
    The {\idadense} linear solver has not been initialized.
  \end{args}
}
{}
\index{IDABAND@{\idaband} linear solver!optional input|)}
\index{optional input!band linear solver|)}
%%
%%---------------------------------------------------------
%%
\noindent\paragraph{\bf SPGMR Linear solver.}
\index{optional input!iterative linear solver|(}
\index{IDASPGMR@{\idaspgmr} linear solver!optional input|(}
\index{preconditioning!user-supplied|(}
The call to \id{IDASpgmr} is used to communicate the the maximum dimension of 
the Krylov subspace to be used (\id{maxl}). 

If preconditioning is to be done within the {\spgmr} method,
then the user must supply a preconditioner solve function \id{psolve}
and specify it through a call to \id{IDASpgmrSetPrecSolveFn}.
\index{IDASPGMR@{\idaspgmr} linear solver!preconditioner solve function}
%%
The evaluation and preprocessing of any Jacobian-related data needed
by the user's preconditioner solve function is done in the optional
user-supplied function \id{psetup}. Both of these functions are
fully specified in \S\ref{ss:user_fct_sim}.
If used, the \id{psetup} function should be specified through a call to
\id{IDASpgmrSetPrecSetupFn}.
\index{IDASPGMR@{\idaspgmr} linear solver!preconditioner setup function}
%%
Optionally, the {\idaspgmr} solver passes the pointer it receives through 
\id{IDASpgmrSetPrecData} to the preconditioner setup and solve functions.  
This allows the user to create an arbitrary structure with relevant problem data 
and access it during the execution of the user-supplied preconditioner functions
without using global data in the program.  
The pointer \id{prec\_data} may be identical to \id{res\_data}, if the latter was 
specified through \id{IDASetRdata}.

The \index{IDASPGMR@{\idaspgmr} linear solver!Jacobian approximation used by}
{\idaspgmr} solver requires a function to compute an approximation to the
product between the Jacobian matrix $J(t,y)$ and a vector $v$.
The user can supply his/her own Jacobian times vector approximation function, 
or use the difference quotient function \Id{IDASpgmrDQJtimes} 
\index{Jacobian approximation function!Jacobian times vector!difference quotient}
that comes with the {\idaspgmr} solver.
A user-defined Jacobian-vector function must be of type \id{IDASpgmrJtimesFn} and 
can be specified through a call to \id{IDASpgmrSetJacTimesVecFn} 
(see \S\ref{ss:user_fct_sim} for specification details).
%%
As with the preconditioner user data structure \id{prec\_data}, 
the user can specify, through a call to \id{IDASpgmrSetJacData}, a pointer to a 
user-defined data structure, \id{jac\_data}, which
the {\idaspgmr} solver passes to the Jacobian times vector function \id{jtimes} each
time it is called.  
The pointer \id{jac\_data} may be identical to \id{prec\_data} and/or \id{res\_data}.
%%
%%
\ucfunction{IDASpgmrSetPrecSolveFn}
{
  flag = IDASpgmrSetPrecSolveFn(ida\_mem, psolve);
}
{
  The function \ID{IDASpgmrSet} specifies the preconditioner
  solve function.
}
{
  \begin{args}[ida\_mem]
  \item[ida\_mem] (\id{void *})
    pointer to the {\ida} memory block.
  \item[psolve] (\id{IDASpgmrPrecSolveFn})
    user-defined preconditioner solve function.
  \end{args}
}
{
  The return value \id{flag} (of type \id{int}) is one of
  \begin{args}[IDASPGMR\_LMEM\_NULL]
  \item[\Id{IDASPGMR\_SUCCESS}] 
    The optional value has been successfuly set.
  \item[\Id{IDASPGMR\_MEM\_NULL}]
    The \id{ida\_mem} pointer is \id{NULL}.
  \item[\Id{IDASPGMR\_LMEM\_NULL}]
    The {\idaspgmr} linear solver has not been initialized.
  \end{args}
}
{
   The function type \id{IDASpgmrPrecSolveFn} is described in \S\ref{ss:psolveFn}.
}
%%
%%
\ucfunction{IDASpgmrSetPrecSetupFn}
{
  flag = IDASpgmrSetPrecSetupFn(ida\_mem, psetup);
}
{
  The function \ID{IDASpgmrSetPrecSetupFn} specifies the preconditioner
  preprocessing function.
}
{
  \begin{args}[ida\_mem]
  \item[ida\_mem] (\id{void *})
    pointer to the {\ida} memory block.
  \item[psetup] (\id{IDASpgmrPrecSetupFn})
    user-defined preconditioner setup function.
  \end{args}
}
{
  The return value \id{flag} (of type \id{int}) is one of
  \begin{args}[IDASPGMR\_LMEM\_NULL]
  \item[\Id{IDASPGMR\_SUCCESS}] 
    The optional value has been successfuly set.
  \item[\Id{IDASPGMR\_MEM\_NULL}]
    The \id{ida\_mem} pointer is \id{NULL}.
  \item[\Id{IDASPGMR\_LMEM\_NULL}]
    The {\idaspgmr} linear solver has not been initialized.
  \end{args}
}
{
   The function type \id{IDASpgmrPrecSetupFn} is described in \S\ref{ss:precondFn}.
}
%%
%%
\ucfunction{IDASpgmrSetPrecData}
{
  flag = IDASpgmrSetPrecData(ida\_mem, prec\_data);
}
{
  The function \ID{IDASpgmrSetPrecData} specifies the data structure
  to be passed to the user supplied preconditioner setup and solve
  functions each time they are called.
}
{
  \begin{args}[prec\_data]
  \item[ida\_mem] (\id{void *})
    pointer to the {\ida} memory block.
  \item[prec\_data] (\id{void *})
     pointer to the user-defined data structure.
  \end{args}
}
{
  The return value \id{flag} (of type \id{int}) is one of
  \begin{args}[IDASPGMR\_LMEM\_NULL]
  \item[\Id{IDASPGMR\_SUCCESS}] 
    The optional value has been successfuly set.
  \item[\Id{IDASPGMR\_MEM\_NULL}]
    The \id{ida\_mem} pointer is \id{NULL}.
  \item[\Id{IDASPGMR\_LMEM\_NULL}]
    The {\idaspgmr} linear solver has not been initialized.
  \end{args}
}
{}
%%
\index{preconditioning!user-supplied|)}
%%
\index{Jacobian approximation function!Jacobian times vector!user-supplied}
\ucfunction{IDASpgmrSetJacTimesVecFn}
{
  flag = IDASpgmrSetJacTimesVecFn(ida\_mem, jtimes);
}
{
  The function \ID{IDASpgmrSetJacTimesFn} specifies the Jacobian-vector 
  function to be used.
}
{
  \begin{args}[ida\_mem]
  \item[ida\_mem] (\id{void *})
    pointer to the {\ida} memory block.
  \item[jtimes] (\id{IDASpgmrJacTimesVecFn})
    user-defined Jacobian-vector product function.
  \end{args}
}
{
  The return value \id{flag} (of type \id{int}) is one of
  \begin{args}[IDASPGMR\_LMEM\_NULL]
  \item[\Id{IDASPGMR\_SUCCESS}] 
    The optional value has been successfuly set.
  \item[\Id{IDASPGMR\_MEM\_NULL}]
    The \id{ida\_mem} pointer is \id{NULL}.
  \item[\Id{IDASPGMR\_LMEM\_NULL}]
    The {\idaspgmr} linear solver has not been initialized.
  \end{args}
}
{
  By default, {\idaspgmr} uses the difference quotient function \id{IDASpgmrDQJtimes}.
  If \id{NULL} is passed to \id{jtimes}, this default function is used.

  The function type \id{IDASpgmrJacTimesVecFn} is described in \S\ref{ss:jtimesFn}.
}
%%
%%
\ucfunction{IDASpgmrSetJacData}
{
  flag = IDASpgmrSetJacData(ida\_mem, jac\_data);
}
{
  The function \ID{IDASpgmrSetJacData} specifies the data structure
  to be passed to the user supplied Jacobian-vector
  function each time it is called.
}
{
  \begin{args}[jac\_data]
  \item[ida\_mem] (\id{void *})
    pointer to the {\ida} memory block.
  \item[jac\_data] (\id{void *})
     pointer to the user-defined data structure.
  \end{args}
}
{
  The return value \id{flag} (of type \id{int}) is one of
  \begin{args}[IDASPGMR\_LMEM\_NULL]
  \item[\Id{IDASPGMR\_SUCCESS}] 
    The optional value has been successfuly set.
  \item[\Id{IDASPGMR\_MEM\_NULL}]
    The \id{ida\_mem} pointer is \id{NULL}.
  \item[\Id{IDASPGMR\_LMEM\_NULL}]
    The {\idaspgmr} linear solver has not been initialized.
  \end{args}
}
{}
%%
%%
\ucfunction{IDASpgmrSetGSType}
{
  flag = IDASpgmrSetGSType(ida\_mem, gstype);
}
{
  The function \ID{IDASpgmrSetGSType} specifies the 
  Gram-Schmidt orthogonalization to be used. 
  This must be one of the enumeration constants \ID{MODIFIED\_GS}
  or \ID{CLASSICAL\_GS}. These correspond to using modified Gram-Schmidt 
  and classical Gram-Schmidt, respectively. 
  \index{Gram-Schmidt procedure}
}
{
  \begin{args}[ida\_mem]
  \item[ida\_mem] (\id{void *})
    pointer to the {\ida} memory block.
  \item[gstype] (\id{int})
    type of Gram-Schmidt orthogonalization.
  \end{args}
}
{
  The return value \id{flag} (of type \id{int}) is one of
  \begin{args}[IDASPGMR\_ILL\_INPUT]
  \item[\Id{IDASPGMR\_SUCCESS}] 
    The optional value has been successfuly set.
  \item[\Id{IDASPGMR\_MEM\_NULL}]
    The \id{ida\_mem} pointer is \id{NULL}.
  \item[\Id{IDASPGMR\_LMEM\_NULL}]
    The {\idaspgmr} linear solver has not been initialized.
  \item[\Id{IDASPGMR\_ILL\_INPUT}]
    The Gram-Schmidt orthogonalization type \id{gstype} is not valid.
  \end{args}
}
{
  The default value is \id{MODIFIED\_GS}.
}
%%
%%
\ucfunction{IDASpgmrSetMaxRestarts}
{
  flag = IDASpgmrSetMaxRestarts(ida\_mem, maxrs);
}
{
  The function \ID{IDASpgmrSetMaxRestarts} specifies the maximum number of 
  restarts to be used in the GMRES algorithm.
}
{
  \begin{args}[ida\_mem]
  \item[ida\_mem] (\id{void *})
    pointer to the {\ida} memory block.
  \item[maxrs] (\id{int})
    maximum number of restarts.
  \end{args}
}
{
  The return value \id{flag} (of type \id{int}) is one of
  \begin{args}[IDASPGMR\_ILL\_INPUT]
  \item[\Id{IDASPGMR\_SUCCESS}] 
    The optional value has been successfuly set.
  \item[\Id{IDASPGMR\_MEM\_NULL}]
    The \id{ida\_mem} pointer is \id{NULL}.
  \item[\Id{IDASPGMR\_LMEM\_NULL}]
    The {\idaspgmr} linear solver has not been initialized.
  \item[\Id{IDASPGMR\_ILL\_INPUT}]
    The \id{maxrs} argument is negative.
  \end{args}
}
{
  The default value is $5$. Pass \id{maxrs} $=0$ to specify no restarts.
}
%%
%%
\ucfunction{IDASpgmrSetEpslin}
{
  flag = IDASpgmrSetEpsLin(ida\_mem, eplifac);
}
{
  The function \ID{IDASpgmrSetEps} specifies the factor by
  which the GMRES\index{GMRES method} convergence test constant is reduced
  from the Newton iteration test constant.
}
{
  \begin{args}[ida\_mem]
  \item[ida\_mem] (\id{void *})
    pointer to the {\ida} memory block.
  \item[eplifac] (\id{realtype})

  \end{args}
}
{
  The return value \id{flag} (of type \id{int}) is one of
  \begin{args}[IDASPGMR\_ILL\_INPUT]
  \item[\Id{IDASPGMR\_SUCCESS}] 
    The optional value has been successfuly set.
  \item[\Id{IDASPGMR\_MEM\_NULL}]
    The \id{ida\_mem} pointer is \id{NULL}.
  \item[\Id{IDASPGMR\_LMEM\_NULL}]
    The {\idaspgmr} linear solver has not been initialized.
  \item[\Id{IDASPGMR\_ILL\_INPUT}]
    The factor \id{eplifac} is negative.  
  \end{args}
}
{
  The default value is $0.05$.

  Passing a value \id{eplifac}$ = 0.0$ also indicates using the default value.
}
%%
%%
\ucfunction{IDASpgmrSetIncrementFactor}
{
  flag = IDASpgmrSetIncrementFactor(ida\_mem, dqincfac);
}
{
  The function \ID{IDASpgmrSetIncrementFactor} specifies a factor in the 
  increments to $y$ used in the difference quotient approximations 
  to the Jacobian-vector products.  
}
{
  \begin{args}[dqincfac]
  \item[ida\_mem] (\id{void *})
    pointer to the {\ida} memory block.
  \item[dqincfac] (\id{realtype})
    difference quotient increment factor.
  \end{args}
}
{
  The return value \id{flag} (of type \id{int}) is one of
  \begin{args}[IDASPGMR\_ILL\_INPUT]
  \item[\Id{IDASPGMR\_SUCCESS}] 
    The optional value has been successfuly set.
  \item[\Id{IDASPGMR\_MEM\_NULL}]
    The \id{ida\_mem} pointer is \id{NULL}.
  \item[\Id{IDASPGMR\_LMEM\_NULL}]
    The {\idaspgmr} linear solver has not been initialized.
  \item[\Id{IDASPGMR\_ILL\_INPUT}]
    The increment factor was non-positive.
  \end{args}
}
{
  The default value is \id{dqincfac} $=1.0$.
}
%%
\index{IDASPGMR@{\idaspgmr} linear solver!optional input|)}
\index{optional input!iterative linear solver|)}

%%
%%
%%===================================================================================
%%
\subsubsection{Initial condition calculation optional input functions}
\index{optional input!initial condition calculation|(}

The following functions can be called to set optional inputs to control the initial 
conditions calculations.
%%
\ucfunction{IDASetNonlinConvCoefIC}
{
flag = IDASetNonlinConvCoefIC(ida\_mem, epiccon);
}
{
  The function \ID{IDASetNonlinConvCoefIC} specifies the positive
  coefficient in the Newton initial condition test.
}
{
  \begin{args}[ida\_mem]
  \item[ida\_mem] (\id{void *})
    pointer to the {\ida} memory block.
  \item[epiccon] (\id{realtype})
    coefficient in the Newton convergence test.
  \end{args}
}
{
  The return value \id{flag} (of type \id{int}) is one of
  \begin{args}[IDA\_ILL\_INPUT]
  \item[\Id{IDA\_SUCCESS}] 
    The optional value has been successfuly set.
  \item[\Id{IDA\_MEM\_NULL}]
    The \id{ida\_mem} pointer is \id{NULL}.
  \item[\Id{IDA\_ILL\_INPUT}]
    The \id{epiccon} factor is negative (illegal).
  \end{args}
}
{
  The default value is $0.01 \cdot 0.33$.

  This test uses a weighted RMS norm (with weights defined by the tolerances).
  For new initial value vectors $y$ and ${\dot y}$ to be accepted, the norm
  of $J^{-1}F(t_0, y, {\dot y}) \le $\id{epiccon}, where $J$ is the system Jacobian.
}
%%
\ucfunction{IDASetMaxNumStepsIC}
{
flag = IDASetMaxNumStepsIC(ida\_mem, maxnh);
}
{
  The function \ID{IDASetMaxNumStepsIC} specifies the maximum number
  of values of $h$ allowed when \id{icopt}$=$\id{CALC\_YA\_YDP\_INIT}
  in \id{IDACalcIC}, where $h$ appears in the system Jacobian,
  $J = dF / dy + (1/h) \cdot dF / d{\dot y}$.
}
{
  \begin{args}[ida\_mem]
  \item[ida\_mem] (\id{void *})
    pointer to the {\ida} memory block.
  \item[maxnh] (\id{int})
    maximum allowed number of values for $h$.
  \end{args}
}
{
  The return value \id{flag} (of type \id{int}) is one of
  \begin{args}[IDA\_ILL\_INPUT]
  \item[\Id{IDA\_SUCCESS}] 
    The optional value has been successfuly set.
  \item[\Id{IDA\_MEM\_NULL}]
    The \id{ida\_mem} pointer is \id{NULL}.
  \item[\Id{IDA\_ILL\_INPUT}]
    \id{maxnh} is non-positive.
  \end{args}
}
{
  The default value is $5$.
}
%%
%%
\ucfunction{IDASetMaxNumJacsIC}
{
flag = IDASetMaxNumJacsIC(ida\_mem, maxnj);
}
{
  The function \ID{IDASetMaxNumJacsIC} specifies the maximum number
  of the approximate Jacobian or preconditioner allowed, when the
  Newton iteration appear to be slowly converging.
}
{
  \begin{args}[ida\_mem]
  \item[ida\_mem] (\id{void *})
    pointer to the {\ida} memory block.
  \item[maxnj] (\id{int})
    maximum allowed number of Jacobian or preconditioner evaluations.
  \end{args}
}
{
  The return value \id{flag} (of type \id{int}) is one of
  \begin{args}[IDA\_ILL\_INPUT]
  \item[\Id{IDA\_SUCCESS}] 
    The optional value has been successfuly set.
  \item[\Id{IDA\_MEM\_NULL}]
    The \id{ida\_mem} pointer is \id{NULL}.
  \item[\Id{IDA\_ILL\_INPUT}]
    \id{maxnj} is non-positive.
  \end{args}
}
{
  The default value is $4$.
}
%%
%%
\ucfunction{IDASetMaxNumItersIC}
{
flag = IDASetMaxNumItersIC(ida\_mem, maxnit);
}
{
  The function \ID{IDASetNumItersIC} specifies the maximum
  number of Newton iterations allowed in any one attempt to solve
  the initial conditions calculation problem.
}
{
  \begin{args}[ida\_mem]
  \item[ida\_mem] (\id{void *})
    pointer to the {\ida} memory block.
  \item[maxnit] (\id{int})
    maximum number of Newton iterations.
  \end{args}
}
{
  The return value \id{flag} (of type \id{int}) is one of
  \begin{args}[IDA\_ILL\_INPUT]
  \item[\Id{IDA\_SUCCESS}] 
    The optional value has been successfuly set.
  \item[\Id{IDA\_MEM\_NULL}]
    The \id{ida\_mem} pointer is \id{NULL}.
  \item[\Id{IDA\_ILL\_INPUT}]
    \id{maxnit} is non-positive.
  \end{args}
}
{
  The default value is $10$.
}
%%
%%
\ucfunction{IDASetLineSearchOffIC}
{
flag = IDASetLineSearchOffIC(ida\_mem, lsoff);
}
{
  The function \ID{IDASetLineSearchOffIC} specifies whether to turn
  on or off the linesearch algorithm.
}
{
  \begin{args}[ida\_mem]
  \item[ida\_mem] (\id{void *})
    pointer to the {\ida} memory block.
  \item[lsoff] (\id{booleantype})
    a flag to turn off (\id{TRUE}) or keep (\id{FALSE}) the linesearch
    algorithm.
  \end{args}
}
{
  The return value \id{flag} (of type \id{int}) is one of
  \begin{args}[IDA\_MEM\_NULL]
  \item[\Id{IDA\_SUCCESS}] 
    The optional value has been successfuly set.
  \item[\Id{IDA\_MEM\_NULL}]
    The \id{ida\_mem} pointer is \id{NULL}.
  \end{args}
}
{
  The default value is \id{FALSE}.
}
%%
%%
\ucfunction{IDASetStepToleranceIC}
{
flag = IDASetStepToleranceIC(ida\_mem, steptol);
}
{
  The function \ID{IDASetStepToleranceIC} specifies a positive lower bound
  on the Newton step.
}
{
  \begin{args}[ida\_mem]
  \item[ida\_mem] (\id{void *})
    pointer to the {\ida} memory block.
  \item[steptol] (\id{int})
    Newton step tolerance.
  \end{args}
}
{
  The return value \id{flag} (of type \id{int}) is one of
  \begin{args}[IDA\_ILL\_INPUT]
  \item[\Id{IDA\_SUCCESS}] 
    The optional value has been successfuly set.
  \item[\Id{IDA\_MEM\_NULL}]
    The \id{ida\_mem} pointer is \id{NULL}.
  \item[\Id{IDA\_ILL\_INPUT}]
    The \id{steptol} tolerance is negative (illegal).
  \end{args}
}
{
  The default value is $\sqrt{(\text{unit roundoff})^3}$.
}
%%
\index{optional input!initial condition calculation|)}

%%
%%===================================================================================
%%

\subsection{Interpolated output function}\label{ss:optional_dky}
\index{optional output!interpolated solution}

An optional function \ID{IDAGetSolution} is available to obtain additional output values.  
This function must be called after a successful return from \id{IDASolve} and provides 
interpolated values of $y$ and $y'$ for any value of $t$ in the last internal step 
taken by {\ida}.

The call to the \id{IDAGetSolution} function has the following form:
%%
\ucfunction{IDAGetSolution}
{
  flag = IDAGetSolution(ida\_mem, t, yret, ypret);
}
{
  The function \ID{IDAGetSolution} computes the interpolated values of 
  $y$ and $y'$ for any value of $t$ in the last internal step taken by {\ida}.
}
{
  \begin{args}[ida\_mem]
  \item[ida\_mem] (\id{void *})
    pointer to the {\ida} memory block.
  \item[t] (\id{realtype})
  \item[yret] (\id{N\_Vector})
    vector containing the interpolated $y(t)$.
  \item[ypret] (\id{N\_Vector})
    vector containing the interpolated $y'(t)$.
  \end{args}
}
{
  The return value \id{flag} (of type \id{int}) is one of
  \begin{args}[IDA\_MEM\_NULL] 
  \item[\Id{IDA\_SUCCESS}]
    \id{IDAGetDky} succeeded.
  \item[\Id{IDA\_MEM\_NULL}] 
    The \id{ida\_mem} argument was \id{NULL}.
  \item[\Id{IDA\_BAD\_T}] 
    \id{t} is not in the interval $[t_n - h_u , t_n]$.
  \end{args}

}
{
  It is only legal to call the function \id{IDAGetSolution} after a 
  successful return from \id{IDASolve}. 
}

%%
%%===================================================================================
%%

\subsection{Optional output functions}\label{ss:optional_output}

{\ida} provides an extensive list of functions that can be used to obtain
solver performance information.
Table \ref{t:optional_output} lists all optional output functions in {\ida},
which are then described in detail in the remainder of this section.

\begin{table}
\centering
\caption{Optional outputs from {\ida}, {\idadense}, {\idaband}, and {\idaspgmr}}
\label{t:optional_output}
\medskip
\begin{tabular}{|l|l|}\hline
{\bf Optional output} & {\bf Function name} \\
\hline
\multicolumn{2}{|c|}{\bf IDA main solver} \\
\hline
Size of {\ida} real and integer workspace & \id{IDAGetWorkSpace} \\
Cumulative number of internal steps & \id{IDAGetNumSteps} \\
No. of calls to residual function & \id{IDAGetNumResEvals} \\
No. of calls to linear solver setup function & \id{IDAGetNumLinSolvSetups} \\
No. of local error test failures that have occurred & \id{IDAGetNumErrTestFails} \\
Order used during the last step & \id{IDAGetLastOrder} \\
Order to be attempted on the next step & \id{IDAGetCurrentOrder} \\
Order reductions due to stability limit detection & \id{IDAGetNumStabLimOrderReds} \\
Actual initial step size used & \id{IDAGetActualInitStep} \\
Step size used for the last step & \id{IDAGetLastStep} \\
Step size to be attempted on the next step & \id{IDAGetCurrentStep} \\
Current internal time reached by the solver & \id{IDAGetCurrentTime} \\
Suggested factor for tolerance scaling  & \id{IDAGetTolScaleFactor} \\
Error weight vector for state variables & \id{IDAGetErrWeights} \\
Estimated local error vector & \id{IDAGetEstLocalErrors} \\
No. of nonlinear solver iterations & \id{IDAGetNumNonlinSolvIters} \\
No. of nonlinear convergence failures & \id{IDAGetNumNonlinSolvConvFails} \\ 
\hline
\multicolumn{2}{|c|}{\bf IDADENSE linear solver} \\
\hline
Size of {\idadense} real and integer workspace & \id{IDADenseGetWorkSpace} \\
No. of Jacobian evaluations & \id{IDADenseGetNumJacEvals} \\
No. of residual calls for finite diff. Jacobian evals. & \id{IDADenseGetNumResEvals} \\ 
Last return from a {\idadense} function & \id{IDADenseGetLastFlag} \\ 
\hline
\multicolumn{2}{|c|}{\bf IDABAND linear solver} \\
\hline
Size of {\idaband} real and integer workspace & \id{IDABandGetWorkSpace} \\
No. of Jacobian evaluations & \id{IDABandGetNumJacEvals} \\
No. of residual calls for finite diff. Jacobian evals. & \id{IDABandGetNumResEvals} \\ 
Last return from a {\idaband} function & \id{IDABandGetLastFlag} \\ 
\hline
\multicolumn{2}{|c|}{\bf IDASPGMR linear solver} \\
\hline
Size of {\idaspgmr} real and integer workspace & \id{IDASpgmrGetWorkSpace} \\
No. of linear iterations & \id{IDASpgmrGetNumLinIters} \\
No. of linear convergence failures & \id{IDASpgmrGetNumConvFails} \\
No. of preconditioner evaluations & \id{IDASpgmrGetNumPrecEvals} \\
No. of preconditioner solves & \id{IDASpgmrGetNumPrecSolves} \\
No. of Jacobian-vector product evaluations & \id{IDASpgmrGetNumJtimesEvals} \\
No. of residual calls for finite diff. Jacobian-vector evals. & \id{IDASpgmrGetNumResEvals} \\ 
Last return from a {\idaspgmr} function & \id{IDASpgmrGetLastFlag} \\ 
\hline
\end{tabular}
\end{table}


\subsubsection{Main solver optional output functions}
\index{optional output!solver|(}
%%
{\ida} provides several user-callable functions that can be used to obtain
different quantities that may be of interest to the user, such as solver workspace
requirements, solver performance statistics, as well as additional data from
the {\ida} memory block (a suggested tolerance scaling factor, the error weight
vector, and the vector of estimated local errors). Also provided are functions to
extract statistics related to the performance of the {\ida} nonlinear solver
being used. As a convenience, additional extraction functions provide the optional 
outputs in groups.
%%
These optional output functions are described next.
%%
\index{memory requirements!IDA@{\ida} solver|(}
%%
\ucfunction{IDAGetWorkSpace}
{
  flag = IDAGetWorkSpace(ida\_mem, \&lenrw, \&leniw);
}
{
  The function \ID{IDAGetWorkSpace} returns the
  {\ida} real and integer workspace sizes.
}
{
  \begin{args}[ida\_mem]
  \item[ida\_mem] (\id{void *})
    pointer to the {\ida} memory block.
  \item[lenrw] (\id{long int})
    number of real values in the {\ida} workspace.
  \item[leniw] (\id{long int})
    number of integer values in the {\ida} workspace.
  \end{args}
}
{
  The return value \id{flag} (of type \id{int}) is one of
  \begin{args}[IDA\_MEM\_NULL]
  \item[IDA\_SUCCESS] 
    The optional output value has been successfuly set.
  \item[\Id{IDA\_MEM\_NULL}]
    The \id{ida\_mem} pointer is \id{NULL}.
  \end{args}
}
{}
%%
\index{memory requirements!IDA@{\ida} solver|)}
%%
\ucfunction{IDAGetNumSteps}
{
  flag = IDAGetNumSteps(ida\_mem, \&nsteps);
}
{
  The function \ID{IDAGetNumSteps} returns the cumulative number of internal 
  steps taken by the solver (total so far).
}
{
  \begin{args}[ida\_mem]
  \item[ida\_mem] (\id{void *})
    pointer to the {\ida} memory block.
  \item[nsteps] (\id{long int})
    number of steps taken by {\ida}.
  \end{args}
}
{
  The return value \id{flag} (of type \id{int}) is one of
  \begin{args}[IDA\_MEM\_NULL]
  \item[IDA\_SUCCESS] 
    The optional output value has been successfuly set.
  \item[\Id{IDA\_MEM\_NULL}]
    The \id{ida\_mem} pointer is \id{NULL}.
  \end{args}
}
{}
%%
%%
\ucfunction{IDAGetNumResEvals}
{
  flag = IDAGetNumResEvals(ida\_mem, \&nrevals);
}
{
  The function \ID{IDAGetNumResEvals} returns the 
  number of calls to the user's residual evaluation function.
}
{
  \begin{args}[ida\_mem]
  \item[ida\_mem] (\id{void *})
    pointer to the {\ida} memory block.
  \item[nrevals] (\id{long int})
    number of calls to the user's \id{res} function.
  \end{args}
}
{
  The return value \id{flag} (of type \id{int}) is one of
  \begin{args}[IDA\_MEM\_NULL]
  \item[IDA\_SUCCESS] 
    The optional output value has been successfuly set.
  \item[\Id{IDA\_MEM\_NULL}]
    The \id{ida\_mem} pointer is \id{NULL}.
  \end{args}
}
{
  The \id{nrevals} value returned by \id{IDAGetNumResEvals} does not
  account for calls made to \id{res} from a linear solver or preconditioner 
  module. 
}
%%
%%
\ucfunction{IDAGetNumLinSolvSetups}
{
  flag = IDAGetNumLinSolvSetups(ida\_mem, \&nlinsetups);
}
{
  The function \ID{IDAGetNumLinSolvSetups} returns the
  number of calls made to the linear solver's setup function.
}
{
  \begin{args}[nlinsetups]
  \item[ida\_mem] (\id{void *})
    pointer to the {\ida} memory block.
  \item[nlinsetups] (\id{long int})
    number of calls made to the linear solver setup function.
  \end{args}
}
{
  The return value \id{flag} (of type \id{int}) is one of
  \begin{args}[IDA\_MEM\_NULL]
  \item[IDA\_SUCCESS] 
    The optional output value has been successfuly set.
  \item[\Id{IDA\_MEM\_NULL}]
    The \id{ida\_mem} pointer is \id{NULL}.
  \end{args}
}
{}
%%
%%
\ucfunction{IDAGetNumErrTestFails}
{
  flag = IDAGetNumErrTestFails(ida\_mem, \&netfails);
}
{
  The function \ID{IDAGetNumErrTestFails} returns the
  number of local error test failures that have occurred.
}
{
  \begin{args}[netfails]
  \item[ida\_mem] (\id{void *})
    pointer to the {\ida} memory block.
  \item[netfails] (\id{long int})
    number of error test failures.
  \end{args}
}
{
  The return value \id{flag} (of type \id{int}) is one of
  \begin{args}[IDA\_MEM\_NULL]
  \item[IDA\_SUCCESS] 
    The optional output value has been successfuly set.
  \item[\Id{IDA\_MEM\_NULL}]
    The \id{ida\_mem} pointer is \id{NULL}.
  \end{args}
}
{}
%%
%%
\ucfunction{IDAGetLastOrder}
{
  flag = IDAGetLastOrder(ida\_mem, \&qlast);
}
{
  The function \ID{IDAGetLastOrder} returns the
  integration method order used during the last internal step.
}
{
  \begin{args}[ida\_mem]
  \item[ida\_mem] (\id{void *})
    pointer to the {\ida} memory block.
  \item[qlast] (\id{int})
    method order used on the last internal step.
  \end{args}
}
{
  The return value \id{flag} (of type \id{int}) is one of
  \begin{args}[IDA\_MEM\_NULL]
  \item[IDA\_SUCCESS] 
    The optional output value has been successfuly set.
  \item[\Id{IDA\_MEM\_NULL}]
    The \id{ida\_mem} pointer is \id{NULL}.
  \end{args}
}
{}
%%
%%
\ucfunction{IDAGetCurrentOrder}
{
  flag = IDAGetCurrentOrder(ida\_mem, \&qcur);
}
{
  The function \ID{IDAGetCurrentOrder} returns the
  integration method order to be used on the next internal step.
}
{
  \begin{args}[ida\_mem]
  \item[ida\_mem] (\id{void *})
    pointer to the {\ida} memory block.
  \item[qcur] (\id{int})
    method order to be used on the next internal step.
  \end{args}
}
{
  The return value \id{flag} (of type \id{int}) is one of
  \begin{args}[IDA\_MEM\_NULL]
  \item[IDA\_SUCCESS] 
    The optional output value has been successfuly set.
  \item[\Id{IDA\_MEM\_NULL}]
    The \id{ida\_mem} pointer is \id{NULL}.
  \end{args}
}
{}
%%
%%
\ucfunction{IDAGetLastStep}
{
  flag = IDAGetLastStep(ida\_mem, \&hlast);
}
{
  The function \ID{IDAGetLastStep} returns the
  integration step size taken on the last internal step.
}
{
  \begin{args}[ida\_mem]
  \item[ida\_mem] (\id{void *})
    pointer to the {\ida} memory block.
  \item[hlast] (\id{realtype})
    step size taken on the last internal step.
  \end{args}
}
{
  The return value \id{flag} (of type \id{int}) is one of
  \begin{args}[IDA\_MEM\_NULL]
  \item[IDA\_SUCCESS] 
    The optional output value has been successfuly set.
  \item[\Id{IDA\_MEM\_NULL}]
    The \id{ida\_mem} pointer is \id{NULL}.
  \end{args}
}
{}
%%
%%
\ucfunction{IDAGetCurrentStep}
{
  flag = IDAGetCurrentStep(ida\_mem, \&hcur);
}
{
  The function \ID{IDAGetCurrentStep} returns the
  integration step size to be attempted on the next internal step.
}
{
  \begin{args}[ida\_mem]
  \item[ida\_mem] (\id{void *})
    pointer to the {\ida} memory block.
  \item[hcur] (\id{realtype})
    step size to be attempted on the next internal step.
  \end{args}
}
{
  The return value \id{flag} (of type \id{int}) is one of
  \begin{args}[IDA\_MEM\_NULL]
  \item[IDA\_SUCCESS] 
    The optional output value has been successfuly set.
  \item[\Id{IDA\_MEM\_NULL}]
    The \id{ida\_mem} pointer is \id{NULL}.
  \end{args}
}
{}
%%
%%
\ucfunction{IDAGetActualInitStep}
{
  flag = IDAGetActualInitStep(ida\_mem, \&hinused);
}
{
  The function \ID{IDAGetActualInitStep} returns the
  value of the integration step size used on the first step.
}
{
  \begin{args}[ida\_mem]
  \item[ida\_mem] (\id{void *})
    pointer to the {\ida} memory block.
  \item[hinused] (\id{realtype})
    actual value of initial step size.
  \end{args}
}
{
  The return value \id{flag} (of type \id{int}) is one of
  \begin{args}[IDA\_MEM\_NULL]
  \item[IDA\_SUCCESS] 
    The optional output value has been successfuly set.
  \item[\Id{IDA\_MEM\_NULL}]
    The \id{ida\_mem} pointer is \id{NULL}.
  \end{args}
}
{
  Even if the value of the initial integration step size was specified
  by the user through a call to \id{IDASetInitStep}, this value might have 
  been changed by {\ida} to ensure that the step size is within the 
  prescribed bounds ($h_{\min} \le h_0 \le h_{\max}$), or to meet the
  local error test.
}
%%
%%
\ucfunction{IDAGetCurrentTime}
{
  flag = IDAGetCurrentTime(ida\_mem, \&tcur);
}
{
  The function \ID{IDAGetCurrentTime} returns the
  current internal time reached by the solver.
}
{
  \begin{args}[ida\_mem]
  \item[ida\_mem] (\id{void *})
    pointer to the {\ida} memory block.
  \item[tcur] (\id{realtype})
    current internal time reached.
  \end{args}
}
{
  The return value \id{flag} (of type \id{int}) is one of
  \begin{args}[IDA\_MEM\_NULL]
  \item[IDA\_SUCCESS] 
    The optional output value has been successfuly set.
  \item[\Id{IDA\_MEM\_NULL}]
    The \id{ida\_mem} pointer is \id{NULL}.
  \end{args}
}
{}
%%
%%
\ucfunction{IDAGetTolScaleFactor}
{
  flag = IDAGetTolScaleFactor(ida\_mem, \&tolsfac);
}
{
  The function \ID{IDAGetTolScaleFactor} returns a
  suggested factor by which the user's tolerances 
  should be scaled when too much accuracy has been 
  requested for some internal step.
}
{
  \begin{args}[tolsfac]
  \item[ida\_mem] (\id{void *})
    pointer to the {\ida} memory block.
  \item[tolsfac] (\id{realtype})
    suggested scaling factor for user tolerances.
  \end{args}
}
{
  The return value \id{flag} (of type \id{int}) is one of
  \begin{args}[IDA\_MEM\_NULL]
  \item[IDA\_SUCCESS] 
    The optional output value has been successfuly set.
  \item[\Id{IDA\_MEM\_NULL}]
    The \id{ida\_mem} pointer is \id{NULL}.
  \end{args}
}
{}
%%
%%
\ucfunction{IDAGetErrWeights}
{
  flag = IDAGetErrWeights(ida\_mem, \&eweight);
}
{
  The function \ID{IDAGetErrWeights} returns the solution error weights 
  at the current time. These are the reciprocals of the $W_i$ of (\ref{e:errwt}).
}
{
  \begin{args}[eweight]
  \item[ida\_mem] (\id{void *})
    pointer to the {\ida} memory block.
  \item[eweight] (\id{N\_Vector})
    solution error weights at the current time.
  \end{args}
}
{
  The return value \id{flag} (of type \id{int}) is one of
  \begin{args}[IDA\_MEM\_NULL]
  \item[IDA\_SUCCESS] 
    The optional output value has been successfuly set.
  \item[\Id{IDA\_MEM\_NULL}]
    The \id{ida\_mem} pointer is \id{NULL}.
  \end{args}
}
{
  The user need not allocate space for \id{eweight} and should not modify
  any of its components.
}
%%
%%
\ucfunction{IDAGetEstLocalErrors}
{
  flag = IDAGetEstLocalErrors(ida\_mem, \&ele);
}
{
  The function \ID{IDAGetEstLocalErrors} returns the
  vector of estimated local errors.
}
{
  \begin{args}[ida\_mem]
  \item[ida\_mem] (\id{void *})
    pointer to the {\ida} memory block.
  \item[ele] (\id{N\_Vector})
    estimated local errors.
  \end{args}
}
{
  The return value \id{flag} (of type \id{int}) is one of
  \begin{args}[IDA\_MEM\_NULL]
  \item[IDA\_SUCCESS] 
    The optional output value has been successfuly set.
  \item[\Id{IDA\_MEM\_NULL}]
    The \id{ida\_mem} pointer is \id{NULL}.
  \end{args}
}
{
  The user need not allocate space for \id{ele}.
}
%%
%%
\ucfunction{IDAGetIntegratorStats}
{
  \begin{tabular}[t]{@{}r@{}l@{}}
    flag = IDAGetIntegratorStats(&ida\_mem, \&nsteps, \&nrevals, \&nlinsetups, \\
                                 &\&netfails, \&qlast, \&qcur, \&hinused,\\
                                 &\&hlast, \&hcur, \&tcur);
  \end{tabular}
}
{
  The function \ID{IDAGetIntegratorStats} returns the {\ida} integrator statistics
  as a group.
}
{
  \begin{args}[nlinsetups]
  \item[ida\_mem] (\id{void *})
    pointer to the {\ida} memory block.
  \item[nsteps] (\id{long int})
    number of steps taken by {\ida}.
  \item[nrevals] (\id{long int})
    number of calls to the user's \id{res} function.
  \item[nlinsetups] (\id{long int})
    number of calls made to the linear solver setup function.
  \item[netfails] (\id{long int})
    number of error test failures.
  \item[qlast] (\id{int})
    method order used on the last internal step.
  \item[qcur] (\id{int})
    method order to be used on the next internal step.
  \item[hinused] (\id{realtype})
    actual value of initial step size.
  \item[hlast] (\id{realtype})
    step size taken on the last internal step.
  \item[hcur] (\id{realtype})
    step size to be attempted on the next internal step.
  \item[tcur] (\id{realtype})
    current internal time reached.
  \end{args}
}
{
  The return value \id{flag} (of type \id{int}) is one of
  \begin{args}[IDA\_MEM\_NULL]
  \item[IDA\_SUCCESS] 
    the optional output values have been successfuly set.
  \item[\Id{IDA\_MEM\_NULL}]
    the \id{ida\_mem} pointer is \id{NULL}.
  \end{args}
}
{}
%%
%%
\ucfunction{IDAGetNumNonlinSolvIters}
{
  flag = IDAGetNumNonlinSolvIters(ida\_mem, \&nniters);
}
{
  The function \ID{IDAGetNumNonlinSolvIters} returns the
  number of nonlinear (functional or Newton) iterations performed. 
}
{
  \begin{args}[nniters]
  \item[ida\_mem] (\id{void *})
    pointer to the {\ida} memory block.
  \item[nniters] (\id{long int})
    number of nonlinear iterations performed.
  \end{args}
}
{
  The return value \id{flag} (of type \id{int}) is one of
  \begin{args}[IDA\_MEM\_NULL]
  \item[IDA\_SUCCESS] 
    The optional output value has been successfuly set.
  \item[\Id{IDA\_MEM\_NULL}]
    The \id{ida\_mem} pointer is \id{NULL}.
  \end{args}
}
{}
%%
%%
\ucfunction{IDAGetNumNonlinSolvConvFails}
{
  flag = IDAGetNumNonlinSolvConvFails(ida\_mem, \&nncfails);
}
{
  The function \ID{IDAGetNumNonlinSolvConvFails} returns the
  number of nonlinear convergence failures that have occurred.
}
{
  \begin{args}[nncfails]
  \item[ida\_mem] (\id{void *})
    pointer to the {\ida} memory block.
  \item[nncfails] (\id{long int})
    number of nonlinear convergence failures.
  \end{args}
}
{
  The return value \id{flag} (of type \id{int}) is one of
  \begin{args}[IDA\_MEM\_NULL]
  \item[IDA\_SUCCESS] 
    The optional output value has been successfuly set.
  \item[\Id{IDA\_MEM\_NULL}]
    The \id{ida\_mem} pointer is \id{NULL}.
  \end{args}
}
{}
%%
%%
\ucfunction{IDAGetNonlinSolvStats}
{
  flag = IDAGetNonlinSolvStats(ida\_mem, \&nniters, \&nncfails);
}
{
  The function \ID{IDAGetNonlinSolvStats} returns the the
  {\ida} nonlinear solver statistics as a group.
}
{
  \begin{args}[nncfails]
  \item[ida\_mem] (\id{void *})
    pointer to the {\ida} memory block.
  \item[nniters] (\id{long int})
    number of nonlinear iterations performed.
  \item[nncfails] (\id{long int})
    number of nonlinear convergence failures.
  \end{args}
}
{
  The return value \id{flag} (of type \id{int}) is one of
  \begin{args}[IDA\_MEM\_NULL]
  \item[IDA\_SUCCESS] 
    The optional output value has been successfuly set.
  \item[\Id{IDA\_MEM\_NULL}]
    The \id{ida\_mem} pointer is \id{NULL}.
  \end{args}
}
{}
%%
\index{optional output!solver|)}
%%
%%===================================================================================

\subsubsection{Linear solver optional output functions}

For each of the linear system solver modules, there are various optional 
outputs that describe the performance of the module. The functions available 
to access these are described below.

\noindent\paragraph{\bf Dense Linear solver.}
\index{optional output!dense linear solver|(}
\index{IDADENSE@{\idadense} linear solver!optional output|(}
The following optional outputs are available from the {\idadense} module:
workspace requirements, number of calls to the Jacobian routine, number of 
calls to the residual routine for finite-difference Jacobian approximation,
and last return value from a {\idadense} function.
%%
\index{IDADENSE@{\idadense} linear solver!memory requirements} 
\index{memory requirements!IDADENSE@{\idadense} linear solver}
%%
\ucfunction{IDADenseGetWorkSpace}
{
  flag = IDADenseGetWorkSpace(ida\_mem, \&lenrwD, \&leniwD);
}
{
  The function \ID{IDADenseGetWorkSpace} returns the sizes of
  the {\idadense} real and integer workspaces.
}
{
  \begin{args}[ida\_mem]
  \item[ida\_mem] (\id{void *})
    pointer to the {\ida} memory block.
  \item[lenrwD] (\id{long int})
    the number of real values in the {\idadense} workspace.
  \item[leniwD] (\id{long int})
    the number of integer values in the {\idadense} workspace.
  \end{args}
}
{
  The return value \id{flag} (of type \id{int}) is one of
  \begin{args}[IDA\_MEM\_NULL]
  \item[IDA\_SUCCESS] 
    The optional output value has been successfuly set.
  \item[\Id{IDA\_MEM\_NULL}]
    The \id{ida\_mem} pointer is \id{NULL}.
  \item[\Id{IDA\_MEM\_NULL}]
    The {\idadense} linear solver has not been initialized.
  \end{args}
}
{
  In terms of the problem size $N$, the actual size of the real workspace
  is $2N^2$ \id{realtype} words.

  In terms of the problem size $N$, the actual size of the integer workspace
  is $N$ integer words.
}
%%
%%
\ucfunction{IDADenseGetNumJacEvals}
{
  flag = IDADenseGetNumJacEvals(ida\_mem, \&njevalsD);
}
{
  The function \ID{IDADenseGetNumJacEvals} returns the
  number of calls to the dense Jacobian approximation function.
}
{
  \begin{args}[njevalsD]
  \item[ida\_mem] (\id{void *})
    pointer to the {\ida} memory block.
  \item[njevalsD] (\id{long int})
    the number of calls to the Jacobian function.
  \end{args}
}
{
  The return value \id{flag} (of type \id{int}) is one of
  \begin{args}[IDADENSE\_LMEM\_NULL]
  \item[IDADENSE\_SUCCESS] 
    The optional output value has been successfuly set.
  \item[\Id{IDADENSE\_MEM\_NULL}]
    The \id{ida\_mem} pointer is \id{NULL}.
  \item[\Id{IDADENSE\_LMEM\_NULL}]
    The {\idadense} linear solver has not been initialized.
  \end{args}
}
{}
%%
%%
\ucfunction{IDADenseGetNumResEvals}
{
  flag = IDADenseGetNumResEvals(ida\_mem, \&nrevalsD);
}
{
  The function \ID{IDADenseGetNumResEvals} returns the
  number of calls to the user residual function due to the 
  finite difference dense Jacobian approximation.
}
{
  \begin{args}[nrevalsD]
  \item[ida\_mem] (\id{void *})
    pointer to the {\ida} memory block.
  \item[nrevalsD] (\id{long int})
    the number of calls to the user residual function.
  \end{args}
}
{
  The return value \id{flag} (of type \id{int}) is one of
  \begin{args}[IDADENSE\_LMEM\_NULL]
  \item[IDADENSE\_SUCCESS] 
    The optional output value has been successfuly set.
  \item[\Id{IDADENSE\_MEM\_NULL}]
    The \id{ida\_mem} pointer is \id{NULL}.
  \item[\Id{IDADENSE\_LMEM\_NULL}]
    The {\idadense} linear solver has not been initialized.
  \end{args}
}
{
  The value \id{nrevalsD} is incremented only if the default 
  \id{IDADenseDQJac} difference quotient function is used.
}
%%
\ucfunction{IDADenseGetLastFlag}
{
  flag = IDADenseGetLastFlag(ida\_mem, \&flag);
}
{
  The function \ID{IDADenseGetLastFlag} returns the
  last return value from a {\idadense} routine. 
}
{
  \begin{args}[ida\_mem]
  \item[ida\_mem] (\id{void *})
    pointer to the {\ida} memory block.
  \item[flag] (\id{int})
    the value of the last return flag from a {\idadense} function.
  \end{args}
}
{
  The return value \id{flag} (of type \id{int}) is one of
  \begin{args}[IDADENSE\_LMEM\_NULL]
  \item[\Id{IDADENSE\_SUCCESS}] 
    The optional output value has been successfully set.
  \item[\Id{IDADENSE\_MEM\_NULL}]
    The \id{ida\_mem} pointer is \id{NULL}.
  \item[\Id{IDADENSE\_LMEM\_NULL}]
    The {\idadense} linear solver has not been initialized.
  \end{args}
}
{
  If the {\idadense} setup function failed (\id{IDASolve} returned \id{IDA\_LSETUP\_FAIL}),
  the value \id{flag} is equal to the column index (numbered from one) at which
  a zero diagonal element was encountered during the LU factorization of the 
  dense Jacobian matrix.
}
%%
\index{IDADENSE@{\idadense} linear solver!optional output|)}
\index{optional output!dense linear solver|)}
%
%--------------------------------
%
\noindent\paragraph{\bf Band Linear solver.}
\index{optional output!band linear solver|(}
\index{IDABAND@{\idaband} linear solver!optional output|(}
The following optional outputs are available from the {\idaband} module:
workspace requirements, number of calls to the Jacobian routine, number of 
calls to the residual routine for finite-difference Jacobian approximation,
and last return value from a {\idaband} function.
%%
\index{IDABAND@{\idaband} linear solver!memory requirements} 
\index{memory requirements!IDABAND@{\idaband} linear solver}
%%
\ucfunction{IDABandGetWorkSpace}
{
  flag = IDABandGetWorkSpace(ida\_mem, \&lenrwB, \&leniwB);
}
{
  The function \ID{IDABandGetWorkSpace} returns the sizes of
  the {\idaband} real and integer workspaces.
}
{
  \begin{args}[ida\_mem]
  \item[ida\_mem] (\id{void *})
    pointer to the {\ida} memory block.
  \item[lenrwB] (\id{long int})
    the number of real values in the {\idaband} workspace.
  \item[leniwB] (\id{long int})
    the number of integer values in the {\idaband} workspace.
  \end{args}
}
{
  The return value \id{flag} (of type \id{int}) is one of
  \begin{args}[IDABAND\_LMEM\_NULL]
  \item[IDABAND\_SUCCESS] 
    The optional output value has been successfuly set.
  \item[\Id{IDABAND\_MEM\_NULL}]
    The \id{ida\_mem} pointer is \id{NULL}.
  \item[\Id{IDABAND\_LMEM\_NULL}]
    The {\idaband} linear solver has not been initialized.
  \end{args}
}
{
  In terms of the problem size $N$ and Jacobian half-bandwidths, 
  the actual size of the real workspace
  $N\,(2$ \id{mupper}$+ 3$ \id{mlower} $+ 2)$ \id{realtype} words.

  In terms of the problem size $N$, the actual size of the integer workspace
  is $N$ integer words.
}
%%
%%
\ucfunction{IDABandGetNumJacEvals}
{
  flag = IDABandGetNumJacEvals(ida\_mem, \&njevalsB);
}
{
  The function \ID{IDABandGetNumJacEvals} returns the
  number of calls to the banded Jacobian approximation function.
}
{
  \begin{args}[njevalsB]
  \item[ida\_mem] (\id{void *})
    pointer to the {\ida} memory block.
  \item[njevalsB] (\id{long int})
    the number of calls to the Jacobian function.
  \end{args}
}
{
  The return value \id{flag} (of type \id{int}) is one of
  \begin{args}[IDABAND\_LMEM\_NULL]
  \item[IDABAND\_SUCCESS] 
    The optional output value has been successfuly set.
  \item[\Id{IDABAND\_MEM\_NULL}]
    The \id{ida\_mem} pointer is \id{NULL}.
  \item[\Id{IDABAND\_LMEM\_NULL}]
    The {\idaband} linear solver has not been initialized.
  \end{args}
}
{}
%%
%%
\ucfunction{IDABandGetNumResEvals}
{
  flag = IDABandGetNumResEvals(ida\_mem, \&nrevalsB);
}
{
  The function \ID{IDABandGetNumResEvals} returns the
  number of calls to the user residual function due to the 
  finite difference banded Jacobian approximation.
}
{
  \begin{args}[nrevalsB]
  \item[ida\_mem] (\id{void *})
    pointer to the {\ida} memory block.
  \item[nrevalsB] (\id{long int})
    the number of calls to the user residual function.
  \end{args}
}
{
  The return value \id{flag} (of type \id{int}) is one of
  \begin{args}[IDABAND\_LMEM\_NULL]
  \item[IDA\_SUCCESS] 
    The optional output value has been successfuly set.
  \item[\Id{IDABAND\_MEM\_NULL}]
    The \id{ida\_mem} pointer is \id{NULL}.
  \item[\Id{IDABAND\_LMEM\_NULL}]
    The {\idaband} linear solver has not been initialized.
  \end{args}
}
{
  The value \id{nrevalsB} is incremented only if the default 
  \id{IDABandDQJac} difference quotient function is used.
}
%%
\ucfunction{IDABandGetLastFlag}
{
  flag = IDABandGetLastFlag(ida\_mem, \&flag);
}
{
  The function \ID{IDABandGetLastFlag} returns the
  last return value from a {\idaband} routine. 
}
{
  \begin{args}[ida\_mem]
  \item[ida\_mem] (\id{void *})
    pointer to the {\ida} memory block.
  \item[flag] (\id{int})
    the value of the last return flag from a {\idaband} function.
  \end{args}
}
{
  The return value \id{flag} (of type \id{int}) is one of
  \begin{args}[IDABAND\_LMEM\_NULL]
  \item[\Id{IDABAND\_SUCCESS}] 
    The optional output value has been successfully set.
  \item[\Id{IDaBAND\_MEM\_NULL}]
    The \id{ida\_mem} pointer is \id{NULL}.
  \item[\Id{IDABAND\_LMEM\_NULL}]
    The {\idaband} linear solver has not been initialized.
  \end{args}
}
{
  If the {\idaband} setup sunction failed (\id{IDASolve} returned \id{IDA\_LSETUP\_FAIL}),
  the value \id{flag} is equal to the column index (numbered from one) at which
  a zero diagonal element was encountered during the LU factorization of the 
  banded Jacobian matrix.
}
%%
\index{IDABAND@{\idaband} linear solver!optional output|)}
\index{optional output!band linear solver|)}
%
%--------------------------------
%
\noindent\paragraph{\bf SPGMR Linear solver.}
\index{optional output!iterative linear solver|(}
\index{IDASPGMR@{\idaspgmr} linear solver!optional output|(} 
The following optional outputs are available from the {\idaspgmr} module:
workspace requirements, number of linear iterations,
number of linear convergence failures, number of calls to the preconditioner
setup and solve routines, number of calls to the Jacobian-vector product routine, number of 
calls to the residual routine for finite-difference Jacobian-vector product approximation,
and last return value from a {\idaspgmr} function.
%%
\index{IDASPGMR@{\idaspgmr} linear solver!memory requirements} 
\index{memory requirements!IDASPGMR@{\idaspgmr} linear solver}
%%
\ucfunction{IDASpgmrGetWorkSpace}
{
  flag = IDASpgmrGetWorkSpace(ida\_mem, \&lenrwSG, \&leniwSG);
}
{
  The function \ID{IDASpgmrGetWorkSpace} returns the sizes of
  the {\idaspgmr} real and integer workspaces.
}
{
  \begin{args}[lenrwSG]
  \item[ida\_mem] (\id{void *})
    pointer to the {\ida} memory block.
  \item[lenrwSG] (\id{long int})
    the number of real values in the {\idaspgmr} workspace.
  \item[leniwSG] (\id{long int})
    the number of integer values in the {\idaspgmr} workspace.
  \end{args}
}
{
  The return value \id{flag} (of type \id{int}) is one of
  \begin{args}[IDASPGMR\_LMEM\_NULL]
  \item[IDASPGMR\_SUCCESS] 
    The optional output value has been successfuly set.
  \item[\Id{IDASPGMR\_MEM\_NULL}]
    The \id{ida\_mem} pointer is \id{NULL}.
  \item[\Id{IDASPGMR\_LMEM\_NULL}]
    The {\idaspgmr} linear solver has not been initialized.
  \end{args}
}
{
  In terms of the problem size $N$ and maximum subspace size \id{maxl}, 
  the actual size of the real workspace is
  $N*($ \id{maxl} $+ 5) +$ \id{maxl} $*($ \id{maxl} $ + 4) + 1$ \id{realtype} words.
}
%%
%%
\ucfunction{IDASpgmrGetNumLinIters}
{
  flag = IDASpgmrGetNumLinIters(ida\_mem, \&nliters);
}
{
  The function \ID{IDASpgmrGetNumLinIters} returns the
  cumulative number of linear iterations.
}
{
  \begin{args}[nliters]
  \item[ida\_mem] (\id{void *})
    pointer to the {\ida} memory block.
  \item[nliters] (\id{long int})
    the current number of linear iterations.
  \end{args}
}
{
  The return value \id{flag} (of type \id{int}) is one of
  \begin{args}[IDASPGMR\_LMEM\_NULL]
  \item[IDASPGMR\_SUCCESS] 
    The optional output value has been successfuly set.
  \item[\Id{IDASPGMR\_MEM\_NULL}]
    The \id{ida\_mem} pointer is \id{NULL}.
  \item[\Id{IDASPGMR\_LMEM\_NULL}]
    The {\idaspgmr} linear solver has not been initialized.
  \end{args}
}
{}
%%
%%
\ucfunction{IDASpgmrGetNumConvFails}
{
  flag = IDASpgmrGetNumConvFails(ida\_mem, \&nlcfails);
}
{
  The function \ID{IDASpgmrGetNumConvFails} returns the
  cumulative number of linear convergence failures.
}
{
  \begin{args}[nlcfails]
  \item[ida\_mem] (\id{void *})
    pointer to the {\ida} memory block.
  \item[nlcfails] (\id{long int})
    the current number of linear convergence failures.
  \end{args}
}
{
  The return value \id{flag} (of type \id{int}) is one of
  \begin{args}[IDASPGMR\_LMEM\_NULL]
  \item[IDASPGMR\_SUCCESS] 
    The optional output value has been successfuly set.
  \item[\Id{IDASPGMR\_MEM\_NULL}]
    The \id{ida\_mem} pointer is \id{NULL}.
  \item[\Id{IDASPGMR\_LMEM\_NULL}]
    The {\idaspgmr} linear solver has not been initialized.
  \end{args}
}
{}
%%
%%
\ucfunction{IDASpgmrGetNumPrecEvals}
{
  flag = IDASpgmrGetNumPrecEvals(ida\_mem, \&npevals);
}
{
  The function \ID{IDASpgmrGetNumPrecEvals} returns the
  number of preconditioner evaluations, i.e., the number of 
  calls made to \id{psetup} with \id{jok=FALSE}.
}
{
  \begin{args}[npevals]
  \item[ida\_mem] (\id{void *})
    pointer to the {\ida} memory block.
  \item[npevals] (\id{long int})
    the current number of calls to \id{psetup}.
  \end{args}
}
{
  The return value \id{flag} (of type \id{int}) is one of
  \begin{args}[IDASPGMR\_LMEM\_NULL]
  \item[IDASPGMR\_SUCCESS] 
    The optional output value has been successfuly set.
  \item[\Id{IDASPGMR\_MEM\_NULL}]
    The \id{ida\_mem} pointer is \id{NULL}.
  \item[\Id{IDASPGMR\_LMEM\_NULL}]
    The {\idaspgmr} linear solver has not been initialized.
  \end{args}
}
{}
%%
%%
\ucfunction{IDASpgmrGetNumPrecSolves}
{
  flag = IDASpgmrGetNumPrecSolves(ida\_mem, \&npsolves);
}
{
  The function \ID{IDASpgmrGetNumPrecSolves} returns the
  cumulative number of calls made to the preconditioner 
  solve function, \id{psolve}.
}
{
  \begin{args}[npsolves]
  \item[ida\_mem] (\id{void *})
    pointer to the {\ida} memory block.
  \item[npsolves] (\id{long int})
    the current number of calls to \id{psolve}.
  \end{args}
}
{
  The return value \id{flag} (of type \id{int}) is one of
  \begin{args}[IDASPGMR\_LMEM\_NULL]
  \item[IDASPGMR\_SUCCESS] 
    The optional output value has been successfuly set.
  \item[\Id{IDASPGMR\_MEM\_NULL}]
    The \id{ida\_mem} pointer is \id{NULL}.
  \item[\Id{IDASPGMR\_LMEM\_NULL}]
    The {\idaspgmr} linear solver has not been initialized.
  \end{args}
}
{}
%%
%%
\ucfunction{IDASpgmrGetNumJtimesEvals}
{
  flag = IDASpgmrGetNumJtimesEvals(ida\_mem, \&njvevals);
}
{
  The function \ID{IDASpgmrGetNumJtimesEvals} returns the
  cumulative number made to the Jacobian-vector function,
  \id{jtimes}.
}
{
  \begin{args}[njvevals]
  \item[ida\_mem] (\id{void *})
    pointer to the {\ida} memory block.
  \item[njvevals] (\id{long int})
    the current number of calls to \id{jtimes}.
  \end{args}
}
{
  The return value \id{flag} (of type \id{int}) is one of
  \begin{args}[IDASPGMR\_LMEM\_NULL]
  \item[IDASPGMR\_SUCCESS] 
    The optional output value has been successfuly set.
  \item[\Id{IDASPGMR\_MEM\_NULL}]
    The \id{ida\_mem} pointer is \id{NULL}.
  \item[\Id{IDASPGMR\_LMEM\_NULL}]
    The {\idaspgmr} linear solver has not been initialized.
  \end{args}
}
{}
%%
%%
\ucfunction{IDASpgmrGetNumResEvals}
{
  flag = IDASpgmrGetNumResEvals(ida\_mem, \&nrevalsSG);
}
{
  The function \ID{IDASpgmrGetNumResEvals} returns the
  number of calls to the user residual function for
  finite difference Jacobian-vector product approximation.
}
{
  \begin{args}[nrevalsSG]
  \item[ida\_mem] (\id{void *})
    pointer to the {\ida} memory block.
  \item[nrevalsSG] (\id{long int})
    the number of calls to the user residual function.
  \end{args}
}
{
  The return value \id{flag} (of type \id{int}) is one of
  \begin{args}[IDASPGMR\_LMEM\_NULL]
  \item[IDA\_SUCCESS] 
    The optional output value has been successfuly set.
  \item[\Id{IDASPGMR\_MEM\_NULL}]
    The \id{ida\_mem} pointer is \id{NULL}.
  \item[\Id{IDASPGMR\_LMEM\_NULL}]
    The {\idaspgmr} linear solver has not been initialized.
  \end{args}
}
{
  The value \id{nrevalsSG} is incremented only if the default 
  \id{IDASpgmrDQJtimes} difference quotient function is used.
}
%%
%%
\ucfunction{IDASpgmrGetLastFlag}
{
  flag = IDASpgmrGetLastFlag(ida\_mem, \&flag);
}
{
  The function \ID{IDASpgmrGetLastFlag} returns the
  last return value from a {\idaspgmr} routine. 
}
{
  \begin{args}[ida\_mem]
  \item[ida\_mem] (\id{void *})
    pointer to the {\ida} memory block.
  \item[flag] (\id{int})
    the value of the last return flag from a {\idaspgmr} function.
  \end{args}
}
{
  The return value \id{flag} (of type \id{int}) is one of
  \begin{args}[IDASPGMR\_LMEM\_NULL]
  \item[\Id{IDASPGMR\_SUCCESS}] 
    The optional output value has been successfully set.
  \item[\Id{IDASPGMR\_MEM\_NULL}]
    The \id{ida\_mem} pointer is \id{NULL}.
  \item[\Id{IDASPGMR\_LMEM\_NULL}]
    The {\idaspgmr} linear solver has not been initialized.
  \end{args}
}
{
  If the {\idaspgmr} setup function failed (\id{IDASolve} returned \id{IDA\_LSETUP\_FAIL}),
  \id{flag} contains the return value of the preconditioner setup function \id{psetup}.

  If the {\idaspgmr} solve function failed (\id{IDASolve} returned \id{IDA\_LSETUP\_FAIL}),
  \id{flag} contains the error return flag from \id{SpgmrSolve} and will be one
  of: \id{SPGMR\_CONV\_FAIL} indicating a failure to converge, \id{SPGMR\_QRFACT\_FAIL}
  indicating a singular matrix foud during the QR factorization, \id{SPGMR\_PSOLVE\_FAIL\_REC}
  indicating that the preconditioner solve function \id{psolve} failed recoverably,
  \id{SPGMR\_MEM\_NULL} indicating that the {\spgmr} memory is \id{NULL}, \id{SPGMR\_ATIMES\_FAIL},
  indicating a failure in the Jacobian times vector function, \id{SPGMR\_PSOLVE\_FAIL\_UNREC}
  indicating that the preconditioner solve function \id{psolve} failed unrecoverably,
  \id{SPGMR\_GS\_FAIL} indicating a failure in the Gram-Schmidt procedure, 
  or \id{SPGMR\_QRSOL\_FAIL} indicating that the matrix $R$ was found to be singular during
  the QR solve phase.
}
%%
\index{IDASPGMR@{\idaspgmr} linear solver!optional output|)} 
\index{optional output!iterative linear solver|)}
%%
%%
%%===================================================================================

\subsection{IDA reinitialization function}\label{sss:idareinit}
\index{reinitialization}

The function \ID{IDAReInit} reinitializes the main {\ida} solver for
the solution of a problem, where a prior call to \Id{IDAMalloc} has
been made. The new problem must have the same size as the previous one.
\id{IDAReInit} performs the same input checking and initializations 
that \id{IDAMalloc} does, but does no memory allocation, assuming that the 
existing internal memory is sufficient for the new problem.             
                                                                 
The use of \id{IDAReInit} requires that the maximum method order,    
\Id{maxord}, is no larger for the new problem than for the problem  
specified in the last call to \id{IDAMalloc}.

The {\nvector} module specification \id{nvSpec} set for the previous problem
will be reused for the new problem.
%%
%%
\ucfunction{IDAReInit}
{
  flag = IDAReInit(ida\_mem, res, t0, y0, yp0, itol, reltol, abstol);
}
{
  The function \id{IDAReInit} provides required problem specifications 
  and reinitializes {\ida}.
}
{
  \begin{args}[ida\_mem]
  \item[ida\_mem] (\id{void *})
    pointer to the {\ida} memory block.
  \item[res] (\Id{IDAResFn})
    is the {\C} function which computes $F$. This function has the form 
    \id{f(N, t, y, ydot, resval, res\_data)} (for full details see \S\ref{ss:user_fct_sim}).
  \item[t0] (\id{realtype})
    is the initial value of $t$.
  \item[y0] (\id{N\_Vector})
    is the initial value of $y$. 
  \item[yp0] (\id{N\_Vector})
    is the initial value of $y^\prime$.   
  \item[itol] (\id{int}) 
    is either \Id{IDA\_SS} or \Id{IDA\_SV}, where \id{IDA\_SS} indicates scalar relative error 
    tolerance and scalar absolute error tolerance, while \id{IDA\_SV} indicates scalar
    relative error tolerance and vector absolute error tolerance. 
    The latter choice is important when the absolute error tolerance needs to
    be different for each component of the ODE. 
  \item[reltol] (\id{realtype *})
    is a pointer to the relative error tolerance.
  \item[abstol] (\id{void *})
    is a pointer to the absolute error tolerance.
  \end{args}
}
{
  The return flag \id{flag} (of type \id{int}) will be one of the following:
  \begin{args}[IDA\_NO\_MALLOC]
  \item[\Id{IDA\_SUCCESS}]
    The call to \id{IDAReInit} was successful.
  \item[\Id{IDA\_MEM\_NULL}] 
    The {\ida} memory block was not initialized through a 
    previous call to \id{IDACreate}.
  \item[\Id{IDA\_NO\_MALLOC}] 
    Memory space for the {\ida} memory block was not allocated through a 
    previous call to \id{IDAMalloc}.
  \item[\Id{IDA\_ILL\_INPUT}] 
    An input argument to \id{IDAReInit} has an illegal value.
  \end{args}
}
{
  If an error occurred, \id{IDAReInit} also prints an error message to the
  file specified by the optional input \id{errfp}.
}
%%
%%


%% RADU
%% GOT HERE....



%%===================================================================================
\section{User-supplied functions}\label{ss:user_fct_sim}
%%===================================================================================

The user-supplied functions consist of one function defining the DAE residual, 
(optionally) a function that provides Jacobian related information for the linear 
solver (if Newton iteration is chosen), and (optionally) one or two functions 
that define the preconditioner for use in the {\spgmr} algorithm. 
%%
%%--------------
%%
\subsection{Residual function}\label{ss:resFn}
\index{residual function}
The user must provide a function of type \ID{IDAResFn} defined as follows:
\usfunction{IDAResFn}
{
  typedef void (*IDAResFn)(&realtype tt, N\_Vector yy, N\_Vector yp,  \\
                           &N\_Vector rr, void *res\_data);
}
{
  This function computes the problem residual for given value
  of the independent variable $t$, state vector $y$, and derivative $\dot y$.
}
{
  \begin{args}[res\_data]
  \item[tt]
    is the current value of the independent variable.
  \item[yy]
    is the current value of the dependent variable vector, $y(t)$.
  \item[yp]
    is the current value of $y'(t)$.
  \item[rr]
    is the output vector $F(t,y,y')$.
  \item[res\_data]
    is a pointer to user data - the same as the \Id{res\_data}      
    parameter passed to \id{IDASetRdata}.   
  \end{args}
}
{
  An \id{IDAResFn} function type should return a value of $0$ if successful, 
  a positive value if a recoverable error occured (e.g. \id{yy} has an illegal value),
  or a negative value if a nonrecoverable error occured. 

  In the latter case, the integrator halts. 
  If a recoverable error occured, the integrator will attempt to correct and retry.
}
{
  Allocation of memory for \id{ydot} is handled within {\ida}.
}
%%
%%--------------
%%
\subsection{Jacobian information (direct method with dense Jacobian)}
\label{ss:djacFn}
\index{Jacobian approximation function!dense!user-supplied|(}
If the direct linear solver with dense treatment of the Jacobian is used 
(i.e. \Id{IDADense} is called in Step \ref{i:lin_solver} of \S\ref{ss:skeleton_sim}), 
the user may provide a function of type \ID{IDADenseJacFn} defined by
\usfunction{IDADenseJacFn}
{
  typedef int (*IDADenseJacFn)(&long int Neq, realtype tt, \\
                               &N\_Vector yy, N\_Vector yp, N\_Vector rr, \\ 
                               &realtype c\_j, void *jac\_data, \\
                               &DenseMat Jac, \\
                               &N\_Vector tmp1, N\_Vector tmp2, \\
                               &N\_Vector tmp3);
}
{
  This function computes the dense Jacobian (or an approximation to it)
  of the DAE system.
}
{
  \begin{args}[jac\_data]
  \item[Neq]
    is the problem size.
  \item[tt]
    is the current value of the independent variable.
  \item[yy]
    is the current value of the dependent variable vector, $y(t)$.
  \item[yp]
    is the current value of $y'(t)$.
  \item[rr]
    is the current value of the vector $F(t,y,y')$.
  \item[c\_j]
    is the scalar in the system Jacobian, proportional to the inverse of the
    step size.
  \item[jac\_data]
    is a pointer to user data - the same as the \id{jac\_data}      
    parameter passed to \id{IDADenseSetJacData}.   
  \item[Jac]
    is the output Jacobian matrix.  
  \item[tmp1]
  \item[tmp2]
  \item[tmp3]
    are pointers to memory allocated for variables of type \id{N\_Vector} 
    which can be used by \id{IDADenseJacFn} as temporary storage or work space.
  \end{args}
}
{
  An \id{IDADenseJacFn} function type should return $0$ if successful,
  a positive value if a recoverable error occured, or a negative value
  if a nonrecoverable error occured. 

  In the case of a recoverable eror return, 
  the integrator will attempt to recover by reducing the stepsize,
  and hence changing $\alpha$ in (\ref{e:DAE_Jacobian}).
}
{
  A user-supplied dense Jacobian function must load the \id{Neq} $\times$ \id{Neq}
  dense matrix \id{Jac} with an approximation to the Jacobian matrix $J$
  at the point (\id{t}, \id{yy}, \id{yp}).  Only nonzero elements need to be loaded
  into \id{Jac} because \id{Jac} is set to the zero matrix before the call
  to the Jacobian function. The type of \id{Jac} is \Id{DenseMat}. 
  
  The accessor macros \Id{DENSE\_ELEM} and \Id{DENSE\_COL} allow the user to
  read and write dense matrix elements without making explicit
  references to the underlying representation of the \id{DenseMat}
  type. \id{DENSE\_ELEM(Jac, i, j)} references the (\id{i}, \id{j})-th
  element of the dense matrix \id{Jac} (\id{i}, \id{j}$= 0\ldots$ \id{Neq}$-1$).
  This macro is for use in small problems in which efficiency of access is not a major
  concern.  Thus, in terms of indices $m$ and $n$ running from $1$ to
  \id{Neq}, the Jacobian element $J_{m,n}$ can be loaded with the statement
  \id{DENSE\_ELEM(Jac, m-1, n-1) =} $J_{m,n}$.  Alternatively,
  \id{DENSE\_COL(Jac, j)} returns a pointer to the storage for
  the \id{j}th column of \id{Jac} (\id{j}$= 0\ldots$ \id{Neq}$-1$), and the 
  elements of the \id{j}-th column
  are then accessed via ordinary array indexing.  Thus $J_{m,n}$ can be 
  loaded with the statements \id{col\_n = DENSE\_COL(Jac, n-1);}
  \id{col\_n[m-1] =} $J_{m,n}$.  For large problems, it is more 
  efficient to use \id{DENSE\_COL} than to use \id{DENSE\_ELEM}. 
  Note that both of these macros number rows and columns
  starting from $0$, not $1$.  

  The \id{DenseMat} type and the accessor macros \id{DENSE\_ELEM} and 
  \id{DENSE\_COL} are documented in \S\ref{ss:dense}.

  If the user's \id{IDADenseJacFn} function uses difference quotient approximations,
  it may need to access quantities not in the call list. These include the current
  stepsize, the error weights, etc. To obtain these, use the \id{IDAGet*} functions 
  described in \S\ref{ss:optional_output}. The unit roundoff can be accessed through
  the macro \id{DBL\_EPSILON} defined in \id{float.h}.
}
\index{Jacobian approximation function!dense!user-supplied|)}
%%
%%--------------
%%
\subsection{Jacobian information (direct method with banded Jacobian)}
\label{ss:bjacFn}
\index{Jacobian approximation function!band!user-supplied|(}
\index{half-bandwidths|(}
If the direct linear solver with banded treatment of the Jacobian is used 
(i.e. \Id{IDABand} is called in Step \ref{i:lin_solver} of \S\ref{ss:skeleton_sim}), 
the user may provide a function of type \ID{IDABandJacFn} defined as follows:
\usfunction{IDABandJacFn}
{
 typedef int (*IDABandJacFn)(&long int Neq, long int mupper, \\
                             &long int mlower, realtype tt, \\ 
                             &N\_Vector yy, N\_Vector yp, N\_Vector rr, \\
                             &realtype c\_j, void *jac\_data, \\
                             &BandMat Jac, \\
                             &N\_Vector tmp1, N\_Vector tmp2, \\
                             &N\_Vector tmp3);
}
{
  This function computes the banded Jacobian (or a banded approximation to it).
}
{
  \begin{args}[jac\_data]
  \item[Neq]
    is the problem size.
  \item[mlower]
  \item[mupper]
    are the lower and upper half bandwidth of the Jacobian.
  \item[tt]
    is the current value of the independent variable.
  \item[yy]
    is the current value of the dependent variable vector, $y(t)$.
  \item[yp]
    is the current value of $y'(t)$.
  \item[rr]
    is the current value of the vector $F(t,y,y')$.
  \item[c\_j]
    is the scalar in the system Jacobian, proportional to the inverse of the
    step size.
  \item[jac\_data]
    is a pointer to user data - the same as the \id{jac\_data}      
    parameter passed to \id{IDABandSetJacData}.   
  \item[Jac]
    is the output Jacobian matrix.  
  \item[tmp1]
  \item[tmp2]
  \item[tmp3]
    are pointers to memory allocated for variables of type \id{N\_Vector} 
    which can be used by \id{IDABandJacFn} as temporary storage or work space.    
  \end{args}
}
{
  A \id{IDABandJacFn} function type should return $0$ if successful,
  a positive value if a recoverable error occured, or a negative value
  if a nonrecoverable error occured. 

  In the case of a recoverable eror return, 
  the integrator will attempt to recover by reducing the stepsize,
  and hence changing $\alpha$ in (\ref{e:DAE_Jacobian}).
}
{
  A user-supplied band Jacobian function must load the band matrix \id{Jac}
  of type \Id{BandMat} with the elements of the Jacobian $J(t,y)$ at the
  point (\id{tt},\id{yy},\id{yp}).  Only nonzero elements need to be loaded into
  \id{Jac} because \id{Jac} is preset to zero before the call to the
  Jacobian function.  

  The accessor macros \Id{BAND\_ELEM}, \Id{BAND\_COL}, and \Id{BAND\_COL\_ELEM} 
  allow the user to read and write band matrix elements without making specific 
  references to the underlying representation of the \id{BandMat} type.
  \id{BAND\_ELEM(Jac, i, j)} references the (\id{i}, \id{j})th element of the 
  band matrix \id{Jac}, counting from $0$.
  This macro is for use in small problems in which efficiency of access is not
  a major concern.  Thus, in terms of indices $m$ and $n$ running from $1$ to
  $N$ with $(m,n)$ within the band defined by \id{mupper} and
  \id{mlower}, the Jacobian element $J_{m,n}$ can be loaded with the 
  statement \id{BAND\_ELEM(Jac, m-1, n-1) =} $J_{m,n}$. The elements within
  the band are those with \id{-mupper} $\le$ \id{m-n} $\le$ \id{mlower}.
  Alternatively, \id{BAND\_COL(Jac, j)} returns a pointer to the diagonal element of the
  \id{j}th column of \id{Jac}, and if we assign this address to 
  \id{realtype *col\_j}, then the \id{i}th element of the \id{j}th column is
  given by \id{BAND\_COL\_ELEM(col\_j, i, j)}, counting from $0$.
  Thus for $(m,n)$ within the band, $J_{m,n}$ can be loaded by setting 
  \id{col\_n = BAND\_COL(Jac, n-1);} \id{BAND\_COL\_ELEM(col\_n, m-1, n-1) =} $J_{m,n}$.
  The elements of the \id{j}th column can also be accessed
  via ordinary array indexing, but this approach requires knowledge of
  the underlying storage for a band matrix of type \id{BandMat}.  
  The array \id{col\_n} can be indexed from $-$\id{mupper} to \id{mlower}.
  For large problems, it is more efficient to use the combination of
  \id{BAND\_COL} and \id{BAND\_COL\_ELEM} than to use the
  \id{BAND\_ELEM}.  As in the dense case, these macros all number rows
  and columns starting from $0$, not $1$.  

  The \id{BandMat} type and the accessor macros \id{BAND\_ELEM}, \id{BAND\_COL}, and
  \id{BAND\_COL\_ELEM} are documented in \S\ref{ss:band}.

  If the user's \id{IDABandJacFn} function uses difference quotient approximations,
  it may need to access quantities not in the call list. These include the current
  stepsize, the error weights, etc. To obtain these, use the \id{IDAGet*} functions 
  described in \S\ref{ss:optional_output}. The unit roundoff can be accessed through
  the macro \id{DBL\_EPSILON} defined in \id{float.h}.
}
\index{half-bandwidths|)}
\index{Jacobian approximation function!band!user-supplied|)}
%%
%%----------------
%%
\subsection{Jacobian information (SPGMR matrix-vector product)}
\label{ss:jtimesFn}
\index{Jacobian approximation function!Jacobian times vector!user-supplied|(}
If an iterative {\spgmr} linear solver is selected (\id{IDASpgmr} is called in step 
\ref{i:lin_solver} of \S\ref{ss:skeleton_sim}) the user may provide a function
of type \ID{IDASpgmrJacTimesVecFn} in the following form:
\usfunction{IDASpgmrJacTimesVecFn}
{
  typedef int (*IDASpgmrJacTimesVecFn)(&realtype tt, \\
                                       &N\_Vector yy, N\_Vector yp, N\_Vector rr, \\
                                       &N\_Vector v, N\_Vector Jv, \\
                                       &realtype c\_j, void *jac\_data, \\
                                       &N\_Vector tmp1, N\_Vector tmp2);
}
{
  This function computes the product of the problem Jacobian and the
  vector \id{v} (or an approximation to it).
}
{
  \begin{args}[jac\_data]
  \item[tt]
    is the current value of the independent variable.
  \item[yy]
    is the current value of the dependent variable vector, $y(t)$.
  \item[yp]
    is the current value of $y'(t)$.
  \item[rr]
    is the current value of the vector $F(t,y,y')$.
  \item[c\_j]
    is the scalar in the system Jacobian, proportional to the inverse of the
    step size.
  \item[v]
    is the vector by which the Jacobian must be multiplied to the right.
  \item[Jv]
      is the output vector computed.
  \item[jac\_data]
    is a pointer to user data - the same as the \id{jac\_data}      
    parameter passed to \id{IDASpgmrSetJacData}.   
  \item[tmp1]
  \item[tmp2]
    are pointers to memory allocated for variables of type \id{N\_Vector} 
    which can be used by \id{IDASpgmrJacTimesVecFn} as temporary storage or work space.
  \end{args}
}
{  
  The value to be returned by the Jacobian times vector function should be
  $0$ if successful. 
  A positive value indicates that a recoverable error occured, while a negative 
  value indicates that a nonrecoverable error occured. 

  In the case of a recoverable eror return, the integrator will attempt to recover 
  by reducing the stepsize, and hence changing $\alpha$ in (\ref{e:DAE_Jacobian}).
}
{}
\index{Jacobian approximation function!Jacobian times vector!user-supplied|)}
%%
%%--------------
%%
\subsection{Preconditioning (SPGMR linear system solution)}
\label{ss:psolveFn}
\index{preconditioning!user-supplied}
\index{IDASPGMR@{\idaspgmr} linear solver!preconditioner solve function}
If preconditioning is used, then the user must provide a {\C} function to
solve the linear system $Pz = r$ where $P$ may be either a left or a
right preconditioner matrix.
This function must be of type \ID{IDASpgmrPrecSolveFn}, defined as follows:
%%
%%
\usfunction{IDASpgmrPrecSolveFn}
{
  typedef int (*IDASpgmrPrecSolveFn)(&realtype tt, \\
                                     &N\_Vector yy, N\_Vector yp, N\_Vector rr, \\
                                     &N\_Vector rvec, N\_Vector zvec, \\ 
                                     &realtype c\_j, realtype delta, \\
                                     &void *prec\_data, N\_Vector tmp);
}
{
  This function solves the preconditioning system $Pz = r$.
}
{  
  \begin{args}[prec\_data]
  \item[tt]
    is the current value of the independent variable.
  \item[yy]
    is the current value of the dependent variable vector, $y(t)$.
  \item[yp]
    is the current value of $y'(t)$.
  \item[rr]
    is the current value of the vector $F(t,y,y')$.
  \item[rvec]
    is the right-hand side vector of the linear system.
  \item[zvec]
    is the output vector computed.
  \item[c\_j]
    is the scalar in the system Jacobian, proportional to the inverse of the
    step size.
  \item[delta]
    is an input tolerance to be used if an iterative method 
    is employed in the solution.  In that case, the residual 
    vector $Res = r - P z$ of the system should be made less than 
    \id{delta} in weighted $l_2$ norm,     
    i.e., $\sqrt{\sum_i (Res_i \cdot ewt_i)^2 } < delta$.
    To obtain the \id{N\_Vector} \id{ewt}, call \id{IDAGetErrWeights} 
    (see \S\ref{ss:optional_output}).
  \item[prec\_data]
    is a pointer to user data - the same as the \id{prec\_data}      
    parameter passed to the function \id{IDASpgmrSetPrecData}.
  \item[tmp]
    is a pointer to memory allocated for a variable of type    
    \id{N\_Vector} which can be used for work space.
  \end{args}
}
{
  The value to be returned by the preconditioner solve function is a flag indicating 
  whether it was successful.  This value should be $0$ if successful, 
  positive for a recoverable error (in which case the step will be retried),     
  negative for an unrecoverable error (in which case the integration is halted). 
}
{}
%%
%%-----------------
%%
\subsection{Preconditioning (SPGMR Jacobian data)}
\label{ss:precondFn}
\index{preconditioning!user-supplied}
\index{IDASPGMR@{\idaspgmr} linear solver!preconditioner setup function}
If the user's preconditioner requires that any Jacobian related data
be evaluated or preprocessed, then this needs to be done in a
user-supplied {\C} function of type \ID{IDASpgmrPrecSetupFn}, defined as follows:
\usfunction{IDASpgmrPrecSetupFn}
{
  typedef int (*IDASpgmrPrecSetupFn)(&realtype tt, \\
                                     &N\_Vector yy, N\_Vector yp, N\_Vector rr, \\
                                     &realtype c\_j, void *prec\_data,\\
                                     &N\_Vector tmp1, N\_Vector tmp2,\\
                                     &N\_Vector tmp3);
}
{
  This function evaluates and/or preprocesses Jacobian related data needed
  by the preconditioner.
}
{
  The arguments of a \id{IDASpgmrPrecSetupFn} are as follows:
  \begin{args}[prec\_data]
  \item[tt]
    is the current value of the independent variable.
  \item[yy]
    is the current value of the dependent variable vector, $y(t)$.
  \item[yp]
    is the current value of $y'(t)$.
  \item[rr]
    is the current value of the vector $F(t,y,y')$.
  \item[c\_j]
    is the scalar in the system Jacobian, proportional to the inverse of the
    step size.
  \item[prec\_data]
    is a pointer to user data, the same as the \id{prec\_data}      
    parameter passed to \id{IDASpgmrSetPrecData}.
  \item[tmp1]
  \item[tmp2]
  \item[tmp3]
    are pointers to memory allocated for variables of type \id{N\_Vector} 
    which can be used by \id{IDASpgmrPrecSetupFn} as temporary storage or work space.    
  \end{args}
}
{
  The value to be returned by the preconditioner setup function is a flag indicating 
  whether it was successful.  This value should be $0$ if successful, 
  positive for a recoverable error (in which case the step will be retried),     
  negative for an unrecoverable error (in which case the integration is halted). 
}
{
  The operations performed by this function might include forming a crude 
  approximate Jacobian, and performing an LU factorization on the resulting            
  approximation.

  Each call to the preconditioner setup function is preceded by a call to     
  the \id{IDAResFn} user function with the same (\id{tt},\id{yy},\id{yp}) arguments.  
  Thus the preconditioner setup function can use any auxiliary data that is 
  computed and saved during the evaluation of the DAE residual.
  
  This function is not called in advance of every call to the preconditioner solve
  function, but rather is called only as often as needed to achieve convergence in the
  Newton iteration. 
}

%%===================================================================================
\section{A parallel band-block-diagonal preconditioner module}\label{sss:idabbdpre}
%%===================================================================================

A principal reason for using a parallel DAE solver such as {\ida} lies
in the solution of partial differential equations (PDEs).  Moreover,
the use of a Krylov iterative method for the solution of many such
problems is motivated by the nature of the underlying linear system of
equations (\ref{e:DAE_Newtoncorr}) that must be solved at each time step.  The
linear algebraic system is large, sparse, and structured. However, if
a Krylov iterative method is to be effective in this setting, then a
nontrivial preconditioner needs to be used.  Otherwise, the rate of
convergence of the Krylov iterative method is usually unacceptably
slow.  Unfortunately, an effective preconditioner tends to be
problem-specific.

However, we have developed one type of preconditioner that treats a
rather broad class of PDE-based problems.  It has been successfully
used for several realistic, large-scale problems \cite{HiTa:98} and is
included in a software module within the {\ida} package. This module
works with the parallel vector module {\nvecp} and 
generates a preconditioner that is a block-diagonal matrix with each
block being a band matrix. The blocks need not have the same number of
super- and sub-diagonals and these numbers may vary from block to
block. This Band-Block-Diagonal Preconditioner module is called
{\idabbdpre}.

\index{IDABBDPRE@{\idabbdpre} preconditioner!description|(}
\index{preconditioning!band-block diagonal}
One way to envision these preconditioners is to think of the domain of
the computational PDE problem as being subdivided into $M$ non-overlapping
sub-domains.  Each of these sub-domains is then assigned to one of the
$M$ processors to be used to solve the DAE system. The basic idea is
to isolate the preconditioning so that it is local to each processor,
and also to use a (possibly cheaper) approximate residual
function. This requires the definition of a new function $G(t,y,{\dot y})$
which approximates the function $F(t, y, {\dot y})$ in the definition of the DAE
system (\ref{e:DAE}). However, the user may set $G = F$.  Corresponding
to the domain decomposition, there is a decomposition of the solution
vector $y$ into $M$ disjoint blocks $y_m$, and a decomposition of $G$
into blocks $G_m$.  The block $G_m$ depends on $y_m$ and also on
components of blocks $y_{m'}$ associated with neighboring sub-domains
(so-called ghost-cell data).  Let $\bar{y}_m$ denote $y_m$ augmented
with those other components on which $G_m$ depends.  Then we have
\begin{equation}
  G(t,y) = [G_1(t,\bar{y}_1), G_2(t,\bar{y}_2), \ldots, G_M(t,\bar{y}_M)]^T
\end{equation}
and each of the blocks $G_m(t, \bar{y}_m)$ is uncoupled from the others.

The preconditioner associated with this decomposition has the form 
\begin{equation}
  P= diag[P_1, P_2, \ldots, P_M]
\end{equation}
where 
\begin{equation}
  P_m \approx \partial G_m / \partial y_m
  + \alpha \partial G_m / \partial {\dot y}_m
\end{equation}
This matrix is taken to be banded, with
upper and lower half-bandwidths \id{mudq} and \id{mldq} defined as
the number of non-zero diagonals above and below the main diagonal,
respectively. The difference quotient approximation is computed using
\id{mudq} $+$ \id{mldq} $+ 2$ evaluations of $G_m$, but only a matrix
of bandwidth \id{mu} $+$ \id{ml} $+ 1$ is retained. 
Neither pair of parameters need be the true half-bandwidths of the Jacobians of the
local block of $G$, if smaller values provide a more efficient
preconditioner. The solution of the complete linear system
\begin{equation}
  Px = b
\end{equation}
reduces to solving each of the equations 
\begin{equation}
  P_m x_m = b_m
\end{equation}
and this is done by banded LU factorization of $P_m$ followed by a banded
backsolve.
\index{IDABBDPRE@{\idabbdpre} preconditioner!description|)}

%%
%%------------------------------------------------------------------------------------
%%

\index{IDABBDPRE@{\idabbdpre} preconditioner!user-supplied functions|(}
%%
The {\idabbdpre} module calls two user-provided functions to construct $P$: 
a required function \id{Gres} (of type \id{IDABBDLocalFn}) 
which approximates the residual function $G(t,y) \approx F(t,y)$ and which 
is computed locally, and an optional function \id{Gcomm} (of type \id{IDABBDCommFn}) 
which performs all inter-process communication necessary to evaluate the approximate 
residual $G$.
These are in addition to the user-supplied residual function \id{res}.
Both functions take as input the same pointer \id{res\_data} as that passed
by the user to \id{IDASetRdata} and passed to the user's function \id{res},
and neither function has a return value. The user is responsible for
providing space (presumably within \id{res\_data}) for components of \id{yy}
and \id{yp} that are communicated by \id{Gcomm} from the other processors, and that are
then used by \id{Gres}, which is not expected to do any communication.
%%
%%
\usfunction{IDABBDLocalFn}
{
  typedef void (*IDABBDLocalFn)(&long int Nlocal, realtype tt,  \\
                                &N\_Vector yy, N\_Vector yp, N\_Vector gval, \\
                                &void *res\_data);
}
{
  This function computes $G(t,y,{\dot y})$. It loads the vector
  \id{glocal} as a function of \id{tt}, \id{yy}, and \id{yp}.  
}
{
  \begin{args}[res\_data]
  \item[Nlocal] 
    is the local vector length.
  \item[tt]
    is the value of the independent variable.
  \item[yy]
    is the dependent variable. 
  \item[yp]
    is the derivative of the dependent variable. 
  \item[gval]
    is the output vector.
  \item[res\_data]
    is a pointer to user data - the same as the \Id{res\_data}      
    parameter passed to \id{IDASetRdata}.  
  \end{args}
}
{
  An \id{IDABBDLocalFn} function type does not have a return value.
}
{
  This function assumes that all inter-processor communication of data needed to 
  calculate \id{gval} has already been done, and this data is accessible within
  \id{res\_data}.

  The case where $G$ is mathematically identical to $F$ is allowed.
}
%%
%%
\usfunction{IDABBDCommFn}
{
  typedef void (*IDABBDCommFn)(&long int Nlocal, realtype tt,  \\
                               &N\_Vector yy, N\_Vector yp, void *res\_data);
}
{
  This function performs all inter-processor communications necessary 
  for the execution of the \id{Gres} function above, using the input 
  vectors \id{yy} and \id{yp}.
}
{
  \begin{args}[res\_data]
  \item[Nlocal] 
    is the local vector length.
  \item[tt]
    is the value of the independent variable.
  \item[yy]
    is the dependent variable. 
  \item[yp]
    is the derivative of the dependent variable. 
  \item[res\_data]
    is a pointer to user data - the same as the \Id{res\_data}      
    parameter passed to \id{IDASetRdata}.  
  \end{args}
}
{
  An \id{IDABBDCommFn} function type does not have a return value.
}
{
  The \id{Gcomm} function is expected to save communicated data in space defined within the
  structure \id{res\_data}. 

  Each call to the \id{Gcomm} function is preceded by a call to the residual
  function \id{res} with the same (\id{tt}, \id{yy}, \id{yp}) arguments.  
  Thus \id{Gcomm} can omit 
  any communications done by \id{res} if relevant to the evaluation of \id{Glocal}.  
  If all necessary comunication was done in \id{res}, then \id{Gcomm} $=$ \id{NULL}
  can be passed in the call to \id{IDABBDPrecAlloc} (see below).

}
%%
\index{IDABBDPRE@{\idabbdpre} preconditioner!user-supplied functions|)}

%%
%%------------------------------------------------------------------------------------
%%

\index{IDABBDPRE@{\idabbdpre} preconditioner!usage|(}
%%
Besides the header files required for the integration of the ODE problem
(see \S\ref{ss:header_sim}),  to use the {\idabbdpre} module, the main program 
must include the header file \id{idabbdpre.h} which declares the needed
function prototypes.\index{header files}

The following is a summary of the usage of this module and describes the sequence 
of calls in the user main program. Steps that are unchanged from the user main
program presented in \S\ref{ss:skeleton_sim} are grayed-out.
%%
%%
\index{User main program!IDABBDPRE@{\idabbdpre} usage}
\begin{Steps}
\item 
  \textcolor{gray}{\bf Initialize MPI}

\item
  \textcolor{gray}{\bf Set problem dimensions}

\item
  \textcolor{gray}{\bf Set vector of initial values}
 
\item
  \textcolor{gray}{\bf Create {\ida} object}

\item
  \textcolor{gray}{\bf Set optional inputs}

\item
  \textcolor{gray}{\bf Allocate internal memory}

\item \label{i:bbdpre_init}
  {\bf Initialize the {\idabbdpre} preconditioner module}

  Specify the upper and lower bandwidths \id{mudq}, \id{mldq} and
  \id{mukeep}, \id{mlkeep} and call 

   \id{
     \begin{tabular}[t]{@{}r@{}l@{}}
       bbd\_data = IBBDPrecAlloc(&ida\_mem, Nlocal, mudq, mldq, \\
                                 &mukeep, mlkeep, dq\_rel\_yy, Gres, Gcomm);
     \end{tabular}
   }

  to allocate memory for and initialize a data structure \id{bbd\_data} to be 
  passed to the {\idaspgmr} linear solver. The last two arguments of \id{IBBDPrecAlloc}
  are the two user-supplied functions described above.

\item \label{i:bbdpre_attach}
  {\bf Attach the {\idaspgmr} linear solver}

  \id{flag = IDABBDSpgmr(ida\_mem, maxl, bbd\_data);}

  The function \Id{IDABBDSpgmr} is a wrapper around the {\idaspgmr} specification
  function \id{IDASpgmr} and performs the following actions:
  \begin{itemize}
    \item Attaches the {\idaspgmr} linear solver to the main {\ida} solver memory;
    \item Sets the preconditioner data structure for {\idabbdpre};
    \item Sets the preconditioner setup function for {\idabbdpre};
    \item Sets the preconditioner solve function for {\idabbdpre};
  \end{itemize}

\item
  \textcolor{gray}{\bf Set linear solver optional inputs}

  Note that the user should not overwrite the preconditioner data, setup function, 
  or solve function through calls to {\idaspgmr} optional input functions.

\item
  \textcolor{gray}{\bf Advance solution in time}

\item
  \textcolor{gray}{\bf Deallocate memory for solution vector}

\item \label{i:bbdpre_free}
  {\bf Free the {\idabbdpre} data structure}

  \id{IBBDPrecFree(bbd\_data);}

\item
  \textcolor{gray}{\bf Free solver memory}
  
\item 
  \textcolor{gray}{\bf Finalize MPI}

\end{Steps}
%%
\index{IDABBDPRE@{\idabbdpre} preconditioner!usage|)}
%%
%%------------------------------------------------------------------------------------
%%
\index{IDABBDPRE@{\idabbdpre} preconditioner!user-callable functions|(}
%%
The three user-callable functions that initialize, attach, and deallocate
the {\idabbdpre} preconditioner module (steps \ref{i:bbdpre_init},
\ref{i:bbdpre_attach}, and \ref{i:bbdpre_free} above) are described
next.
%%
\index{half-bandwidths}
\ucfunction{IBBDPrecAlloc}
{
   \begin{tabular}[t]{@{}r@{}l@{}}
     bbd\_data = IBBDPrecAlloc(&ida\_mem, Nlocal, mudq, mldq, \\
                                &mukeep, mlkeep, dq\_rel\_yy, Gres, Gcomm);
   \end{tabular}
}
{
  The function \ID{IBBDPrecAlloc} initializes and allocates
  memory for the {\idabbdpre} preconditioner.
}
{
  \begin{args}[dq\_rel\_yy]
  \item[ida\_mem] (\id{void *})
    pointer to the {\ida} memory block.
  \item[Nlocal] (\id{long int})
    local vector dimension.
  \item[mudq] (\id{long int})
    upper half-bandwidth to be used in the difference-quotient Jacobian approximation.
  \item[mldq] (\id{long int})
    lower half-bandwidth to be used in the difference-quotient Jacobian approximation.
  \item[mukeep] (\id{long int})
    upper half-bandwidth of the retained banded approximate Jacobian block.
  \item[mlkeep] (\id{long int})
    lower half-bandwidth of the retained banded approximate Jacobian block.
  \item[dq\_rel\_yy] (\id{realtype})
    the relative increment in components of \id{y} used in the difference quotient
    approximations.  The default is \id{dqrely}$ = \sqrt{\text{unit roundoff}}$, which
    can be specified by passing \id{dqrely}$ = 0.0$.
  \item[Gres] (\id{IDABBDLocalFn})
    the {\C} function which computes the local residual approximation 
    $G(t,y,{\dot y})$.
  \item[Gcomm] (\id{IDABBDCommFn})
    the optional {\C} function which performs all inter-process communication required for
    the computation of $G(t,y,{\dot y})$.
  \end{args}
}
{
  If successful, \id{IBBDPrecAlloc} returns a pointer to the newly created 
  {\idabbdpre} memory block (of type \id{void *}).
  If an error occurred, \id{IBBDPrecAlloc} returns \id{NULL}.
}
{
  If one of the half-bandwidth \id{mudq} or \id{mldq} to be used in the 
  difference-quotient calculation of the approximate Jacobian is negative or 
  exceeds the value \id{Nlocal}$-1$, it is replaced with 0 \id{local\_N}$-1$.

  The half-bandwidths \id{mudq} and \id{mldq} need not be the true 
  half-bandwidths of the Jacobian of the local block of $g$,    
  when smaller values may provide a greater efficiency.       

  Also, the half-bandwidths \id{mukeep} and \id{mlkeep} of the retained 
  banded approximate Jacobian block may be even smaller,      
  to reduce storage and computation costs further.            

  For all four half-bandwidths, the values need not be the    
  same on every processor.
}
%%
%%
\ucfunction{IDABBDSpgmr}
{
  flag = IDABBDSpgmr(ida\_mem, maxl, bbd\_data);
}
{
  The function \ID{IDABBDSpgmr} links the {\idabbdpre} data to the
  {\idaspgmr} linear solver and attaches the latter to the {\ida}
  memory block.
}
{
  \begin{args}[bbd\_data]
  \item[ida\_mem] (\id{void *})
    pointer to the {\ida} memory block.
  \item[maxl] (\id{int})
    \index{maxl@\texttt{maxl}}
    maximum dimension of the Krylov subspace to be used. Pass $0$ to use the 
    default value \id{IDASPGMR\_MAXL}$=5$.
  \item[bbd\_data] (\id{void *})
    pointer to the {\idabbdpre} data structure.
  \end{args}
}
{
  The return value \id{flag} (of type \id{int}) is one of
  \begin{args}[IDASPGMR\_ILL\_INPUT]
  \item[\Id{IDASPGMR\_SUCCESS}] 
    The {\idaspgmr} initialization was successful.
  \item[\Id{IDASPGMR\_MEM\_NULL}]
    The \id{ida\_mem} pointer is \id{NULL}.
  \item[\Id{IDASPGMR\_MEM\_FAIL}]
    A memory allocation request failed.
  \item[\Id{IDA\_PDATA\_NULL}]
    The {\idabbdpre} preconditioner has not been initialized.
  \end{args}
}
{}
%%
\ucfunction{IBBDPrecFree}
{
  IBBDPrecFree(bbd\_data);
}
{
  The function \ID{IBBDPrecFree} frees the pointer allocated by
  \id{IBBDPrecAlloc}.
}
{
  The only argument of \id{IBBDPrecFree} is the pointer to the {\idabbdpre} 
  data structure (of type \id{void *}).
}
{
  The function \id{IBBDPrecFree} has no return value.
}
{}
%%
The {\idabbdpre} module also provides a reinitialization function to allow
solving  a sequence of problems of the same size with {\idaspgmr}/{\idabbdpre},
provided there is no change in \id{local\_N}, \id{mukeep}, or \id{mlkeep}.
After solving one problem, and after calling \id{IDAReInit} to re-initialize 
{\ida} for a subsequent problem, a call to \id{IBBDPrecReInit} can be made
to change any of the following: the half-bandwidths \id{mudq} and \id{mldq} 
used in the difference-quotient Jacobian approximations, the relative increment 
\id{dq\_rel\_yy}, or one of the user-supplied functions \id{Gres} and \id{Gcomm}.
%%
\ucfunction{IBBDPrecReInit}
{
  flag = IBBDPrecReInit(bbd\_data, mudq, mldq, dq\_rel\_yy, Gres, Gcomm);
}
{
  The function \ID{IBBDPrecReInit} reinitializes the {\idabbdpre} preconditioner.
}
{
  \begin{args}[dq\_rel\_yy]
  \item[bbd\_data] (\id{void *})
    pointer to the {\idabbdpre} data structure.
  \item[mudq] (\id{long int})
    upper half-bandwidth to be used in the difference-quotient Jacobian approximation.
  \item[mldq] (\id{long int})
    lower half-bandwidth to be used in the difference-quotient Jacobian approximation.
  \item[dq\_rel\_yy] (\id{realtype})
    the relative increment in components of \id{y} used in the difference quotient
    approximations.  The default is \id{dqrely} $= \sqrt{\text{unit roundoff}}$, which
    can be specified by passing \id{dqrely} $= 0.0$.
  \item[Gres] (\id{IDABBDLocalFn})
    the {\C} function which computes the local residual approximation 
    $G(t,y,{\dot y})$.
  \item[Gcomm] (\id{IDACommFn})
    the optional {\C} function which performs all inter-process communication required for
    the computation of $G(t,y,{\dot y})$.
  \end{args}
}
{
  The return value of \id{IBBDPrecReInit} is always \Id{IDA\_SUCCESS}.
}
{
  If one of the half-bandwidth \id{mudq} or \id{mldq} is negative or
  exceeds the value \id{Nlocal}$-1$, it is replaced with 0 \id{local\_N}$-1$.
}
%%
\index{IDABBDPRE@{\idabbdpre} preconditioner!user-callable functions|)}
%%
%%------------------------------------------------------------------------------------
%%
\index{optional output!band-block-diagonal preconditioner|(}
\index{IDABBDPRE@{\idabbdpre} preconditioner!optional output|(}
The following three optional output functions are available for use with
the {\idabbdpre} module:
%%
\index{memory requirements!IDABBDPRE@{\idabbdpre} preconditioner}
\ucfunction{IBBDPrecGetWorkSpace}
{
  flag = IBBDPrecGetWorkSpace(bbd\_data, \&lenrwBBDP, \&leniwBBDP);
}
{
  The function \ID{IBBDPrecGetWorkSpace} returns the sizes
  of the {\idabbdpre} real and integer workspaces.
}
{
  \begin{args}[lenrwBBDP]
  \item[bbd\_data] (\id{void *})
    pointer to the {\idabbdpre} data structure.
  \item[lenrwBBDP] (\id{long int})
    the number of real values in the {\idabbdpre} workspace.
  \item[leniwBBDP] (\id{long int})
    the number of integer values in the {\idabbdpre} workspace.
  \end{args}
}
{
  The return value \id{flag} (of type \id{int}) is one of
  \begin{args}[IDA\_PDATA\_NULL]
  \item[IDA\_SUCCESS] 
    The optional output value has been successfuly set.
  \item[\Id{IDA\_PDATA\_NULL}]
    The {\idabbdpre} preconditioner has not been initialized.
  \end{args}
}
{
  In terms of the local vector dimension $N_l$, the actual size of the real workspace is
  $N_l \,(2$ \id{mlkeep} $+$ \id{mukeep} $+$ \id{smu} $+2)$ \id{realtype} words,
  where \id{smu} = $\min ( N_l - 1 ,$ \id{mukeep} $+$ \id{mlkeep}$)$.

  The actual size of the integer workspace is $N_l$ integer words.
}
%%
%%
\ucfunction{IBBDPrecGetNumGfnEvals}
{
  flag = IBBDPrecGetNumGfnEvals(bbd\_data, \&ngevalsBBDP);
}
{
  The function \ID{IBBDPrecGetNumGfnEvals} returns the
  number of calls to the user \id{Gres} function due to the 
  finite difference approximation of the Jacobian blocks used within
  {\idabbdpre}'s preconditioner setup function.
}
{
  \begin{args}[ngevalsBBDP]
  \item[bbd\_data] (\id{void *})
    pointer to the {\idabbdpre} data structure.
  \item[ngevalsBBDP] (\id{long int})
    the number of calls to the user \id{Gres} function.
  \end{args}
}
{
  The return value \id{flag} (of type \id{int}) is one of
  \begin{args}[IDA\_PDATA\_NULL]
  \item[IDA\_SUCCESS] 
    The optional output value has been successfuly set.
  \item[\Id{IDA\_PDATA\_NULL}]
    The {\idabbdpre} preconditioner has not been initialized.
  \end{args}
}
{}
%%
\index{IDABBDPRE@{\idabbdpre} preconditioner!optional output|)}
\index{optional output!band-block-diagonal preconditioner|)}
%%
%%
%-------------------------------------------------------

The costs associated with {\idabbdpre} also include \id{nlinsetups} LU
factorizations, \id{nlinsetups} calls to \id{Gcomm}, and \id{npsolves} banded
backsolve calls, where \id{nlinsetups} and \id{npsolves} are optional {\ida}
outputs (see \S\ref{ss:optional_output}).

Similar block-diagonal preconditioners could be considered with different
treatment of the blocks $P_m$. For example, incomplete LU factorization or
an iterative method could be used instead of banded LU factorization.


