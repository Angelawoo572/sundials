%===================================================================================
\section{Parallel example problems}\label{s:ex_parallel}
%===================================================================================

\subsection{An IDABBDPRE preconditioner example: \id{iwebbbd}}\label{ss:iwebbbd}

In this example, \id{iwebbbd}, we solve the same food web problem as with
\id{iwebsb}, but in parallel and with the {\idaspgmr} linear solver and
using the {\idabbdpre} module, which generates and uses a band-block-diagonal 
preconditioner.  
The half-bandwidths of the Jacobian block on each processor are both equal to
$\id{NUM\_SPECIES}\cdot\id{MXSUB}$, and that is the value supplied as \id{mudq} and \id{mldq}
in the call to \id{IDABBDPrecAlloc}.  But in order to reduce storage and computation
costs for preconditioning, we supply the values \id{mukeep} = \id{mlkeep} = 2
(= \id{NUM\_SPECIES}) as the half-bandwidths of the retained band matrix blocks.
This means that the Jacobian elements are computed with a difference quotient
scheme using the true bandwidth of the block, but only a narrow band matrix
(bandwidth 5) is kept as the preconditioner.  The source is listed in
Appendix \ref{s:iwebbbd_c}. 

In the parallel setting, we can think of the processors as being laid out
in a grid \id{NPEX} $\times$ \id{NPEY}, with each processor computing its
contiguous subset of the solution vector. As a consequence, the computation 
of the residual vector requires that each processor exchanges boundary information
(namely the components at all interior subgrid boundaries) with its neighboring processors.
The message-passing (implemented in the function \id{rescomm}) uses blocking sends, 
non-blocking receives, and receive-waiting, in routines \id{BSend}, \id{BRecvPost},
and \id{BRecvWait}, respectively.
%%
The data received from each neighboring processor is then loaded into a work array,
\id{cext}, which contains this ghost cell data and the local portion of the solution.

The local portion of the residual vector is then computed in \id{reslocal} which
assumes that all inter-processor communication of data
needed to calculate \id{rr} has already been done.  Components at interior
subgrid boundaries are assumed to be in the work array \id{cext}.
The local portion of the solution vector \id{cc} is first copied into \id{cext}.
The exterior Neumann boundary conditions are explicitly handled here
by copying data from the first interior mesh line to the ghost cell
locations in \id{cext}.  Then the reaction and diffusion terms are
evaluated in terms of the \id{cext} array, and the residuals are formed.
The reaction terms are saved separately in the vector \id{rates} in the user-data
structure \id{webdata} for use by the preconditioner setup routine. 

The function \id{reslocal} is also passed to the {\idabbdpre} preconditioner
as the \id{Gres} argument, while a \id{NULL} pointer is passed for the \id{Gcomm}
argument (since all required communication for the evaluation of \id{Gres} was
already done for \id{resweb}).

The main program begins with MPI calls to initialize MPI and to set
multi-processor environment parameters \id{npes} (number of processes) and
\id{thispe} (local process index).  Then the local and global vector lengths
are set, the user-data structure \id{webdata} is created and initialized, and
\id{N\_Vector} variables are created and initialized for the initial conditions
(\id{cc} and \id{cp}) and for the vector \id{id} specifying the differential 
and algebraic components of the solution vector. The \id{id} vector is used
in \id{IDACalcIC} which corrects the initial values so that they are
consistent with the DAE algebraic constraints.
A temporary \id{N\_Vector} \id{res} is also created here and destroyed after
its use in \id{SetInitialProfiles}.

The main program continues by creating and allocating the {\ida} memory
block, initializing the {\idabbdpre} preconditioner, and attaching the
{\idaspgmr} linear solver to the {\ida} solver.

In a loop over the desired output times, the main solver function \id{IDASolve}
is called and selected solution components (at the bottom-left and top-right
corners of the computational domain) are collected on process 0 and printed
to \id{stdout}. The main program ends by printing final solver statistics,
freeing memory, and finalizing MPI.

Sample output from \id{iwebbbd} follows.
%%
\vspace{0.1in}
\VerbatimInput[frame=single,framesep=0.1in,label={\tt iwebbbd} sample output,fontsize=\small]
{../examples_par/iwebbbd.out}
%%

%-----------------------------------------------------------------------------------

\subsection{A user preconditioner example: \id{iheatpk}}\label{ss:iheatpk}

As an example of using {\ida} with the parallel MPI {\nvecp} module and the Krylov 
linear solver {\idaspgmr} with user-defined preconditioner, we provide the example
\id{iheatpk} which solves the same 2-D heat PDE problem as \id{iheatsk}. 

The preconditioner (\id{PsolveHeat} and \id{PsetupHeat}) uses the diagonal 
elements of the Jacobian only and therefore involves only local calculations. 
The residual is formed on submeshes with an \id{MXSUB} $\times$ \id{MYSUB} mesh 
on each of \id{NPEX} $\times$ \id{NPEY} processor. 
The MPI communication (implemented in \id{rescomm}) is basically the same
as for the \id{iwebbbd} example described above and uses the \id{uext} array
to collect ghost cell data from neighboring processes.

As in \id{iheatsk}, additional constraints are passed to {\ida} through
the \id{N\_Vector} \id{constraints} and the function \id{IDASetConstraints}.
All components of \id{constraints} are set to $1.0$ indicating that non-negativity
constraints are to be imposed on each solutin component.
In addition, for illustration purposes, \id{iheatsk} also excludes the algebraic 
components of the solution (specified through the \id{N\_Vector} \id{id}) from the 
error test by calling \id{IDASetsuppressAlg} with a flag \id{TRUE}.

The source is listed in Appendix \ref{s:iheatpk_c}.
Sample output from \id{iheatpk} follows.
%%
\vspace{0.1in}
\VerbatimInput[frame=single,framesep=0.1in,label={\tt iheatpk} sample output,fontsize=\small]
{../examples_par/iheatpk.out}
%%

