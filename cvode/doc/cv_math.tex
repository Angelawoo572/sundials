%===================================================================================
\section{Mathematical Considerations}\label{s:math}
%===================================================================================

{\cvode} solves initial-value problems (IVPs) for systems of ODEs. 
Such problems can be stated as
\begin{equation}\label{e:ivp}
\begin{split}
&\dot{y} = f(t,\,y) \\
&y(t_0) = y_0 \, ,
\end{split}
\end{equation}
where $y \in {\bf R}^N$, $\dot{y}\,=dy/dt$ and ${\bf R}^N$ is the real $N$-dimensional
vector space. That is, (\ref{e:ivp}) represents a system of $N$ ordinary
differential equations and their initial conditions at some $t_0$. The
dependent variable is $y$ and the independent variable is $t$. The
independent variable need not appear explicitly in the $N$-vector valued
function $f$.

%------------------------
\subsection{IVP Solution}\label{ss:ivp_sol}
%------------------------

The IVP is solved by one of two numerical methods. These are the
backward differentiation formula (BDF) \index{BDF method} and the 
Adams-Moulton formula \index{Adams method}. 
Both are implemented in a variable-stepsize, variable-order form. The BDF
uses a fixed-leading-coefficient form. These formulas can both be
represented by a linear multistep formula 
\begin{equation}\label{e:lmm}
\sum_{i=0}^{K_1}\alpha_{n,i}y_{n-i} + h_n\sum_{i=0}^{K_2}\beta_{n,i} 
\dot{y}_{n-i}=0
\end{equation}
where the $N$-vector $y_n$ is the computed approximation to $y(t_n)$,
the exact solution of (\ref{e:ivp}) at $t_n$. The stepsize is
$h_n=t_n-t_{n-1}$.  The coefficients $\alpha_{n,i}$ and $\beta_{n,i}$
are uniquely determined by the particular integration formula, the
history of the stepsize, and the normalization $\alpha_{n,0}=-1$. The
Adams-Moulton \index{Adams method} formula is recommended for nonstiff ODEs and is
represented by (\ref{e:lmm}) with $K_1=1$ and $K_2=q-1$. The order
of this formula is $q$ and its values range from 1 through 12. For
stiff ODEs, BDF \index{BDF method} should be selected and is represented by 
(\ref{e:lmm}) with $K_1=q$ and $K_2=0$. For BDF, the order $q$ may
take on values from 1 through 5. In the case of either formula, the
integration begins with $q=1$, and after that $q$ varies automatically
and dynamically.

For either BDF or the Adams formula, $\dot{y}_n$ denotes
$f(t_n,\,y_n)$. That is, (\ref{e:lmm}) is an implicit formula, and 
the nonlinear equation 
\begin{equation}\label{e:nonlinear}
\begin{split}
G(y_n) &\equiv  y_n-h_n\beta_{n,0}f(t_n,\,y_n) - a_n=0   \\
a_n &= \sum_{i>0}(\alpha_{n,i}y_{n-i}+h_n\beta_{n,i}\dot{y}_{n-i}) 
\end{split}
\end{equation}
must be solved for $y_{n}$ at each time step. For nonstiff problems,
functional (or fixpoint) iteration is normally used and does not
require the solution of a linear system of equations. For stiff
problems, a Newton iteration is used and for each iteration an
underlying linear system must be solved. This linear system of
equations has the form
\begin{equation}\label{e:Newton}
M[y_{n(m+1)}-y_{n(m)}]=-G(y_{n(m)}) \, ,
\end{equation}
where $y_{n(m)}$ is the $m$th approximation to $y_n$, and $M$
approximates $\partial G/ \partial y$:
\begin{equation} \label{e:N_Matrix}
M \approx I-\gamma J, ~~~~ J = \frac{\partial f}{\partial y}, ~~~~
    \gamma = h_n\beta_{n,0} ~.
\end{equation}
At present, aside from a diagonal Jacobian approximation, the other
options implemented in {\cvode} for solving the linear systems
(\ref{e:Newton}) are:
\begin{itemize}
\item a direct method with dense treatment of the Jacobian;
\item a direct method with band treatment of the Jacobian;
\item an iterative method {\spgmr} (scaled, preconditioned
GMRES) \cite{BrHi89}, which is a Krylov subspace method. In most
cases, performance of {\spgmr} is improved by user-supplied
preconditioners. The user may precondition the system on the left, on
the right, on both the left and right, or use no preconditioner.
\end{itemize}
In most cases of interest to the {\cvode} user, the technique of
integration will involve BDF and the Newton method coupled with one of the 
linear solver modules.

\index{error control|(}
The integrator computes an estimate $E_{n}$ of the local error at each time
step, and strives to satisfy the following inequality
\begin{equation*}%\label{e:Err}
\left\| E_n\right\|_{rms,w} < 1 ~.
\end{equation*}
Here the weighted root-mean-square norm is defined by
\begin{equation}\label{e:rms}
\left\| E_n\right\|_{rms,w}=\left[ \sum_{i=1}^N\frac{1}{N}\left(
w_iE_{n,i}\right) ^2\right] ^{1/2} \, ,
\end{equation}
where $E_{n,i}$ denotes the $i$th component of $E_n$, and the $i$th 
component of the weight vector is 
\begin{equation}\label{e:weight}
w_i=\frac{1}{rtol|y_i|+atol_i} \,.
\end{equation}
This permits an arbitrary combination of relative and absolute error control.
The user-specified relative error tolerance is the scalar $rtol$ and the
user-specified absolute error tolerance is $atol$ which may be an $N$-vector
(as indicated above) or a scalar. The value for $rtol$
indicates the number of digits of relative accuracy for a single time step.
The specified value for $atol_{i}\;$indicates the values of the
corresponding component of the solution vector which may be thought of as
being zero, or at the noise level. In particular, if we set 
$atol_i=rtol\times floor_i$ then $floor_i$ represents the floor value for the 
$i$th component of the solution and is that magnitude of the component for
which there is a crossover from relative error control to absolute error
control. Since these tolerances define the allowed error per step, they
should be chosen conservatively. Experience indicates that a conservative
choice yields a more economical solution than error tolerances that are too
large.
\index{error control|)}

The error control mechanism in {\cvode} varies the stepsize and order
in an attempt to take minimum number of steps while satisfying the local
error test. The order control can be (optionally) modified with an algorithm
that attempts to detect limitations resulting from BDF stability properties.

If the system of ODEs (\ref{e:ivp}) contains equations that are {\em pure quadratures}
(i.e. the corresponding variables do not appear in the right hand side),
{\cvode} implements an efficient staggered algorithm for their treatment, by not
%-----------------------------------------------

\subsection{BDF Stability Limit Detection}

{\cvode} includes an algorithm, {\stald} (STAbility Limit Detection),
which provides protection against potentially unstable behavior of the 
BDF multistep integration methods is certain situations, as described below.

When the BDF option is selected, {\cvode} uses Backward Differentiation 
Formula methods of orders 1 to 5.  At order 1 or 2, the BDF
method is A-stable, meaning that for any complex constant $\lambda$ in
the open left half-plane, the method is unconditionally stable (for
any step size) for the standard scalar model problem $dy/dt = \lambda y$.
For an ODE system, this means that, roughly speaking, as long as all
modes in the system are stable, the method is also stable for any
choice of step size, at least in the sense of a local linear stability
analysis.

At orders 3 to 5, the BDF methods are not A-stable, although they are
{\em stiffly stable}. In each case, in order for the method to be stable
at step size $h$ on the scalar model problem, the product $h\lambda$ must
lie in a {\em region of absolute stability}. 
That region excludes a portion of the left half-plane that is concentrated 
near the imaginary axis.  The size of that region of instability grows as the order
increases from 3 to 5.  What this means is that, when running BDF at
any of these orders, if an eigenvalue $\lambda$ of the system lies close
enough to the imaginary axis, the step sizes $h$ for which the method is
stable are limited (at least according to the linear stability theory)
to a set that prevents $h\lambda$ from leaving the stability region.
The meaning of {\em close enough} depends on the order.  
At order 3, the unstable region is much narrower than at order 5, 
so the potential for unstable behavior grows with order.

System eigenvalues that are likely to run into this instability are
ones that correspond to weakly damped oscillations.  A pure undamped
oscillation corresponds to an eigenvalue on the imaginary axis.
Problems with modes of that kind call for different considerations,
since the oscillation generally must be followed by the solver, and
this requires step sizes ($h \sim 1/\nu$, where $\nu$ is the frequency) 
that are stable for BDF anyway.  But for a weakly damped oscillatory mode,
the oscillation in the solution is eventually damped to the noise level, 
and at that time it is important that the solver not be restricted to step 
sizes on the order of $1/\nu$.  It is in this situation that the new option may
be of great value.

In terms of partial differential equations, the typical problems for
which the stability limit detection option is appropriate are
semi-discrete ODE systems (i.e. discretized in space) from PDEs with
advection and diffusion, but with advection dominating over diffusion.
Diffusion alone produces pure decay modes, while advection tends to
produce undamped oscillatory modes.  A mix of the two with advection
dominant will have weakly damped oscillatory modes.

The {\stald} algorithm attempts to detect, in a direct
manner, the presence of a stability region boundary that is limiting
the step sizes in the presence of a weakly damped oscillation \cite{Hi92}.
The algorithm supplements (but differs greatly from) the existing
algorithms in {\cvode} for choosing step size and order based on
estimated local truncation errors.  The {\stald} algorithm works directly
with history data that is readily available in {\cvode}.  If it concludes
that the step size is in fact stability-limited, it dictates a
reduction in the method order, regardless of the outcome of the
error-based algorithm.  The {\stald} algorithm has been tested in
combination with the {\vode} solver on linear advection-dominated
advection-diffusion problems \cite{Hi95}, where it works well.  The
implementation in {\cvode} has been successfully tested on linear 
and nonlinear advection-diffusion problems, among others.

This stability limit detection option adds some overhead computational
cost to the {\cvode} solution.  (In timing tests, these overhead costs
have ranged from 2\% to 7\% of the total, depending on the size and
complexity of the problem, with lower relative costs for larger
problems.)  Therefore, it should be activated only when there is
reasonable expectation of modes in the user's system for which it is
appropriate.  In particular, if a {\cvode} solution with this option
turned off appears to take an inordinately large number of steps at
orders 3-5 for no apparent reason in terms of the solution time scale,
then there is a good chance that step sizes are being limited by
stability, and that turning on the option will improve the efficiency
of the solution. 

