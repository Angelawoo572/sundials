%%==============================================================================
\chapter{Using CVODE}\label{s:simulation}
%%==============================================================================

This chapter is concerned with the use of {\cvode} for the integration
of IVPs.  The following sections treat the header files, the layout of
the user's main program, description of the {\cvode} user-callable
functions, and user-supplied functions.  The final section describes
the Fortran/C interface module, which supports users with applications
written in Fortran77.  The listings of the sample programs in the
companion document \cite{cvode2.2.0_ex} may also be helpful.  Those
codes are intended to serve as templates and are included in the
{\cvode} package.

The user should be aware that not all linear solver modules are compatible 
with all {\nvector} implementations. 
\index{CVODE@{\cvode} linear solvers!NVECTOR@{\nvector} compatibility}
For example, {\nvecp} is not compatible with the direct dense or direct band 
linear solvers since these linear solver modules need to form the system
Jacobian.  The following {\cvode} modules can only be used with {\nvecs}:
{\cvdense}, {\cvband}, and {\cvbandpre}. The preconditioner module {\cvbbdpre}
can only be used with {\nvecp}. 

%%==============================================================================
\section{Access to library and header files}\label{ss:file_access}
%%==============================================================================

At this point, it is assumed that the installation of {\cvode},
following the procedure described in Chapter \ref{s:install}, has
been completed successfully.

Regardless of where the user's application program resides, its
associated compilation and load commands must make reference to the
appropriate locations for the library and header files required by
{\cvode}.  In terms of the directory {\em build\_tree} defined in
Chapter \ref{s:install}, the relevant library files are
\begin{itemize}
\item {\em build\_tree}\id{/lib/libsundials\_cvode.}{\em lib},
\item {\em build\_tree}\id{/lib/libsundials\_fcvode.a},
\item {\em build\_tree}\id{/lib/libsundials\_shared.}{\em lib},
\item {\em build\_tree}\id{/lib/libsundials\_nvec*.}{\em lib} (up to two files), and
\item {\em build\_tree}\id{/lib/libsundials\_fnvec*.a} (up to two files),
\end{itemize}
where the file extension .{\em lib} is typically \id{.so} for shared libraries
and \id{.a} for static libraries. All relevant header files are all located under
the subdirectory
\begin{itemize}
\item {\em build\_tree}\id{/include}
\end{itemize}

For an application that contains both a {\cvode} problem (IVP) and a
{\cvodes} problem (IVP with sensitivity analysis), \index{CVODES@{\cvodes}}
references to the library files must be made carefully, because both
of the associated solver library files contain a user-callable
function called \Id{CVode}, although the version in {\cvodes} is fully
compatible with that in {\cvode}.  In this case, the loader command
must reference {\em build\_tree}\id{/lib/libsundials\_cvodes.}{\em lib}, and not 
{\em build\_tree}\id{/lib/libsundials\_cvode.}{\em lib}.

%%==============================================================================
\section{Data types}\label{s:types}
%%==============================================================================
% This is a shared SUNDIALS TEX file with description of
% types used in llntyps.h
%
\index{portability}
The \ID{sundials\_types.h} file contains the definition of the type \ID{realtype},
which is used by the {\sundials} solvers for all floating-point data, the definition 
of the integer type \ID{sunindextype}, which is used for vector and matrix indices,
and \ID{booleantype}, which is used for certain logic operations within {\sundials}.


\subsection{Floating point types}

The type \id{realtype} can be \id{float}, \id{double}, or \id{long double}, with
the default being \id{double}.
The user can change the precision of the {\sundials} solvers arithmetic at the
configuration stage (see \S\ref{ss:configuration_options_nix}).

Additionally, based on the current precision, \id{sundials\_types.h} defines 
\Id{BIG\_REAL} to be the largest value representable as a \id{realtype},
\Id{SMALL\_REAL} to be the smallest value representable as a \id{realtype}, and
\Id{UNIT\_ROUNDOFF} to be the difference between $1.0$ and the minimum \id{realtype}
greater than $1.0$.

Within {\sundials}, real constants are set by way of a macro called
\Id{RCONST}.  It is this macro that needs the ability to branch on the
definition \id{realtype}.  In ANSI {\CC}, a floating-point constant with no
suffix is stored as a \id{double}.  Placing the suffix ``F'' at the
end of a floating point constant makes it a \id{float}, whereas using the suffix
``L'' makes it a \id{long double}.  For example,
\begin{verbatim}
#define A 1.0
#define B 1.0F
#define C 1.0L
\end{verbatim}
defines \id{A} to be a \id{double} constant equal to $1.0$, \id{B} to be a
\id{float} constant equal to $1.0$, and \id{C} to be a \id{long double} constant
equal to $1.0$.  The macro call \id{RCONST(1.0)} automatically expands to \id{1.0}
if \id{realtype} is \id{double}, to \id{1.0F} if \id{realtype} is \id{float},
or to \id{1.0L} if \id{realtype} is \id{long double}.  {\sundials} uses the
\id{RCONST} macro internally to declare all of its floating-point constants. 

A user program which uses the type \id{realtype} and the \id{RCONST} macro
to handle floating-point constants is precision-independent except for
any calls to precision-specific standard math library
functions.  (Our example programs use both \id{realtype} and
\id{RCONST}.)  Users can, however, use the type \id{double}, \id{float}, or
\id{long double} in their code (assuming that this usage is consistent
with the typedef for \id{realtype}).  Thus, a previously existing
piece of ANSI {\CC} code can use {\sundials} without modifying the code
to use \id{realtype}, so long as the {\sundials} libraries use the
correct precision (for details see \S\ref{ss:configuration_options_nix}).


\subsection{Integer types used for vector and matrix indices}

The type \id{sunindextype} can be either a 64- or 32-bit \emph{signed} integer.
The default is the portable \id{int64\_t} type, and the user can change it
to \id{int32\_t} at the configuration stage. The configuration system
will detect if the compiler does not support portable types, and will
replace \id{int64\_t} and \id{int32\_t} with \id{long long} and \id{int},
respectively, to ensure use of the desired sizes on Linux, Mac OS X and Windows
platforms. {\sundials} currently does not support \emph{unsigned} integer types 
for vector and matrix indices, although these could be added in the future if there 
is sufficient demand.


%%==============================================================================
\section{Header files}\label{ss:header_sim}
%%==============================================================================
\index{header files}
The calling program must include several header files so that various macros
and data types can be used. The header file that is always required is:
%%
\begin{itemize}
\item  \Id{cvode.h}, 
  the header file for {\cvode}, which defines the several
  types and various constants, and includes function prototypes.
\end{itemize}
%%
Note that \id{cvode.h} includes \Id{sundialstypes.h}, 
which defines the types \id{realtype} and \id{booleantype}
and the constants \id{FALSE} and \id{TRUE}.

The calling program must also include an {\nvector} implementation header file
(see Chapter \ref{s:nvector} for details).
For the two {\nvector} implementations that are included in the {\cvode} package,
the corresponding header files are:
%%
\begin{itemize}
\item \Id{nvector\_serial.h}, 
  which defines the serial implementation {\nvecs};
\item \Id{nvector\_parallel.h}, 
  which defines the parallel MPI implementation, {\nvecp}.
\end{itemize}
%%
Note that both these files include in turn the header file \Id{nvector.h} which 
defines the abstract \Id{N\_Vector} type. 

Finally, if the user chooses Newton iteration for the solution of the nonlinear
systems, then a linear solver module header file will be required. 
\index{CVODE@{\cvode} linear solvers!header files}
The header files corresponding to the various linear solver options in
{\cvode} are:
%%
\begin{itemize}
\item \Id{cvdense.h}, 
  which is used with the dense direct linear solver in 
  the context of {\cvode}. This in turn includes a header file (\id{dense.h})
  which defines the \Id{DenseMat} type and corresponding accessor macros; 
\item \Id{cvband.h}, 
  which is used with the band direct linear solver in the
  context of {\cvode}. This in turn includes a header file (\id{band.h})
  which defines the \Id{BandMat} type and corresponding accessor macros;
\item \Id{cvdiag.h}, which is used with a diagonal linear solver in the
  context of {\cvode};
\item \Id{cvspgmr.h}, 
  which is used with the Krylov solver {\spgmr} in the
  context of {\cvode}. This in turn includes a header file (\id{iterative.h})
  which enumerates the kind of preconditioning and the choices for the
  Gram-Schmidt process.
\end{itemize}

Other headers may be needed, according as to the choice of preconditioner,
etc. In one of the examples in \cite{cvode2.2.0_ex}, preconditioning is done
with a block-diagonal matrix. For this, the header
\id{smalldense.h} is included.

%%==============================================================================
\section{A skeleton of the user's main program}\label{ss:skeleton_sim}
%%==============================================================================

A high-level view of the combined user program and {\cvode} package is
shown in Figure~\ref{f:sim_overview}.
%%
\begin{figure}
\centerline{\psfig{figure=cvsim.eps,width=\textwidth}}
\caption {Diagram of the user program and 
  {\cvode} package for integration of IVP}\label{f:sim_overview}
\end{figure}
%%
The following is a skeleton of the user's main program (or calling
program) for the integration of an ODE IVP. Some steps are independent
of the {\nvector} implementation used; where this is not the case, usage
specifications are given for the two implementations provided with {\cvode}:
Steps marked with {\p} correspond to  {\nvecp}, while steps marked with
{\s} correspond to {\nvecs}.
%%
%%
%%
\index{User main program!CVODE@{\cvode} usage}
\begin{Steps}
  
\item 
  {\bf {\p} Initialize MPI}

  Call \id{MPI\_Init(\&argc, \&argv);} to initialize MPI if used by
  the user's program, aside from the internal use in {\nvecp}.  
  Here \id{argc} and \id{argv} are the command line argument 
  counter and array received by \id{main}.
  
\item
  {\bf Set problem dimensions}

  {\s} Set \id{N}, the problem size $N$.

  {\p} Set \id{Nlocal}, the local vector length (the sub-vector
  length for this processor); \id{N}, the global vector length (the
  problem size $N$, and the sum of all the values of \id{Nlocal});
  and the active set of processors.
  
\item
  {\bf Set vector of initial values}
 
  To set the vector \id{y0} of initial values, use functions defined by a
  particular {\nvector} implementation.  If a \id{realtype} array  \id{ydata}
  already exists, containing the initial values of $y$, make the call:

  {\s} \id{y0 = NV\_Make\_Serial(N, ydata);}

  {\p} \id{y0 = NV\_Make\_Parallel(comm, Nlocal, N, ydata);}

  Otherwise, make the call:

  {\s} \id{y0 = NV\_New\_Serial(N);}

  {\p} \id{y0 = NV\_New\_Parallel(comm, Nlocal, N);}

  and load initial values into the structure defined by:

  {\s} \id{NV\_DATA\_S(y0)}

  {\p} \id{NV\_DATA\_P(y0)}

  Here \id{comm} is the MPI communicator, set in one of two ways: 
  If a proper subset of active processors is to be used, \id{comm} 
  must be set by suitable MPI calls. Otherwise, to specify that all 
  processors are to be used, \id{comm} must be \id{MPI\_COMM\_WORLD}.
  
\item\label{i:cvode_create} 
  {\bf Create {\cvode} object}

  Call \id{cvode\_mem = }\id{CVodeCreate}\id{(lmm,iter);} 
  to create the {\cvode} memory block and specify the solution method
  (linear multistep method and nonlinear solver iteration type).
  \id{CVodeCreate} returns a pointer to the {\cvode} memory structure.
  See \S\ref{sss:cvodemalloc} for details.

\item
  {\bf Set optional inputs}

  Call \id{CVodeSet*} functions to change from their default values any
  optional inputs that control the behavior of {\cvode}.
  See \S\ref{ss:optional_input} for details.

\item\label{i:cvode_malloc} 
  {\bf Allocate internal memory}

  Call \id{CVodeMalloc}\id{(...);} 
  to provide required problem specifications,
  allocate internal memory for {\cvode}, 
  and initialize {\cvode}.
  \id{CVodeMalloc} returns an error flag to indicate success or an illegal
  argument value.  See \S\ref{sss:cvodemalloc} for details.
  
\item\label{i:lin_solver} 
  {\bf Attach linear solver module}

  If Newton iteration is chosen, initialize the linear solver module
  with one of the following calls (for details see \S\ref{sss:lin_solv_init}):

  {\s} \id{ier = }\Id{CVDense}\id{(...);}

  {\s} \id{ier = }\Id{CVBand}\id{(...);}

  \id{ier = }\Id{CVDiag}\id{(...);}

  \id{ier = }\Id{CVSpgmr}\id{(...);}
  
\item
  {\bf Set linear solver optional inputs}

  Call \id{CV*Set*} functions from the selected linear solver module to
  change optional inputs specific to that linear solver.
  See \S\ref{ss:optional_input} for details.

\item
  {\bf Specify rootfinding problem}
  \index{Rootfinding}

  Optionally, call \id{CVodeRootInit} to initialize a rootfinding problem
  to be solved during the integration of the ODE system.
  See \S\ref{ss:root_uc} for details.

\item
  {\bf Advance solution in time}

  For each point at which output is desired, call
  \id{ier = }\Id{CVode}\id{(cvode\_mem, tout, yout, \&tret, itask);}
  Set \Id{itask} to specify the return mode.
  The vector \id{y} (which can be the same as
  the vector \id{y0} above) will contain $y(t)$.
  See \S\ref{sss:cvode} for details.
  
\item
  {\bf Get optional outputs}

  Call \id{CV*Get*} functions to obtain optional output.
  See \S\ref{ss:optional_output} and \S\ref{ss:root_uc} for details.

\item
  {\bf Deallocate memory for solution vector}

  Upon completion of the integration, deallocate memory for the vector \id{y}
  by calling the destructor function defined by the {\nvector} implementation:

  {\s} \id{NV\_Destroy\_Serial(y);}

  {\p} \id{NV\_Destroy\_Parallel(y);}
  
\item
  {\bf Free solver memory}

  \Id{CVodeFree}\id{(cvode\_mem);} to free the memory allocated for {\cvode}.
  
\item 
  {\bf {\p} Finalize MPI}

  Call \id{MPI\_Finalize();} to terminate MPI.
  
\end{Steps}

%%==============================================================================
\section{User-callable functions}\label{ss:cvode_fct_sim}
%%==============================================================================

This section describes the {\cvode} functions that are called by the
user to set up and solve an IVP. Some of these are required. However,
starting with \S\ref{ss:optional_input}, the functions listed involve
optional inputs/outputs or restarting, and those paragraphs can be
skipped for a casual use of {\cvode}. In any case, refer to
\S\ref{ss:skeleton_sim} for the correct order of these calls.
Calls related to rootfinding are described in \S\ref{s:using_rootfinding}.

%%==============================================================================
\subsection{CVODE initialization and deallocation functions}
\label{sss:cvodemalloc}
%%==============================================================================

The following three functions must be called in the order listed. The last one
is to be called only after the IVP solution is complete, as it frees the
{\cvode} memory block created and allocated by the first two calls.
%%
\ucfunction{CVodeCreate}
{
  cvode\_mem = CVodeCreate(lmm, iter);
}
{
  The function \ID{CVodeCreate} instantiates a {\cvode} solver object and
  specifies the solution method.
}
{
  \begin{args}[iter]
  \item[lmm] (\id{int})
    specifies the linear multistep method and must be one of two
    possible values: \ID{CV\_ADAMS} or \ID{CV\_BDF}.     
  \item[iter] (\id{int})
    specifies the type of nonlinear solver iteration and may be
    either \ID{CV\_NEWTON} or \ID{CV\_FUNCTIONAL}. 
  \end{args}
  The recommended choices for (\ID{lmm}, \ID{iter}) are
  (\id{CV\_ADAMS}, \id{CV\_FUNCTIONAL}) for nonstiff problems and
  (\id{CV\_BDF}, \id{CV\_NEWTON}) for stiff problems.
}
{
  If successful, \id{CVodeCreate} returns a pointer to the newly created 
  {\cvode} memory block (of type \id{void *}).
  If an error occurred, \id{CVodeCreate} prints an error message to \id{stderr}
  and returns \id{NULL}.
}
{}
%%
%%
\ucfunction{CVodeMalloc}
{
flag = CVodeMalloc(cvode\_mem, f, t0, y0, itol, reltol, abstol);
}
{
  The function \ID{CVodeMalloc} provides required problem and solution
  specifications, allocates internal memory, and initializes {\cvode}.
}
{
  \begin{args}[cvode\_mem]
  \item[cvode\_mem] (\id{void *})
    pointer to the {\cvode} memory block returned by \id{CVodeCreate}.
  \item[f] (\Id{CVRhsFn})
    is the {\C} function which computes $f$ in the ODE. This function has the 
    form \id{f(t, y, ydot, f\_data)} (for full details see \S\ref{ss:rhsFn}).
  \item[t0] (\id{realtype})
    is the initial value of $t$.
  \item[y0] (\id{N\_Vector})
    is the initial value of $y$. 
  \item[itol] (\id{int}) 
    is either \ID{CV\_SS} or \ID{CV\_SV}, where \ID{itol}$=$\id{SS} indicates scalar
    relative error tolerance and scalar absolute error tolerance, while
    \id{itol}$=$\id{CV\_SV} indicates scalar relative error tolerance and vector
    absolute error tolerance.  The latter choice is important when the absolute
    error tolerance needs to be different for each component of the ODE. 
  \item[reltol] (\id{realtype *})
    \index{tolerances}
    is a pointer to the relative error tolerance.
  \item[abstol] (\id{void *})
    is a pointer to the absolute error tolerance.
  \end{args}
}
{
  The return flag \id{flag} (of type \id{int}) will be one of the following:
  \begin{args}[CV\_ILL\_INPUT]
  \item[\Id{CV\_SUCCESS}]
    The call to \id{CVodeMalloc} was successful.
  \item[\Id{CV\_MEM\_NULL}] 
    The {\cvode} memory block was not initialized through a previous call
    to \id{CVodeCreate}.
  \item[\Id{CV\_MEM\_FAIL}] 
    A memory allocation request has failed.
  \item[\Id{CV\_ILL\_INPUT}] 
    An input argument to \id{CVodeMalloc} has an illegal value.
  \end{args}
}
{
  If an error occurred, \id{CVodeMalloc} also prints an error message to the
  file specified by the optional input \id{errfp}.

  The tolerance values in \id{reltol} and \id{abstol} may be changed between
  calls to \id{CVode} (see \id{CVodesetTolerances} in \S\ref{ss:optional_input}).
}
%%
%%
\ucfunction{CVodeFree}
{
  CVodeFree(cvode\_mem);
}
{
  The function \ID{CVodeFree} frees the pointer allocated by
  a previous call to \id{CVodeMalloc}.
}
{
  The argument is the pointer to the {\cvode} memory block (of type \id{void *}).
}
{
  The function \id{CVodeFree} has no return value.
}
{}
%%
%%==============================================================================
\subsection{Linear solver specification functions}\label{sss:lin_solv_init}
%%==============================================================================

As previously explained, Newton iteration requires the solution of
linear systems of the form (\ref{e:Newton}).  There are four {\cvode} linear
solvers currently available for this task: {\cvdense}, {\cvband}, {\cvdiag},
and {\cvspgmr}.  The first three are direct solvers and derive their name
from the type of approximation used for the Jacobian 
$J = \partial{f}/\partial{y}$.  {\cvdense}, {\cvband}, and {\cvdiag} work with
dense, banded, and diagonal approximations to $J$, respectively.  The
fourth {\cvode} linear solver, {\cvspgmr}, is an iterative solver.  The {\spgmr}
in the name indicates that it uses a scaled preconditioned
GMRES method.

\index{CVODE@{\cvode} linear solvers!selecting one|(} 
To specify a {\cvode} linear solver, after the call to \id{CVodeCreate}
but before any calls to \id{CVode}, the user's program must call one
of the functions \Id{CVDense}, \Id{CVBand}, \Id{CVDiag}, \Id{CVSpgmr},
as documented below. The first argument passed to these functions is the {\cvode}
memory pointer returned by \id{CVodeCreate}.  A call to one of these
functions links the main {\cvode} integrator to a linear solver and
allows the user to specify parameters which are specific to a
particular solver, such as the half-bandwidths in the {\cvband} case.
%%
The use of each of the linear solvers involves certain constants and possibly 
some macros, that are likely to be needed in the user code.  These are
available in the corresponding header file associated with the linear
solver, as specified below.
\index{CVODE@{\cvode} linear solvers!selecting one|)}

\index{CVODE@{\cvode} linear solvers!built on generic solvers|(} 
In each case except the diagonal approximation case {\cvdiag}, the linear
solver module used by {\cvode} is actually built on top of a generic
linear system solver, which may be of interest in itself.  These
generic solvers, denoted {\dense}, {\band}, and {\spgmr}, are described
separately in Chapter \ref{s:gen_linsolv}.
\index{CVODE@{\cvode} linear solvers!built on generic solvers|)}
%%
%%
%%
\index{CVODE@{\cvode} linear solvers!CVDENSE@{\cvdense}}
\index{CVDENSE@{\cvdense} linear solver!selection of}
\index{CVDENSE@{\cvdense} linear solver!NVECTOR@{\nvector} compatibility}
\ucfunction{CVDense}
{
  flag = CVDense(cvode\_mem, N);
}
{
  The function \ID{CVDense} selects the {\cvdense} linear solver. 

  The user's main function must include the \id{cvdense.h} header file.
}
{
  \begin{args}[cvode\_mem]
  \item[cvode\_mem] (\id{void *})
    pointer to the {\cvode} memory block.
  \item[N] (\id{long int})
    problem dimension.
  \end{args}
}
{
  The return value \id{flag} (of type \id{int}) is one of
  \begin{args}[CVDENSE\_ILL\_INPUT]
  \item[\Id{CVDENSE\_SUCCESS}] 
    The {\cvdense} initialization was successful.
  \item[\Id{CVDENSE\_MEM\_NULL}]
    The \id{cvode\_mem} pointer is \id{NULL}.
  \item[\Id{CVDENSE\_ILL\_INPUT}]
    The {\cvdense} solver is not compatible with the current {\nvector} module.
  \item[\Id{CVDENSE\_MEM\_FAIL}]
    A memory allocation request failed.
  \end{args}
}
{
  The {\cvdense} linear solver may not be compatible with a particular
  implementation of the {\nvector} module. 
  Of the two {\nvector} modules provided by {\sundials}, only {\nvecs} is 
  compatible, while {\nvecp} is not.
}
%%
%%
%%
\index{CVODE@{\cvode} linear solvers!CVBAND@{\cvband}}
\index{CVBAND@{\cvband} linear solver!selection of}
\index{CVBAND@{\cvband} linear solver!NVECTOR@{\nvector} compatibility}
\index{half-bandwidths}
\ucfunction{CVBand}
{
  flag = CVBand(cvode\_mem, N, mupper, mlower);
}
{
  The function \ID{CVBand} selects the {\cvband} linear solver. 

  The user's main function must include the \id{cvband.h} header file.
}
{
  \begin{args}[cvode\_mem]
  \item[cvode\_mem] (\id{void *})
    pointer to the {\cvode} memory block.
  \item[N] (\id{long int})
    problem dimension.
  \item[mupper] (\id{long int})
    upper half-bandwidth of the problem Jacobian (or of the approximation of it).
  \item[mlower] (\id{long int})
    lower half-bandwidth of the problem Jacobian (or of the approximation of it).
  \end{args}
}
{
  The return value \id{flag} (of type \id{int}) is one of
  \begin{args}[CVBAND\_ILL\_INPUT]
  \item[\Id{CVBAND\_SUCCESS}] 
    The {\cvband} initialization was successful.
  \item[\Id{CVBAND\_MEM\_NULL}]
    The \id{cvode\_mem} pointer is \id{NULL}.
  \item[\Id{CVBAND\_ILL\_INPUT}]
    The {\cvband} solver is not compatible with the current {\nvector} module, or
    one of the Jacobian half-bandwidths is outside its valid range
    ($0 \ldots$ \id{N}$-1$).
  \item[\Id{CVBAND\_MEM\_FAIL}]
    A memory allocation request failed.
  \end{args}
}
{
  The {\cvband} linear solver may not be compatible with a particular
  implementation of the {\nvector} module. Of the two {\nvector} modules 
  provided by {\sundials}, only {\nvecs} is compatible, while {\nvecp} is not.
  The half-bandwidths are to be set so that the nonzero locations $(i,j)$ in the
  banded (approximate) Jacobian satisfy $-$\id{mlower} $\leq j-i \leq$ \id{mupper}.
}
%%
%%
%%
\index{CVODE@{\cvode} linear solvers!CVDIAG@{\cvdiag}}
\index{CVDIAG@{\cvdiag} linear solver!selection of}
\index{CVDIAG@{\cvdiag} linear solver!Jacobian approximation used by}
\index{Jacobian approximation function!diagonal!difference quotient}
\ucfunction{CVDiag}
{
  flag = CVDiag(cvode\_mem);
}
{
  The function \ID{CVDiag} selects the {\cvdiag} linear solver. 

  The user's main function must include the \id{cvdiag.h} header file.
}
{
  \begin{args}[cvode\_mem]
  \item[cvode\_mem] (\id{void *})
    pointer to the {\cvode} memory block.
  \end{args}
}
{
  The return value \id{flag} (of type \id{int}) is one of
  \begin{args}[CVDIAg\_MEM\_FAIL]
  \item[\Id{CVDIAG\_SUCCESS}]
    The {\cvdiag} initialization was successful.
  \item[\Id{CVDIAG\_MEM\_NULL}]
    The \id{cvode\_mem} pointer is \id{NULL}.
  \item[\Id{CVDIAG\_MEM\_FAIL}]
    A memory allocation request failed.
  \end{args}
}
{
  The {\cvdiag} solver is the simplest of all the current {\cvode} linear solvers. 
  The {\cvdiag} solver uses an approximate diagonal Jacobian formed by way of a
  difference quotient. The user does {\em not} have the option to supply a
  function to compute an approximate diagonal Jacobian.
}
%%
%%
%%
\index{CVODE@{\cvode} linear solvers!CVSPGMR@{\cvspgmr}}
\index{CVSPGMR@{\cvspgmr} linear solver!selection of} 
\ucfunction{CVSpgmr}
{
  flag = CVSpgmr(cvode\_mem, pretype, maxl);
}
{
  The function \ID{CVSpgmr} selects the {\cvspgmr} linear solver. 

  The user's main function must include the \id{cvspgmr.h} header file.
}
{
  \begin{args}[cvode\_mem]
  \item[cvode\_mem] (\id{void *})
    pointer to the {\cvode} memory block.
  \item[pretype] (\id{int})
    \index{pretype@\texttt{pretype}|textbf}
    specifies the preconditioning type and must be one of: 
    \Id{PREC\_NONE}, \Id{PREC\_LEFT}, \Id{PREC\_RIGHT}, or \Id{PREC\_BOTH}.
  \item[maxl] (\id{int})
    \index{maxl@\texttt{maxl}}
    maximum dimension of the Krylov subspace to be used. Pass $0$ to use the 
    default value \id{CVSPGMR\_MAXL}$=5$.
  \end{args}
}
{
  The return value \id{flag} (of type \id{int}) is one of
  \begin{args}[CVSPGMR\_ILL\_INPUT]
  \item[\Id{CVSPGMR\_SUCCESS}] 
    The {\cvspgmr} initialization was successful.
  \item[\Id{CVSPGMR\_MEM\_NULL}]
    The \id{cvode\_mem} pointer is \id{NULL}.
  \item[\Id{CVSPGMR\_ILL\_INPUT}]
    The preconditioner type \id{pretype} is not valid.
  \item[\Id{CVSPGMR\_MEM\_FAIL}]
    A memory allocation request failed.
  \end{args}
}
{
  The {\cvspgmr} solver uses a scaled preconditioned GMRES\index{GMRES method}
  iterative method to solve the linear system (\ref{e:Newton}).\\
  \index{preconditioning!advice on|(}With this {\spgmr} method, preconditioning 
  can be done on the left only, on the right only, on both the left and the right, 
  or not at all.  For a given preconditioner matrix, the merits of left vs. right
  preconditioning are unclear in general, and the user should experiment
  with both choices.  Performance will differ because the inverse of the
  left preconditioner is included in the linear system residual whose
  norm is being tested in the {\spgmr} algorithm.  As a rule, however, if
  the preconditioner is the product of two matrices, we recommend that
  preconditioning be done either on the left only or the right only,
  rather than using one factor on each side. For specification of preconditioner,
  see \S\ref{ss:optional_input} and \S\ref{ss:user_fct_sim}.

  If preconditioning is done, user-supplied functions define left and right 
  preconditioner matrices $P_1$ and $P_2$ (either of which could be the identity
  matrix), such that the product $P_1 P_2$ approximates the Newton matrix
  $M=I-\gamma J$ of (\ref{e:Newtonmat}).
  \index{preconditioning!advice on|)}
}

%%==============================================================================
\subsection{CVODE solver function}\label{sss:cvode}
%%==============================================================================

This is the central step in the solution process --- the call to
perform the integration of the IVP.
%
\ucfunction{CVode}
{
  flag = CVode(cvode\_mem, tout, yout, tret, itask);
}
{
  The function \ID{CVode} integrates the ODE over an interval in $t$.
}
{
  \begin{args}[cvode\_mem]
  \item[cvode\_mem] (\id{void *})
    pointer to the {\cvode} memory block.
  \item[tout] (\id{realtype})
    the next time at which a computed solution is desired.
  \item[yout] (\id{N\_Vector})
    the computed solution vector.
  \item[tret] (\id{realtype *})
    the time reached by the solver.
  \item[itask] (\id{int})
    \index{itask@\texttt{itask}|textbf}
    a flag indicating the job of the solver for the next user step. 
    The \Id{CV\_NORMAL} task is to have the solver take internal steps until   
    it has reached or just passed the user specified \id{tout}
    parameter. The solver then interpolates in order to   
    return an approximate value of $y($\id{tout}$)$. 
    The \Id{CV\_ONE\_STEP} option tells the solver to just take one internal step  
    and return the solution at the point reached by that step. 
    The \Id{CV\_NORMAL\_TSTOP} and \Id{CV\_ONE\_STEP\_TSTOP} modes are     
    similar to \id{CV\_NORMAL} and \id{CV\_ONE\_STEP}, respectively, except    
    that the integration never proceeds past the value      
    \id{tstop} (specified through the function \id{CVodeSetStopTime}).
  \end{args}
}
{
  On return, \id{CVode} returns a vector \id{yout} and a corresponding 
  independent variable value $t=$\id{*tret}, such that \id{yout} is the computed 
  value of $y(t)$.

  In \id{CV\_NORMAL} mode with no errors, \id{*tret} will be equal to \id{tout} 
  and \id{yout} = $y($\id{tout}$)$.

  The return value \id{flag} (of type \id{int}) will be one of the following:
  \begin{args}[CV\_TOO\_MUCH\_WORK]
  \item[\Id{CV\_SUCCESS}]
    \id{CVode} succeeded and no root was found.
  \item[\Id{CV\_TSTOP\_RETURN}]
    \id{CVode} succeeded by reaching the stopping point specified through
    the optional input function \id{CVodeSetStopTime} (see \S\ref{ss:optional_input}).
  \item[\Id{CV\_ROOT\_RETURN}]
    \id{CVode} succeeded and found one or more roots.  If \id{nrtfn}
     $>1$, call \id{CVodeGetRootInfo} to see which $g_i$ were found to
     have a root.  See \S\ref{s:using_rootfinding} for more information.
  \item[\Id{CV\_MEM\_NULL}]
    The \id{cvode\_mem} argument was \id{NULL}.
  \item[\Id{CV\_NO\_MALLOC}]
    The {\cvode} memory was not allocated by a call to \id{CVodeMalloc}.
  \item[\Id{CV\_ILL\_INPUT}]
    One of the inputs to \id{CVode} is illegal. This includes the situation where
    a root of one of the root functions was found both at $t_0$ and very near $t_0$.
    It also includes the situation 
    where a component of the error weight vector becomes negative during internal 
    time-stepping. The \id{CV\_ILL\_INPUT} flag will also be returned if the linear 
    solver function initialization (called by the user after calling 
    \id{CVodeCreate}) failed to set the linear solver-specific \id{lsolve} field
    in \id{cvode\_mem}. 
    In any case, the user should see the printed error message for details.
  \item[\Id{CV\_LINIT\_FAIL}] 
    The linear solver's initialization function failed. 
  \item[\Id{CV\_TOO\_MUCH\_WORK}] 
    The solver took \id{mxstep} internal steps but could not reach tout. 
    The default value for \id{mxstep} is \id{MXSTEP\_DEFAULT = 500}.
  \item[\Id{CV\_TOO\_MUCH\_ACC}] 
    The solver could not satisfy the accuracy demanded by the user for some 
    internal step.
  \item[\Id{CV\_ERR\_FAILURE}]
    Error test failures occurred too many times (\id{MXNEF = 7}) during one 
    internal time step or occurred with $|h| = h_{min}$.
  \item[\Id{CV\_CONV\_FAILURE}] 
    Convergence test failures occurred too many times (\id{MXNCF = 10}) during 
    one internal time step or occurred with $|h| = h_{min}$.             
  \item[\Id{CV\_LSETUP\_FAIL}] 
    The linear solver's setup function failed in an unrecoverable manner.
  \item[\Id{CV\_LSOLVE\_FAIL}] 
    The linear solver's solve function failed in an unrecoverable manner.
  \end{args} 
}
{
  The vector \id{yout} can occupy the same space as the \id{y0} vector of 
  initial conditions that was passed to \id{CVodeMalloc}. 

  In the \id{CV\_ONE\_STEP} mode, \id{tout} is used on the first call only, 
  to get the direction and rough scale of the independent variable.

  All failure return values are negative and therefore a test \id{flag}$< 0$
  will trap all \id{CVode} failures.
}

%%==============================================================================
\subsection{Optional input functions}\label{ss:optional_input}
%%==============================================================================

{\cvode} provides an extensive list of functions that can be used to change
from their default values various optional input parameters that control the
behavior of the {\cvode} solver. 
Table \ref{t:optional_input} lists all optional input functions in {\cvode} which 
are then described in detail in the remainder of this section.
For the most casual use of {\cvode}, the reader can skip to \S\ref{ss:user_fct_sim}.

We note that, on error return, all these functions also print an error message
to \id{stderr} (or to the file pointed to by \id{errfp} if already specified).
\index{error message}
We also note that all error return values are negative, so a test \id{flag}$<0$
will catch any error.

\begin{table}
\centering
\caption{Optional inputs for {\cvode}, {\cvdense}, {\cvband}, and {\cvspgmr}}
\label{t:optional_input}
\medskip
\begin{tabular}{|l|l|l|}\hline
{\bf Optional input} & {\bf Function name} & {\bf Default} \\
\hline
\multicolumn{3}{|c|}{\bf CVODE main solver} \\
\hline
Pointer to an error file & \id{CVodeSetErrFile} & \id{stderr}  \\
Data for right-hand side function & \id{CVodeSetFdata} & NULL \\
Maximum order for BDF method & \id{CVodeSetMaxOrd} & 5 \\
Maximum order for Adams method & \id{CVodeSetMaxOrd} & 12  \\
Maximum no. of internal steps before $t_{\mbox{\scriptsize out}}$ & \id{CVodeSetMaxNumSteps} & 500 \\
Maximum no. of warnings for $t_n+h=t_n$ & \id{CVodeSetMaxHnilWarns} & 10 \\
Flag to activate stability limit detection & \id{CVodeSetStabLimDet} & FALSE \\
Initial step size & \id{CVodeSetInitStep} & estimated \\
Minimum absolute step size & \id{CVodeSetMinStep} & 0.0 \\
Maximum absolute step size & \id{CVodeSetMaxStep} & $\infty$ \\
Value of $t_{stop}$ & \id{CVodeSetStopTime} & $\infty$ \\
Maximum no. of error test failures & \id{CVodeSetMaxErrTestFails} & 7 \\
Maximum no. of nonlinear iterations & \id{CVodeSetMaxNonlinIters} & 3 \\
Maximum no. of convergence failures & \id{CVodeSetMaxConvFails} & 10 \\
Coefficient in the nonlinear convergence test & \id{CVodeSetNonlinConvCoef} & 0.1 \\
Data for rootfinding function & \id{CVodeSetGdata} & NULL \\
Nonlinear iteration type & \id{CVodeSetIterType} & none \\
Integration tolerances & \id{CVodeSetTolerances} & none \\
\hline
\multicolumn{3}{|c|}{\bf CVDENSE linear solver} \\
\hline
Dense Jacobian function & \id{CVDenseSetJacFn} & internal DQ \\
Data for Jacobian function & \id{CVDenseSetJacData} & NULL \\
\hline
\multicolumn{3}{|c|}{\bf CVBAND linear solver} \\
\hline
Band Jacobian function & \id{CVBandSetJacFn} & internal DQ \\
Data for Jacobian function & \id{CVBandSetJacData} & NULL \\
\hline
\multicolumn{3}{|c|}{\bf CVSPGMR linear solver} \\
\hline
Preconditioner solve function & \id{CVSpgmrSetPrecSolveFn} & NULL \\
Preconditioner setup function & \id{CVSpgmrSetPrecSetupFn} & NULL \\
Data for preconditioner functions & \id{CVSpgmrSetPrecData} & NULL \\
Jacobian times vector function & \id{CVSpgmrSetJacTimesVecFn} & internal DQ \\
Data for Jacobian times vector function &\id{CVSpgmrSetJacData} & NULL \\ 
Type of Gram-Schmidt orthogonalization & \id{CVSpgmrSetGSType} & classical GS \\
Ratio between linear and nonlinear tolerances & \id{CVSpgmrSetDelt} & 0.05 \\
Preconditioning type & \id{CVSpgmrSetPrecType} & none \\
\hline
\end{tabular}
\end{table}

%%==============================================================================
\subsubsection{Main solver optional input functions}
%%==============================================================================
\index{optional input!solver|(}

The calls listed here can be executed in any order. However, if
\id{CVodeSetErrFile} is to be called, that call should be first, in order to
take effect for any later error message.

\index{error message}
\ucfunction{CVodeSetErrFile}
{
flag = CVodeSetErrFile(cvode\_mem, errfp);
}
{
  The function \ID{CVodeSetErrFile} specifies the pointer to the file
  where all {\cvode} messages should be directed.
}
{
  \begin{args}[cvode\_mem]
  \item[cvode\_mem] (\id{void *})
    pointer to the {\cvode} memory block.
  \item[errfp] (\id{FILE *})
    pointer to output file.
  \end{args}
}
{
  The return value \id{flag} (of type \id{int}) is one of
  \begin{args}[CV\_MEM\_NULL]
  \item[\Id{CV\_SUCCESS}] 
    The optional value has been successfully set.
  \item[\Id{CV\_MEM\_NULL}]
    The \id{cvode\_mem} pointer is \id{NULL}.
  \end{args}
}
{
  The default value for \id{errfp} is \id{stderr}. 

  Passing a value of \id{NULL} disables all future error message output
  (except for the case in which the {\cvode} memory pointer is \id{NULL}).
}
%%
%%
\ucfunction{CVodeSetFdata}
{
  flag = CVodeSetFdata(cvode\_mem, f\_data);
}
{
  The function \ID{CVodeSetFdata} specifies the user data block \ID{f\_data},
  for use by the user right-hand side function $f$, and attaches it to the main 
  {\cvode} memory block.
}
{
  \begin{args}[cvode\_mem]
  \item[cvode\_mem] (\id{void *})
    pointer to the {\cvode} memory block.
  \item[f\_data] (\id{void *})
    pointer to the user data.
  \end{args}
}
{
  The return value \id{flag} (of type \id{int}) is one of
  \begin{args}[CV\_MEM\_NULL]
  \item[\Id{CV\_SUCCESS}] 
    The optional value has been successfully set.
  \item[\Id{CV\_MEM\_NULL}]
    The \id{cvode\_mem} pointer is \id{NULL}.
  \end{args}
}
{
  If \id{f\_data} is not specified, a \id{NULL} pointer is
  passed to the $f$ function.
}
%%
%%
\ucfunction{CVodeSetMaxOrd}
{
flag = CVodeSetMaxOrder(cvode\_mem, maxord);
}
{
  The function \ID{CVodeSetMaxOrder} specifies the maximum order of the 
  linear multistep method.
}
{
  \begin{args}[cvode\_mem]
  \item[cvode\_mem] (\id{void *})
    pointer to the {\cvode} memory block.
  \item[maxord] (\id{int})
    value of the maximum method order.
  \end{args}
}
{
  The return value \id{flag} (of type \id{int}) is one of
  \begin{args}[CV\_ILL\_INPUT]
  \item[\Id{CV\_SUCCESS}] 
    The optional value has been successfully set.
  \item[\Id{CV\_MEM\_NULL}]
    The \id{cvode\_mem} pointer is \id{NULL}.
  \item[\Id{CV\_ILL\_INPUT}]
    The specified value \id{maxord} is negative, or larger than 
    its previous value.
  \end{args}
}
{
  The default value is \Id{ADAMS\_Q\_MAX}$= 12$ for
  the Adams-Moulton method and \Id{BDF\_Q\_MAX}$= 5$
  for the BDF method.
  Since \ID{maxord} affects the memory requirements
  for the internal {\cvode} memory block, its value
  can not be increased past its previous value.
}
%%
%%
\ucfunction{CVodeSetMaxNumSteps}
{
flag = CVodeSetMaxNumSteps(cvode\_mem, mxsteps);
}
{
  The function \ID{CVodeSetMaxNumSteps} specifies the maximum number
  of steps to be taken by the solver in its attempt to reach 
  the final time.
}
{
  \begin{args}[cvode\_mem]
  \item[cvode\_mem] (\id{void *})
    pointer to the {\cvode} memory block.
  \item[mxsteps] (\id{long int})
    maximum allowed number of steps.
  \end{args}
}
{
  The return value \id{flag} (of type \id{int}) is one of
  \begin{args}[CV\_ILL\_INPUT]
  \item[\Id{CV\_SUCCESS}] 
    The optional value has been successfully set.
  \item[\Id{CV\_MEM\_NULL}]
    The \id{cvode\_mem} pointer is \id{NULL}.
  \item[\Id{CV\_ILL\_INPUT}]
    \id{mxsteps} is non-positive.
  \end{args}
}
{
  The default value is $500$.
}
%%
%%
\ucfunction{CVodeSetMaxHnilWarns}
{
flag = CVodeSetMaxHnilWarns(cvode\_mem, mxhnil);
}
{
  The function \ID{CVodeSetMaxHnilWarns} specifies the maximum number of warning
  messages issued by the solver that $t+h=t$ on the next internal step.
}
{
  \begin{args}[cvode\_mem]
  \item[cvode\_mem] (\id{void *})
    pointer to the {\cvode} memory block.
  \item[mxhnil] (\id{int})
    maximum number of warning messages
  \end{args}
}
{
  The return value \id{flag} (of type \id{int}) is one of
  \begin{args}[CV\_MEM\_NULL]
  \item[\Id{CV\_SUCCESS}] 
    The optional value has been successfully set.
  \item[\Id{CV\_MEM\_NULL}]
    The \id{cvode\_mem} pointer is \id{NULL}.
  \end{args}
}
{
  The default value is $10$.
  A negative \id{mxhnil} value indicates that no warning messages should
  be issued.
}
%%
%%
\ucfunction{CVodeSetStabLimDet}
{
flag = CVodeSetstabLimDet(cvode\_mem, stldet);
}
{
  The function \ID{CVodeSetStabLimDet} indicates to turn on/off
  the BDF stability limit detection algorithm. See \S\ref{s:bdf_stab}.
}
{
  \begin{args}[cvode\_mem]
  \item[cvode\_mem] (\id{void *})
    pointer to the {\cvode} memory block.
  \item[stldet] (\id{booleantype})
    flag to control stability limit detection (\id{TRUE} = on; \id{FALSE} = off).
  \end{args}
}
{
  The return value \id{flag} (of type \id{int}) is one of
  \begin{args}[CV\_ILL\_INPUT]
  \item[\Id{CV\_SUCCESS}] 
    The optional value has been successfully set.
  \item[\Id{CV\_MEM\_NULL}]
    The \id{cvode\_mem} pointer is \id{NULL}.
  \item[\Id{CV\_ILL\_INPUT}]
    The linear multistep method is not set to \id{CV\_BDF}.
  \end{args}
}
{
  The default value is \id{FALSE}. If \id{stldet = TRUE}, when BDF is used
  and the method order is 3 or greater, an internal function, \id{CVsldet},
  is called to detect stability limit. If limit is detected, the order is reduced.
}
%%
%%
\ucfunction{CVodeSetInitStep}
{
flag = CVodeSetInitStep(cvode\_mem, hin);
}
{
  The function \ID{CVodeSetInitStep} specifies the initial step size.
}
{
  \begin{args}[cvode\_mem]
  \item[cvode\_mem] (\id{void *})
    pointer to the {\cvode} memory block.
  \item[hin] (\id{realtype})
    value of the initial step size.
  \end{args}
}
{
  The return value \id{flag} (of type \id{int}) is one of
  \begin{args}[CV\_MEM\_NULL]
  \item[\Id{CV\_SUCCESS}] 
    The optional value has been successfully set.
  \item[\Id{CV\_MEM\_NULL}]
    The \id{cvode\_mem} pointer is \id{NULL}.
  \end{args}
}
{
  By default, {\cvode} estimates the initial stepsize as the solution $h$ 
  of $\| 0.5 h^2 \ddot y \|_{\mbox{\scriptsize WRMS}} = 1$,
  where $\ddot y$ is an estimated second derivative of the solution at the
  initial time.
}
%%
%%
\index{step size bounds|(}
\ucfunction{CVodeSetMinStep}
{
flag = CVodeSetMinStep(cvode\_mem, hmin);
}
{
  The function \ID{CVodeSetMinStep} specifies the minimum absolute
  value of the step size.
}
{
  \begin{args}[cvode\_mem]
  \item[cvode\_mem] (\id{void *})
    pointer to the {\cvode} memory block.
  \item[hmin] (\id{realtype})
    minimum absolute value of the step size.
  \end{args}
}
{
  The return value \id{flag} (of type \id{int}) is one of
  \begin{args}[CV\_ILL\_INPUT]
  \item[\Id{CV\_SUCCESS}] 
    The optional value has been successfully set.
  \item[\Id{CV\_MEM\_NULL}]
    The \id{cvode\_mem} pointer is \id{NULL}.
  \item[\Id{CV\_ILL\_INPUT}]
    Either \id{hmin} is not positive or it is larger than the maximum allowable step.
  \end{args}
}
{
  The default value is $0.0$.
}
%%
%%
\ucfunction{CVodeSetMaxStep}
{
flag = CVodeSetMaxStep(cvode\_mem, hmax);
}
{
  The function \ID{CVodeSetMaxStep} specifies the maximum absolute
  value of the step size.
}
{
  \begin{args}[cvode\_mem]
  \item[cvode\_mem] (\id{void *})
    pointer to the {\cvode} memory block.
  \item[hmax] (\id{realtype})
    maximum absolute value of the step size.
  \end{args}
}
{
  The return value \id{flag} (of type \id{int}) is one of
  \begin{args}[CV\_ILL\_INPUT]
  \item[\Id{CV\_SUCCESS}] 
    The optional value has been successfully set.
  \item[\Id{CV\_MEM\_NULL}]
    The \id{cvode\_mem} pointer is \id{NULL}.
  \item[\Id{CV\_ILL\_INPUT}]
    Either \id{hmax} is not positive or it is smaller than the minimum allowable step.
  \end{args}
}
{
  The default value is $\infty$.
}
\index{step size bounds|)}
%%
%%
\ucfunction{CVodeSetStopTime}
{
flag = CVodeSetStopTime(cvode\_mem, tstop);
}
{
  The function \ID{CVodeSetStopTime} specifies the value of the
  independent variable $t$ past which the solution is not to proceed.
}
{
  \begin{args}[cvode\_mem]
  \item[cvode\_mem] (\id{void *})
    pointer to the {\cvode} memory block.
  \item[tstop] (\id{realtype})
    value of the independent variable past which the solution should
    not proceed.
  \end{args}
}
{
  The return value \id{flag} (of type \id{int}) is one of
  \begin{args}[CV\_MEM\_NULL]
  \item[\Id{CV\_SUCCESS}] 
    The optional value has been successfully set.
  \item[\Id{CV\_MEM\_NULL}]
    The \id{cvode\_mem} pointer is \id{NULL}.
  \end{args}
}
{
  The default value is $\infty$.
}
%%
%%
\ucfunction{CVodeSetMaxErrTestFails}
{
flag = CVodeSetMaxErrTestFails(cvode\_mem, maxnef);
}
{
  The function \ID{CVodeSetMaxErrTestFails} specifies the
  maximum number of error test failures in attempting one step.
}
{
  \begin{args}[cvode\_mem]
  \item[cvode\_mem] (\id{void *})
    pointer to the {\cvode} memory block.
  \item[maxnef] (\id{int})
    maximum number of error test failures allowed on one step.
  \end{args}
}
{
  The return value \id{flag} (of type \id{int}) is one of
  \begin{args}[CV\_MEM\_NULL]
  \item[\Id{CV\_SUCCESS}] 
    The optional value has been successfully set.
  \item[\Id{CV\_MEM\_NULL}]
    The \id{cvode\_mem} pointer is \id{NULL}.
  \end{args}
}
{
  The default value is $7$.
}
%%
%%
\ucfunction{CVodeSetMaxNonlinIters}
{
flag = CVodeSetMaxNonlinIters(cvode\_mem, maxcor);
}
{
  The function \ID{CVodeSetMaxNonlinIters} specifies the maximum
  number of nonlinear solver iterations at one step.
}
{
  \begin{args}[cvode\_mem]
  \item[cvode\_mem] (\id{void *})
    pointer to the {\cvode} memory block.
  \item[maxcor] (\id{int})
    maximum number of nonlinear solver iterations allowed on one step.
  \end{args}
}
{
  The return value \id{flag} (of type \id{int}) is one of
  \begin{args}[CV\_MEM\_NULL]
  \item[\Id{CV\_SUCCESS}] 
    The optional value has been successfully set.
  \item[\Id{CV\_MEM\_NULL}]
    The \id{cvode\_mem} pointer is \id{NULL}.
  \end{args}
}
{
  The default value is $3$.
}
%%
%%
\ucfunction{CVodeSetMaxConvFails}
{
flag = CVodeSetMaxConvFails(cvode\_mem, maxncf);
}
{
  The function \ID{CVodeSetMaxConvFails} specifies the
  maximum number of nonlinear solver convergence failures at one step.
}
{
  \begin{args}[cvode\_mem]
  \item[cvode\_mem] (\id{void *})
    pointer to the {\cvode} memory block.
  \item[maxncf] (\id{int})
    maximum number of allowable nonlinear solver convergence failures
    on one step.
  \end{args}
}
{
  The return value \id{flag} (of type \id{int}) is one of
  \begin{args}[CV\_MEM\_NULL]
  \item[\Id{CV\_SUCCESS}] 
    The optional value has been successfully set.
  \item[\Id{CV\_MEM\_NULL}]
    The \id{cvode\_mem} pointer is \id{NULL}.
  \end{args}
}
{
  The default value is $10$.
}
%%
%%
\ucfunction{CVodeSetNonlinConvCoef}
{
flag = CVodeSetNonlinConvCoef(cvode\_mem, nlscoef);
}
{
  The function \ID{CVodeSetNonlinConvCoef} specifies the safety factor
  in the nonlinear convergence test (see \S\ref{ss:ivp_sol}).
}
{
  \begin{args}[cvode\_mem]
  \item[cvode\_mem] (\id{void *})
    pointer to the {\cvode} memory block.
  \item[nlscoef] (\id{realtype})
    coefficient in nonlinear convergence test.
  \end{args}
}
{
  The return value \id{flag} (of type \id{int}) is one of
  \begin{args}[CV\_MEM\_NULL]
  \item[\Id{CV\_SUCCESS}] 
    The optional value has been successfully set.
  \item[\Id{CV\_MEM\_NULL}]
    The \id{cvode\_mem} pointer is \id{NULL}.
  \end{args}
}
{
  The default value is $0.1$.
}
%%
\ucfunction{CVodeSetIterType}
{
flag = CVodeSetIterType(cvode\_mem, iter);
}
{
  The function \ID{CVodeSetIterType} resets the nonlinear solver
  iteration type \Id{iter}.
}
{
  \begin{args}[cvode\_mem]
  \item[cvode\_mem] (\id{void *})
    pointer to the {\cvode} memory block.
  \item[iter] (\id{int})
    specifies the type of nonlinear solver iteration and may be
    either \Id{CV\_NEWTON} or \Id{CV\_FUNCTIONAL}. 
  \end{args}
}
{
  The return value \id{flag} (of type \id{int}) is one of
  \begin{args}[CV\_ILL\_INPUT]
  \item[\Id{CV\_SUCCESS}] 
    The optional value has been successfully set.
  \item[\Id{CV\_MEM\_NULL}]
    The \id{cvode\_mem} pointer is \id{NULL}.
  \item[\Id{CV\_ILL\_INPUT}]
    The \id{iter} value passed is neither \Id{CV\_NEWTON} nor \Id{CV\_FUNCTIONAL}.
  \end{args}
}
{
  The nonlinear solver iteration type is initially specified in the call
  to \id{CVodeCreate} (see \S\ref{sss:cvodemalloc}). This function call is
  needed only if \id{iter} is being changed from its value in the prior call 
  to \id{CVodeCreate}.
}
%%
%%
\ucfunction{CVodeSetTolerances}
{
flag = CVodeSetTolerances(cvode\_mem, itol, reltol, abstol);
}
{
  The function \ID{CVodeSetTolerances} resets the integration tolerances.
}
{
  \begin{args}[cvode\_mem]
  \item[cvode\_mem] (\id{void *})
    pointer to the {\cvodes} memory block.
  \item[itol] (\id{int}) 
    is either \ID{CV\_SS} or \ID{CV\_SV}, where \ID{itol}$=$\id{CV\_SS} indicates scalar
    relative error tolerance and scalar absolute error tolerance, while
    \id{itol}$=$\id{CV\_SV} indicates scalar relative error tolerance and vector
    absolute error tolerance.  The latter choice is important when the absolute
    error tolerance needs to be different for each component of the ODE. 
  \item[reltol] (\id{realtype *})
    \index{tolerances}
    is a pointer to the relative error tolerance.
  \item[abstol] (\id{void *})
    is a pointer to the absolute error tolerance.
  \end{args}
}
{
  The return value \id{flag} (of type \id{int}) is one of
  \begin{args}[CV\_ILL\_INPUT]
  \item[\Id{CV\_SUCCESS}] 
    The tolerances have been successfully set.
  \item[\Id{CV\_MEM\_NULL}]
    The \id{cvode\_mem} pointer is \id{NULL}.
  \item[\Id{CV\_ILL\_INPUT}]
    An input argument has an illegal value.
  \end{args}
}
{
  The integration tolerances are initially specified in the call
  to \id{CVodeMalloc} (see \S\ref{sss:cvodemalloc}). This function call is
  needed only if the tolerances are being changed from their values between
  succesive calls to \id{CVode}.
}
%%
\index{optional input!solver|)}

%%==============================================================================
\subsubsection{Linear solver optional input functions}
%%==============================================================================

The linear solver modules, with one exception, allow for various optional 
inputs, which are described here. The diagonal linear solver module has no
optional inputs.
%%
\paragraph\noindent{\bf Dense Linear solver.}
\index{optional input!dense linear solver|(}
\index{CVDENSE@{\cvdense} linear solver!optional input|(}
The \index{CVDENSE@{\cvdense} linear solver!Jacobian approximation used by}
{\cvdense} solver needs a function to compute a dense approximation to
the Jacobian matrix $J(t,y)$.  This function must be of type \id{CVDenseJacFn}. 
The user can supply his/her own dense Jacobian function, or use the default 
difference quotient function \Id{CVDenseDQJac} 
\index{Jacobian approximation function!dense!difference quotient}
that comes with the {\cvdense} solver.
To specify a user-supplied Jacobian function \id{djac} and associated user 
data \id{jac\_data}, {\cvdense} provides the functions \id{CVDenseSetJacFn}
and \id{CVDenseSetJacData}, respectively.
The {\cvdense} solver passes the pointer it receives through \id{CVDenseSetJacData} 
to its dense Jacobian function. This allows the user to
create an arbitrary structure with relevant problem data and access it
during the execution of the user-supplied Jacobian function, without
using global data in the program.  The pointer \id{jac\_data} may be
identical to \id{f\_data}, if the latter was specified through \id{CVodeSetFdata}.
%%
\index{Jacobian approximation function!dense!user-supplied}
\ucfunction{CVDenseSetJacFn}
{
  flag = CVDenseSetJacFn(cvode\_mem, djac);
}
{
  The function \ID{CVDenseSetJacFn} specifies the dense Jacobian
  approximation function to be used.
}
{
  \begin{args}[cvode\_mem]
  \item[cvode\_mem] (\id{void *})
    pointer to the {\cvode} memory block.
  \item[djac] (\id{CVDenseJacFn})
    user-defined dense Jacobian approximation function.
  \end{args}
}
{
  The return value \id{flag} (of type \id{int}) is one of
  \begin{args}[CVDENSE\_LMEM\_NULL]
  \item[\Id{CVDENSE\_SUCCESS}] 
    The optional value has been successfully set.
  \item[\Id{CVDENSE\_MEM\_NULL}]
    The \id{cvode\_mem} pointer is \id{NULL}.
  \item[\Id{CVDENSE\_LMEM\_NULL}]
    The {\cvdense} linear solver has not been initialized.
  \end{args}
}
{
  By default, {\cvdense} uses the difference quotient function \id{CVDenseDQJac}.
  If \id{NULL} is passed to \id{djac}, this default function is used.

  The function type \id{CVDenseJacFn} is described in \S\ref{ss:djacFn}.
}
%%
\ucfunction{CVDenseSetJacData}
{
  flag = CVDenseSetJacData(cvode\_mem, jac\_data);
}
{
  The function \ID{CVDenseSetJacData} specifies the data structure
  to be passed to the user supplied dense Jacobian approximation 
  function each time it is called.
}
{
  \begin{args}[cvode\_mem]
  \item[cvode\_mem] (\id{void *})
    pointer to the {\cvode} memory block.
  \item[jac\_data] (\id{void *})
    pointer to the user-defined data structure.
  \end{args}
}
{
  The return value \id{flag} (of type \id{int}) is one of
  \begin{args}[CVDENSE\_LMEM\_NULL]
  \item[\Id{CVDENSE\_SUCCESS}] 
    The optional value has been successfully set.
  \item[\Id{CVDENSE\_MEM\_NULL}]
    The \id{cvode\_mem} pointer is \id{NULL}.
  \item[\Id{CVDENSE\_LMEM\_NULL}]
    The {\cvdense} linear solver has not been initialized.
  \end{args}
}
{}
\index{CVDENSE@{\cvdense} linear solver!optional input|)}
\index{optional input!dense linear solver|)}
%%
%%------------------------------------------------------------------------------
%%
\paragraph\noindent{\bf Band Linear solver.}
\index{optional input!band linear solver|(}
\index{CVBAND@{\cvband} linear solver!optional input|(}
The \index{CVBAND@{\cvband} linear solver!Jacobian approximation used by}
{\cvdense} solver needs a function to compute a banded approximation to
the Jacobian matrix $J(t,y)$.  This function must be of type \id{CVBandJacFn}. 
The user can supply his/her own banded Jacobian approximation function, 
or use the default difference quotient function \Id{CVBandDQJac} 
\index{Jacobian approximation function!band!difference quotient}
that comes with the {\cvband} solver.
To specify a user-supplied Jacobian function \id{bjac} and associated user 
data \id{jac\_data}, {\cvband} provides the functions \id{CVBandSetJacFn}
and \id{CVBandSetJacData}, respectively.
The {\cvband} solver passes the pointer it receives through \id{CVBandSetJacData} 
to its banded Jacobian approximation function. This allows the user to
create an arbitrary structure with relevant problem data and access it
during the execution of the user-supplied Jacobian function, without
using global data in the program.  The pointer \id{jac\_data} may be
identical to \id{f\_data}, if the latter was specified through \id{CVodeSetFdata}.
%%
\index{Jacobian approximation function!band!user-supplied}
\ucfunction{CVBandSetJacFn}
{
  flag = CVBandSetJacFn(cvode\_mem, bjac);
}
{
  The function \ID{CVBandSetJacFn} specifies the banded Jacobian
  approximation function to be used.
}
{
  \begin{args}[cvode\_mem]
  \item[cvode\_mem] (\id{void *})
    pointer to the {\cvode} memory block.
  \item[bjac] (\id{CVBandJacFn})
    user-defined banded Jacobian approximation function.
  \end{args}
}
{
  The return value \id{flag} (of type \id{int}) is one of
  \begin{args}[CVBAND\_LMEM\_NULL]
  \item[\Id{CVBAND\_SUCCESS}] 
    The optional value has been successfully set.
  \item[\Id{CVBAND\_MEM\_NULL}]
    The \id{cvode\_mem} pointer is \id{NULL}.
  \item[\Id{CVBAND\_LMEM\_NULL}]
    The {\cvband} linear solver has not been initialized.
  \end{args}
}
{
  By default, {\cvband} uses the difference quotient function \id{CVBandDQJac}.
  If \id{NULL} is passed to \id{bjac}, this default function is used.

  The function type \id{CVBandJacFn} is described in \S\ref{ss:bjacFn}.
}
%%
\ucfunction{CVBandSetJacData}
{
  flag = CVBandSetJacData(cvode\_mem, jac\_data);
}
{
  The function \ID{CVBandSetJacData} specifies the data structure
  to be passed to the user supplied banded Jacobian approximation 
  function each time it is called.
}
{
  \begin{args}[cvode\_mem]
  \item[cvode\_mem] (\id{void *})
    pointer to the {\cvode} memory block.
  \item[jac\_data] (\id{void *})
    pointer to the user-defined data structure.
  \end{args}
}
{
  The return value \id{flag} (of type \id{int}) is one of
  \begin{args}[CVBAND\_LMEM\_NULL]
  \item[\Id{CVBAND\_SUCCESS}] 
    The optional value has been successfully set.
  \item[\Id{CVBAND\_MEM\_NULL}]
    The \id{cvode\_mem} pointer is \id{NULL}.
  \item[\Id{CVBAND\_LMEM\_NULL}]
    The {\cvdense} linear solver has not been initialized.
  \end{args}
}
{}
\index{CVBAND@{\cvband} linear solver!optional input|)}
\index{optional input!band linear solver|)}
%%
%%------------------------------------------------------------------------------
%%
\paragraph\noindent{\bf SPGMR Linear solver.}
\index{optional input!iterative linear solver|(}
\index{CVSPGMR@{\cvspgmr} linear solver!optional input|(}
\index{preconditioning!user-supplied|(}
The call to \id{CVSpgmr} is used to communicate the type of preconditioning 
(\id{pretype}) and the maximum dimension of the Krylov subspace to be used
(\id{maxl}).  The \id{pretype} parameter can be \id{PREC\_NONE},
\id{PREC\_LEFT}, \id{PREC\_RIGHT}, or \id{PREC\_BOTH}.
If no preconditioning is desired, pass \id{PREC\_NONE} for \id{pretype}.
Otherwise, a preconditioner solve function \id{psolve} is required.  
Regardless of the type of preconditioning, a preconditioner setup function 
\id{psetup} is sometimes useful, but is {\em not} required. 

If any type of preconditioning is to be done within the {\spgmr} method,
then the user must supply a preconditioner solve function \id{psolve}
and specify it through a call to \id{CVSpgmrSetPrecSolveFn}.
\index{CVSPGMR@{\cvspgmr} linear solver!preconditioner solve function}
%%
The evaluation and preprocessing of any Jacobian-related data needed
by the user's preconditioner solve function is done in the optional
user-supplied function \id{psetup}. Both of these functions are
fully specified in \S\ref{ss:user_fct_sim}.
If used, the \id{psetup} function should be specified through a call to
\id{CVSpgmrSetPrecSetupFn}.
\index{CVSPGMR@{\cvspgmr} linear solver!preconditioner setup function}
%%
Optionally, the {\cvspgmr} solver passes the pointer it receives through 
\id{CVSpgmrSetPrecData} to the preconditioner setup and solve functions.  
This allows the user to create an arbitrary structure with relevant problem data 
and access it during the execution of the user-supplied preconditioner functions
without using global data in the program.  
The pointer \id{P\_data} may be identical to \id{f\_data}, if the latter was 
specified through \id{CVodeSetFdata}.

The \index{CVSPGMR@{\cvspgmr} linear solver!Jacobian approximation used by}
{\cvspgmr} solver requires a function to compute an approximation to the
product between the Jacobian matrix $J(t,y)$ and a vector $v$.
The user can supply his/her own Jacobian times vector approximation function, 
or use the difference quotient function \Id{CVSpgmrDQJtimes} 
\index{Jacobian approximation function!Jacobian times vector!difference quotient}
that comes with the {\cvspgmr} solver.
A user-defined Jacobian-vector function must be of type \id{CVSpgmrJtimesFn} and 
can be specified through a call to \id{CVSpgmrSetJacTimesVecFn} 
(see \S\ref{ss:user_fct_sim} for specification details).
%%
As with the preconditioner user data structure \id{P\_data}, 
the user can specify, through a call to \id{CVSpgmrSetJacData}, a pointer to a 
user-defined data structure, \id{jac\_data}, which
the {\cvspgmr} solver passes to the Jacobian times vector function \id{jtimes}
each time it is called.  
The pointer \id{jac\_data} may be identical to \id{P\_data} and/or \id{f\_data}.
%%
%%
\ucfunction{CVSpgmrSetPrecSolveFn}
{
  flag = CVSpgmrSetPrecSolveFn(cvode\_mem, psolve);
}
{
  The function \ID{CVSpgmrSet} specifies the preconditioner
  solve function.
}
{
  \begin{args}[cvode\_mem]
  \item[cvode\_mem] (\id{void *})
    pointer to the {\cvode} memory block.
  \item[psolve] (\id{CVSpgmrPrecSolveFn})
    user-defined preconditioner solve function.
  \end{args}
}
{
  The return value \id{flag} (of type \id{int}) is one of
  \begin{args}[CVSPGMR\_LMEM\_NULL]
  \item[\Id{CVSPGMR\_SUCCESS}] 
    The optional value has been successfully set.
  \item[\Id{CVSPGMR\_MEM\_NULL}]
    The \id{cvode\_mem} pointer is \id{NULL}.
  \item[\Id{CVSPGMR\_LMEM\_NULL}]
    The {\cvspgmr} linear solver has not been initialized.
  \end{args}
}
{
   The function type \id{CVSpgmrPrecSolveFn} is described in \S\ref{ss:psolveFn}.
}
%%
%%
\ucfunction{CVSpgmrSetPrecSetupFn}
{
  flag = CVSpgmrSetPrecSetupFn(cvode\_mem, psetup);
}
{
  The function \ID{CVSpgmrSetPrecSetupFn} specifies the preconditioner
  preprocessing function.
}
{
  \begin{args}[cvode\_mem]
  \item[cvode\_mem] (\id{void *})
    pointer to the {\cvode} memory block.
  \item[psetup] (\id{CVSpgmrPrecSetupFn})
    user-defined preconditioner setup function.
  \end{args}
}
{
  The return value \id{flag} (of type \id{int}) is one of
  \begin{args}[CVSPGMR\_LMEM\_NULL]
  \item[\Id{CVSPGMR\_SUCCESS}] 
    The optional value has been successfully set.
  \item[\Id{CVSPGMR\_MEM\_NULL}]
    The \id{cvode\_mem} pointer is \id{NULL}.
  \item[\Id{CVSPGMR\_LMEM\_NULL}]
    The {\cvspgmr} linear solver has not been initialized.
  \end{args}
}
{
   The function type \id{CVSpgmrPrecSetupFn} is described in \S\ref{ss:precondFn}.
}
%%
%%
\ucfunction{CVSpgmrSetPrecData}
{
  flag = CVSpgmrSetPrecData(cvode\_mem, P\_data);
}
{
  The function \ID{CVSpgmrSetPrecData} specifies the data structure
  to be passed to the user supplied preconditioner setup and solve
  functions each time they are called.
}
{
  \begin{args}[cvode\_mem]
  \item[cvode\_mem] (\id{void *})
    pointer to the {\cvode} memory block.
  \item[P\_data] (\id{void *})
     pointer to the user-defined data structure.
  \end{args}
}
{
  The return value \id{flag} (of type \id{int}) is one of
  \begin{args}[CVSPGMR\_LMEM\_NULL]
  \item[\Id{CVSPGMR\_SUCCESS}] 
    The optional value has been successfully set.
  \item[\Id{CVSPGMR\_MEM\_NULL}]
    The \id{cvode\_mem} pointer is \id{NULL}.
  \item[\Id{CVSPGMR\_LMEM\_NULL}]
    The {\cvspgmr} linear solver has not been initialized.
  \end{args}
}
{}
%%
\index{preconditioning!user-supplied|)}
%%
\index{Jacobian approximation function!Jacobian times vector!user-supplied}
\ucfunction{CVSpgmrSetJacTimesVecFn}
{
  flag = CVSpgmrSetJacTimesVecFn(cvode\_mem, jtimes);
}
{
  The function \ID{CVSpgmrSetJacTimesFn} specifies the Jacobian-vector 
  function to be used.
}
{
  \begin{args}[cvode\_mem]
  \item[cvode\_mem] (\id{void *})
    pointer to the {\cvode} memory block.
  \item[jtimes] (\id{CVSpgmrJacTimesVecFn})
    user-defined Jacobian-vector product function.
  \end{args}
}
{
  The return value \id{flag} (of type \id{int}) is one of
  \begin{args}[CVSPGMR\_LMEM\_NULL]
  \item[\Id{CVSPGMR\_SUCCESS}] 
    The optional value has been successfully set.
  \item[\Id{CVSPGMR\_MEM\_NULL}]
    The \id{cvode\_mem} pointer is \id{NULL}.
  \item[\Id{CVSPGMR\_LMEM\_NULL}]
    The {\cvspgmr} linear solver has not been initialized.
  \end{args}
}
{
  By default, {\cvspgmr} uses the difference quotient function \id{CVSpgmrDQJtimes}.
  If \id{NULL} is passed to \id{jtimes}, this default function is used.

  The function type \id{CVSpgmrJacTimesVecFn} is described in \S\ref{ss:jtimesFn}.
}
%%
%%
\ucfunction{CVSpgmrSetJacData}
{
  flag = CVSpgmrSetJacData(cvode\_mem, jac\_data);
}
{
  The function \ID{CVSpgmrSetJacData} specifies the data structure
  to be passed to the user supplied Jacobian-vector
  function each time it is called.
}
{
  \begin{args}[cvode\_mem]
  \item[cvode\_mem] (\id{void *})
    pointer to the {\cvode} memory block.
  \item[jac\_data] (\id{void *})
     pointer to the user-defined data structure.
  \end{args}
}
{
  The return value \id{flag} (of type \id{int}) is one of
  \begin{args}[CVSPGMR\_LMEM\_NULL]
  \item[\Id{CVSPGMR\_SUCCESS}] 
    The optional value has been successfully set.
  \item[\Id{CVSPGMR\_MEM\_NULL}]
    The \id{cvode\_mem} pointer is \id{NULL}.
  \item[\Id{CVSPGMR\_LMEM\_NULL}]
    The {\cvspgmr} linear solver has not been initialized.
  \end{args}
}
{}
%%
%%
\ucfunction{CVSpgmrSetGSType}
{
  flag = CVSpgmrSetGSType(cvode\_mem, gstype);
}
{
  The function \ID{CVSpgmrSetGSType} specifies the 
  Gram-Schmidt orthogonalization to be used. 
  This must be one of the enumeration constants \ID{MODIFIED\_GS}
  or \ID{CLASSICAL\_GS}. These correspond to using modified Gram-Schmidt 
  and classical Gram-Schmidt, respectively. 
  \index{Gram-Schmidt procedure}
}
{
  \begin{args}[cvode\_mem]
  \item[cvode\_mem] (\id{void *})
    pointer to the {\cvode} memory block.
  \item[gstype] (\id{int})
    type of Gram-Schmidt orthogonalization.
  \end{args}
}
{
  The return value \id{flag} (of type \id{int}) is one of
  \begin{args}[CVSPGMR\_ILL\_INPUT]
  \item[\Id{CVSPGMR\_SUCCESS}] 
    The optional value has been successfully set.
  \item[\Id{CVSPGMR\_MEM\_NULL}]
    The \id{cvode\_mem} pointer is \id{NULL}.
  \item[\Id{CVSPGMR\_LMEM\_NULL}]
    The {\cvspgmr} linear solver has not been initialized.
  \item[\Id{CVSPGMR\_ILL\_INPUT}]
    The Gram-Schmidt orthogonalization type \id{gstype} is not valid.
  \end{args}
}
{
  The default value is \id{MODIFIED\_GS}.
}
%%
%%
\ucfunction{CVSpgmrSetDelt}
{
  flag = CVSpgmrSetDelt(cvode\_mem, delt);
}
{
  The function \ID{CVSpgmrSetDelt} specifies the factor by
  which the GMRES\index{GMRES method} convergence test constant is reduced
  from the Newton iteration test constant.
}
{
  \begin{args}[cvode\_mem]
  \item[cvode\_mem] (\id{void *})
    pointer to the {\cvode} memory block.
  \item[delt] (\id{realtype})

  \end{args}
}
{
  The return value \id{flag} (of type \id{int}) is one of
  \begin{args}[CVSPGMR\_ILL\_INPUT]
  \item[\Id{CVSPGMR\_SUCCESS}] 
    The optional value has been successfully set.
  \item[\Id{CVSPGMR\_MEM\_NULL}]
    The \id{cvode\_mem} pointer is \id{NULL}.
  \item[\Id{CVSPGMR\_LMEM\_NULL}]
    The {\cvspgmr} linear solver has not been initialized.
  \item[\Id{CVSPGMR\_ILL\_INPUT}]
    The factor \id{delt} is negative.  
  \end{args}
}
{
  The default value is $0.05$.

  Passing a value \id{delt}$ = 0.0$ also indicates using the default value.
}
%%
%%
\ucfunction{CVSpgmrSetPrecType}
{
  flag = CVSpgmrSetPrecType(cvode\_mem, pretype);
}
{
  The function \ID{CVSpgmrSetPrecType} resets the type
  of preconditioning to be used.
}
{
  \begin{args}[cvode\_mem]
  \item[cvode\_mem] (\id{void *})
    pointer to the {\cvode} memory block.
  \item[pretype] (\id{int})
    \index{pretype@\texttt{pretype}}
    specifies the type of preconditioning and must be one of:
    \Id{PREC\_NONE}, \Id{PREC\_LEFT}, \Id{PREC\_RIGHT}, or \Id{PREC\_BOTH}.
  \end{args}
}
{
  The return value \id{flag} (of type \id{int}) is one of
  \begin{args}[CVSPGMR\_ILL\_INPUT]
  \item[\Id{CVSPGMR\_SUCCESS}] 
    The optional value has been successfully set.
  \item[\Id{CVSPGMR\_MEM\_NULL}]
    The \id{cvode\_mem} pointer is \id{NULL}.
  \item[\Id{CVSPGMR\_LMEM\_NULL}]
    The {\cvspgmr} linear solver has not been initialized.
  \item[\Id{CVSPGMR\_ILL\_INPUT}]
    The preconditioner type \id{pretype} is not valid.
  \end{args}
}
{
  The preconditioning type is initially specified in the call
  to \id{CVSpgmr} (see \S\ref{sss:lin_solv_init}). This function call is
  needed only if \id{pretype} is being changed from its value in the
  previous call to \id{CVSpgmr}.
}
%%
\index{CVSPGMR@{\cvspgmr} linear solver!optional input|)}
\index{optional input!iterative linear solver|)}

%%==============================================================================
\subsection{Interpolated output function}\label{ss:optional_dky}
%%==============================================================================
\index{optional output!interpolated solution}

An optional function \ID{CVodeGetDky} is available to obtain additional output
values.  This function must be called after a successful return from \id{CVode}
and provides interpolated values of $y$ or its derivatives, up to the current
order of the integration method, interpolated to any value of $t$ in the last
internal step taken by {\cvode}.

The call to the \id{CVodeGetDky} function has the following form:
%%
\ucfunction{CVodeGetDky}
{
  flag = CVodeGetDky(cvode\_mem, t, k, dky);
}
{
  The function \ID{CVodeGetDky} computes the \id{k}-th derivative of the \id{y}
  function at time \id{t}, i.e. $d^{(k)}y/dt^{(k)} (t)$, where $t_n - h_u \le$
  \id{t} $\le t_n$, $t_n$ denotes the current internal time reached, and $h_u$
  is the  last internal step size successfully used by the solver.  The 
  user may request \id{k} $= 0, 1, ..., q_u$, where $q_u$ is the current order. 
}
{
  \begin{args}[cvode\_mem]
  \item[cvode\_mem] (\id{void *})
    pointer to the {\cvode} memory block.
  \item[t] (\id{realtype})
  \item[k] (\id{int})
  \item[dky] (\id{N\_Vector})
    vector containing the derivative.
    This vector must be allocated by the caller. 
  \end{args}
}
{
  The return value \id{flag} (of type \id{int}) is one of
  \begin{args}[CV\_MEM\_NULL] 
  \item[\Id{CV\_SUCCESS}]
    \id{CVodeGetDky} succeeded.
  \item[\Id{CV\_BAD\_K}] 
    \id{k} is not in the range $0, 1, ..., q_u$.
  \item[\Id{CV\_BAD\_T}] 
    \id{t} is not in the interval $[t_n - h_u , t_n]$.
  \item[\Id{CV\_BAD\_DKY}] 
    The \id{dky} argument was \id{NULL}.
  \item[\Id{CV\_MEM\_NULL}] 
    The \id{cvode\_mem} argument was \id{NULL}.
  \end{args}

}
{
  It is only legal to call the function \id{CVodeGetDky} after a 
  successful return from \id{CVode}. See \id{CVodeGetLastOrder} 
  and \id{CVodeGetLastStep} in the next section for access to 
  $q_u$ and $h_u$.
}

%%==============================================================================
\subsection{Optional output functions}\label{ss:optional_output}
%%==============================================================================

{\cvode} provides an extensive list of functions that can be used to obtain
solver performance information.
Table \ref{t:optional_output} lists all optional output functions in {\cvode},
which are then described in detail in the remainder of this section.

\begin{table}
\centering
\caption{Optional outputs from {\cvode}, {\cvdense}, {\cvband}, {\cvdiag}, and
 {\cvspgmr}}
\label{t:optional_output}
\medskip
\begin{tabular}{|l|l|}\hline
{\bf Optional output} & {\bf Function name} \\
\hline
\multicolumn{2}{|c|}{\bf CVODE main solver} \\
\hline
Size of {\cvode} real and integer workspaces & \id{CVodeGetWorkSpace} \\
Cumulative number of internal steps & \id{CVodeGetNumSteps} \\
No. of calls to r.h.s. function & \id{CVodeGetNumRhsEvals} \\
No. of calls to linear solver setup function & \id{CVodeGetNumLinSolvSetups} \\
No. of local error test failures that have occurred & \id{CVodeGetNumErrTestFails} \\
Order used during the last step & \id{CVodeGetLastOrder} \\
Order to be attempted on the next step & \id{CVodeGetCurrentOrder} \\
Order reductions due to stability limit detection & \id{CVodeGetNumStabLimOrderReds} \\
Actual initial step size used & \id{CVodeGetActualInitStep} \\
Step size used for the last step & \id{CVodeGetLastStep} \\
Step size to be attempted on the next step & \id{CVodeGetCurrentStep} \\
Current internal time reached by the solver & \id{CVodeGetCurrentTime} \\
Suggested factor for tolerance scaling  & \id{CVodeGetTolScaleFactor} \\
Error weight vector for state variables & \id{CVodeGetErrWeights} \\
Estimated local error vector & \id{CVodeGetEstLocalErrors} \\
No. of nonlinear solver iterations & \id{CVodeGetNumNonlinSolvIters} \\
No. of nonlinear convergence failures & \id{CVodeGetNumNonlinSolvConvFails} \\
All {\cvode} integrator statistics & \id{CVodeGetIntegratorStats} \\
{\cvode} nonlinear solver statistics & \id{CVodeGetNonlinSolvStats} \\
Array showing roots found & \id{CvodeGetRootInfo} \\
No. of calls to user root function & \id{CVodeGetNumGEvals} \\
\hline
\multicolumn{2}{|c|}{\bf CVDENSE linear solver} \\
\hline
Size of {\cvdense} real and integer workspaces & \id{CVDenseGetWorkSpace} \\
No. of Jacobian evaluations & \id{CVDenseGetNumJacEvals} \\
No. of r.h.s. calls for finite diff. Jacobian evals. & \id{CVDenseGetNumRhsEvals} \\ 
Last return from a {\cvdense} function & \id{CVDenseGetLastFlag} \\ 
\hline
\multicolumn{2}{|c|}{\bf CVBAND linear solver} \\
\hline
Size of {\cvband} real and integer workspaces & \id{CVBandGetWorkSpace} \\
No. of Jacobian evaluations & \id{CVBandGetNumJacEvals} \\
No. of r.h.s. calls for finite diff. Jacobian evals. & \id{CVBandGetNumRhsEvals} \\ 
Last return from a {\cvband} function & \id{CVBandGetLastFlag} \\ 
\hline
\multicolumn{2}{|c|}{\bf CVDIAG linear solver} \\
\hline
Size of {\cvdiag} real and integer workspaces & \id{CVDiagGetWorkSpace} \\
No. of r.h.s. calls for finite diff. Jacobian evals. & \id{CVDiagGetNumRhsEvals} \\ 
Last return from a {\cvdiag} function & \id{CVDiagGetLastFlag} \\ 
\hline
\multicolumn{2}{|c|}{\bf CVSPGMR linear solver} \\
\hline
Size of {\cvspgmr} real and integer workspaces & \id{CVSpgmrGetWorkSpace} \\
No. of linear iterations & \id{CVSpgmrGetNumLinIters} \\
No. of linear convergence failures & \id{CVSpgmrGetNumConvFails} \\
No. of preconditioner evaluations & \id{CVSpgmrGetNumPrecEvals} \\
No. of preconditioner solves & \id{CVSpgmrGetNumPrecSolves} \\
No. of Jacobian-vector product evaluations & \id{CVSpgmrGetNumJtimesEvals} \\
No. of r.h.s. calls for finite diff. Jacobian-vector evals. & \id{CVSpgmrGetNumRhsEvals} \\ 
Last return from a {\cvspgmr} function & \id{CVSpgmrGetLastFlag} \\ 
\hline
\end{tabular}
\end{table}


%%==============================================================================
\subsubsection{Main solver optional output functions}
%%==============================================================================
\index{optional output!solver|(}
%%
{\cvode} provides several user-callable functions that can be used to obtain
different quantities that may be of interest to the user, such as solver workspace
requirements, solver performance statistics, as well as additional data from
the {\cvode} memory block (a suggested tolerance scaling factor, the error weight
vector, and the vector of estimated local errors). Also provided are functions to
extract statistics related to the performance of the {\cvode} nonlinear solver
being used. As a convenience, additional extraction functions provide the optional 
outputs in groups.
%%
These optional output functions are described next.
%%
\index{memory requirements!CVODE@{\cvode} solver|(}
%%
\ucfunction{CVodeGetWorkSpace}
{
  flag = CVodeGetWorkSpace(cvode\_mem, \&lenrw, \&leniw);
}
{
  The function \ID{CVodeGetWorkSpace} returns the
  {\cvode} integer and real workspace sizes.
}
{
  \begin{args}[cvode\_mem]
  \item[cvode\_mem] (\id{void *})
    pointer to the {\cvode} memory block.
  \item[lenrw] (\id{long int})
    the number of \id{realtype} values in the {\cvode} workspace.
  \item[leniw] (\id{long int})
    the number of integer values in the {\cvode} workspace.
  \end{args}
}
{
  The return value \id{flag} (of type \id{int}) is one of
  \begin{args}[CV\_MEM\_NULL]
  \item[\Id{CV\_SUCCESS}] 
    The optional output value has been successfully set.
  \item[\Id{CV\_MEM\_NULL}]
    The \id{cvode\_mem} pointer is \id{NULL}.
  \end{args}
}
{
  In terms of the problem size $N$ and maximum method order \id{maxord},
  the actual size of the real workspace is (\id{maxord+5})$N$ \id{realtype}
  words.  For the default values, this size is $17 N$ for the Adams method
  and $10 N$ for the BDF method.
}
%%
\index{memory requirements!CVODE@{\cvode} solver|)}
%%
\ucfunction{CVodeGetNumSteps}
{
  flag = CVodeGetNumSteps(cvode\_mem, \&nsteps);
}
{
  The function \ID{CVodeGetNumSteps} returns the cumulative number of internal 
  steps taken by the solver (total so far).
}
{
  \begin{args}[cvode\_mem]
  \item[cvode\_mem] (\id{void *})
    pointer to the {\cvode} memory block.
  \item[nsteps] (\id{long int})
    number of steps taken by {\cvode}.
  \end{args}
}
{
  The return value \id{flag} (of type \id{int}) is one of
  \begin{args}[CV\_MEM\_NULL]
  \item[\Id{CV\_SUCCESS}] 
    The optional output value has been successfully set.
  \item[\Id{CV\_MEM\_NULL}]
    The \id{cvode\_mem} pointer is \id{NULL}.
  \end{args}
}
{}
%%
%%
\ucfunction{CVodeGetNumRhsEvals}
{
  flag = CVodeGetNumRhsEvals(cvode\_mem, \&nfevals);
}
{
  The function \ID{CVodeGetNumRhsEvals} returns the 
  number of calls to the user's right-hand side evaluation function.
}
{
  \begin{args}[cvode\_mem]
  \item[cvode\_mem] (\id{void *})
    pointer to the {\cvode} memory block.
  \item[nfevals] (\id{long int})
    number of calls to the user's \id{f} function.
  \end{args}
}
{
  The return value \id{flag} (of type \id{int}) is one of
  \begin{args}[CV\_MEM\_NULL]
  \item[\Id{CV\_SUCCESS}] 
    The optional output value has been successfully set.
  \item[\Id{CV\_MEM\_NULL}]
    The \id{cvode\_mem} pointer is \id{NULL}.
  \end{args}
}
{
  The \id{nfevals} value returned by \id{CVodeGetNumRhsEvals} does not
  account for calls made to \id{f} from a linear solver or preconditioner 
  module. 
}
%%
%%
\ucfunction{CVodeGetNumLinSolvSetups}
{
  flag = CVodeGetNumLinSolvSetups(cvode\_mem, \&nlinsetups);
}
{
  The function \ID{CVodeGetNumLinSolvSetups} returns the
  number of calls made to the linear solver's setup function.
}
{
  \begin{args}[nlinsetups]
  \item[cvode\_mem] (\id{void *})
    pointer to the {\cvode} memory block.
  \item[nlinsetups] (\id{long int})
    number of calls made to the linear solver setup function.
  \end{args}
}
{
  The return value \id{flag} (of type \id{int}) is one of
  \begin{args}[CV\_MEM\_NULL]
  \item[\Id{CV\_SUCCESS}] 
    The optional output value has been successfully set.
  \item[\Id{CV\_MEM\_NULL}]
    The \id{cvode\_mem} pointer is \id{NULL}.
  \end{args}
}
{}
%%
%%
\ucfunction{CVodeGetNumErrTestFails}
{
  flag = CVodeGetNumErrTestFails(cvode\_mem, \&netfails);
}
{
  The function \ID{CVodeGetNumErrTestFails} returns the
  number of local error test failures that have occurred.
}
{
  \begin{args}[cvode\_mem]
  \item[cvode\_mem] (\id{void *})
    pointer to the {\cvode} memory block.
  \item[netfails] (\id{long int})
    number of error test failures.
  \end{args}
}
{
  The return value \id{flag} (of type \id{int}) is one of
  \begin{args}[CV\_MEM\_NULL]
  \item[\Id{CV\_SUCCESS}] 
    The optional output value has been successfully set.
  \item[\Id{CV\_MEM\_NULL}]
    The \id{cvode\_mem} pointer is \id{NULL}.
  \end{args}
}
{}
%%
%%
\ucfunction{CVodeGetLastOrder}
{
  flag = CVodeGetLastOrder(cvode\_mem, \&qlast);
}
{
  The function \ID{CVodeGetLastOrder} returns the
  integration method order used during the last internal step.
}
{
  \begin{args}[cvode\_mem]
  \item[cvode\_mem] (\id{void *})
    pointer to the {\cvode} memory block.
  \item[qlast] (\id{int})
    method order used on the last internal step.
  \end{args}
}
{
  The return value \id{flag} (of type \id{int}) is one of
  \begin{args}[CV\_MEM\_NULL]
  \item[\Id{CV\_SUCCESS}] 
    The optional output value has been successfully set.
  \item[\Id{CV\_MEM\_NULL}]
    The \id{cvode\_mem} pointer is \id{NULL}.
  \end{args}
}
{}
%%
%%
\ucfunction{CVodeGetCurrentOrder}
{
  flag = CVodeGetCurrentOrder(cvode\_mem, \&qcur);
}
{
  The function \ID{CVodeGetCurrentOrder} returns the
  integration method order to be used on the next internal step.
}
{
  \begin{args}[cvode\_mem]
  \item[cvode\_mem] (\id{void *})
    pointer to the {\cvode} memory block.
  \item[qcur] (\id{int})
    method order to be used on the next internal step.
  \end{args}
}
{
  The return value \id{flag} (of type \id{int}) is one of
  \begin{args}[CV\_MEM\_NULL]
  \item[\Id{CV\_SUCCESS}] 
    The optional output value has been successfully set.
  \item[\Id{CV\_MEM\_NULL}]
    The \id{cvode\_mem} pointer is \id{NULL}.
  \end{args}
}
{}
%%
%%
\ucfunction{CVodeGetLastStep}
{
  flag = CVodeGetLastStep(cvode\_mem, \&hlast);
}
{
  The function \ID{CVodeGetLastStep} returns the
  integration step size taken on the last internal step.
}
{
  \begin{args}[cvode\_mem]
  \item[cvode\_mem] (\id{void *})
    pointer to the {\cvode} memory block.
  \item[hlast] (\id{realtype})
    step size taken on the last internal step.
  \end{args}
}
{
  The return value \id{flag} (of type \id{int}) is one of
  \begin{args}[CV\_MEM\_NULL]
  \item[\Id{CV\_SUCCESS}] 
    The optional output value has been successfully set.
  \item[\Id{CV\_MEM\_NULL}]
    The \id{cvode\_mem} pointer is \id{NULL}.
  \end{args}
}
{}
%%
%%
\ucfunction{CVodeGetCurrentStep}
{
  flag = CVodeGetCurrentStep(cvode\_mem, \&hcur);
}
{
  The function \ID{CVodeGetCurrentStep} returns the
  integration step size to be attempted on the next internal step.
}
{
  \begin{args}[cvode\_mem]
  \item[cvode\_mem] (\id{void *})
    pointer to the {\cvode} memory block.
  \item[hcur] (\id{realtype})
    step size to be attempted on the next internal step.
  \end{args}
}
{
  The return value \id{flag} (of type \id{int}) is one of
  \begin{args}[CV\_MEM\_NULL]
  \item[\Id{CV\_SUCCESS}] 
    The optional output value has been successfully set.
  \item[\Id{CV\_MEM\_NULL}]
    The \id{cvode\_mem} pointer is \id{NULL}.
  \end{args}
}
{}
%%
%%
\ucfunction{CVodeGetActualInitStep}
{
  flag = CVodeGetActualInitStep(cvode\_mem, \&hinused);
}
{
  The function \ID{CVodeGetActualInitStep} returns the
  value of the integration step size used on the first step.
}
{
  \begin{args}[cvode\_mem]
  \item[cvode\_mem] (\id{void *})
    pointer to the {\cvode} memory block.
  \item[hinused] (\id{realtype})
    actual value of initial step size.
  \end{args}
}
{
  The return value \id{flag} (of type \id{int}) is one of
  \begin{args}[CV\_MEM\_NULL]
  \item[\Id{CV\_SUCCESS}] 
    The optional output value has been successfully set.
  \item[\Id{CV\_MEM\_NULL}]
    The \id{cvode\_mem} pointer is \id{NULL}.
  \end{args}
}
{
  Even if the value of the initial integration step size was specified
  by the user through a call to \id{CVodeSetInitStep}, this value might have 
  been changed by {\cvode} to ensure that the step size is within the 
  prescribed bounds ($h_{\min} \le h_0 \le h_{\max}$), or to meet the
  local error test.
}
%%
%%
\ucfunction{CVodeGetCurrentTime}
{
  flag = CVodeGetCurrentTime(cvode\_mem, \&tcur);
}
{
  The function \ID{CVodeGetCurrentTime} returns the
  current internal time reached by the solver.
}
{
  \begin{args}[cvode\_mem]
  \item[cvode\_mem] (\id{void *})
    pointer to the {\cvode} memory block.
  \item[tcur] (\id{realtype})
    current internal time reached.
  \end{args}
}
{
  The return value \id{flag} (of type \id{int}) is one of
  \begin{args}[CV\_MEM\_NULL]
  \item[\Id{CV\_SUCCESS}] 
    The optional output value has been successfully set.
  \item[\Id{CV\_MEM\_NULL}]
    The \id{cvode\_mem} pointer is \id{NULL}.
  \end{args}
}
{}
%%
%%
\ucfunction{CVodeGetNumStabLimOrderReds}
{
  flag = CVodeGetNumStabLimOrderReds(cvode\_mem, \&nslred);
}
{
  The function \ID{CVodeGetNumStabLimOrderReds} returns the
  number of order reductions dictated by the BDF stability limit 
  detection algorithm (see \S\ref{s:bdf_stab}).
}
{
  \begin{args}[cvode\_mem]
  \item[cvode\_mem] (\id{void *})
    pointer to the {\cvode} memory block.
  \item[nslred] (\id{long int})
    number of order reductions due to stability limit detection.
  \end{args}
}
{
  The return value \id{flag} (of type \id{int}) is one of
  \begin{args}[CV\_NO\_SLDET]
  \item[\Id{CV\_SUCCESS}] 
    The optional output value has been successfully set.
  \item[\Id{CV\_MEM\_NULL}]
    The \id{cvode\_mem} pointer is \id{NULL}.
  \item[\Id{CV\_NO\_SLDET}]
    The stability limit detection algorithm was not activated 
    through a call to \id{CVodeSetStabLimDet}.
  \end{args}
}
{}
%%
%%
\ucfunction{CVodeGetTolScaleFactor}
{
  flag = CVodeGetTolScaleFactor(cvode\_mem, \&tolsfac);
}
{
  The function \ID{CVodeGetTolScaleFactor} returns a
  suggested factor by which the user's tolerances 
  should be scaled when too much accuracy has been 
  requested for some internal step.
}
{
  \begin{args}[cvode\_mem]
  \item[cvode\_mem] (\id{void *})
    pointer to the {\cvode} memory block.
  \item[tolsfac] (\id{realtype})
    suggested scaling factor for user tolerances.
  \end{args}
}
{
  The return value \id{flag} (of type \id{int}) is one of
  \begin{args}[CV\_MEM\_NULL]
  \item[\Id{CV\_SUCCESS}] 
    The optional output value has been successfully set.
  \item[\Id{CV\_MEM\_NULL}]
    The \id{cvode\_mem} pointer is \id{NULL}.
  \end{args}
}
{}
%%
%%
\ucfunction{CVodeGetErrWeights}
{
  flag = CVodeGetErrWeights(cvode\_mem, \&eweight);
}
{
  The function \ID{CVodeGetErrWeights} returns the solution error weights 
  at the current time. These are the reciprocals of the $W_i$ of (\ref{e:errwt}).
}
{
  \begin{args}[cvode\_mem]
  \item[cvode\_mem] (\id{void *})
    pointer to the {\cvode} memory block.
  \item[eweight] (\id{N\_Vector})
    solution error weights at the current time.
  \end{args}
}
{
  The return value \id{flag} (of type \id{int}) is one of
  \begin{args}[CV\_MEM\_NULL]
  \item[\Id{CV\_SUCCESS}] 
    The optional output value has been successfully set.
  \item[\Id{CV\_MEM\_NULL}]
    The \id{cvode\_mem} pointer is \id{NULL}.
  \end{args}
}
{
  The user need not allocate space for \id{eweight} and should not modify
  any of its components.
}
%%
%%
\ucfunction{CVodeGetEstLocalErrors}
{
  flag = CVodeGetEstLocalErrors(cvode\_mem, \&ele);
}
{
  The function \ID{CVodeGetEstLocalErrors} returns the
  vector of estimated local errors.
}
{
  \begin{args}[cvode\_mem]
  \item[cvode\_mem] (\id{void *})
    pointer to the {\cvode} memory block.
  \item[ele] (\id{N\_Vector})
    estimated local errors.
  \end{args}
}
{
  The return value \id{flag} (of type \id{int}) is one of
  \begin{args}[CV\_MEM\_NULL]
  \item[\Id{CV\_SUCCESS}] 
    The optional output value has been successfully set.
  \item[\Id{CV\_MEM\_NULL}]
    The \id{cvode\_mem} pointer is \id{NULL}.
  \end{args}
}
{
  The user need not allocate space for \id{ele}.
}
%%
%%
\ucfunction{CVodeGetIntegratorStats}
{
  \begin{tabular}[t]{@{}r@{}l@{}}
    flag = CVodeGetIntegratorStats(&cvode\_mem, \&nsteps, \&nfevals, \\
                                   &\&nlinsetups, \&netfails, \&qlast, \&qcur, \\
                                   &\&hinused, \&hlast, \&hcur, \&tcur);
  \end{tabular}
}
{
  The function \ID{CVodeGetIntegratorStats} returns the {\cvode} integrator
  statistics as a group.
}
{
  \begin{args}[nlinsetups]
  \item[cvode\_mem] (\id{void *})
    pointer to the {\cvode} memory block.
  \item[nsteps] (\id{long int})
    number of steps taken by {\cvode}.
  \item[nfevals] (\id{long int})
    number of calls to the user's \id{f} function.
  \item[nlinsetups] (\id{long int})
    number of calls made to the linear solver setup function.
  \item[netfails] (\id{long int})
    number of error test failures.
  \item[qlast] (\id{int})
    method order used on the last internal step.
  \item[qcur] (\id{int})
    method order to be used on the next internal step.
  \item[hinused] (\id{realtype})
    actual value of initial step size.
  \item[hlast] (\id{realtype})
    step size taken on the last internal step.
  \item[hcur] (\id{realtype})
    step size to be attempted on the next internal step.
  \item[tcur] (\id{realtype})
    current internal time reached.
  \end{args}
}
{
  The return value \id{flag} (of type \id{int}) is one of
  \begin{args}[CV\_MEM\_NULL]
  \item[\Id{CV\_SUCCESS}] 
    the optional output values have been successfully set.
  \item[\Id{CV\_MEM\_NULL}]
    the \id{cvode\_mem} pointer is \id{NULL}.
  \end{args}
}
{}
%%
%%
\ucfunction{CVodeGetNumNonlinSolvIters}
{
  flag = CVodeGetNumNonlinSolvIters(cvode\_mem, \&nniters);
}
{
  The function \ID{CVodeGetNumNonlinSolvIters} returns the
  number of nonlinear (functional or Newton) iterations performed. 
}
{
  \begin{args}[cvode\_mem]
  \item[cvode\_mem] (\id{void *})
    pointer to the {\cvode} memory block.
  \item[nniters] (\id{long int})
    number of nonlinear iterations performed.
  \end{args}
}
{
  The return value \id{flag} (of type \id{int}) is one of
  \begin{args}[CV\_MEM\_NULL]
  \item[\Id{CV\_SUCCESS}] 
    The optional output value has been successfully set.
  \item[\Id{CV\_MEM\_NULL}]
    The \id{cvode\_mem} pointer is \id{NULL}.
  \end{args}
}
{}
%%
%%
\ucfunction{CVodeGetNumNonlinSolvConvFails}
{
  flag = CVodeGetNumNonlinSolvConvFails(cvode\_mem, \&nncfails);
}
{
  The function \ID{CVodeGetNumNonlinSolvConvFails} returns the
  number of nonlinear convergence failures that have occurred.
}
{
  \begin{args}[cvode\_mem]
  \item[cvode\_mem] (\id{void *})
    pointer to the {\cvode} memory block.
  \item[nncfails] (\id{long int})
    number of nonlinear convergence failures.
  \end{args}
}
{
  The return value \id{flag} (of type \id{int}) is one of
  \begin{args}[CV\_MEM\_NULL]
  \item[\Id{CV\_SUCCESS}] 
    The optional output value has been successfully set.
  \item[\Id{CV\_MEM\_NULL}]
    The \id{cvode\_mem} pointer is \id{NULL}.
  \end{args}
}
{}
%%
%%
\ucfunction{CVodeGetNonlinSolvStats}
{
  flag = CVodeGetNonlinSolvStats(cvode\_mem, \&nniters, \&nncfails);
}
{
  The function \ID{CVodeGetNonlinSolvStats} returns the
  {\cvode} nonlinear solver statistics as a group.
}
{
  \begin{args}[cvode\_mem]
  \item[cvode\_mem] (\id{void *})
    pointer to the {\cvode} memory block.
  \item[nniters] (\id{long int})
    number of nonlinear iterations performed.
  \item[nncfails] (\id{long int})
    number of nonlinear convergence failures.
  \end{args}
}
{
  The return value \id{flag} (of type \id{int}) is one of
  \begin{args}[CV\_MEM\_NULL]
  \item[\Id{CV\_SUCCESS}] 
    The optional output value has been successfully set.
  \item[\Id{CV\_MEM\_NULL}]
    The \id{cvode\_mem} pointer is \id{NULL}.
  \end{args}
}
{}

\index{optional output!solver|)}

%%==============================================================================
\subsubsection{Linear solver optional output functions}\label{sss:linsolv_io}
%%==============================================================================

For each of the linear system solver modules, there are various optional 
outputs that describe the performance of the module. The functions available 
to access these are described below.

\vspace{0.1in}\noindent{\bf Dense Linear solver}
\index{optional output!dense linear solver|(}
\index{CVDENSE@{\cvdense} linear solver!optional output|(}
%%
\index{CVDENSE@{\cvdense} linear solver!memory requirements|(} 
\index{memory requirements!CVDENSE@{\cvdense} linear solver|(}
%%
\ucfunction{CVDenseGetWorkSpace}
{
  flag = CVDenseGetWorkSpace(cvode\_mem, \&lenrwD, \&leniwD);
}
{
  The function \ID{CVDenseGetWorkSpace} returns the
  {\cvdense} real and integer workspace sizes.
}
{
  \begin{args}[cvode\_mem]
  \item[cvode\_mem] (\id{void *})
    pointer to the {\cvode} memory block.
  \item[lenrwD] (\id{long int})
    the number of \id{realtype} values in the {\cvdense} workspace.
  \item[leniwD] (\id{long int})
    the number of integer values in the {\cvdense} workspace.
  \end{args}
}
{
  The return value \id{flag} (of type \id{int}) is one of
  \begin{args}[CVDENSE\_LMEM\_NULL]
  \item[\Id{CVDENSE\_SUCCESS}] 
    The optional output value has been successfully set.
  \item[\Id{CVDENSE\_MEM\_NULL}]
    The \id{cvode\_mem} pointer is \id{NULL}.
  \item[\Id{CVDENSE\_LMEM\_NULL}]
    The {\cvdense} linear solver has not been initialized.
  \end{args}
}
{
  In terms of the problem size $N$, the actual size of the real workspace
  is $2N^2$ \id{realtype} words, and the actual size of the integer workspace
  is $N$ integer words.
}
%%
\index{CVDENSE@{\cvdense} linear solver!memory requirements|)} 
\index{memory requirements!CVDENSE@{\cvdense} linear solver|)}
%%
%%
\ucfunction{CVDenseGetNumJacEvals}
{
  flag = CVDenseGetNumJacEvals(cvode\_mem, \&njevalsD);
}
{
  The function \ID{CVDenseGetNumJacEvals} returns the
  number of calls to the dense Jacobian approximation function.
}
{
  \begin{args}[cvode\_mem]
  \item[cvode\_mem] (\id{void *})
    pointer to the {\cvode} memory block.
  \item[njevalsD] (\id{long int})
    the number of calls to the Jacobian function.
  \end{args}
}
{
  The return value \id{flag} (of type \id{int}) is one of
  \begin{args}[CVDENSE\_LMEM\_NULL]
  \item[\Id{CVDENSE\_SUCCESS}] 
    The optional output value has been successfully set.
  \item[\Id{CVDENSE\_MEM\_NULL}]
    The \id{cvode\_mem} pointer is \id{NULL}.
  \item[\Id{CVDENSE\_LMEM\_NULL}]
    The {\cvdense} linear solver has not been initialized.
  \end{args}
}
{}
%%
%%
\ucfunction{CVDenseGetNumRhsEvals}
{
  flag = CVDenseGetNumRhsEvals(cvode\_mem, \&nfevalsD);
}
{
  The function \ID{CVDenseGetNumRhsEvals} returns the
  number of calls to the user right-hand side function due to the 
  finite difference dense Jacobian approximation.
}
{
  \begin{args}[cvode\_mem]
  \item[cvode\_mem] (\id{void *})
    pointer to the {\cvode} memory block.
  \item[nfevalsD] (\id{long int})
    the number of calls to the user right-hand side function.
  \end{args}
}
{
  The return value \id{flag} (of type \id{int}) is one of
  \begin{args}[CVDENSE\_LMEM\_NULL]
  \item[\Id{CVDENSE\_SUCCESS}] 
    The optional output value has been successfully set.
  \item[\Id{CVDENSE\_MEM\_NULL}]
    The \id{cvode\_mem} pointer is \id{NULL}.
  \item[\Id{CVDENSE\_LMEM\_NULL}]
    The {\cvdense} linear solver has not been initialized.
  \end{args}
}
{
  The value \id{nfevalsD} is incremented only if the default 
  \id{CVDenseDQJac} difference quotient function is used.
}
%%
%%
\ucfunction{CVDenseGetLastFlag}
{
  flag = CVDenseGetLastFlag(cvode\_mem, \&flag);
}
{
  The function \ID{CVDenseGetLastFlag} returns the
  last return value from a {\cvdense} routine. 
}
{
  \begin{args}[cvode\_mem]
  \item[cvode\_mem] (\id{void *})
    pointer to the {\cvode} memory block.
  \item[flag] (\id{int})
    the value of the last return flag from a {\cvdense} function.
  \end{args}
}
{
  The return value \id{flag} (of type \id{int}) is one of
  \begin{args}[CVDENSE\_LMEM\_NULL]
  \item[\Id{CVDENSE\_SUCCESS}] 
    The optional output value has been successfully set.
  \item[\Id{CVDENSE\_MEM\_NULL}]
    The \id{cvode\_mem} pointer is \id{NULL}.
  \item[\Id{CVDENSE\_LMEM\_NULL}]
    The {\cvdense} linear solver has not been initialized.
  \end{args}
}
{
  If the {\cvdense} setup function failed (\id{CVode} returned \id{CV\_LSETUP\_FAIL}),
  the value \id{flag} is equal to the column index (numbered from one) at which
  a zero diagonal element was encountered during the LU factorization of the 
  dense Jacobian matrix.
}
%%
%%
\index{CVDENSE@{\cvdense} linear solver!optional output|)}
\index{optional output!dense linear solver|)}
%
%-------------------------------------------------------------------------------
%
\noindent{\bf Band Linear solver}
\index{optional output!band linear solver|(}
\index{CVBAND@{\cvband} linear solver!optional output|(}
%%
\index{CVBAND@{\cvband} linear solver!memory requirements|(} 
\index{memory requirements!CVBAND@{\cvband} linear solver|(}
%%
\ucfunction{CVBandGetWorkSpace}
{
  flag = CVBandGetWorkSpace(cvode\_mem, \&lenrwB, \&leniwB);
}
{
  The function \ID{CVBandGetWorkSpace} returns the
  {\cvband} real and integer workspace sizes.
}
{
  \begin{args}[cvode\_mem]
  \item[cvode\_mem] (\id{void *})
    pointer to the {\cvode} memory block.
  \item[lenrwB] (\id{long int})
    the number of \id{realtype} values in the {\cvband} workspace.
  \item[leniwB] (\id{long int})
    the number of integer values in the {\cvband} workspace.
  \end{args}
}
{
  The return value \id{flag} (of type \id{int}) is one of
  \begin{args}[CVBAND\_LMEM\_NULL]
  \item[\Id{CVBAND\_SUCCESS}] 
    The optional output value has been successfully set.
  \item[\Id{CVBAND\_MEM\_NULL}]
    The \id{cvode\_mem} pointer is \id{NULL}.
  \item[\Id{CVBAND\_LMEM\_NULL}]
    The {\cvband} linear solver has not been initialized.
  \end{args}
}
{
  In terms of the problem size $N$ and Jacobian half-bandwidths, 
  the actual size of the real workspace is
  $(2$ \id{mupper}$+ 3$ \id{mlower} $+ 2)\, N$ \id{realtype} words,
  and the actual size of the integer workspace is $N$ integer words.
}
%%
\index{CVBAND@{\cvband} linear solver!memory requirements|)} 
\index{memory requirements!CVBAND@{\cvband} linear solver|)}
%%
%%
\ucfunction{CVBandGetNumJacEvals}
{
  flag = CVBandGetNumJacEvals(cvode\_mem, \&njevalsB);
}
{
  The function \ID{CVBandGetNumJacEvals} returns the
  number of calls to the banded Jacobian approximation function.
}
{
  \begin{args}[cvode\_mem]
  \item[cvode\_mem] (\id{void *})
    pointer to the {\cvode} memory block.
  \item[njevalsB] (\id{long int})
    the number of calls to the Jacobian function.
  \end{args}
}
{
  The return value \id{flag} (of type \id{int}) is one of
  \begin{args}[CVBAND\_LMEM\_NULL]
  \item[\Id{CVBAND\_SUCCESS}] 
    The optional output value has been successfully set.
  \item[\Id{CVBAND\_MEM\_NULL}]
    The \id{cvode\_mem} pointer is \id{NULL}.
  \item[\Id{CVBAND\_LMEM\_NULL}]
    The {\cvband} linear solver has not been initialized.
  \end{args}
}
{}
%%
%%
\ucfunction{CVBandGetNumRhsEvals}
{
  flag = CVBandGetNumRhsEvals(cvode\_mem, \&nfevalsB);
}
{
  The function \ID{CVBandGetNumRhsEvals} returns the
  number of calls to the user right-hand side function due to the 
  finite difference banded Jacobian approximation.
}
{
  \begin{args}[cvode\_mem]
  \item[cvode\_mem] (\id{void *})
    pointer to the {\cvode} memory block.
  \item[nfevalsB] (\id{long int})
    the number of calls to the user right-hand side function.
  \end{args}
}
{
  The return value \id{flag} (of type \id{int}) is one of
  \begin{args}[CVBAND\_LMEM\_NULL]
  \item[\Id{CVBAND\_SUCCESS}] 
    The optional output value has been successfully set.
  \item[\Id{CVBAND\_MEM\_NULL}]
    The \id{cvode\_mem} pointer is \id{NULL}.
  \item[\Id{CVBAND\_LMEM\_NULL}]
    The {\cvband} linear solver has not been initialized.
  \end{args}
}
{
  The value \id{nfevalsB} is incremented only if the default 
  \id{CVBandDQJac} difference quotient function is used.
}
%%
%%
\ucfunction{CVBandGetLastFlag}
{
  flag = CVBandGetLastFlag(cvode\_mem, \&flag);
}
{
  The function \ID{CVBandGetLastFlag} returns the
  last return value from a {\cvband} routine. 
}
{
  \begin{args}[cvode\_mem]
  \item[cvode\_mem] (\id{void *})
    pointer to the {\cvode} memory block.
  \item[flag] (\id{int})
    the value of the last return flag from a {\cvband} function.
  \end{args}
}
{
  The return value \id{flag} (of type \id{int}) is one of
  \begin{args}[CVBAND\_LMEM\_NULL]
  \item[\Id{CVBAND\_SUCCESS}] 
    The optional output value has been successfully set.
  \item[\Id{CVBAND\_MEM\_NULL}]
    The \id{cvode\_mem} pointer is \id{NULL}.
  \item[\Id{CVBAND\_LMEM\_NULL}]
    The {\cvband} linear solver has not been initialized.
  \end{args}
}
{
  If the {\cvband} setup sunction failed (\id{CVode} returned \id{CV\_LSETUP\_FAIL}),
  the value \id{flag} is equal to the column index (numbered from one) at which
  a zero diagonal element was encountered during the LU factorization of the 
  banded Jacobian matrix.
}
%%
%%
\index{CVBAND@{\cvband} linear solver!optional output|)}
\index{optional output!band linear solver|)}
%
%--------------------------------
%
\noindent{\bf Diagonal Linear solver.}
\index{optional output!diagonal linear solver|(}
\index{CVDIAG@{\cvdiag} linear solver!optional output|(}
%%
\index{CVDIAG@{\cvdiag} linear solver!memory requirements|(} 
\index{memory requirements!CVDIAG@{\cvdiag} linear solver|(}
%%
\ucfunction{CVDiagGetWorkSpace}
{
  flag = CVDiagGetWorkSpace(cvode\_mem, \&lenrwDI, \&leniwDI);
}
{
  The function \ID{CVDiagGetWorkSpace} returns the
  {\cvdiag} real and integer workspace sizes.
}
{
  \begin{args}[cvode\_mem]
  \item[cvode\_mem] (\id{void *})
    pointer to the {\cvode} memory block.
  \item[lenrwDI] (\id{long int})
    the number of \id{realtype} values in the {\cvdiag} workspace.
  \item[leniwDI] (\id{long int})
    the number of integer values in the {\cvdiag} workspace.
  \end{args}
}
{
  The return value \id{flag} (of type \id{int}) is one of
  \begin{args}[CVDIAG\_LMEM\_NULL]
  \item[\Id{CVDIAG\_SUCCESS}] 
    The optional output value has been successfully set.
  \item[\Id{CVDIAG\_MEM\_NULL}]
    The \id{cvode\_mem} pointer is \id{NULL}.
  \item[\Id{CVDIAG\_LMEM\_NULL}]
    The {\cvdiag} linear solver has not been initialized.
  \end{args}
}
{
  In terms of the problem size $N$, the actual size of the real workspace
  is $3 N$ \id{realtype} words.
}
%%
\index{CVDIAG@{\cvdiag} linear solver!memory requirements|)} 
\index{memory requirements!CVDIAG@{\cvdiag} linear solver|)}
%%
%%
\ucfunction{CVDiagGetNumRhsEvals}
{
  flag = CVDiagGetNumRhsEvals(cvode\_mem, \&nfevalsDI);
}
{
  The function \ID{CVDiagGetNumRhsEvals} returns the
  number of calls to the user right-hand side function due to the 
  finite difference Jacobian approximation.
}
{
  \begin{args}[cvode\_mem]
  \item[cvode\_mem] (\id{void *})
    pointer to the {\cvode} memory block.
  \item[nfevalsDI] (\id{long int})
    the number of calls to the user right-hand side function.
  \end{args}
}
{
  The return value \id{flag} (of type \id{int}) is one of
  \begin{args}[CVDIAG\_LMEM\_NULL]
  \item[\Id{CVDIAG\_SUCCESS}] 
    The optional output value has been successfully set.
  \item[\Id{CVDIAG\_MEM\_NULL}]
    The \id{cvode\_mem} pointer is \id{NULL}.
  \item[\Id{CVDIAG\_LMEM\_NULL}]
    The {\cvdiag} linear solver has not been initialized.
  \end{args}
}
{
  The number of diagonal approximate Jacobians formed is
  equal to the number of calls to the linear solver setup function
  (available by calling \id{CVodeGetNumLinsolvSetups}).
}
%%
%%
\ucfunction{CVDiagGetLastFlag}
{
  flag = CVDiagGetLastFlag(cvode\_mem, \&flag);
}
{
  The function \ID{CVDiagGetLastFlag} returns the
  last return value from a {\cvdiag} routine. 
}
{
  \begin{args}[cvode\_mem]
  \item[cvode\_mem] (\id{void *})
    pointer to the {\cvode} memory block.
  \item[flag] (\id{int})
    the value of the last return flag from a {\cvdiag} function.
  \end{args}
}
{
  The return value \id{flag} (of type \id{int}) is one of
  \begin{args}[CVDIAG\_LMEM\_NULL]
  \item[\Id{CVDIAG\_SUCCESS}] 
    The optional output value has been successfully set.
  \item[\Id{CVDIAG\_MEM\_NULL}]
    The \id{cvode\_mem} pointer is \id{NULL}.
  \item[\Id{CVDIAG\_LMEM\_NULL}]
    The {\cvdiag} linear solver has not been initialized.
  \end{args}
}
{
  If the {\cvdiag} setup function failed (\id{CVode} returned \id{CV\_LSETUP\_FAIL}),
  the value \id{flag} is equal to \id{CVDIAG\_INV\_FAIL}, indicating that a zero
  diagonal element was encountered.
  The same value for \id{flag} is set if the {\cvdiag} solve function failed
  (\id{CVode} returned \id{CV\_LSOLVE\_FAIL}).
}
%%
%%
\index{CVDIAG@{\cvdiag} linear solver!optional output|)}
\index{optional output!diagonal linear solver|)}
%
%--------------------------------
%
\noindent{\bf SPGMR Linear solver.}
\index{optional output!iterative linear solver|(}
\index{CVSPGMR@{\cvspgmr} linear solver!optional output|(} 
%%
\index{CVSPGMR@{\cvspgmr} linear solver!memory requirements|(} 
\index{memory requirements!CVSPGMR@{\cvspgmr} linear solver|(}
%%
\ucfunction{CVSpgmrGetWorkSpace}
{
  flag = CVSpgmrGetWorkSpace(cvode\_mem, \&lenrwSG, \&leniwSG);
}
{
  The function \ID{CVSpgmrGetWorkSpace} returns the
  {\cvspgmr} real and integer workspace sizes.
}
{
  \begin{args}[cvode\_mem]
  \item[cvode\_mem] (\id{void *})
    pointer to the {\cvode} memory block.
  \item[lenrwSG] (\id{long int})
    the number of \id{realtype} values in the {\cvspgmr} workspace.
  \item[leniwSG] (\id{long int})
    the number of integer values in the {\cvspgmr} workspace.
  \end{args}
}
{
  The return value \id{flag} (of type \id{int}) is one of
  \begin{args}[CVSPGMR\_LMEM\_NULL]
  \item[\Id{CVSPGMR\_SUCCESS}] 
    The optional output value has been successfully set.
  \item[\Id{CVSPGMR\_MEM\_NULL}]
    The \id{cvode\_mem} pointer is \id{NULL}.
  \item[\Id{CVSPGMR\_LMEM\_NULL}]
    The {\cvspgmr} linear solver has not been initialized.
  \end{args}
}
{
  In terms of the problem size $N$ and maximum subspace size \id{maxl}, 
  the actual size of the real workspace is
  (\id{maxl}$+ 5)*N +$ \id{maxl} $*($ \id{maxl}$ + 4) + 1$ \id{realtype}
  words.  (In a parallel setting, this value is global --- summed over
  all processors.)
}
%%
\index{CVSPGMR@{\cvspgmr} linear solver!memory requirements|)} 
\index{memory requirements!CVSPGMR@{\cvspgmr} linear solver|)}
%%
%%
\ucfunction{CVSpgmrGetNumLinIters}
{
  flag = CVSpgmrGetNumLinIters(cvode\_mem, \&nliters);
}
{
  The function \ID{CVSpgmrGetNumLinIters} returns the
  cumulative number of linear iterations.
}
{
  \begin{args}[cvode\_mem]
  \item[cvode\_mem] (\id{void *})
    pointer to the {\cvode} memory block.
  \item[nliters] (\id{long int})
    the current number of linear iterations.
  \end{args}
}
{
  The return value \id{flag} (of type \id{int}) is one of
  \begin{args}[CVSPGMR\_LMEM\_NULL]
  \item[\Id{CVSPGMR\_SUCCESS}] 
    The optional output value has been successfully set.
  \item[\Id{CVSPGMR\_MEM\_NULL}]
    The \id{cvode\_mem} pointer is \id{NULL}.
  \item[\Id{CVSPGMR\_LMEM\_NULL}]
    The {\cvspgmr} linear solver has not been initialized.
  \end{args}
}
{}
%%
%%
\ucfunction{CVSpgmrGetNumConvFails}
{
  flag = CVSpgmrGetNumConvFails(cvode\_mem, \&nlcfails);
}
{
  The function \ID{CVSpgmrGetNumConvFails} returns the
  cumulative number of linear convergence failures.
}
{
  \begin{args}[cvode\_mem]
  \item[cvode\_mem] (\id{void *})
    pointer to the {\cvode} memory block.
  \item[nlcfails] (\id{long int})
    the current number of linear convergence failures.
  \end{args}
}
{
  The return value \id{flag} (of type \id{int}) is one of
  \begin{args}[CVSPGMR\_LMEM\_NULL]
  \item[\Id{CVSPGMR\_SUCCESS}] 
    The optional output value has been successfully set.
  \item[\Id{CVSPGMR\_MEM\_NULL}]
    The \id{cvode\_mem} pointer is \id{NULL}.
  \item[\Id{CVSPGMR\_LMEM\_NULL}]
    The {\cvspgmr} linear solver has not been initialized.
  \end{args}
}
{}
%%
%%
\ucfunction{CVSpgmrGetNumPrecEvals}
{
  flag = CVSpgmrGetNumPrecEvals(cvode\_mem, \&npevals);
}
{
  The function \ID{CVSpgmrGetNumPrecEvals} returns the
  number of preconditioner evaluations, i.e., the number of 
  calls made to \id{psetup} with \id{jok=FALSE}.
}
{
  \begin{args}[cvode\_mem]
  \item[cvode\_mem] (\id{void *})
    pointer to the {\cvode} memory block.
  \item[npevals] (\id{long int})
    the current number of calls to \id{psetup}.
  \end{args}
}
{
  The return value \id{flag} (of type \id{int}) is one of
  \begin{args}[CVSPGMR\_LMEM\_NULL]
  \item[\Id{CVSPGMR\_SUCCESS}] 
    The optional output value has been successfully set.
  \item[\Id{CVSPGMR\_MEM\_NULL}]
    The \id{cvode\_mem} pointer is \id{NULL}.
  \item[\Id{CVSPGMR\_LMEM\_NULL}]
    The {\cvspgmr} linear solver has not been initialized.
  \end{args}
}
{}
%%
%%
\ucfunction{CVSpgmrGetNumPrecSolves}
{
  flag = CVSpgmrGetNumPrecSolves(cvode\_mem, \&npsolves);
}
{
  The function \ID{CVSpgmrGetNumPrecSolves} returns the
  cumulative number of calls made to the preconditioner 
  solve function, \id{psolve}.
}
{
  \begin{args}[cvode\_mem]
  \item[cvode\_mem] (\id{void *})
    pointer to the {\cvode} memory block.
  \item[npsolves] (\id{long int})
    the current number of calls to \id{psolve}.
  \end{args}
}
{
  The return value \id{flag} (of type \id{int}) is one of
  \begin{args}[CVSPGMR\_LMEM\_NULL]
  \item[\Id{CVSPGMR\_SUCCESS}] 
    The optional output value has been successfully set.
  \item[\Id{CVSPGMR\_MEM\_NULL}]
    The \id{cvode\_mem} pointer is \id{NULL}.
  \item[\Id{CVSPGMR\_LMEM\_NULL}]
    The {\cvspgmr} linear solver has not been initialized.
  \end{args}
}
{}
%%
%%
\ucfunction{CVSpgmrGetNumJtimesEvals}
{
  flag = CVSpgmrGetNumJtimesEvals(cvode\_mem, \&njvevals);
}
{
  The function \ID{CVSpgmrGetNumJtimesEvals} returns the
  cumulative number made to the Jacobian-vector function,
  \id{jtimes}.
}
{
  \begin{args}[cvode\_mem]
  \item[cvode\_mem] (\id{void *})
    pointer to the {\cvode} memory block.
  \item[njvevals] (\id{long int})
    the current number of calls to \id{jtimes}.
  \end{args}
}
{
  The return value \id{flag} (of type \id{int}) is one of
  \begin{args}[CVSPGMR\_LMEM\_NULL]
  \item[\Id{CVSPGMR\_SUCCESS}] 
    The optional output value has been successfully set.
  \item[\Id{CVSPGMR\_MEM\_NULL}]
    The \id{cvode\_mem} pointer is \id{NULL}.
  \item[\Id{CVSPGMR\_LMEM\_NULL}]
    The {\cvspgmr} linear solver has not been initialized.
  \end{args}
}
{}
%%
%%
\ucfunction{CVSpgmrGetNumRhsEvals}
{
  flag = CVSpgmrGetNumRhsEvals(cvode\_mem, \&nfevalsSG);
}
{
  The function \ID{CVSpgmrGetNumRhsEvals} returns the
  number of calls to the user right-hand side function for
  finite difference Jacobian-vector product approximation.
}
{
  \begin{args}[cvode\_mem]
  \item[cvode\_mem] (\id{void *})
    pointer to the {\cvode} memory block.
  \item[nfevalsSG] (\id{long int})
    the number of calls to the user right-hand side function.
  \end{args}
}
{
  The return value \id{flag} (of type \id{int}) is one of
  \begin{args}[CVSPGMR\_LMEM\_NULL]
  \item[\Id{CVSPGMR\_SUCCESS}] 
    The optional output value has been successfully set.
  \item[\Id{CVSPGMR\_MEM\_NULL}]
    The \id{cvode\_mem} pointer is \id{NULL}.
  \item[\Id{CVSPGMR\_LMEM\_NULL}]
    The {\cvspgmr} linear solver has not been initialized.
  \end{args}
}
{
  The value \id{nfevalsSG} is incremented only if the default 
  \id{CVSpgmrDQJtimes} difference quotient function is used.
}
%%
%%
\ucfunction{CVSpgmrGetLastFlag}
{
  flag = CVSpgmrGetLastFlag(cvode\_mem, \&flag);
}
{
  The function \ID{CVSpgmrGetLastFlag} returns the
  last return value from a {\cvspgmr} routine. 
}
{
  \begin{args}[cvode\_mem]
  \item[cvode\_mem] (\id{void *})
    pointer to the {\cvode} memory block.
  \item[flag] (\id{int})
    the value of the last return flag from a {\cvspgmr} function.
  \end{args}
}
{
  The return value \id{flag} (of type \id{int}) is one of
  \begin{args}[CVSPGMR\_LMEM\_NULL]
  \item[\Id{CVSPGMR\_SUCCESS}] 
    The optional output value has been successfully set.
  \item[\Id{CVSPGMR\_MEM\_NULL}]
    The \id{cvode\_mem} pointer is \id{NULL}.
  \item[\Id{CVSPGMR\_LMEM\_NULL}]
    The {\cvspgmr} linear solver has not been initialized.
  \end{args}
}
{
  If the {\cvspgmr} setup function failed (\id{CVode} returned
  \id{CV\_LSETUP\_FAIL}), \id{flag} contains the return value of the
  preconditioner setup function \id{psetup}.

  If the {\cvspgmr} solve function failed (\id{CVode} returned
  \id{CV\_LSETUP\_FAIL}), \id{flag} contains the error return flag from
  \id{SpgmrSolve} and will be one of:
 \id{SPGMR\_CONV\_FAIL}, indicating a failure to converge;
 \id{SPGMR\_QRFACT\_FAIL}, indicating a singular matrix found during the QR
  factorization;
 \id{SPGMR\_PSOLVE\_FAIL\_REC}, indicating that the preconditioner solve function
 \id{psolve} failed recoverably;
  \id{SPGMR\_MEM\_NULL}, indicating that the {\spgmr} memory is \id{NULL};
  \id{SPGMR\_ATIMES\_FAIL}, indicating a failure in the Jacobian times vector
  function;
  \id{SPGMR\_PSOLVE\_FAIL\_UNREC}, indicating that the preconditioner solve
  function \id{psolve} failed unrecoverably;
  \id{SPGMR\_GS\_FAIL}, indicating a failure in the Gram-Schmidt procedure; 
  or \id{SPGMR\_QRSOL\_FAIL}, indicating that the matrix $R$ was found to be
  singular during the QR solve phase.
}
%%
%%
\index{CVSPGMR@{\cvspgmr} linear solver!optional output|)} 
\index{optional output!iterative linear solver|)}

%%==============================================================================
\subsection{CVODE reinitialization function}\label{sss:cvreinit}
%%==============================================================================
\index{reinitialization}

The function \ID{CVodeReInit} reinitializes the main {\cvode} solver for
the solution of a problem, where a prior call to \Id{CVodeMalloc} has
been made. The new problem must have the same size as the previous one.
\id{CVodeReInit} performs the same input checking and initializations 
that \id{CVodeMalloc} does, but does no memory allocation, assuming that the 
existing internal memory is sufficient for the new problem.             
                                                                 
The use of \id{CVodeReInit} requires that the maximum method order,    
\Id{maxord}, is no larger for the new problem than for the problem  
specified in the last call to \id{CVodeMalloc}.  This condition is  
automatically fulfilled if the multistep method parameter \Id{lmm}  
is unchanged (or changed from \Id{CV\_ADAMS} to \Id{CV\_BDF}) and the default    
value for \id{maxord} is specified.

If there are changes to the linear solver specifications, make the
appropriate \id{Set} calls, as described in \S\ref{sss:lin_solv_init}

%%
%%
\ucfunction{CVodeReInit}
{
  flag = CVodeReInit(cvode\_mem, f, t0, y0, itol, reltol, abstol);
}
{
  The function \id{CVodeReInit} provides required problem specifications 
  and reinitializes {\cvode}.
}
{
  \begin{args}[cvode\_mem]
  \item[cvode\_mem] (\id{void *})
    pointer to the {\cvode} memory block.
  \item[f] (\Id{CVRhsFn})
    is the {\C} function which computes $f$ in the ODE. This function has the form 
    \id{f(N, t, y, ydot, f\_data)} (for full details see \S\ref{ss:user_fct_sim}).
  \item[t0] (\id{realtype})
    is the initial value of $t$.
  \item[y0] (\id{N\_Vector})
    is the initial value of $y$. 
  \item[itol] (\id{int}) 
    is either \Id{CV\_SS} or \Id{CV\_SV}, where \Id{itol}$=$\id{SS} indicates
    scalar relative error tolerance and scalar absolute error tolerance, while
    \id{itol}$=$\id{CV\_SV} indicates scalar relative error tolerance and vector
    absolute error tolerance.  The latter choice is important when the absolute
    error tolerance needs to be different for each component of the ODE. 
  \item[reltol] (\id{realtype *})
    is a pointer to the relative error tolerance.
  \item[abstol] (\id{void *})
    is a pointer to the absolute error tolerance.
  \end{args}
}
{
  The return flag \id{flag} (of type \id{int}) will be one of the following:
  \begin{args}[CV\_NO\_MALLOC]
  \item[\Id{CV\_SUCCESS}]
    The call to \id{CVodeReInit} was successful.
  \item[\Id{CV\_MEM\_NULL}] 
    The {\cvode} memory block was not initialized through a 
    previous call to \id{CVodeCreate}.
  \item[\Id{CV\_NO\_MALLOC}] 
    Memory space for the {\cvode} memory block was not allocated through a 
    previous call to \id{CVodeMalloc}.
  \item[\Id{CV\_ILL\_INPUT}] 
    An input argument to \id{CVodeReInit} has an illegal value.
  \end{args}
}
{
  If an error occurred, \id{CVodeReInit} also prints an error message to the
  file specified by the optional input \id{errfp}.
}

%%==============================================================================
\section{User-supplied functions}\label{ss:user_fct_sim}
%%==============================================================================

The user-supplied functions consist of one function defining the ODE, 
(optionally) a function that provides Jacobian related information for the linear
solver (if Newton iteration is chosen), and (optionally) one or two functions 
that define the preconditioner for use in the {\spgmr} algorithm. 

%%==============================================================================
\subsection{ODE right-hand side} \label{ss:rhsFn}
%%==============================================================================
\index{right-hand side function}

The user must provide a function of type \ID{CVRhsFn} defined as follows:
\usfunction{CVRhsFn}
{
  typedef void (*CVRhsFn)(&realtype t, N\_Vector y, N\_Vector ydot, \\
                          &void *f\_data);
}
{
  This function computes the ODE right-hand side for a given value
  of the independent variable $t$ and state vector $y$.
}
{
  \begin{args}[f\_data]
  \item[t]
    is the current value of the independent variable.
  \item[y]
    is the current value of the dependent variable vector, $y(t)$.
  \item[ydot]
    is the output vector $f(t,y)$.
  \item[f\_data]
    is a pointer to user data --- the same as the \Id{f\_data}      
    parameter passed to \id{CVodeSetFdata}.   
  \end{args}
}
{
  A \id{CVRhsFn} function type does not have a return value.
}
{
Allocation of memory for \id{ydot} is handled within {\cvode}.
}

\index{right-hand side function}

%%==============================================================================
\subsection{Jacobian information (direct method with dense Jacobian)}
\label{ss:djacFn}
%%==============================================================================
\index{Jacobian approximation function!dense!user-supplied|(}

If the direct linear solver with dense treatment of the Jacobian is used 
(i.e. \Id{CVDense} is called in Step \ref{i:lin_solver} of \S\ref{ss:skeleton_sim}), 
the user may provide a function of type \ID{CVDenseJacFn} defined by
\usfunction{CVDenseJacFn}
{
  typedef void (*CVDenseJacFn)(&long int N, DenseMat J, realtype t, \\
                               &N\_Vector y, N\_Vector fy, void *jac\_data, \\
                               &N\_Vector tmp1, N\_Vector tmp2, N\_Vector tmp3);
}
{
  This function computes the dense Jacobian $J = \partial f / \partial y$ 
  (or an approximation to it).
}
{
  \begin{args}[jac\_data]
  \item[N]
    is the problem size.
  \item[J]
    is the output Jacobian matrix.  
  \item[t]
    is the current value of the independent variable.
  \item[y]
    is the current value of the dependent variable vector, 
    namely the predicted value of $y(t)$.
  \item[fy]
    is the vector $f(t,y)$.
  \item[jac\_data]
    is a pointer to user data --- the same as the \id{jac\_data}      
    parameter passed to \id{CVDenseSetJacData}.   
  \item[tmp1]
  \item[tmp2]
  \item[tmp3]
    are pointers to memory allocated    
    for variables of type \id{N\_Vector} which can be used by           
    \id{CVDenseJacFn} as temporary storage or work space.    
  \end{args}
}
{
  A \id{CVDenseJacFn} function type does not have a return value.
}
{
  A user-supplied dense Jacobian function must load the \id{N} by \id{N}
  dense matrix \id{J} with an approximation to the Jacobian matrix $J$
  at the point (\id{t}, \id{y}).  Only nonzero elements need to be loaded
  into \id{J} because \id{J} is set to the zero matrix before the call
  to the Jacobian function. The type of \id{J} is \Id{DenseMat}. 
  
  The accessor macros \Id{DENSE\_ELEM} and \Id{DENSE\_COL} allow the user to
  read and write dense matrix elements without making explicit
  references to the underlying representation of the \id{DenseMat}
  type. \id{DENSE\_ELEM(J, i, j)} references the (\id{i}, \id{j})-th
  element of the dense matrix \id{J} (\id{i}, \id{j}$= 0\ldots N-1$). This macro
  is for use in small problems in which efficiency of access is not a major
  concern.  Thus, in terms of indices $m$ and $n$ running from $1$ to
  $N$, the Jacobian element $J_{m,n}$ can be loaded with the statement
  \id{DENSE\_ELEM(J, m-1, n-1) =} $J_{m,n}$.  Alternatively,
  \id{DENSE\_COL(J, j)} returns a pointer to the storage for
  the \id{j}th column of \id{J} (\id{j}$= 0\ldots N-1$), and the 
  elements of the \id{j}th column
  are then accessed via ordinary array indexing.  Thus $J_{m,n}$ can be 
  loaded with the statements \id{col\_n = DENSE\_COL(J, n-1);}
  \id{col\_n[m-1] =} $J_{m,n}$.  For large problems, it is more 
  efficient to use \id{DENSE\_COL} than to use \id{DENSE\_ELEM}. 
  Note that both of these macros number rows and columns
  starting from $0$, not $1$.  

  The \id{DenseMat} type and the accessor macros \id{DENSE\_ELEM} and 
  \id{DENSE\_COL} are documented in \S\ref{ss:dense}.

  If the user's \id{CVDenseJacFn} function uses difference quotient
  approximations, it may need to access quantities not in the call
  list. These include the current stepsize, the error weights, etc.
  To obtain these, use the \id{CVodeGet*} functions described in
  \S\ref{ss:optional_output}. The unit roundoff can be accessed
  as \id{UNIT\_ROUNDOFF} defined in \id{sundialstypes.h}.
}
\index{Jacobian approximation function!dense!user-supplied|)}

%%==============================================================================
\subsection{Jacobian information (direct method with banded Jacobian)}
\label{ss:bjacFn}
%%==============================================================================
\index{Jacobian approximation function!band!user-supplied|(}
\index{half-bandwidths|(}

If the direct linear solver with banded treatment of the Jacobian is used 
(i.e. \Id{CVBand} is called in Step \ref{i:lin_solver} of \S\ref{ss:skeleton_sim}), 
the user may provide a function of type \ID{CVBandJacFn} defined as follows:
\usfunction{CVBandJacFn}
{
 typedef void (*CVBandJacFn)(&long int N, long int mupper, \\
                             &long int mlower, BandMat J, realtype t, \\ 
                             &N\_Vector y, N\_Vector fy, void *jac\_data, \\
                             &N\_Vector tmp1, N\_Vector tmp2, N\_Vector tmp3);
}
{
  This function computes the banded Jacobian $J = \partial f / \partial y$ 
  (or a banded approximation to it).
}
{
  \begin{args}[jac\_data]
  \item[N]
    is the problem size.
  \item[mlower]
  \item[mupper]
    are the lower and upper half-bandwidths of the Jacobian.
  \item[J]
    is the output Jacobian matrix.  
  \item[t]
    is the current value of the independent variable.
  \item[y]
    is the current value of the dependent variable vector, 
    namely the predicted value of $y(t)$.
  \item[fy]
    is the vector $f(t,y)$.
  \item[jac\_data]
    is a pointer to user data --- the same as the \id{jac\_data}      
    parameter passed to \id{CVBandSetJacData}.   
  \item[tmp1]
  \item[tmp2]
  \item[tmp3]
    are pointers to memory allocated    
    for variables of type \id{N\_Vector} which can be used by           
    \id{CVBandJacFn} as temporary storage or work space.    
  \end{args}
}
{
  A \id{CVBandJacFn} function type does not have a return value.
}
{
  A user-supplied band Jacobian function must load the band matrix \id{J}
  of type \Id{BandMat} with the elements of the Jacobian $J(t,y)$ at the
  point (\id{t},\id{y}).  Only nonzero elements need to be loaded into
  \id{J} because \id{J} is preset to zero before the call to the
  Jacobian function.  

  The accessor macros \Id{BAND\_ELEM}, \Id{BAND\_COL}, and \Id{BAND\_COL\_ELEM} 
  allow the user to read and write band matrix elements without making specific 
  references to the underlying representation of the \id{BandMat} type.
  \id{BAND\_ELEM(J, i, j)} references the (\id{i}, \id{j})th element of the 
  band matrix \id{J}, counting from $0$.
  This macro is for use in small problems in which efficiency of access is not
  a major concern.  Thus, in terms of indices $m$ and $n$ running from $1$ to
  $N$ with $(m,n)$ within the band defined by \id{mupper} and
  \id{mlower}, the Jacobian element $J_{m,n}$ can be loaded with the 
  statement \id{BAND\_ELEM(J, m-1, n-1) =} $J_{m,n}$. The elements within
  the band are those with \id{-mupper} $\le$ \id{m-n} $\le$ \id{mlower}.
  Alternatively, \id{BAND\_COL(J, j)} returns a pointer to the diagonal element
  of the \id{j}th column of \id{J}, and if we assign this address to 
  \id{realtype *col\_j}, then the \id{i}th element of the \id{j}th column is
  given by \id{BAND\_COL\_ELEM(col\_j, i, j)}, counting from $0$.
  Thus for $(m,n)$ within the band, $J_{m,n}$ can be loaded by setting 
  \id{col\_n = BAND\_COL(J, n-1);} \id{BAND\_COL\_ELEM(col\_n, m-1, n-1) =}
  $J_{m,n}$.  The elements of the \id{j}th column can also be accessed
  via ordinary array indexing, but this approach requires knowledge of
  the underlying storage for a band matrix of type \id{BandMat}.  
  The array \id{col\_n} can be indexed from $-$\id{mupper} to \id{mlower}.
  For large problems, it is more efficient to use the combination of
  \id{BAND\_COL} and \id{BAND\_COL\_ELEM} than to use the
  \id{BAND\_ELEM}.  As in the dense case, these macros all number rows
  and columns starting from $0$, not $1$.  

  The \id{BandMat} type and the accessor macros \id{BAND\_ELEM}, \id{BAND\_COL},
  and \id{BAND\_COL\_ELEM} are documented in \S\ref{ss:band}.

  If the user's \id{CVBandJacFn} function uses difference quotient approximations,
  it may need to access quantities not in the call list. These include the current
  stepsize, the error weights, etc. To obtain these, use the \id{CVodeGet*}
  functions described in \S\ref{ss:optional_output}. The unit roundoff can be
  accessed as \id{UNIT\_ROUNDOFF} defined in \id{sundialstypes.h}.
}
\index{half-bandwidths|)}
\index{Jacobian approximation function!band!user-supplied|)}

%%==============================================================================
\subsection{Jacobian information (SPGMR matrix-vector product)}
\label{ss:jtimesFn}
%%==============================================================================
\index{Jacobian approximation function!Jacobian times vector!user-supplied|(}

If an iterative {\spgmr} linear solver is selected (\id{CVSpgmr} is called in step 
\ref{i:lin_solver} of \S\ref{ss:skeleton_sim}) the user may provide a function
of type \ID{CVSpgmrJacTimesVecFn} in the following form:
\usfunction{CVSpgmrJacTimesVecFn}
{
  typedef int (*CVSpgmrJacTimesVecFn)(&N\_Vector v, N\_Vector Jv, realtype t, \\
                                      &N\_Vector y, N\_Vector fy, \\
                                      &void *jac\_data, N\_Vector tmp);
}
{
  This function computes the product $J v = (\partial f / \partial y) v$ 
  (or an approximation to it).
}
{
  \begin{args}[jac\_data]
  \item[v]
    is the vector by which the Jacobian must be multiplied to the right.
  \item[Jv]
      is the output vector computed.
  \item[t]
    is the current value of the independent variable.       
  \item[y] 
    is the current value of the dependent variable vector. 
  \item[fy]
    is the vector $f(t,y)$.
  \item[jac\_data]
    is a pointer to user data --- the same as the \id{jac\_data}      
    parameter passed to \id{CVSpgmrSetJacData}.   
  \item[tmp]
    is a pointer to memory allocated for a variable of type \id{N\_Vector}
    which can be used for work space.
  \end{args}
}
{  
  The value to be returned by the Jacobian times vector function should be
  $0$ if successful. Any other return value will result in an unrecoverable
  error of the {\spgmr} generic solver, in which case the integration is halted.
}
{
  If the user's \id{CVSpgmrJacTimesVecFn} function uses difference quotient
  approximations, it may need to access quantities not in the call
  list. These include the current stepsize, the error weights, etc.
  To obtain these, use the \id{CVodeGet*} functions described in
  \S\ref{ss:optional_output}. The unit roundoff can be accessed
  as \id{UNIT\_ROUNDOFF} defined in \id{sundialstypes.h}.
}
\index{Jacobian approximation function!Jacobian times vector!user-supplied|)}

%%==============================================================================
\subsection{Preconditioning (SPGMR linear system solution)} \label{ss:psolveFn}
%%==============================================================================
\index{preconditioning!user-supplied}
\index{CVSPGMR@{\cvspgmr} linear solver!preconditioner solve function}

If preconditioning is used, then the user must provide a {\C} function to
solve the linear system $Pz = r$ where $P$ may be either a left or a
right preconditioner matrix.
This function must be of type \ID{CVSpgmrPrecSolveFn}, defined as follows:
%%
%%
\usfunction{CVSpgmrPrecSolveFn}
{
  typedef int (*CVSpgmrPrecSolveFn)(&realtype t, N\_Vector y, N\_Vector fy, \\
                                    &N\_Vector r, N\_Vector z, \\ 
                                    &realtype gamma, realtype delta, \\
                                    &int lr, void *P\_data, N\_Vector tmp);
}
{
  This function solves the preconditioning system $Pz = r$.
}
{  
  \begin{args}[P\_data]
  \item[t]
    is the current value of the independent variable.
  \item[y] 
    is the current value of the dependent variable vector.  
  \item[fy]
    is the vector $f(t,y)$.
  \item[r]
    is the right-hand side vector of the linear system.
  \item[z]
    is the output vector computed.
  \item[gamma]
    is the scalar $\gamma$ appearing in the Newton matrix $M=I-\gamma J$.
  \item[delta]
    is an input tolerance to be used if an iterative method 
    is employed in the solution.  In that case, the residual 
    vector $Res = r - P z$ of the system should be made less than 
    \id{delta} in weighted $l_2$ norm,     
    i.e., $\sqrt{\sum_i (Res_i \cdot ewt_i)^2 } < delta$.
    To obtain the \id{N\_Vector} \id{ewt}, call \id{CVodeGetErrWeights} 
    (see \S\ref{ss:optional_output}).
  \item[lr]
    is an input flag indicating whether the preconditioner solve
    function is to use the left preconditioner (\id{lr=1}) or 
    the right preconditioner (\id{lr=2});
  \item[P\_data]
    is a pointer to user data --- the same as the \id{P\_data}      
    parameter passed to the function \id{CVSpgmrSetPrecData}.
  \item[tmp]
    is a pointer to memory allocated for a variable of type \id{N\_Vector}
    which can be used for work space.
  \end{args}
}
{
  The value to be returned by the preconditioner solve function is a flag
  indicating whether it was successful.  This value should be $0$ if successful, 
  positive for a recoverable error (in which case the step will be retried),     
  negative for an unrecoverable error (in which case the integration is halted). 
}
{}

%%==============================================================================
\subsection{Preconditioning (SPGMR Jacobian data)}\label{ss:precondFn}
%%==============================================================================
\index{preconditioning!user-supplied}
\index{CVSPGMR@{\cvspgmr} linear solver!preconditioner setup function}

If the user's preconditioner requires that any Jacobian related data
be evaluated or preprocessed, then this needs to be done in a
user-supplied {\C} function of type \ID{CVSpgmrPrecSetupFn}, defined as follows:
\usfunction{CVSpgmrPrecSetupFn}
{
  typedef int (*CVSpgmrPrecSetupFn&)(realtype t, N\_Vector y, N\_Vector fy,  \\
                                  &booleantype jok, booleantype *jcurPtr, \\
                                  &realtype gamma, void *P\_data,\\
                                  &N\_Vector tmp1, N\_Vector tmp2,\\
                                  &N\_Vector tmp3);
}
{
  This function evaluates and/or preprocesses Jacobian related data needed
  by the preconditioner.
}
{
  The arguments of a \id{CVSpgmrPrecSetupFn} are as follows:
  \begin{args}[jcurPtr]
  \item[t]
    is the current value of the independent variable.
  \item[y]
    is the current value of the dependent variable vector, 
    namely the predicted value of $y(t)$.
  \item[fy]
    is the vector $f(t,y)$.                    
  \item[jok]
    is an input flag indicating whether Jacobian-related   
    data needs to be recomputed. The \id{jok} argument provides for 
    the re-use of Jacobian data in the preconditioner solve function.
    \id{jok == FALSE} means that Jacobian-related data   
    must be recomputed from scratch.                                 
    \id{jok == TRUE}  means that Jacobian data, if saved from 
    the previous call to this function, can be reused      
    (with the current value of \id{gamma}).            
    A call with \id{jok == TRUE} can only occur after   
    a call with \id{jok == FALSE}.
  \item[jcurPtr]
    is a pointer to an output integer flag which is        
    to be set to \id{TRUE} if Jacobian data was recomputed or   
    to \id{FALSE} if Jacobian data was not           
    recomputed, but saved data was reused.
  \item[gamma]
    is the scalar $\gamma$ appearing in the Newton matrix $M = I - \gamma P$.
  \item[P\_data]
    is a pointer to user data, the same as the \id{P\_data}      
    parameter passed to \id{CVSpgmrSetPrecData}.
  \item[tmp1]
  \item[tmp2]
  \item[tmp3]
    are pointers to memory allocated    
    for variables of type \id{N\_Vector} which can be used by           
    \id{CVSpgmrPrecSetupFn} as temporary storage or work space.    
  \end{args}
}
{
  The value to be returned by the preconditioner setup function is a flag
  indicating whether it was successful.  This value should be $0$ if successful, 
  positive for a recoverable error (in which case the step will be retried),     
  negative for an unrecoverable error (in which case the integration is halted). 
}
{
  The operations performed by this function might include forming a crude 
  approximate Jacobian, and performing an LU factorization on the resulting
  approximation to $M=I - \gamma J$.

  Each call to the preconditioner setup function is preceded by a call to     
  the \id{CVRhsFn} user function with the same \id{(t,y)} arguments.  
  Thus the preconditioner setup function can use any auxiliary data that is 
  computed and saved during the evaluation of the ODE right hand side.
  
  This function is not called in advance of every call to the preconditioner
  solve function, but rather is called only as often as needed to achieve
  convergence in the Newton iteration. 

  If the user's \id{CVSpgmrPrecSetupFn} function uses difference quotient
  approximations, it may need to access quantities not in the call
  list. These include the current stepsize, the error weights, etc.
  To obtain these, use the \id{CVodeGet*} functions described in
  \S\ref{ss:optional_output}. The unit roundoff can be accessed
  as \id{UNIT\_ROUNDOFF} defined in \id{sundialstypes.h}.
}

%%==============================================================================
\section{Rootfinding}\label{s:using_rootfinding}
%%==============================================================================
\index{Rootfinding}

While integrating the IVP, {\cvode} has the capability of finding the
roots of a set of user-defined functions. This section describes the
user-callable functions used to initialize and define the rootfinding
problem and obtain solution information, and it also describes the
required additional user-supplied function.

%%==============================================================================
\subsection{User-callable functions for rootfinding}\label{ss:root_uc}
%%==============================================================================

\ucfunction{CVodeRootInit}
{
  flag = CVodeRootInit(cvode\_mem, g, nrtfn);
}
{
  The function \ID{CVodeRootInit} specifies that the roots of a set of
  functions $g_i(t,y)$ are to be found while the IVP is being solved.
}
{
  \begin{args}[cvode\_mem]
  \item[cvode\_mem] (\id{void *})
    pointer to the {\cvode} memory block returned by \id{CVodeCreate}.
  \item[g] (\id{CVRootFn})
    is the {\C} function which defines the \id{nrtfn} functions $g_i(t,y)$
    whose roots are sought. See \S\ref{ss:root_us} for details.
  \item[nrtfn] (\id{int})
    is the number of root functions $g_i$.
  \end{args}
}
{
  The return value \id{flag} (of type \id{int}) is one of
  \begin{args}[CV\_MEM\_FAIL]
  \item[CV\_SUCCESS]
    The call to \id{CVodeRootInit} was successful.
  \item[CV\_MEM\_NULL]
    The \id{cvode\_mem} argument was \id{NULL}.
  \item[CV\_MEM\_FAIL]
    A memory allocation failed.
  \end{args}
}
{
  If a new IVP is to be solved with a call to \id{CVodeReInit}, where the new
  IVP has no rootfinding problem but the prior one did, then call
  \id{CVodeRootInit} with \id{nrtfn}$=0$.
}
%%
%%
There is one optional input function associated with rootfinding. 
%%
\ucfunction{CVodeSetGdata}
{
  flag = CVodeSetGdata(cvode\_mem, g\_data);
}
{
  The function \ID{CVodeSetGdata} specifies the user data block \ID{g\_data}
  for use by the user's root function $g$.
}
{
  \begin{args}[cvode\_mem]
  \item[cvode\_mem] (\id{void *})
    pointer to the {\cvode} memory block.
  \item[g\_data] (\id{void *})
    pointer to the user data.
  \end{args}
}
{
  The return value \id{flag} (of type \id{int}) is one of
  \begin{args}[CV\_MEM\_NULL]
  \item[\Id{CV\_SUCCESS}] 
    The optional value has been successfully set.
  \item[\Id{CV\_MEM\_NULL}]
    The \id{cvode\_mem} pointer is \id{NULL}.
  \end{args}
}
{
  If \id{g\_data} is not specified, a \id{NULL} pointer is
  passed to the $g$ function.
}
%%
There are two optional output functions associated with rootfinding.
%%
%%
\ucfunction{CVodeGetRootInfo}
{
  flag = CVodeGetRootInfo(cvode\_mem, \&rootsfound);
}
{
  The function \ID{CVodeGetRootInfo} returns an array showing which 
  functions were found to have a root.
}
{
  \begin{args}[cvode\_mem]
  \item[cvode\_mem] (\id{void *})
    pointer to the {\cvode} memory block.
  \item[rootsfound] (\id{int *})
    an \id{int} array of length \id{nrtfn}, showing the indices
    of the user functions $g_i$ found to have a root.  For
    $i=0,\ldots,$\id{nrtfn}$-1$, \id{rootsfound}[$i$]$=1$ if $g_i$
    has a root, and $=0$ if not.
  \end{args}
}
{
  The return value \id{flag} (of type \id{int}) is one of
  \begin{args}[CV\_MEM\_NULL]
  \item[\Id{CV\_SUCCESS}] 
    The optional output values have been successfully set.
  \item[\Id{CV\_MEM\_NULL}]
    The \id{cvode\_mem} pointer is \id{NULL}.
  \end{args}
}
{}
%%
%%
\ucfunction{CVodeGetNumGEvals}
{
  flag = CVodeGetNumGEvals(cvode\_mem, \&ngevals);
}
{
  The function \ID{CVodeGetNumGEvals} returns the cumulative
  number of calls to the user root function $g$.
}
{
  \begin{args}[cvode\_mem]
  \item[cvode\_mem] (\id{void *})
    pointer to the {\cvode} memory block.
  \item[ngevals] (\id{long int})
    number of calls to the user's function \id{g} so far.
  \end{args}
}
{
  The return value \id{flag} (of type \id{int}) is one of
  \begin{args}[CV\_MEM\_NULL]
  \item[\Id{CV\_SUCCESS}] 
    The optional output value has been successfully set.
  \item[\Id{CV\_MEM\_NULL}]
    The \id{cvode\_mem} pointer is \id{NULL}.
  \end{args}
}
{}

%%==============================================================================
\subsection{User-supplied function for rootfinding}\label{ss:root_us}
%%==============================================================================

If a rootfinding problem is to be solved during the integration of the ODE system,
the user must supply a {\C} function of type \ID{CVRootFn}, defined as follows:
%%
\usfunction{CVRootFn}
{
  typedef void (*CVRootFn)(&realtype t, N\_Vector y, realtype *gout, \\
                           &void *g\_data);
}
{
  This function computes a vector-valued function $g(t,y)$ such that the roots of
  the \id{nrtfn} components $g_i(t,y)$ are to be found during the integration.
}
{
  \begin{args}[g\_data]
  \item[t]
    is the current value of the independent variable.
  \item[y]
    is the current value of the dependent variable vector, $y(t)$.
  \item[gout]
    is the output array, of length \id{nrtfn}, with components $g_i(t,y)$.
  \item[g\_data]
    is a pointer to user data --- the same as the \Id{g\_data}      
    parameter passed to \id{CVodeSetGdata}.   
  \end{args}
}
{
  A \id{CVRootFn} function type does not have a return value.
}
{
  Allocation of memory for \id{gout} is handled within {\cvode}.
}

%%==============================================================================
\section{Preconditioner modules}\label{ss:preconds}
%%==============================================================================

The efficiency of Krylov iterative methods for the solution of linear systems 
can be greatly enhanced through preconditioning. For problems in which the 
user cannot define a more effective, problem-specific preconditioner,
{\cvode} provides a banded preconditioner in the module {\cvbandpre} and
a band-block-diagonal preconditioner module {\cvbbdpre}.

%%==============================================================================
\subsection{A serial banded preconditioner module}\label{sss:cvbandpre}
%%==============================================================================

\index{CVBANDPRE@{\cvbandpre} preconditioner!description}
\index{preconditioning!banded}

This preconditioner provides a band matrix preconditioner based on
difference quotients of the ODE right-hand side function \id{f}.
It generates a band matrix of bandwidth $m_l + m_u + 1$, where
the number of super-diagonals ($m_u$, the upper half-bandwidth) and
sub-diagonals ($m_l$, the lower half-bandwidth) are specified by
the user and uses this to form a preconditioner for use with the Krylov
linear solver in {\cvspgmr}.  Although this matrix is intended
to approximate the Jacobian $\partial f / \partial y$, 
it may be a very crude approximation.  The true Jacobian need not be banded,
or its true bandwidth may be larger than $m_l + m_u + 1$, as long as the
banded approximation generated here is sufficiently accurate
to speed convergence as a preconditioner. 

\index{CVBANDPRE@{\cvbandpre} preconditioner!usage|(}
In order to use the {\cvbandpre} module, the user need not define any
additional functions. 
%%
Besides the header files required for the integration of the ODE problem
(see \S\ref{ss:header_sim}),  to use the {\cvbandpre} module, the main program 
must include the header file \id{cvbandpre.h} which declares the needed
function prototypes.\index{header files}
%%
The following is a summary of the usage of this module and describes the sequence
of calls in the user main program. Steps that are unchanged from the user main
program presented in \S\ref{ss:skeleton_sim} are grayed-out.
%%
%%
\index{User main program!CVBANDPRE@{\cvbandpre} usage}
\begin{Steps}
  
\item
  \textcolor{gray}{\bf Set problem dimensions}

\item
  \textcolor{gray}{\bf Set vector of initial values}
 
\item
  \textcolor{gray}{\bf Create {\cvode} object}

\item
  \textcolor{gray}{\bf Set optional inputs}

\item
  \textcolor{gray}{\bf Allocate internal memory}

\item \label{i:bandpre_init}
  {\bf Initialize the {\cvbandpre} preconditioner module}

  Specify the upper and lower half-bandwidths \id{mu} and \id{ml} and call 

  \id{bp\_data = CVBandPrecAlloc(cvode\_mem, N, mu, ml);} 

  to allocate memory for and initialize a data structure \id{bp\_data} to be 
  passed to the {\cvspgmr} linear solver.

\item \label{i:bandpre_attach}
  {\bf Attach the {\cvspgmr} linear solver}

  \id{flag = CVBPSpgmr(cvode\_mem, pretype, maxl, bp\_data);}

  The function \Id{CVBPSpgmr} is a wrapper around the {\cvspgmr} specification
  function \id{CVSpgmr} and performs the following actions:
  \begin{itemize}
    \item Attaches the {\cvspgmr} linear solver to the main {\cvode} solver memory;
    \item Sets the preconditioner data structure for {\cvbandpre};
    \item Sets the preconditioner setup function for {\cvbandpre};
    \item Sets the preconditioner solve function for {\cvbandpre};
  \end{itemize}
  The arguments \id{pretype} and \id{maxl} are described below.
  The last argument of \id{CVBPSpgmr} is the pointer to the {\cvbandpre} data
  returned by \id{CVBandPrecAlloc}.

\item
  \textcolor{gray}{\bf Set linear solver optional inputs}

  Note that the user should not overwrite the preconditioner data, setup function,
  or solve function through calls to {\cvspgmr} optional input functions.

\item
  \textcolor{gray}{\bf Advance solution in time}

\item
  \textcolor{gray}{\bf Deallocate memory for solution vector}

\item \label{i:bandpre_free}
  {\bf Free the {\cvbandpre} data structure}

  \id{CVBandPrecFree(bp\_data);}

\item
  \textcolor{gray}{\bf Free solver memory}
  
\end{Steps}
%%
%%
\index{CVBANDPRE@{\cvbandpre} preconditioner!usage|)}

\index{CVBANDPRE@{\cvbandpre} preconditioner!user-callable functions|(}
The three user-callable functions that initialize, attach, and deallocate
the {\cvbandpre} preconditioner module (steps \ref{i:bandpre_init},
\ref{i:bandpre_attach}, and \ref{i:bandpre_free} above) are described
in more detail below.
%%
\index{half-bandwidths}
\ucfunction{CVBandPrecAlloc}
{
  bp\_data = CVBandPrecAlloc(cvode\_mem, N, mu, ml);
}
{
  The function \ID{CVBandPrecAlloc} initializes and allocates
  memory for the {\cvbandpre} preconditioner.
}
{
  \begin{args}[cvode\_mem]
  \item[cvode\_mem] (\id{void *})
    pointer to the {\cvode} memory block.
  \item[N] (\id{long int})
    problem dimension.
  \item[mu] (\id{long int})
    upper half-bandwidth of the problem Jacobian approximation.
  \item[ml] (\id{long int})
    lower half-bandwidth of the problem Jacobian approximation.
  \end{args}
}
{
  If successful, \id{CVBandPrecAlloc} returns a pointer to the newly created 
  {\cvbandpre} memory block (of type \id{void *}).
  If an error occurred, \id{CVBandPrecAlloc} returns \id{NULL}.
}
{
  The banded approximate Jacobian will have its nonzeros only in locations
  $(i,j)$ with $-$\id{ml} $\leq j-i \leq$ \id{mu}.
}
%%
%%
\ucfunction{CVBPSpgmr}
{
  flag = CVBPSpgmr(cvode\_mem, pretype, maxl, bp\_data);
}
{
  The function \ID{CVBPSpgmr} links the {\cvbandpre} data to the
  {\cvspgmr} linear solver and attaches the latter to the {\cvode}
  memory block.
}
{
  \begin{args}[cvode\_mem]
  \item[cvode\_mem] (\id{void *})
    pointer to the {\cvode} memory block.
  \item[pretype] (\id{int})
    \index{pretype@\texttt{pretype}}
    preconditioning type. Must be one of \Id{PREC\_LEFT} or \Id{PREC\_RIGHT}.
  \item[maxl] (\id{int})
    \index{maxl@\texttt{maxl}}
    maximum dimension of the Krylov subspace to be used. Pass $0$ to use the 
    default value \id{CVSPGMR\_MAXL}$=5$.
  \item[bp\_data] (\id{void *})
    pointer to the {\cvbandpre} data structure.
  \end{args}
}
{
  The return value \id{flag} (of type \id{int}) is one of
  \begin{args}[CVSPGMR\_ILL\_INPUT]
  \item[\Id{CVSPGMR\_SUCCESS}] 
    The {\cvspgmr} initialization was successful.
  \item[\Id{CVSPGMR\_MEM\_NULL}]
    The \id{cvode\_mem} pointer is \id{NULL}.
  \item[\Id{CVSPGMR\_ILL\_INPUT}]
    The preconditioner type \id{pretype} is not valid.
  \item[\Id{CVSPGMR\_MEM\_FAIL}]
    A memory allocation request failed.
  \item[\Id{CV\_PDATA\_NULL}]
    The {\cvbandpre} preconditioner has not been initialized.
  \end{args}
}
{}
%%
\ucfunction{CVBandPrecFree}
{
  CVBandPrecFree(bp\_data);
}
{
  The function \ID{CVBandPrecFree} frees the pointer allocated by
  \id{CVBandPrecAlloc}.
}
{
  The only argument of \id{CVBandPrecFree} is the pointer to the {\cvbandpre} 
  data structure (of type \id{void *}).
}
{
  The function \id{CVBandPrecFree} has no return value.
}
{}
%%
\index{CVBANDPRE@{\cvbandpre} preconditioner!user-callable functions|)}

\index{optional output!banded preconditioner|(}
\index{CVBANDPRE@{\cvbandpre} preconditioner!optional output|(}

The following three optional output functions are available for use with 
the {\cvbandpre} module:
%%
\index{memory requirements!CVBANDPRE@{\cvbandpre} preconditioner|(}
%%
\ucfunction{CVBandPrecGetWorkSpace}
{
  flag = CVBandPrecGetWorkSpace(bp\_data, \&lenrwBP, \&leniwBP);
}
{
  The function \ID{CVBandPrecGetWorkSpace} returns the
  {\cvbandpre} real and integer workspace sizes.
}
{
  \begin{args}[lenrwBP]
  \item[bp\_data] (\id{void *})
    pointer to the {\cvbandpre} data structure.
  \item[lenrwBP] (\id{long int})
    the number of \id{realtype} values in the {\cvbandpre} workspace.
  \item[leniwBP] (\id{long int})
    the number of integer values in the {\cvbandpre} workspace.
  \end{args}
}
{
  The return value \id{flag} (of type \id{int}) is one of
  \begin{args}[CV\_PDATA\_NULL]
  \item[\Id{CV\_SUCCESS}] 
    The optional output value has been successfully set.
  \item[\Id{CV\_PDATA\_NULL}]
    The {\cvbandpre} preconditioner has not been initialized.
  \end{args}
}
{
  In terms of problem size $N$, and \id{smu} = $\min(N-1,\,$\id{mu+ml}),
  the actual size of the real workspace is
  $(2$ \id{ml} $+$ \id{mu} $+$ \id{smu} $+2)\, N$ \id{realtype} words,
  and the actual size of the integer workspace is $N$ integer words.
}
%%
\index{memory requirements!CVBANDPRE@{\cvbandpre} preconditioner|)}
%%
\ucfunction{CVBandPrecGetNumRhsEvals}
{
  flag = CVBandPrecGetNumRhsEvals(bp\_data, \&nfevalsBP);
}
{
  The function \ID{CVBandPrecGetNumRhsEvals} returns the
  number of calls to the user right-hand side function for
  finite difference banded Jacobian approximation used within
  {\cvbandpre}'s preconditioner setup function.
}
{
  \begin{args}[nfevalsBP]
  \item[bp\_data] (\id{void *})
    pointer to the {\cvbandpre} data structure.
  \item[nfevalsBP] (\id{long int})
    the number of calls to the user right-hand side function.
  \end{args}
}
{
  The return value \id{flag} (of type \id{int}) is one of
  \begin{args}[CV\_PDATA\_NULL]
  \item[\Id{CV\_SUCCESS}] 
    The optional output value has been successfully set.
  \item[\Id{CV\_PDATA\_NULL}]
    The {\cvbandpre} preconditioner has not been initialized.
  \end{args}
}
{}
%%
\index{CVBANDPRE@{\cvbandpre} preconditioner!optional output|)}
\index{optional output!banded preconditioner|)}

%%==============================================================================
\subsection{A parallel band-block-diagonal preconditioner module}
\label{sss:cvbbdpre}
%%==============================================================================

A principal reason for using a parallel ODE solver such as {\cvode} lies
in the solution of partial differential equations (PDEs).  Moreover,
the use of a Krylov iterative method for the solution of many such
problems is motivated by the nature of the underlying linear system of
equations (\ref{e:Newton}) that must be solved at each time step.  The
linear algebraic system is large, sparse, and structured. However, if
a Krylov iterative method is to be effective in this setting, then a
nontrivial preconditioner needs to be used.  Otherwise, the rate of
convergence of the Krylov iterative method is usually unacceptably
slow.  Unfortunately, an effective preconditioner tends to be
problem-specific.

However, we have developed one type of preconditioner that treats a
rather broad class of PDE-based problems.  It has been successfully
used for several realistic, large-scale problems \cite{HiTa:98} and is
included in a software module within the {\cvode} package. This module
works with the parallel vector module {\nvecp} and 
generates a preconditioner that is a block-diagonal matrix with each
block being a band matrix. The blocks need not have the same number of
super- and sub-diagonals and these numbers may vary from block to
block. This Band-Block-Diagonal Preconditioner module is called
{\cvbbdpre}.

\index{CVBBDPRE@{\cvbbdpre} preconditioner!description|(}
\index{preconditioning!band-block diagonal}
One way to envision these preconditioners is to think of the domain of
the computational PDE problem as being subdivided into $M$ non-overlapping
subdomains.  Each of these subdomains is then assigned to one of the
$M$ processors to be used to solve the ODE system. The basic idea is
to isolate the preconditioning so that it is local to each processor,
and also to use a (possibly cheaper) approximate right-hand side
function. This requires the definition of a new function $g(t,y)$
which approximates the function $f(t, y)$ in the definition of the ODE
system (\ref{e:ivp}). However, the user may set $g = f$.  Corresponding
to the domain decomposition, there is a decomposition of the solution
vector $y$ into $M$ disjoint blocks $y_m$, and a decomposition of $g$
into blocks $g_m$.  The block $g_m$ depends on $y_m$ and also on
components of blocks $y_{m'}$ associated with neighboring subdomains
(so-called ghost-cell data).  Let $\bar{y}_m$ denote $y_m$ augmented
with those other components on which $g_m$ depends.  Then we have
\begin{equation}
  g(t,y) = [g_1(t,\bar{y}_1), g_2(t,\bar{y}_2), \ldots, g_M(t,\bar{y}_M)]^T
\end{equation}
and each of the blocks $g_m(t, \bar{y}_m)$ is uncoupled from the others.

The preconditioner associated with this decomposition has the form 
\begin{equation}
  P= diag[P_1, P_2, \ldots, P_M]
\end{equation}
where 
\begin{equation}
  P_m \approx I - \gamma J_m
\end{equation}
and $J_m$ is a difference quotient approximation to 
$\partial g_m/\partial y_m$. This matrix is taken to be banded, with
upper and lower half-bandwidths \id{mudq} and \id{mldq} defined as
the number of non-zero diagonals above and below the main diagonal,
respectively. The difference quotient approximation is computed using
\id{mudq} $+$ \id{mldq} $+ 2$ evaluations of $g_m$, but only a matrix
of bandwidth \id{mu} $+$ \id{ml} $+ 1$ is retained. 
Neither pair of parameters need be the true half-bandwidths of the Jacobian of
the local block of $g$, if smaller values provide a more efficient
preconditioner. The solution of the complete linear system
\begin{equation}
  Px = b
\end{equation}
reduces to solving each of the equations 
\begin{equation}
  P_m x_m = b_m
\end{equation}
and this is done by banded LU factorization of $P_m$ followed by a banded
backsolve.
\index{CVBBDPRE@{\cvbbdpre} preconditioner!description|)}

\index{CVBBDPRE@{\cvbbdpre} preconditioner!user-supplied functions|(}
The {\cvbbdpre} module calls two user-provided functions to construct $P$: 
a required function \id{gloc} (of type \id{CVLocalFn}) 
which approximates the right-hand side function $g(t,y) \approx f(t,y)$ and which 
is computed locally, and an optional function \id{cfn} (of type \id{CVCommFn}) which performs 
all inter-process communication necessary to evaluate the approximate right-hand
side $g$.  These are in addition to the user-supplied right-hand side function
\id{f}.  Both functions take as input the same pointer \id{f\_data} as that passed
by the user to \id{CVodeSetFdata} and passed to the user's function \id{f},
and neither function has a return value. The user is responsible for
providing space (presumably within \id{f\_data}) for components of \id{y}
that are communicated by \id{cfn} from the other processors, and that are
then used by \id{gloc}, which is not expected to do any communication.
%%
%%
\usfunction{CVLocalFn}
{
  typedef void (*CVLocalFn)(&long int Nlocal, realtype t,  \\
                            &N\_Vector y, N\_Vector glocal, \\
                            &void *f\_data);
}
{
  This function computes $g(t,y)$. It loads the vector
  \id{glocal} as a function of \id{t} and \id{y}.  
}
{
  \begin{args}[Nlocal]
  \item[Nlocal] 
    is the local vector length.
  \item[t]
    is the value of the independent variable.
  \item[y]
    is the dependent variable. 
  \item[glocal]
    is the output vector.
  \item[f\_data]
    is a pointer to user data --- the same as the \Id{f\_data}      
    parameter passed to \id{CVodeSetFdata}.  
  \end{args}
}
{
  A \id{CVLocalFn} function type does not have a return value.
}
{
  This function assumes that all inter-processor communication of data needed to 
  calculate \id{glocal} has already been done, and this data is accessible within
  \id{f\_data}.

  The case where $g$ is mathematically identical to $f$ is allowed.
}
%%
%%
\usfunction{CVCommFn}
{
  typedef void (*CVCommFn)(&long int Nlocal, realtype t,  \\
                           &N\_Vector y, void *f\_data);
}
{
  This function performs all inter-processor communications necessary 
  for the execution of the \id{gloc} function above, using the input vector \id{y}.
}
{
  \begin{args}[Nlocal]
  \item[Nlocal] 
    is the local vector length.
  \item[t]
    is the value of the independent variable.
  \item[y]
    is the dependent variable. 
  \item[f\_data]
    is a pointer to user data --- the same as the \Id{f\_data}      
    parameter passed to \id{CVodeSetFdata}.  
  \end{args}
}
{
  A \id{CVCommFn} function type does not have a return value.
}
{
  The \id{cfn} function is expected to save communicated data in space defined
  within the structure \id{f\_data}. 

  Each call to the \id{cfn} function is preceded by a call to the right-hand side
  function \id{f} with the same (\id{t}, \id{y}) arguments.  Thus \id{cfn} can omit 
  any communications done by \id{f} if relevant to the evaluation of \id{glocal}.
  If all necessary comunication was done in \id{f}, then \id{cfn} $=$ \id{NULL}
  can be passed in the call to \id{CVBBDPrecAlloc} (see below).
}
%%
\index{CVBBDPRE@{\cvbbdpre} preconditioner!user-supplied functions|)}

\index{CVBBDPRE@{\cvbbdpre} preconditioner!usage|(}
%%
Besides the header files required for the integration of the ODE problem
(see \S\ref{ss:header_sim}),  to use the {\cvbbdpre} module, the main program 
must include the header file \id{cvbbdpre.h} which declares the needed
function prototypes.\index{header files}

The following is a summary of the usage of this module and describes the sequence
of calls in the user main program. Steps that are unchanged from the user main
program presented in \S\ref{ss:skeleton_sim} are grayed-out.
%%
%%
\index{User main program!CVBBDPRE@{\cvbbdpre} usage}
\begin{Steps}
\item 
  \textcolor{gray}{\bf Initialize MPI}

\item
  \textcolor{gray}{\bf Set problem dimensions}

\item
  \textcolor{gray}{\bf Set vector of initial values}
 
\item
  \textcolor{gray}{\bf Create {\cvode} object}

\item
  \textcolor{gray}{\bf Set optional inputs}

\item
  \textcolor{gray}{\bf Allocate internal memory}

\item \label{i:bbdpre_init}
  {\bf Initialize the {\cvbbdpre} preconditioner module}

  Specify the upper and lower half-bandwidths \id{mudq}, \id{mldq} and
  \id{mukeep}, \id{mlkeep} and call 

   \id{
     \begin{tabular}[t]{@{}r@{}l@{}}
       bbd\_data = CVBBDPrecAlloc(&cvode\_mem, local\_N, mudq, mldq, \\
                                  &mukeep, mlkeep, dqrely, gloc, cfn);
     \end{tabular}
   }

  to allocate memory for and initialize a data structure \id{bbd\_data} to be 
  passed to the {\cvspgmr} linear solver. The last two arguments of
  \id{CVBBDPrecAlloc} are the two user-supplied functions described above.

\item \label{i:bbdpre_attach}
  {\bf Attach the {\cvspgmr} linear solver}

  \id{flag = CVBBDSpgmr(cvode\_mem, pretype, maxl, bbd\_data);}

  The function \Id{CVBPSpgmr} is a wrapper around the {\cvspgmr} specification
  function \id{CVSpgmr} and performs the following actions:
  \begin{itemize}
    \item Attaches the {\cvspgmr} linear solver to the main {\cvode} solver memory;
    \item Sets the preconditioner data structure for {\cvbbdpre};
    \item Sets the preconditioner setup function for {\cvbbdpre};
    \item Sets the preconditioner solve function for {\cvbbdpre};
  \end{itemize}
  The arguments \id{pretype} and \id{maxl} are described below.
  The last argument of \id{CVBBDSpgmr} is the pointer to the {\cvbbdpre} data
  returned by \id{CVBBDPrecAlloc}.

\item
  \textcolor{gray}{\bf Set linear solver optional inputs}

  Note that the user should not overwrite the preconditioner data, setup function,
  or solve function through calls to {\cvspgmr} optional input functions.

\item
  \textcolor{gray}{\bf Advance solution in time}

\item
  \textcolor{gray}{\bf Deallocate memory for solution vector}

\item \label{i:bbdpre_free}
  {\bf Free the {\cvbbdpre} data structure}

  \id{CVBBDPrecFree(bbd\_data);}

\item
  \textcolor{gray}{\bf Free solver memory}
  
\item 
  \textcolor{gray}{\bf Finalize MPI}

\end{Steps}
%%
\index{CVBBDPRE@{\cvbbdpre} preconditioner!usage|)}
%%
\index{CVBBDPRE@{\cvbbdpre} preconditioner!user-callable functions|(}
%%
The three user-callable functions that initialize, attach, and deallocate
the {\cvbbdpre} preconditioner module (steps \ref{i:bbdpre_init},
\ref{i:bbdpre_attach}, and \ref{i:bbdpre_free} above) are described
next.
%%
\index{half-bandwidths}
\ucfunction{CVBBDPrecAlloc}
{
   \begin{tabular}[t]{@{}r@{}l@{}}
     bbd\_data = CVBBDPrecAlloc(&cvode\_mem, local\_N, mudq, mldq, \\
                                &mukeep, mlkeep, dqrely, gloc, cfn);
   \end{tabular}
}
{
  The function \ID{CVBBDPrecAlloc} initializes and allocates
  memory for the {\cvbbdpre} preconditioner.
}
{
  \begin{args}[cvode\_mem]
  \item[cvode\_mem] (\id{void *})
    pointer to the {\cvode} memory block.
  \item[local\_N] (\id{long int})
    local vector length.
  \item[mudq] (\id{long int})
    upper half-bandwidth to be used in the difference-quotient Jacobian approximation.
  \item[mldq] (\id{long int})
    lower half-bandwidth to be used in the difference-quotient Jacobian approximation.
  \item[mukeep] (\id{long int})
    upper half-bandwidth of the retained banded approximate Jacobian block.
  \item[mlkeep] (\id{long int})
    lower half-bandwidth of the retained banded approximate Jacobian block.
  \item[dqrely] (\id{realtype})
    the relative increment in components of \id{y} used in the difference quotient
    approximations.  The default is \id{dqrely}$ = \sqrt{\text{unit roundoff}}$,
    which can be specified by passing \id{dqrely}$ = 0.0$.
  \item[gloc] (\id{CVLocalFn})
    the {\C} function which computes the approximation $g(t,y) \approx f(t,y)$. 
  \item[cfn] (\id{CVCommFn})
    the optional {\C} function which performs all inter-process communication required for
    the computation of $g(t,y)$.
  \end{args}
}
{
  If successful, \id{CVBBDPrecAlloc} returns a pointer to the newly created 
  {\cvbbdpre} memory block (of type \id{void *}).
  If an error occurred, \id{CVBBDPrecAlloc} returns \id{NULL}.
}
{
  If one of the half-bandwidths \id{mudq} or \id{mldq} to be used in the 
  difference-quotient calculation of the approximate Jacobian is negative or 
  exceeds the value \id{local\_N}$-1$, it is replaced with 0 or
  \id{local\_N}$-1$ accordingly.

  The half-bandwidths \id{mudq} and \id{mldq} need not be the true 
  half-bandwidths of the Jacobian of the local block of $g$,    
  when smaller values may provide a greater efficiency.       

  Also, the half-bandwidths \id{mukeep} and \id{mlkeep} of the retained 
  banded approximate Jacobian block may be even smaller,      
  to reduce storage and computation costs further.            

  For all four half-bandwidths, the values need not be the    
  same on every processor.
}
%%
%%
\ucfunction{CVBBDSpgmr}
{
  flag = CVBBDSpgmr(cvode\_mem, pretype, maxl, bbd\_data);
}
{
  The function \ID{CVBBDSpgmr} links the {\cvbbdpre} data to the
  {\cvspgmr} linear solver and attaches the latter to the {\cvode}
  memory block.
}
{
  \begin{args}[cvode\_mem]
  \item[cvode\_mem] (\id{void *})
    pointer to the {\cvode} memory block.
  \item[pretype] (\id{int})
    \index{pretype@\texttt{pretype}}
    preconditioning type. Must be one of \Id{PREC\_LEFT} or \Id{PREC\_RIGHT}.
  \item[maxl] (\id{int})
    \index{maxl@\texttt{maxl}}
    maximum dimension of the Krylov subspace to be used. Pass $0$ to use the 
    default value \id{CVSPGMR\_MAXL}$=5$.
  \item[bbd\_data] (\id{void *})
    pointer to the {\cvbbdpre} data structure.
  \end{args}
}
{
  The return value \id{flag} (of type \id{int}) is one of
  \begin{args}[CVSPGMR\_ILL\_INPUT]
  \item[\Id{CVSPGMR\_SUCCESS}] 
    The {\cvspgmr} initialization was successful.
  \item[\Id{CVSPGMR\_MEM\_NULL}]
    The \id{cvode\_mem} pointer is \id{NULL}.
  \item[\Id{CVSPGMR\_ILL\_INPUT}]
    The preconditioner type \id{pretype} is not valid.
  \item[\Id{CVSPGMR\_MEM\_FAIL}]
    A memory allocation request failed.
  \item[\Id{CV\_PDATA\_NULL}]
    The {\cvbbdpre} preconditioner has not been initialized.
  \end{args}
}
{}
%%
\ucfunction{CVBBDPrecFree}
{
  CVBBDPrecFree(bbd\_data);
}
{
  The function \ID{CVBBDPrecFree} frees the pointer allocated by
  \id{CVBBDPrecAlloc}.
}
{
  The only argument of \id{CVBBDPrecFree} is the pointer to the {\cvbbdpre} 
  data structure (of type \id{void *}).
}
{
  The function \id{CVBBDPrecFree} has no return value.
}
{}
%%
The {\cvbbdpre} module also provides a reinitialization function to allow
solving  a sequence of problems of the same size with {\cvspgmr}/{\cvbbdpre},
provided there is no change in \id{local\_N}, \id{mukeep}, or \id{mlkeep}.
After solving one problem, and after calling \id{CVodeReInit} to re-initialize 
{\cvode} for a subsequent problem, a call to \id{CVBBDPrecReInit} can be made
to change any of the following: the half-bandwidths \id{mudq} and \id{mldq} 
used in the difference-quotient Jacobian approximations, the relative increment
\id{dqrely}, or one of the user-supplied functions \id{gloc} and \id{cfn}.
%%
\ucfunction{CVBBDPrecReInit}
{
  flag = CVBBDPrecReInit(bbd\_data, mudq, mldq, dqrely, gloc, cfn);
}
{
  The function \ID{CVBBDPrecReInit} reinitializes the {\cvbbdpre} preconditioner.
}
{
  \begin{args}[bbd\_data]
  \item[bbd\_data] (\id{void *})
    pointer to the {\cvbbdpre} data structure.
  \item[mudq] (\id{long int})
    upper half-bandwidth to be used in the difference-quotient Jacobian approximation.
  \item[mldq] (\id{long int})
    lower half-bandwidth to be used in the difference-quotient Jacobian approximation.
  \item[dqrely] (\id{realtype})
    the relative increment in components of \id{y} used in the difference quotient
    approximations.  The default is \id{dqrely} $= \sqrt{\text{unit roundoff}}$,
    which can be specified by passing \id{dqrely} $= 0.0$.
  \item[gloc] (\id{CVLocalFn})
    the {\C} function which computes the approximation $g(t,y) \approx f(t,y)$. 
  \item[cfn] (\id{CVCommFn})
    the optional {\C} function which performs all inter-process communication required for
    the computation of $g(t,y)$.
  \end{args}
}
{
  The return value of \id{CVBBDPrecReInit} is always \Id{CV\_SUCCESS}.
}
{
  If one of the half-bandwidths \id{mudq} or \id{mldq} is negative or
  exceeds the value \id{local\_N}$-1$, it is replaced with 0 or
  \id{local\_N}$-1$ accordingly.
}
%%
\index{CVBBDPRE@{\cvbbdpre} preconditioner!user-callable functions|)}
%%
\index{optional output!band-block-diagonal preconditioner|(}
\index{CVBBDPRE@{\cvbbdpre} preconditioner!optional output|(}
The following two optional output functions are available for use with
the {\cvbbdpre} module:
%%
\index{memory requirements!CVBBDPRE@{\cvbbdpre} preconditioner}
\ucfunction{CVBBDPrecGetWorkSpace}
{
  flag = CVBBDPrecGetWorkSpace(bbd\_data, \&lenrwBBDP, \&leniwBBDP);
}
{
  The function \ID{CVBBDPrecGetWorkSpace} returns the local
  {\cvbbdpre} real and integer workspace sizes.
}
{
  \begin{args}[lenrwBBDP]
  \item[bbd\_data] (\id{void *})
    pointer to the {\cvbbdpre} data structure.
  \item[lenrwBBDP] (\id{long int})
    local number of \id{realtype} values in the {\cvbbdpre} workspace.
  \item[leniwBBDP] (\id{long int})
    local number of integer values in the {\cvbbdpre} workspace.
  \end{args}
}
{
  The return value \id{flag} (of type \id{int}) is one of
  \begin{args}[CV\_PDATA\_NULL]
  \item[\Id{CV\_SUCCESS}] 
    The optional output value has been successfully set.
  \item[\Id{CV\_PDATA\_NULL}]
    The {\cvbbdpre} preconditioner has not been initialized.
  \end{args}
}
{
  In terms of \id{local\_N} and
  \id{smu} = $\min$(\id{local\_N - 1, mukeep} $+$ \id{mlkeep}),
  the actual size of the real workspace is
  (2 \id{mlkeep} $+$ \id{mukeep} $+$ \id{smu} $+2) \, $\id{local\_N} $~$
  \id{realtype} words, and the actual size of the integer workspace is
  \id{local\_N} integer words.  These values are local to the current processor.
}
%%
%%
\ucfunction{CVBBDPrecGetNumGfnEvals}
{
  flag = CVBBDPrecGetNumGfnEvals(bbd\_data, \&ngevalsBBDP);
}
{
  The function \ID{CVBBDPrecGetNumGfnEvals} returns the
  number of calls to the user \id{gloc} function due to the 
  finite difference approximation of the Jacobian blocks used within
  {\cvbbdpre}'s preconditioner setup function.
}
{
  \begin{args}[ngevalsBBDP]
  \item[bbd\_data] (\id{void *})
    pointer to the {\cvbbdpre} data structure.
  \item[ngevalsBBDP] (\id{long int})
    the number of calls to the user \id{gloc} function.
  \end{args}
}
{
  The return value \id{flag} (of type \id{int}) is one of
  \begin{args}[CV\_PDATA\_NULL]
  \item[\Id{CV\_SUCCESS}] 
    The optional output value has been successfully set.
  \item[\Id{CV\_PDATA\_NULL}]
    The {\cvbbdpre} preconditioner has not been initialized.
  \end{args}
}
{}
%%
\index{CVBBDPRE@{\cvbbdpre} preconditioner!optional output|)}
\index{optional output!band-block-diagonal preconditioner|)}

The costs associated with {\cvbbdpre} also include \id{nlinsetups} LU
factorizations, \id{nlinsetups} calls to \id{cfn}, and \id{npsolves} banded
backsolve calls, where \id{nlinsetups} and \id{npsolves} are optional {\cvode}
outputs (see \S\ref{ss:optional_output}).

Similar block-diagonal preconditioners could be considered with different
treatment of the blocks $P_m$. For example, incomplete LU factorization or
an iterative method could be used instead of banded LU factorization.

%%==============================================================================
\section{FCVODE, a {\F}-{\C} interface module}\label{ss:fcmix}
%%==============================================================================

The {\fcvode} interface module is a package of {\C} functions which support
the use of the {\cvode} solver, for the solution of ODE systems 
$dy/dt = f(t,y)$, in a mixed {\F}/{\C} setting.  While {\cvode} is written
in {\C}, it is assumed here that the user's calling program and
user-supplied problem-defining routines are written in {\F}.
This package provides the necessary interface to {\cvode} for both the
serial and the parallel {\nvector} implementations.

%%==============================================================================
\subsection{FCVODE routines}
%%==============================================================================

\index{FCVODE@{\fcvode} interface module!user-callable functions|(}
The user-callable functions, with the corresponding {\cvode} functions,
are as follows:
\begin{itemize}
\item
  Interface to the {\nvector} modules
  \begin{itemize}
  \item \id{FNVINITS} (defined by {\nvecs}) 
    interfaces to \id{NV\_New\_Serial}.
  \item \id{FNVINITP} (defined by {\nvecp}) 
    interfaces to \id{NV\_New\_Parallel}.
  \item \id{FNVFREES} (defined by {\nvecs})
    interface to \id{NV\_Destroy\_Serial}.
  \item \id{FNVFREEP}  (defined by {\nvecp})
    interfaces to \id{NV\_Destroy\_Parallel}.
  \end{itemize}
\item Interface to the main {\cvode} module
  \begin{itemize}
  \item \id{FCVMALLOC}
    interfaces to \id{CVodeCreate}, \id{CVodeSet*} functions, and \id{CVodeMalloc}.
  \item \id{FCVREINIT}  
    interfaces to \id{CVodeReInit} and \id{CVodeSet*} functions.
  \item \id{FCVODE}
    interfaces to \id{CVode}, \id{CVodeGet*} functions, and to the optional
    output functions for the selected linear solver module.
  \item \id{FCVDKY}     
    interfaces to the interpolated output function \id{CVodeGetDky}.
  \item \id{FCVFREE}    
    interfaces to \id{CVodeFree}.
  \end{itemize}  
\item Interface to the linear solver modules
  \begin{itemize}
  \item \id{FCVDIAG}    
    interfaces to \id{CVDiag}
  \item \id{FCVDENSE}
    interfaces to \id{CVDense}.
  \item \id{FCVDENSESETJAC}
    interfaces to \id{CVDenseSetJacFn}.
  \item \id{FCVBAND}
    interfaces to \id{CVBand}.
  \item \id{FCVBANDSETJAC}
    interfaces to \id{CVBandSetJacFn}.
  \item \id{FCVSPGMR}
    interfaces to \id{CVSpgmr} and {\spgmr} optional input functions.
  \item \id{FCVSPGMRREINIT} 
    interfaces to {\spgmr} optional input functions.
  \item \id{FCVSPGMRSETJAC}
   interfaces to \id{CVSpgmrSetJacTimesVecFn}.
 \item \id{FCVSPGMRSETPSOL}
   interfaces to \id{CVSpgmrSetPrecSolveFn}.
 \item \id{FCVSPGMRSETPSET}
   interfaces to \id{CVSpgmrSetPrecSetupFn}.
 \end{itemize}

\end{itemize}
\index{FCVODE@{\fcvode} interface module!user-callable functions|)}

\index{FCVODE@{\fcvode} interface module!user-supplied functions}
The user-supplied functions, each listed with the corresponding interface
function which calls it (and its type within {\cvode}), are as follows:
\begin{center}
\begin{tabular}{|l|l|l|}
\hline
{\fcvode} routine ({\F})  &  {\cvode} function ({\C}) & {\cvode} function type \\\hline
\id{FCVFUN}    & \id{FCVf}        & \id{CVRhsFn} \\
\id{FCVDJAC}   & \id{FCVDenseJac} & \id{CVDenseJacFn} \\
\id{FCVBJAC}   & \id{FCVBandJac}  & \id{CVBandJacFn} \\
\id{FCVPSOL}   & \id{FCVPSol}     & \id{CVSpgmrPrecSolveFn} \\
\id{FCVPSET}   & \id{FCVPSet}     & \id{CVSpgmrPrecSetupFn} \\
\id{FCVJTIMES} & \id{FCVJtimes}   & \id{CVSpgmrJacTimesVecFn} \\\hline
\end{tabular}
\end{center}
In contrast to the case of direct use of {\cvode}, and of most {\F} ODE
solvers, the names of all user-supplied routines here are fixed, in
order to maximize portability for the resulting mixed-language program.

%%==============================================================================
\subsubsection{Important note on portability}
%%==============================================================================
\index{portability!Fortran}

In this package, the names of the interface functions, and the names of
the {\F} user routines called by them, appear as dummy names
which are mapped to actual values by a series of definitions in the
header files \id{fcvode.h} and \id{fcvbbd.h}.
By default, those mapping definitions depend in turn on the {\C} macro
\id{F77\_FUNC} defined in the header file \id{config.h} by \id{configure}. However,
the set of flags --- \Id{SUNDIALS\_CASE\_UPPER}, \Id{SUNDIALS\_CASE\_LOWER},
\Id{SUNDIALS\_UNDERSCORE\_NONE}, \Id{SUNDIALS\_UNDERSCORE\_ONE}, and
\Id{SUNDIALS\_UNDERSCORE\_TWO} can be explicitly defined in \id{config.h} when
configuring {\sundials} via the \id{--with-f77underscore} and
\id{--with-f77case} options to override the default behavior if necessary
(see Chapter \ref{s:install}). Either way, the names into which the dummy names
are mapped are in upper or lower case and have up to two underscores appended.

The user must also ensure that variables in the user {\F} code are
declared in a manner consistent with their counterparts in {\cvode}.
All real variables must be declared as \id{REAL}, \id{DOUBLE PRECISION},
or perhaps as \id{REAL*}{\em n}, where {\em n} denotes the number of bytes,
depending on whether {\cvode} was built in single, double or extended precision 
(see Chapter \ref{s:install}). Moreover, some of the {\F} integer variables
must be declared as \id{INTEGER*4} or \id{INTEGER*8} according to the 
{\C} type \id{long int}. These integer variables include: the array
of integer optional inputs and outputs (\id{IOPT}), problem dimensions (\id{NEQ},
\id{NLOCAL}, \id{NGLOBAL}), and Jacobian half-bandwidths (\id{MU}, \id{ML},
\id{MUDQ}, and \id{MLDQ}). This is particularly important when using
{\cvode} and the {\fcvode} package on 64-bit architectures.

%%==============================================================================
\subsection{FCVODE optional input and output}
%%==============================================================================
\index{FCVODE@{\fcvode} interface module!optional input and output}

In order to keep the number of user-callable {\fcvode} interface routines to
a minimum, optional inputs and outputs to the {\cvode} solver and to related 
modules are not accessed through individual functions, but rather through a
pair of arrays, \Id{IOPT} of integer type and \Id{ROPT} of real type.
Table \ref{t:fcvode_io} lists the entries in these two arrays and specifies the
{\fcvode} user-callable routine which sets/accesses the corresponding optional
variable, as well as the {\cvode} optional function which is actually called.
For more details on the optional inputs and outputs, see \S\ref{ss:optional_input}
and \S\ref{ss:optional_output}.

\begin{table}
\centering
\caption{Description of the {\fcvode} optional input-output arrays \Id{IOPT} and
         \Id{ROPT}}
\label{t:fcvode_io}
\medskip
\begin{tabular}{|r|c|c|l|}
\multicolumn{4}{c}{Integer input-output array \id{IOPT}}\\
\hline
{\bf Index} & {\bf Optional input} & {\bf Optional output} & {\cvode} {\bf function} \\ 
\hline
\multicolumn{4}{|c|}{{\cvode} main solver}\\
\hline
%
1  & \id{MAXORD}          &  & \id{CVodeSetMaxOrd} \\
%
2  & \id{MXSTEP}          &  & \id{CVodeSetMaxNumSteps} \\
%
3  & \id{MXHNIL}          &  & \id{CVodeSetMaxHnilWarns} \\
%
4  &                      & \id{NST}               & \id{CVodeGetNumSteps} \\
%                                                                
5  &                      & \id{NFE}              & \id{CVodeGetNumRhsEvals} \\
%
6  &                      & \id{NSETUPS}                 & \id{CVodeGetNumLinSolvSetups} \\
%
7  &                      & \id{NNI}                & \id{CVodeGetNumNonlinSolvIters} \\
%
8  &                      & \id{NCFN}                 & \id{CVodeGetNumNonlinSolvConvFails} \\
%
9  &                      & \id{NETF}              & \id{CVodeGetNumErrTestFails} \\
%
10 &                      & \id{QU}            & \id{CVodeGetLastOrder} \\
%
11 &                      & \id{QCUR}                 & \id{CVodeGetCurrentOrder} \\
%
12, 13 &                  & \id{LENRW}, \id{LENIW}  & \id{CVodeGetWorkSpace} \\
%
14 & \id{SLDET}           &  & \id{CVodeSetStabLimDet} \\
%
15 &                      & \id{NOR}            & \id{CVodeGetNumStabLimOrderReds} \\ 
%
22 & \id{MAXERRTESTFAILS} & & \id{CVodeSetMaxErrTestFails} \\
%
23 & \id{MAXNONLINITERS}  & & \id{CVodeSetMaxNonlinIters} \\
%
24 & \id{MAXCONVFAILS}    & & \id{CVodeSetMaxConvFails} \\
%
25 &                      & \id{NGE}                 & \id{CVodeGetNumGEvals} \\ 
%
\hline
\multicolumn{4}{|c|}{{\cvdense} linear solver}\\
\hline
16, 17 &  &  \id{LRW}, \id{LIW}             & \id{CVDenseGetWorkSpace} \\ 
18 &  &       \id{NJE}             & \id{CVDenseGetNumJacEvals} \\ 
26 &  &     \id{LS\_FLAG}               & \id{CVDenseGetLastFlag} \\ 
\hline
\multicolumn{4}{|c|}{{\cvband} linear solver}\\
\hline
16, 17 &  &   \id{LRW}, \id{LIW}                  & \id{CVBandGetWorkSpace} \\ 
18 &  &       \id{NJE}             & \id{CVBandGetNumJacEvals} \\ 
26 &  &      \id{LS\_FLAG}              & \id{CVBandGetLastFlag} \\ 
\hline
\multicolumn{4}{|c|}{{\cvdiag} linear solver}\\
\hline
16, 17 &  &  \id{LRW}, \id{LIW}                   & \id{CVDiagGetWorkSpace} \\ 
26 &  &       \id{LS\_FLAG}             & \id{CVDiagGetLastFlag} \\ 
\hline
\multicolumn{4}{|c|}{{\cvspgmr} linear solver}\\
\hline
16, 17 &  &   \id{LRW}, \id{LIW}                  & \id{CVSpgmrGetWorkSpace} \\ 
18 &  &      \id{NPE}              & \id{CVSpgmrGetNumPrecEvals} \\ 
19 &  &      \id{NLI}              & \id{CVSpgmrGetNumLinIters} \\ 
20 &  &       \id{NPS}             & \id{CVSpgmrGetNumPrecSolves} \\ 
21 &  &      \id{NCFL}              & \id{CVSpgmrGetNumConvFails} \\
26 &  &      \id{LS\_FLAG}              & \id{CVSpgmrGetLastFlag} \\ 
\hline
\multicolumn{4}{c}{}\\
\multicolumn{4}{c}{Real input-output array \id{ROPT}}\\\hline
{\bf Index} & {\bf Optional input} & {\bf Optional output} & {\cvode} {\bf function} \\ 
\hline
%
1  & \id{H0}             &             & \id{CVodeSetInitStep} \\
%
2  & \id{HMAX}           &             & \id{CVodeSetMaxStep} \\
%
3  & \id{HMIN}           &             & \id{CVodeSetMinStep} \\
%
4  &                     & \id{HU}     & \id{CVodeGetLastStep} \\
%
5  &                     & \id{HCUR}   & \id{CVodeGetCurrentStep} \\
%
6  &                     & \id{TCUR}   & \id{CVodeGetCurrentTime} \\
%
7  &                     & \id{TOLSF}  & \id{CVodeGetTolScaleFactor} \\
%
8  & \id{TSTOP}          &             & \id{CVodeSetStopTime} \\
%
9  & \id{NONLINCONVCOEF} &             & \id{CVodeSetNonlinConvCoef} \\
%
10 &                     & \id{UROUND} & unit roundoff \\
\hline
%
\end{tabular}
\end{table}                                                                  

%%==============================================================================
\subsection{Usage of the FCVODE interface module}\label{ss:fcvode_usage}
%%==============================================================================
\index{FCVODE@{\fcvode} interface module!usage|(}

The usage of {\fcvode} requires calls to six or seven interface
functions, depending on the method options selected, and one or more
user-supplied routines which define the problem to be solved.  These
function calls and user routines are summarized separately below.
Some details are omitted, and the user is referred to the description
of the corresponding {\cvode} functions for information on the arguments 
of any given user-callable interface routine, or of a given user-supplied 
function called by an interface function.
The usage of {\fcvode} with preconditioner modules is described in later
subsections.

Steps marked with {\s} in the instructions below apply to the serial
{\nvector} implementation ({\nvecs}) only, while those marked with {\p}
apply to {\nvecp}.

\index{User main program!FCVODE@{\fcvode} usage}
\begin{Steps}
  
\item {\bf Right-hand side specification}
  
  The user must in all cases supply the following Fortran routine
  \index{FCVFUN@\texttt{FCVFUN}}
\begin{verbatim}
      SUBROUTINE FCVFUN(T, Y, YDOT)
      DIMENSION Y(*), YDOT(*)
\end{verbatim}
  It must set the \id{YDOT} array to $f(t,y)$, the right-hand side of the ODE
  system, as function of \id{T}$=t$ and the array \id{Y}$=y$.  
  
\item  {\bf {\nvector} module initialization}

  {\s} To initialize the serial {\nvector} module, the user must make the
  following call:
  \index{FNVINITS@\texttt{FNVINITS}}
\begin{verbatim}
      CALL FNVINITS(NEQ, IER)
\end{verbatim}
  where \id{NEQ} is the size of vectors and
  \id{IER} is a  return completion flag which is set to $0$ on success and $-1$ 
  if a failure occurred.
  
  {\p} To initialize the parallel vector module, the user must make the
  following call:
  \index{FNVINITP@\texttt{FNVINITP}}
\begin{verbatim}
      CALL FNVINITP(NLOCAL, NGLOBAL, IER)
\end{verbatim}
  in which the arguments are: \id{NLOCAL} the local size of vectors on this
  processor, \id{NGLOBAL} the system size (and the global size of vectors, that
  is the sum of all values of NLOCAL). The return completion flag \id{IER} is
  set on $0$ upon successful return and on $-1$ otherwise.
  Note that if MPI was initialized by the user, the communicator must be
  set to \id{MPI\_COMM\_WORLD}.  If not, this routine initializes MPI and sets
  the communicator equal to \id{MPI\_COMM\_WORLD}.
  
\item {\bf Problem specification}

  To set various problem and solution parameters and allocate
  internal memory, make the following call:
  \index{FCVMALLOC@\texttt{FCVMALLOC}}
  \ucfunction{FCVMALLOC}
  {
    \begin{tabular}[t]{@{}r@{}l@{}l@{}}
      &CALL FCVMALLOC(&T0, Y0, METH, ITMETH, IATOL, RTOL, ATOL, INOPT, \\
    \&&               &IOPT, ROPT, IER)
    \end{tabular}
  }
  {
    This function provides required problem and solution specifications, 
    specifies optional inputs,
    allocates internal memory, and initializes {\cvode}.
  }
  {
    \begin{args}[ITMETH ]
    \item[T0] is the initial value of $t$.
    \item[Y0] is an array of initial conditions.
    \item[METH] specifies the  basic integration method: 
      $1$ for Adams (nonstiff) or $2$ for BDF (stiff).
    \item[ITMETH] specifies the nonlinear iteration method: 
      $1$ for functional iteration or $2$ for Newton iteration.
    \item[IATOL] specifies the type for absolute tolerance \id{ATOL}:
      $1$ for scalar or $2$ for array.
    \item[RTOL] is the relative tolerance (scalar).
    \item[ATOL] is the absolute tolerance (scalar or array).
    \item[INOPT] is the optional input flag: $0$ if none or $1$ if optional 
      inputs are used.
    \item[IOPT] is an array of length 40 for integer optional inputs and outputs.
    \item[ROPT] is an array of length 40 for real optional inputs and outputs.
    \end{args}
  }
  {
    ~~\id{IER} is a return completion flag.  Values are $0$ for successful return
    and $-1$ otherwise. See printed message for details in case of failure.
  }
  {
    The optional inputs and outputs associated with the main {\cvode} integrator
    are listed in Table~\ref{t:fcvode_io}.
    If any of the optional inputs are used, the others must be set
    to zero to indicate default values.
  }

\item\label{i:fcvode_lin_solv_spec} {\bf Linear solver specification} 
  
  In the case of a stiff system, the implicit \id{BDF} method involves the solution
  of linear systems related to the Jacobian $J = \partial f / \partial y$
  of the ODE system.  {\cvode} presently includes four choices for the treatment
  of these systems, and the user of {\fcvode} must call a routine with a
  specific name to make the desired choice.

  {\s} {\bf Diagonal approximate Jacobian}
  \index{CVDIAG@{\cvdiag} linear solver!use in {\fcvode}}
  
  This choice is appropriate when the Jacobian can be well approximated by
  a diagonal matrix.  The user must make the call:
  \index{FCVDIAG@\texttt{FCVDIAG}}
\begin{verbatim}
      CALL FCVDIAG(IER)
\end{verbatim}
  \id{IER} is an error return flag set on $0$ on success or $-1$ if a memory 
  failure occurred.
  There is no additional user-supplied routine. Optional outputs specific
  to the {\diag} case listed in Table~\ref{t:fcvode_io}.
  
  {\s} {\bf Dense treatment of the linear system}
  \index{CVDENSE@{\cvdense} linear solver!use in {\fcvode}}
  
  The user must make the call:
  \index{FCVDENSE@\texttt{FCVDENSE}}
\begin{verbatim}
      CALL FCVDENSE(NEQ, IER)
\end{verbatim}
  The argument \id{IER} is an error return flag which can be $0$ 
  for success , $-1$ if a memory allocation failure occurred, or $-2$ for illegal
  input.  \index{Jacobian approximation function!dense!use in {\fcvode}}
  As an option when using the {\dense} linear solver, the user may supply a
  routine that computes a dense approximation of the system Jacobian 
  $J = \partial f / \partial y$. If supplied, it must have the following form:
  \index{FCVDJAC@\texttt{FCVDJAC}}
\begin{verbatim}
      SUBROUTINE FCVDJAC (NEQ, T, Y, FY, DJAC, EWT, H, WK1, WK2, WK3)
      DIMENSION Y(*), FY(*), EWT(*), DJAC(NEQ,*), WK1(*), WK2(*), WK3(*)
\end{verbatim}
  Typically this routine will use only \id{NEQ}, \id{T}, \id{Y}, and \id{DJAC}. 
  It must compute the Jacobian and store it columnwise in \id{DJAC}.
  \id{FY} contains $f(t,y)$. The vectors \id{WK1}, \id{WK2}, and \id{WK3}
  of length \id{NEQ} are provided as work space for use in \id{FCVDJAC}.
  
  If the user's \id{FCVDJAC} uses difference quotient approximations, it
  may need to use the error weight array \id{EWT} and current stepsize \id{H}
  in the calculation of suitable increments.  It may also need the unit
  roundoff, which can be obtained as the optional output \id{ROPT(10)},
  passed from the calling program to this routine using \id{COMMON}.

  If the \id{FCVDJAC} routine is provided, then, 
  following the call to \id{FCVDENSE}, the user must make the call:
  \index{FCVDENSESETJAC@\texttt{FCVDENSESETJAC}}
\begin{verbatim}
      CALL FCVDENSESETJAC (FLAG, IER)
\end{verbatim}
  with \id{FLAG} $\neq 0$ to specify use of the user-supplied Jacobian approximation.
  The argument \id{IER} is an error return flag which can be $0$ 
  for success or non-zero if an error occurred.
  
  Optional outputs specific to the {\dense} case are listed in Table~\ref{t:fcvode_io}.

  {\s} {\bf Band treatment of the linear system}
  \index{CVBAND@{\cvband} linear solver!use in {\fcvode}}
  
  The user must make the call:
  \index{FCVBAND@\texttt{FCVBAND}}
\begin{verbatim}
      CALL FCVBAND (NEQ, MU, ML, IER)
\end{verbatim}
  The arguments are: \id{MU}, the upper half-bandwidth; \id{ML}, 
  the lower half-bandwidth; and \id{IER} an error return flag which can be  
  $0$ for success , $-1$ if a memory allocation failure occurred, or $-2$ 
  in case an input has an illegal value.     
  
  \index{Jacobian approximation function!band!use in {\fcvode}}
  As an option when using the {\band} linear solver, the user may supply a
  routine that computes a band approximation of the system Jacobian 
  $J = \partial f / \partial y$. If supplied, it must have the following form:
  \index{FCVBJAC@\texttt{FCVBJAC}}
\begin{verbatim}
      SUBROUTINE FCVBJAC(NEQ, MU, ML, MDIM, T, Y, FY, BJAC,
     &                   EWT, H, WK1, WK2, WK3)
      DIMENSION Y(*), FY(*), EWT(*), BJAC(MDIM,*), WK1(*), WK2(*), WK3(*)
\end{verbatim}
  Typically this routine will use only \id{NEQ}, \id{MU}, \id{ML}, \id{T}, 
  \id{Y}, and \id{BJAC}. 
  It must load the \id{MDIM} by \id{N} array \id{BJAC} with the Jacobian matrix
  at the current ($t$,$y$) in band form.  Store in \id{BJAC}(k,j) the Jacobian
  element $J_{i,j}$ with $k = i - j + MU + 1$, $k = 1 \cdots ML+MU+1$ and
  $j = 1 \cdots N$. \id{FY} contains $f(t,y)$. The vectors \id{WK1}, \id{WK2},
  and \id{WK3} of length \id{NEQ} are provided as work space for use in
  \id{FCVBJAC}.

  If the user's \id{FCVBJAC} uses difference quotient approximations, it
  may need to use the error weight array \id{EWT} and current stepsize \id{H}
  in the calculation of suitable increments.  It may also need the unit
  roundoff, which can be obtained as the optional output \id{ROPT(10)},
  passed from the calling program to this routine using \id{COMMON}.

  If the \id{FCVBJAC} routine is provided, then, following the call to \id{FCVBAND},
  the user must make the call:
  \index{FCVBANDSETJAC@\texttt{FCVBANDSETJAC}}
\begin{verbatim}
      CALL FCVBANDSETJAC(FLAG, IER)
\end{verbatim}
  with \id{FLAG} $\neq 0$ to specify use of the user-supplied Jacobian approximation.
  The argument \id{IER} is an error return flag which can be $0$ 
  for success or non-zero if an error occurred.
  
  Optional outputs specific to the {\band} case are listed in Table~\ref{t:fcvode_io}.
  
  {\s}{\p} {\bf SPGMR treatment of the linear systems}
  \index{CVSPGMR@{\cvspgmr} linear solver!use in {\fcvode}}
  
  For the Scaled Preconditioned GMRES solution of the linear systems,
  the user must make the call
  \index{FCVSPGMR@\texttt{FCVSPGMR}}
\begin{verbatim}
      CALL FCVSPGMR(IPRETYPE, IGSTYPE, MAXL, DELT, IER)
\end{verbatim}
  The arguments are as follows.
  \id{IPRETYPE} specifies the preconditioner type: 
  $0$ for no preconditioning, $1$ for left only, $2$ for right only, or $3$ for
  both sides. \id{IGSTYPE} indicates the Gram-Schmidt process type: 
  $0$ for modified G-S or  $1$ for classical G-S.
  \id{MAXL} is the maximum Krylov subspace dimension ($0$ indicates default).
  \id{DELT} is the linear convergence tolerance factor ($0.0$ indicates default).
  \id{IER} is an error return flag which can be $0$ to indicate success, $-1$
  if a memory allocation failure occurred, or $-2$ to indicate an illegal input.
  
  \index{Jacobian approximation function!Jacobian times vector!use in {\fcvode}}
  As an option when using the {\spgmr} linear solver, the user may supply a 
  routine that computes the product of the system Jacobian 
  $J = \partial f / \partial y$ 
  and a given vector $v$.  If supplied, it must have the following form:
  \index{FCVJTIMES@\texttt{FCVJTIMES}}
\begin{verbatim}
      SUBROUTINE FCVJTIMES (V, FJV, T, Y, FY, EWT, H, WORK, IER)
      DIMENSION V(*), FJV(*), Y(*), FY(*), EWT(*), WORK(*)
\end{verbatim}
  Typically this routine will use only \id{NEQ}, \id{T}, \id{Y}, \id{V}, and
  \id{FJV}.  It must compute the product vector $Jv$, where the vector $v$ is
  stored in \id{V}, and store the product in \id{FJV}.  On return, set
  \id{IER=0} if \id{FCVJTIMES} was successful, and nonzero otherwise.
  \id{FY} contains $f(t,y)$. The vector \id{WORK}, of length \id{NEQ}, is
  provided as work space for use in \id{FCVJTIMES}.

  If the user's \id{FCVJTIMES} uses difference quotient approximations, it
  may need to use the error weight array \id{EWT} and current stepsize \id{H}
  in the calculation of suitable increments.  It may also need the unit
  roundoff, which can be obtained as the optional output \id{ROPT(10)},
  passed from the calling program to this routine using \id{COMMON}.

  If the \id{FCVJTIMES} routine is provided, then, 
  following the call to \id{FCVSPGMR}, the user must make the call:
  \index{FCVSPGMRSETJAC@\texttt{FCVSPGMRSETJAC}}
\begin{verbatim}
      CALL FCVSPGMRSETJAC(FLAG, IER)
\end{verbatim}
  with \id{FLAG} $\neq 0$ to specify use of the user-supplied Jacobian times
  vector approximation.
  The argument \id{IER} is an error return flag which can be $0$ 
  for success or non-zero if an error occurred.
  
  If preconditioning is to be done (\id{IPRETYPE} $\neq 0$), then, following the
  call to \id{FCVSPGMR}, the user must call
  \index{FCVSPGMRSETPSOL@\texttt{FCVSPGMRSETPSOL}}
\begin{verbatim}
      CALL FCVSPGMRSETPSOL(FLAG, IER)
\end{verbatim}
  with \id{FLAG} $\neq 0$, and the user program must include the following routine
  for solution of the preconditioner linear system:
  \index{FCVPSOL@\texttt{FCVPSOL}}
\begin{verbatim}
      SUBROUTINE FCVPSOL(T, Y, FY, VT, GAMMA, EWT, DELTA, R, LR, Z, IER)
      DIMENSION Y(*), FY(*), VT(*), EWT(*), R(*), Z(*)
\end{verbatim}
  It must solve the preconditioner linear system $Pz = r$, where $r =$ \id{R} 
  is input, and store the solution $z$ in \id{Z}. Here $P$ is the left 
  preconditioner if \id{LR=1} and the right preconditioner if \id{LR=2}.  
  The preconditioner (or the product of the left and right preconditioners 
  if both are nontrivial) should be an  approximation to the matrix 
  $I - \gamma J$, where $I$ is the identity matrix, $J$ is the system Jacobian,
  and $\gamma =$ \id{GAMMA}.
  
  The arguments \id{EWT} and \id{DELTA} are input and provide the error weight
  array and a scalar tolerance, respectively, for use by \id{FCVPSOL} if it uses
  an iterative method in its solution.  In that case, the residual vector
  $\rho = r - Pz$ of the system should be made less than \id{DELTA} in weighted
  $\ell_2$ norm, i.e. $\sqrt{\sum(\rho_i * \id{EWT}[i])^2} < $ \id{DELTA}.
  The argument \id{VT} is a work array of length \id{NEQ} for use by this
  routine.

  If the user's preconditioner requires that any Jacobian related data be evaluated
  or preprocessed, then, following the call to \id{FCVSPGMRSETPSOL}, the user must
  call \index{FCVSPGMRSETPSET@\texttt{FCVSPGMRSETPSET}}
\begin{verbatim}
      CALL FCVSPGMRSETPSET(FLAG, IER)
\end{verbatim}
  with \id{FLAG} $\neq 0$.
  In this case, the user program must also include
  the following routine for the evaluation and preprocessing of the preconditioner:
  \index{FCVPSET@\texttt{FCVPSET}}
\begin{verbatim}
      SUBROUTINE FCVPSET(T, Y, FY, JOK, JCUR, GAMMA, EWT, H, V1, V2, V3, IER)
      DIMENSION Y(*), FY(*), EWT(*), V1(*), V2(*), V3(*) 
\end{verbatim}
  It must perform any evaluation of Jacobian-related data and preprocessing needed
  for the solution of the preconditioner linear systems by \id{FCVPSOL}.
  The input argument \id{JOK} allows for Jacobian data to be saved and reused:
  If \id{JOK=0}, this data should be recomputed from scratch. If \id{JOK=1},
  a saved copy of it may be reused, and the preconditioner constructed from it.
  On return, set \id{JCUR=1} if Jacobian data was computed, and \id{0} otherwise.
  Also on return, set \id{IER=0} if \id{FCVPSET} was successful, set \id{IER}
  positive if a recoverable error occurred, and set \id{IER} negative if a 
  non-recoverable error occurred.
  
  If the user's \id{FCVPSET} uses difference quotient approximations, it
  may need to use the error weight array \id{EWT} and current stepsize \id{H}
  in the calculation of suitable increments.  It may also need the unit
  roundoff, which can be obtained as the optional output \id{ROPT(10)},
  passed from the calling program to this routine using \id{COMMON}.

  Optional outputs specific to the {\spgmr} case are listed in Table~\ref{t:fcvode_io}.
  
  If a sequence of problems of the same size is being solved using the {\spgmr}
  linear solver, then following the call to \id{FCVREINIT} (see below), a call
  to the \id{FCVSPGMR} routine may or may not be needed.  
  If there is a change in input arguments other than \id{MAXL}, then the user 
  program should call the routine \id{FCVSPGMRREINIT} which
  reinitializes the {\spgmr} linear solver, but without reallocating its memory.
  The arguments of \id{FCVSPGMRREINIT} routine have the same names and meanings
  as those of \id{FCVSPGMR} routine.  Finally, if the value of \id{MAXL} is
  being changed, then a call to \id{FCVSPGMR} must be made.

\item {\bf Problem solution}

  Carrying out the integration is accomplished by making calls as follows:
  \index{FCVODE@\texttt{FCVODE}}
\begin{verbatim}
      CALL FCVODE(TOUT, T, Y, ITASK, IER)
\end{verbatim}
  The arguments are as follows.
  \id{TOUT} specifies the next value of $t$ at which a solution is desired (input).
  \id{T} is the value of $t$ reached by the solver on output.
  \id{Y} is an array containing the computed solution on output.
  \id{ITASK} is a task indicator and should be set to $1$ for normal mode 
  (overshoot \id{TOUT} and interpolate), to $2$ for one-step mode 
  (return after each internal step taken), to $3$ for normal mode with
  the additional \id{tstop} constraint, or to $4$ for one-step mode 
  with the additional constraint \id{tstop}.
  \id{IER} is a completion flag and will be set to a positive value upon
  successful return or to a negative value if an error occurred. These values
  correspond to the \id{CVode} returns (see \S\ref{sss:cvode}) as follows:
  $0$: \Id{CV\_SUCCESS}, $1$: \Id{CV\_TSTOP\_RETURN}, $2$: \Id{CV\_ROOT\_RETURN},
  $-1$: \Id{CV\_MEM\_NULL}, $-2$: \Id{Cv\_ILL\_INPUT}, $-3$: \Id{CV\_NO\_MALLOC}
  $-4$: \Id{CV\_TOO\_MUCH\_WORK}, $-5$: \Id{CV\_TOO\_MUCH\_ACC}, $-6$: \Id{CV\_ERR\_FAILURE},
  $-7$: \Id{CV\_CONV\_FAILURE}, $-8$: \Id{CV\_LINIT\_FAIL}, $-9$: \Id{CV\_LSETUP\_FAIL}, 
  and $-10$: \Id{CV\_LSOLVE\_FAIL} from \id{CVode} (see \S\ref{sss:cvode}).
  The current values of the optional outputs are available in \id{IOPT} and
  \id{ROPT} (see Table~\ref{t:fcvode_io}).
  
\item {\bf Additional solution output}

  To obtain a derivative of the solution, of order up to the current method
  order, make the following call:
  \index{FCVDKY@\texttt{FCVDKY}}
\begin{verbatim}
      CALL FCVDKY(T, K, DKY, IER)
\end{verbatim}
  where
  \id{T} is the value of $t$ at which solution derivative is desired,
  \id{K} is the derivative order ($0 \le$ \id{K} $\le$ \id{QU}), and
  \id{DKY} is an array containing the computed \id{K}-th derivative of $y$
  on return.  The value \id{T} must lie between \id{TCUR-HU} and \id{TCUR}.
  The return flag \id{IER} is set to $0$ upon successful return or to a negative
  value to indicate an illegal input.
  
\item {\bf Problem reinitialization}

  To re-initialize the {\cvode} solver for the solution of a new problem
  of the same size as one already solved, make the following call:
  \index{FCVREINIT@\texttt{FCVREINIT}}
\begin{verbatim}
      CALL FCVREINIT(T0, Y0, IATOL, RTOL, ATOL, INOPT, IOPT, ROPT, IER)
\end{verbatim}
  The arguments have the same names and meanings as those of \id{FCVMALLOC}.
  \id{FCVREINIT} performs the same initializations as
  \id{FCVMALLOC}, but does no memory allocation, using instead the existing
  internal memory created by the previous \id{FCVMALLOC} call.  The call to
  specify the linear system solution method may or may not be needed.

\item {\bf Memory deallocation}

  To free the internal memory created by the call to \id{FCVMALLOC},
  make the call
  \index{FCVFREE@\texttt{FCVFREE}}
\begin{verbatim}
      CALL FCVFREE
\end{verbatim}
  and then, depending on the {\nvector} version (serial or parallel), either
  \index{FNVFREES@\texttt{FNVFREES}}
\begin{verbatim}
      CALL FNVFREES
\end{verbatim}
  or
  \index{FNVFREEP@\texttt{FNVFREEP}}
\begin{verbatim}
      CALL FNVFREEP  
\end{verbatim}
  respectively.

\end{Steps}
\index{FCVODE@{\fcvode} interface module!usage|)}

%%==============================================================================
\subsection{Usage of the FCVROOT interface to rootfinding}
%%==============================================================================
\index{FCVODE@{\fcvode} interface module!rootfinding|(}
\index{Rootfinding}

The {\fcvroot} interface package allows programs written in {\F} to
use the rootfinding feature of the {\cvode} solver module.
%%
The user-callable functions in {\fcvroot}, with the corresponding
{\cvode} functions, are as follows: 
\begin{itemize}
  \item \id{FCVROOTINIT} interfaces to \id{CVodeRootInit}.
  \item \id{FCVROOTINFO} interfaces to \id{CVodeGetRootInfo}.
  \item \id{FCVROOTFREE} interfaces to \id{CVodeRootInit}.
\end{itemize}
%%
In order to use the rootfinding feature of {\cvode}, the following
call must be made, after calling \id{FCVMALLOC} but prior to calling
\id{FCVODE}, to allocate and initialize memory for the \id{FCVROOT} module:
\begin{verbatim}
      CALL FCVROOTINIT (NRTFN, IER)
\end{verbatim}
The arguments are as follows:
\id{NRTFN} is the number of root functions.
\id{IER} is a return completion flag; its values are $0$ for success, $-1$ 
if the \id{CVODE} memory was \id{NULL}, and $-11$ if a memory allocation failed.

To specifiy the functions whose roots are to be found, the user must
define the following routine:
\begin{verbatim}
      SUBROUTINE FCVROOTFN (T, Y, G)
      DIMENSION Y(*), G(*)
\end{verbatim}
It must set the \id{G} array, of length \id{NRTFN}, with components $g_i(t,y)$,
as a function of \id{T}$=t$ and the array \id{Y}$=y$.  

When making calls to \id{FCVODE} to solve the ODE system, the occurrence of
a root is flagged by the return value \id{IER} = 2.  In that case, if
\id{NRTFN} $> 1$, the functions $g_i$ which were found to have a root can
be identified by making the following call:
\begin{verbatim}
      CALL FCVROOTINFO (NRTFN, INFO, IER)
\end{verbatim}
The arguments are as follows: \id{NRTFN} is the number of root functions.
\id{INFO} is an integer array of length \id{NRTFN} with root information.
\id{IER} is a return completion flag; its values are $0$ for success, 
negative if there was a memory failure.  The returned values of \id{INFO(i)}
(\id{i}$ = 1,\ldots,$ \id{NRTFN}) are 0 or 1, such that \id{INFO(i)} $ = 1$
if $g_{\id{i}}$ was found to have a root, and \id{INFO(i)} $ = 0$ otherwise.

The total number of calls made to the root function \id{FCVROOTFN},
denoted \id{NGE}, can be obtained from \id{IOPT(25)}.
%%
If the {\fcvode}/{\cvode} memory block is reinitialized to solve a
different problem via a call to \id{FCVREINIT}, then the counter
\id{NGE} is reset to zero.

To free the memory resources allocated by a prior call to \id{FCVROOTINIT} make
the following call:
\begin{verbatim}
      CALL FCVROOTFREE
\end{verbatim}
See \S\ref{s:using_rootfinding} for additional information on the
rootfinding feature.
\index{FCVODE@{\fcvode} interface module!rootfinding|)}

%%==============================================================================
\subsection{Usage of the FCVBP interface to CVBANDPRE}
%%==============================================================================
\index{FCVODE@{\fcvode} interface module!interface to the {\cvbandpre} module|(}

The {\fcvbp} interface sub-module is a package of {\C} functions which,
as part of the {\fcvode} interface module, support the use of the
{\cvode} solver with the serial {\nvecs} module and the {\cvbandpre} 
preconditioner module (see \S\ref{sss:cvbandpre}), for the solution of 
ODE systems in a mixed {\F}/{\C} setting.  

The user-callable functions in this package, with the corresponding
{\cvode} and {\cvbandpre} functions, are as follows: 
\begin{itemize}
\item \id{FCVBPINIT}
  interfaces to \id{CVBandPrecAlloc}.
\item \id{FCVBPSPGMR}
  interfaces to \id{CVBPSpgmr} and {\spgmr} optional input functions.
\item \id{FCVBPREINIT}
  interfaces to \id{CVBandPrecReInit}.
\item \id{FCVBPOPT}
  interfaces to {\cvbandpre} optional output functions.
\item \id{FCVBPFREE}
  interfaces to \id{CVBandPrecFree}.
\end{itemize}

As with the rest of the {\fcvode} routines, the names of the user-supplied
routines are mapped to actual values through a series of definitions in the
header file \id{fcvbp.h}.

The following is a summary of the usage of this module. Steps that are unchanged
from the main program described in \S\ref{ss:fcvode_usage} are grayed-out.

\index{User main program!FCVBP@{\fcvbp} usage}
\begin{Steps}
  
\item \textcolor{gray}{\bf Right-hand side specification}

\item \textcolor{gray}{\bf {\nvector} module initialization}

\item \textcolor{gray}{\bf Problem specification}

\item {\bf Linear solver specification}

  To initialize the {\cvbandpre} preconditioner, make the following call:
  \index{FCVBPINIT@\texttt{FCVBPINIT}}
\begin{verbatim}
       CALL FCVBPINIT(NEQ, MU, ML, IER)
\end{verbatim}
  The arguments are as follows.
  \id{NEQ} is the problem size.
  \id{MU} and \id{ML} are the upper and lower half-bandwidths of the band matrix
  that  is retained as an approximation of the Jacobian.
  \id{IER} is a return completion flag.  A value of $0$ indicates success, while 
  a value of $-1$ indicates that a memory failure occurred.
  
  To specify the {\spgmr} linear system solver and use the {\cvbandpre}
  preconditioner, make the following call:
  \index{FCVBPSPGMR@\texttt{FCVBPSPGMR}}
\begin{verbatim}
       CALL FCVBPSPGMR(IPRETYPE, IGSTYPE, MAXL, DELT, IER)
\end{verbatim}
  Its arguments are the same as those of \id{FCVSPGMR}
  (see step \ref{i:fcvode_lin_solv_spec} in \S\ref{ss:fcvode_usage}).
  

  Optionally, to specify that {\spgmr} should use the supplied \id{FCVJTIMES}, 
  make the call
  \index{FCVSPGMRSETJAC@\texttt{FCVSPGMRSETJAC}}
\begin{verbatim}
       CALL FCVSPGMRSETJAC(FLAG, IER)
\end{verbatim}
  with \id{FLAG} $\neq 0$ 
  (see step \ref{i:fcvode_lin_solv_spec} in \S\ref{ss:fcvode_usage} for details).
  
\item \textcolor{gray}{\bf Problem solution}
  
\item {\bf {\cvbbdpre} Optional outputs}
  
  Optional outputs specific to the {\spgmr} solver are \id{NPE}, \id{NLI},
  \id{NPS}, \id{NCFL}, \id{LRW}, and \id{LIW}, stored in \id{IOPT(16)} $\cdots$
  \id{IOPT(21)}, respectively.
  To obtain the optional outputs associated with the {\cvbandpre} module, make
  the following call:
  \index{FCVBPOPT@\texttt{FCVBPOPT}}
\begin{verbatim}
       CALL FCVBPOPT(LENRPW, LENIPW, NFE)
\end{verbatim}
  The arguments returned are as follows.
  \id{LENRPW} is the length of real preconditioner work space, in \id{realtype}
  words. \id{LENIPW} is the length of integer preconditioner work space, in
  integer words. \id{NFE} is the number of $f(t,y)$ evaluations (calls to
  \id{FCVFUN}) for finite difference banded Jacobian approximation.
  
\item {\bf Memory deallocation}

  To free the internal memory created by the call to \id{FCVBPINIT}, before
  calling \id{FCVFREE} and \id{FNVFREEP}, the user must call
  \index{FCVBPFREE@\texttt{FCVBPFREE}}
\begin{verbatim}
      CALL FCVBPFREE
\end{verbatim}
\index{FCVODE@{\fcvode} interface module!interface to the {\cvbandpre} module|)}

\end{Steps}

%%==============================================================================
\subsection{Usage of the FCVBBD interface to CVBBDPRE}
%%==============================================================================
\index{FCVODE@{\fcvode} interface module!interface to the {\cvbbdpre} module|(}

The {\fcvbbd} interface sub-module is a package of {\C} functions which,
as part of the {\fcvode} interface module, support the use of the
{\cvode} solver with the parallel {\nvecp} module and the {\cvbbdpre} 
preconditioner module (see \S\ref{sss:cvbbdpre}), for the solution of 
ODE systems in a mixed {\F}/{\C} setting.  

The user-callable functions in this package, with the corresponding
{\cvode} and {\cvbbdpre} functions, are as follows: 
\begin{itemize}
\item \id{FCVBBDINIT}
  interfaces to \id{CVBBDPrecAlloc}.
\item \id{FCVBBDSPGMR}
  interfaces to \id{CVBBDSpgmr} and {\spgmr} optional input functions.
\item \id{FCVBBDREINIT}
  interfaces to \id{CVBBDPrecReInit}.
\item \id{FCVBBDOPT}
  interfaces to {\cvbbdpre} optional output functions.
\item \id{FCVBBDFREE}
  interfaces to \id{CVBBDPrecFree}.
\end{itemize}

In addition to the Fortran right-hand side function \id{FCVFUN}, the
user-supplied functions used by this package, are listed below,
each with the corresponding interface function which calls it (and its
type within {\cvbbdpre} or {\cvode}):
\begin{center}
\begin{tabular}{|l|l|l|}
\hline
{\fcvbbd} routine ({\F})  &  {\cvode} function ({\C}) & {\cvode} function type \\\hline
\id{FCVLOCFN}  & \id{FCVgloc}     & \id{CVLocalFn} \\
\id{FCVCOMMF}  & \id{FCVcfn}      & \id{CVCommFn} \\
\id{FCVJTIMES} & \id{FCVJtimes}   & \id{CVSpgmrJacTimesVecFn} \\ \hline
\end{tabular}
\end{center}
As with the rest of the {\fcvode} routines, the names of all user-supplied routines 
here are fixed, in order to maximize portability for the resulting mixed-language
program.  Additionally, based on the flags \Id{SUNDIALS\_CASE\_*} and 
\Id{SUNDIALS\_UNDERSCORE\_*}, the names of the user-supplied routines 
are mapped to actual values through a series of definitions in the header file 
\id{fcvbbd.h}.

The following is a summary of the usage of this module. Steps that are unchanged
from the main program described in \S\ref{ss:fcvode_usage} are grayed-out.

\index{User main program!FCVBBD@{\fcvbbd} usage}
\begin{Steps}
  
\item \textcolor{gray}{\bf Right-hand side specification}

\item \textcolor{gray}{\bf {\nvector} module initialization}

\item \textcolor{gray}{\bf Problem specification}

\item {\bf Linear solver specification}

  To initialize the {\cvbbdpre} preconditioner, make the following call:
  \index{FCVBBDINIT@\texttt{FCVBBDINIT}}
\begin{verbatim}
       CALL FCVBBDINIT(NLOCAL, MUDQ, MLDQ, MU, ML, DQRELY, IER)
\end{verbatim}
  The arguments are as follows.
  \id{NLOCAL} is the local size of vectors on this processor.
  \id{MUDQ} and \id{MLDQ} are the upper and lower half-bandwidths to be used in 
  the computation of the local Jacobian blocks by difference quotients.
  These may be smaller than the true half-bandwidths of the
  Jacobian of the local block of $g$, when smaller values may
  provide greater efficiency.
  \id{MU} and \id{ML} are the upper and lower half-bandwidths of the band matrix
  that  is retained as an approximation of the local Jacobian block.
  These may be smaller than \id{MUDQ} and \id{MLDQ}.
  \id{DQRELY} is the relative increment factor in $y$ for difference quotients
  (optional).  A value of $0.0$ indicates the default, $\sqrt{\text{unit roundoff}}$.
  \id{IER} is a return completion flag.  A value of $0$ indicates success, while
  a value of $-1$ indicates that a memory failure occurred or that an input had
  an illegal value.
  
  To specify the {\spgmr} linear system solver and use the {\cvbbdpre}
  preconditioner, make the following call:
  \index{FCVBBDSPGMR@\texttt{FCVBBDSPGMR}}
\begin{verbatim}
       CALL FCVBBDSPGMR(IPRETYPE, IGSTYPE, MAXL, DELT, IER)
\end{verbatim}
  Its arguments are the same as those of \id{FCVSPGMR}
  (see step \ref{i:fcvode_lin_solv_spec} in \S\ref{ss:fcvode_usage}).
  

  Optionally, to specify that {\spgmr} should use the supplied \id{FCVJTIMES}, 
  make the call
  \index{FCVSPGMRSETJAC@\texttt{FCVSPGMRSETJAC}}
\begin{verbatim}
       CALL FCVSPGMRSETJAC(FLAG, IER)
\end{verbatim}
  with \id{FLAG} $\neq 0$ 
  (see step \ref{i:fcvode_lin_solv_spec} in \S\ref{ss:fcvode_usage} for details).
  
\item \textcolor{gray}{\bf Problem solution}
  
\item {\bf {\cvbbdpre} Optional outputs}
  
  Optional outputs specific to the {\spgmr} solver are \id{NPE}, \id{NLI},
  \id{NPS}, \id{NCFL}, \id{LRW}, and \id{LIW}, stored in \id{IOPT(16)} $\cdots$
  \id{IOPT(21)}, respectively.
  To obtain the optional outputs associated with the {\cvbbdpre} module, make
  the following call:
  \index{FCVBBDOPT@\texttt{FCVBBDOPT}}
\begin{verbatim}
       CALL FCVBBDOPT(LENRPW, LENIPW, NGE)
\end{verbatim}
  The arguments returned are as follows.
  \id{LENRPW} is the length of real preconditioner work space, in \id{realtype}
  words.  This size is local to the current processor.
  \id{LENIPW} is the length of integer preconditioner work space, in integer
  words.  This size is local to the current processor.
  \id{NGE} is the number of $g(t,y)$ evaluations (calls to \id{FCVLOCFN}) so far.
  
\item {\bf Problem reinitialization}
  
  If a sequence of problems of the same size is being solved using the {\spgmr}
  linear solver in combination with the {\cvbbdpre} preconditioner, then the
  {\cvode} package can be re-initialized for the second and subsequent problems
  by calling \id{FCVREINIT}, following which, a call to \id{FCVBBDINIT} may or 
  may not be needed.
  If the input arguments are the same, no \id{FCVBBDINIT} call is needed.
  If there is a change in input arguments other than \id{MU}, \id{ML}, or
  \id{MAXL}, then the user program should make the call 
  \index{FCVBBDREINIT@\texttt{FCVBBDREINIT}}
\begin{verbatim}
       CALL FCVBBDREINIT(NLOCAL, MUDQ, MLDQ, DQRELY, IER)
\end{verbatim}
  This reinitializes the {\cvbbdpre} preconditioner, but without
  reallocating its memory.  The arguments of the \id{FCVBBDREINIT}
  routine have the same names and meanings as those of \id{FCVBBDINIT}.
  If the value of \id{MU} or \id{ML} is being changed, then a call to
  \id{FCVBBDINIT} must be made.  Finally, if \id{MAXL} is being
  changed, then a call to \id{FCVBBDSPGMR} must be made; in this case
  the {\spgmr} memory is reallocated.
  
\item {\bf Memory deallocation}

  To free the internal memory created by the call to \id{FCVBBDINIT}, before
  calling \id{FCVFREE} and \id{FNVFREEP}, the user must call
  \index{FCVBBDFREE@\texttt{FCVBBDFREE}}
\begin{verbatim}
      CALL FCVBBDFREE
\end{verbatim}
\index{FCVODE@{\fcvode} interface module!interface to the {\cvbbdpre} module|)}

\end{Steps}
