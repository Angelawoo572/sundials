%===================================================================================
\chapter{Introduction}\label{s:intro}
%===================================================================================

{\cvode} is part of a software family called {\sundials}: 
SUite of Nonlinear and DIfferential/ALgebraic equation Solvers.  
This suite consists of {\cvode}, {\kinsol}, and {\ida}, and variants of these
with sensitivity analysis capabilities.
%
%---------------------------------
\section{Historical Background}\label{ss:history}
%---------------------------------

\index{CVODE@{\cvode}!relationship to {\vode}, {\vodpk}|(}
{\F} solvers for ODE initial value problems are widespread and heavily used. 
Two solvers that have been written at LLNL in the past are {\vode} \cite{BBH:89} 
and {\vodpk} \cite{Byr:92}.
{\vode}\index{VODE@{\vode}} is a general purpose solver that includes methods for stiff
and nonstiff systems, and in the stiff case uses direct methods (full or
banded) for the solution of the linear systems that arise at each implicit
step. Externally, {\vode} is very similar to the well known solver
{\lsode}\index{LSODE@{\lsode}} \cite{RaHi:94}.
{\vodpk}\index{VODPK@{\vodpk}} is a variant of {\vode} that uses a preconditioned Krylov 
(iterative) method for the solution of the linear systems. {\vodpk} is a powerful 
tool for large stiff systems because it combines established methods for stiff 
integration, nonlinear iteration, and Krylov (linear) iteration with a problem-specific
treatment of the dominant source of stiffness, in the form of the user-supplied
preconditioner matrix \cite{BrHi:89}.
The capabilities of both {\vode} and {\vodpk} have been combined in the {\C}-language 
package {\cvode}\index{CVODE@{\cvode}} \cite{CoHi:96}.

In the process of translating the {\vode} and {\vodpk} algorithms into {\C}, the overall 
{\cvode} organization has been changed considerably.
One key feature of the {\cvode} organization is that the linear system solvers comprise a
layer of code modules that is separated from the integration algorithm, allowing for 
easy modification and expansion of the linear solver array.
A second key feature is a separate module devoted to vector operations; this 
facilitated the extension to multiprosessor environments with minimal impacts 
on the rest of the solver, resulting in {\pvode}\index{PVODE@{\pvode}} \cite{ByHi:99}, 
the parallel variant of {\cvode}.
\index{CVODE@{\cvode}!relationship to {\vode}, {\vodpk}|)}

\index{CVODE@{\cvode}!relationship to {\cvode}, {\pvode}|(}
Recently, the functionality of {\cvode} and {\pvode} has been combined into
one single code, simply called {\cvode}.
Development of the new version of {\cvode} was concurrent with a redesign of the vector operations module
across the {\sundials} suite. The key feature of the new {\nvector} module is that it
is written in terms of abstract vector operations with the actual vector kernels attached
by a particular implementation (such as serial or parallel) of {\nvector}. This allows
writing the {\sundials} solvers in a manner independent of the actual {\nvector} 
implementation (which can be user-supplied), as well as allowing more than one 
{\nvector} module linked into an executable file.
\index{CVODE@{\cvode}!relationship to {\cvode}, {\pvode}|)}

\index{CVODE@{\cvode}!motivation for writing in C|(}
There are several motivations for choosing the {\C} language for {\cvode}.
First, a general movement away from {\F} and toward {\C} in scientific
computing is apparent.  Second, the pointer, structure, and dynamic
memory allocation features in C are extremely useful in software of
this complexity, with the great variety of method options offered.
Finally, we prefer {\C} over {\CPP} for {\cvode} because of the wider
availability of {\C} compilers, the potentially greater efficiency of {\C},
and the greater ease of interfacing the solver to applications written
in extended {\F}.
\index{CVODE@{\cvode}!motivation for writing in C|)}

\section{Changes in version 2.1}
The major changes from the previous version involve a redesign of the
user interface across the entire {\sundials} suite. We have eliminated the
mechanism of providing optional inputs and extracting optional statistics 
from the solver through the \id{iopt} and \id{ropt} arrays. Instead,
{\cvode} now provides a set of routines (with prefix \id{CVodeSet})
to change the default values for various quantities controlling the
solver and a set of extraction routines (with prefix \id{CVodeGet})
to extract statistics after return from the main solver routine.
Similarly, each linear solver module provides its own set of {\id{Set}-}
and {\id{Get}-type} routines. For more details see \S\ref{ss:optional_input}
and \S\ref{ss:optional_output}.

Additionally, the interfaces to several user-supplied routines
(such as those providing Jacobians and preconditioner information) 
were simplified by reducing the number
of arguments. The same information that was previously accessible
through such arguments can now be obtained through {\id{Get}-type}
functions.

Installation of {\cvode} (and all of {\sundials}) has been completely 
redesigned and is now based on configure scripts.

\section{Reading this User Guide}\label{ss:reading}

This user guide is a combination of general usage instructions and
specific example programs.  We expect that some readers will want to
concentrate on the general instructions, while others will refer
mostly to the examples, and the organization is intended to
accommodate both styles.

There are different possible levels of usage of {\cvode}. The most casual
user, with a small IVP problem only, can get by with reading \S\ref{ss:ivp_sol}, 
then \S\ref{s:simulation} through \S\ref{sss:cvode} only, and looking at examples 
in \cite{cvode2.1_ex}. 
In a different direction, a more expert user with an IVP problem may want
to (a) use a package preconditioner (\S\ref{ss:preconds}), (b) supply
his/her own Jacobian or preconditioner routines (\S\ref{ss:user_fct_sim}),
(c) do multiple runs of problems of the same size (\S\ref{sss:cvreinit}), 
(d) supply a new {\nvector} module (\S\ref{s:nvector}), or even 
(e) supply a different linear solver module
(\S\ref{ss:cvode_org} and \S\ref{s:gen_linsolv}).

The structure of this document is as follows:
\begin{itemize}
\item
  In \S\ref{s:install} we begin with instructions for the installation of 
  {\cvode}, within the structure of {\sundials}.
\item
  In \S\ref{s:math}, we give short descriptions of the numerical 
  methods implemented by {\cvode} for the solution of initial value problems
  for systems of ODEs.
\item
  The following section describes the structure of the {\sundials} suite
  of solvers (\S\ref{ss:sun_org}) and the software organization of the {\cvode}
  solver (\S\ref{ss:cvode_org}). 
\item
  In \S\ref{s:simulation}, we give an overview of the usage of {\cvode},
  as well as a complete description of the user interface and of the 
  user-defined routines for integration of IVP ODEs.
\item
  Section \ref{s:nvector} gives a brief overview of the generic {\nvector} module 
  shared among the various components of {\sundials}, as well as details on the two {\nvector}
  implementations provided with {\sundials}: a serial implementation
  (\S\ref{ss:nvec_ser}) and a parallel implementation, based on MPI\index{MPI}
  (\S\ref{ss:nvec_par}).
\item
  Section \ref{s:gen_linsolv} describes in detail the generic linear solvers shared 
  by all {\sundials} solvers.
\end{itemize}

Finally, the reader should be aware of the following notational conventions
in this user guide:  program listings and identifiers (such as \id{CVodeMalloc}) 
within textual explanations appear in typewriter type style; 
fields in {\C} structures (such as {\em content}) appear in italics;
and packages or modules, such as {\cvdense}, are written in all capitals. 
In the Index, page numbers that appear in bold indicate the main reference
for that entry.

\paragraph{Aknowledgments.}
We wish to aknowledge the contributions to previous versions of the
{\cvode} and {\pvode} codes and user guides of Scott D. Cohen \cite{CoHi:94}
and George D. Byrne \cite{ByHi:98}.

