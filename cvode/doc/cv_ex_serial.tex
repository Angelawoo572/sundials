%===============================================================================
\section{Serial example problems}\label{s:ex_serial}
%===============================================================================

\subsection{A dense example: \id{cvdx}}\label{ss:cvdx}

As an initial illustration of the use of the {\cvode} package for the
integration of IVP ODEs, we give a sample program called \id{cvdx.c}.
It uses the {\cvode} dense linear solver module {\cvdense} 
and the {\nvecs} module (which provides a serial implementation of {\nvector})
in the solution of a 3-species chemical kinetics problem.

The problem consists of the following three rate equations:
\vspace*{-.08in}
\begin{equation}
  \begin{split}
    \dot{y}_1 &= -0.04 \cdot y_1 + 10^4 \cdot y_2 \cdot y_3 \\
    \dot{y}_2 &=  0.04 \cdot y_1 - 10^4 \cdot y_2 \cdot y_3
                                  - 3 \cdot 10^7 \cdot y_2^2 \\
    \dot{y}_3 &=  3 \cdot 10^7 \cdot y_2^2
  \end{split} 
\end{equation}
on the interval $t \in [0, ~4 \cdot 10^{10}]$, with initial conditions
$y_1(0) = 1.0$, $y_2(0) = y_3(0) = 0.0$.
While integrating the system, we also use the rootfinding
feature to find the points at which $y_1 = 10^{-4}$ or at which
$y_3 = 0.01$.

For the source, listed in Appendix \ref{s:cvdx_c}, we give a rather detailed
explanation of the parts of the program and their interaction with {\cvode}.

Following the initial comment block, this program has a number
of \id{\#include} lines, which allow access to useful items in {\cvode}
header files.  The \id{sundialstypes.h} file provides the definition of the
type \id{realtype} (see \ugref{s:types} for details).  
For now, it suffices to read \id{realtype} as \id{double}.
The \id{cvode.h} file provides prototypes for the {\cvode}
functions to be called (excluding the linear solver selection
function), and also a number of constants that are to be used in
setting input arguments and testing the return value of \id{CVode}.
The \id{cvdense.h} file provides the prototype for the \id{CVDense} function.
The \id{nvector\_serial.h} file is the header file for the serial
implementation of the {\nvector} module and includes definitions of the 
\id{N\_Vector} type, a macro to access vector components, and prototypes 
for the serial implementation specific machine environment memory allocation
and freeing functions.
The \id{dense.h} file provides the definition of the dense
matrix type \id{DenseMat} and a macro for accessing matrix elements.
We have explicitly included \id{dense.h}, but this is not necessary because 
it is included by \id{cvdense.h}.
Finally, the \id{sundialsmath.h} file provides the definition of the \id{ABS}
macro used in the user-provided function \id{ewt}.

This program includes two user-defined accessor macros, \id{Ith} and
\id{IJth} that are useful in writing the problem functions in a form
closely matching the mathematical description of the ODE system,
i.e. with components numbered from 1 instead of from 0.  The \id{Ith}
macro is used to access components of a vector of type \id{N\_Vector}
with a serial implementation.  It is defined using the {\nvecs}
accessor macro \id{NV\_Ith\_S} which numbers components starting with
0. The \id{IJth} macro is used to access elements of a dense matrix of
type \id{DenseMat}. It is defined using the {\dense} accessor macro
\id{DENSE\_ELEM} which numbers matrix rows and columns starting with
0. 
The macro \id{NV\_Ith\_S} is fully described in \ugref{ss:nvec_ser}.
The macro \id{DENSE\_ELEM} is fully described in \ugref{ss:djacFn}.

Next, the program includes some problem-specific constants, which are
isolated to this early location to make it easy to change them as needed.  
The program prologue ends with prototypes of four private helper
functions and the four user-supplied functions that are called by {\cvode}.

The \id{main} program begins with some dimensions and type declarations,
including use of the type \id{N\_Vector}.  The next several lines
allocate memory for the \id{y} vector using
\id{N\_VNew\_Serial} with a length argument of \id{NEQ} ($= 3$). The
lines following that load the initial values of the dependendent
variable vector into \id{y} using the \id{Ith} macro.

The calls to \id{N\_VNew\_Serial}, and also later calls to \id{CVode***}
functions, make use of a private function, \id{check\_flag}, which examines
the return value and prints a message if there was a failure.  The
\id{check\_flag} function was written to be used for any serial {\sundials}
application.

The call to \id{CVodeCreate} creates the {\cvode} solver memory block,
specifying the \id{CV\_BDF} integration method with \id{CV\_NEWTON} iteration.
Its return value is a pointer to that memory block for this
problem.  In the case of failure, the return value is \id{NULL}.  This
pointer must be passed in the remaining calls to {\cvode} functions.

The call to \id{CVodeMalloc} allocates the solver memory block.
Its arguments include the name of the {\C} function \id{f} defining the
right-hand side function $f(t,y)$, and the initial values of $t$ and $y$.
The argument \id{CV\_WF} specifies that a private function is provided
to compute the error weight vector and hence the next two arguments are
ignored by {\cvode} (we pass $0.0$ for the relative tolerance and \id{NULL}
for the absolute tolerance).
See \ugref{sss:cvodemalloc} for full details of this call.

The use of the \id{ewt} function is specified through the call to
\id{CVodeSetEwtFn} which also indicates that no user data is needed for the
evaluation of the error weights (last argument \id{NULL}).

The call to \id{CVodeRootInit} specifies that a rootfinding problem
is to be solved along with the integration of the ODE system, that the
root functions are specified in the function \id{g}, and that there are
two such functions.  Specifically, they are set to $y_1 - 0.0001$ and 
$y_3 - 0.01$, respectively.
See \ugref{ss:root_uc} for a detailed description of this call.

The calls to \id{CVDense} (see \ugref{sss:lin_solv_init}) and 
\id{CVDenseSetJacFn} (see \ugref{ss:optional_input}) specify the {\cvdense}
linear solver with an analytic Jacobian supplied by the user-supplied function
\id{Jac}.

The actual solution of the ODE initial value problem is accomplished in
the loop over values of the output time \id{tout}.  In each pass of the
loop, the program calls \id{CVode} in the \id{CV\_NORMAL} mode, meaning that
the integrator is to take steps until it overshoots \id{tout} and then
interpolate to $t = $\id{tout}, putting the computed value of $y$(\id{tout})
into \id{y}, with \id{t} = \id{tout}.  The return value in this case is
\id{CV\_SUCCESS}.  However, if \id{CVode} finds a root before reaching the next
value of \id{tout}, it returns \id{CV\_ROOT\_RETURN} and stores the root
location in \id{t} and the solution there in \id{y}.  In either case, the
program prints \id{t} and \id{y}.  In the case of a root, it calls
\id{CVodeGetRootInfo} to get a length-2 array \id{rootsfound} of bits showing
which root function was found to have a root.  If \id{CVode} returned any
negative value (indicating a failure), the program breaks out of the loop.  
In the case of a \id{CV\_SUCCESS} return, the value of \id{tout} is
advanced (multiplied by 10) and a counter (\id{iout}) is advanced, so
that the loop can be ended when that counter reaches the preset number
of output times, \id{NOUT} = 12.  See \ugref{sss:cvode} for full
details of the call to \id{CVode}.

Finally, the main program calls \id{PrintFinalStats} to get and print
all of the relevant statistical quantities.  It then calls \id{NV\_Destroy}
to free the vectors \id{y} and \id{abstol}, and \id{CVodeFree} to free the 
{\cvode} memory block.

The function \id{PrintFinalStats} used here is actually suitable for
general use in applications of {\cvode} to any problem with a dense 
Jacobian.  It calls various \id{CVodeGet***} and \id{CVDenseGet***}
functions to obtain the relevant counters, and then prints them.
Specifically, these are: the cumulative number of steps (\id{nst}), 
the number of \id{f} evaluations (\id{nfe}) (excluding those for
difference-quotient Jacobian evaluations),
the number of matrix factorizations (\id{nsetups}),
the number of \id{f} evaluations for Jacobian evaluations (\id{nfeD}
= 0 here),
the number of Jacobian evaluations (\id{njeD}),
the number of nonlinear (Newton) iterations (\id{nni}),
the number of nonlinear convergence failures (\id{ncfn}),
the number of local error test failures (\id{netf}), and
the number of \id{g} (root function) evaluations (\id{nge}).
These optional outputs are described in \ugref{ss:optional_output}.

The function \id{f} is a straightforward expression of the ODEs. 
It uses the user-defined macro \id{Ith} to extract the components of \id{y}
and to load the components of \id{ydot}.
See \ugref{ss:rhsFn} for a detailed specification of \id{f}.

Similarly, the function \id{g} defines the two functions, $g_0$ and $g_1$,
whose roots are to be found.  See \ugref{ss:root_us} for a detailed description
of the \id{g} function.

The function \id{Jac} sets the nonzero elements of the Jacobian as a
dense matrix.  (Zero elements need not be set because \id{J} is preset
to zero.)  It uses the user-defined macro \id{IJth} to reference the
elements of a dense matrix of type \id{DenseMat}.  Here the problem
size is small, so we need not worry about the inefficiency of using
\id{NV\_Ith\_S} and \id{DENSE\_ELEM} to access \id{N\_Vector} and
\id{DenseMat} elements.  Note that in this example, \id{Jac}
only accesses the \id{y} and \id{J} arguments.  See \ugref{ss:djacFn}
for a detailed description of the dense \id{Jac} function.

The function \id{ewt} defines the weights to be used in any WRMS norm
evaluation within {\cvode}. Note that its implementation is identical
to what {\cvode} would have use internally, had \id{ewt} not been defined.
This function is provided here only as an illustration of using this
feature of {\cvode}.

The output generated by \id{cvdx} is shown below.  It shows the output
values at the 12 preset values of \id{tout}.  It also shows the two root
locations found, first at a root of $g_1$, and then at a root of $g_0$.

%%
\includeOutput{cvdx}{../examples_ser/cvdx.out}
%%

%-------------------------------------------------------------------------------

\subsection{A banded example: \id{cvbx}}\label{ss:cvbx}

The example program \id{cvbx.c} solves the semi-discretized form of
the 2-D advection-diffusion equation
\vspace*{-.1in}
\begin{equation}
\label{eq:adeqn}
\partial v / \partial t = \partial^2 v / \partial x^2
  + .5 \partial v / \partial x + \partial^2 v / \partial y^2
\end{equation}
on a rectangle, with zero Dirichlet boundary conditions. The PDE is 
discretized with standard central finite differences on a 
(\id{MX}+2) $\times$ (\id{MY}+2) mesh, giving an ODE system of size
\id{MX*MY}.  The discrete value $v_{ij}$ approximates $v$ at $x = i \Delta x$,
$y = j \Delta y$. The ODEs are
%%\vspace*{-.05in}
\begin{equation}
\label{eq:cdiff}
\frac{dv_{ij}}{dt} = f_{ij} =
         \frac{v_{i-1,j} - 2 v_{ij} + v_{i+1,j}}{(\Delta x)^2}
       + .5  \frac{v_{i+1,j} - v_{i-1,j}}{2 \Delta x}
       + \frac{v_{i,j-1} - 2 v_{ij} + v_{i,j+1}}{(\Delta y)^2} \, ,
\end{equation}
where $1 \leq i \leq $\id{MX} and $1 \leq j \leq $\id{MY}.  The boundary
conditions are imposed by taking $v_{ij} = 0$ above if $i = 0$
or \id{MX}$ + 1$, or if $j = 0$ or \id{MY}$ + 1$. 
If we set $u_{(j-1)+(i-1)*\id{MY}} = v_{ij}$, so that the ODE system is
$\dot{u} = f(u)$, then the system Jacobian $J = \partial f / \partial u$ is
a band matrix with upper and lower half-bandwidths both equal to \id{MY}.
In the example, we take \id{MX} $= 10$ and \id{MY} $= 5$.
The source is listed in Appendix \ref{s:cvbx_c}.

The \id{cvbx.c} program includes files \id{cvband.h} and
\id{band.h} in order to use the {\cvband} linear solver. The \id{cvband.h}
file contains the prototype for the \id{CVBand} routine. The \id{band.h}
file contains the definition for band matrix type \id{BandMat} and the
\id{BAND\_COL} and \id{BAND\_COL\_ELEM} macros for accessing matrix
elements (see \ugref{ss:band}).
We have explicitly included \id{band.h}, but this is not necessary because
it is included by \id{cvband.h}.  The file \id{nvector\_serial.h} is
included for the definition of the serial \id{N\_Vector} type.

The include lines at the top of the file are followed by definitions of
problem constants which include the $x$ and $y$ mesh dimensions, \id{MX} and
\id{MY}, the number of equations \id{NEQ}, the scalar absolute tolerance
\id{ATOL}, the initial time \id{T0}, and the initial output time \id{T1}.

Spatial discretization of the PDE naturally produces an ODE system in
which equations are numbered by mesh coordinates $(i,j)$. The
user-defined macro \id{IJth} isolates the translation for the
mathematical two-dimensional index to the one-dimensional
\id{N\_Vector} index and allows the user to write clean, readable code
to access components of the dependent variable.  The \id{NV\_DATA\_S}
macro returns the component array for a given \id{N\_Vector}, and this
array is passed to \id{IJth} in order to do the actual \id{N\_Vector}
access.

The type \id{UserData} is a pointer to a structure containing problem
data used in the \id{f} and \id{Jac} functions.  This structure is
allocated and initialized at the beginning of \id{main}. The pointer
to it, called \id{data}, is passed to both \id{CVodeSetFData} and
\id{CVBandSetJacData}, and as a result it will be passed back to the
\id{f} and \id{Jac} functions each time they are called.  (If
appropriate, two different data structures could be defined and passed
to \id{f} and \id{Jac}.)  The use of the \id{data} pointer eliminates
the need for global program data.

The \id{main} program is straightforward.  The \id{CVodeCreate} call
specifies the \id{CV\_BDF} method with a \id{CV\_NEWTON} iteration. In the
\id{CVodeMalloc} call, the parameter \id{SS} indicates scalar relative
and absolute tolerances, and pointers \id{\&reltol} and \id{\&abstol} to
these values are passed.  
The call to \id{CVBand} (see \ugref{sss:lin_solv_init}) specifies the {\cvband}
linear solver, and specifies that both half-bandwidths of the Jacobian
are equal to \id{MY}.  
The call to \id{CVBandSetJacFn} (see \ugref{ss:optional_input}) specifies
that a user-supplied Jacobian function \id{Jac} is to be used and that a
pointer to \id{data} shold be passed to \id{Jac} every time it is called.
The actual solution of the problem is performed by
the call to \id{CVode} within the loop over the output times \id{tout}.
The max-norm of the solution vector (from a call to \id{N\_VMaxNorm}) and
the cumulative number of time steps (from a call to \id{CVodeGetNumSteps}) are
printed at each output time. Finally, the calls to \id{PrintFinalStats},
\id{N\_VDestroy}, and \id{CVodeFree} print statistics and free problem memory.

Following the \id{main} program in the \id{cvbx.c} file are definitions of
five functions: \id{f}, \id{Jac}, \id{SetIC}, \id{PrintFinalStats}, and
\id{check\_flag}.   The last three functions are called only from within the
\id{cvbx.c} file.  The \id{SetIC} function sets the initial dependent variable
vector; \id{PrintFinalStats} gets and prints statistics at the end of the run;
and \id{check\_flag} aids in checking return values.  The statistics
printed include counters such as the total number of steps (\id{nst}), 
\id{f} evaluations (excluding those for Jaobian evaluations) (\id{nfe}),
LU decompositions (\id{nsetups}), \id{f} evaluations for
difference-quotient Jacobians (\id{nfeB} = 0 here),
Jacobian evaluations (\id{njeB}), and nonlinear iterations (\id{nni}).
These optional outputs are described in \ugref{ss:optional_output}.
Note that \id{PrintFinalStats} is suitable for general use in applications
of {\cvode} to any problem with a banded Jacobian.

The \id{f} function implements the central difference approximation
(\ref{eq:cdiff}) with $u$ identically zero on the boundary. 
The constant coefficients $(\Delta x)^{-2}$, $.5(2 \Delta x)^{-1}$, and 
$(\Delta y)^{-2}$ are computed only once at the beginning of \id{main}, 
and stored in the locations \id{data->hdcoef}, \id{data->hacoef}, and
\id{data->vdcoef}, respectively.   When \id{f} receives the \id{data}
pointer (renamed \id{f\_data} here), it pulls out these values from storage
in the local variables \id{hordc}, \id{horac}, and \id{verdc}.  It then
uses these to construct the diffusion and advection terms, which are
combined to form \id{udot}.  Note the extra lines setting out-of-bounds
values of $u$ to zero.

The \id{Jac} function is an expression of the derivatives
\vspace*{-.08in}
\begin{eqnarray*}
\partial f_{ij} / \partial v_{ij} &=&
         -2 [(\Delta x)^{-2} + (\Delta y)^{-2}] \\
\partial f_{ij} / \partial v_{i \pm 1,j} &=& (\Delta x)^{-2} 
                  \pm .5 (2 \Delta x)^{-1}, ~~~~
\partial f_{ij} / \partial v_{i,j \pm 1}  =  (\Delta y)^{-2} ~. 
\end{eqnarray*}
This function loads the Jacobian by columns, and like \id{f} it
makes use of the preset coefficients in \id{data}.
It loops over the mesh points (\id{i},\id{j}). For each such mesh
point, the one-dimensional index \id{k = j-1 + (i-1)*MY} is computed
and the \id{k}th column of the Jacobian matrix $J$ is set. 
The row index $k'$ of each component $f_{i',j'}$ that depends on
$v_{i,j}$ must be identified in order to load the corresponding element.
The elements are loaded with the \id{BAND\_COL\_ELEM} macro.
Note that the formula for the global index $k$ implies that decreasing 
(increasing) \id{i} by $1$ corresponds to decreasing (increasing) 
\id{k} by \id{MY}, while decreasing (increasing) \id{j} by $1$ 
corresponds of decreasing (increasing) \id{k} by $1$. 
These statements are reflected in the arguments to \id{BAND\_COL\_ELEM}. 
The first argument passed to the \id{BAND\_COL\_ELEM} macro is a pointer to
the diagonal element in the column to be accessed. This pointer is obtained
via a call to the \id{BAND\_COL} macro and is stored in \id{kthCol} in
the \id{Jac} function. When setting the components of $J$ we must be
careful not to index out of bounds. The guards \id{(i != 1)} etc.
in front of the calls to \id{BAND\_COL\_ELEM} prevent illegal indexing.
See \ugref{ss:bjacFn} for a detailed description of the banded \id{Jac}
function.

The output generated by \id{cvbx} is shown below.

%%
\includeOutput{cvbx}{../examples_ser/cvbx.out}
%%

%-------------------------------------------------------------------------------

\subsection{A Krylov example: \id{cvkx}}\label{ss:cvkx}

We give here an example that illustrates the use of {\cvode} with the Krylov
method {\spgmr}, in the {\cvspgmr} module, as the linear system solver.  
The source file, \id{cvkx.c}, is listed in Appendix \ref{s:cvkx_c}.

This program solves the semi-discretized form of a pair of
kinetics-advection-diffusion partial differential equations, which
represent a simplified model for the transport, production, and loss
of ozone and the oxygen singlet in the upper atmosphere.  The problem
includes nonlinear diurnal kinetics, horizontal advection and diffusion, 
and nonuniform vertical diffusion.  The PDEs can be written as
\begin{equation}\label{cvkxpde}
  \frac{\partial c^i}{\partial t}=K_h\frac{\partial^2 c^i}{\partial x^2}
  +V \frac{\partial c^i}{\partial x}
  + \frac{\partial} {\partial y} K_v(y) \frac{\partial c^i}{\partial y}
  + R^i(c^1,c^2,t) \quad (i=1,2)~,
\end{equation}
where the superscripts $i$ are used to distinguish the two chemical
species, and where the reaction terms are given by
\begin{equation}\label{e:cvkx:r}
\begin{split}
  R^1(c^1,c^2,t) & = -q_1c^1c^3-q_2c^1c^2+2q_3(t)c^3+q_4(t)c^2 ~, \\
  R^2(c^1,c^2,t) & = q_1c^1c^3-q_2c^1c^2-q_4(t)c^2 ~.
\end{split}
\end{equation}
The spatial domain is $0 \leq x \leq 20,\;30 \leq y \leq 50$ (in {\it km}). 
The various constants and parameters are: $K_h=4.0\cdot 10^{-6},
~ V=10^{-3},~ K_v=10^{-8}\exp (y/5),~ q_1=1.63\cdot 10^{-16},
~ q_2=4.66\cdot 10^{-16},~ c^3=3.7\cdot 10^{16},$ and the diurnal
rate constants are defined as:
\begin{equation*}
q_i(t) = 
\left\{ \begin{array}{ll}
  \exp [-a_i/\sin \omega t], & \mbox{for } \sin \omega t>0 \\
  0, & \mbox{for } \sin \omega t\leq 0
  \end{array} \right\} ~~~(i=3,4) ~,
\end{equation*}
where $\omega =\pi /43200, ~ a_3=22.62,~ a_4=7.601.$  The time interval of
integration is $[0, 86400]$, representing 24 hours measured in seconds.

Homogeneous Neumann boundary conditions are imposed on each boundary, and the
initial conditions are 
\begin{equation} \label{cvkxic}
  \begin{split}
  c^{1}(x,y,0) &= 10^{6}\alpha (x)\beta (y) ~,~~~ 
                    c^{2}(x,y,0)=10^{12}\alpha(x)\beta (y) ~, \\
  \alpha (x) &= 1-(0.1x-1)^{2}+(0.1x-1)^{4}/2 ~, \\
  \beta (y) &= 1-(0.1y-4)^{2}+(0.1y-4)^{4}/2 ~.
  \end{split} 
\end{equation}
For this example, the equations (\ref{cvkxpde}) are discretized spatially
with standard central finite differences on a $10 \times 10$ mesh,
giving an ODE system of size $200$.

Among the initial \id{\#include} lines in this case are lines to
include \id{cvspgmr.h} and \id{sundialsmath.h}.  The first contains
constants and function prototypes associated with the {\spgmr} method,
including the values of the \id{pretype} argument to \id{CVSpgmr}.
The inclusion of \id{sundialsmath.h} is done to access the \id{SQR}
macro for the square of a \id{realtype} number.

The \id{main} program calls \id{CVodeCreate} specifying the \id{CV\_BDF} method
and \id{CV\_NEWTON} iteration, and then calls \id{CVodeMalloc} with scalar
tolerances.   
It calls  \id{CVSpgmr} (see \ugref{sss:lin_solv_init}) to specify the {\cvspgmr} 
linear solver with left preconditioning, and the default value (indicated by a zero argument)
for \id{maxl}.  The Gram-Schmidt orthogonalization is set to \id{MODIFIED\_GS}
through the function \id{CVSpgmrSetGSType}.  Next, user-supplied preconditioner
setup and solve functions, \id{Precond} and \id{PSolve}, as well as the 
\id{data} pointer passed to \id{Precond} and \id{PSolve} whenever these are called,
See \ugref{ss:optional_input} for details on the \id{CVSpgmrSetPreconditioner} function.

Then for a sequence of \id{tout} values, \id{CVode} is called in the
\id{CV\_NORMAL} mode, sampled output is printed, and the return value is
tested for error conditions.  After that, \id{PrintFinalStats} is called
to get and print final statistics, and memory is freed by calls to
\id{N\_VDestroy}, \id{FreeUserData}, and \id{CVodeFree}.  The printed
statistics include various counters, such as the total numbers of steps
(\id{nst}), of \id{f} evaluations (excluding those for $Jv$ product
evaluations) (\id{nfe}), of \id{f} evaluations for $Jv$ evaluations (\id{nfel}),
of nonlinear iterations (\id{nni}), of linear (Krylov) iterations (\id{nli}),
of preconditioner setups (\id{nsetups}), of preconditioner evaluations
(\id{npe}), and of preconditioner solves (\id{nps}), among others.  
Also printed are the lengths of the problem-dependent real and integer
workspaces used by the main integrator \id{CVode}, denoted \id{lenrw} and
\id{leniw}, and those used by {\cvspgmr}, denoted \id{llrw} and \id{lliw}.
All of these optional outputs are described in \ugref{ss:optional_output}.
The \id{PrintFinalStats} function is suitable for general use in applications
of {\cvode} to any problem with the {\spgmr} linear solver.

Mathematically, the dependent variable has three dimensions: species
number, $x$ mesh point, and $y$ mesh point.  But in {\nvecs}, a vector of
type \id{N\_Vector} works with a one-dimensional contiguous array of
data components. The macro \id{IJKth} isolates the translation from
three dimensions to one. Its use results in clearer code and makes it
easy to change the underlying layout of the three-dimensional data. 
Here the problem size is $200$, so we use the \id{NV\_DATA\_S} macro
for efficient \id{N\_Vector} access.  The \id{NV\_DATA\_S} macro gives
a pointer to the first component of an \id{N\_Vector} which we pass to
the \id{IJKth} macro to do an \id{N\_Vector} access.

The preconditioner used here is the block-diagonal part of the true Newton
matrix.  It is generated and factored in the \id{Precond} routine 
(see \ugref{ss:precondFn}) and backsolved in the \id{PSolve} routine 
(see \ugref{ss:psolveFn}). Its diagonal blocks are $2 \times 2$
matrices that include the interaction Jacobian elements and the diagonal
contribution of the diffusion Jacobian elements.  The block-diagonal part of
the Jacobian itself, $J_{bd}$, is saved in separate storage each time it is
generated, on calls to \id{Precond} with \id{jok}\id{ == FALSE}.
On calls with \id{jok}\id{ == TRUE}, signifying that saved Jacobian data
can be reused, the preconditioner $P = I - \gamma J_{bd}$ is formed from the
saved matrix $J_{bd}$ and factored.  (A call to \id{Precond} with
\id{jok}\id{ == TRUE} can only occur after a prior call with
\id{jok}\id{ == FALSE}.)  The \id{Precond} routine must also set the value
of \id{jcur}, i.e. \id{*jcurPtr}, to \id{TRUE} when $J_{bd}$ is re-evaluated,
and \id{FALSE} otherwise, to inform {\cvspgmr} of the status of Jacobian data.

We need to take a brief detour to explain one last important aspect of
the \id{cvkx.c} program.  The generic {\dense} solver contains two
sets of functions: one for ``large'' matrices and one for ``small''
matrices.  The large dense functions work with the type \id{DenseMat},
while the small dense functions work with \id{realtype **} as the
underlying dense matrix types.  The {\cvdense} linear solver uses the
type \id{DenseMat} for the $N \times N$ dense Jacobian and Newton
matrices, and calls the large matrix functions.  But to avoid the
extra layer of function calls, \id{cvkx.c} uses the small dense
functions for all operations on the $2 \times 2$ preconditioner blocks.  
Thus it includes \id{smalldense.h}, and calls the small dense matrix
functions \id{denalloc}, \id{dencopy}, \id{denscale}, \id{denaddI}, 
\id{denfree}, \id{denfreepiv}, \id{gefa}, and \id{gesl}.  The macro
\id{IJth} defined near the top of the file is used to access
individual elements in each preconditioner block, numbered from $1$.
The small dense functions are available for {\cvode} user programs
generally, and are documented in \ugref{ss:dense}.

In addition to the functions called by {\cvode}, \id{cvkx.c} includes
definitions of several private functions.  These are: \id{AllocUserData}
to allocate space for $J_{bd}$, $P$, and the pivot arrays; \id{InitUserData}
to load problem constants in the \id{data} block; \id{FreeUserData} to free
that block; \id{SetInitialProfiles} to load the initial values in \id{y}; 
\id{PrintOutput} to retreive and print selected solution values and
statistics; \id{PrintFinalStats} to print statistics; and \id{check\_flag}
to check return values for error conditions.

The output generated by \id{cvkx.c} is shown below.  Note that the
number of preconditioner evaluations, \id{npe}, is much smaller than
the number of preconditioner setups, \id{nsetups}, as a result of the
Jacobian re-use scheme.

%%
\includeOutput{cvkx}{../examples_ser/cvkx.out}
%%

%-------------------------------------------------------------------------------
