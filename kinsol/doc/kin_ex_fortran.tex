%===================================================================================
\section{Fortran example problems}\label{s:ex_fortran}
%===================================================================================

The {\F} example problem programs supplied with the {\kinsol}
package are all written in standard F77 Fortran and use double-precision
arithmetic. However, when the {\F} examples are built, the source code is
automatically modified according to the configure options supplied by the
user and the system type. Integer variables are declared as {\tt INTEGER*}{\em n},
where {\em n} denotes the number of bytes in the corresponding {\C} type
({\tt long int} or {\tt int}). Floating-point variable declarations remain
unchanged if double-precision is used, but are changed to {\tt REAL*}{\em n},
where {\em n} denotes the number of bytes in the {\sundials} type {\tt realtype},
if using single-precision. Also, if using single-precision, declarations of
floating-point constants are appropriately modified, e.g.; {\tt 0.5D-4} is
changed to {\tt 0.5E-4}.


The two examples supplied with the {\fkinsol} module are very simple tests 
of the {\F}-{\C} interface module. 
They solve the nonlinear system
\begin{equation*}
  F(u) = 0 \, , \quad \text{where } f_i(u) = u_i^2 - i^2 \, , 1 \le i \le N \, . 
\end{equation*}

\subsection{A serial example: fkinkryx}\label{ss:fkinkryx}

The \id{fkinkryx} program, for which the source code is listed in
Appendix~\ref{s:fkinkryx_f}, solves the above problem using the serial
{\nvecs} module.

The main program begins by calling \id{fnvinits} to initialize computations
with the serial {\nvecs} module. Next, the array \id{uu} is set to contain
the initial guess $u_i = 2 i$, the array \id{scale} is set with all 
components equal to $1.0$ (meaning that no scaling is done), and the array
\id{constr} is set with all components equal to $0.0$ to indicate that no 
inequality constraints should be imposed on the solution vector.

The {\kinsol} solver is initialized and memory for it is allocated by
calling \id{fkinmalloc}, which also specifies the maximum number of 
iterations between calls to the preconditioner setup routine (\id{msbpre} $=5$),
the tolerance for stopping based on the function norm (\id{fnormtol} = $10^{-5}$),
the tolerance for stopping based on the step length (\id{scsteptol} $=10^{-4}$),
and that no optional inputs are provided (\id{inopt} $=0$).

Next, the {\kinspgmr} linear solver module is attached to {\kinsol} by calling
\id{fkinspgmr}, which also specifies the maximum Krylov subspace dimension
(\id{maxl} $=10$) and the maximum number of restarts allowed for {\spgmr}
(\id{maxlrst} $=2$).  The {\kinspgmr} module is directed to use the supplied
preconditioner by  calling the \id{fkinspgmrsetprec} routine
with a first argument equal to $1$. The solution of the
nonlinear system is obtained after a successful return from \id{fkinsol}, which
is then printed to unit 6 (stdout).

Memory allocated for the {\kinsol} solver is released by calling \id{fkinfree} and
computations with the {\nvecs} module are terminated by calling \id{fnvfrees}.

The user-supplied routine \id{fkfun} contains a straightforward transcription
of the nonlinear system function $f$, while the routine \id{fkpset} sets the
array \id{pp} (in the common block \id{pcom}) to contain an approximation to 
the reciprocals of the Jacobian diagonal elements. The components of \id{pp} are
then used in \id{fkpsol} to solve the preconditioner linear system $Px=v$
through simple multiplications.

The following is sample output from \id{fkinkryx}, using $N = 128$.

\includeOutput{fkinkryx}{../fcmix/examples_ser/fkinkryx.out}

%%----------------------------------------------------------------------------

\subsection{A parallel example: fkinkryx\_p}\label{ss:fkinkryx_p}

The program \id{kindiapf}, listed in Appendix~\ref{s:fkinkryx_p_f}, is a
straightforward modification of \id{fkinkryx} to use the parallel MPI
{\nvecp} module.

After initialization of MPI, the {\nvecp} module is initialized by calling
\id{fnvinitp} with the local and global vector sizes as its first two
arguments.
The problem set-up ({\kinsol} initialization, {\kinspgmr} specification) and
solution steps are the same as in \id{fkinkryx}. 
Upon successful return from \id{fkinsol}, the solution segment local to 
the process with id equal to $0$ is printed to the screen.
Finally, the {\kinsol} memory is released, {\nvecp} computations are 
finalized, and the MPI environent is terminated. 

For this simple example, no inter-process communication is required to
evaluate the nonlinear system function $f$ or the preconditioner. 
As a consequence, the user-supplied routines \id{fkfun}, \id{fkpset}, and
\id{fkpsol} are basically identical to those in \id{fkinkryx}.

Sample output from \id{kindiapf}, for $N=128$, follows.

\includeOutput{fkinkryx\_p}{../fcmix/examples_par/fkinkryx_p.out}

