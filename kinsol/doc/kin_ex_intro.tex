%===============================================================================
\section{Introduction}\label{s:ex_intro}
%===============================================================================

This report is intended to serve as a companion document to the User
Documentation of {\kinsol} \cite{kinsol2.2.0_ug}.  It provides details, with
listings, on the example programs supplied with the {\kinsol} distribution
package.

The {\kinsol} distribution contains examples of four types: serial
{\C} examples, parallel {\C} examples, and serial and parallel {\F}
examples.  The following lists summarize all of these examples.
%%
Supplied in the \id{sundials/kinsol/examples\_ser} directory is the
following serial example (using the {\nvecs} module):

\begin{itemize}
\item \id{kinwebs}
  is a demonstration program for {\kinsol} with the Krylov linear solver.

  This program solves a nonlinear system that arises from a system
  of partial differential equations.  The PDE system is a six-species
  food web population model, with predator-prey interaction and diffusion
  on the unit square in two dimensions.

  The preconditioner matrix is a block-diagonal matrix based on
  the partial derivatives of the interaction terms only.
\end{itemize}
%%
Supplied in the \id{sundials/kinsol/examples\_par} directory are
the following two parallel examples (using the {\nvecp} module):
\begin{itemize}
\item \id{kinwebp}
  is a parallel implementation of \id{kinwebs}.
\item \id{kinwebbbd}
  solves the same problem as \id{kinwebp}, with a block-diagonal matrix
  with banded blocks as a preconditioner, generated by difference quotients,
  using the module {\kinbbdpre}.
\end{itemize}
%%
With the {\fkinsol} module, in the directories 
\id{sundials/kinsol/fcmix/examples\_ser} and
\id{sundials/kinsol/fcmix/examples\_par}, are the following examples for
the {\F}-{\C} interface:
\begin{itemize}
\item \id{kindiagsf}
  solves a nonlinear system of the form $u_i^2 = i^2$
  using an approximate diagonal preconditioner.
\item \id{kindiagpf}
  is a parallel implementation of \id{kindiagsf}.
\end{itemize}

In the following sections, we give detailed descriptions of some (but
not all) of these examples.  The Appendices contain complete listings
of those examples described below.  We also give our output files for
each of these examples, but users should be cautioned that their
results may differ slightly from these.  Differences in solution
values may differ within the tolerances, and differences in cumulative
counters, such as numbers of steps or Newton iterations, may differ
from one machine environment to another by as much as 10\% to 20\%.

In the descriptions below, we make frequent references to the {\kinsol}
User Document \cite{kinsol2.2.0_ug}.  All citations to specific sections
(e.g. \S\ref{s:types}) are references to parts of that User Document, unless
explicitly stated otherwise.
