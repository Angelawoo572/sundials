%===================================================================================
\chapter{Code Organization}\label{s:organization}
%===================================================================================

%----------------------------------
\section{SUNDIALS organization}\label{ss:sun_org}
%----------------------------------
% This is a shared SUNDIALS TEX file with description of
% the SUNDIALS organization
%
The family of solvers referred to as {\sundials} consists of the solvers
{\cvode} (for ODE systems), {\kinsol} (for nonlinear algebraic
systems), and {\ida} (for differential-algebraic systems).  In addition,
{\sundials} also includes variants of {\cvode} and {\ida} with sensitivity analysis 
capabilities (using either forward or adjoint methods): {\cvodes} and {\idas},
respectively.

The various solvers of this family share many subordinate modules.
For this reason, it is organized as a family, with a directory
structure that exploits that sharing (see Fig. \ref{f:sunorg}).
\begin{figure}
\subfigure[High-level diagram]
{\centerline{\psfig{figure=sunorg1.eps,width=\textwidth}}}
\subfigure[Directory structure of the source tree]
{\centerline{\psfig{figure=sunorg2.eps,width=\textwidth}}}
\caption {Organization of the SUNDIALS suite}\label{f:sunorg}
\end{figure}
The following is a list of the solver packages presently available:
\begin{itemize}

\item {\cvode},  
  a solver for stiff and nonstiff ODEs $dy/dt = f(t,y)$;

\item {\cvodes},
  a solver for stiff and nonstiff ODEs
  with sensitivity analysis capabilities;

\item {\ida},
  a solver for differential-algebraic systems $F(t,y,y^\prime) = 0$;

\item {\idas},
  a solver for differential-algebraic systems
  with sensitivity analysis capabilities;

\item {\kinsol}, 
  a solver for nonlinear algebraic systems $F(u) = 0$.

\end{itemize}


%----------------------------------
\section{KINSOL organization}\label{ss:kinsol_org}
%----------------------------------

\index{KINSOL@{\kinsol}!package structure}
The {\kinsol} package is written in the ANSI {\C} language. The following
summarizes the basic structure of the package, although knowledge
of this structure is not necessary for its use.

The overall organization of the {\kinsol} package is shown in Figure
\ref{f:kinorg}.
\begin{figure}
{\centerline{\psfig{figure=kinorg.eps,width=\textwidth}}}
\caption [Overall structure diagram of the {\kinsol} package]
{Overall structure diagram of the {\kinsol} package.
  Modules specific to {\kinsol} are distinguished by rounded boxes, while
  generic solver and auxiliary modules are in rectangular boxes.}
\label{f:kinorg}
\end{figure}

The central solver module, implemented in the files
\id{kinsol.h} and \id{kinsol.c}, deals with the solution of a nonlinear
algebraic system using either an Inexact Newton method or a line search method
for the global strategy. Although this module contains logic for the Newton
iteration, it has no knowledge of the method used to solve the linear
systems that arise. For any given user problem, the user must specify
which linear solver module to use.

\index{KINSOL@{\kinsol} linear solvers!list of|(}
At present, the package includes the following {\kinsol} linear system
module:
\begin{itemize}
\item {\kinspgmr}: scaled preconditioned GMRES method.
\end{itemize}
This set of linear solver modules is intended to be expanded in the
future as new algorithms are developed.
\index{KINSOL@{\kinsol} linear solvers!list of|)}

The {\kinspgmr} package includes an algorithm for the approximation
by difference quotients of the product between the Jacobian matrix and
a vector of appropriate length. The user has the option of providing
a routine for this operation.
With \index{preconditioning!setup and solve phases} {\kinspgmr},
the preconditioning must be supplied by the user, in two phases:
setup (preprocessing of Jacobian data) and solve.

\index{KINSOL@{\kinsol} linear solvers!implementation details|(}
A {\kinsol} linear solver module consists of four routines, devoted to (1)
memory allocation and initialization, (2) setup of the matrix data
involved, (3) solution of the system, and (4) freeing of memory.
The setup and solution phases are separate because the evaluation of
Jacobians and preconditioners is done only periodically during the
integration, as required to achieve convergence. The call list within
the central {\kinsol} module to each of the associated functions is
fixed, thus allowing the central module to be completely independent
of the linear system method.
\index{KINSOL@{\kinsol} linear solvers!implementation details|)}

Linear solver modules are also decomposed in another way.
\index{generic linear solvers!use in {\kinsol}|(}
The module {\kinspgmr} is a set of
interface routines built on top of a generic solver module {\spgmr}.
The interface deals with the use of these methods in the {\kinsol} context,
whereas the generic solver is independent of the context.
While the generic solvers here were generated with {\sundials} in mind, our
intention is that they be usable in other applications as
general-purpose solvers.  This separation also allows for any generic
solver to be replaced by an improved version, with no necessity to
revise the {\kinsol} package elsewhere.
\index{generic linear solvers!use in {\kinsol}|)}

{\kinsol} also provides a preconditioner module called {\kinbbdpre} which
works in conjunction with {\nvecp} and generates a preconditioner that is
a block-diagonal matrix with each block being a band matrix.

All state information used by {\kinsol} to solve a given problem is saved
in a structure, and a pointer to that structure is returned to the
user.  There is no global data in the {\kinsol} package, and so in this
respect it is reentrant. State information specific to the linear
solver is saved in a separate structure, a pointer to which resides in
the {\kinsol} memory structure. The reentrancy of {\kinsol} was motivated
by the anticipated multicomputer extension, but is also essential
in a uniprocessor setting where two or more problems are solved by
intermixed calls to the package from one user program.
