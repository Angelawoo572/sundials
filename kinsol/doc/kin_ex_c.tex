%===================================================================================
\section{C example problems}\label{s:ex_c}
%===================================================================================

\subsection{A serial example: \id{kinwebs}}\label{ss:kinwebs}

We give here an example that illustrates the use of {\kinsol} with the Krylov
method {\spgmr}, in the {\kinspgmr} module, as the linear system solver.
The source file, \id{kinwebs.c}, is listed in Appendix \ref{s:kinwebs_c}.

This program solves a nonlinear system that arises from a discretized system of partial
differential equations. The PDE system is a six-species food web population
model, with predator-prey interaction and diffusion on the unit square in
two dimensions. Given the dependent variable vector of species concentrations
$c = [c_1, c_2,..., c_{n_s}]^T$, where $n_s = 2 n_p$ is the number of species 
and $n_p$ is the number of predators and of prey, then
the PDEs can be written as
\begin{equation}\label{e:kinwebs_pde}
  d_i \cdot \left( \frac{\partial^2 c_i}{\partial x^2} + 
    \frac{\partial^2 c_i}{\partial y^2} \right) + f_i(x,y,c) = 0
  \quad (i=1,...,n_s) \, ,
\end{equation}
where the subscripts $i$ are used to distinguish the species, and where
\begin{equation}\label{e:kinwebs_fterm}
f_i(x,y,c) = c_i \cdot \left(b_i + \sum_{j=1}^{n_s} a_{i,j} \cdot c_j \right) \, .
\end{equation}
The problem coefficients are given by
\begin{equation*}
  a_{ij} = 
  \begin{cases}
    -1                 & i=j \\
    -0.5 \cdot 10^{-6} & i \leq n_p , ~ j > n_p  \\
    10^4               & i > n_p , ~ j \leq n_p  \\
    0                  & \mbox{all other } \, ,
  \end{cases}
\end{equation*}
%%
\begin{equation*}
  b_i = b_i(x,y) = 
  \begin{cases}
    1 + \alpha xy   & i \leq n_p  \\
    -1 - \alpha xy   & i > n_p \, ,
  \end{cases}
\end{equation*}
and
%%
\begin{equation*}
  d_i = 
  \begin{cases}
    1 & i \leq n_p  \\
    0.5 & i > n_p  \, .
  \end{cases}
\end{equation*}
The spatial domain is the unit square $(x,y) \in [0,1] \times [0,1]$.

Homogeneous Neumann boundary conditions are imposed and the initial
guess is constant in both $x$ and $y$. For this example, the equations
(\ref{e:kinwebs_pde}) are discretized spatially with standard central finite
differences on a $8 \times 8$ mesh with $n_s = 6$, giving a system of size $384$.

Among the initial \id{\#include} lines in this case are lines to
include \id{kinspgmr.h} and \id{sundialsmath.h}.  The first contains
constants and function prototypes associated with the {\spgmr} method.
The inclusion of \id{sundialsmath.h} is done to access the \id{MAX} and
\id{ABS} macros, and the \id{RSqrt} function to compute the square root
of a \id{realtype} number.

The \id{main} program calls \id{KINCreate} and then calls \id{KINMalloc} with the
name of the user-supplied system function \id{func} and solution vector as
arguments.  The \id{main} program then calls a number of \id{KINSet*}
routines to notify {\kinsol} of the function data pointer, the
positivity constraints on the solution, and convergence tolerances on
the system function and step size.
It calls  \id{KINSpgmr} (see \ugref{ss:lin_solv_init}) to specify the {\kinspgmr} 
linear solver, and passes a  value of $15$ as the maximum Krylov subspace dimension,
\id{maxl}.  Next, a maximum value of \id{maxlrst} $=2$ restarts is imposed and
the user-supplied preconditioner setup and solve functions, \id{PrecSetupBD} and
\id{PrecSolveBD}, are specified through calls to \id{KINSpgmrSetPrecSetupFn} and
\id{KINSpgmrSetPrecSolveFn}, respectively (see \ugref{ss:optional_input}). 
The \id{data} pointer passed to \id{KINSpgmrSetPrecData} is passed to \id{PrecSetupBD} 
and \id{PrecSolveBD} whenever these are called. 

Next, \id{KINSol} is called, the return value is tested for error conditions, and
the approximate solution vector is printed via a call to \id{PrintOutput}.
After that, \id{PrintFinalStats} is called to get and print final statistics, and
memory is freed by calls to \id{N\_VDestroy\_Serial}, \id{FreeUserData} and \id{KINFree}.
The statistics printed are the total numbers of nonlinear iterations (\id{nni}),
of \id{func} evaluations (excluding those for $Jv$ product evaluations) (\id{nfe}),
of \id{func} evaluations for $Jv$ evaluations (\id{nfeSG}), of linear (Krylov)
iterations (\id{nli}), of preconditioner evaluations (\id{npe}), and of
preconditioner solves (\id{nps}). All of these optional outputs and others are
described in \ugref{ss:optional_output}.

Mathematically, the dependent variable has three dimensions: species
number, $x$ mesh point, and $y$ mesh point.  But in {\nvecs}, a vector of
type \id{N\_Vector} works with a one-dimensional contiguous array of
data components. The macro \id{IJ\_Vptr} isolates the translation from
three dimensions to one. Its use results in clearer code and makes it
easy to change the underlying layout of the three-dimensional data. 
Here the problem size is $384$, so we use the \id{NV\_DATA\_S} macro
for efficient \id{N\_Vector} access. The \id{NV\_DATA\_S} macro gives
a pointer to the first component of a serial \id{N\_Vector} which is then
passed to the \id{IJ\_Vptr} macro.

The preconditioner used here is the block-diagonal part of the true Newton
matrix and is based only on the partial derivatives of the interaction terms $f$
in (\ref{e:kinwebs_fterm}) and hence its  diagonal blocks are $n_s \times n_s$ matrices
($n_s = 6$).
It is generated and factored in the \id{PrecSetupBD} routine and
backsolved in the \id{PrecSolveBD} routine.  
See \ugref{ss:precondFn} for detailed descriptions
of these preconditioner functions.

The program \id{kinwebs.c} uses the ``small'' dense functions for all operations 
on the $6 \times 6$ preconditioner blocks.  
Thus it includes \id{smalldense.h}, and calls the small dense matrix
functions \id{denalloc}, \id{denallocpiv}, 
\id{denfree}, \id{denfreepiv}, \id{gefa}, and \id{gesl}.
The small dense functions are generally available for {\kinsol} user programs
(for more information, see \ugref{ss:dense} or the comments in the header file
\id{smalldense.h}).

In addition to the functions called by {\kinsol}, \id{kinwebs.c} includes
definitions of several private functions.  These are: \id{AllocUserData}
to allocate space for $P$ and the pivot arrays; \id{InitUserData}
to load problem constants in the \id{data} block; \id{FreeUserData} to free
that block; \id{SetInitialProfiles} to load the initial values in \id{cc}; 
\id{PrintOutput} to retreive and print selected solution values;
\id{PrintFinalStats} to print statistics; and \id{check\_flag}
to check return values for error conditions.

The output generated by \id{kinwebs} is shown below.  Note that the
solution involved 7 Newton iterations, with an average of about 33
Krylov iterations per Newton iteration.

\includeOutput{kinwebs}{../examples_ser/kinwebs.out}

%-----------------------------------------------------------------------------------

\subsection{A parallel example: \id{kinwebbbd}}\label{ss:kinwebbbd}

In this example, \id{kinwebbbd}, we solve the same problem as with \id{kinwebs}
above, but in parallel, and instead of supplying the preconditioner we use the
{\kinbbdpre} module.  The source is given in Appendix \ref{s:kinwebbbd_c}.

{\kinbbdpre} generates and uses a band-block-diagonal preconditioner, generated
by difference quotients.  The upper and lower half-bandwidths of the Jacobian block
on each process are both equal to $2\cdot$\id{NUM\_SPECIES}$-1$, and that is the
value supplied as \id{mu} and \id{ml} in the call to \id{KINBBDPrecAlloc}. 

In this case, we think of the parallel MPI processes as
being laid out in a rectangle, and each process being assigned a
subgrid of size \id{MXSUB}$\times$\id{MYSUB} of the $x-y$ grid. If
there are \id{NPEX} processes in the $x$ direction and \id{NPEY}
processes in the $y$ direction, then the overall grid size is
\id{MX}$\times$\id{MY} with \id{MX}$=$\id{NPEX}$\times$\id{MXSUB} and
\id{MY}$=$\id{NPEY}$\times$\id{MYSUB}, and the size of the nonlinear system is
\id{NUM\_SPECIES}$\cdot$\id{MX}$\cdot$\id{MY}.  

The evaluation of the nonlinear system function is performed in \id{func}.
In this parallel setting, the processes first communicate
the subgrid boundary data and then compute the local components of the nonlinear
system function. The MPI communication is isolated in the private function \id{ccomm}
(which in turn calls \id{BRecvPost}, \id{BSend}, and \id{BRecvWait}) and the 
subgrid boundary data received from neighboring processes is loaded into the
work array \id{cext}. The computation of the nonlinear system function is done
in \id{func\_local} which starts by copying the local segment of the \id{cc} vector into
\id{cext} and then by imposing the boundary conditions by copying the first interior
mesh line from \id{cc} into \id{cext}. After this, the nonlinear system function is 
evaluated by using central finite-difference approximations using the data in \id{cext}
exclusively.

The function \id{func\_local} is also passed as the \id{gloc} argument to 
\id{KINBBDPrecAlloc}. Since all communication needed for the evaluation of the
local aproximation of $f$ used in building the band-block-diagonal preconditioner
is already done for the evaluation of $f$ in \id{func}, a \id{NULL} pointer is
passed as the \id{gcomm} argument to \id{KINBBDPrecAlloc}.

The \id{main} program resembles closely that of the \id{kinwebs} example, with
particularization arising from the use of the parallel MPI {\nvecp} module.
It begins by initializing MPI and obtaining the total number of processes and 
the id of the local process. The local length of the solution vector is then 
computed as \id{NUM\_SPECIES}$\cdot$\id{MXSUB}$\cdot$\id{MYSUB}.
Distributed vectors are created by calling the constructor defined in {\nvecp}
with the MPI communicator and the local and global problem sizes as arguments.
All output is performed only from the process with id equal to $0$.
Finally, after all memory deallocation, the MPI environment is terminated by
calling \id{MPI\_Finalize}.

The output generated by \id{kinwebbbd} is shown below.

\includeOutput{kinwebbbd}{../examples_par/kinwebbbd.out}


