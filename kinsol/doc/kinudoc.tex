% LaTeX source for KINSOL user's guide.
% Last revised 6 July 1998.

\documentclass[11pt]{article}
\usepackage{epsfig}

\setlength{\oddsidemargin}{0in}
\setlength{\evensidemargin}{0in}
\setlength{\textwidth}{6.5in}
\setlength{\textheight}{8.5in}

\title{User Documentation for KINSOL, a Nonlinear Solver for Sequential and 
Parallel Computers%
\thanks{
Research performed under the auspices of the U.S.
Department of Energy, by Lawrence Livermore National Laboratory under
contract W-7405-ENG-48.  Work supported by LDRD, Project 95-ER-036.} }

\author{Allan G. Taylor and Alan C. Hindmarsh%
\thanks{Center for Applied Scientific Computing, L-561,
        LLNL, Livermore, CA 94551.} }

\begin{document}

\thispagestyle{empty}

% Generate cover page
\vspace{2in}
\begin{center}
  {\LARGE
    User Documentation for KINSOL, \\ 
    A Nonlinear Solver \\
    for Sequential and Parallel Computers
    }
  \\
  \vspace{1in}
  {\large Allan G. Taylor}\\
  {\large\em Lawrence Livermore National Laboratory}\\
  \vspace{0.25in}
  {\large Alan C. Hindmarsh}\\
  {\large\em Lawrence Livermore National Laboratory}
\end{center}  
\vfill
% LLNL report info
{\large\it Center for Applied Scientific Computing}\\
\vspace{0.7in}\\
{\large UCRL-ID-131185}\\
{\large July 1998}

\newpage

\thispagestyle{empty}

% Disclaimer and preprint blurb page

\vspace*{1.0in}
\scriptsize
\begin{center}
\begin{minipage}{4.0in}
\begin{center}
DISCLAIMER
\end{center}
This document was prepared as an account of work sponsored by an
agency of the United States Government.
Neither the United States Government nor the University of California
nor any of their employees, makes any warranty, express or implied, or
assumes any legal liability or responsibility for the accuracy,
completeness, or usefulness of any information, apparatus, product, or
process disclosed, or represents that its use would not infringe
privately owned rights.
Reference herein to any specific commercial products, process, or
service by trade name, trademark, manufacturer, or otherwise, does not
necessarily constitute or imply its endorsement, recommendation, or
favoring by the United States Government or the University of
California.
The views and opinions of authors expressed herein do not necessarily
state or reflect those of the United States Government or the
University of California, and shall not be used for advertising or
product endorsement purposes.  
\par\vspace*{1.0in}
%\begin{center}
%PREPRINT
%\end{center}
%This is a preprint of a paper submitted to [journal/conference].  
%Since changes may be made before publication, this preprint is made
%available with the understanding that it will not be cited or
%reproduced without the permission of the author.
\end{minipage}
\end{center}

\normalsize

% End of cover/disclaimer pages


\maketitle

\setcounter{page}{1}

\section{Introduction}

KINSOL is a general purpose nonlinear system solver callable from either
C or Fortran programs. It is based on NKSOL \cite{BrSa90}, but is written in
ANSI-standard C rather than Fortran77. Its most notable feature is that it
uses Krylov Inexact Newton techniques in the system's approximate solution, 
thus sharing significant modules previously written within CASC at LLNL to 
support CVODE\cite{CoHi94,CoHi96}/PVODE\cite{LDRD98,PVODEusrguide}. It 
also requires almost no matrix storage for solving the Newton equations as 
compared to direct methods. The name \mbox{KINSOL} is derived from those 
techniques: Krylov Inexact Newton SOLver. The package was arranged so that 
selecting one of two forms of a single module in the compilation process will
allow the entire package to be created in either sequential (serial) or 
parallel form. The parallel version of KINSOL uses MPI (Message-Passing 
Interface) \cite{MPI} and an appropriately revised version of the vector module
{\tt NVECTOR}, as mentioned above, to achieve parallelism and portability. 
KINSOL in parallel form is intended for the SPMD (Single Program Multiple Data)
model with distributed memory, in which all vectors are identically distributed
across processors. In particular, the vector module {\tt NVECTOR} is designed 
to help the user assign a contiguous segment of a given vector to each of the 
processors for parallel computation. Several primitives were added to 
{\tt NVECTOR} as originally written for PVODE to implement KINSOL.

KINSOL has been run on a Cray-T3D, an eight-processor DEC ALPHA
and a cluster of workstations.  It is currently being used in a simulation
of tokamak edge plasmas and in groundwater two-phase flow studies at LLNL. 

The remainder of this paper is organized as follows: Section 2 sets
the mathematical notation and summarizes the basic methods. Section 3
summarizes the organization of the KINSOL solver, while Section 4
summarizes its usage.  Section 5 describes a preconditioner module,
Section 6 describes a set of Fortran/C interfaces, Section 7
describes an example problem, and Section 8 discusses availability.


\section{Mathematical Considerations}

The KINSOL code is a C implementation of a previous code, NKSOL, a nonlinear 
system solver written in Fortran by Brown and Saad \cite{BrSa90}.
 
The nonlinear system of equations
\begin{equation}
                  F(u) = 0, 
                             \label{nonlinear system}
\end{equation}
where $F(u)$ is a nonlinear function from ${\bf R}^N$ to ${\bf R}^N$, is 
solved by this package. An Inexact Newton method is applied to (\ref{nonlinear system}) 
resulting in the following iteration:

\vspace{0.2in}
{\bf Inexact Newton iteration}
\begin{enumerate}
   \item Set $u_0 = $ an initial guess
   \item For $n = 0, 1, 2,...$ until convergence do:\nonumber 
      \begin{itemize}
          \item[(a)] Solve $J(u_n)\delta_n = - F(u_n)$ 
      \item[(b)] Set $u_{n+1} = u_n + \delta_n$
          \item[(c)] Test for convergence
      \end{itemize}
\end{enumerate}
Here, $J(u_n) = F'(u_n)$ is the system Jacobian. As this code module is
anticipated for use on large systems, only iterative methods were considered 
to solve the system in step 2(a). These solutions are only approximate. Methods
of this type used for solution of nonlinear systems are called Inexact Newton 
methods. At each stage in the iteration process, a multiple of the 
approximate solution $\delta_n$  to the equation of step 2(a) is applied 
to the previously determined iterated approximate solution to produce a new 
approximate solution. Convergence is tested before iteration continues. The 
iterative method currently implemented is one of the class of Krylov methods.

As only the matrix-vector product $J(u)v$ is required in the Krylov method, in
this nonlinear equations setting that action is approximated by a difference
quotient of the form
\begin{equation}
    J(u)v  \approx \frac {F(u+\sigma v) - F(u)}{\sigma},  \label{jacobv}
\end{equation}
where $u$ is the current approximation to a root of (\ref{nonlinear system}) 
and $\sigma$ is a scalar, appropriately chosen to minimize numerical error in 
the computation of (\ref{jacobv}). An optional user-defined routine 
implementing this matrix-vector product is accommodated. See further details 
below in the section describing the routine {\tt KINSpgmr}.

To the above methods are added scaling and preconditioning. Scaling is allowed
for both the approximate solution vector and the system function vector. 
Additionally, right preconditioning is provided for if the preconditioning 
setup and solve routines are supplied by the user.

While only one linear solver is now implemented for use with this package, the
formal structure is in place for alternate solvers. The solver currently
implemented is the GMRES solver \cite{BrHi89,SaSc86} in module {\tt SPGMR} and
accessed via {\tt KINSPGMR}. Here GMRES stands for Generalized Minimal RESidual.
In most cases, performance of SPGMR is improved by user-supplied 
preconditioners.  

SPGMR is one of a class of preconditioned Krylov methods.  Write the
linear system  simply as
\begin{equation}
 A x = b . \label{Linsys}
\end{equation}
A preconditioned Krylov method for (\ref{Linsys}) involves a
preconditioner matrix $P$ that approximates $A$, but for which
linear systems $Px=b$ can be solved easily.  For preconditioning
on the right, the Krylov method is applied to the equivalent system
\[
 (A P^{-1}) (P x) = b .
\]
In KINSOL, the user may precondition the system on the right or use no 
preconditioner. In any case, the Krylov method (in our case GMRES) is applied to the transformed system
\[
 \bar{A} \bar{x} = \bar{b} .
\]
From an initial guess $\bar{x}_0$, an approximate solution 
$\bar{x}_m = \bar{x}_0 + z$ is obtained for $m = 1, 2, \ldots$ 
(until convergence), with $z$ chosen from the Krylov subspace
$K_m = span\{r_0, \bar{A}r_0, \ldots, \bar{A}^{m-1}r_0\}$ 
of dimension $m$, where $r_0$ is the initial residual 
$\bar{b} - \bar{A} \bar{x}_0$.  Each Krylov iteration requires one
matrix-vector multiply operation $\bar{A} v$, which is a combination
of multiplies by $A$ and by $P^{-1}$.  Multiplication of a given
vector $v$ by $A$ requires the product $Jv$, and that is approximated 
by a difference quotient $[F(u+\sigma v) - F(u)]/\sigma$.
Multiplication by $P^{-1}$ is to be provided by the user of the
solver, and is generally problem-dependent.  In the case of GMRES, 
the choice in $K_m$ is based on minimizing the $L_2$ norm of the
residual $\bar{b} - \bar{A} \bar{x}_m$ \cite{BrHi89,SaSc86}. When a given
$\bar{x}_m$ meets the linear system convergence criterion, $\bar{x}_m$ 
corresponds to the next increment $\delta_n$ in the solution of (\ref{nonlinear
system}) : $\delta_n$ is obtained from $\bar{x}_m$ by applying scaling
and preconditioning. The increment $\delta_n$ is then added to $u_n$ to form 
$u_{n+1}$ in step 2(b) by one of the strategies discussed below. The new
iterate $u_{n+1}$ is tested for (nonlinear) convergence in (\ref{nonlinear 
system}) , which is step 2(c) of the Inexact Newton iteration.


Two methods of applying a computed step $\delta_n$ to the previously computed
approximate solution vector are implemented. Denoted 'global strategies',
they attempt to use the direction implied by $\delta_n$ in the most efficient 
way in furthering convergence of the global (i.e., nonlinear) problem. The 
first and simplest is the Inexact Newton strategy.  A more advanced technique 
is implemented in the second strategy, called Linesearch. The so-called 
'Forcing Term' algorithms of Eisenstat and Walker \cite{EiWa96} to control the 
linear convergence tolerance are also implemented. 

A fundamental set of mathematical operations on $N$-vectors has been written
for both KINSOL and CVODE/PVODE. This set of computational kernels exists in
a distinct code module called {\tt NVECTOR}. By separating these frequent 
operations from the rest of the code, almost all operations in KINSOL with 
significant potential for parallel computation have been isolated. Then, two
different sets of kernels, both with the same routine names and a common 
interface, allow parallel computation to be very simply implemented in these 
codes. The operations done by this set of kernels are vector addition, 
scaling, and copy, vector norms, scalar products, and so forth.

\section{Code Organization}

\begin{figure}[p]
\centerline{\psfig{figure=kinorg.eps,height=7.0in}}
\caption{Overall structure of the KINSOL package.
  Modules comprising the central solver are distinguished by rounded
  boxes, while the user program, generic linear solvers, and auxiliary
  modules are in unrounded boxes.} 
\label{fig-kinorg}
\end{figure}

A way to visualize KINSOL is to think of the code as being organized
in layers, as shown in Fig. \ref{fig-kinorg}. Here, a module's name is used
to indicate the general function of the module's contents. Viewed
this way, the user's main program is at the top level. This program, with
associated user-supplied routines, makes various initialization calls, manages 
input/output, and calls the {\tt KINSOL} main module which carries out the 
system solution. At the next level 
down, the {\tt KINSOL} main module controls the iterative solution process, 
and is independent of the linear system method. {\tt KINSOL} calls the 
user-supplied function $F$, known as {\tt func} internally, and accesses the 
linear system solver. At the third level is found the linear system solver 
{\tt KINSPGMR}, which provides an interface to a generic solver for the SPGMR 
method, consisting of modules {\tt SPGMR} and {\tt ITERATIV}, {\tt KINSPGMR} 
also accesses the user-supplied preconditioner solve routine {\tt psolve}, if 
specified, and, if supplied, also accesses a user-supplied routine 
{\tt precondset} that computes and preprocesses the preconditioner. The 
{\tt precondset} routine is usually implemented by way of an approximate 
Jacobian matrix. Other linear system solvers may be added to the package in the
future. Such additions will be independent of the {\tt KINSOL} and 
{\tt KINSPGMR} modules. Several supporting modules reside at the fourth level.
These include {\tt LLNLTYPS}, {\tt LLNLMATH}, and {\tt NVECTOR}. The first of 
these defines types {\tt real} and {\tt integer}. The second specifies power 
functions, and the third is discussed further below.

The key to being able to move from the sequential computing
environment to the parallel computing environment lies in the {\tt NVECTOR}
module. This was briefly mentioned in the previous section.The idea is to 
distribute solution of the nonlinear system over several processors so that 
each processor is solving a contiguous subset of the system. This is achieved 
through the {\tt NVECTOR} module, which handles all calculations on $N$-vector
in a distributed manner, when the parallel version is compiled with parallel 
libraries. For any vector operation, each processor performs the operation on 
its contiguous 
elements of the input vectors, of length (say) {\tt Nlocal}, followed 
by a global reduction operation where needed.  In this way, vector 
calculations can be performed simultaneously with each processor 
working on its block of the vector. Vector kernels are designed to be 
used in a straightforward way for various vector operations that 
require the use of the entire distributed $N$-vector.  These kernels 
include dot products, various norms, linear sums, and so on. The key 
to simply handling both parallel and serial applications of a code 
lies in standardizing the interface to the vector kernels: both sequential and
parallel versions of {\tt NVECTOR} have an identical interface. In this way, 
one can access the kernels without referring directly to the underlying 
vector structure. This is assisted by using abstract data types that 
describe the machine environment data block ({\tt type machEnvType}) 
and all $N$-vectors ({\tt type N\_Vector}). Functions
to define a block of machine-dependent information and to free that block of 
information are also included in the vector module. Because the KINSOL
interface to the vector kernels is independent of the vector structure, the
user could supply their own kernel to best fit their application data 
structures. All references to parallelism are in the kernel, thus, the user
would handle all parallel aspects in this case.

As the algorithms used in NKSOL had several unique features, notably the way 
that constraints were handled \cite{BrSa90}, several new vector 
kernels were written and added to the module {\tt NVECTOR}. The changes, 
completely transparent to CVODE/PVODE, have now been incorporated in the
'common' version of {\tt NVECTOR}.

The parallel version of KINSOL uses the MPI (Message Passing
Interface) system \cite{MPI} for all inter-processor
communication. This achieves a high degree of portability, since MPI
is becoming widely accepted as a standard for message passing
software. For a different parallel computing environment, some rewriting of the
vector module could allow the use of other specific machine-dependent
instructions.

The coding style and structure of KINSOL was based on both style and structure
of the preexisting CVODE/PVODE codes. This was predicated upon the requirement
that the same vector kernel implementation and GMRES solvers be used in both 
codes. At the same time, those features somewhat unique to the Fortran language
(e.g., those constructs used in the original code NKSOL), were placed 
appropriately in a C language setting. Considerable simplification of the 
calling sequences resulted from this process. Of course, the resulting C 
language structure maintains relative privacy for definitions for each portion
of the code. The resulting code has proven to be readily adaptable to either
sequential or parallel execution by means of two versions of the module 
{\tt NVECTOR}.

\section{Using KINSOL}

This section is concerned with the use of KINSOL and consists of five
subsections. Those subsections treat the user-callable routines constituting 
the KINSOL interface in an overview and then in detail, give a layout or 
skeleton of the user's main program, and user-supplied functions or routines,
and discusses  C++/C interfacing. The listing of the sample program KINXP 
(a Predator-Prey PDE problem, P is for parallel version) in the 
Appendix may be particularly helpful. That code is intended to serve as a 
template to assist in preparations to use KINSOL and is included in the 
KINSOL distribution package. The sequential equivalent of KINXP, called KINXS, and
other variations and examples are found with KINSOL in the distribution
package.


\subsection{Overview of Routines and Their Usage}

The source code is organized in files (modules) as shown in Table 1. For each 
module there are two corresponding files. For example, {\tt KINSOL} requires 
both the files {\tt kinsol.c} and {\tt kinsol.h} .


\begin{table}%[htb]
\begin{center}
\caption{Modules in the KINSOL package}
\label{KINmodules}
\vspace{0.1 in}
\begin{tabular}{|l|l|l|} \hline
Module name & User-callable routines & other contents \\ \hline \hline
{\tt KINSOL} & {\tt KINMalloc}, {\tt KINSol}, & system function type {\tt SysFn};
 linear solver \\
            & {\tt KINFree}  & function pointers  {\tt linit, lsetup}, \\ 
            &       & {\tt lsolve, lfree} \\ \hline

{\tt KINSPGMR} &  {\tt KINSpgmr} &  {\tt KINSpgmrPrecondFn} type \\
         &           &  {\tt KINSpgmrPrecondSolveFn} type \\
         &           &  {\tt KINSpgmrAtimesFn} type \\ \hline
{\tt SPGMR}    & & {\tt SpgmrMalloc}, {\tt SpgmrSolve}, {\tt SpgmrFree} \\ \hline
{\tt ITERATIV} & & Routines in support of SPGMR \\ \hline
{\tt NVECTOR}  & {\tt PVecInitMPI},             & Type {\tt N\_Vector}; vector 
macros \\
         & {\tt PVecFreeMPI},            & {\tt N\_VMAKE}, {\tt N\_VDATA}, etc. \\
         & 19 other vector kernels  & \\ \hline
{\tt LLNLMATH} & & {\tt UnitRoundoff}, {\tt RPowerI}, {\tt RPowerR}, {\tt RSqrt}; \\
         & & Macros {\tt MIN, MAX, ABS, SQR} \\ \hline
{\tt LLNLTYPS} & & Types {\tt real}, {\tt integer}, {\tt boole} \\ \hline
\end{tabular}
\end{center}
\end{table}

\begin{table}%[hkb]
\begin{center}
\caption{User-supplied routines for KINSOL}
\label{user-supplied-routines}
\vspace{.1 in}
\begin{tabular}{|l|l|} \hline
typedef name  (* - optional) &  purpose of user-supplied routine \\ \hline \hline
{\tt SysFn} & the function ${F(u)}$, also known as {\tt func(uu)} \\  \hline
{\tt KINSpgmrPrecondFn}* & setup routine for preconditioner \\ \hline
{\tt KINSpgmrPrecondSolveFn}* & solve routine for preconditioner \\ \hline
{\tt KINSpgmruserAtimesFn}* & user-supplied {\tt Atimes} function \\ \hline
{\tt KINLocalFn}* & local computation function \\
 & (BBD preconditioner) \\ \hline
{\tt KINCommFn}* & interprocessor communication function \\
 & (BBD preconditioner) \\ \hline
\end{tabular}
\end{center}
\end{table}

In addition to routines supplied with KINSOL, there are several routines
either required or optional that the user can supply. They are outlined in
Table 2. Details and use of the last two routines listed there are discussed
in Section 5.


\subsection{Detailed description of routines}

This subsection uses extracts from header files for {\tt KINSOL} and 
{\tt KINSPGMR} to detail the arguments of user-callable routines. For each
routine, the declaration with arguments is followed by a section of comments. 
Please note that the system function ${F(u)}$ is called {\tt func(uu)} in the 
actual {\tt KINSOL} and {\tt KINSPGMR} source code. The independent variable 
$u$ is called {\tt uu} in those code modules as well.

\subsubsection{Memory allocation routine {\tt KINMalloc}}

\small
\begin{verbatim}

void *KINMalloc(integer Neq, FILE *msgfp, void *machEnv);

/******************************************************************
 *                                                                *
 * Function   : KINMalloc                                         *
 *                                                                *
 *            This function allocates main memory for the KINSol  *
 *            package. It also allocates several vectors of size  *
 *            Neq used by the package. Other N_Vectors are also   *
 *            to be allocated by the user and supplied to KINSol  *
 *             --------------------------                         *
 *  Neq   size of vectors being handled by the current memory     *
 *        allocation call to KINMalloc                            *
 *                                                                *
 *  msgfp  pointer to a FILE used to receive error messages from  *
 *         KINMalloc                                              *
 *                                                                *
 *****************************************************************
\end{verbatim}
\normalsize

\subsubsection{Main solver  {\tt KINSol}}

\small
\begin{verbatim}

int KINSol(void *kinmem, integer Neq, 
          N_Vector uu, SysFn func, int globalstrategy, 
          N_Vector uscale, N_Vector fscale,
          real fnormtol, real scsteptol, N_Vector constraints, 
          boole optIn, long int iopt[], real ropt[], void *f_data,
          FILE *msgfp, void *machEnv);

/******************************************************************
 *                                                                *
 * Function : KINSol                                              *
 *----------------------------------------------------------------*
 * KINSol initializes memory for a problem previously allocated by*
 * a call to KINMalloc. It also checks the initial value of uu    *
 * (the initial guess) against the constraints and checks if the  *
 * initial guess is a solution of the system. It then attempts to *
 * solve the system func(uu) = 0. , where the function func is    *
 * supplied by the user.  The input arguments for KINSol and their*
 * function are described below:                                  *
 *                                                                *
 * Neq     is the number of equations in the algebraic system or, *
 *         for a parallel problem, the number of variables        *
 *         assigned to the current processor                      *
 *                                                                *
 * kinmem  pointer to KINSol memory block returned by the         *
 *             preceding KINMalloc call                           *
 *                                            
 * uu      is the solution vector for the system func(uu) = 0.    *
 *          uu is to be set to an initial value if other          *
 *          than 0. vector starting value is desired              *
 *                                                                *
 * func    is the system function for the system:   func(uu) = 0. *          
 *                                                                *
 * globalstrategy  is a variable which indicates which global     *
 *         strategy to apply the computed increment delta in the  *
 *         solution uu.  Choices are :                            *
 *         INEXACT_NEWTON  or  LINESEARCH                         *
 *                                                                *
 * uscale  is an array (type N_Vector) of diagonal elements of the*
 *         scaling matrix for uu. The elements of uscale must be  *
 *         positive values. The scaling matrix uscale should be   *
 *         chosen so that uscale * uu (as a matrix multiplication)*
 *         should have all its components with roughly the same   *
 *         magnitude when uu is close to a root of func.          *
 *                                                                *
 * fscale  is an array (type N_Vector) of diagonal elements of the*
 *         scaling matrix for func. the elements of fscale must be*
 *         positive values.  The scaling matrix fscale should be  *
 *         chosen so that fscale * func(uu) (as a matrix          *
 *         multiplication) should have all its components with    *
 *         roughly the same magnitude when uu is NOT too near a   *
 *         root of func.                                          *
 *                                                                *
 * fnormtol  is a real (scalar) value containing the stopping     *
 *         tolerance on maxnorm( fscale * func(uu) ) .            *
 *         If fnormtol is input as 0., then a default value of    *
 *         (uround) to the 1/3 power will be used.                *
 *         uround is the unit roundoff for the machine            *
 *         in use for the calculation. (see UnitRoundoff in       *
 *         llnlmath module                                        *
 *                                    *
 * scsteptol  is a real (scalar) value containing the stopping    *
 *         tolerance on the maximum scaled step  uu(k) - uu(k-1). *
 *         If scsteptol is input as 0., then a default value of   *
 *         (uround) to the 2/3 power will be used.                *
 *         uround is the unit roundoff for the machine            *
 *         in use for the calculation. (see UnitRoundoff in       *
 *         llnlmath module                                        *
 *                                                                *
 * constraints  is an N_Vector defining inequality constraints    *
 *         for each component of the solution vector uu.  If      *
 *         constraints(i) = +2 or -2, then uu(i) will be          *
 *         constrained to be > 0.0 or < 0.0, respectively.  If    *
 *         constraints(i) = +1 or -1, then uu(i) will be          *
 *         constrained to be >= 0.0 or <= 0.0, respectively.  A   *
 *         zero value in constraints(i) implies there is no       *
 *         constraint on uu(i).  However, if constraints is a     *
 *         NULL pointer, then neither uu nor the maximum relative *
 *         change in uu will be constrained.                      *
 *                                                                *
 * optIn   is a flag (boole) indicating whether optional inputs   *
 *         from the user in the arrays iopt and ropt are to be    *
 *         used. Pass FALSE to ignore all optional inputs and TRUE*
 *         to use all optional inputs that are present.           *
 *         Either choice does NOT affect outputs in other         *
 *         positions of iopt or ropt.                             *
 *                                                                *
 * iopt    is the user-allocated array (of size OPT_SIZE) that    *
 *         will hold optional integer inputs and outputs.         *
 *         The user can pass NULL if he/she does not              *
 *         wish to use optional integer inputs or outputs.        *
 *         If optIn is TRUE, the user should preset to 0 those    *
 *         locations for which default values are to be used.     *
 *         Elements of iopt which have significance for either    *
 *         input or output parameters are:                        *
 *         PRINTFL, MXITER, PRECOND_NO_INIT, NNI ,NFE ,NBCF,      *
 *         NBKTRK, MXKRYL, and ETACHOICE                          *
 *                                                                *
 * ropt    is the user-allocated array (of size OPT_SIZE) that    *
 *         will hold optional real inputs and outputs.            *
 *         The user can pass NULL if he/she does not              *
 *         wish to use optional real inputs or outputs.           *
 *         If optIn is TRUE, the user should preset to 0.0 the    *
 *         optional input locations for which default values are  *
 *         to be used.                                            *
 *         Elements of iopt which have significance for either    *
 *         input or output parameters are:                        *
 *         MXNEWTSTEP, RELFUNC , RELU , FNORM , STEPL, ETACONST,  *
 *         ETAGAMMA, and ETAALPHA                                 * 
 *                                                                *
 *         Permissible iopt and ropt input parameters are given   *
 *         in a section below.                                    *
 *                                                                *
 * f_data  is a pointer to work space for use by the user-supplied*
 *         function func. The space allocated to f_data is        *
 *         allocated by the user's program before the call to     *
 *         KINMalloc                                              *
 *                                                                *
 * msgfp   is the file pointer for a message file where all KINSol*
 *         warning, error and informational messages will be      *
 *         written. This parameter can be stdout (standard output)*
 *         , stderr (standard error), a file pointer to a user    *
 *         created file, or NULL. If NULL is passed, then stdout  *
 *         (standard output) is used as a default                 *
 *                                                                *
 *                                                                *
 * machEnv is a pointer to machine environment-specific           *
 *         information. Pass NULL for the sequential case         *
 *            (see nvector.h)                                     *
 *                                                                *
 * If successful, KINMalloc returns a pointer to initialized      *
 * problem memory. This pointer should be passed to KINSol. If    *
 * an initialization error occurs, KINMalloc prints an error      *
 * message to the file specified by msgfp and returns NULL.       *
 *                                                                *
 ******************************************************************

\end{verbatim}
\normalsize

\subsubsection{Main solver {\tt KINSol} optional inputs and outputs}

The input of several optional input parameters is handled by placing
their values in appropriate elements of either {\tt iopt} or {\tt ropt} arrays.
Those optional input parameters and their permissible input values are now
discussed.
\small
\begin{verbatim}

 ******************************************************************
 *                                                                *
 * Optional Inputs and Outputs                                    *
 *----------------------------------------------------------------*
 * The user should declare two arrays for optional input and      *
 * output, an iopt array for optional integer input and output    *
 * and an ropt array for optional real input and output. These    *
 * arrays should both be of size OPT_SIZE.                        *
 * So the user's declaration should look like:                    *
 *                                                                *
 * long int iopt[OPT_SIZE];                                       *
 * real     ropt[OPT_SIZE];                                       *
 *                                                                *
 * The following definitions are indices into the iopt and ropt   *
 * arrays. A brief description of the contents of these positions *
 * follows.                                                       *
 *                                                                *
 * iopt[PRINTFL]    (input)  allows user to select from 4 levels  *
 *                  of output to FILE msgfp.                      *
 *                  =0 no statistics printed   (DEFAULT)          *
 *                  =1 output the nonlinear iteration count, the  *
 *                     scaled norm of func(uu), and number of     *
 *                     func calls.                                *
 *                  =2 same as 1 with the addition of global      *
 *                     strategy statistics:                       *
 *                     f1 = 0.5*norm(fscale*func(uu))**2   and    *
 *                     f1new = 0.5*norm(fscale*func(unew))**2 .   *
 *                  =3 same as 2 with the addition of further     *
 *                     Krylov iteration statistics.               *
 *                                                                *
 * iopt[MXITER]     (input) maximum allowable number of nonlinear *
 *                   iterations. The default is MXITER_DEFAULT  . *
 *                                                                *
 * iopt[PRECOND_NO_INIT] (input) Set to 1 to prevent the initial  *
 *                    call to the routine precondset upon a given *
 *                    call to KINSol. Set to 0 or leave unset to  *
 *                    force the initial call to precondset        *
 *                    Use the choice of 1 only after beginning the*
 *                    first of a series of calls with a 0 value   *
 *                    If a value other than 0 or 1 is encountered,*
 *                    the default, 0, is set in this element of   *
 *                    iopt and thus the routine precondset will   *
 *                    be called upon every call to KINSol, unless *
 *                    iopt[PRECOND_NO_INIT] is changed by the user*
 *                                                                *
 * iopt[ETACHOICE]   (input) a flag indicating which of three     *
 *                    methods to use for computing eta, the       *
 *                    coefficient in the linear solver            *
 *                    convergence tolerance eps   given by        *
 *                      eps = (eta+u_round)*norm(func(uu))        *
 *                    here, all norms are  the scaled L2 norm     *
 *                    The linear solver attempts to produce a step*
 *                    p such that norm(func(u)+J(uu)*p) <= eps    *
 *                    Two of the methods for computing eta        *
 *                    calculate a value based on the convergence  *
 *                    process in the routine KINForcingTerm.      *
 *                    The third method does not require           *
 *                    calculation; a constant eta is selected.    *
 *                                                                *
 *                    The default if iopt[ETACHOICE] is  not      *
 *                    specified is ETACHOICE1, (see below)        *
 *                                                                *
 *                    The allowed values (methods)  are:          *
 *               ETACONSTANT  constant eta, default of 0.1 or user*
 *                  supplied choice, for which see ropt[ETACONST],*
 *                                                                *
 *               ETACHOICE1 [default] which uses choice 1 of      *
 *                  Eisenstat and Walker's paper of SIAM J. Sci.  *
 *                  Comput.,17 (1996), pp 16-32 wherein eta is:   *
 *                          eta(k) =                              *
 *   ABS( norm(func(uu(k))) - norm(func(uu(k-1))+J(uu(k-1))*p) )  *
 *                       / norm(func(uu(k-1)))                    *
 *                                                                *
 *               ETACHOICE2   which uses choice 2 of              *
 *                  Eisenstat and Walker wherein eta is:          *
 *                  eta(k) = egamma *                             *
 *              ( norm(func(uu(k))) / norm(func(u(k-1))) )^ealpha *
 *                                                                *
 *                  egamma and ealpha for choice 2, both required,*
 *                  are from either defaults (egamma = 0.9 ,      *
 *                  ealpha = 2)  or from  user input,             *
 *                  see ropt[ETAALPHA] and ropt[ETAGAMMA], below. *
 *                                                                *
 *                  For eta(k) determined by either Choice 1 or   *
 *                  Choice 2, a value eta_safe is determined, and *
 *                  the safeguard   eta(k) <- max(eta_safe,eta(k))*
 *                  is applied to prevent eta(k) from becoming too*
 *                  small to quickly.                             *
 *                   For Choice 1,                                *
 *                     eta_safe = eta(k-1)^((1.+sqrt(5.))/2.)     *
 *            and    for Choice 2,                                *
 *                     eta_safe = egamma*eta(k-1)^ealpha.         *
 *                  (These safeguards are turned off if they drop *
 *                  below 0.1 . Also, eta is never allowed to be  *
 *                  less than eta_min = 1.e-4 .                   *
 *                                                                *
 *                                                                *
 * iopt[NNI]        (output) total number of nonlinear iterations *
 *                                                                *
 * iopt[NFE]        (output) total number of calls to the user-   *
 *                   supplied system function func.               *
 *                                                                *
 * iopt[NBCF]       (output) total number of times the beta       *
 *                   condition could not be met in the linesearch *
 *                   algorithm. The nonlinear iteration is halted *
 *                   if this value ever exceeds MXNBCF (10).      *
 *                                                                *
 * iopt[NBKTRK]     (output) total number of backtracks in the    *
 *                   linesearch algorithm.                        *
 *                                                                *
 * ropt[MXNEWTSTEP] (input) maximum allowable length of a newton  *
 *                   step. The default value is calculated from   *
 *                   1000*max(norm(uscale*uu(0),norm(uscale)).    *
 *                                                                *
 * ropt[RELFUNC]    (input) relative error in computing func(uu)  *
 *                   if known. Default is the machine epsilon.    *
 *                                                                *
 * ropt[RELU]       (input) a scalar constraint which restricts   *
 *                   the update of uu to  del(uu)/uu < ropt[RELU] *
 *                   The default is no constraint on the relative *
 *                   step in uu.                                  *
 *                                                                *
 * ropt[ETAGAMMA]   (input) the coefficient egamma in the eta     *
 *                   computation. See routine KINForcingTerm      *
 *            (SEE iopt[ETACHOICE] above for additional info)     *
 *                                                                *
 * ropt[ETAALPHA]   (input) the coefficient ealpha in the eta     *
 *                   computation. See routine KINForcingTerm      *
 *            (SEE iopt[ETACHOICE] above for additional info)     *
 *                                                                *
 * ropt[ETACONST]   (input)  a user specified constant value for  *
 *                    eta, used in lieu of that computed by       *
 *                        routine KINForcingTerm                  *
 *            (SEE iopt[ETACHOICE] above for additional info)     *
 *                Permissible ETACHOICE values are                *
 *                ETACHOICE1 (the default), ETACHOICE2, and       *
 *                ETACONST.                                       *
 *                                                                *
 * ropt[FNORM]      (output) the scaled norm at a given iteration:*
 *                   norm(fscale(func(uu))                        *
 *                                                                *
 * ropt[STEPL]      (output) last step length in the global       *
 *                   strategy routine:                            *
 *                   KINLineSearch or KINInexactNewton)           *
 *                                                                *

\end{verbatim}
\normalsize

\subsubsection{Main solver ({\tt KINSol}) return codes}

The return code values for the routine {\tt KINSol}, both for success and a
variety of possible failures, are given next.

\small
\begin{verbatim}
 *                                                                *
 *  KINSol returns an integer-valued termination code with the set*
 *       of possible values:                                      *
 *      KINSOL_NO_MEM,KINSOL_INPUT_ERROR,                         *
 *      KINSOL_SUCCESS, KINSOL_SCALED_LT_FNORM,                   * 
 *      KINSOL_LNSRCH_NONCONV, KINSOL_MAXITER_REACHED,            *
 *      KINSOL_MXNEWT_5X_EXCEEDED, KINSOL_LINESEARCH_BCFAIL,      *
 *      KINSOL_KRYLOV_FAILURE, KINSOL_PRECONDSET_FAILURE,         *
 *      KINSOL_PRECONDSOLVE_FAILURE,                              *
 *      KINSOL_INITIAL_GUESS_OK                                   *
 *                                                                *
 *     The meanings of these return codes are now given, each by  *
 *     the suffix portion of the respective code. That is,        *
 *     KINSOL_NO_MEM is listed in the descriptions below as NO_MEM*
 *                                                                *
 *                                                                *
 *  SUCCESS :    means maxnorm(fscale*func(uu) <= fnormtol, where *
 *               maxnorm() is the maximum norm function N_VMaxNorm*
 *               uu is probably an approximate root of func.      *
 *                                                                *
 *  SCALED_LT_FNORM:  means the scaled distance between the last  *
 *               two steps is less than scsteptol.  uu may be an  *
 *               approximate root of func, but it is also possible*
 *               that the algorithm is making very slow progress  *
 *               and is not near a root or that scsteptol is too  *
 *               large                                            *
 *                                                                *
 *  LNSRCH_NONCONV: means the LineSearch module failed to reduce  *
 *               norm(func) sufficiently on the last global step  *
 *               Either uu is close to a root of f and no more    *
 *               accuracy is possible, or the finite-difference   *
 *               approximation to j*v is inaccurate, or scsteptol *
 *               is too large. Check the outputs ncfl and nni: if *
 *               ncfl is close to nni, it may be the case that the*
 *               Krylov iteration is converging very slowly. In   *
 *               this case, the user may want to use precondition-*
 *               ing and/or increase the maxl value in the        * 
 *               KINSpgmr input list (that is,  increase the max  *
 *               dimension of the Krylov subspace by setting maxl *
 *               to nonzero (thus not using the default value of  *
 *               KINSPGMR_MAXL, or if maxl is being set, increase *
 *               its value                                        *
 *                                                                *
 *  MAXITER_REACHED: means that the maximum allowable number of   *
 *               nonlinear iterations has been reached. This is by*
 *               default 200, but may be changed through optional *
 *               input iopt[MXITER].                              *
 *                                                                *
 *  MXNEWT_5X_EXCEEDED: means 5 consecutive steps of length mxnewt*
 *               (maximum Newton stepsize limit) have been taken. *
 *               Either norm(f) asymptotes from above to a finite *
 *               value in some direction, or mxnewt is too small. *
 *               Mxnewt is computed internally (by default) as    *
 *               mxnewt = 1000*max(norm(uscale*uu0),1), where     *
 *               uu0 is the initial guess for uu, and norm() is   *
 *               the Euclidean norm N_VWrmsNorm(). Mxnewt can be  *
 *               set by the user through optional input           *
 *               ropt[MXNEWTSTEP].                                *
 *                                                                *
 *  LINESEARCH_BCFAIL: means that more than the allowed maximum   *
 *               number of failures (MXNBCF) occurred when trying *
 *               to satisfy the beta condition in the linesearch  *
 *               algorithm. It is likely that the iteration is    *
 *               making poor progress.                            *
 *                                                                *
 *  KRYLOV_FAILURE: means there was a failure of the Krylov       *
 *               iteration process to converge                    *
 *                                                                *
 *  PRECONDSET_FAILURE: means there was a nonrecoverable          *
 *               error in PrecondSet causing the iteration to halt*
 *                                                                *
 *                                                                *
 *  PRECONDSOLVE_FAILURE: means there was a nonrecoverable        *
 *            error in PrecondSolve causing the iteration to halt.*
 *                                                                *
 *  NO_MEM:    the KINSol memory pointer received was NULL        *
 *                                                                *
 *                                                                *
 *  INPUT_ERROR: one or more input parameters or arrays was in    *
 *               eror. See the listing in msgfp for further info  *
 **************************************************************** */
\end{verbatim}
\normalsize

\subsubsection{Deallocation routine {\tt KINFree}}

The next material describes the routine {\tt KINFree}. Note that it need
not be called after a specific {\tt KINSol} call but only when the
memory used by the {\tt KINSOL} package is to be released.

\small
\begin{verbatim}

void KINFree(void *kin_mem);
 
/******************************************************************
 *                                                                *
 * Function : KINFree                                             *
 *----------------------------------------------------------------*
 * KINFree frees the problem memory kinsol_mem allocated by       *
 * KINMalloc.  Its only argument is the pointer kinsol_mem        *
 * returned by KINMalloc   .                                      *
 *                                                                *
 ******************************************************************/
\end{verbatim}
\normalsize 

\subsubsection{Linear solver interface function definitions}


The linear solver package to be used with KINSOL interfaces with it 
via four routines of the type given below. Note that at present there
are only the four routines ({\tt KINSpgmrInit}, {\tt KINSpgmrSetup}, 
{\tt KINSpgmrSolve}, and {\tt KINSpgmrFree}) from the KINSPGMR package available.
In the following, each routine is named, followed by the generic description.
If a user wishes to implement another linear solver within KINSOL, the calling
conventions given below need to be followed as well as the entire interface
as used in {\tt KINSPGMR}.

\small

\begin{verbatim}

KINSpgmrInit:

/*******************************************************************
 *                                                                 *
 * int (*kin_linit)(KINMem kin_mem, boole *setupNonNull);          *
 *-----------------------------------------------------------------*
 * The purpose of kin_linit is to allocate memory for the          *
 * solver-specific fields in the structure *(kin_mem->kin_lmem) and*
 * perform any needed initializations of solver-specific memory,   *
 * such as counters/statistics. The kin_linit routine should set   *
 * *setupNonNull to be TRUE if the setup operation for the linear  *
 * solver is non-empty and FALSE if the setup operation does       *
 * nothing. An LInitFn should return LINIT_OK (== 0) if it has     *
 * successfully initialized the KINSol linear solver and LINIT_ERR *
 * (== -1) otherwise. These constants are defined above. If an     *
 * error does occur, an appropriate message should be sent to      *
 * (kin_mem->msgfp).                                               *
 *                                                                 *
 *******************************************************************/

KINSpgmrSetup:

/*******************************************************************
 *                                                                 *
 * int (*kin_lsetup)(KINMem kin_mem);                              *
 *-----------------------------------------------------------------*
 * The job of kin_lsetup is to prepare the linear solver for       *
 * subsequent calls to kin_lsolve.                                 *
 *                                                                 *
 * kin_mem - problem memory pointer of type KINMem. See the big    *
 *          typedef earlier in this file.                          *
 *                                                                 *
 * The kin_lsetup routine should return 0 if successful,           *
 * a positive value for a recoverable error, and a negative value  *
 * for an unrecoverable error.                                     *
 *                                                                 *
 *******************************************************************/

KINSpgmrSolve:

/*******************************************************************
 *                                                                 *
 * int (*kin_lsolve)(KINMem kin_mem, N_Vector bb, N_Vector xx,     *
 *                  real *res_norm);                               *
 *-----------------------------------------------------------------*
 * kin_lsolve must solve the linear equation J x = b, where        *
 * J is an approximate Jacobian matrix, x is the approximate system*
 * solution,  and the RHS vector b is input. The solution is to be *
 * returned in the vector b. kin_lsolve returns a positive value   *
 * for a recoverable error and a negative value for an             *
 * unrecoverable error. Success is indicated by a 0 return value.  *
 *                                                                 *
 *******************************************************************/

KINSpgmrFree:

/*******************************************************************
 *                                                                 *
 * void (*kin_lfree)(KINMem kin_mem);                              *
 *-----------------------------------------------------------------*
 * kin_lfree should free up any memory allocated by the linear     *
 * solver. This routine is called once a problem has been          *
 * completed and the linear solver is no longer needed.            *
 *                                                                 *
 *******************************************************************/

\end{verbatim}
\normalsize

\subsubsection{Linear solver routine {\tt KINSpgmr} and its optional outputs}

Pointers to the routines just described for the linear solver {\tt KINSPGMR} 
are 'set' in the KINSOL memory structure by the call to {\tt KINSpgmr}. No 
other action to prepare for those routines is required. {\tt KINSpgmr} is now 
described.

\small
\begin{verbatim}

void KINSpgmr(void *kin_mem, int maxl, int maxlrst, int msbpre,
              KINSpgmrPrecondFn precondset, 
          KINSpgmrPrecondSolveFn precondsolve,
          KINSpgmruserAtimesFn userAtimes,
              void *P_data);
     
/******************************************************************
 *                                                                *
 * Function : KINSpgmr                                            *
 *----------------------------------------------------------------*
 * A call to the KINSpgmr function links the main KINSol solver   *
 * with the KINSpgmr linear solver. Among other things, it sets   *
 * the generic names linit, lsetup, lsolve, and lfree to the      *
 * specific names for this package:                               *
 *                   KINSpgmrInit                                 *
 *                          KINSpgmrSetup                         *
 *                                  KINSpgmrSolve                 *
 *                                              KINSpgmrFree      *
 *                                                                *
 * kin_mem is the pointer to KINSol memory returned by            *
 *             KINSolMalloc.                                      *
 *                                                                *
 * maxl      is the maximum Krylov dimension. This is an          *
 *             optional input to the KINSpgmr solver. Pass 0 to   *
 *             use the default value MIN(Neq, KINSPGMR_MAXL=10).  *
 *                                                                *
 * maxlrst   is the maximum number of linear solver restarts      *
 *            allowed. Values outside the range 0 to 2*Neq/maxl   *
 *            will be restricted to that range. 0, meaning no     *
 *            restarts is a safe starting value.                  *
 *                                                                *
 * msbpre    is the maximum number of steps calling the solver    *
 *           precondsolve without calling the preconditioner      *
 *           precondset. (The default is KINSPGMR_MSBPRE = 10)    *
 *                                                                *
 * precondset  is the user's preconditioner routine. It is used to*
 *             evaluate and preprocess any Jacobian-related data  *
 *             needed by the precondsolve routine.  See the       *
 *             documentation for the type KINSpgmrPrecondFn for   *
 *             full details.  Pass NULL if no such setup of       *
 *             Jacobian data is required.  A precond routine is   *
 *             NOT required, but rather provided when needed by   *
 *             user's precondsolve routine                        *
 *                                                                *
 * precondsolve  is the user's preconditioner solve routine. It   *
 *             is used to solve Px=b, where P is a preconditioner *
 *             matrix.  See the documentation for the type        *
 *             KINSpgmrPrecondSolveFn for full details.  The only *
 *              case in which psolve is allowed to be NULL is when*
 *             no preconditioning is to be done. The NULL is taken*
 *             as a flag that preconditioning is not desired.     *
 *                                                                *
 * userAtimes   is an optional routine supplied by the user to    *
 *             perform the matrix-vector multiply J v, where J is *
 *             an approximate Jacobian matrix for that iteration. *
 *             Enter NULL if no such routine is required. If one  *
 *             is supplied, conforming to the definitions given   *
 *             in this file, enter its filename.                  *
 *                                                                *
 * P_data      is a pointer to user preconditioner data. This     *
 *             pointer is passed to precondset and precondsolve   *
 *             every time these routines are called.              *
 *                                                                *
 ******************************************************************/
  
\end{verbatim}
Four elements in the KINSOL array {\tt iopt} are used to return KINSPGMR
statistics. Those {\tt iopt} elements are indexed by constants
{\tt SPGMR\_NLI}, {\tt SPGMR\_NPE}, {\tt SPGMR\_NPS}, and {\tt SPGMR\_NCFL}, 
which are defined in file {\tt kinspgmr.h}. The meaning of each output 
parameter available for {\tt KINSpgmr} is explained next.

\small
\begin{verbatim}

/******************************************************************
 *                                                                *
 * KINSpgmr solver statistics indices                             *
 *----------------------------------------------------------------*
 * The following enumeration gives a symbolic name to each        *
 * KINSpgmr-specific statistic. The symbolic names are used as    *
 * indices into the iopt and ropt arrays and values of both arrays*
 * are set in this module                                         *
 *                                                                *
 * The KINSpgmr statistics are:                                   *
 *                                                                *
 * iopt[SPGMR_NLI]  (output) number of linear iterations.         *
 *                                                                *
 * iopt[SPGMR_NPE]  (output) number of preconditioner evaluations *
 *                                                                *
 * iopt[SPGMR_NPS]  (output) number of calls made to user's psolve*
 *                    function.                                   *
 *                                                                *
 * iopt[SPGMR_NCFL] (output) number of linear convergence failures*
 *                                                                *
 ******************************************************************/
\end{verbatim}
\normalsize

\normalsize
\subsection{A Skeleton of the User's Main Program}

The user's program must have the following steps in the order indicated:
\vspace{.2 in}
\begin{enumerate}

\item {\tt MPI\_Init(\&argc, \&argv);} to initialize MPI if used by
the user's program.  Here {\tt argc} and {\tt argv} are the command
line argument counter and array received by {\tt main}.

\item Set {\tt n}, the local vector length (the sub-vector
length for this processor); {\tt Neq}, the global vector length (the
problem size $N$, and the sum of all the values of {\tt Nlocal});
and the active set of processors.

\item {\tt machEnv = PVecInitMPI(comm, n, Neq, \&argc, \&argv);} to
initialize the NVECTOR module.  Here {\tt comm} is the MPI
communicator, which may be set in one of two ways: If a proper subset of active
processors is to be used, {\tt comm} must be set by suitable MPI
calls.  Otherwise, to specify that all processors are to be used, 
{\tt comm} must be either {\tt MPI\_COMM\_WORLD} or {\tt NULL}.

\item Set the vector {\tt u} of initial values.  Use the macro
{\tt N\_VMAKE(u, udata, machEnv);} if an existing array {\tt udata} 
contains the initial values of $u$.  Otherwise, make the call
{\tt u = N\_VNew(Neq, machEnv);} and load initial values into
the array defined by {\tt N\_VDATA(u)}.

\item {\tt kmem = KINMalloc(...);} which allocates internal
memory for KINSOL and returns a pointer to the KINSOL memory structure. 

\item {\tt KINSpgmr(...);}

\item {\tt ier = KINSol(kmem,  u, \ldots );} performs the solve.

\item {\tt N\_VDISPOSE;} or {\tt N\_VFree;} upon completion of the
integration, to deallocate the memory for the vector {\tt u}, allocated
by {\tt N\_VMAKE} or {\tt N\_VNew}, respectively.

\item {\tt KINFree(kmem);} to free the memory allocated for KINSOL.

\item {\tt PVecFreeMPI(machEnv);} to free machine-dependent data.

\end{enumerate}
\vspace{.2 in}
 A summary of these in practice, for both the serial and parallel case, is 
given next.
\vspace{.2 in}

\noindent
{Summary of Serial Usage of KINSOL} \label{serial-usage}
\vspace{.2 in}
\begin{enumerate}

\item {\tt msgfile=fopen("***.out","w")};

\item Allocate and initialize vectors and structures, as required.

\item  {\tt kmem= KINMalloc(SystemSize, msgfile,  NULL);}

\item  {\tt KINSpgmr(kmem, \ldots);}

\item  {\tt retcode=KINSol(kmem,...);}
    
\item  {\tt KINFree(mem);}

\end{enumerate}
\vspace{.2 in}
{Summary of Parallel Usage of KINSOL} \label{parallel-usage}
\vspace{.2 in}
\begin{enumerate}

\item  {\tt msgfile = fopen("test.out","w");}      Open message file, if 
desired.
                                     

\item  {\tt MPI\_Init();}     as {\tt PVecInitMPI}, below, also calls 
{\tt MPI\_Init}, this call is only required if the user's program uses MPI before step 3.


\item  Set local length {\tt n} and global length {\tt Neq}, and the active 
set of processors.


\item  {\tt machEnv = PVecInitMPI(comm, n, Neq, argc, argv);}
         {\tt comm} = MPI communicator (if set up by user), or
         {\tt comm} = {\tt MPI\_COMM\_WORLD} or {\tt NULL} (specifying all processors)   
    {\tt if (machEnv == NULL) return(1);}
   


\item  {\tt N\_VMAKE(u, udata, machEnv);}     
      or
    {\tt u = N\_VNew(Neq,machEnv);}     user sets up vectors, structures, etc.
  


\item  {\tt kmem = KINMalloc( Neq, msgfile, machEnv);}     initializes KINSOL 
    if stdout is to be used instead of a specific error message file, 
           enter NULL in place of msgfile.  
   

\item  {\tt KINSpgmr(\ldots)};  call the setup routine for the linear solver to be used.
          Note that only {\tt KINSpgmr} is available at present.
    
\item  {\tt flag= KINSol(kmem, Neq, u, func, ... , machEnv);}
       call the KINSOL main routine -- can be called repetitively with
       different functions {\tt func} and other options. A linear solver
       choice made in step 7, when another choice is available, cannot 
           be changed between {\tt KINSol} calls.   
   
\item {\tt N\_VDISPOSE}(  ); or {\tt N\_VFree}( ); call, as appropriate.

\item  {\tt KINFree(kmem)};       Free KINSOL memory, independent of machine   .
   


\item  {\tt PVecFreeMPI(machEnv);}        Free machine-dependent data.
\end{enumerate}
\vspace{.2 in}

Every usage of KINSOL requires at least the inclusion of the following header
files:  {\tt kinsol.h}, {\tt kinspgmr.h} or a future alternate solver, 
{\tt math.h}, {\tt llnltyps.h}, and {\tt nvector.h}. If the BBD preconditioner is
used, additional header files are required: {\tt kinbbdpre.h} and {\tt band.h}.
The header file {\tt mpi.h} is required for parallel applications of KINSOL.



\subsection{User-Supplied Functions}

The function defining the nonlinear system, called $F(u)$ in this report,
but {\tt func(uu)} in {\tt KINSOL} and {\tt KINSPGMR} internal usage, must be of
the form described by the following typedef extracted from {\tt KINSOL}:

\small
\begin{verbatim}

typedef void (*SysFn)(integer Neq, N_Vector uu, 
                      N_Vector fval,  void *f_data );

 /******************************************************************
 *                                                                *
 * Type : SysFn                                                   *
 *----------------------------------------------------------------*        
 * The func function which defines the system to be solved :      *
 *              func(uu) = 0      must have type SysFn.           *
 * func takes as input the problem size Neq and the dependent     *
 * variable vector uu. The function stores the result of func(uu) *
 * in fval. The necessary work space, besides uu and fval,  is    *
 * provided by the pointer f_data.                                *
 * The uu argument is of type N_Vector.                           *
 * A SysFn function does not have a return value.                 *
 *                                                                *
 ******************************************************************/

\end{verbatim}
\normalsize

Preconditioning is an important step in using KINSOL with any linear
solver. The interface for the routines defining the preconditioner
setup and solve routines for {\tt KINSPGMR} are given next.

\small
\begin{verbatim}

typedef int (*KINSpgmrPrecondFn)(integer Neq, 
                                 N_Vector uu, N_Vector uscale ,
                                 N_Vector fval, N_Vector fscale,
                                 N_Vector vtemp1, N_Vector vtemp2,
                                 SysFn func, real uround,
                                 long int *nfePtr, void *P_data);
  
/******************************************************************
 *                                                                *     
 * Type : KINSpgmrPrecondFn                                       *
 *----------------------------------------------------------------*
 * The user-supplied preconditioner setup function precondset and *
 * the user-supplied preconditioner solve function precondsolve   *
 * together must define the right preconditoner matrix P chosen   *
 * so as to provide an easier system for the Krylov solver        *
 * to solve. precondset is called to provide any matrix data      *
 * required by the subsequent call(s) to precondsolve. The data is*
 * stored in the memory allocated to P_data and the structuring of*
 * that memory is up to the user.     More specifically,          *
 * the user-supplied preconditioner setup function precondset     *
 * is to evaluate and preprocess any Jacobian-related data        *
 * needed by the preconditioner solve function precondsolve.      *
 * This might include forming a crude approximate Jacobian,       *
 * and performing an LU factorization on the resulting            *
 * approximation to J.  This function will not be called in       *
 * advance of every call to precondsolve, but instead will be     *
 * called only as often as necessary to achieve convergence       *
 * within the Newton iteration in KINSol.  If the precondsolve    *
 * function needs no preparation, the precondset function can be  *
 * NULL.                                                          *
 *                                                                *
 * precondset should not modify the contents of the arrays        *
 * uu or fval as those arrays are used elsewhere in the           *
 * iteration process.                                             *
 *                                                                *
 * Each call to the precondset function is preceded by a call to  *
 * the system function func. Thus the precondset function can use *
 * any auxiliary data that is computed by the func function and   *
 * saved in a way accessible to precondset.                       *
 *                                                                *
 * The two scaling arrays, fscale and uscale, and unit roundoff   *
 * uround are provided to the precondset function for possible use*
 * in approximating Jacobian data, e.g. by difference quotients.  *
 * These arrays should also not be altered                        *
 *                                                                *
 * A function precondset must have the prototype given below.     *
 * Its parameters are as follows:                                 *
 *                                                                *
 * Neq     is the length of all vector arguments.                 *
 *                                                                *
 * uu      an N_Vector giving the current iterate for the system. *
 *                                                                *
 * uscale  an N_Vector giving the diagonal entries of the uu-     *
 *         scaling matrix.                                        *
 *                                                                *
 * fval    an N_Vector giving the current function value          *
 *                                                                *
 * fscale  an N_Vector giving the diagonal entries of the func-   *
 *         scaling matrix.                                        *
 *                                                                *
 * vtemp1  an N_Vector temporary                                  *
 *                                                                *
 * vtemp2  an N_Vector temporary                                  *
 *                                                                *
 * func    the function func defines the system being solved:     *
 *         func(uu) = 0., and its name is passed initially to     *
 *         KINSol in the call to KINMalloc                        *
 *                                                                *
 * uround  is the machine unit roundoff.                          *
 *                                                                *
 * nfePtr  is a pointer to the memory location containing the     *
 *           KINSol problem data nfe = number of calls to func.   *
 *           The precondset routine should update this counter by *
 *           adding on the number of func calls made in order to  *
 *           approximate the Jacobian, if any.  For example, if   *
 *           the routine calls func a total of W times, then the  *
 *           update is *nfePtr += W.                              *
 *                                                                *
 * P_data is a pointer to user data - the same as the P_data      *
 *           parameter passed to KINSpgmr.                        *
 *                                                                *
 *                                                                *
 * Returned value:                                                *
 * The value to be returned by the precondset function is a flag  *
 * indicating whether it was successful.  This value should be    *
 *   0   if successful,                                           *
 *   1   if failure, in which case KINSol stops                   *
 *                                                                *
 *                                                                *
 ******************************************************************/
  

typedef int (*KINSpgmrPrecondSolveFn)(integer Neq,
                      N_Vector uu, N_Vector uscale, 
                      N_Vector fval, N_Vector fscale, 
                      N_Vector vtem, N_Vector ftem,
                      SysFn func, real u_round,
                      long int *nfePtr, void *P_data);
 
/******************************************************************
 *                                                                *           
 * Type : KINSpgmrPrecondSolveFn                                  *
 *----------------------------------------------------------------*
 * The user-supplied preconditioner solve function precondsolve   *
 * is to solve a linear system P x = r in which the matrix P is   *
 * the (right) preconditioner matrix P.                           *
 *                                                                *
 * precondset should not modify the contents of the iterate       *
 * array uu  or the current function value array  fval as those   *
 * are used elsewhere in the iteration process.                   *
 *                                                                *
 * A function precondsolve must have the prototype given below.   *
 * Its parameters are as follows:                                 *
 *                                                                *
 * Neq      is the length of all vector arguments.                *
 *                                                                *
 * uu      an N_Vector giving the current iterate for the system. *
 *                                                                *
 * uscale  an N_Vector giving the diagonal entries of the uu-     *
 *         scaling matrix.                                        *
 *                                                                *
 * fval    an N_Vector giving the current function value          *
 *                                                                *
 * fscale  an N_Vector giving the diagonal entries of the func-   *
 *         scaling matrix.                                        *
 *                                                                *
 * vtem    an N_Vector work array, holds the RHS vector on input  *
 *            and the result x on output/return                   *
 *                                                                *
 * ftem    an N_Vector work array, usually set on input as vtemp  *
 *                                                                *
 * func    the function func defines the system being solved:     *
 *         func(uu) = 0.                                          *
 *                                                                *
 * uround  is the machine unit roundoff.                          *
 *                                                                *
 * nfePtr is a pointer to the memory location containing the      *
 *          KINSol problem data nfe = number of calls to func. The*
 *          precondsolve routine should update this counter by    *
 *          adding  on the number of func calls made in order to  *
 *          carry out the solution, if any.  For example, if the  *
 *          routine calls func a total of W times, then the update*
 *          is  *nfePtr += W.                                     *
 *                                                                *
 * P_data is a pointer to user data - the same as the P_data      *
 *          parameter passed to KINSpgmr.                         *
 *                                                                *
 * Returned value:                                                *
 * The value to be returned by the precondsolve function is a flag*
 * indicating whether it was successful.  This value should be    *
 *   0 if successful,                                             *
 *   1 if failure, in which case KINSol stops                     *
 *                                                                *
 ******************************************************************/

\end{verbatim}
\normalsize
  
The matrix-vector multiply $J v$ may be done more efficiently on
occasion by an algorithm supplied by the user. This option is handled
by supplying a routine of type next described to {\tt KINSPGMR}, the
routine {\tt KINSpgmr}, in particular.

\small
\begin{verbatim}

typedef int (*KINSpgmruserAtimesFn)(void *f_data, N_Vector v, 
                    N_Vector z, boole *new_uu, 
                    N_Vector uu);
  
/******************************************************************
 *                                                                *
 * type : KINSpgmruserAtimesFn                                    *
 *                                                                *
 *    The user-supplied A times x routine (optional) where A is   *
 *    the Jacobian matrix dF/du or an approximation to it and v   *
 *    is a vector.        z = (J v) is computed                   *
 *                                                                *
 *   f_data is a pointer to the structure used to handle data for *
 *     the user-supplied system function and also to contain data *
 *     for evaluation of the Jacobian of func                     *
 *                                                                *
 *   v is the N_Vector to be multiplied by J                      *
 *   (preconditioned and unscaled as received)                    *
 *                                                                *
 *   z is the N_Vector resulting from the application of J to v   *
 *                                                                * 
 *   new_uu   is a flag indicating if a new_uu has been           *
 *    processed or not                                            *
 *                                                                *
 *   uu       is the N_Vector of the current iterate              *
 *                                                                *
 *****************************************************************/

\end{verbatim}
\normalsize
 
\subsection{Use by a C++ Application}

KINSOL has been written in so that it permits use by
applications written in C++ as well as in C.  For this purpose, each
KINSOL header file is wrapped with conditionally compiled lines reading
{\tt extern "C" \{ ... \}}, conditional on the variable 
{\tt \_\_cplusplus} being defined.  This directive causes the C++
compiler to use C-style names when compiling the function prototypes
encountered.  Users with C++ applications should also be aware that we
have defined, in {\tt llnltyps.h}, a boolean variable type, 
{\tt boole}, since C has no such type.  The type {\tt boole} is
equated to type {\tt int}, and so arguments in user calls, or calls to
user-supplied routines, which are of type {\tt boole} can be typed as
either {\tt boole} or {\tt int} by the user.  The same applies to
vector kernels which have a type {\tt boole} return value, if the user
is providing these kernels.


\section{A Band-Block-Diagonal Preconditioner Module}

A principal reason for using a parallel nonlinear system solver such as 
KINSOL lies in the solution of nonlinear systems arising in a partial 
differential equations (PDE) context.  Moreover,
the use of a Krylov iterative method for the solution of many such
problems is motivated by the nature of the underlying linear system of
equations that must be solved at each time step.  The
linear algebraic system is large, sparse, and structured. However, if
a Krylov iterative method is to be effective in this setting, then an
effective preconditioner needs to be used.  Otherwise, the rate of
convergence of the Krylov iterative method is usually unacceptably
slow.  Unfortunately, an effective preconditioner tends to be
problem-specific.

However, we have developed one type of preconditioner that treats a
rather broad class of problems.  It has been successfully
used for several realistic, large-scale problems and is
included in a software module within the KINSOL package. This module
generates a preconditioner that is a block-diagonal matrix with each
block being a band matrix. The blocks need not have the same number of
super- and sub-diagonals and these numbers may vary from block to
block. This Band-Block-Diagonal Preconditioner module is called
{\tt KINBBDPRE}.

One way to envision these preconditioners is to think of the domain of
the computational PDE problem as being subdivided into $M$ non-overlapping
subdomains.  Each of these subdomains is then assigned to one of the
$M$ processors to be used to solve the PDE system. The basic idea is
to isolate the preconditioning so that it is local to each processor,
and also to use a (possibly cheaper) approximate system
function. This requires the definition of a new function $g(u)$
which approximates the function $F(u)$ in the definition of the nonlinear
system (\ref{nonlinear system}). However, the user may set $g = F$.  
Corresponding to the domain decomposition, there is a decomposition of the 
solution vector $u$ into $M$ disjoint blocks $u_m$, and a decomposition of $g$
into blocks $g_m$.  The block $g_m$ depends on $u_m$ and also on
components of blocks $u_{m'}$ asssociated with neighboring subdomains
(so-called ghost-cell data).  Let $\bar{u}_m$ denote $u_m$ augmented
with those other components on which $g_m$ depends.  Then we have
\begin{equation}
g(u) = [g_1(\bar{u}_1), g_2(\bar{u}_2), \ldots, g_M(t,\bar{u}_M)]^T
\end{equation}
and each of the blocks $g_m(t, \bar{u}_m)$ is uncoupled from the others.

The preconditioner associated with this decomposition has the form 
\begin{equation}
P= diag[P_1, P_2, \ldots, P_M]
\end{equation}
where 
\begin{equation}
P_m \approx J_m
\end{equation}
and $J_m$ is a difference quotient approximation to 
$\partial g_m/\partial u_m$. This matrix is taken to be banded, with
upper and lower half-bandwidths {\tt mu} and {\tt ml} defined as
the number of non-zero diagonals above and below the main diagonal,
respectively. The difference quotient approximation is computed using
{\tt mu + ml + 2} evaluations of $g_m$. The parameters {\tt ml} and
{\tt mu} need not be the true half-bandwidths of the Jacobian of the
local block of $g$, if smaller values provide a more efficient
preconditioner. Also, they need not be the same on every processor.
The solution of the complete linear system

\begin{equation}
Px = b
\end{equation}
reduces to solving each of the equations 
\begin{equation}
P_m x_m = b_m
\end{equation}
and this is done by banded LU factorization of $P_m$ followed by a banded
backsolve.

To use this {\tt KINBBDPRE} module, the user must supply two functions which 
the module calls to construct $P$. These are in addition to the user-supplied
system function {\tt func}.

\begin{itemize}

\item A function {\tt glocfn(Nlocal, ulocal, glocal, f\_data)} must
be supplied by the user to compute $g(u)$. It loads the real array
{\tt glocal} as a function of {\tt t} and {\tt ulocal}.  
Both {\tt glocal} and {\tt ulocal} are of length {\tt Nlocal}, the
local vector length.

\item  A function {\tt gcomm(Nlocal, u, f\_data)} which must be supplied to
perform all inter-processor communications necessary for the execution of
the {\tt glocfn} function, using the input vector {\tt u} of type 
{\tt N\_Vector}.
\end{itemize}

Both functions take as input the same pointer {\tt f\_data} as that passed
by the user to {\tt KINMalloc} and passed to the user's function {\tt func},
and neither function has a return value. The user is responsible for
providing space (presumably within {\tt f\_data}) for components of {\tt u}
that are communicated by {\tt gcomm} from the other processors, and that are
then used by {\tt glocfn}, which is not expected to do any communication.

The user's calling program should include the following elements:

\begin{itemize}
\item  {\tt \#include "kinbbdpre.h"} for needed function prototypes and for
type {\tt KBBDData}.

\item  {\tt KBBDData p\_data};

\item  {\tt machEnv = PVecInitMPI(comm, Nlocal, N, argc, argv)};

\item  {\tt N\_VMake(u, udata, machEnv)};

\item  {\tt kmem = KINMalloc(N, F, \ldots )};

\item  {\tt p\_data = KBBDAlloc(Nlocal, mu, ml, \ldots, glocfn, gcomm, \ldots)};
where the 
\newline upper and lower half-bandwidths are {\tt mu} and {\tt ml},
respectively; and {\tt glocfn} and {\tt gcomm} are user-supplied
functions.

\item  {\tt KINSpgmr(kmem, maxl, maxlrst, msbpre, KBBDPrecon, KBBDPSol,}
\newline {\tt userAtimes, p\_data)}; with the memory pointers 
{\tt kmem} and {\tt p\_data} returned by the two previous calls,
the parameters ({\tt maxl}, {\tt maxlrst}, and {\tt msbpre}) and the
names of the preconditioner routines ({\tt KBBDPrecon, KBBDPSol})
supplied with the KINBBDPRE module. If a user-supplied matrix-vector multiply
routine,{\tt userAtimes}, is supplied, it also is entered here.

\item  {\tt ier = KINSol(cvode\_mem, u \ldots )}; to carry out the
KINSOL solution.

\item  {\tt KBBDFree(p\_data)}; to free the {\tt KBBDPRE} memory block.

\item  {\tt KINFree(kmem)}; to free the KINSOL memory block.

\item  {\tt PVecFreeMPI(machEnv)}; to free the KINSOL MPI memory block.
\end{itemize}

Three optional outputs associated with this module are available by way of
macros. These are:

\begin{description}
\item  {\tt KBBD\_RPWSIZE(p\_data)} = size of the real workspace (local to
the current processor) used by {\tt KINBBDPRE}.

\item  {\tt KBBD\_IPWSIZE(p\_data)} = size of the integer workspace (local to
the current processor) used by {\tt KINBBDPRE}.

\item  {\tt KBBD\_NGE(p\_data)} = cumulative number of $g$ evaluations (calls
to {\tt glocfn}) so far.
\end{description}

The costs associated with {\tt KINBBDPRE} also include {\tt npe} LU
factorizations, {\tt npe} calls to {\tt gcomm}, and {\tt nps} banded
backsolve calls, where {\tt npe} and {\tt nps} are optional KINSOL
outputs.

Similar block-diagonal preconditioners could be considered with different
treatment of the blocks $P_m$. For example, incomplete LU factorization or
an iterative method could be used instead of banded LU factorization.


\section{The Fortran/C Interface Package}

We anticipate that many users of KINSOL will work from existing Fortran
application programs.  To accommodate them, we have provided a set of
interface routines that make the required connections to KINSOL with a
minimum of changes to the application programs.  Specifically, a
Fortran/C interface package called FKINSOL is a collection of C
language functions and header files which enables the user to write a
main program and all user-supplied subroutines in Fortran and to use
the C language KINSOL package.  This package entails some compromises
in portability, but we have kept these to a minimum by requiring fixed
names for user-supplied routines, and by using a name-mapping scheme
to set the names of externals in the Fortran/C linkages.  The latter
depends on two parameters set in a small header file.


Since a user cannot successfully link a program where any routine calls a 
Fortran routine not supplied, it is necessary that there be six choices for 
the {\tt FKINSPGMR} routine. {\tt FKINSPGMR00} is found in {\tt fkinsol.c} but
the others are in separate files to simplify linking. Each calls the routine 
{\tt KINSpgmr} (a C module) but with different options. The first of two suffix
digits indicates whether the number of routines supplied is 0 (no 
preconditioning), 1 (preconditioner solve only), or 2 (both preconditioner 
setup and solve routines). The second digit indicates whether or not a 
{\tt userAtimes} routine routine is supplied in Fortran. For example, if 
{\tt FKINSPGMR11} is called from the Fortran main, it will be necessary that 
the user supply as well the routines {\tt FPSOL} and {\tt FATIMES}. In this 
way, dummy routines named {\tt FPSOL}, {\tt FATIMES}, etc., are not required.


The Fortran/C interfaces have been tested on a Cray-T3D, a DEC ALPHA, and a 
cluster of Sun workstations.

A similar interface package, called {\tt FKINBBD}, has been written for the 
{\tt KINBBDPRE} preconditioner module. It works in conjunction with the
{\tt FKINSOL} interface package. The additional user-callable functions 
here are: {\tt FKBBDINIT0} and {\tt FKINBBDINIT1}, which interface with 
{\tt KBBDAlloc} and {\tt KINSpgmr};
{\tt FKINBBDOPT}, which accesses optional outputs; and {\tt FKINBBDFREE}, which
interfaces with {\tt KBBDFree}. The two user-supplied Fortran subroutines
required, in addition to {\tt KFUN} to define $F$, are: {\tt KLOCFN}, which
computes $g(u)$; and {\tt KCOMMFN}, which performs communications necessary
for {\tt KLOCFN}.

An overview of the Fortran interface and a skeleton program illustrating their
use follow.

\subsection{Overview of Fortran interface routines}

The routines used to interface with a Fortran main program and with Fortran
user-supplied routines are summarized below.  Further details can be
found in the header file {\tt fkinsol.h}.  Also, the user should check,
and reset if necessary, the parameters in the file {\tt fcmixpar.h}.
The functions which are callable from the user's Fortran program are
as follows:

\begin{itemize}
\item  {\tt FKINITMPI} interfaces with {\tt PVecInitMPI} and is used to
initialize the NVECTOR module.

\item  {\tt FPKINMALLOC} interfaces with {\tt KINMalloc} and is used to
initialize {\tt KINSol}.

\item {\tt FKINSPGMR00, FKINSPGMR01, FKINSPGMR10, FKINSPGMR11, FKINSPGMR20}, 
and \newline
{\tt FKINSPGMR21} interface with {\tt KINSpgmr} 
when SPGMR has been chosen as the linear system solver (the only choice at 
present). These six interface routines correspond to the cases of no 
preconditioning, preconditioning with no saved matrix data, and preconditioning
with saved matrix data, respectively, each  without or with a user-supplied 
Jacobian-vector multiply ({\tt FATIMES}) routine. For example, {\tt FKINSPGMR11}
uses conditioning but no setup routine ({\tt psolve} but no {\tt precondset}) 
and also the user has supplied a routine {\tt FATIMES} that performs the 
Jacobian-vector multiply    used in the GMRES solver.

\item  {\tt FKINSOL} interfaces with {\tt KINSol}.

\item  {\tt FKINFREE} interfaces with {\tt KINFree} and is used to free
memory allocated for {\tt CVode}.

\item  {\tt FKFREEMPI} interfaces with {\tt PVecFreeMPI} and is used to free
memory allocated for MPI.
\end{itemize}


 { Fortran interface modules and routines:} \label{fortran-interface}
\begin{table} [htb]
\begin{center}
\begin{tabular}{|l|l|} \hline
MODULE & Fortran-callable routine \\ \hline \hline
    
{\tt FKINSOL} & {\tt FKINITMPI, FKFREEMPI, FPKINMALLOC,} \\ 
           &{\tt FKINFREE, FKINSPGMR00, FKINSOL} \\ \hline
{\tt FKINSPGMR01}  &  {\tt FKINSPGMR01} \\ \hline
{\tt FKINSPGMR10}  &  {\tt FKINSPGMR10} \\ \hline
{\tt FKINSPGMR11}  &  {\tt FKINSPGMR11} \\ \hline
{\tt FKINSPGMR20}  &  {\tt FKINSPGMR20} \\ \hline
{\tt FKINSPGMR21}  &  {\tt FKINSPGMR21} \\ \hline
\end{tabular}
\end{center}
\end{table}


The user-supplied Fortran subroutines are as follows.  The names of
these routines are fixed and are case-sensitive.

\begin{itemize}
\item  {\tt KFUN} which defines the function, $F$, that described the system
to be solved  $F(u) = 0$.

\item  {\tt KPSOL} which solves the preconditioner equation, and is required
if preconditioning is used.

\item  {\tt KPRECO} which computes the preconditioner, and is required if
preconditioning involves pre-computed matrix data.

\item {\tt FATIMES} which performs a Jacobian-vector product paralleling the
C routine userAtimes.

\item {\tt KLOCFN} which performs the local computation of the system function
as required for the BBD preconditioner.

\item {\tt KCOMMFN} which performs the communication of function values between
processors as
required for the BBD preconditioner.

\end{itemize}

\begin{table}[htb]
\begin{center}
\begin{tabular}{|l|l|} \hline
  &Routines to be provided by the user: \\ 
  &   (* indicates optional) \\ \hline \hline
{\tt KFUN} & user-supplied Fortran system function \\ \hline
{\tt KPRECO}* & user-supplied Fortran preconditioner setup * \\ \hline
{\tt KPSOL}* & user-supplied Fortran preconditioner solve * \\ \hline
{\tt FATIMES}* & user-supplied Fortran Atimes * \\ \hline
{\tt KLOCFN}* & for BBD preconditioner/Fortran interface* \\ \hline
{\tt KCOMMFN}* & for BBD preconditioner/Fortran interface* \\ \hline
\end{tabular}
\end{center}
\end{table}


\subsection{Skeleton of Fortran usage}

The two summaries of usage in a Fortran context are brief but follow the
pattern established above for the C interface. 

{Summary of Parallel Usage of KINSOL, using the Fortran interface:} \label{Fparallel}

\begin{enumerate}

\item  {\tt call MPI\_INIT(...)}        Initialize MPI.
\item {\tt call FKINITMPI(nlocal, neq, ier)} Initialize the NVECTOR interface to MPI.
      Here, {\tt nlocal} and {\tt neq} are the local and global sizes of vectors to be used.

\item  {\tt call MPI\_COMM\_SIZE(...)} or {\tt call MPI\_COMM\_RANK(...)} Optional calls to 
    determine logical processor number and count, part of MPI, proper.

\item  {\tt call FPKINMALLOC(...)} Allocate space for {\tt KINSOL}.

\item  {\tt call FKINSPGMR20(...)} Set up the linear solver. The
    choice illustrated here is for both a setup and solve preconditioner 
    routine to be supplied by the user in Fortran, but no user-supplied
    {\tt FATIMES} routine.

\item  {\tt call FKINSOL(...)} Call {\tt KINSol}, through the Fortran interface.

\item  {\tt call FKINFREE}     Free memory usage by KINSOL and its Fortran interface.

\item  {\tt call FKFREEMPI} Free MPI interface.
\end{enumerate}

{Summary of Serial Usage of KINSOL, using the Fortran interface}: \label{Fserial}

\begin{enumerate}

\item  {\tt call FPKINMALLOC(...)} Allocate space for {\tt KINSOL}.

\item  {\tt call FKINSPGMR20(...)} Set up the linear solver. The
    choice illustrated here is for both a setup and solve preconditioner 
    routine to be supplied by the user in Fortran, but no user-supplied
    {\tt FATIMES} routine.

\item  {\tt call FKINSOL(...)} Call {\tt KINSol}, through the Fortran interface.

\item  {\tt call FKINFREE}     Free memory usage by KINSOL and its Fortran 
          interface
\end{enumerate}


\section{Example Problems}


Although a trivial diagonal example is supplied with the distribution package, 
the following example,  the so-called predator-prey PDE system, is more 
illustrative of the power of KINSOL with real problems. This particular 
problem, outlined below, was solved by both a sequential and parallel
implementation of KINSOL ({\tt kinxs.c} and {\tt kinxp.c} being the C program 
source). It was also solved using the Band-Block-Diagonal Preconditioner 
supplied with KINSOL ({\tt kinxbbd.c}). The PDE problem to be solved is now 
briefly presented.

This example problem is a model of a multi-species food web \cite{Br86},
in which mutual competition and/or predator-prey relationships in a
spatial domain are simulated. For this problem the dependent variable $c$ 
replaces the generic dependent variable $u$ used above. Here we consider a 
model with $s = 2p$ species, where both species $1,\cdots , p$ (the
prey) and $p+1,\cdots, s$ (the predators) have infinitely fast reaction
rates:
\begin{equation}
\renewcommand{\arraystretch}{1.2}
\left\{
\begin{array}{l}
               \hspace*{1em}  0 = f_i(x,y,c) + d_i 
( c^i_{xx} + c^i_{yy} ) ~ ~ (i = 1,2,\cdots,p) , \\
               \hspace*{1em}  0 = f_i(x,y,c) + d_i 
( c^i_{xx} + c^i_{yy} ) ~ ~ (i = p+1,\cdots,s) ,
\end{array} \right.  \label{5.3}
\end{equation}
with
\begin{equation}
f_i(x,y,c) = c^i (b_i + \sum_{j = 1}^s a_{ij} c^j).
\end{equation}
The interaction and diffusion coefficients $(a_{ij},b_i,d_i)$
could be functions of $(x,y)$ in general. The choices made for
this test problem are for a simple model of $p$ prey and $p$
predator species, arranged in that order in the vector $c$.  We
take the various coefficients to be as follows:
\begin{equation}
\left\{
\begin{array}{l}
a_{ii} = -1 ~ ~ (\mbox{all } i) \\
a_{ij} = -0.5 \cdot 10^{-6} ~ ~ ( i \leq p , j > p )
\\
a_{ij} = 10^4 ~ ( i > p , j \leq p )
\end{array}  \right.
\end{equation}
(all other $a_{ij} = 0 ) ,$
\begin{equation}
\left\{
\begin{array}{l}
b_i = b_i(x,y) = (1 + \alpha xy )  ~ ~ ( i \leq p ) \\
b_i = b_i(x,y) = -(1 + \alpha xy   ~ ~ ( i > p )
\end{array}  \right.
\end{equation}
and
\begin{equation}
\left\{
\begin{array}{l}
d_i =   1 ~ ~ ( i \leq p ) \\
d_i = 0.5 ~ ~ ( i > p ) .
\end{array}  \right.
\end{equation}
The domain is the unit square $0 \leq x,y \leq 1$.  The boundary conditions are
of Neumann type (zero normal derivatives) everywhere.  The coefficients are
such that a unique stable equilibrium is guaranteed to exist when $\alpha$
is zero \cite{Br86}.  Empirically, for (\ref{5.3}) a stable 
equilibrium appears to exist when $\alpha$ is positive, although 
it may not be unique. In this problem we take $\alpha = 1$.
The initial conditions used for this problem are taken to be constant functions
by species type. These satisfy the boundary conditions and very nearly
satisfy the constraints, given by
\begin{eqnarray*}
c^i & = & 1.16347 ~~ (i=1, \cdots, p)\\
c^i & = & 34903.1 ~~ (i=p+1, \cdots, s).
\end{eqnarray*}

The PDE system (\ref{5.3}) (plus boundary conditions) was discretized
with central differencing on an $L \times L$ mesh, with the resulting nonlinear
system has size $N$ = $s L^2$.

The main program source solving this problem ({\tt kinxs.c}) is given in its 
entirety in the Appendix. The output for this case is also included in the
Appendix.

\section{Availability}

At present, the KINSOL package has not been released for general
distribution. However, plans are in progress for a limited release, and
interested potential users at DOE Laboratories can obtain KINSOL on request
from Allan Taylor or Alan Hindmarsh (at agtaylor@llnl.gov or alanh@llnl.gov, 
resp). 

\begin{thebibliography}{9}

\bibitem{Br86}
P. N. Brown, {\em Decay to Uniform States in Food Webs}, SIAM J. Appl.
Math., 46 (1986), 376-392.

\bibitem{BrHi89}
P. N. Brown and A. C. Hindmarsh, {\it Reduced Storage Matrix Methods
in Stiff ODE Systems}, J. Appl. Math. \& Comp. 31 (1989), pp. 40--91.

\bibitem{BrSa90} 
P. N. Brown and Y. Saad, {\it Hybrid Krylov methods for 
nonlinear systems of equations}, SIAM J. Sci. Stat. Comput., 11 (1990), 
pp. 450-481.

\bibitem{EiWa96}
S. C. Eisenstat and H. F. Walker, {\it Choosing the Forcing Terms in an Inexact
Newton Method}, SIAM J. Sci. Comput., 17 (1996),
pp. 16-32.

\bibitem{PVODEusrguide} George D. Byrne and Alan C. Hindmarsh,
{\it User Documentation for PVODE, An ODE Solver for Parallel Computers}, 
Lawrence Livermore National Laboratory report UCRL-ID-130884, May 1998.

\bibitem{CoHi94}
S. D. Cohen and A. C. Hindmarsh, {\it CVODE User Guide},
Lawrence Livermore National Laboratory report UCRL-MA-118618,
September 1994.

\bibitem{CoHi96}  Scott D. Cohen and Alan C. Hindmarsh, {\it CVODE, a
Stiff/Nonstiff ODE Solver in C}, Computers in Physics, 10, No. 2
(1996), pp. 138-143.

\bibitem{MPI}  William Gropp, Ewing Lusk, and Anthony Skjellum, {\it Using
MPI Portable Parallel Programming with the Message-Passing Interface, }The
MIT Press, Cambridge, MA, 1994.

\bibitem{LDRD98}
Alan C. Hindmarsh and Allan G. Taylor, {\it PVODE and KINSOL: Parallel Software
for Differential and Nonlinear Systems}, Lawrence Livermore National Laboratory
report UCRL-ID-129739, February 1998.

\bibitem{SaSc86}
Y. Saad and M. H. Schultz, {\it GMRES: A Generalized Minimal Residual
Algorithm for Solving Nonsymmetric Linear Systems}, SIAM J. Sci. Stat.
Comp. 7 (1986), pp. 856--869.

\end{thebibliography}


\newpage
\section{Appendix: Listing of Predator-Prey PDE Example Program}
\small
\begin{verbatim}

 /************************************************************************
 *                                                                       *
 * File: kinxp.c                                                         *
 * Programmers: Allan G. Taylor and Alan C. Hindmarsh @ LLNL             *
 * Version of 1 Dec 1997                                                 *
 *-----------------------------------------------------------------------*
 * Example problem for KINSol, parallel machine case                  
 * This example solves a nonlinear system that arises from a system of  
 * partial differential equations. The PDE system is a food web         
 * population model, with predator-prey interaction and diffusion on the
 * unit square in two dimensions. The dependent variable vector is      
 * 
 *       1   2         ns
 * c = (c , c ,  ..., c  )              (denoted by the variable cc)
 * 
 * and the pde's are as follows:
 *
 *                    i       i      
 *         0 = d(i)*(c     + c    )  +  f  (x,y,c)   (i=1,...,ns)
 *                    xx      yy         i
 *
 *   where
 *
 *                   i             ns         j  
 *   f  (x,y,c)  =  c  * (b(i)  + sum a(i,j)*c )
 *    i                           j=1
 *
 * The number of species is ns = 2 * np, with the first np being prey and
 * the last np being predators. The number np is both the number of prey and
 * predator species. The coefficients a(i,j) , b(i) , d(i) are
 *
 *   a(i,i) = -AA  (all i)
 *   a(i,j) = -GG  (i <= np , j >  np)
 *   a(i,j) =  EE  (i >  np,  j <= np)
 *   b(i) = BB * (1 + alpha * x * y)  (i <= np)
 *   b(i) =-BB * (1 + alpha * x * y)  (i >= np)
 *   d(i) = dprey  (i <= np)
 *   d(i) = dpred  ( i > np)
 *
 *  The various scalar parameters are set using define's 
 *       or in routine InitUserData
 *  The boundary conditions are .. normal derivative  =  0.
 *  The initial guess is constant in x and y, although the final
 *  solution is not.
 *
 *  The PDEs are discretized by central differencing on a mx by my mesh.
 * 
 *  The nonlinear system is solved by KINSol using the method specified in
 *  local variable globalstrat .
 *
 *  The preconditioner matrix is a block-diagonal matrix based on the
 *  partial derivatives of the interaction terms f (in the above equation) only
 * 
 *
 * 
 *  Execution:  mpirun -np N -machinefile machines kinxp
 *          {with N = NPEX*NPEY, total number of processors, see below}
 *
 *  references..
 * 1.
 *  Peter N Brown and Youcef Saad,
 *  Hybrid Krylov Methods for Nonlinear Systems of Equations
 *  LLNL report UCRL-97645, November 1987
 *  
 * 2.
 *  Peter N. Brown and Alan C. Hindmarsh,
 *  Reduced Storage Matrix Methods in Stiff ODE systems,
 *  Lawrence Livermore National Laboratory Report  UCRL-95088, Rev. 1,
 *  June 1987, and  Journal of Applied Mathematics and Computation, Vol. 31
 *  (May 1989), pp. 40-91. ( for a description of the time-dependent
 *   version of this test problem.)
 *
 *
 *  run command line: mpirun -np N -machinefile machines kinxp 
 *   where N = NPEX * NPEY  is the number of processors to use.
 ************************************************************************/

#include <stdio.h>
#include <stdlib.h>
#include <math.h>
#include "llnltyps.h"  /* definitions of real, integer, boole, TRUE, FALSE*/
#include "kinsol.h"    /* main KINSol header file                         */
#include "iterativ.h"  /* contains the enum for types of preconditioning  */
#include "kinspgmr.h"  /* use KINSpgmr linear solver                      */
#include "dense.h"     /* use generic DENSE solver for preconditioning    */
#include "nvector.h"   /* definitions of type N_Vector, macro N_VDATA     */
#include "llnlmath.h"  /* contains RSqrt and UnitRoundoff routines        */
#include "mpi.h"       /* MPI include file                                */

/* Problem Constants */

#define NUM_SPECIES     6         /*  must equal 2*(number of prey or
                      predators)  number of prey = 
                      number of predators */

#define PI       3.1415926535898   /* pi */ 

#define NPEX        2        /* number of processors in the x-direction */
#define NPEY        2        /* number of processors in the y-direction */
#define MXSUB       10       /* MXSUB = number of x mesh points per subgrid */
#define MYSUB       10       /* MYSUB = number of y mesh points per subgrid */
#define MX          (NPEX*MXSUB) /* number of grid points in the x-direction */
#define MY          (NPEY*MYSUB) /* number of grid points in the y-direction */
#define NSMXSUB     (NUM_SPECIES * MXSUB)
#define NSMXSUB2    (NUM_SPECIES * (MXSUB+2))
#define NEQ         (NUM_SPECIES * MX * MY)  
                               /* number of equations in the system */

#define AA          RCONST(1.0)    /* value of coefficient a, above eqns */
#define EE          RCONST(10000.) /* value of coefficient e, above eqns */
#define GG          RCONST(0.5e-6) /* value of coefficient g, above eqns */
#define BB          RCONST(1.0)    /* value of coefficient b, above eqns */
#define DPREY       RCONST(1.0)    /* value of coefficient dprey, above eqns */
#define DPRED       RCONST(0.5)    /* value of coefficient dpred, above eqns */
#define ALPHA       RCONST(1.0)    /* value of coefficient alpha, above eqns */
#define AX          RCONST(1.0)    /* total range of x variable */
#define AY          RCONST(1.0)    /* total range of y variable */
#define FTOL        RCONST(1.e-7)  /*  ftol tolerance */
#define STOL        RCONST(1.e-13) /*  stol tolerance */
#define THOUSAND    RCONST(1000.0) /* one thousand */
#define ZERO        RCONST(0.)     /* 0. */
#define ONE         RCONST(1.0)    /* 1. */


/* User-defined vector accessor macro: IJ_Vptr */

/* IJ_Vptr is define in order to isolate the underlying 3-d structure of the 
   dependent variable vector from its underlying 1-d storage (an N_Vector).
   IJ_Vptr returns a pointer to the location in vv corresponding to 
   ns = 0 ,  jx = i,  jy = j .    */


#define IJ_Vptr(vv,i,j)   (&(((vv)->data)[(i)*NUM_SPECIES + (j)*NSMXSUB]))


/* Type : UserData 
   contains preconditioner blocks, pivot arrays, and problem constants */

typedef struct {
  real **P[MXSUB][MYSUB];
  integer *pivot[MXSUB][MYSUB];
  real **acoef, *bcoef;
  N_Vector rates;
  real *cox, *coy;
  real cext[NUM_SPECIES * (MXSUB+2)*(MYSUB+2)];
  integer my_pe, isubx, isuby, nsmxsub, nsmxsub2;

  MPI_Comm comm;
  
  real ax, ay, dx, dy;
  real uround, sqruround;
  integer Neq, mx, my, ns, np;
} *UserData;


/* Private Helper Functions */

static UserData AllocUserData(void);
static void InitUserData(integer my_pe, MPI_Comm comm, UserData data);
static void FreeUserData(UserData data);
static void SetInitialProfiles(N_Vector cc, N_Vector sc);
static void PrintOutput(integer my_pe, MPI_Comm comm, N_Vector cc);
static void PrintFinalStats(long int *iopt);
static void WebRate(real xx, real yy, real *cxy, real *ratesxy, void *f_data);
static real DotProd(integer size, real *x1, real *x2);
static void BSend(MPI_Comm comm, integer my_pe, integer isubx, integer isuby,
                  integer dsizex, integer dsizey, real *cdata);
static void BRecvPost(MPI_Comm comm, MPI_Request request[], integer my_pe,
              integer isubx, integer isuby,
              integer dsizex, integer dsizey,
              real *cext, real *buffer);
static void BRecvWait(MPI_Request request[], integer isubx, integer isuby,
              integer dsizex, real *cext, real *buffer);
static void ccomm(integer Neq, real *cdata, UserData data);
static void fcalcprpr(integer Neq, N_Vector cc, N_Vector fval, 
                 void *f_data);




/* Functions Called by the KINSol Solver */

static void funcprpr(integer Neq, N_Vector cc, N_Vector fval, 
                     void *f_data);


static int Precondbd(integer Neq, N_Vector cc, N_Vector cscale,
           N_Vector fval, N_Vector fscale,
           N_Vector vtem,N_Vector vtemp1, SysFn func, real uround,
           long int *nfePtr, void *P_data);


static int PSolvebd(integer Neq, N_Vector cc, N_Vector cscale,
          N_Vector fval, N_Vector fscale, N_Vector vv, N_Vector ftem,
          SysFn func, real uround, 
          long int *nfePtr, void *P_data);



/***************************** Main Program ******************************/

main(int argc, char *argv[])

{
  FILE *msgfile;
  integer Neq=NEQ;
  integer globalstrategy, i, local_N;
  real fnormtol, scsteptol, ropt[OPT_SIZE];
  long int iopt[OPT_SIZE];
  N_Vector cc, sc, constraints;
  UserData data;
  int iout, flag;
  int npelast = NPEX*NPEY-1;
  int my_pe, npes;
  boole optIn;
  void *mem;
  KINMem kmem;
  machEnvType   machEnv;
  MPI_Comm      comm;

  /* Allocate memory, and set problem data, initial values, tolerances */ 

  msgfile = fopen("PredPrey.out","w");

  /* Get processor number and total number of pe's */

  MPI_Init(&argc, &argv);
  comm = MPI_COMM_WORLD;
  MPI_Comm_size(comm, &npes);
  MPI_Comm_rank(comm, &my_pe);

  if (npes != NPEX*NPEY) {
    if (my_pe == 0)
      printf("\n npes=%d is not equal to NPEX*NPEY=%d\n", npes,NPEX*NPEY);
    return(1);
  }

  /* Set local length */

  local_N = NUM_SPECIES*MXSUB*MYSUB;

  /* allocate and initialize user data block */

  data=(UserData)AllocUserData();
  InitUserData(my_pe, comm, data);
  machEnv = PVecInitMPI(comm, local_N, Neq, &argc, &argv);
  if(machEnv==NULL) return(1);

/* example of changing defaults using iopt */
  optIn = TRUE; for(i=0;i<KINSOL_IOPT_SIZE;i++)iopt[i]=0; 
                for(i=0;i<KINSOL_ROPT_SIZE;i++)ropt[i]=ZERO;
        iopt[MXITER]=250; 

/* choose global strategy */
  globalstrategy = INEXACT_NEWTON;

  /* allocate (initialize) vectors */
  cc = N_VNew(Neq, machEnv);
  sc = N_VNew(Neq, machEnv);
  data->rates=N_VNew(Neq,machEnv);

  constraints = N_VNew(Neq, machEnv);
  N_VConst(0.,constraints);
  
  SetInitialProfiles(cc, sc);

  fnormtol=FTOL; scsteptol=STOL;

/* Call KINMalloc to allocate KINSol memory block: 

     A pointer to KINSol problem memory is returned and stored in kmem.*/

  mem = KINMalloc(Neq, msgfile, machEnv);
  if(my_pe==0 && mem == NULL) { printf("KINMalloc failed."); return(1); }
  kmem = (KINMem)mem;

  /* Call KINSpgmr to specify the KINSol linear solver KINSpgmr with solve 
      routines Precondbd and PSolvebd, and the pointer to 
      the user-defined  block data.               */

  KINSpgmr(kmem,
       16, /* a zero in this position forces use of the KINSpgmr default
          for maxl, dimension of the Krylov space*/
       2,  /* if zero in this position forces use of the KINSpgmr default
            for maxlrst, the max number of linear solver restarts allowed*/
       0,  /* a zero in this position forces use of the KINSpgmr default 
          for msbpre, the number of calls to the preconditioner allowed
          without a call to the preconditioner setup routine */
           Precondbd, /* user-supplied preconditioner setup routine */
           PSolvebd,  /* user-supplied preconditioner solve routine */
       NULL,   /* user-supplied ATimes routine -- Null chosen here */
           data);

  if(my_pe==0)printf(" \n predator-prey test problem --  KINSol\n\n");

  /* first,print out the problem size and then the 
     initial concentration profile */

  if(my_pe==0){
    printf("Mesh dimensions %d X %d\n",MX,MY);
    printf("Total system size %d\n",Neq);
    printf("Preconditioning uses interaction-only block-diagonal matrix\n");
    printf("tolerance parameters:  fnormtol = %g   scsteptol = %g\n",
               fnormtol,scsteptol);

    printf("\nInitial profile of concentration\n");
  }
  if(my_pe==0 || my_pe==npelast)  PrintOutput(my_pe, comm, cc);
  
 
  /* call KINSol and print output concentration profile */

  flag = KINSol(kmem,           /* KINSol memory block */
        Neq,            /* system size -- number of equations  */
        cc,             /* solution cc of funcprpr(cc)=0 is desired */
        funcprpr,       /* function describing the system equations */
        globalstrategy, /* global stragegy choice */
        sc,             /* scaling vector, for the variable cc */
        sc,             /* scaling vector for function values fval */
        fnormtol,       /* tolerance on fnorm funcprpr(cc) for sol'n */
        scsteptol,      /* step size tolerance */
        constraints,    /* constraints vector  */
        optIn,          /* optional inputs flat: TRUE or FALSE */
        iopt,           /* integer optional input array */
        ropt,           /* real optional input array */
        data,           /* pointer to user data */
        msgfile,        /* file pointer to message file */
        machEnv);          /* machEnv pointer */

  if(my_pe==0){
   if (flag != KINSOL_SUCCESS) { 
    printf("KINSol failed, flag=%d.\n", flag); 
    return(flag); }

    printf("\n\n\nComputed equilibrium species concentrations:\n\n");
 }

  if(my_pe==0 || my_pe==npelast)PrintOutput(my_pe, comm, cc);

  /* cc values are available on each processor */
  if(my_pe==0) PrintFinalStats(iopt);
 

  /* Free memory and print final statistics */  
  N_VFree(cc);
  N_VFree(sc);
  N_VFree(constraints);
  KINFree(kmem);
  FreeUserData(data);

  MPI_Finalize();
  return(0);
}


/*********************** Private Helper Functions ************************/


/* Allocate memory for data structure of type UserData */

static  UserData AllocUserData(void)
{
  int jx, jy;
  UserData data;

  data = (UserData) malloc(sizeof *data);

  for (jx=0; jx < MXSUB; jx++) {
    for (jy=0; jy < MYSUB; jy++) {
      (data->P)[jx][jy] = denalloc(NUM_SPECIES);
      (data->pivot)[jx][jy] = denallocpiv(NUM_SPECIES);
    }
  }
 (data->acoef) = denalloc(NUM_SPECIES);
 (data->bcoef) = (real *)malloc(NUM_SPECIES * sizeof(real));
 (data->cox)   = (real *)malloc(NUM_SPECIES * sizeof(real));
 (data->coy)   = (real *)malloc(NUM_SPECIES * sizeof(real));
 

  return(data);
}

/* readability constants defined */

#define acoef  (data->acoef)
#define bcoef  (data->bcoef)
#define cox    (data->cox)
#define coy    (data->coy)



/********************************************************************/

/* Load problem constants in data */

static void InitUserData(integer my_pe, MPI_Comm comm,UserData data)
{
  int i, j, np;
  real *a1,*a2, *a3, *a4, *b, dx2, dy2;

  data->mx = MX;
  data->my = MY;
  data->ns = NUM_SPECIES;
  data->np = NUM_SPECIES / 2;
  data->ax = AX;
  data->ay = AY;
  data->dx = (data->ax)/(MX-1);
  data->dy = (data->ay)/(MY-1);
  data->Neq= NEQ;
  data->my_pe = my_pe;
  data->comm = comm;
  data->isuby = my_pe / NPEX;
  data->isubx = my_pe - data->isuby*NPEX;
  data->nsmxsub = NUM_SPECIES * MXSUB;
  data->nsmxsub2 = NUM_SPECIES * (MXSUB+2);
  

  data->uround = UnitRoundoff();
  data->sqruround = RSqrt(data->uround);
  
  /* set up the coefficients a and b plus others found in the equations */
  np = data->np;

  dx2=(data->dx)*(data->dx); dy2=(data->dy)*(data->dy);

  for(i=0;i<np;i++){
    a1= &(acoef[i][np]);
    a2= &(acoef[i+np][0]);
    a3= &(acoef[i][0]);
    a4= &(acoef[i+np][np]);
    /*  fill in the portion of acoef in the four quadrants, row by row */
    for(j=0;j<np;j++){
      *a1++ =  -GG;
      *a2++ =   EE;
      *a3++ = ZERO;
      *a4++ = ZERO;
    }

    /* and then change the diagonal elements of acoef to -AA */
    acoef[i][i]=-AA;
    acoef[i+np][i+np] = -AA;

    bcoef[i] = BB;
    bcoef[i+np] = -BB;

    cox[i]=DPREY/(dx2);
    cox[i+np]=DPRED/(dx2);

    coy[i]=DPREY/(dy2);
    coy[i+np]=DPRED/(dy2);
  }

}

/********************************************************************/

/* Free data memory */

static void FreeUserData(UserData data)
{
  int jx, jy;

  for (jx=0; jx < MXSUB; jx++) {
    for (jy=0; jy < MYSUB; jy++) {
      denfree((data->P)[jx][jy]);
      denfreepiv((data->pivot)[jx][jy]);
    }
  }

  denfree(acoef);
  free(bcoef);
  free(cox);
  N_VFree(data->rates);

  free(data);

}

/********************************************************************/

/* Set initial conditions in cc */

static void SetInitialProfiles(N_Vector cc, N_Vector sc)
{
  int i, jx, jy;
  real *ct1, *st1, *ct2, *st2;
  real  ctemp[NUM_SPECIES], stemp[NUM_SPECIES];

  /* Initialize temporary arrays ctemp and stemp to be used
          in the loading process */

  for(i=0;i<NUM_SPECIES;i++)
    if(i<NUM_SPECIES/2){
      ctemp[i]=RCONST(1.16347);
      stemp[i]=ONE;}
    else {
      ctemp[i]=RCONST(34903.1);
      stemp[i]=RCONST(0.00001);}

  /* Load initial profiles into cc and sc vector from temporary arrays */

  for (jy=0; jy < MYSUB; jy++) {
    for (jx=0; jx < MXSUB; jx++) {
      ct1 = IJ_Vptr(cc,jx,jy);
      ct2 = ctemp;
      st1 = IJ_Vptr(sc,jx,jy);
      st2 = stemp;
      for(i=0;i<NUM_SPECIES;i++){
    *ct1++=*ct2++;
    *st1++=*st2++;
      }
    }
  }

}  /* end SetInitialProfiles */

/********************************************************************/

/* Print sample of current cc values */

static void PrintOutput(integer my_pe, MPI_Comm comm, N_Vector cc)
{
  int is, jx, jy, i0, npelast;
  real  *ct, tempc[NUM_SPECIES];
  MPI_Status status;

  npelast = NPEX*NPEY - 1;

  ct = N_VDATA(cc);

 /* send the cc values (for all species) at the top right mesh point to PE 0 */
  if(my_pe == npelast){
  i0 = NUM_SPECIES*(MXSUB*MYSUB-1);
  if(npelast!=0)
    MPI_Send(&ct[i0],NUM_SPECIES,PVEC_REAL_MPI_TYPE,0,0,comm);
  else  /* single processor case */
    for(is=0;is<NUM_SPECIES;is++) tempc[is]=ct[i0+is];   
  }

  /* On PE 0, receive the cc values at top right, then print performance data 
     and sampled solution values */
  if(my_pe == 0) {

    if(npelast != 0)
      MPI_Recv(&tempc[0],NUM_SPECIES,PVEC_REAL_MPI_TYPE, npelast,0,comm,&status);
    printf("\n");
    printf("At bottom left::\n");
    for(is=0;is<NUM_SPECIES;is++){
      if((is%6)*6== is)printf("\n");
      printf(" %g",ct[is]);
    }

    printf("\n\n");
    printf("At top right:\n");
    for(is=0;is<NUM_SPECIES;is++){
      if((is%6)*6 == is)printf("\n");
      printf(" %g",tempc[is]);
    }
    printf("\n\n");
  }
}

/********************************************************************/

/* Print final statistics contained in iopt */

static void PrintFinalStats(long int *iopt)
{
  printf("\nFinal Statistics.. \n\n");
  printf("nni    = %5ld    nli   = %5ld\n", iopt[NNI], iopt[SPGMR_NLI]);
  printf("nfe    = %5ld    npe   = %5ld\n", iopt[NFE], iopt[SPGMR_NPE]);
  printf("nps    = %5ld    ncfl  = %5ld\n", iopt[SPGMR_NPS], iopt[SPGMR_NCFL]);
}

/********************************************************************/


/* Routine to send boundary data to neighboring PEs */

static void BSend(MPI_Comm comm, integer my_pe, integer isubx, integer isuby,
                  integer dsizex, integer dsizey, real *cdata)
{
  int i, ly;
  integer offsetc, offsetbuf;
  real bufleft[NUM_SPECIES*MYSUB], bufright[NUM_SPECIES*MYSUB];

  /* If isuby > 0, send data from bottom x-line of u */

  if (isuby != 0)
    MPI_Send(&cdata[0], dsizex, PVEC_REAL_MPI_TYPE, my_pe-NPEX, 0, comm);

  /* If isuby < NPEY-1, send data from top x-line of u */

  if (isuby != NPEY-1) {
    offsetc = (MYSUB-1)*dsizex;
    MPI_Send(&cdata[offsetc], dsizex, PVEC_REAL_MPI_TYPE, my_pe+NPEX, 0, comm);
  }

  /* If isubx > 0, send data from left y-line of u (via bufleft) */

  if (isubx != 0) {
    for (ly = 0; ly < MYSUB; ly++) {
      offsetbuf = ly*NUM_SPECIES;
      offsetc = ly*dsizex;
      for (i = 0; i < NUM_SPECIES; i++)
        bufleft[offsetbuf+i] = cdata[offsetc+i];
    }
    MPI_Send(&bufleft[0], dsizey, PVEC_REAL_MPI_TYPE, my_pe-1, 0, comm);   
  }

  /* If isubx < NPEX-1, send data from right y-line of u (via bufright) */

  if (isubx != NPEX-1) {
    for (ly = 0; ly < MYSUB; ly++) {
      offsetbuf = ly*NUM_SPECIES;
      offsetc = offsetbuf*MXSUB + (MXSUB-1)*NUM_SPECIES;
      for (i = 0; i < NUM_SPECIES; i++)
        bufright[offsetbuf+i] = cdata[offsetc+i];
    }
    MPI_Send(&bufright[0], dsizey, PVEC_REAL_MPI_TYPE, my_pe+1, 0, comm);   
  }

}


/********************************************************************/

 
/* Routine to start receiving boundary data from neighboring PEs.
   Notes:
   1) buffer should be able to hold 2*NUM_SPECIES*MYSUB real entries, should be
   passed to both the BRecvPost and BRecvWait functions, and should not
   be manipulated between the two calls.
   2) request should have 4 entries, and should be passed in both calls also. */

static void BRecvPost(MPI_Comm comm, MPI_Request request[], integer my_pe,
              integer isubx, integer isuby,
              integer dsizex, integer dsizey,
              real *cext, real *buffer)
{
  integer offsetce;
  /* Have bufleft and bufright use the same buffer */
  real *bufleft = buffer, *bufright = buffer+NUM_SPECIES*MYSUB;

  /* If isuby > 0, receive data for bottom x-line of cext */
  if (isuby != 0)
    MPI_Irecv(&cext[NUM_SPECIES], dsizex, PVEC_REAL_MPI_TYPE,
                         my_pe-NPEX, 0, comm, &request[0]);

  /* If isuby < NPEY-1, receive data for top x-line of cext */
  if (isuby != NPEY-1) {
    offsetce = NUM_SPECIES*(1 + (MYSUB+1)*(MXSUB+2));
    MPI_Irecv(&cext[offsetce], dsizex, PVEC_REAL_MPI_TYPE,
                                         my_pe+NPEX, 0, comm, &request[1]);
  }

  /* If isubx > 0, receive data for left y-line of cext (via bufleft) */
  if (isubx != 0) {
    MPI_Irecv(&bufleft[0], dsizey, PVEC_REAL_MPI_TYPE,
                                         my_pe-1, 0, comm, &request[2]);
  }

  /* If isubx < NPEX-1, receive data for right y-line of cext (via bufright) */
  if (isubx != NPEX-1) {
    MPI_Irecv(&bufright[0], dsizey, PVEC_REAL_MPI_TYPE,
                                         my_pe+1, 0, comm, &request[3]);
  }

}

/********************************************************************/



/* Routine to finish receiving boundary data from neighboring PEs.
   Notes:
   1) buffer should be able to hold 2*NUM_SPECIES*MYSUB real entries, should be
   passed to both the BRecvPost and BRecvWait functions, and should not
   be manipulated between the two calls.
   2) request should have 4 entries, and should be passed in both calls also. */

static void BRecvWait(MPI_Request request[], integer isubx, integer isuby,
              integer dsizex, real *cext, real *buffer)
{
  int i, ly;
  integer dsizex2, offsetce, offsetbuf;
  real *bufleft = buffer, *bufright = buffer+NUM_SPECIES*MYSUB;
  MPI_Status status;

  dsizex2 = dsizex + 2*NUM_SPECIES;

  /* If isuby > 0, receive data for bottom x-line of cext */
  if (isuby != 0)
    MPI_Wait(&request[0],&status);

  /* If isuby < NPEY-1, receive data for top x-line of cext */
  if (isuby != NPEY-1)
    MPI_Wait(&request[1],&status);

  /* If isubx > 0, receive data for left y-line of cext (via bufleft) */
  if (isubx != 0) {
    MPI_Wait(&request[2],&status);

    /* Copy the buffer to cext */
    for (ly = 0; ly < MYSUB; ly++) {
      offsetbuf = ly*NUM_SPECIES;
      offsetce = (ly+1)*dsizex2;
      for (i = 0; i < NUM_SPECIES; i++)
        cext[offsetce+i] = bufleft[offsetbuf+i];
    }
  }

  /* If isubx < NPEX-1, receive data for right y-line of cext (via bufright) */
  if (isubx != NPEX-1) {
    MPI_Wait(&request[3],&status);

    /* Copy the buffer to cext */
    for (ly = 0; ly < MYSUB; ly++) {
      offsetbuf = ly*NUM_SPECIES;
      offsetce = (ly+2)*dsizex2 - NUM_SPECIES;
      for (i = 0; i < NUM_SPECIES; i++)
    cext[offsetce+i] = bufright[offsetbuf+i];
    }
  }

}


/********************************************************************/

/* ccomm routine.  This routine performs all communication 
   between processors of data needed to calculate f. */

static void ccomm(integer Neq,real *cdata, UserData data)
{

  real *cext, buffer[2*NUM_SPECIES*MYSUB];
  MPI_Comm comm;
  integer my_pe, isubx, isuby, nsmxsub, nsmysub;
  MPI_Request request[4];


  /* Get comm, my_pe, subgrid indices, data sizes, extended array cext */

  comm = data->comm;  my_pe = data->my_pe;
  isubx = data->isubx;   isuby = data->isuby;
  nsmxsub = data->nsmxsub;
  nsmysub = NUM_SPECIES*MYSUB;
  cext = data->cext;

  /* Start receiving boundary data from neighboring PEs */

  BRecvPost(comm, request, my_pe, isubx, isuby, nsmxsub, nsmysub, cext, buffer);

  /* Send data from boundary of local grid to neighboring PEs */

  BSend(comm, my_pe, isubx, isuby, nsmxsub, nsmysub, cdata);

  /* Finish receiving boundary data from neighboring PEs */

  BRecvWait(request, isubx, isuby, nsmxsub, cext, buffer);

}

/********************************************************************/


/* system function for predator - prey system  calculation part */

static void fcalcprpr(integer Neq, N_Vector cc, N_Vector fval, 
                 void *f_data)
{
  real xx, yy, *cxy, *rxy, *fxy, dcydi, dcyui, dcxli, dcxri;
  real *cext, dely, delx, *cdata;
  integer i, j, is, ly;
  integer isubx, isuby, nsmxsub, nsmxsub2;
  integer shifty, offsetc, offsetce, offsetcl, offsetcr, offsetcd, offsetcu;
  UserData data;


  data=(UserData)f_data;
  cdata = N_VDATA(cc);

  /* Get subgrid indices, data sizes, extended work array cext */

  isubx = data->isubx;   isuby = data->isuby;
  nsmxsub = data->nsmxsub; nsmxsub2 = data->nsmxsub2;
  cext = data->cext;

  /* Copy local segment of cc vector into the working extended array cext */

  offsetc = 0;
  offsetce = nsmxsub2 + NUM_SPECIES;
  for (ly = 0; ly < MYSUB; ly++) {
    for (i = 0; i < nsmxsub; i++) cext[offsetce+i] = cdata[offsetc+i];
    offsetc = offsetc + nsmxsub;
    offsetce = offsetce + nsmxsub2;
  }

  /* To facilitate homogeneous Neumann boundary conditions, when this is
  a boundary PE, copy data from the first interior mesh line of cc to cext */

  /* If isuby = 0, copy x-line 2 of cc to cext */
  if (isuby == 0) {
    for (i = 0; i < nsmxsub; i++) cext[NUM_SPECIES+i] = cdata[nsmxsub+i];
  }

  /* If isuby = NPEY-1, copy x-line MYSUB-1 of cc to cext */
  if (isuby == NPEY-1) {
    offsetc = (MYSUB-2)*nsmxsub;
    offsetce = (MYSUB+1)*nsmxsub2 + NUM_SPECIES;
    for (i = 0; i < nsmxsub; i++) cext[offsetce+i] = cdata[offsetc+i];
  }

  /* If isubx = 0, copy y-line 2 of cc to cext */
  if (isubx == 0) {
    for (ly = 0; ly < MYSUB; ly++) {
      offsetc = ly*nsmxsub + NUM_SPECIES;
      offsetce = (ly+1)*nsmxsub2;
      for (i = 0; i < NUM_SPECIES; i++) cext[offsetce+i] = cdata[offsetc+i];
    }
  }

  /* If isubx = NPEX-1, copy y-line MXSUB-1 of cc to cext */
  if (isubx == NPEX-1) {
    for (ly = 0; ly < MYSUB; ly++) {
      offsetc = (ly+1)*nsmxsub - 2*NUM_SPECIES;
      offsetce = (ly+2)*nsmxsub2 - NUM_SPECIES;
      for (i = 0; i < NUM_SPECIES; i++) cext[offsetce+i] = cdata[offsetc+i];
    }
  }


  /* loop over all grid points, evaluating for each species at each */

  delx = data->dx;
  dely = data->dy;
  shifty = (MXSUB+2)*NUM_SPECIES;
  for(j=0; j<MYSUB; j++) {
    yy = dely*(j  + isuby * MYSUB);
    for(i=0; i<MXSUB; i++){

      xx = delx * ( i + isubx * MXSUB);
      cxy = IJ_Vptr(cc,i,j);
      rxy = IJ_Vptr(data->rates,i,j);
      fxy = IJ_Vptr(fval,i,j);
      
      WebRate(xx, yy, cxy, rxy, f_data);

      offsetc = (i+1)*NUM_SPECIES + (j+1)*NSMXSUB2;
      offsetcd = offsetc - shifty;
      offsetcu = offsetc + shifty;
      offsetcl = offsetc - NUM_SPECIES;
      offsetcr = offsetc + NUM_SPECIES;

      
      for(is=0; is<NUM_SPECIES; is++){

    /* differencing in x */

    dcydi = cext[offsetc+is]  - cext[offsetcd+is];
        dcyui = cext[offsetcu+is] - cext[offsetc+is];
    
    /* differencing in y */

    dcxli = cext[offsetc+is]  - cext[offsetcl+is];
    dcxri = cext[offsetcr+is] - cext[offsetc+is];

    /* compute the value at xx , yy */

    fxy[is] = (coy)[is] * (dcyui - dcydi) +
               (cox)[is] * (dcxri - dcxli) + rxy[is];

      } /* end is loop */

    } /* end of i  or x  loop */

  } /* end of j or y loop */

}  /* end of routine fcalcprpr */



/***************** Functions Called by the KINSol Solver ******************/


/* system function routine.  Evaluate funcprpr(cc).  First call ccomm to do 
  communication of  subgrid boundary data into cext.  Then calculate funcprpr
  by a call to fcalcprpr. */

static void funcprpr(integer Neq, N_Vector cc, N_Vector fval, void *f_data)
{
  real *cdata, *fvdata;
  UserData data;

  cdata = N_VDATA(cc);
  fvdata = N_VDATA(fval);
  data = (UserData) f_data;


  /* Call ccomm to do inter-processor communicaiton */

  ccomm (Neq, cdata, data);

  /* Call fcalc to calculate the system function */

  fcalcprpr (Neq, cc, fval, data);

}



/********************************************************************/

/* Preconditioner setup routine. Generate and preprocess P. */

static int Precondbd(integer Neq, N_Vector cc, N_Vector cscale,
           N_Vector fval, N_Vector fscale,
           N_Vector vtem, N_Vector vtemp1, SysFn func, real uround,
           long int *nfePtr, void *P_data)
{
  real r, r0, sqruround;
  real xx, yy, *cxy, *scxy, cctemp, **Pxy, *ratesxy, *Pxycol;
  real fac, perturb_rates[NUM_SPECIES];


  integer i, j, jx, jy, ret;


  UserData data;

  data = (UserData)P_data;

  sqruround = data->sqruround;
  fac = N_VWL2Norm(fval, fscale);
  r0 = THOUSAND * uround * fac * Neq;

  if(r0 == ZERO) r0 = ONE;


  for(jy=0; jy<MYSUB; jy++){

    yy =data->dy *(jy + data->isuby * MYSUB);

    for(jx=0; jx<MXSUB; jx++){

      xx = data->dx * (jx + data->isubx * MXSUB);
      Pxy = (data->P)[jx][jy];
      cxy = IJ_Vptr(cc,jx,jy);
      scxy= IJ_Vptr(cscale,jx,jy);
      ratesxy = IJ_Vptr((data->rates),jx,jy);

      for(j=0; j<NUM_SPECIES; j++){

    cctemp=cxy[j];  /* save the j,jx,jy element of cc */
    r=MAX(sqruround * ABS(cctemp),r0/scxy[j]);
    cxy[j] += r; /* perturb the j,jx,jy element of cc */
    fac = ONE/r;

    WebRate(xx, yy, cxy, perturb_rates,data);

    Pxycol = Pxy[j];

    for(i=0; i<NUM_SPECIES; i++) {
      Pxycol[i]=(perturb_rates[i]-ratesxy[i]) * fac;
    }

    /* restore j,jx,jy element of cc */
    cxy[j] = cctemp;

      } /* end of j loop */


      /*  lu decomposition of each block */

      ret = gefa(Pxy, NUM_SPECIES, (data->pivot)[jx][jy]);


      if(ret!=0)return(1);

    } /* end jx loop */

  } /* end jy loop */
  return(0);

}  /* end of routine Precondbd */

/********************************************************************/


/* Preconditioner solve routine */

static int PSolvebd(integer Neq, N_Vector cc, N_Vector cscale,
          N_Vector fval, N_Vector fscale, N_Vector vv, N_Vector ftem,
          SysFn func, real uround, 
          long int *nfePtr, void *P_data)
{
 real **Pxy, *vxy;
 integer *pivot, jx, jy;
 UserData data;

 data = (UserData)P_data;
 
 for(  jx=0; jx<MXSUB; jx++) {
   for(jy=0; jy<MYSUB; jy++){
     /* for a given jx,jy block, do the inversion process */
     /* vvxy is the address of the portion of the vector to which the
    inversion process is applied, and Pxy is the first address for the 
    jx,jy block of P */
     pivot=(data->pivot)[jx][jy];
     Pxy = (data->P)[jx][jy];
     vxy = IJ_Vptr(vv,jx,jy);
     gesl(Pxy, NUM_SPECIES, pivot, vxy);

   } /* end of jy loop */

 } /* end of jx loop */

 return(0);

} /*  end of PSolvebd  */


/********************************************************************/

static void WebRate(real xx, real yy, real *cxy, real *ratesxy, void *f_data)
{
  integer i;
  integer j;
  real fac;
  UserData data;

  data = (UserData)f_data;

  for(i=0;i<NUM_SPECIES;i++) 
             ratesxy[i]= DotProd(NUM_SPECIES, cxy, acoef[i]);
  /* above, ratesxy is used as an intermediate array.  see below */

  fac = ONE + ALPHA * xx * yy;

  for(i=0; i<NUM_SPECIES; i++){  ratesxy[i] = cxy[i] *
                 ( bcoef[i] * fac + ratesxy[i] );
  }

} /* end WebRate */

/********************************************************************/


static real DotProd(integer size, real *x1, real *x2)
{
  integer i;
  real *xx1, *xx2, temp = ZERO;
  
  xx1 = x1; xx2 = x2;
  for(i=0; i<size; i++) temp += *xx1++ * *xx2++;
  return(temp);

}
\end{verbatim}
\newpage
\normalsize

Sample output for the sample case {\tt KINXP}

\small
\begin{verbatim}

 predator-prey test problem --  KINSol

Mesh dimensions 20 X 20
Total system size 2400
Preconditioning uses interaction-only block-diagonal matrix
tolerance parameters:  fnormtol = 1e-07   scsteptol = 1e-13

Initial profile of concentration

At bottom left::

 1.16347 1.16347 1.16347 34903.1 34903.1 34903.1

At top right:

 1.16347 1.16347 1.16347 34903.1 34903.1 34903.1




Computed equilibrium species concentrations:


At bottom left::

 1.165 1.165 1.165 34949 34949 34949

At top right:

 1.25552 1.25552 1.25552 37663.2 37663.2 37663.2


Final Statistics.. 

nni    =    68    nli   =  1339
nfe    =  1476    npe   =     6
nps    =  1407    ncfl  =    16



\end{verbatim}
\normalsize

\end{document}


