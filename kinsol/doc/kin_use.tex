%===================================================================================
\chapter{Using KINSOL for C Applications}\label{c:usage}
%===================================================================================

This chapter is concerned with the use of {\kinsol} for the solution
of nonlinear systems. The following subsections treat the header
files, the layout of the user's main program, description of the
{\kinsol} user-callable routines, and user-supplied functions.
The final section describes the {\F}/{\C} interface module, which supports
users with applications written in {\F}77. The listings of the sample
programs in the companion document \cite{kinsol2.4.0_ex} may also be helpful.
Those codes may be used as templates (with the removal of some lines involved
in testing), and are included in the {\kinsol} package.

The user should be aware that not all linear solver modules are compatible 
with all {\nvector} implementations. 
\index{KINSOL@{\kinsol} linear solvers!NVECTOR@{\nvector} compatibility}
For example, {\nvecp} is not compatible with the direct dense or direct band 
linear solvers since these linear solver modules need to form the complete
system Jacobian. The following {\kinsol} modules can only be used with {\nvecs}:
{\kindense} and {\kinband}. The preconditioner module {\kinbbdpre}
can only be used with {\nvecp}.

{\kinsol} uses various constants for both input and output. These are
defined as needed in this chapter, but for convenience are also listed
separately in Chapter \ref{c:constants}.

%%==============================================================================
\section{Access to library and header files}\label{ss:file_access}
%%==============================================================================

At this point, it is assumed that the installation of {\kinsol},
following the procedure described in Chapter \ref{s:install}, has
been completed successfully.

Regardless of where the user's application program resides, its
associated compilation and load commands must make reference to the
appropriate locations for the library and header files required by
{\kinsol}.  The relevant library files are
\begin{itemize}
\item {\em libdir}\id{/libsundials\_kinsol.}{\em lib},
\item {\em libdir}\id{/libsundials\_nvec*.}{\em lib} (one or two files),
\end{itemize}
where the file extension .{\em lib} is typically \id{.so} for shared libraries
and \id{.a} for static libraries. The relevant header files are located in
the subdirectories
\begin{itemize}
\item {\em incdir}\id{/include}
\item {\em incdir}\id{/include/kinsol}
\item {\em incdir}\id{/include/sundials}
\end{itemize}
The directories {\em libdir} and {\em incdir} are the install libray and include
directories. For a default installation, these are {\em build\_tree}\id{/lib} and
{\em build\_tree}\id{/include}, respectively, where {\em build\_tree} was defined in
Chapter \ref{s:install}.

%%----------------------------------
\section{Data types}\label{s:types}
%%----------------------------------
% This is a shared SUNDIALS TEX file with description of
% types used in llntyps.h
%
\index{portability}
The \ID{sundials\_types.h} file contains the definition of the type \ID{realtype},
which is used by the {\sundials} solvers for all floating-point data, the definition 
of the integer type \ID{sunindextype}, which is used for vector and matrix indices,
and \ID{booleantype}, which is used for certain logic operations within {\sundials}.


\subsection{Floating point types}

The type \id{realtype} can be \id{float}, \id{double}, or \id{long double}, with
the default being \id{double}.
The user can change the precision of the {\sundials} solvers arithmetic at the
configuration stage (see \S\ref{ss:configuration_options_nix}).

Additionally, based on the current precision, \id{sundials\_types.h} defines 
\Id{BIG\_REAL} to be the largest value representable as a \id{realtype},
\Id{SMALL\_REAL} to be the smallest value representable as a \id{realtype}, and
\Id{UNIT\_ROUNDOFF} to be the difference between $1.0$ and the minimum \id{realtype}
greater than $1.0$.

Within {\sundials}, real constants are set by way of a macro called
\Id{RCONST}.  It is this macro that needs the ability to branch on the
definition \id{realtype}.  In ANSI {\CC}, a floating-point constant with no
suffix is stored as a \id{double}.  Placing the suffix ``F'' at the
end of a floating point constant makes it a \id{float}, whereas using the suffix
``L'' makes it a \id{long double}.  For example,
\begin{verbatim}
#define A 1.0
#define B 1.0F
#define C 1.0L
\end{verbatim}
defines \id{A} to be a \id{double} constant equal to $1.0$, \id{B} to be a
\id{float} constant equal to $1.0$, and \id{C} to be a \id{long double} constant
equal to $1.0$.  The macro call \id{RCONST(1.0)} automatically expands to \id{1.0}
if \id{realtype} is \id{double}, to \id{1.0F} if \id{realtype} is \id{float},
or to \id{1.0L} if \id{realtype} is \id{long double}.  {\sundials} uses the
\id{RCONST} macro internally to declare all of its floating-point constants. 

A user program which uses the type \id{realtype} and the \id{RCONST} macro
to handle floating-point constants is precision-independent except for
any calls to precision-specific standard math library
functions.  (Our example programs use both \id{realtype} and
\id{RCONST}.)  Users can, however, use the type \id{double}, \id{float}, or
\id{long double} in their code (assuming that this usage is consistent
with the typedef for \id{realtype}).  Thus, a previously existing
piece of ANSI {\CC} code can use {\sundials} without modifying the code
to use \id{realtype}, so long as the {\sundials} libraries use the
correct precision (for details see \S\ref{ss:configuration_options_nix}).


\subsection{Integer types used for vector and matrix indices}

The type \id{sunindextype} can be either a 64- or 32-bit \emph{signed} integer.
The default is the portable \id{int64\_t} type, and the user can change it
to \id{int32\_t} at the configuration stage. The configuration system
will detect if the compiler does not support portable types, and will
replace \id{int64\_t} and \id{int32\_t} with \id{long long} and \id{int},
respectively, to ensure use of the desired sizes on Linux, Mac OS X and Windows
platforms. {\sundials} currently does not support \emph{unsigned} integer types 
for vector and matrix indices, although these could be added in the future if there 
is sufficient demand.


%------------------------
\section{Header files}\label{s:header_sol}
%------------------------
\index{header files}
The calling program must include several header files so that various macros
and data types can be used. The header file that is always required is:
%
\begin{itemize}
\item  \Id{kinsol.h}, 
  the header file for {\kinsol}, which defines several
  types and various constants, and includes function prototypes.
\end{itemize}
%
\id{kinsol.h} also includes \Id{sundials\_types.h}, 
which defines the types \id{realtype} and \id{booleantype}
and constants \id{FALSE} and \id{TRUE}.

The calling program must also include an {\nvector} implementation header file
(see Chapter \ref{s:nvector} for details).
For the two {\nvector} implementations that are included in the {\kinsol} package,
the corresponding header files are:
%
\begin{itemize}
\item \Id{nvector\_serial.h}, 
  which defines the serial implementation, {\nvecs};
\item \Id{nvector\_parallel.h}, 
  which defines the parallel {\mpi} implementation, {\nvecp}.
\end{itemize}
%
Note that both of these files include in turn the header file \Id{sundials\_nvector.h}, which 
defines the abstract \Id{N\_Vector} type. 

Finally, a linear solver module header file is required. 
\index{KINSOL@{\kinsol} linear solvers!header files}
The header files corresponding to the various linear solver options in
{\kinsol} are:
%%
\begin{itemize}
\item \Id{kinsol\_dense.h},
  which is used with the dense direct linear solver in 
  the context of {\kinsol}. This in turn includes a header file (\id{sundials\_dense.h})
  which defines the \Id{DenseMat} type and corresponding accessor macros; 
\item \Id{kinsol\_band.h}, 
  which is used with the band direct linear solver in the
  context of {\kinsol}. This in turn includes a header file (\id{sundials\_band.h})
  which defines the \Id{BandMat} type and corresponding accessor macros;
\item \Id{kinsol\_spgmr.h}, 
  which is used with the Krylov solver {\spgmr} in the
  context of {\kinsol};
\item \Id{kinsol\_spbcgs.h}, 
  which is used with the Krylov solver {\spbcg} in the
  context of {\kinsol};
\item \Id{kinsol\_sptfqmr.h}, 
  which is used with the Krylov solver {\sptfqmr} in the
  context of {\kinsol};
\end{itemize}

The header files for the Krylov iterative solvers include \id{kinsol\_spils.h}
which defined common fuunctions and which in turn includes a header file (\id{sundials\_iterative.h})
which enumerates the kind of preconditioning and for the choices for the Gram-Schmidt process for {\spgmr}.

Other headers may be needed, according to the choice of
preconditioner, etc. For example, in the \id{kinkryx} example~\cite{kinsol2.4.0_ex}, 
preconditioning is done with a block-diagonal matrix. For this, the header
\id{sundials\_smalldense.h} is included.

%-------------------------------------------------
\section{A skeleton of the user's main program}\label{s:skeleton_sol}
%-------------------------------------------------

The following is a skeleton of the user's main program (or calling
program) for the solution of a nonlinear problem. 
Some steps are independent of the {\nvector} implementation used; 
where this is not the case, usage specifications are given for the two implementations 
provided with {\kinsol}: Steps marked with {\p} correspond to 
{\nvecp}, while steps marked with {\s} correspond to {\nvecs}.
%%
%%
\index{User main program!KINSOL@{\kinsol} usage}
\begin{Steps}
  
\item 
  {\bf {\p} Initialize MPI}

  Call \id{MPI\_Init(\&argc, \&argv);} to initialize {\mpi} if used by
  the user's program, aside from the internal use in {\nvecp}.  
  Here \id{argc} and \id{argv} are the command line argument 
  counter and array received by \id{main}.
  
\item
  {\bf Set problem dimensions}

  {\s} Set \id{N}, the problem size $N$.

  {\p} Set \id{Nlocal}, the local vector length (the sub-vector
  length for this process); \id{N}, the global vector length (the
  problem size $N$, and the sum of all the values of \id{Nlocal});
  and the active set of processes.
  
\item
  {\bf Set vector with initial guess}
 
  To set the vector \id{u} of initial values, use functions defined by a
  particular {\nvector} implementation.  If a \id{realtype} array  \id{udata}
  already exists, containing the initial guess of $u_0$, make the call:

  {\s} \id{u = N\_VMake\_Serial(N, udata);}

  {\p} \id{u = N\_VMake\_Parallel(comm, Nlocal, N, udata);}

  Otherwise, make the call:

  {\s} \id{u = N\_VNew\_Serial(N);}

  {\p} \id{u = N\_VNew\_Parallel(comm, Nlocal, N);}

  and load initial values into the structure defined by:

  {\s} \id{NV\_DATA\_S(u)}

  {\p} \id{NV\_DATA\_P(u)}

  Here \id{comm} is the {\mpi} communicator, set in one of two ways: 
  If a proper subset of active processes is to be used, \id{comm} 
  must be set by suitable {\mpi} calls. Otherwise, to specify that all 
  processes are to be used, \id{comm} must be \id{MPI\_COMM\_WORLD}.
  
\item\label{i:kinsol_create} 
  {\bf Create {\kinsol} object}

  Call \id{kin\_mem = KINCreate();} 
  to create the {\kinsol} memory block.
  \id{KINCreate} returns a pointer to the {\kinsol} memory structure.
  See \S\ref{sss:kinmalloc} for details.

\item
  {\bf Set optional inputs}

  Call \id{KINSet*} routines to change from their default values any
  optional inputs that control the behavior of {\kinsol}.
  See \S\ref{ss:optional_input} for details.

\item\label{i:kinsol_malloc} 
  {\bf Allocate internal memory}

  Call \id{KINMalloc(...);} 
  to specify the problem defining function $F$,
  allocate internal memory for {\kinsol}, 
  and initialize {\kinsol}.
  \id{KINMalloc} returns a flag to indicate success or an illegal argument value.
  See \S\ref{sss:kinmalloc} for details.

\item\label{i:lin_solver} 
  {\bf Attach linear solver module}

  Initialize the linear solver module with one of the following calls
  (for details see \S\ref{sss:lin_solv_init}).

  {\s} \id{ier = }\Id{KINDense}\id{(...);}

  {\s} \id{ier = }\Id{KINBand}\id{(...);}

  \id{ier = }\Id{KINSpgmr}\id{(...);}
  
  \id{ier = }\Id{KINSpbcg}\id{(...);}
  
  \id{ier = }\Id{KINSptfqmr}\id{(...);}
  
\item
  {\bf Set linear solver optional inputs}

  Call \id{KIN*Set*} functions from the selected linear solver module to
  change optional inputs specific to that linear solver.
  See \S\ref{ss:optional_input} for details.

\item
  {\bf Solve problem}

  Call \id{ier = }\Id{KINSol}\id{(...);} to solve the nonlinear problem for a given
  initial guess. See \S\ref{sss:kinsol} for details.

\item
  {\bf Get optional outputs}

  Call \id{KINGet*} and \id{KIN*Get*} functions to obtain optional output.
  See \S\ref{ss:optional_output} for details.

\item
  {\bf Deallocate memory for solution vector}

  Upon completion of the solution, deallocate memory for the vector \id{u}
  by calling the destructor function defined by the {\nvector} implementation:

  {\s} \id{N\_VDestroy\_Serial(u);}

  {\p} \id{N\_VDestroy\_Parallel(u);}
  
\item
  {\bf Free solver memory}

  Call \id{KINFree(\&kin\_mem);} to free the memory allocated for {\kinsol}.
  
\item 
  {\bf {\p} Finalize MPI}

  Call \id{MPI\_Finalize();} to terminate {\mpi}.
  
\end{Steps}

%%==============================================================================

\section{User-callable functions}\label{s:kinsol_fct_sol}

This section describes the {\kinsol} functions that are called by the
user to set up and solve a nonlinear problem. Some of these are required. However,
starting with \S\ref{ss:optional_input}, the functions listed involve
optional inputs/outputs or restarting, and those paragraphs can be
skipped for a casual use of {\kinsol}. In any case, refer to
\S\ref{s:skeleton_sol} for the correct order of these calls.

%%-------------------------------------------------------------------------------

\subsection{KINSOL initialization and deallocation functions}
\label{sss:kinmalloc}

The following three functions must be called in the order listed. The last one
is to be called only after the problem solution is complete, as it frees the
{\kinsol} memory block created and allocated by the first two calls.
%%
\ucfunction{KINCreate}
{
  kin\_mem = KINCreate();
}
{
  The function \ID{KINCreate} instantiates a {\kinsol} solver object.
}
{
  This function has no arguments.
}
{
  If successful, \id{KINCreate} returns a pointer to the newly created 
  {\kinsol} memory block (of type \id{void *}).
  If an error occurred, \id{KINCreate} prints an error message to \id{stderr}
  and returns \id{NULL}.
}
{}
%%
%%
\ucfunction{KINMalloc}
{
flag = KINMalloc(kin\_mem, func, tmpl);
}
{
  The function \ID{KINMalloc} specifies the problem-defining
  function, allocates internal memory, and initializes {\kinsol}.
}
{
  \begin{args}[kin\_mem]
  \item[kin\_mem] (\id{void *})
    pointer to the {\kinsol} memory block returned by \id{KINCreate}.
  \item[func] (\Id{KINSysFn})
    is the {\C} function which computes $F$ in the nonlinear problem. 
    This function has the form \id{func(u, fval, f\_data)} 
    (for full details see \S\ref{ss:sysFn}).
  \item[tmpl] (\id{N\_Vector})
    is an \id{N\_Vector} which is used as a template to create (by cloning)
    necessary vectors in \id{kin\_mem}.
  \end{args}
}
{
  The return flag \id{flag} (of type \id{int}) will be one of the following:
  \begin{args}[KIN\_ILL\_INPUT]
  \item[\Id{KIN\_SUCCESS}]
    The call to \id{KINMalloc} was successful.
  \item[\Id{KIN\_MEM\_NULL}] 
    The {kinsol} memory block was not initialized through a previous call
    to \id{KINCreate}.
  \item[\Id{KIN\_MEM\_FAIL}] 
    A memory allocation request has failed.
  \item[\Id{KIN\_ILL\_INPUT}] 
    An input argument to \id{KINMalloc} has an illegal value.
  \end{args}
}
{
  If an error occurred, \id{KINMalloc} sends an error message to the
  error handler function.
}
%%
%%
\ucfunction{KINFree}
{
  KINFree(\&kin\_mem);
}
{
  The function \ID{KINFree} frees the memory allocated by
  a previous call to \id{KINMalloc}.
}
{
  The argument is the address of the pointer to the {\kinsol} memory block
  returned by \id{KINCreate} (of type \id{void *}).
}
{
  The function \id{KINFree} has no return value.
}
{}
%%

%%-------------------------------------------------------------------------------

\subsection{Linear solver specification functions}\label{sss:lin_solv_init}

As previously explained, Newton iteration requires the solution of
linear systems of the form (\ref{e:Newton}). There are five {\kinsol} linear
solvers currently available for this task: {\kindense}, {\kinband}, {\kinspgmr},
{\kinspbcg}, and {\kinsptfqmr}. The first two are direct solvers and their
names indicate the type of approximation used for the Jacobian
$J = \partial{F}/\partial{u}$; {\kindense} and {\kinband} work with dense and
banded approximations to $J$, respectively. The last three {\kinsol} linear solvers ---
{\kinspgmr}, {\kinspbcg}, and {\kinsptfqmr} --- are Krylov iterative solvers, which
use scaled preconditioned GMRES, scaled preconditioned Bi-CGStab, and scaled
preconditioned TFQMR, respectively. Together, they are referred to as {\kinspils}
(from scaled preconditioned iterative linear solvers).

\index{KINSOL{\kinsol} linear solvers!selecting one|(}
To specify a {\kinsol} linear solver, after the call to \id{KINCreate}
but before any calls to \id{KINSol}, the user's program must call one
of the functions \Id{KINDense}, \Id{KINBand}, \Id{KINSpgmr},
\Id{KINSpbcg}, or \Id{KINSptfqmr}, as documented below.
The first argument passed to these functions is the {\kinsol}
memory pointer returned by \id{KINCreate}. A call to one of these
functions links the main {\kinsol} nonlinear solver to a linear solver and
allows the user to specify parameters which are specific to a
particular solver, such as the half-bandwidths in the {\kinband} case.
%%
The use of each of the linear solvers involves certain constants and possibly 
some macros, that are likely to be needed in the user code.  These are
available in the corresponding header file associated with the linear
solver, as specified below.
\index{KINSOL@{\kinsol} linear solvers!selecting one|)}

\index{KINSOL@{\kinsol} linear solvers!built on generic solvers|)}
In each case, the linear solver module used by {\kinsol} is actually built
on top of a generic linear system solver, which may be of interest in itself.
These generic solvers, denoted {\dense}, {\band}, {\spgmr}, {\spbcg}, and {\sptfqmr},
are described separately in Chapter \ref{s:gen_linsolv}.
\index{KINSOL@{\kinsol} linear solvers!built on generic solvers|)}
%%
%%
%%
\index{KINSOL@{\kinsol} linear solvers!KINDENSE@{\kindense}}
\index{KINDENSE@{\kindense} linear solver!selection of}
\index{KINDENSE@{\kindense} linear solver!NVECTOR@{\nvector} compatibility}
\ucfunction{KINDense}
{
  flag = KINDense(kin\_mem, N);
}
{
  The function \ID{KINDense} selects the {\kindense} linear solver. 

  The user's main function must include the \id{kinsol\_dense.h} header file.
}
{
  \begin{args}[kin\_mem]
  \item[kin\_mem] (\id{void *})
    pointer to the {\kinsol} memory block.
  \item[N] (\id{long int})
    problem dimension.
  \end{args}
}
{
  The return value \id{flag} (of type \id{int}) is one of
  \begin{args}[KINDENSE\_ILL\_INPUT]
  \item[\Id{KINDENSE\_SUCCESS}] 
    The {\kindense} initialization was successful.
  \item[\Id{KINDENSE\_MEM\_NULL}]
    The \id{kin\_mem} pointer is \id{NULL}.
  \item[\Id{KINDENSE\_ILL\_INPUT}]
    The {\kindense} solver is not compatible with the current {\nvector} module.
  \item[\Id{KINDENSE\_MEM\_FAIL}]
    A memory allocation request failed.
  \end{args}
}
{
  The {\kindense} linear solver may not be compatible with a particular
  implementation of the {\nvector} module. 
  Of the two {\nvector} modules provided by {\sundials}, only {\nvecs} is 
  compatible, while {\nvecp} is not.
}
%%
%%
%%
\index{KINSOL@{\kinsol} linear solvers!KINBAND@{\kinband}}
\index{KINBAND@{\kinband} linear solver!selection of}
\index{KINBAND@{\kinband} linear solver!NVECTOR@{\nvector} compatibility}
\index{half-bandwidths}
\ucfunction{KINBand}
{
  flag = KINBand(kin\_mem, N, mupper, mlower);
}
{
  The function \ID{KINBand} selects the {\kinband} linear solver. 

  The user's main function must include the \id{kinsol\_band.h} header file.
}
{
  \begin{args}[kin\_mem]
  \item[kin\_mem] (\id{void *})
    pointer to the {\kinsol} memory block.
  \item[N] (\id{long int})
    problem dimension.
  \item[mupper] (\id{long int})
    upper half-bandwidth of the problem Jacobian (or of the approximation of it).
  \item[mlower] (\id{long int})
    lower half-bandwidth of the problem Jacobian (or of the approximation of it).
  \end{args}
}
{
  The return value \id{flag} (of type \id{int}) is one of
  \begin{args}[KINBAND\_ILL\_INPUT]
  \item[\Id{KINBAND\_SUCCESS}] 
    The {\kinband} initialization was successful.
  \item[\Id{KINBAND\_MEM\_NULL}]
    The \id{kin\_mem} pointer is \id{NULL}.
  \item[\Id{KINBAND\_ILL\_INPUT}]
    The {\kinband} solver is not compatible with the current {\nvector} module, or
    one of the Jacobian half-bandwidths is outside its valid range
    ($0 \ldots$ \id{N}$-1$).
  \item[\Id{KINBAND\_MEM\_FAIL}]
    A memory allocation request failed.
  \end{args}
}
{
  The {\kinband} linear solver may not be compatible with a particular
  implementation of the {\nvector} module. Of the two {\nvector} modules 
  provided by {\sundials}, only {\nvecs} is compatible, while {\nvecp} is not.
  The half-bandwidths are to be set so that the nonzero locations $(i,j)$ in the
  banded (approximate) Jacobian satisfy $-$\id{mlower} $\leq j-i \leq$ \id{mupper}.
}
%%
%%
%%
\index{KINSOL@{\kinsol} linear solvers!KINSPGMR@{\kinspgmr}}
\index{KINSPGMR@{\kinspgmr} linear solver!selection of}
\ucfunction{KINSpgmr}
{
  flag = KINSpgmr(kin\_mem, maxl);
}
{
  The function \ID{KINSpgmr} selects the {\kinspgmr} linear solver.

  The user's main function must include the \id{kinsol\_spgmr.h} header file.
}
{
  \begin{args}[kin\_mem]
  \item[kin\_mem] (\id{void *})
    pointer to the {\kinsol} memory block.
  \item[maxl] (\id{int})
    maximum dimension of the Krylov subspace to be used. Pass $0$ to use the 
    default value \id{KINSPILS\_MAXL}$=5$.
  \end{args}
}
{
  The return value \id{flag} (of type \id{int}) is one of:
  \begin{args}[KINSPILS\_ILL\_INPUT]
  \item[\Id{KINSPILS\_SUCCESS}] 
    The {\kinspgmr} initialization was successful.
  \item[\Id{KINSPILS\_MEM\_NULL}]
    The \id{kin\_mem} pointer is \id{NULL}.
  \item[\Id{KINSPILS\_ILL\_INPUT}]
    The {\nvector} module used does not implement a required operation.
  \item[\Id{KINSPILS\_MEM\_FAIL}]
    A memory allocation request failed.
  \end{args}
}
{
  The {\kinspgmr} solver uses a scaled preconditioned GMRES\index{GMRES method}
  iterative method to solve the linear system (\ref{e:Newton}).

  Within {\kinsol}, only right preconditioning is available.
  For specification of the preconditioner,
  see \S\ref{ss:optional_input} and \S\ref{ss:user_fct_sol}.

  If preconditioning is done, user-supplied functions define the right 
  preconditioner matrices $P$, which approximate the Newton matrix
  from (\ref{e:Newton}).
}
%%
%%
%%
\index{KINSOL@{\kinsol} linear solvers!KINSPBCG@{\kinspbcg}}
\index{KINSPBCG@{\kinspbcg} linear solver!selection of}
\ucfunction{KINSpbcg}
{
  flag = KINSpbcg(kin\_mem, maxl);
}
{
  The function \ID{KINSpbcg} selects the {\kinspbcg} linear solver.

  The user's main function must include the \id{kinsol\_spbcgs.h} header file.
}
{
  \begin{args}[kin\_mem]
  \item[kin\_mem] (\id{void *})
    pointer to the {\kinsol} memory block.
  \item[maxl] (\id{int})
    maximum dimension of the Krylov subspace to be used. Pass $0$ to use the 
    default value \id{KINSPILS\_MAXL}$=5$.
  \end{args}
}
{
  The return value \id{flag} (of type \id{int}) is one of:
  \begin{args}[KINSPILS\_ILL\_INPUT]
  \item[\Id{KINSPILS\_SUCCESS}] 
    The {\kinspbcg} initialization was successful.
  \item[\Id{KINSPILS\_MEM\_NULL}]
    The \id{kin\_mem} pointer is \id{NULL}.
  \item[\Id{KINSPILS\_ILL\_INPUT}]
    The {\nvector} module used does not implement a required operation.
  \item[\Id{KINSPILS\_MEM\_FAIL}]
    A memory allocation request failed.
  \end{args}
}
{
  The {\kinspbcg} solver uses a scaled preconditioned Bi-CGStab\index{Bi-CGstab method}
  iterative method to solve the linear system (\ref{e:Newton}).

  Within {\kinsol}, only right preconditioning is available.
  For specification of the preconditioner,
  see \S\ref{ss:optional_input} and \S\ref{ss:user_fct_sol}.

  If preconditioning is done, user-supplied functions define the right 
  preconditioner matrices $P$, which approximate the Newton matrix
  from (\ref{e:Newton}).
}
%%
%%
%%
\index{KINSOL@{\kinsol} linear solvers!KINSPTFQMR@{\kinsptfqmr}}
\index{KINSPTFQMR@{\kinsptfqmr} linear solver!selection of}
\ucfunction{KINSptfqmr}
{
  flag = KINSptfqmr(kin\_mem, maxl);
}
{
  The function \ID{KINSptfqmr} selects the {\kinsptfqmr} linear solver.

  The user's main function must include the \id{kinsol\_sptfqmr.h} header file.
}
{
  \begin{args}[kin\_mem]
  \item[kin\_mem] (\id{void *})
    pointer to the {\kinsol} memory block.
  \item[maxl] (\id{int})
    maximum dimension of the Krylov subspace to be used. Pass $0$ to use the 
    default value \id{KINSPILS\_MAXL}$=5$.
  \end{args}
}
{
  The return value \id{flag} (of type \id{int}) is one of:
  \begin{args}[KINSPILS\_ILL\_INPUT]
  \item[\Id{KINSPILS\_SUCCESS}] 
    The {\kinsptfqmr} initialization was successful.
  \item[\Id{KINSPILS\_MEM\_NULL}]
    The \id{kin\_mem} pointer is \id{NULL}.
  \item[\Id{KINSPILS\_ILL\_INPUT}]
    The {\nvector} module used does not implement a required operation.
  \item[\Id{KINSPILS\_MEM\_FAIL}]
    A memory allocation request failed.
  \end{args}
}
{
  The {\kinsptfqmr} solver uses a scaled preconditioned TFQMR\index{TFQMR method}
  iterative method to solve the linear system (\ref{e:Newton}).

  Within {\kinsol}, only right preconditioning is available.
  For specification of the preconditioner,
  see \S\ref{ss:optional_input} and \S\ref{ss:user_fct_sol}.

  If preconditioning is done, user-supplied functions define the right 
  preconditioner matrices $P$, which approximate the Newton matrix
  from (\ref{e:Newton}).
}

%--------------------------------------------------------------------
\subsection{KINSOL solver function}\label{sss:kinsol}
%--------------------------------------------------------------------

This is the central step in the solution process --- the call to solve
the nonlinear algebraic system.
%
\ucfunction{KINSol}
{
  flag = KINSol(kin\_mem, u, strategy, u\_scale, f\_scale);
}
{
  The function \ID{KINSol} computes an approximate solution to the nonlinear
  system.
}
{
  \begin{args}[strategy]
  \item[kin\_mem] (\id{void *})
    pointer to the {\kinsol} memory block.
  \item[u] (\id{N\_Vector})
    vector set to initial guess by user before calling \id{KINSol},
    but which upon return contains an approximate solution of
    the nonlinear system $F(u)=0$ the computed solution vector.
  \item[strategy] (\id{int})
    globalization strategy applied to the Newton method.
    It must be one of \ID{KIN\_NONE} or \ID{KIN\_LINESEARCH}.
  \item[u\_scale] (\id{N\_Vector})
    vector containing diagonal elements of scaling matrix $D_u$ for vector \id{u}
    chosen so that the components of $D_u \cdot$\id{u}
    (as a matrix multiplication) all have about the same magnitude when 
    \id{u} is close to a root of $F(u)$.
  \item[f\_scale] (\id{N\_Vector})
    vector containing diagonal elements of scaling matrix $D_F$ for $F(u)$ chosen 
    so that the components of $D_F \cdot F($\id{u}$)$ 
    (as a matrix multiplication) all have roughly the same magnitude when 
    \id{u} is not too near a root of $F(u)$.
  \end{args}
}
{
  On return, \id{KINSol} returns the approximate solution in the vector \id{u}.
  The return value \id{flag} (of type \id{int}) will be one of the following:
  \begin{args}[a]

  \item[\Id{KIN\_SUCCESS}]\rule{0pt}{0pt}

    \id{KINSol} succeeded; the scaled norm of $F(u)$ is less than \id{fnormtol}.

  \item[\Id{KIN\_INITIAL\_GUESS\_OK}]\rule{0pt}{0pt}

    The guess \id{u} $=u_0$ satisfied the system $F(u)=0$
    within the tolerances specified.

  \item[\Id{KIN\_STEP\_LT\_STPTOL}]\rule{0pt}{0pt}

    {\kinsol} stopped based on scaled step length.
    This means that the current iterate may be an approximate solution of the given
    nonlinear system, but it is also quite possible that the algorithm is ``stalled" 
    (making insufficient progress) near an invalid solution, or that the 
    scalar \id{scsteptol} is too large (see \id{KINSetScaledStepTol} in 
    \S\ref{ss:optional_input} to change \id{scsteptol} from its default value).

  \item[\Id{KIN\_MEM\_NULL}]\rule{0pt}{0pt}

    The {\kinsol} memory block pointer was \id{NULL}.
 
  \item[\Id{KIN\_ILL\_INPUT}]\rule{0pt}{0pt}

    An input parameter was invalid.

  \item[\Id{KIN\_NO\_MALLOC}]\rule{0pt}{0pt}

    The {\kinsol} memory was not allocated by a call to \id{KINMalloc}.

  \item[\Id{KIN\_LINESEARCH\_NONCONV}]\rule{0pt}{0pt}

    The line search algorithm was unable
    to find an iterate sufficiently distinct
    from the current iterate, or could not find an iterate satisfying
    the sufficient decrease condition.
    
    Failure to satisfy the sufficient decrease
    condition could mean the current iterate is
    ``close" to an approximate solution of the given
    nonlinear system, the finite difference
    approximation of the matrix-vector product
    $J(u) v$ is inaccurate, or the real scalar
    \id{scsteptol} is too large.
 
  \item[\Id{KIN\_MAXITER\_REACHED}] \rule{0pt}{0pt}

    The maximum number of nonlinear iterations
    has been reached.
 
  \item[\Id{KIN\_MXNEWT\_5X\_EXCEEDED}]\rule{0pt}{0pt}
    
    Five consecutive steps have been taken
    that satisfy the inequality
    $\|D_u p\|_{L2} > 0.99 \,$ \id{mxnewtstep},
    where $p$ denotes the current step and
    \id{mxnewtstep} is a scalar upper bound
    on the scaled step length.
    
    Such a failure may mean that $\|D_F F(u)\|_{L2}$
    asymptotes from above to a finite value, or
    the real scalar \id{mxnewtstep} is too small.
 
  \item[\Id{KIN\_LINESEARCH\_BCFAIL}]\rule{0pt}{0pt}

    The line search algorithm
    was unable to satisfy the
    ``beta-condition'' for \id{MXNBCF} $+ 1$ nonlinear
    iterations (not necessarily consecutive),
    which may indicate the algorithm is making
    poor progress.
 
  \item[\Id{KIN\_LINSOLV\_NO\_RECOVERY}]\rule{0pt}{0pt}

    The user-supplied routine \id{psolve}
    encountered a recoverable error, but
    the preconditioner is already current.
 
  \item[\Id{KIN\_LINIT\_FAIL}]\rule{0pt}{0pt}

    The linear solver initialization routine (\id{linit})
    encountered an error.
 
  \item[\Id{KIN\_LSETUP\_FAIL}]\rule{0pt}{0pt}

    The user-supplied routine \id{pset} (used to set up 
    the preconditioner data) encountered an unrecoverable
    error.
 
  \item[\Id{KIN\_LSOLVE\_FAIL}]\rule{0pt}{0pt}

    Either the user-supplied routine \id{psolve} (used to
    to solve the preconditioned linear system) encountered
    an unrecoverable error, or the linear solver routine
    (\id{lsolve}) encountered an error condition.

  \item[\Id{KIN\_SYSFUNC\_FAIL}]\rule{0pt}{0pt}

    The system function failed in an unrecoverable manner.

  \item[\Id{KIN\_FIRST\_SYSFUNC\_ERR}]\rule{0pt}{0pt}

    The system function failed recoverably at the first call.

  \item[\Id{KIN\_REPTD\_SYSFUNC\_ERR}]\rule{0pt}{0pt}

    The system function had repeated recoverable errors. No recovery is possible.

 
  \end{args}
}
{
  The components of vectors \id{u\_scale} and \id{f\_scale} should be strictly positive.
  
  \id{KIN\_SUCCESS} $=0$, \id{KIN\_INITIAL\_GUESS\_OK} $=1$, and 
  \id{KIN\_STEP\_LT\_STPTOL} $=2$.
  %%
  All remaining return values are negative and therefore a test \id{flag} $< 0$
  will trap all \id{KINSol} failures.
}


%%==============================================================================
\subsection{Optional input functions}\label{ss:optional_input}
%%==============================================================================

{\kinsol} provides an extensive list of functions that can be used to change
from their default values various optional input parameters that control the
behavior of the {\kinsol} solver. 
Table \ref{t:optional_input} lists all optional input functions in {\kinsol} which 
are then described in detail in the remainder of this section, beginning with those
for the main {\kinsol} solver and continuing with those for the linear solver
modules. For the most casual use of {\kinsol}, the reader can skip to \S\ref{ss:user_fct_sol}.

We note that, on error return, all of these functions also send an error message
to the error handler function.\index{error messages}
We also note that all error return values are negative, so a test \id{flag} $<0$
will catch any error.

\begin{table}
\centering
\caption{Optional inputs for {\kinsol}, {\kindense}, and {\kinspils}}
\label{t:optional_input}
\medskip
\begin{tabular}{|l|l|l|}\hline
{\bf Optional input} & {\bf Function name} & {\bf Default} \\
\hline
\multicolumn{3}{|c|}{\bf KINSOL main solver} \\
\hline
Error handler function & \id{KINSetErrHandlerFn} & internal fn. \\
Pointer to an error file & \id{KINSetErrFile} & \id{stderr}  \\
Info handler function & \id{KINSetInfoHandlerFn} & internal fn. \\
Pointer to an info file & \id{KINSetInfoFile} & \id{stdout} \\
Data for problem-defining function & \id{KINSetFdata} & \id{NULL} \\
Verbosity level of output & \id{KINSetPrintLevel} & 0 \\
Max. number of nonlinear iterations & \id{KINSetNumMaxIters} & 200 \\
No initial preconditioner setup & \id{KINSetNoInitSetup} & \id{FALSE} \\
No residual monitoring${}^{*}$ & \id{KINSetNoResMon} & \id{FALSE} \\
Max. iterations without prec. setup & \id{KINSetMaxSetupCalls} & 10 \\
Max. iterations without residual check${}^{*}$ & \id{KINSetMaxSubSetupCalls} & 5 \\
Form of $\eta$ coefficient & \id{KINSetEtaForm} &  \id{KIN\_ETACHOICE1}\\
Constant value of $\eta$ & \id{KINSetEtaConstValue} &  0.1 \\
Values of $\gamma$ and $\alpha$ & \id{KINSetEtaParams} & 0.9 and 2.0 \\
Values of $\omega_{min}$ and $\omega_{max}$${}^{*}$ & \id{KINSetResMonParams} & 0.00001 and 0.9 \\
Constant value of $\omega$${}^{*}$ & \id{KINSetResMonConstValue} & 0.9 \\
Lower bound on $\epsilon$ & \id{KINSetNoMinEps} & \id{FALSE} \\
Max. scaled length of Newton step & \id{KINSetMaxNewtonStep} & $1000 \| D_u u_0 \|_2$ \\
Max. number of $\beta$-condition failures & \id{KINSetMaxBetaFails} & 10 \\
Rel. error for F.D. $Jv$ & \id{KINSetRelErrFunc} & $\sqrt{\text{uround}}$ \\
Function-norm stopping tolerance & \id{KINSetFuncNormTol} & $\sqrt[3]{\text{uround}}$ \\
Scaled-step stopping tolerance & \id{KINSetScaledSteptol} & $\text{uround}^{2/3}$ \\
Inequality constraints on solution & \id{KINSetConstraints} & \id{NULL} \\
Nonlinear system function & \id{KINSetSysFunc} & none \\
\hline
\multicolumn{3}{|c|}{\bf KINDENSE linear solver} \\
\hline
Dense Jacobian function and data & \id{KINDenseSetJacFn} & internal DQ, \\
&&\id{NULL}\\
\hline
\multicolumn{3}{|c|}{\bf KINBAND linear solver} \\
\hline
Band Jacobian function and data & \id{KINBandSetJacFn} & internal DQ, \\
&&\id{NULL}\\
\hline
\multicolumn{3}{|c|}{\bf KINSPILS linear solvers} \\
\hline
Max. number of restarts${}^{**}$ & \id{KINSpilsSetMaxRestarts} & 0 \\
Preconditioner functions and data & \id{KINSpilsSetPreconditioner} & \id{NULL}, \id{NULL}, \id{NULL} \\
Jacobian-times-vector function and data & \id{KINSpilsSetJacTimesVecFn} & internal DQ, \\
&&\id{NULL} \\
\hline
\multicolumn{3}{l}{}\\
\multicolumn{3}{l}{${}^{*}$ Only for {\kindense} and {\kinband}} \\
\multicolumn{3}{l}{${}^{**}$ Only for {\kinspgmr}}\\
\end{tabular}
\end{table}

\subsubsection{Main solver optional input functions}\label{ss:optin_main}
\index{optional input!solver|(}

The calls listed here can be executed in any order. \\
However, if \id{KINSetErrHandlerFn} 
or \id{KINSetErrFile} are to be called, that call should be first, in order to take
effect for any later error message.
%%
\index{error message!user-defined handler}
\ucfunction{KINSetErrHandlerFn}
{
flag = KINSetErrHandlerFn(kin\_mem, ehfun, eh\_data);
}
{
  The function \ID{KINSetErrHandlerFn} specifies the optional user-defined function
  to be used in handling error messages.
}
{
  \begin{args}[kin\_mem]
  \item[kin\_mem] (\id{void *})
    pointer to the {\kinsol} memory block.
  \item[ehfun] (\id{KINErrHandlerFn})
    is the {\C} error handler function (see \S\ref{ss:ehFn}).
  \item[eh\_data] (\id{void *})
    pointer to user data passed to \id{ehfun} every time it is called.
  \end{args}
}
{
  The return value \id{flag} (of type \id{int}) is one of:
  \begin{args}[KIN\_MEM\_NULL]
  \item[\Id{KIN\_SUCCESS}] 
    The function \id{ehfun} and data pointer \id{eh\_data} have been successfully set.
  \item[\Id{KIN\_MEM\_NULL}]
    The \id{kin\_mem} pointer is \id{NULL}.
  \end{args}
}
{
  The default internal error handler function directs error messages to the
  file specified by the file pointer \id{errfp} (see \id{KINSetErrFile} below).

  Error messages indicating that the {\kinsol} solver memory is \id{NULL} will
  always be directed to \id{stderr}.
}
%%
\index{error messages!redirecting}
\ucfunction{KINSetErrFile}
{
flag = KINSetErrFile(kin\_mem, errfp);
}
{
  The function \ID{KINSetErrFile} specifies the pointer to the file
  where all {\kinsol} messages should be directed in case the default
  {\kinsol} error handler function is used.
}
{
  \begin{args}[kin\_mem]
  \item[kin\_mem] (\id{void *})
    pointer to the {\kinsol} memory block.
  \item[errfp] (\id{FILE *})
    pointer to output file.
  \end{args}
}
{
  The return value \id{flag} (of type \id{int}) is one of
  \begin{args}[KIN\_MEM\_NULL]
  \item[\Id{KIN\_SUCCESS}] 
    The optional value has been successfully set.
  \item[\Id{KIN\_MEM\_NULL}]
    The \id{kin\_mem} pointer is \id{NULL}.
  \end{args}
}
{
  The default value for \id{errfp} is \id{stderr}. 

  Passing a value of \id{NULL} disables all future error message output
  (except for the case in which the {\kinsol} memory pointer is \id{NULL}).

  {\warn}If \id{KINSetErrFile} is to be called, it should be called before any
  other optional input functions, in order to take effect for any later error message.
}
%%
%%
\index{info message!user-defined handler}
\ucfunction{KINSetInfoHandlerFn}
{
flag = KINSetInfoHandlerFn(kin\_mem, ihfun, ih\_data);
}
{
  The function \ID{KINSetInfoHandlerFn} specifies the optional user-defined function
  to be used in handling informative (non-error) messages.
}
{
  \begin{args}[kin\_mem]
  \item[kin\_mem] (\id{void *})
    pointer to the {\kinsol} memory block.
  \item[ihfun] (\id{KINInfoHandlerFn})
    is the {\C} info handler function (see \S\ref{ss:ihFn}).
  \item[ih\_data] (\id{void *})
    pointer to user data passed to \id{ihfun} every time it is called.
  \end{args}
}
{
  The return value \id{flag} (of type \id{int}) is one of:
  \begin{args}[KIN\_MEM\_NULL]
  \item[\Id{KIN\_SUCCESS}] 
    The function \id{ihfun} and data pointer \id{ih\_data} have been successfully set.
  \item[\Id{KIN\_MEM\_NULL}]
    The \id{kin\_mem} pointer is \id{NULL}.
  \end{args}
}
{
  The default internal info handler function directs informative (non-error) messages to the
  file specified by the file pointer \id{infofp} (see \id{KINSetInfoFile} below).
}
%%
\index{info messages!redirecting}
\ucfunction{KINSetInfoFile}
{
flag = KINSetInfoFile(kin\_mem, infofp);
}
{
  The function \ID{KINSetInfoFile} specifies the pointer to the file
  where all informative (non-error) messages should be directed.
}
{
  \begin{args}[kin\_mem]
  \item[kin\_mem] (\id{void *})
    pointer to the {\kinsol} memory block.
  \item[infofp] (\id{FILE *})
    pointer to output file.
  \end{args}
}
{
  The return value \id{flag} (of type \id{int}) is one of:
  \begin{args}[KIN\_MEM\_NULL]
  \item[\Id{KIN\_SUCCESS}] 
    The optional value has been successfully set.
  \item[\Id{KIN\_MEM\_NULL}]
    The \id{kin\_mem} pointer is \id{NULL}.
  \end{args}
}
{
  The default value for \id{infofp} is \id{stdout}.
}
%%
%%
\ucfunction{KINSetPrintLevel}
{
flag = KINSetPrintLevel(kin\_mem, printfl);
}
{
  The function \ID{KINSetPrintLevel} specifies the level of verbosity
  of the output.
}
{
  \begin{args}[kin\_mem]

  \item[kin\_mem] (\id{void *})
    pointer to the {\kinsol} memory block.

  \item[printfl] (\id{int})
    flag indicating the level of verbosity. Must be one of:
    
    \begin{itemize}
    \item[0]
      no information displayed.
      
    \item[1]
      for each nonlinear iteration display
      the following information: the scaled
      Euclidean $\ell_2$ norm of the system function
      evaluated at the current iterate, the
      scaled norm of the Newton step (only if
      using \id{KIN\_NONE}), and the
      number of function evaluations performed
      so far.
      
    \item[2]
      display level 1 output and the
      following values for each iteration:
      
      $\|F(u)\|_{D_F}$
      (only for \id{KIN\_NONE}).
      
      $\|F(u)\|_{D_F,\infty}$
      (for \id{KIN\_NONE} and
      \id{KIN\_LINESEARCH}).
      
    \item[3]
      display level 2 output plus additional
      values used by the global strategy
      (only if using \id{KIN\_LINESEARCH}), and
      statistical information for the linear
      solver.
    \end{itemize}

  \end{args}
}
{
  The return value \id{flag} (of type \id{int}) is one of:
  \begin{args}[KIN\_ILL\_INPUT]
  \item[\Id{KIN\_SUCCESS}] 
    The optional value has been successfully set.
  \item[\Id{KIN\_MEM\_NULL}]
    The \id{kin\_mem} pointer is \id{NULL}.
  \item[\Id{KIN\_ILL\_INPUT}]
    The argument \id{printfl} had an illegal value.
  \end{args}
}
{
  The default value for \id{printfl} is $0$.
}
%%
\ucfunction{KINSetFdata}
{
flag = KINSetFdata(kin\_mem, f\_data);
}
{
  The function \ID{KINSetFdata} specifies the pointer to user-defined memory
  that is to be passed to the user-supplied function implementing the nonlinear
  system residual.
}
{
  \begin{args}[kin\_mem]
  \item[kin\_mem] (\id{void *})
    pointer to the {\kinsol} memory block.
  \item[f\_data] (\id{void *})
    pointer to the user-defined memory.
  \end{args}
}
{
  The return value \id{flag} (of type \id{int}) is one of:
  \begin{args}[KIN\_MEM\_NULL]
  \item[\Id{KIN\_SUCCESS}] 
    The optional value has been successfully set.
  \item[\Id{KIN\_MEM\_NULL}]
    The \id{kin\_mem} pointer is \id{NULL}.
  \end{args}
}
{
  The default value for \id{f\_data} is \id{NULL}.
}
%%
%%
\ucfunction{KINSetNumMaxIters}
{
flag = KINSetNumMaxIters(kin\_mem, mxiter);
}
{
  The function \ID{KINSetNumMaxIters} specifies the maximum number of 
  nonlinear iterations allowed.
}
{
  \begin{args}[kin\_mem]
  \item[kin\_mem] (\id{void *})
    pointer to the {\kinsol} memory block.
  \item[mxiter] (\id{long int})
    maximum number of nonlinear iterations.
  \end{args}
}
{
  The return value \id{flag} (of type \id{int}) is one of:
  \begin{args}[KIN\_ILL\_INPUT]
  \item[\Id{KIN\_SUCCESS}] 
    The optional value has been successfully set.
  \item[\Id{KIN\_MEM\_NULL}]
    The \id{kin\_mem} pointer is \id{NULL}.
  \item[\Id{KIN\_ILL\_INPUT}]
    The maximum number of iterations was non-positive.
  \end{args}
}
{
  The default value for \id{mxiter} is \id{MXITER\_DEFAULT} $=200$.
}
%%
%%
\ucfunction{KINSetNoInitSetup}
{
flag = KINSetNoInitSetup(kin\_mem, noInitSetup);
}
{
  The function \ID{KINSetNoInitSetup} specifies whether an initial call
  to the preconditioner setup function should be made or not.
}
{
  \begin{args}[noInitSeti[]
  \item[kin\_mem] (\id{void *})
    pointer to the {\kinsol} memory block.
  \item[noInitSetup] (\id{booleantype})
    flag controlling whether or not an initial call 
    to the preconditioner setup function is made.
  \end{args}
}
{
  The return value \id{flag} (of type \id{int}) is one of:
  \begin{args}[KIN\_MEM\_NULL]
  \item[\Id{KIN\_SUCCESS}] 
    The optional value has been successfully set.
  \item[\Id{KIN\_MEM\_NULL}]
    The \id{kin\_mem} pointer is \id{NULL}.
  \end{args}
}
{
  The default value for \id{noInitSetup} is \id{FALSE}, meaning that an initial call
  to the preconditioner setup function will be made.
}
%%
%%
\ucfunction{KINSetNoResMon}
{
flag = KINSetNoResMon(kin\_mem, noNNIResMon);
}
{
  The function \ID{KINSetNoResMon} specifies whether or not the nonlinear
  residual monitoring scheme is used to control Jacobian updating.
}
{
  \begin{args}[noNNIResMon]
  \item[kin\_mem] (\id{void *})
    pointer to the {\kinsol} memory block.
  \item[noNNIResMon] (\id{booleantype})
    flag controlling whether or not residual monitoring is used.
  \end{args}
}
{
  The return value \id{flag} (of type \id{int}) is one of:
  \begin{args}[KIN\_MEM\_NULL]
  \item[\Id{KIN\_SUCCESS}] 
    The optional value has been successfully set.
  \item[\Id{KIN\_MEM\_NULL}]
    The \id{kin\_mem} pointer is \id{NULL}.
  \end{args}
}
{
  When using a direct solver, the default value for \id{noNNIResMon} is \id{FALSE},
  meaning that the nonlinear residual will be monitored. Otherwise, \id{noNNIResMon}
  is set to \id{TRUE}.

  {\warn}Residual monitoring is only available for use with the direct linear solver
  modules (meaning {\kindense} and {\kinband}).
}
%%
%%
\ucfunction{KINSetMaxSetupCalls}
{
flag = KINSetMaxSetupCalls(kin\_mem, msbset);
}
{
  The function \ID{KINSetMaxSetupCalls} specifies the maximum number of 
  nonlinear iterations that can be performed between calls to the 
  preconditioner setup function.
}
{
  \begin{args}[kin\_mem]
  \item[kin\_mem] (\id{void *})
    pointer to the {\kinsol} memory block.
  \item[msbset] (\id{long int})
    maximum number of nonlinear iterations without a call
    to the preconditioner setup function.
  \end{args}
}
{
  The return value \id{flag} (of type \id{int}) is one of:
  \begin{args}[KIN\_ILL\_INPUT]
  \item[\Id{KIN\_SUCCESS}] 
    The optional value has been successfully set.
  \item[\Id{KIN\_MEM\_NULL}]
    The \id{kin\_mem} pointer is \id{NULL}.
  \item[\Id{KIN\_ILL\_INPUT}]
    The argument \id{msbset} was negative.
  \end{args}
}
{
  The default value for \id{msbset} is \id{MSBSET\_DEFAULT} $=10$.
}
%%
%%
\ucfunction{KINSetMaxSubSetupCalls}
{
flag = KINSetMaxSubSetupCalls(kin\_mem, msbsetsub);
}
{
  The function \ID{KINSetMaxSubSetupCalls} specifies the maximum number of 
  nonlinear iterations between checks by the residual monitoring algorithm.
}
{
  \begin{args}[kin\_mem]
  \item[kin\_mem] (\id{void *})
    pointer to the {\kinsol} memory block.
  \item[msbsetsub] (\id{long int})
    maximum number of nonlinear iterations without checking the
    nonlinear residual.
  \end{args}
}
{
  The return value \id{flag} (of type \id{int}) is one of:
  \begin{args}[KIN\_ILL\_INPUT]
  \item[\Id{KIN\_SUCCESS}] 
    The optional value has been successfully set.
  \item[\Id{KIN\_MEM\_NULL}]
    The \id{kin\_mem} pointer is \id{NULL}.
  \item[\Id{KIN\_ILL\_INPUT}]
    The argument \id{msbsetsub} was negative.
  \end{args}
}
{
  The default value for \id{msbsetsub} is \id{MSBSET\_SUB\_DEFAULT} $=5$.

  {\warn}Residual monitoring is only available for use with the direct linear solver
  modules (meaning {\kindense} and {\kinband}).
}
%%
%%
\ucfunction{KINSetEtaForm}
{
flag = KINSetEtaForm(kin\_mem, etachoice);
}
{
  The function \ID{KINSetEtaForm} specifies the method for computing
  the value of the $\eta$ coefficient used in the calculation of the
  linear solver convergence tolerance.
}
{
  \begin{args}[etachoice]
  \item[kin\_mem] (\id{void *})
    pointer to the {\kinsol} memory block.
  \item[etachoice] (\id{int})
    flag indicating the method for computing $\eta$. \id{etachoice} must be one
    of \Id{KIN\_ETACHOICE1}, \Id{KIN\_ETACHOICE2}, or \Id{KIN\_ETACONSTANT}
    (see Chapter \ref{s:math} for details).
  \end{args}
}
{
  The return value \id{flag} (of type \id{int}) is one of:
  \begin{args}[KIN\_ILL\_INPUT]
  \item[\Id{KIN\_SUCCESS}] 
    The optional value has been successfully set.
  \item[\Id{KIN\_MEM\_NULL}]
    The \id{kin\_mem} pointer is \id{NULL}.
  \item[\Id{KIN\_ILL\_INPUT}]
    The argument \id{etachoice} had an illegal value.
  \end{args}
}
{
  The default value for \id{etachoice} is \id{KIN\_ETACHOICE1}.
}
%%
%%
\ucfunction{KINSetEtaConstValue}
{
flag = KINSetEtaConstValue(kin\_mem, eta);
}
{
  The function \ID{KINSetEtaConstValue} specifies the constant value
  for $\eta$ in the case \\ \id{etachoice = KIN\_ETACONSTANT}.
}
{
  \begin{args}[kin\_mem]
  \item[kin\_mem] (\id{void *})
    pointer to the {\kinsol} memory block.
  \item[eta] (\id{realtype})
    constant value for $\eta$.
  \end{args}
}
{
  The return value \id{flag} (of type \id{int}) is one of:
  \begin{args}[KIN\_ILL\_INPUT]
  \item[\Id{KIN\_SUCCESS}] 
    The optional value has been successfully set.
  \item[\Id{KIN\_MEM\_NULL}]
    The \id{kin\_mem} pointer is \id{NULL}.
  \item[\Id{KIN\_ILL\_INPUT}]
    The argument \id{eta} had an illegal value
  \end{args}
}
{
  The default value for \id{eta} is $0.1$.
  The valid values are $0.0 <$ \id{eta} $\le 1.0$.
}
%%
%%
\ucfunction{KINSetEtaParams}
{
flag = KINSetEtaParams(kin\_mem, egamma, ealpha);
}
{
  The function \ID{KINSetEtaParams} specifies the parameters $\gamma$ and
  $\alpha$ in the formula for $\eta$, in the case \id{etachoice = KIN\_ETACHOICE2}.
}
{
  \begin{args}[kin\_mem]
  \item[kin\_mem] (\id{void *})
    pointer to the {\kinsol} memory block.
  \item[egamma] (\id{realtype})
    value of the $\gamma$ parameter.
  \item[ealpha] (\id{realtype})
    value of the $\alpha$ parameter.
  \end{args}
}
{
  The return value \id{flag} (of type \id{int}) is one of:
  \begin{args}[KIN\_ILL\_INPUT]
  \item[\Id{KIN\_SUCCESS}] 
    The optional values have been successfully set.
  \item[\Id{KIN\_MEM\_NULL}]
    The \id{kin\_mem} pointer is \id{NULL}.
  \item[\Id{KIN\_ILL\_INPUT}]
    One of the arguments \id{egamma} or \id{ealpha} had an illegal value.
  \end{args}
}
{
  The default values for \id{egamma} and \id{ealpha} are $0.9$ and $2.0$, respectively.

  The valid values for \id{ealpha} are $1.0<$ \id{ealpha} $\le 2.0$. 
  If \id{ealpha} $= 0.0$, then its value is set to $2.0$.

  The valid values for \id{egamma} are $0.0 <$ \id{egamma} $\le1.0$.
  If \id{egamma} $= 0.0$, then its value is set to $0.9$.
}
%%
%%
\ucfunction{KINSetResMonConstValue}
{
flag = KINSetResMonConstValue(kin\_mem, omegaconst);
}
{
  The function \ID{KINSetResMonConstValue} specifies the constant value
  for $\omega$ when using residual monitoring.
}
{
  \begin{args}[kin\_mem]
  \item[kin\_mem] (\id{void *})
    pointer to the {\kinsol} memory block.
  \item[omegaconst] (\id{realtype})
    constant value for $\omega$.
  \end{args}
}
{
  The return value \id{flag} (of type \id{int}) is one of:
  \begin{args}[KIN\_ILL\_INPUT]
  \item[\Id{KIN\_SUCCESS}] 
    The optional value has been successfully set.
  \item[\Id{KIN\_MEM\_NULL}]
    The \id{kin\_mem} pointer is \id{NULL}.
  \item[\Id{KIN\_ILL\_INPUT}]
    The argument \id{omegaconst} had an illegal value
  \end{args}
}
{
  The default value for \id{omega} is $0.9$.
  The valid values are $0.0 <$ \id{omega} $< 1.0$.
}
%%
%%
\ucfunction{KINSetResMonParams}
{
flag = KINSetResMonParams(kin\_mem, omegamin, omegamax);
}
{
  The function \ID{KINSetResMonParams} specifies the parameters $\omega_{min}$ and
  $\omega_{max}$ in the formula for $\omega$.
}
{
  \begin{args}[kin\_mem]
  \item[kin\_mem] (\id{void *})
    pointer to the {\kinsol} memory block.
  \item[omegamin] (\id{realtype})
    value of the $\omega_{min}$ parameter.
  \item[omegamax] (\id{realtype})
    value of the $\omega_{max}$ parameter.
  \end{args}
}
{
  The return value \id{flag} (of type \id{int}) is one of:
  \begin{args}[KIN\_ILL\_INPUT]
  \item[\Id{KIN\_SUCCESS}] 
    The optional values have been successfully set.
  \item[\Id{KIN\_MEM\_NULL}]
    The \id{kin\_mem} pointer is \id{NULL}.
  \item[\Id{KIN\_ILL\_INPUT}]
    One of the arguments \id{omegamin} or \id{omegamax} had an illegal value.
  \end{args}
}
{
  The default values for \id{omegamin} and \id{omegamax} are $0.00001$ and $0.9$, respectively.

  The valid values for \id{omegamin} and \id{omegamax} are
  $0.0 <$ \id{omegamin} $<$ \id{omegamax} $< 1.0$.
  If \id{omegamin} $= 0.0$, then its value is set to $0.00001$.
  If \id{omegamax} $= 0.0$, then its value is set to $0.9$.
}
%%
%%
\ucfunction{KINSetNoMinEps}
{
flag = KINSetNoMinEps(kin\_mem, noMinEps);
}
{
  The function \ID{KINSetNoMinEps} specifies a flag that controls whether or not
  the value of $\epsilon$, the scaled linear residual tolerance, is
  bounded from below.
}
{
  \begin{args}[noMinEps]
  \item[kin\_mem] (\id{void *})
    pointer to the {\kinsol} memory block.
  \item[noMinEps] (\id{booleantype})
    flag controlling the bound on $\epsilon$.
  \end{args}
}
{
  The return value \id{flag} (of type \id{int}) is one of:
  \begin{args}[KIN\_MEM\_NULL]
  \item[\Id{KIN\_SUCCESS}] 
    The optional value has been successfully set.
  \item[\Id{KIN\_MEM\_NULL}]
    The \id{kin\_mem} pointer is \id{NULL}.
  \end{args}
}
{
  The default value for \id{noMinEps} is \id{FALSE}.
}
%%
%%
\ucfunction{KINSetMaxNewtonStep}
{
flag = KINSetMaxNewtonStep(kin\_mem, mxnewtstep);
}
{
  The function \ID{KINSetMaxNewtonStep} specifies the maximum allowable scaled
  length of the Newton step.
}
{
  \begin{args}[mxnewtstep]
  \item[kin\_mem] (\id{void *})
    pointer to the {\kinsol} memory block.
  \item[mxnewtstep] (\id{realtype})
    maximum scaled step length.
  \end{args}
}
{
  The return value \id{flag} (of type \id{int}) is one of:
  \begin{args}[KIN\_ILL\_INPUT]
  \item[\Id{KIN\_SUCCESS}] 
    The optional value has been successfully set.
  \item[\Id{KIN\_MEM\_NULL}]
    The \id{kin\_mem} pointer is \id{NULL}.
  \item[\Id{KIN\_ILL\_INPUT}]
    The maximum step was non-positive.
  \end{args}
}
{
  The default value of \id{mxnewtstep} is $1000\, \| u_0 \|_{D_u}$,
  where $u_0$ is the initial guess.
}
%%
%%
\ucfunction{KINSetMaxBetaFails}
{
flag = KINSetMaxBetaFails(kin\_mem, mxnbcf);
}
{
  The function \ID{KINSetMaxBetaFails} specifies the maximum number of
  $\beta$-condition failures in the linesearch algorithm.
}
{
  \begin{args}[kin\_mem]
  \item[kin\_mem] (\id{void *})
    pointer to the {\kinsol} memory block.
  \item[mxnbcf] (\id{realtype})
    maximum number of $\beta$-condition failures.
  \end{args}
}
{
  The return value \id{flag} (of type \id{int}) is one of:
  \begin{args}[KIN\_ILL\_INPUT]
  \item[\Id{KIN\_SUCCESS}] 
    The optional value has been successfully set.
  \item[\Id{KIN\_MEM\_NULL}]
    The \id{kin\_mem} pointer is \id{NULL}.
  \item[\Id{KIN\_ILL\_INPUT}]
    \id{mxnbcf} was non-positive.
  \end{args}
}
{
  The default value of \id{mxnbcf} is \id{MXNBCF\_DEFAULT} $=10$.
}
%%
%%
\ucfunction{KINSetRelErrFunc}
{
flag = KINSetRelErrFunc(kin\_mem, relfunc);
}
{
  The function \ID{KINSetRelErrFunc} specifies the relative error in 
  computing $F(u)$, which is used in the difference quotient approximation of the
  Jacobian-vector product.
}
{
  \begin{args}[relfunc]
  \item[kin\_mem] (\id{void *})
    pointer to the {\kinsol} memory block.
  \item[relfunc] (\id{realtype})
    relative error in $F(u)$.
  \end{args}
}
{
  The return value \id{flag} (of type \id{int}) is one of:
  \begin{args}[KIN\_ILL\_INPUT]
  \item[\Id{KIN\_SUCCESS}] 
    The optional value has been successfully set.
  \item[\Id{KIN\_MEM\_NULL}]
    The \id{kin\_mem} pointer is \id{NULL}.
  \item[\Id{KIN\_ILL\_INPUT}]
    The relative error was non-positive.
  \end{args}
}
{
  The default value for \id{relfunc} is $\sqrt{\text{unit roundoff}}$.
}
%%
%%
\ucfunction{KINSetFuncNormTol}
{
flag = KINSetFuncNormTol(kin\_mem, fnormtol);
}
{
  The function \ID{KINSetFuncNormTol} specifies the scalar used as a stopping
  tolerance on the scaled maximum norm of the system function $F(u)$.
}
{
  \begin{args}[fnormtol]
  \item[kin\_mem] (\id{void *})
    pointer to the {\kinsol} memory block.
  \item[fnormtol] (\id{realtype})
    tolerance for stopping based on scaled function norm.
  \end{args}
}
{
  The return value \id{flag} (of type \id{int}) is one of:
  \begin{args}[KIN\_ILL\_INPUT]
  \item[\Id{KIN\_SUCCESS}] 
    The optional value has been successfully set.
  \item[\Id{KIN\_MEM\_NULL}]
    The \id{kin\_mem} pointer is \id{NULL}.
  \item[\Id{KIN\_ILL\_INPUT}]
    The tolerance was non-positive.
  \end{args}
}
{
  The default value for \id{fnormtol} is $\sqrt[3]{\text{unit roundoff}}$.
}
%%
%%
\ucfunction{KINSetScaledStepTol}
{
flag = KINSetScaledStepTol(kin\_mem, scsteptol);
}
{
  The function \ID{KINSetScaledStepTol} specifies the scalar used
  as a stopping tolerance on the minimum scaled step length.
}
{
  \begin{args}[scsteptol]
  \item[kin\_mem] (\id{void *})
    pointer to the {\kinsol} memory block.
  \item[scsteptol] (\id{realtype})
    tolerance for stopping based on scaled step length.. 
  \end{args}
}
{
  The return value \id{flag} (of type \id{int}) is one of:
  \begin{args}[KIN\_ILL\_INPUT]
  \item[\Id{KIN\_SUCCESS}] 
    The optional value has been successfully set.
  \item[\Id{KIN\_MEM\_NULL}]
    The \id{kin\_mem} pointer is \id{NULL}.
  \item[\Id{KIN\_ILL\_INPUT}]
    The tolerance was non-positive.
  \end{args}
}
{
  The default value for \id{scsteptol} is $(\text{unit roundoff})^{2/3}$.
}
%%
%%
\ucfunction{KINSetConstraints}
{
flag = KINSetConstraints(kin\_mem, constraints);
}
{
  The function \ID{KINSetConstraints} specifies a vector that defines
  inequality constraints for each component of the solution vector $u$.
}
{
  \begin{args}[constraints]
  \item[kin\_mem] (\id{void *})
    pointer to the {\kinsol} memory block.
  \item[constraints] (\id{N\_Vector})
    vector of constraint flags. If \id{constraints[i]} is
    \begin{itemize}    
    \item[$0.0$] then no constraint is imposed on $u_i$. 
    \item[$1.0$] then $u_i$ will be constrained to be $u_i \ge 0.0$.
    \item[$-1.0$] then $u_i$ will be constrained to be $u_i \le 0.0$.
    \item[$2.0$] then $u_i$ will be constrained to be $u_i > 0.0$.
    \item[$-2.0$] then $u_i$ will be constrained to be $u_i < 0.0$.
    \end{itemize}
  \end{args}
}
{
  The return value \id{flag} (of type \id{int}) is one of:
  \begin{args}[KIN\_ILL\_INPUT]
  \item[\Id{KIN\_SUCCESS}] 
    The optional value has been successfully set.
  \item[\Id{KIN\_MEM\_NULL}]
    The \id{kin\_mem} pointer is \id{NULL}.
  \item[\Id{KIN\_ILL\_INPUT}]
    The constraint vector contains illegal values.
  \end{args}
}
{
  The presence of a non-\id{NULL} constraints vector that is not $0.0$ in
  all components will cause constraint checking to be performed.

  The function creates a private copy of the constraints vector. Consequently,
  the user-supplied vector can be freed after the function call, and
  the constraints can only be changed by calling this function.
}
%%
%%
\ucfunction{KINSetSysFunc}
{
flag = KINSetSysFunc(kin\_mem, func);
}
{
  The function \ID{KINSetSysFunc} specifies the user-provided function
  that evaluates the nonlinear system function $F(u)$.
}
{
  \begin{args}[kin\_mem]
  \item[kin\_mem] (\id{void *})
    pointer to the {\kinsol} memory block.
  \item[func] (\id{KINSysFn})
    user-supplied function that evaluates $F(u)$.
  \end{args}
}
{
  The return value \id{flag} (of type \id{int}) is one of:
  \begin{args}[KIN\_ILL\_INPUT]
  \item[\Id{KIN\_SUCCESS}] 
    The optional value has been successfully set.
  \item[\Id{KIN\_MEM\_NULL}]
    The \id{kin\_mem} pointer is \id{NULL}.
  \item[\Id{KIN\_ILL\_INPUT}]
    The argument \id{func} was \id{NULL}.
  \end{args}
}
{
  The nonlinear system function is initially specified through \id{KINMalloc}.
  The option of changing the system function is provided for a user who wishes 
  to solve several problems of the same size but with different functions.
}
%%
\index{optional input!solver|)}

%%==============================================================================
\subsubsection{Dense linear solver}\label{sss:optin_dense}
%%==============================================================================
\index{optional input!dense linear solver|(}
\index{KINDENSE@{\kindense} linear solver!optional input|(}
The \index{KINDENSE@{\kindense} linear solver!Jacobian approximation used by}
{\kindense} solver needs a function to compute a dense approximation to
the Jacobian matrix $J(u)$.  This function must be of type \id{KINDenseJacFn}. 
The user can supply his/her own dense Jacobian function, or use the default 
difference quotient function \Id{KINDenseDQJac} 
\index{Jacobian approximation function!dense!difference quotient}
that comes with the {\kindense} solver.
To specify a user-supplied Jacobian function \id{djac} and associated user 
data \id{jac\_data}, {\kindense} provides the function \id{KINDenseSetJacFn}.
The {\kindense} solver passes the pointer \id{jac\_data} 
to its dense Jacobian function. This allows the user to
create an arbitrary structure with relevant problem data and access it
during the execution of the user-supplied Jacobian function, without
using global data in the program.  The pointer \id{jac\_data} may be
identical to \id{f\_data}, if the latter was specified through \id{KINSetFdata}.
%%
\index{Jacobian approximation function!dense!user-supplied}
\ucfunction{KINDenseSetJacFn}
{
  flag = KINDenseSetJacFn(kin\_mem, djac, jac\_data);
}
{
  The function \ID{KINDenseSetJacFn} specifies the dense Jacobian
  approximation function to be used and the pointer to user data.
}
{
  \begin{args}[kin\_mem]
  \item[kin\_mem] (\id{void *})
    pointer to the {\kinsol} memory block.
  \item[djac] (\id{KINDenseJacFn})
    user-defined dense Jacobian approximation function.
  \item[jac\_data] (\id{void *})
    pointer to the user-defined data structure.
  \end{args}
}
{
  The return value \id{flag} (of type \id{int}) is one of
  \begin{args}[KINDENSE\_LMEM\_NULL]
  \item[\Id{KINDENSE\_SUCCESS}] 
    The optional value has been successfully set.
  \item[\Id{KINDENSE\_MEM\_NULL}]
    The \id{kin\_mem} pointer is \id{NULL}.
  \item[\Id{KINDENSE\_LMEM\_NULL}]
    The {\kindense} linear solver has not been initialized.
  \end{args}
}
{
  By default, {\kindense} uses the difference quotient function \id{KINDenseDQJac}.
  If \id{NULL} is passed to \id{djac}, this default function is used.

  The function type \id{KINDenseJacFn} is described in \S\ref{ss:djacFn}.
}
\index{KINDENSE@{\kindense} linear solver!optional input|)}
\index{optional input!dense linear solver|)}

%%==============================================================================
\subsubsection{Band linear solver}\label{sss:optin_band}
%%==============================================================================
\index{optional input!band linear solver|(}
\index{KINBAND@{\kinband} linear solver!optional input|(}
The \index{KINBAND@{\kinband} linear solver!Jacobian approximation used by}
{\kindense} solver needs a function to compute a banded approximation to
the Jacobian matrix $J(u)$.  This function must be of type \id{KINBandJacFn}. 
The user can supply his/her own banded Jacobian approximation function, 
or use the default difference quotient function \Id{KINBandDQJac} 
\index{Jacobian approximation function!band!difference quotient}
that comes with the {\kinband} solver.
To specify a user-supplied Jacobian function \id{bjac} and associated user 
data \id{jac\_data}, {\kinband} provides the function \id{KINBandSetJacFn}.
The {\kinband} solver passes the pointer \id{jac\_data}
to its banded Jacobian approximation function. This allows the user to
create an arbitrary structure with relevant problem data and access it
during the execution of the user-supplied Jacobian function, without
using global data in the program.  The pointer \id{jac\_data} may be
identical to \id{f\_data}, if the latter was specified through \id{KINSetFdata}.
%%
\index{Jacobian approximation function!band!user-supplied}
\ucfunction{KINBandSetJacFn}
{
  flag = KINBandSetJacFn(cvode\_mem, bjac, jac\_data);
}
{
  The function \ID{KINBandSetJacFn} specifies the banded Jacobian
  approximation function to be used and the pointer to user data.
}
{
  \begin{args}[kin\_mem]
  \item[kin\_mem] (\id{void *})
    pointer to the {\kinsol} memory block.
  \item[bjac] (\id{KINBandJacFn})
    user-defined banded Jacobian approximation function.
  \item[jac\_data] (\id{void *})
    pointer to the user-defined data structure.
  \end{args}
}
{
  The return value \id{flag} (of type \id{int}) is one of
  \begin{args}[KINBAND\_LMEM\_NULL]
  \item[\Id{KINBAND\_SUCCESS}] 
    The optional value has been successfully set.
  \item[\Id{KINBAND\_MEM\_NULL}]
    The \id{kin\_mem} pointer is \id{NULL}.
  \item[\Id{KINBAND\_LMEM\_NULL}]
    The {\kinband} linear solver has not been initialized.
  \end{args}
}
{
  By default, {\kinband} uses the difference quotient function \id{KINBandDQJac}.
  If \id{NULL} is passed to \id{bjac}, this default function is used.

  The function type \id{KINBandJacFn} is described in \S\ref{ss:bjacFn}.
}
\index{KINBAND@{\kinband} linear solver!optional input|)}
\index{optional input!band linear solver|)}

%%==============================================================================
\subsubsection{SPILS linear solvers}\label{sss:optin_spils}
%%==============================================================================
\index{optional input!iterative linear solver|(}
\index{KINSPGMR@{\kinspgmr} linear solver!optional input|(}
\index{KINSPBCG@{\kinspbcg} linear solver!optional input|(}
\index{KINSPTFQMR@{\kinsptfqmr} linear solver!optional input|(}
\index{preconditioning!user-supplied|(}
If any preconditioning is to be done with one of the {\kinspils} linear
solvers, then the user must supply a preconditioner solve function \id{psolve}
and specify its name in a call to \id{KINSpilsSetPreconditioner}.
\index{KINSPGMR@{\kinspgmr} linear solver!preconditioner solve function}
\index{KINSPBCG@{\kinspbcg} linear solver!preconditioner solve function}
\index{KINSPTFQMR@{\kinsptfqmr} linear solver!preconditioner solve function}
%%
The evaluation and preprocessing of any Jacobian-related data needed
by the user's preconditioner solve function is done in the optional
user-supplied function \id{psetup}. Both of these functions are
fully specified in \S\ref{ss:user_fct_sol}.
If used, the \id{psetup} function should also be specified in the call to
\id{KINSpilsSetPreconditioner}.
\index{KINSPGMR@{\kinspgmr} linear solver!preconditioner setup function}
\index{KINSPBCG@{\kinspbcg} linear solver!preconditioner setup function}
\index{KINSPTFQMR@{\kinsptfqmr} linear solver!preconditioner setup function}
%%
Optionally, a {\kinspils} solver passes the pointer \id{p\_data}
received through \id{KINSpilsSetPreconditioner} to the preconditioner
\id{psetup} and \id{psolve} functions.  This allows the user to create
an arbitrary structure with relevant problem data and access it during
the execution of the user-supplied preconditioner functions without using
global data in the program.  The pointer \id{p\_data} may be identical
to \id{f\_data}, if the latter was specified through \id{KINSetFdata}.

\index{KINSPGMR@{\kinspgmr} linear solver!Jacobian approximation used by}
\index{KINSPBCG@{\kinspbcg} linear solver!Jacobian approximation used by}
\index{KINSPTFQMR@{\kinsptfqmr} linear solver!Jacobian approximation used by}
Ther {\kinspils} solvers require a function to compute an approximation to the
product between the Jacobian matrix $J(u)$ and a vector $v$.
The user can supply his/her own Jacobian-times-vector approximation function, 
or use the difference quotient function \Id{KINSpilsDQJtimes} 
\index{Jacobian approximation function!Jacobian times vector!difference quotient}
that comes with the {\kinspils} solvers.  A user-defined Jacobian-vector
function must be of type \id{KINSpilsJacTimesVecFn} and 
can be specified through a call to \id{KINSpilsSetJacTimesVecFn} 
(see \S\ref{ss:jtimesFn} for specification details).
%%
As with the preconditioner user data structure \id{p\_data}, 
the user can also specify, in the call to \id{KINSpilsSetJacTimesVecFn}, a
pointer to a user-defined data structure, \id{jac\_data}, which
the {\kinspils} solver passes to the Jacobian-times-vector function \id{jtimes}
each time it is called.  
The pointer \id{jac\_data} may be identical to \id{p\_data} and/or \id{f\_data}.
%%
%%
\ucfunction{KINSpilsSetPreconditioner}
{
  flag = KINSpilsSetPreconditioner(kin\_mem, psetup, psolve, p\_data);
}
{
  The function \ID{KINSpilsSetPreconditioner} specifies the preconditioner
  setup and solve functions and the pointer to user data.
}
{
  \begin{args}[kin\_mem]
  \item[kin\_mem] (\id{void *})
    pointer to the {\kinsol} memory block.
  \item[psetup] (\id{KINSpilsPrecSetupFn})
    user-defined preconditioner setup function.
  \item[psolve] (\id{KINSpilsPrecSolveFn})
    user-defined preconditioner solve function.
  \item[p\_data] (\id{void *})
     pointer to the user-defined data structure.
  \end{args}
}
{
  The return value \id{flag} (of type \id{int}) is one of
  \begin{args}[KINSPILS\_LMEM\_NULL]
  \item[\Id{KINSPILS\_SUCCESS}] 
    The optional value has been successfully set.
  \item[\Id{KINSPILS\_MEM\_NULL}]
    The \id{kin\_mem} pointer is \id{NULL}.
  \item[\Id{KINSPILS\_LMEM\_NULL}]
    The {\kinspils} linear solver has not been initialized.
  \end{args}
}
{
   The function type \id{KINSpilsPrecSolveFn} is described in \S\ref{ss:psolveFn}.
   The function type \id{KINSpilsPrecSetupFn} is described in \S\ref{ss:precondFn}.
}
%%
%%
\index{preconditioning!user-supplied|)}
\index{Jacobian approximation function!Jacobian times vector!user-supplied}
\ucfunction{KINSpilsSetJacTimesVecFn}
{
  flag = KINSpilsSetJacTimesVecFn(kin\_mem, jtimes, jac\_data);
}
{
  The function \ID{KINSpilsSetJacTimesFn} specifies the Jacobian-vector 
  function to be used and the pointer to user data.
}
{
  \begin{args}[kin\_mem]
  \item[kin\_mem] (\id{void *})
    pointer to the {\kinsol} memory block.
  \item[jtimes] (\id{KINSpilsJacTimesVecFn})
    user-defined Jacobian-vector product function.
  \item[jac\_data] (\id{void *})
     pointer to the user-defined data structure.
  \end{args}
}
{
  The return value \id{flag} (of type \id{int}) is one of
  \begin{args}[KINSPILS\_LMEM\_NULL]
  \item[\Id{KINSPILS\_SUCCESS}] 
    The optional value has been successfully set.
  \item[\Id{KINSPILS\_MEM\_NULL}]
    The \id{kin\_mem} pointer is \id{NULL}.
  \item[\Id{KINSPILS\_LMEM\_NULL}]
    The {\kinspils} linear solver has not been initialized.
  \end{args}
}
{
  By default, the {\kinspils} linear solvers use an internal difference quotient 
  function \id{KINSpilsDQJtimes}.
  If \id{NULL} is passed to \id{jtimes}, this default function is used.

  The function type \id{KINSpilsJacTimesVecFn} is described in \S\ref{ss:jtimesFn}.
}
%%
%%
\ucfunction{KINSpilsSetMaxRestarts}
{
  flag = KINSpilsSetMaxRestarts(kin\_mem, maxrs);
}
{
  The function \ID{KINSpilsSetMaxRestarts} specifies the maximum
  number of times the iterative linear solver can be restarted.
}
{
  \begin{args}[kin\_mem]
  \item[kin\_mem] (\id{void *})
    pointer to the {\kinsol} memory block.
  \item[maxrs] (\id{int})
    maximum number of restarts.
  \end{args}
}
{
  The return value \id{flag} (of type \id{int}) is one of:
  \begin{args}[KINSPILS\_ILL\_INPUT]
  \item[\Id{KINSPILS\_SUCCESS}] 
    The optional value has been successfully set.
  \item[\Id{KINSPILS\_ILL\_INPUT}]
    The maximum number of restarts specified is negative.
  \item[\Id{KINSPILS\_MEM\_NULL}]
    The \id{kin\_mem} pointer is \id{NULL}.
  \item[\Id{KINSPILS\_LMEM\_NULL}]
    The linear solver has not been initialized.
  \end{args}
}
{
  The default value is $0$ (meaning no restarts).

  {\warn}This option is available only for the {\kinspgmr} linear solver.
}
%%
\index{KINSPTFQMR@{\kinsptfqmr} linear solver!optional input|)}
\index{KINSPBCG@{\kinspbcg} linear solver!optional input|)}
\index{KINSPGMR@{\kinspgmr} linear solver!optional input|)}
\index{optional input!iterative linear solver|)}

%%==============================================================================
\subsection{Optional output functions}\label{ss:optional_output}
%%==============================================================================

{\kinsol} provides an extensive list of functions that can be used to obtain
solver performance information.
Table \ref{t:optional_output} lists all optional output functions in {\kinsol},
which are then described in detail in the remainder of this section, beginning with those
for the main {\kinsol} solver and continuing with those for the linear solver
modules. Where the name of an output from a linear solver module would otherwise conflict
with the name of an optional output from the main solver, a suffix \id{LS} (for Linear Solver)
has been added here ({\em e.g.}, \id{lenrwLS}).

\newlength{\colABC}
\settowidth{\colABC}{No. of r.h.s. calls for finite diff. Jacobian-vector evals.}
\newlength{\colDEF}
\settowidth{\colDEF}{\id{KINGetNumNonlinSolvConvFails}}

\tablecaption{Optional outputs from {\kinsol}, {\kindense}, {\kinband}, and {\kinspils}}
\label{t:optional_output}
\tablefirsthead{\hline {\bf Optional output} & {\bf Function name} \\}
\tablehead{\hline \multicolumn{2}{|l|}{\small\slshape continued from last page} \\
           \hline {\bf Optional output} & {\bf Function name} \\ \hline}
\tabletail{\hline \multicolumn{2}{|r|}{\small\slshape continued on next page} \\ \hline}
\tablelasttail{\hline}

\begin{supertabular}{|p{\colABC}|p{\colDEF}|}
\hline

\multicolumn{2}{|c|}{\bf KINSOL main solver} \\
\hline
Size of {\kinsol} real and integer workspaces & \id{KINGetWorkSpace} \\
Number of function evaluations  & \id{KINGetNumFuncEvals} \\
Number of nonlinear iterations & \id{KINGetNumNolinSolvIters} \\
Number of $\beta$-condition failures & \id{KINGetNumBetaCondFails} \\
Number of backtrack operations & \id{KINGetNumBacktrackOps} \\
Scaled norm of $F$ & \id{KINGetFuncNorm} \\
Scaled norm of the step & \id{KINGetStepLength} \\
\hline
\multicolumn{2}{|c|}{\bf KINDENSE linear solver} \\
\hline
Size of {\kindense} real and integer workspaces & \id{KINDenseGetWorkSpace} \\
No. of Jacobian evaluations & \id{KINDenseGetNumJacEvals} \\
No. of r.h.s. calls for finite diff. Jacobian evals. & \id{KINDenseGetNumFuncEvals} \\ 
Last return from a {\kindense} function & \id{KINDenseGetLastFlag} \\ 
\hline
\multicolumn{2}{|c|}{\bf KINBAND linear solver} \\
\hline
Size of {\kinband} real and integer workspaces & \id{KINBandGetWorkSpace} \\
No. of Jacobian evaluations & \id{KINBandGetNumJacEvals} \\
No. of r.h.s. calls for finite diff. Jacobian evals. & \id{KINBandGetNumFuncEvals} \\ 
Last return from a {\kinband} function & \id{KINBandGetLastFlag} \\ 
\hline
\multicolumn{2}{|c|}{\bf KINSPILS linear solvers} \\
\hline
Size of real and integer workspaces & \id{KINSpilsGetWorkSpace} \\
No. of linear iterations & \id{KINSpilsGetNumLinIters} \\
No. of linear convergence failures & \id{KINSpilsGetNumConvFails} \\
No. of preconditioner evaluations & \id{KINSpilsGetNumPrecEvals} \\
No. of preconditioner solves & \id{KINSpilsGetNumPrecSolves} \\
No. of Jacobian-vector product evaluations & \id{KINSpilsGetNumJtimesEvals} \\
No. of fct. calls for finite diff. Jacobian-vector evals. & \id{KINSpilsGetNumFuncEvals} \\ 
Last return from a linear solver function & \id{KINSpilsGetLastFlag} \\ 
\end{supertabular}


%%==============================================================================
\subsubsection{Main solver optional output functions}\label{sss:output_main}
%%==============================================================================
\index{optional output!solver|(}
%%
{\kinsol} provides several user-callable functions that can be used to obtain
different quantities that may be of interest to the user, such as solver workspace
requirements and solver performance statistics.
%%
These optional output functions are described next.
%%
%%
\ucfunction{KINGetWorkSpace}
{
  flag = KINGetWorkSpace(kin\_mem, \&lenrw, \&leniw);
}
{
  The function \ID{KINGetWorkSpace} returns the
  {\kinsol} integer and real workspace sizes.
}
{
  \begin{args}[kin\_mem]
  \item[kin\_mem] (\id{void *})
    pointer to the {\kinsol} memory block.
  \item[lenrw] (\id{long int})
    the number of \id{realtype} values in the {\kinsol} workspace.
  \item[leniw] (\id{long int})
    the number of integer values in the {\kinsol} workspace.
  \end{args}
}
{
  The return value \id{flag} (of type \id{int}) is one of:
  \begin{args}[KIN\_MEM\_NULL]
  \item[\Id{KIN\_SUCCESS}] 
    The optional output values have been successfully set.
  \item[\Id{KIN\_MEM\_NULL}]
    The \id{kin\_mem} pointer is \id{NULL}.
  \end{args}
}
{
  \index{memory requirements!KINSOL@{\kinsol} solver}
  In terms of the problem size $N$, the actual size of the real workspace 
  is $17 + 5 N$ \id{realtype} words. The real workspace is increased by
  an additional $N$ words if constraint checking is enabled (see \id{KINSetConstraints}).

  The actual size of the integer workspace (without distinction between \id{int} 
  and \id{long int}) is $22 + 5 N$ (increased by $N$ if constraint checking is enabled).
}
%%
%%
\ucfunction{KINGetNumFuncEvals}
{
  flag = KINGetNumFuncEvals(kin\_mem, \&nfevals);
}
{
  The function \ID{KINGetNumFuncEvals} returns the number of evaluations
  of the system function.
}
{
  \begin{args}[kin\_mem]
  \item[kin\_mem] (\id{void *})
    pointer to the {\kinsol} memory block.
  \item[nfevals] (\id{long int})
    number of calls to the user-supplied function that evaluates $F(u)$.
  \end{args}
}
{
  The return value \id{flag} (of type \id{int}) is one of:
  \begin{args}[KIN\_MEM\_NULL]
  \item[\Id{KIN\_SUCCESS}] 
    The optional output value has been successfully set.
  \item[\Id{KIN\_MEM\_NULL}]
    The \id{kin\_mem} pointer is \id{NULL}.
  \end{args}
}
{}
%%
%%
\ucfunction{KINGetNumNonlinSolvIters}
{
  flag = KINGetNumNonlinSolvIters(kin\_mem, \&nniters);
}
{
  The function \ID{KINGetNumNonlinSolvIters} returns the number
  of nonlinear iterations.
}
{
  \begin{args}[kin\_mem]
  \item[kin\_mem] (\id{void *})
    pointer to the {\kinsol} memory block.
  \item[nniters] (\id{long int})
    number of nonlinear iterations.
  \end{args}
}
{
  The return value \id{flag} (of type \id{int}) is one of:
  \begin{args}[KIN\_MEM\_NULL]
  \item[\Id{KIN\_SUCCESS}] 
    The optional output value has been successfully set.
  \item[\Id{KIN\_MEM\_NULL}]
    The \id{kin\_mem} pointer is \id{NULL}.
  \end{args}
}
{}
%%
%%
\ucfunction{KINGetNumBetaCondFails}
{
  flag = KINGetNumBetaCondFails(kin\_mem, \&nbcfails);
}
{
  The function \ID{KINGetNumBetaCondFails} returns the number
  of $\beta$-condition failures.
}
{
  \begin{args}[nbcfails]
  \item[kin\_mem] (\id{void *})
    pointer to the {\kinsol} memory block.
  \item[nbcfails] (\id{long int})
    number of $\beta$-condition failures.
  \end{args}
}
{
  The return value \id{flag} (of type \id{int}) is one of:
  \begin{args}[KIN\_MEM\_NULL]
  \item[\Id{KIN\_SUCCESS}] 
    The optional output value has been successfully set.
  \item[\Id{KIN\_MEM\_NULL}]
    The \id{kin\_mem} pointer is \id{NULL}.
  \end{args}
}
{}
%%
%%
\ucfunction{KINGetNumBacktrackOps}
{
  flag = KINGetNumBacktrackOps(kin\_mem, \&nbacktr);
}
{
  The function \ID{KINGetNumBacktrackOps} returns the number of
  backtrack operations (step length adjustments) performed by the
  line search algorithm.
}
{
  \begin{args}[kin\_mem]
  \item[kin\_mem] (\id{void *})
    pointer to the {\kinsol} memory block.
  \item[nbacktr] (\id{long int})
    number of backtrack operations.
  \end{args}
}
{
  The return value \id{flag} (of type \id{int}) is one of:
  \begin{args}[KIN\_MEM\_NULL]
  \item[\Id{KIN\_SUCCESS}] 
    The optional output value has been successfully set.
  \item[\Id{KIN\_MEM\_NULL}]
    The \id{kin\_mem} pointer is \id{NULL}.
  \end{args}
}
{}
%%
%%
\ucfunction{KINGetFuncNorm}
{
  flag = KINGetFuncNorm(kin\_mem, \&fnorm);
}
{
  The function \ID{KINGetFuncNorm} returns the scaled Euclidean $\ell_2$ norm of the 
  nonlinear system function $F(u)$ evaluated at the current iterate.
}
{
  \begin{args}[kin\_mem]
  \item[kin\_mem] (\id{void *})
    pointer to the {\kinsol} memory block.
  \item[fnorm] (\id{realtype})
    current scaled norm of $F(u)$.
  \end{args}
}
{
  The return value \id{flag} (of type \id{int}) is one of:
  \begin{args}[KIN\_MEM\_NULL]
  \item[\Id{KIN\_SUCCESS}] 
    The optional output value has been successfully set.
  \item[\Id{KIN\_MEM\_NULL}]
    The \id{kin\_mem} pointer is \id{NULL}.
  \end{args}
}
{}
%%
%%
\ucfunction{KINGetStepLength}
{
  flag = KINGetStepLength(kin\_mem, \&steplength);
}
{
  The function \ID{KINGetStepLength} returns the scaled Euclidean $\ell_2$ norm of
  the step used during the previous iteration.
}
{
  \begin{args}[steplength]
  \item[kin\_mem] (\id{void *})
    pointer to the {\kinsol} memory block.
  \item[steplength] (\id{realtype})
    scaled norm of the Newton step.
  \end{args}
}
{
  The return value \id{flag} (of type \id{int}) is one of:
  \begin{args}[KIN\_MEM\_NULL]
  \item[\Id{KIN\_SUCCESS}] 
    The optional output value has been successfully set.
  \item[\Id{KIN\_MEM\_NULL}]
    The \id{kin\_mem} pointer is \id{NULL}.
  \end{args}
}
{}

\index{optional output!solver|)}

%%==============================================================================
\subsubsection{Dense linear solver}\label{sss:optout_dense}
%%==============================================================================
\index{optional output!dense linear solver|(}
\index{KINDENSE@{\kindense} linear solver!optional output|(}
The following optional outputs are available from the {\kindense} module:
workspace requirements, number of calls to the Jacobian routine, number of 
calls to the system function routine for finite-difference Jacobian approximation,
and last return value from a {\kindense} function.
%%
%%
\index{KINDENSE@{\kindense} linear solver!memory requirements} 
\index{memory requirements!KINDENSE@{\kindense} linear solver}
\ucfunction{KINDenseGetWorkSpace}
{
  flag = KINDenseGetWorkSpace(kin\_mem, \&lenrwLS, \&leniwLS);
}
{
  The function \ID{KINDenseGetWorkSpace} returns the
  {\kindense} real and integer workspace sizes.
}
{
  \begin{args}[kin\_mem]
  \item[kin\_mem] (\id{void *})
    pointer to the {\kinsol} memory block.
  \item[lenrwLS] (\id{long int})
    the number of \id{realtype} values in the {\kindense} workspace.
  \item[leniwLS] (\id{long int})
    the number of integer values in the {\kindense} workspace.
  \end{args}
}
{
  The return value \id{flag} (of type \id{int}) is one of
  \begin{args}[KINDENSE\_LMEM\_NULL]
  \item[\Id{KINDENSE\_SUCCESS}] 
    The optional output value has been successfully set.
  \item[\Id{KINDENSE\_MEM\_NULL}]
    The \id{kin\_mem} pointer is \id{NULL}.
  \item[\Id{KINDENSE\_LMEM\_NULL}]
    The {\kindense} linear solver has not been initialized.
  \end{args}
}
{
  In terms of the problem size $N$, the actual size of the real workspace
  is $2N^2$ \id{realtype} words, and the actual size of the integer workspace
  is $N$ integer words.
}
%%
%%
\ucfunction{KINDenseGetNumJacEvals}
{
  flag = KINDenseGetNumJacEvals(kin\_mem, \&njevals);
}
{
  The function \ID{KINDenseGetNumJacEvals} returns the
  number of calls to the dense Jacobian approximation function.
}
{
  \begin{args}[kin\_mem]
  \item[kin\_mem] (\id{void *})
    pointer to the {\kinsol} memory block.
  \item[njevals] (\id{long int})
    the number of calls to the Jacobian function.
  \end{args}
}
{
  The return value \id{flag} (of type \id{int}) is one of
  \begin{args}[KINDENSE\_LMEM\_NULL]
  \item[\Id{KINDENSE\_SUCCESS}] 
    The optional output value has been successfully set.
  \item[\Id{KINDENSE\_MEM\_NULL}]
    The \id{kin\_mem} pointer is \id{NULL}.
  \item[\Id{KINDENSE\_LMEM\_NULL}]
    The {\kindense} linear solver has not been initialized.
  \end{args}
}
{}
%%
%%
\ucfunction{KINDenseGetNumFuncEvals}
{
  flag = KINDenseGetNumFuncEvals(kin\_mem, \&nfevalsLS);
}
{
  The function \ID{KINDenseGetNumFuncEvals} returns the
  number of calls to the user system function due to the 
  finite difference dense Jacobian approximation.
}
{
  \begin{args}[kin\_mem]
  \item[kin\_mem] (\id{void *})
    pointer to the {\kinsol} memory block.
  \item[nfevalsLS] (\id{long int})
    the number of calls to the user system function.
  \end{args}
}
{
  The return value \id{flag} (of type \id{int}) is one of
  \begin{args}[KINDENSE\_LMEM\_NULL]
  \item[\Id{KINDENSE\_SUCCESS}] 
    The optional output value has been successfully set.
  \item[\Id{KINDENSE\_MEM\_NULL}]
    The \id{kin\_mem} pointer is \id{NULL}.
  \item[\Id{KINDENSE\_LMEM\_NULL}]
    The {\kindense} linear solver has not been initialized.
  \end{args}
}
{
  The value \id{nfevalsLS} is incremented only if the default 
  \id{KINDenseDQJac} difference quotient function is used.
}
%%
%%
\ucfunction{KINDenseGetLastFlag}
{
  flag = KINDenseGetLastFlag(kin\_mem, \&lsflag);
}
{
  The function \ID{KINDenseGetLastFlag} returns the
  last return value from a {\kindense} routine. 
}
{
  \begin{args}[kin\_mem]
  \item[kin\_mem] (\id{void *})
    pointer to the {\kinsol} memory block.
  \item[lsflag] (\id{int})
    the value of the last return flag from a {\kindense} function.
  \end{args}
}
{
  The return value \id{flag} (of type \id{int}) is one of
  \begin{args}[KINDENSE\_LMEM\_NULL]
  \item[\Id{KINDENSE\_SUCCESS}] 
    The optional output value has been successfully set.
  \item[\Id{KINDENSE\_MEM\_NULL}]
    The \id{kin\_mem} pointer is \id{NULL}.
  \item[\Id{KINDENSE\_LMEM\_NULL}]
    The {\kindense} linear solver has not been initialized.
  \end{args}
}
{
  If the {\kindense} setup function failed (\id{KINSol} returned \id{KIN\_LSETUP\_FAIL}),
  then \id{lsflag} is equal to the column index (numbered from one) at which
  a zero diagonal element was encountered during the LU factorization of the 
  dense Jacobian matrix.
}
%%
%%
\index{KINDENSE@{\kindense} linear solver!optional output|)}
\index{optional output!dense linear solver|)}

%%==============================================================================
\subsubsection{Band linear solver}\label{sss:optout_band}
%%==============================================================================
\index{optional output!band linear solver|(}
\index{KINBAND@{\kinband} linear solver!optional output|(}
The following optional outputs are available from the {\kinband} module:
workspace requirements, number of calls to the Jacobian routine, number of 
calls to the right-hand side routine for finite-difference Jacobian approximation,
and last return value from a {\kinband} function.
%%
%%
\index{KINBAND@{\kinband} linear solver!memory requirements} 
\index{memory requirements!KINBAND@{\kinband} linear solver}
\ucfunction{KINBandGetWorkSpace}
{
  flag = KINBandGetWorkSpace(kin\_mem, \&lenrwLS, \&leniwLS);
}
{
  The function \ID{KINBandGetWorkSpace} returns the
  {\kinband} real and integer workspace sizes.
}
{
  \begin{args}[kin\_mem]
  \item[kin\_mem] (\id{void *})
    pointer to the {\cvode} memory block.
  \item[lenrwLS] (\id{long int})
    the number of \id{realtype} values in the {\kinband} workspace.
  \item[leniwLS] (\id{long int})
    the number of integer values in the {\kinband} workspace.
  \end{args}
}
{
  The return value \id{flag} (of type \id{int}) is one of
  \begin{args}[KINBAND\_LMEM\_NULL]
  \item[\Id{KINBAND\_SUCCESS}] 
    The optional output value has been successfully set.
  \item[\Id{KINBAND\_MEM\_NULL}]
    The \id{kin\_mem} pointer is \id{NULL}.
  \item[\Id{KINBAND\_LMEM\_NULL}]
    The {\kinband} linear solver has not been initialized.
  \end{args}
}
{
  In terms of the problem size $N$ and Jacobian half-bandwidths, 
  the actual size of the real workspace is
  $(2$ \id{mupper}$+ 3$ \id{mlower} $+ 2)\, N$ \id{realtype} words,
  and the actual size of the integer workspace is $N$ integer words.
}
%%
%%
\ucfunction{KINBandGetNumJacEvals}
{
  flag = KINBandGetNumJacEvals(kin\_mem, \&njevals);
}
{
  The function \ID{KINBandGetNumJacEvals} returns the
  number of calls to the banded Jacobian approximation function.
}
{
  \begin{args}[kin\_mem]
  \item[kin\_mem] (\id{void *})
    pointer to the {\cvode} memory block.
  \item[njevals] (\id{long int})
    the number of calls to the Jacobian function.
  \end{args}
}
{
  The return value \id{flag} (of type \id{int}) is one of
  \begin{args}[KINBAND\_LMEM\_NULL]
  \item[\Id{KINBAND\_SUCCESS}] 
    The optional output value has been successfully set.
  \item[\Id{KINBAND\_MEM\_NULL}]
    The \id{kin\_mem} pointer is \id{NULL}.
  \item[\Id{KINBAND\_LMEM\_NULL}]
    The {\kinband} linear solver has not been initialized.
  \end{args}
}
{}
%%
%%
\ucfunction{KINBandGetNumRhsEvals}
{
  flag = KINBandGetNumFuncEvals(kin\_mem, \&nfevalsLS);
}
{
  The function \ID{KINBandGetNumFuncEvals} returns the
  number of calls to the user system function due to the 
  finite difference banded Jacobian approximation.
}
{
  \begin{args}[kin\_mem]
  \item[kin\_mem] (\id{void *})
    pointer to the {\cvode} memory block.
  \item[nfevalsLS] (\id{long int})
    the number of calls to the user system function.
  \end{args}
}
{
  The return value \id{flag} (of type \id{int}) is one of
  \begin{args}[KINBAND\_LMEM\_NULL]
  \item[\Id{KINBAND\_SUCCESS}] 
    The optional output value has been successfully set.
  \item[\Id{KINBAND\_MEM\_NULL}]
    The \id{kin\_mem} pointer is \id{NULL}.
  \item[\Id{KINBAND\_LMEM\_NULL}]
    The {\kinband} linear solver has not been initialized.
  \end{args}
}
{
  The value \id{nfevalsLS} is incremented only if the default 
  \id{KINBandDQJac} difference quotient function is used.
}
%%
%%
\ucfunction{KINBandGetLastFlag}
{
  flag = KINBandGetLastFlag(kin\_mem, \&lsflag);
}
{
  The function \ID{KINBandGetLastFlag} returns the
  last return value from a {\kinband} routine. 
}
{
  \begin{args}[kin\_mem]
  \item[kin\_mem] (\id{void *})
    pointer to the {\cvode} memory block.
  \item[lsflag] (\id{int})
    the value of the last return flag from a {\kinband} function.
  \end{args}
}
{
  The return value \id{flag} (of type \id{int}) is one of
  \begin{args}[KINBAND\_LMEM\_NULL]
  \item[\Id{KINBAND\_SUCCESS}] 
    The optional output value has been successfully set.
  \item[\Id{KINBAND\_MEM\_NULL}]
    The \id{kin\_mem} pointer is \id{NULL}.
  \item[\Id{KINBAND\_LMEM\_NULL}]
    The {\kinband} linear solver has not been initialized.
  \end{args}
}
{
  If the {\kinband} setup sunction failed (\id{KINSol} returned \id{KIN\_LSETUP\_FAIL}),
  the value \id{lsflag} is equal to the column index (numbered from one) at which
  a zero diagonal element was encountered during the LU factorization of the 
  banded Jacobian matrix.
}
%%
%%
\index{KINBAND@{\kinband} linear solver!optional output|)}
\index{optional output!band linear solver|)}

%%==============================================================================
\subsubsection{SPILS linear solvers}\label{sss:optout_spils}
%%==============================================================================
\index{optional output!iterative linear solver|(}
\index{KINSPGMR@{\kinspgmr} linear solver!optional output|(} 
\index{KINSPBCG@{\kinspbcg} linear solver!optional output|(} 
\index{KINSPTFQMR@{\kinsptfqmr} linear solver!optional output|(} 
The following optional outputs are available from the {\kinspils}
modules: workspace requirements, number of linear iterations, number of
linear convergence failures, number of calls to the preconditioner
setup and solve routines, number of calls to the Jacobian-vector
product routine, number of calls to the system function routine for
finite-difference Jacobian-vector product approximation, and last
return value from a linear solver function.
%%
%%
\index{KINSPGMR@{\kinspgmr} linear solver!memory requirements} 
\index{KINSPBCG@{\kinspbcg} linear solver!memory requirements} 
\index{KINSPTFQMR@{\kinsptfqmr} linear solver!memory requirements} 
\index{memory requirements!KINSPGMR@{\kinspgmr} linear solver}
\ucfunction{KINSpilsGetWorkSpace}
{
  flag = KINSpilsGetWorkSpace(kin\_mem, \&lenrwLS, \&leniwLS);
}
{
  The function \ID{KINSpilsGetWorkSpace} returns the global sizes of the
  linear solver real and integer workspaces.
}
{
  \begin{args}[kin\_mem]
  \item[kin\_mem] (\id{void *})
    pointer to the {\kinsol} memory block.
  \item[lenrwSG] (\id{long int})
    the number of \id{realtype} values in the linear solver workspace.
  \item[leniwSG] (\id{long int})
    the number of integer values in the linear solver workspace.
  \end{args}
}
{
  The return value \id{flag} (of type \id{int}) is one of:
  \begin{args}[KINSPILS\_LMEM\_NULL]
  \item[\Id{KINSPILS\_SUCCESS}] 
    The optional output values have been successfully set.
  \item[\Id{KINSPILS\_MEM\_NULL}]
    The \id{kin\_mem} pointer is \id{NULL}.
  \item[\Id{KINSPILS\_LMEM\_NULL}]
    The {\kinspgmr} linear solver has not been initialized.
  \end{args}
}
{
  In terms of the problem size $N$ and maximum subspace size \id{maxl}, 
  the actual size of the real workspace is roughly:\\
  (\id{maxl}$+ 3)*N +$ \id{maxl} $*($ \id{maxl}$ + 4) + 1$ \id{realtype}
  words for {\kinspgmr},\\
  $7*N$ \id{realtype} words for {\kinspbcg},\\
  and $11*N$ \id{realtype} words for {\kinsptfqmr}.

  In a parallel setting, this value is global --- summed over all processes.
}
%%
%%
\ucfunction{KINSpilsGetNumLinIters}
{
  flag = KINSpilsGetNumLinIters(kin\_mem, \&nliters);
}
{
  The function \ID{KINSpilsGetNumLinIters} returns the
  cumulative number of linear iterations.
}
{
  \begin{args}[kin\_mem]
  \item[kin\_mem] (\id{void *})
    pointer to the {\kinsol} memory block.
  \item[nliters] (\id{long int})
    the current number of linear iterations.
  \end{args}
}
{
  The return value \id{flag} (of type \id{int}) is one of:
  \begin{args}[KINSPILS\_LMEM\_NULL]
  \item[\Id{KINSPILS\_SUCCESS}] 
    The optional output value has been successfully set.
  \item[\Id{KINSPILS\_MEM\_NULL}]
    The \id{kin\_mem} pointer is \id{NULL}.
  \item[\Id{KINSPLIS\_LMEM\_NULL}]
    The linear solver module has not been initialized.
  \end{args}
}
{}
%%
%%
\ucfunction{KINSpilsGetNumConvFails}
{
  flag = KINSpilsGetNumConvFails(kin\_mem, \&nlcfails);
}
{
  The function \ID{KINSpilsGetNumConvFails} returns the
  cumulative number of linear convergence failures.
}
{
  \begin{args}[kin\_mem]
  \item[kin\_mem] (\id{void *})
    pointer to the {\kinsol} memory block.
  \item[nlcfails] (\id{long int})
    the current number of linear convergence failures.
  \end{args}
}
{
  The return value \id{flag} (of type \id{int}) is one of:
  \begin{args}[KINSPILS\_LMEM\_NULL]
  \item[\Id{KINSPILS\_SUCCESS}] 
    The optional output value has been successfully set.
  \item[\Id{KINSPILS\_MEM\_NULL}]
    The \id{kin\_mem} pointer is \id{NULL}.
  \item[\Id{KINSPILS\_LMEM\_NULL}]
    The linear solver module has not been initialized.
  \end{args}
}
{}
%%
%%
\ucfunction{KINSpilsGetNumPrecEvals}
{
  flag = KINSpilsGetNumPrecEvals(kin\_mem, \&npevals);
}
{
  The function \ID{KINSpilsGetNumPrecEvals} returns the
  number of preconditioner evaluations, i.e., the number of 
  calls made to \id{psetup}.
}
{
  \begin{args}[kin\_mem]
  \item[kin\_mem] (\id{void *})
    pointer to the {\kinsol} memory block.
  \item[npevals] (\id{long int})
    the current number of calls to \id{psetup}.
  \end{args}
}
{
  The return value \id{flag} (of type \id{int}) is one of:
  \begin{args}[KINSPILS\_LMEM\_NULL]
  \item[\Id{KINSPILS\_SUCCESS}] 
    The optional output value has been successfully set.
  \item[\Id{KINSPILS\_MEM\_NULL}]
    The \id{kin\_mem} pointer is \id{NULL}.
  \item[\Id{KINSPILS\_LMEM\_NULL}]
    The linear solver module has not been initialized.
  \end{args}
}
{}
%%
%%
\ucfunction{KINSpilsGetNumPrecSolves}
{
  flag = KINSpilsGetNumPrecSolves(kin\_mem, \&npsolves);
}
{
  The function \ID{KINSpilsGetNumPrecSolves} returns the
  cumulative number of calls made to the preconditioner 
  solve function, \id{psolve}.
}
{
  \begin{args}[kin\_mem]
  \item[kin\_mem] (\id{void *})
    pointer to the {\kinsol} memory block.
  \item[npsolves] (\id{long int})
    the current number of calls to \id{psolve}.
  \end{args}
}
{
  The return value \id{flag} (of type \id{int}) is one of:
  \begin{args}[KINSPILS\_LMEM\_NULL]
  \item[\Id{KINSPILS\_SUCCESS}] 
    The optional output value has been successfully set.
  \item[\Id{KINSPILS\_MEM\_NULL}]
    The \id{kin\_mem} pointer is \id{NULL}.
  \item[\Id{KINSPILS\_LMEM\_NULL}]
    The linear solver module has not been initialized.
  \end{args}
}
{}
%%
%%
\ucfunction{KINSpilsGetNumJtimesEvals}
{
  flag = KINSpilsGetNumJtimesEvals(kin\_mem, \&njvevals);
}
{
  The function \ID{KINSpilsGetNumJtimesEvals} returns the
  cumulative number made to the Jacobian-vector product function,
  \id{jtimes}.
}
{
  \begin{args}[kin\_mem]
  \item[kin\_mem] (\id{void *})
    pointer to the {\kinsol} memory block.
  \item[njvevals] (\id{long int})
    the current number of calls to \id{jtimes}.
  \end{args}
}
{
  The return value \id{flag} (of type \id{int}) is one of:
  \begin{args}[KINSPILS\_LMEM\_NULL]
  \item[\Id{KINSPILS\_SUCCESS}] 
    The optional output value has been successfully set.
  \item[\Id{KINSPILS\_MEM\_NULL}]
    The \id{kin\_mem} pointer is \id{NULL}.
  \item[\Id{KINSPILS\_LMEM\_NULL}]
    The linear solver module has not been initialized.
  \end{args}
}
{}
%%
%%
\ucfunction{KINSpilsGetNumFuncEvals}
{
  flag = KINSpilsGetNumFuncEvals(kin\_mem, \&nfevalsLS);
}
{
  The function \ID{KINSpilsGetNumFuncEvals} returns the
  number of calls to the user system function for
  finite difference Jacobian-vector product approximations.
}
{
  \begin{args}[kin\_mem]
  \item[kin\_mem] (\id{void *})
    pointer to the {\kinsol} memory block.
  \item[nfevalsLS] (\id{long int})
    the number of calls to the user system function.
  \end{args}
}
{
  The return value \id{flag} (of type \id{int}) is one of:
  \begin{args}[KINSPILS\_LMEM\_NULL]
  \item[\Id{KINSPILS\_SUCCESS}] 
    The optional output value has been successfully set.
  \item[\Id{KINSPILS\_MEM\_NULL}]
    The \id{kin\_mem} pointer is \id{NULL}.
  \item[\Id{KINSPILS\_LMEM\_NULL}]
    The linear solver module has not been initialized.
  \end{args}
}
{
  The value \id{nfevalsLS} is incremented only if the default 
  \id{KINSpilsDQJtimes} difference quotient function is used.
}
%%
%%
\ucfunction{KINSpilsGetLastFlag}
{
  flag = KINSpilsGetLastFlag(kin\_mem, \&lsflag);
}
{
  The function \ID{KINSpilsGetLastFlag} returns the
  last return value from a {\kinspils} routine. 
}
{
  \begin{args}[kin\_mem]
  \item[kin\_mem] (\id{void *})
    pointer to the {\kinsol} memory block.
  \item[lsflag] (\id{int})
    the value of the last return flag from a {\kinspils} function.
  \end{args}
}
{
  The return value \id{flag} (of type \id{int}) is one of:
  \begin{args}[KINSPILS\_LMEM\_NULL]
  \item[\Id{KINSPILS\_SUCCESS}] 
    The optional output value has been successfully set.
  \item[\Id{KINSPILS\_MEM\_NULL}]
    The \id{kin\_mem} pointer is \id{NULL}.
  \item[\Id{KINSPILS\_LMEM\_NULL}]
    The linear solver module has not been initialized.
  \end{args}
}
{
  If the {\kinspils} setup function failed (\id{KINSOL} returned
  \id{KIN\_LSETUP\_FAIL}), \id{lsflag} will be
  \id{SPGMR\_PSET\_FAIL\_UNREC}, \id{SPBCG\_PSET\_FAIL\_UNREC}, or
  \id{SPTFQMR\_PSET\_FAIL\_UNREC}.

  If the {\kinspgmr} solve function failed (\id{KINSol} returned
  \id{KIN\_LSOLVE\_FAIL}), \id{lsflag} contains the error return flag from
  \id{SpgmrSolve} and will be one of:
  \id{SPGMR\_MEM\_NULL}, indicating that the {\spgmr} memory is \id{NULL};
  \id{SPGMR\_ATIMES\_FAIL\_UNREC}, indicating an unrecoverable failure in the 
  Jacobian-times-vector function;
  \id{SPGMR\_PSOLVE\_FAIL\_UNREC}, indicating that the preconditioner solve
  function \id{psolve} failed unrecoverably;
  \id{SPGMR\_GS\_FAIL}, indicating a failure in the Gram-Schmidt procedure; 
  or \id{SPGMR\_QRSOL\_FAIL}, indicating that the matrix $R$ was found to be
  singular during the QR solve phase.

  If the {\kinspbcg} solve function failed (\id{KINSol} returned
  \id{KIN\_LSOLVE\_FAIL}), \id{lsflag} contains the error return flag from
  \id{SpbcgSolve} and will be one of:
  \id{SPBCG\_MEM\_NULL}, indicating that the {\spbcg} memory is \id{NULL};
  \id{SPBCG\_ATIMES\_FAIL\_UNREC}, indicating an unrecoverable failure in the 
  Jacobian-times-vector function; or
  \id{SPBCG\_PSOLVE\_FAIL\_UNREC}, indicating that the preconditioner solve
  function \id{psolve} failed unrecoverably.

  If the {\kinsptfqmr} solve function failed (\id{KINSol} returned
  \id{KIN\_LSOLVE\_FAIL}), \id{lsflag} contains the error return flag from
  \id{SptfqmrSolve} and will be one of:
  \id{SPTFQMR\_MEM\_NULL}, indicating that the {\sptfqmr} memory is \id{NULL};
  \id{SPTFQMR\_ATIMES\_FAIL\_UNREC}, indicating an unrecoverable failure in the 
  Jacobian-times-vector function; or
  \id{SPTFQMR\_PSOLVE\_FAIL\_UNREC}, indicating that the preconditioner solve
  function \id{psolve} failed unrecoverably.
}
%%
%%
\index{KINSPTFQMR@{\kinsptfqmr} linear solver!optional output|)} 
\index{KINSPBCG@{\kinspbcg} linear solver!optional output|)} 
\index{KINSPGMR@{\kinspgmr} linear solver!optional output|)} 
\index{optional output!iterative linear solver|)}


%%==============================================================================
\section{User-supplied functions}\label{ss:user_fct_sol}
%%==============================================================================

The user-supplied functions consist of one function defining the nonlinear system,
(optionally) a function that handles error and warning messages,
(optionally) a function that handles informational messages, 
(optionally) a function that provides Jacobian-related information for the linear
solver, and (optionally) one or two functions that define the preconditioner for 
use in any of the Krylov iterative algorithms.

%%==============================================================================
\subsection{Problem-defining function}\label{ss:sysFn}
%%==============================================================================
\index{problem-defining function}

The user must provide a function of type \ID{KINSysFn} defined as follows:
\usfunction{KINSysFn}
{
  typedef void (*KINSysFn)(N\_Vector u, N\_Vector fval, void *f\_data);
}
{
  This function computes $F(u)$ for a given value of the vector $u$.
}
{
  \begin{args}[f\_data]
  \item[u]
    is the current value of the variable vector, $u$.
  \item[fval]
    is the output vector $F(u)$.
  \item[f\_data]
    is a pointer to user data, same as the pointer \Id{f\_data}      
    passed to \id{KINSetFdata}.   
  \end{args}
}
{
  A \id{KINSysFn} function should return $0$ if successful, a positive value
  if a recoverable error occurred (in which case {\kinsol} will attempt to
  correct), or a negative value if it failed unrecoverably (in which case
  the solution process is halted and \id{KIN\_SYSFUNC\_FAIL} is returned).
}
{
  Allocation of memory for \id{fval} is handled within {\kinsol}.
}

%%==============================================================================
\subsection{Error message handler function}
\label{ss:ehFn}
%%==============================================================================
\index{error messages!user-defined handler}
As an alternative to the default behavior of directing error and warning messages 
to the file pointed to by \id{errfp} (see \id{KINSetErrFile}), the user may
provide a function of type \ID{KINErrHandlerFn} to process any such messages.
The function type \id{KINErrHandlerFn} is defined as follows:
\usfunction{KINErrHandlerFn}
{
  typedef void (*KINErrHandlerFn)(&int error\_code,  \\
                                  &const char *module, const char *function, \\ 
                                  &char *msg, void *eh\_data); 
}
{
  This function processes error and warning messages from {\kinsol} and 
  its sub-modules.
}
{
  \begin{args}[error\_code]
  \item[error\_code]
    is the error code.
  \item[module]
    is the name of the {\kinsol} module reporting the error.
  \item[function]
    is the name of the function in which the error occurred.
  \item[msg]
    is the error message.
  \item[eh\_data]
    is a pointer to user data, the same as the \Id{eh\_data}
    parameter passed to \id{KINSetErrHandlerFn}.   
  \end{args}
}
{
  A \id{KINErrHandlerFn} function has no return value.
}
{
  \id{error\_code} is negative for errors and positive (\Id{KIN\_WARNING}) for warnings.
  If a function returning a pointer to memory ({\em e.g.}, \id{KINBBDPrecAlloc}) encounters an
  error, it sets \id{error\_code} to 0 before returning \id{NULL}.
}

%%==============================================================================
\subsection{Informational message handler function}
\label{ss:ihFn}
%%==============================================================================
\index{informational messages!user-defined handler}
As an alternative to the default behavior of directing informational (meaning non-error) messages 
to the file pointed to by \id{infofp} (see \id{KINSetInfoFile}), the user may
provide a function of type \ID{KINInfoHandlerFn} to process any such messages.
The function type \id{KINInfoHandlerFn} is defined as follows:
\usfunction{KINInfoHandlerFn}
{
  typedef void (*KINInfoHandlerFn)(&const char *module, const char *function, \\ 
                                   &char *msg, void *ih\_data); 
}
{
  This function processes informational messages from {\kinsol} and 
  its sub-modules.
}
{
  \begin{args}[function]
  \item[module]
    is the name of the {\kinsol} module reporting the information.
  \item[function]
    is the name of the function reporting the information.
  \item[msg]
    is the message.
  \item[ih\_data]
    is a pointer to user data, the same as the \Id{ih\_data}
    parameter passed to \id{KINSetInfoHandlerFn}.   
  \end{args}
}
{
  A \id{KINInfoHandlerFn} function has no return value.
}
{}

%%==============================================================================
\subsection{Jacobian information (direct method with dense Jacobian)}
\label{ss:djacFn}
%%==============================================================================
\index{Jacobian approximation function!dense!user-supplied|(}

If the direct linear solver with dense treatment of the Jacobian is used 
(i.e. \Id{KINDense} is called in Step \ref{i:lin_solver} of \S\ref{s:skeleton_sol}), 
the user may provide a function of type \ID{KINDenseJacFn} defined by
\usfunction{KINDenseJacFn}
{
  typedef int (*KINDenseJacFn)(&long int N, DenseMat J, \\
                               &N\_Vector u, N\_Vector fu, void *jac\_data, \\
                               &N\_Vector tmp1, N\_Vector tmp2);
}
{
  This function computes the dense Jacobian $J(u)$ 
  (or an approximation to it).
}
{
  \begin{args}[jac\_data]
  \item[N]
    is the problem size.
  \item[J]
    is the output Jacobian matrix.  
  \item[u]
    is the current (unscaled) iterate.
  \item[fu]
    is the current value of the vector $F(u)$.
  \item[jac\_data]
    is a pointer to user data --- the same as the \id{jac\_data}      
    parameter passed to \id{KINDenseSetJacData}.   
  \item[tmp1]
  \item[tmp2]
    are pointers to memory allocated    
    for variables of type \id{N\_Vector} which can be used by           
    \id{KINDenseJacFn} as temporary storage or work space.    
  \end{args}
}
{
  A function of type \id{KINDenseJacFn} should return $0$ if successful
  or a non-zero value otherwise.
}
{
  A user-supplied dense Jacobian function must load the \id{N} by \id{N}
  dense matrix \id{J} with an approximation to the Jacobian matrix $J$ at \id{u}.
  Only nonzero elements need to be loaded
  into \id{J} because \id{J} is set to the zero matrix before the call
  to the Jacobian function. The type of \id{J} is \Id{DenseMat}. 
  
  The accessor macros \Id{DENSE\_ELEM} and \Id{DENSE\_COL} allow the user to
  read and write dense matrix elements without making explicit
  references to the underlying representation of the \id{DenseMat}
  type. \id{DENSE\_ELEM(J, i, j)} references the (\id{i}, \id{j})-th
  element of the dense matrix \id{J} (\id{i}, \id{j}$= 0\ldots N-1$). This macro
  is for use in small problems in which efficiency of access is not a major
  concern.  Thus, in terms of indices $m$ and $n$ running from $1$ to
  $N$, the Jacobian element $J_{m,n}$ can be loaded with the statement
  \id{DENSE\_ELEM(J, m-1, n-1) =} $J_{m,n}$.  Alternatively,
  \id{DENSE\_COL(J, j)} returns a pointer to the storage for
  the \id{j}th column of \id{J} (\id{j}$= 0\ldots N-1$), and the 
  elements of the \id{j}th column
  are then accessed via ordinary array indexing.  Thus $J_{m,n}$ can be 
  loaded with the statements \id{col\_n = DENSE\_COL(J, n-1);}
  \id{col\_n[m-1] =} $J_{m,n}$.  For large problems, it is more 
  efficient to use \id{DENSE\_COL} than to use \id{DENSE\_ELEM}. 
  Note that both of these macros number rows and columns
  starting from $0$, not $1$.  

  The \id{DenseMat} type and the accessor macros \id{DENSE\_ELEM} and 
  \id{DENSE\_COL} are documented in \S\ref{ss:dense}.

  If the user's \id{KINDenseJacFn} function uses difference quotient
  approximations, it may need to access quantities not in the call
  list. To obtain these, use the \id{KINGet*} functions described in
  \S\ref{sss:output_main}. The unit roundoff can be accessed
  as \id{UNIT\_ROUNDOFF} defined in \id{sundials\_types.h}.
}
\index{Jacobian approximation function!dense!user-supplied|)}

%%==============================================================================
\subsection{Jacobian information (direct method with banded Jacobian)}
\label{ss:bjacFn}
%%==============================================================================
\index{Jacobian approximation function!band!user-supplied|(}
\index{half-bandwidths|(}

If the direct linear solver with banded treatment of the Jacobian is used 
({\em i.e.}, \Id{KINBand} is called in Step \ref{i:lin_solver} of \S\ref{s:skeleton_sol}), 
the user may provide a function of type \ID{KINBandJacFn} defined as follows:
\usfunction{KINBandJacFn}
{
 typedef int (*KINBandJacFn)(&long int N, long int mupper, \\
                             &long int mlower, BandMat J, \\ 
                             &N\_Vector u, N\_Vector fu, void *jac\_data, \\
                             &N\_Vector tmp1, N\_Vector tmp2);
}
{
  This function computes the banded Jacobian $J(u)$ 
  (or a banded approximation to it).
}
{
  \begin{args}[jac\_data]
  \item[N]
    is the problem size.
  \item[mlower]
  \item[mupper]
    are the lower and upper half-bandwidths of the Jacobian.
  \item[J]
    is the output Jacobian matrix.
  \item[u]
    is the current (unscaled) iterate.
  \item[fu]
    is the current value of the vector $F(u)$.
  \item[jac\_data]
    is a pointer to user data --- the same as the \id{jac\_data}      
    parameter passed to \id{KINBandSetJacData}.   
  \item[tmp1]
  \item[tmp2]
    are pointers to memory allocated    
    for variables of type \id{N\_Vector} which can be used by           
    \id{KINBandJacFn} as temporary storage or work space.    
  \end{args}
}
{
  A function of type \id{KINBandJacFn} should return $0$ if successful
  or a non-zero value otherwise.
}
{
  A user-supplied band Jacobian function must load the band matrix \id{J}
  of type \Id{BandMat} with the elements of the Jacobian $J$ at \id{u}.
  Only nonzero elements need to be loaded into
  \id{J} because \id{J} is preset to zero before the call to the
  Jacobian function.  

  The accessor macros \Id{BAND\_ELEM}, \Id{BAND\_COL}, and \Id{BAND\_COL\_ELEM} 
  allow the user to read and write band matrix elements without making specific 
  references to the underlying representation of the \id{BandMat} type.
  \id{BAND\_ELEM(J, i, j)} references the (\id{i}, \id{j})th element of the 
  band matrix \id{J}, counting from $0$.
  This macro is for use in small problems in which efficiency of access is not
  a major concern.  Thus, in terms of indices $m$ and $n$ running from $1$ to
  $N$ with $(m,n)$ within the band defined by \id{mupper} and
  \id{mlower}, the Jacobian element $J_{m,n}$ can be loaded with the 
  statement \id{BAND\_ELEM(J, m-1, n-1) =} $J_{m,n}$. The elements within
  the band are those with \id{-mupper} $\le$ \id{m-n} $\le$ \id{mlower}.
  Alternatively, \id{BAND\_COL(J, j)} returns a pointer to the diagonal element
  of the \id{j}th column of \id{J}, and if we assign this address to 
  \id{realtype *col\_j}, then the \id{i}th element of the \id{j}th column is
  given by \id{BAND\_COL\_ELEM(col\_j, i, j)}, counting from $0$.
  Thus for $(m,n)$ within the band, $J_{m,n}$ can be loaded by setting 
  \id{col\_n = BAND\_COL(J, n-1);} \id{BAND\_COL\_ELEM(col\_n, m-1, n-1) =}
  $J_{m,n}$.  The elements of the \id{j}th column can also be accessed
  via ordinary array indexing, but this approach requires knowledge of
  the underlying storage for a band matrix of type \id{BandMat}.  
  The array \id{col\_n} can be indexed from $-$\id{mupper} to \id{mlower}.
  For large problems, it is more efficient to use the combination of
  \id{BAND\_COL} and \id{BAND\_COL\_ELEM} than to use the
  \id{BAND\_ELEM}.  As in the dense case, these macros all number rows
  and columns starting from $0$, not $1$.  

  The \id{BandMat} type and the accessor macros \id{BAND\_ELEM}, \id{BAND\_COL},
  and \id{BAND\_COL\_ELEM} are documented in \S\ref{ss:band}.

  If the user's \id{KINBandJacFn} function uses difference quotient approximations,
  it may need to access quantities not in the call list. To obtain these, use the
  \id{KINGet*} functions described in \S\ref{sss:output_main}. The unit roundoff
  can be accessed as \id{UNIT\_ROUNDOFF} defined in \id{sundials\_types.h}.
}
\index{half-bandwidths|)}
\index{Jacobian approximation function!band!user-supplied|)}

%%==============================================================================
\subsection{Jacobian information (matrix-vector product)}
\label{ss:jtimesFn}
%%==============================================================================
\index{Jacobian approximation function!Jacobian times vector!user-supplied|(}

If one of the Krylov iterative linear solvers {\spgmr}, {\spbcg}, or {\sptfqmr}
is selected (\id{KINSp*} is called in step  \ref{i:lin_solver} of
\S\ref{s:skeleton_sol}), the user may provide a function of type
\ID{KINSpilsJacTimesVecFn} in the following form:
\usfunction{KINSpilsJacTimesVecFn}
{
  typedef int (*KINSpilsJacTimesVecFn)(&N\_Vector v, N\_Vector Jv, \\
                                       &N\_Vector u, booleantype new\_u, \\
                                       &void *jac\_data);
}
{
  This function computes the product $J v$ 
  (or an approximation to it).
}
{
  \begin{args}[jac\_data]
  \item[v]
    is the vector by which the Jacobian must be multiplied to the right.
  \item[Jv]
      is the output vector computed.
  \item[u]
    is the current value of the dependent variable vector.
  \item[new\_u]
    is a flag (reset by user) indicating if the iterate has been
    updated in the interim --- Jacobian needs to be updated/reevaluated,
    if appropriate, unless \id{new\_u = FALSE}.
  \item[jac\_data]
    is a pointer to user data --- the same as the \id{jac\_data}      
    parameter passed to \id{KINSp*SetJacTimesVecFn}.
  \end{args}
}
{  
  The value to be returned by the Jacobian-times-vector function should be
  $0$ if successful. If a recoverable failure occured, the return value should be positive.
  In this case, {\kinsol} will attempt to correct by calling the preconditioner setup function
  if the preconditioner information. If this information is current, {\kinsol} halts.
  If the  Jacobian-times-vector function encounters an unrecoverable error, it should
  return a negative value, prompting {\kinsol} to halt.
}
{
  If a user-defined routine is not given, then an internal
  {\kinspgmr} function, using difference quotient approximations,
  is used.

  If the user-provided \id{KINSpilsJacTimesVecFn} function needs the unit roundoff,
  this can be accessed as \id{UNIT\_ROUNDOFF} defined in \id{sundialstypes.h}.

  {\warn}The user is responsible for resetting the value of \id{new\_u} to \id{FALSE}.
}
\index{Jacobian approximation function!Jacobian times vector!user-supplied|)}

%%==============================================================================
\subsection{Preconditioning (linear system solution)} \label{ss:psolveFn}
%%==============================================================================
\index{preconditioning!user-supplied}
\index{KINSPGMR@{\kinspgmr} linear solver!preconditioner solve function}
\index{KINSPBCG@{\kinspbcg} linear solver!preconditioner solve function}
\index{KINSPTFQMR@{\kinsptfqmr} linear solver!preconditioner solve function}

If preconditioning is used, then the user must provide a {\C} function to
solve the linear system $Pz = r$ where $P$ is the preconditioner matrix.
This function must be of type \ID{KINSpilsPrecSolveFn}, defined as follows:
%%
%%
\usfunction{KINSpilsPrecSolveFn}
{
  typedef int (*KINSpilsPrecSolveFn)(&N\_Vector u, N\_Vector uscale,  \\
                                     &N\_Vector fval, N\_Vector fscale,  \\
                                     &N\_Vector v, void *prec\_data, \\
                                     &N\_Vector tmp);
}
{
  This function solves the preconditioning system $Pz = r$.
}
{  
  \begin{args}[prec\_data]
  \item[u] 
    is the current (unscaled) value of the iterate.
  \item[uscale]
    is a vector containing diagonal elements
    of the scaling matrix for \id{u}.
  \item[fval]
    is the vector $F(u)$ evaluated at \id{u}.
  \item[fscale]
    is a vector containing diagonal elements
    of the scaling matrix for \id{fval}.
  \item[v]
    on input, \id{v} is set to the right-hand side vector of the linear 
    system, \id{r}. On output, \id{v} must contain the solution \id{z} of
    the linear system $Pz=r$.
  \item[prec\_data]
    is a pointer to user data - the same as the \id{prec\_data}      
    parameter passed to the function \id{KINSpilsSetPreconditioner}.
  \item[tmp]
    is a pointer to memory allocated for a variable of type \id{N\_Vector}
    which can be used for work space.
  \end{args}
}
{
  The value to be returned by the preconditioner solve function is a flag
  indicating whether it was successful.  This value should be $0$ if successful, 
  positive for a recoverable error, and negative for an unrecoverable error.
}
{
  If the preconditioner solve function fails recoverably and if the preconditioner
  information (set by the preconditioner setup function) is out of date, {\kinsol}
  attempts to correct by calling the setup function. If the preconditioner data
  is current, {\kinsol} halts.
}



%%==============================================================================
\subsection{Preconditioning (Jacobian data)}\label{ss:precondFn}
%%==============================================================================
\index{preconditioning!user-supplied}
\index{KINSPGMR@{\kinspgmr} linear solver!preconditioner setup function}
\index{KINSPBCG@{\kinspbcg} linear solver!preconditioner setup function}
\index{KINSPTFQMR@{\kinsptfqmr} linear solver!preconditioner setup function}

If the user's preconditioner requires that any Jacobian-related data
be evaluated or preprocessed, then this needs to be done in a
user-supplied {\C} function of type \ID{KINSpilsPrecSetupFn}, defined as follows:
\usfunction{KINSpilsPrecSetupFn}
{
typedef int (*KINSpilsPrecSetupFn)(&N\_Vector u, N\_Vector uscale, \\
                                   &N\_Vector fval, N\_Vector fscale,\\
                                   &void *prec\_data, N\_Vector tmp1,\\
                                   &N\_Vector tmp2);
}
{
  This function evaluates and/or preprocesses Jacobian-related data needed
  by the preconditioner.
}
{
  The arguments of a \id{KINSpilsPrecSetupFn} are as follows:
  \begin{args}[prec\_data]
  \item[u] 
    is the current (unscaled) value of the iterate.
  \item[uscale]
    is a vector containing diagonal elements
    of the scaling matrix for \id{u}.
  \item[fval]
    is the vector $F(u)$ evaluated at \id{u}.
  \item[fscale]
    is a vector containing diagonal elements
    of the scaling matrix for \id{fval}.
  \item[prec\_data]
    is a pointer to user data - the same as the \id{prec\_data}      
    parameter passed to the function \id{KINSpilsSetPreconditioner}.
  \item[tmp1]
  \item[tmp2]
    are pointers to memory allocated for variables of type \id{N\_Vector}
    which can be used by \id{KINSpilsPrecSetupFn} as temporary storage or 
    work space.    
  \end{args}
}
{
  The value to be returned by the preconditioner setup function is a flag
  indicating whether it was successful.  This value should be $0$ if successful, 
  any other value resulting in halting the {\kinsol} solver.
}
{
  The user-supplied preconditioner setup subroutine should
  compute the right preconditioner matrix $P$ (stored in the memory
  block referenced by the \id{prec\_data} pointer) used to form the
  scaled preconditioned linear system
 
   $$(D_F J(u) P^{-1} D_u^{-1}) \cdot (D_u P x) = - D_F F(u) \, ,$$
 
  where $D_u$ and $D_F$ denote the diagonal scaling matrices whose
  diagonal elements are stored in the vectors \id{uscale} and
  \id{fscale}, respectively.
 
  The preconditioner setup routine will not be called prior
  to every call made to the preconditioner solve function, but will instead be
  called only as often as necessary to achieve convergence of the
  Newton iteration.
 
  If the preconditioner solve routine requires no preparation, then a
  preconditioner setup function need not be given.
}

%%==============================================================================
\section{A parallel band-block-diagonal preconditioner module}
\label{sss:kinbbdpre}
%%==============================================================================
The efficiency of Krylov iterative methods for the solution of linear systems 
can be greatly enhanced through preconditioning. For problems in which the 
user cannot define a more effective, problem-specific preconditioner,
{\kinsol} provides a band-block-diagonal preconditioner module {\kinbbdpre}, 
to be used with the parallel \id{N\_Vector} module described in \S\ref{ss:nvec_par}.

This module provides a preconditioner matrix for {\kinsol} that
is block-diagonal with banded blocks. The blocking corresponds
to the distribution of the dependent variable vector $u$ amongst
the processes. Each preconditioner block is generated from
the Jacobian of the local part (associated with the current
process) of a given function $G(u)$ approximating $F(u)$
($G = F$ is allowed). The blocks are generated by each process via a
difference quotient scheme, utilizing a specified band structure.
This structure is given by upper and lower half-bandwidths, \id{mudq}
and \id{mldq}, defined as the number of non-zero diagonals above and
below the main diagonal, respectively.  However, from the resulting
approximate Jacobain blocks, only a matrix of bandwidth \id{mu} $+$
\id{ml} $+ 1$ is retained.

Neither pair of parameters need be the true half-bandwidths of the Jacobian
of the local block of $G$, if smaller values provide a more efficient
preconditioner.  Such an efficiency gain may occur if the couplings
in the system outside a certain bandwidth are considerably weaker than
those within the band.  Reducing \id{mu} and \id{ml} while keeping
\id{mudq} and \id{mldq} at their true values, discards the elements
outside the narrower band.  Reducing both pairs has the additional
effect of lumping the outer Jacobian elements into the computed elements
within the band, and requires more caution and experimentation to see
whether the lower cost of narrower band matrices offsets the loss of
accuracy in the blocks.

\index{KINBBDPRE@{\kinbbdpre} preconditioner!user-supplied functions|(}
The {\kinbbdpre} module calls two user-provided functions to construct $P$: 
a required function \id{Gloc} (of type \id{KINLocalFn}) which
approximates the nonlinear system function function $G(u) \approx F(u)$ and which 
is computed locally, and an optional function \id{Gcomm} (of type \id{KINCommFn}) 
which performs all interprocess communication necessary to evaluate 
the approximate function $G$.  
These are in addition to the user-supplied nonlinear system function that 
evaluates $F(u)$.
Both functions take as input the same pointer \id{f\_data} as that passed
by the user to \id{KINSetFdata} and passed to the user's function \id{func},
and neither function has a return value. The user is responsible for
providing space (presumably within \id{f\_data}) for components of \id{u}
that are communicated by \id{Gcomm} from the other processes, and that are
then used by \id{Gloc}, which is not expected to do any communication.
%%
%%
\usfunction{KINLocalFn}
{
  typedef void (*KINLocalFn)(&long int Nlocal, N\_Vector u, \\
                             &N\_Vector gval, void *f\_data);
}
{
  This function computes $G($\id{u}$)$, and outputs the resulting
  vector as \id{gval}.
}
{
  \begin{args}[Nlocal]
  \item[Nlocal] 
    is the local vector length.
  \item[u]
    is the current value of the iterate.
  \item[gval]
    is the output vector.
  \item[f\_data]
    is a pointer to user data - the same as the \Id{f\_data}      
    parameter passed to \id{KINSetFdata}.  
  \end{args}
}
{
  A \id{KINLocalFn} function type does not have a return value.
}
{
  This function assumes that all interprocess communication of data needed to 
  calculate \id{gval} has already been done, and this data is accessible within
  \id{f\_data}.

   Memory for \id{u} and \id{gval} is handled within the preconditioner module.

  The case where $G$ is mathematically identical to $F$ is allowed.
}
%%
%%
\usfunction{KINCommFn}
{
  typedef void (*KINCommFn)(long int Nlocal, N\_Vector u, void *f\_data);
}
{
  This function performs all interprocess communications necessary 
  for the execution of the \id{gloc} function above, using the input vector \id{u}.
}
{
  \begin{args}[Nlocal]
  \item[Nlocal] 
    is the local vector length.
  \item[u]
    is the current value of the iterate.
  \item[f\_data]
    is a pointer to user data - the same as the \Id{f\_data}      
    parameter passed to \id{KINSetFdata}.  
  \end{args}
}
{
  A \id{KINCommFn} function type does not have a return value.
}
{
  The \id{Gcomm} function is expected to save communicated data in space defined
  within the structure \id{f\_data}. 

  Each call to the \id{Gcomm} function is preceded by a call to the system function
  \id{func} with the same \id{u} argument.  Thus \id{Gcomm} can omit 
  any communications done by \id{func} if relevant to the evaluation of \id{Gloc}.
  If all necessary communication was done in \id{func}, then \id{Gcomm = NULL}
  can be passed in the call to \id{KINBBDPrecAlloc} (see below).
}
%%
\index{KINBBDPRE@{\kinbbdpre} preconditioner!user-supplied functions|)}

%%
%% ----------------------------------------
%%
\index{KINBBDPRE@{\kinbbdpre} preconditioner!usage|(}
%%
Besides the header files required for the solution of a nonlinear problem
(see \S\ref{s:header_sol}),  to use the {\kinbbdpre} module, the main program 
must include the header file \id{kinbbdpre.h} which declares the needed
function prototypes.\index{header files}

The following is a summary of the usage of this module and describes the sequence
of calls in the user main program. Steps that are unchanged from the user main
program presented in \S\ref{s:skeleton_sol} are grayed out.
%%
%%
\index{User main program!KINBBDPRE@{\kinbbdpre} usage}
\begin{Steps}
\item 
  \textcolor{gray}{\bf Initialize MPI}

\item
  \textcolor{gray}{\bf Set problem dimensions}

\item
  \textcolor{gray}{\bf Set vector with initial guess}
 
\item
  \textcolor{gray}{\bf Create {\kinsol} object}

\item
  \textcolor{gray}{\bf Set optional inputs}

\item
  \textcolor{gray}{\bf Allocate internal memory}

\item \label{i:bbdpre_init}
  {\bf Initialize the {\kinbbdpre} preconditioner module}

  Specify the upper and lower half-bandwidth pairs \id{mu},\id{ml} and
  \id{mukeep},\id{mlkeep}, and call

   \id{
     \begin{tabular}[t]{@{}r@{}l@{}}
       bbd\_data = KINBBDPrecAlloc(&kin\_mem, Nlocal, mudq, mldq, \\
                                   &mukeep, mlkeep, dq\_rel\_u, Gloc, Gcomm);
     \end{tabular}
   }

  to allocate memory for and initialize a data structure \id{bbd\_data} to be 
  passed to the appropriate {\kinspils} linear solver. The last two arguments of
  \id{KINBBDPrecAlloc} are the two user-supplied functions described above.

\item \label{i:bbdpre_attach}
  {\bf Attach the Kyrlov linear solver, one of:}

  \id{flag = KINBBDSpgmr(kin\_mem, maxl, bbd\_data);}

  \id{flag = KINBBDSpbcg(kin\_mem, maxl, bbd\_data);}

  \id{flag = KINBBDSptfqmr(kin\_mem, maxl, bbd\_data);}

  Each function \Id{KINBBDSp*} is a wrapper around the corresponding specification
  function \id{KINSp*} and performs the following actions:
  \begin{itemize}
    \item Attaches the {\kinspils} linear solver to the main {\kinsol} solver memory;
    \item Sets the preconditioner data structure for {\kinbbdpre};
    \item Sets the preconditioner setup function for {\kinbbdpre};
    \item Sets the preconditioner solve function for {\kinbbdpre};
  \end{itemize}
  The argument \id{maxl} is described below.
  The last argument of \id{KINBBDSp*} is the pointer to the {\kinbbdpre} data
  returned by \id{KINBBDPrecAlloc}.

\item
  \textcolor{gray}{\bf Set linear solver optional inputs}

  Note that the user should not overwrite the preconditioner data, setup function,
  or solve function through calls to {\kinspils} optional input functions.

\item
  \textcolor{gray}{\bf Solve problem}

\item
  \textcolor{gray}{\bf Get optional output}

\item
  \textcolor{gray}{\bf Deallocate memory for solution vector}

\item \label{i:bbdpre_free}
  {\bf Free the {\kinbbdpre} data structure}

  \id{KINBBDPrecFree(\&bbd\_data);}

\item
  \textcolor{gray}{\bf Free solver memory}
  
\item 
  \textcolor{gray}{\bf Finalize MPI}

\end{Steps}
%%
\index{KINBBDPRE@{\kinbbdpre} preconditioner!usage|)}

%%
%% ----------------------------------------
%%
\index{KINBBDPRE@{\kinbbdpre} preconditioner!user-callable functions|(}
%%
The user-callable functions that initialize, attach, and deallocate
the {\kinbbdpre} preconditioner module (steps \ref{i:bbdpre_init},
\ref{i:bbdpre_attach}, and \ref{i:bbdpre_free} above) are described
in more detail below.
%%
\index{half-bandwidths}
\ucfunction{KINBBDPrecAlloc}
{
   \begin{tabular}[t]{@{}r@{}l@{}}
     bbd\_data = KINBBDPrecAlloc(&kin\_mem, Nlocal, mudq, mldq, \\
                                 &mukeep, mlkeep, dq\_rel\_u, Gloc, Gcomm);
   \end{tabular}
}
{
  The function \ID{KINBBDPrecAlloc} initializes and allocates
  memory for the {\kinbbdpre} preconditioner.
}
{
  \begin{args}[dq\_rel\_u]
  \item[kin\_mem] (\id{void *})
    pointer to the {\kinsol} memory block.
  \item[Nlocal] (\id{long int})
    local vector length.
  \item[mudq] (\id{long int})
    upper half-bandwidth to be used in the difference-quotient Jacobian approximation.
  \item[mldq] (\id{long int})
    lower half-bandwidth to be used in the difference-quotient Jacobian approximation.
  \item[mukeep] (\id{long int})
    upper half-bandwidth of the retained banded approximate Jacobian block.
  \item[mlkeep] (\id{long int})
    lower half-bandwidth of the retained banded approximate Jacobian block.
  \item[dq\_rel\_u] (\id{realtype})
    the relative increment in components of \id{u} used in the difference quotient
    approximations.  The default is \id{dq\_rel\_u}$ = \sqrt{\text{unit roundoff}}$,
    which can be specified by passing \id{dq\_rel\_u}$ = 0.0$.
  \item[Gloc] (\id{KINLocalFn})
    the {\C} function which computes the approximation $G(u) \approx F(u)$. 
  \item[Gcomm] (\id{KINCommFn})
    the optional {\C} function which performs all interprocess communication required for
    the computation of $G(u)$.
  \end{args}
}
{
  If successful, \id{KINBBDPrecAlloc} returns a pointer to the newly created 
  {\kinbbdpre} memory block (of type \id{void *}).
  If an error occurred, \id{KINBBDPrecAlloc} returns \id{NULL}.
}
{
  If one of the half-bandwidths \id{mudq} or \id{mldq} to be used in the
  difference-quotient calculation of the approximate Jacobian is negative
  or exceeds the value \id{Nlocal}$-1$, it is replaced with $0$ or
  \id{Nlocal}$-1$ accordingly.

  The half-bandwidths \id{mudq} and \id{mldq} need
  not be the true half-bandwidths of the Jacobian of the local block of $G$,    
  when smaller values may provide greater efficiency.

  Also, the half-bandwidths \id{mukeep} and \id{mlkeep} of the retained
  banded approximate Jacobian block may be even smaller, to reduce
  storage and computation costs further.

  For all four half-bandwidths, the values need not be the same for
  every process.
}
%%
%%
\ucfunction{KINBBDSpgmr}
{
  flag = KINBBDSpgmr(kin\_mem, maxl, bbd\_data);
}
{
  The function \ID{KINBBDSpgmr} links the {\kinbbdpre} data to the
  {\kinspgmr} linear solver and attaches the latter to the {\kinsol}
  memory block.
}
{
  \begin{args}[kin\_mem]
  \item[kin\_mem] (\id{void *})
    pointer to the {\kinsol} memory block.
  \item[maxl] (\id{int})
    \index{maxl@\texttt{maxl}}
    maximum dimension of the Krylov subspace to be used. Pass $0$ to use the 
    default value \id{KINSPILS\_MAXL}$=5$.
  \item[bbd\_data] (\id{void *})
    pointer to the {\kinbbdpre} data structure.
  \end{args}
}
{
  The return value \id{flag} (of type \id{int}) is one of:
  \begin{args}[KINBBDPRE\_PDATA\_NULL]
  \item[\Id{KINSPILS\_SUCCESS}] 
    The {\kinspgmr} initialization was successful.
  \item[\Id{KINSPILS\_MEM\_NULL}]
    The \id{kin\_mem} pointer is \id{NULL}.
  \item[\Id{KINSPILS\_ILL\_INPUT}]
    The {\nvector} module used does not implement a required operation.
  \item[\Id{KINSPILS\_MEM\_FAIL}]
    A memory allocation request failed.
  \item[\Id{KINBBDPRE\_PDATA\_NULL}]
    The {\kinbbdpre} preconditioner has not been initialized.
  \end{args}
}
{}
%%
%%
\ucfunction{KINBBDSpbcg}
{
  flag = KINBBDSpbcg(kin\_mem, maxl, bbd\_data);
}
{
  The function \ID{KINBBDSpbcg} links the {\kinbbdpre} data to the
  {\kinspbcg} linear solver and attaches the latter to the {\kinsol}
  memory block.
}
{
  \begin{args}[kin\_mem]
  \item[kin\_mem] (\id{void *})
    pointer to the {\kinsol} memory block.
  \item[maxl] (\id{int})
    \index{maxl@\texttt{maxl}}
    maximum dimension of the Krylov subspace to be used. Pass $0$ to use the 
    default value \id{KINSPILS\_MAXL}$=5$.
  \item[bbd\_data] (\id{void *})
    pointer to the {\kinbbdpre} data structure.
  \end{args}
}
{
  The return value \id{flag} (of type \id{int}) is one of:
  \begin{args}[KINBBDPRE\_PDATA\_NULL]
  \item[\Id{KINSPILS\_SUCCESS}] 
    The {\kinspgmr} initialization was successful.
  \item[\Id{KINSPILS\_MEM\_NULL}]
    The \id{kin\_mem} pointer is \id{NULL}.
  \item[\Id{KINSPILS\_ILL\_INPUT}]
    The {\nvector} module used does not implement a required operation.
  \item[\Id{KINSPILS\_MEM\_FAIL}]
    A memory allocation request failed.
  \item[\Id{KINBBDPRE\_PDATA\_NULL}]
    The {\kinbbdpre} preconditioner has not been initialized.
  \end{args}
}
{}
%%
%%
\ucfunction{KINBBDSptfqmr}
{
  flag = KINBBDSptfqmr(kin\_mem, maxl, bbd\_data);
}
{
  The function \ID{KINBBDSptfqmr} links the {\kinbbdpre} data to the
  {\kinsptfqmr} linear solver and attaches the latter to the {\kinsol}
  memory block.
}
{
  \begin{args}[kin\_mem]
  \item[kin\_mem] (\id{void *})
    pointer to the {\kinsol} memory block.
  \item[maxl] (\id{int})
    \index{maxl@\texttt{maxl}}
    maximum dimension of the Krylov subspace to be used. Pass $0$ to use the 
    default value \id{KINSPILS\_MAXL}$=5$.
  \item[bbd\_data] (\id{void *})
    pointer to the {\kinbbdpre} data structure.
  \end{args}
}
{
  The return value \id{flag} (of type \id{int}) is one of:
  \begin{args}[KINBBDPRE\_PDATA\_NULL]
  \item[\Id{KINSPILS\_SUCCESS}] 
    The {\kinspgmr} initialization was successful.
  \item[\Id{KINSPILS\_MEM\_NULL}]
    The \id{kin\_mem} pointer is \id{NULL}.
  \item[\Id{KINSPILS\_ILL\_INPUT}]
    The {\nvector} module used does not implement a required operation.
  \item[\Id{KINSPILS\_MEM\_FAIL}]
    A memory allocation request failed.
  \item[\Id{KINBBDPRE\_PDATA\_NULL}]
    The {\kinbbdpre} preconditioner has not been initialized.
  \end{args}
}
{}
%%
\ucfunction{KINBBDPrecFree}
{
  KINBBDPrecFree(\&bbd\_data);
}
{
  The function \ID{KINBBDPrecFree} frees the pointer allocated by
  \id{KINBBDPrecAlloc}.
}
{
  The only argument of \id{KINBBDPrecFree} is the pointer to the {\kinbbdpre} 
  data structure (of type \id{void *}).
}
{
  The function \id{KINBBDPrecFree} has no return value.
}
{}
%%
\index{KINBBDPRE@{\kinbbdpre} preconditioner!user-callable functions|)}
%%
\index{optional output!band-block-diagonal preconditioner|(}
\index{KINBBDPRE@{\kinbbdpre} preconditioner!optional output|(}
The following two optional output functions are available for use with
the {\kinbbdpre} module:
%%
\ucfunction{KINBBDPrecGetWorkSpace}
{
  flag = KINBBDPrecGetWorkSpace(bbd\_data, \&lenrwBBDP, \&leniwBBDP);
}
{
  The function \ID{KINBBDPrecGetWorkSpace} returns the local
  {\kinbbdpre} real and integer workspace sizes.
}
{
  \begin{args}[lenrwBBDP]
  \item[bbd\_data] (\id{void *})
    pointer to the {\kinbbdpre} data structure.
  \item[lenrwBBDP] (\id{long int})
    local number of \id{realtype} values in the {\kinbbdpre} workspace.
  \item[leniwBBDP] (\id{long int})
    local number of integer values in the {\kinbbdpre} workspace.
  \end{args}
}
{
  The return value \id{flag} (of type \id{int}) is one of:
  \begin{args}[KIN\_PDATA\_NULL]
  \item[\Id{KINBBDPRE\_SUCCESS}] 
    The optional output values have been successfully set.
  \item[\Id{KINBBDPRE\_PDATA\_NULL}]
    The {\kinbbdpre} preconditioner has not been initialized.
  \end{args}
}
{
  \index{memory requirements!KINBBDPRE@{\kinbbdpre} preconditioner}
  In terms of the local vector dimension \id{Nlocal} and
  \id{smu} = $\min ( N_l - 1 ,$ \id{mukeep} $+$ \id{mlkeep}$)$, the actual size
  of the real workspace is $(2$ \id{mlkeep} $+$ \id{mukeep} $+$ \id{smu} $+2)$ \id{Nlocal}
  \id{realtype} words, and the actual size of the integer workspace is
  \id{Nlocal} integer words. These values are local to the current processor.

  The workspaces referred to here exist in addition to those given by the
  corresponding \id{KINSp*GetWorkSpace} function.
}
%%
%%
\ucfunction{KINBBDPrecGetNumGfnEvals}
{
  flag = KINBBDPrecGetNumGfnEvals(bbd\_data, \&ngevalsBBDP);
}
{
  The function \ID{KINBBDPrecGetNumGfnEvals} returns the
  number of calls to the user \id{Gloc} function due to the 
  finite difference approximation of the Jacobian blocks used within
  {\kinbbdpre}'s preconditioner setup function.
}
{
  \begin{args}[ngevalsBBDP]
  \item[bbd\_data] (\id{void *})
    pointer to the {\kinbbdpre} data structure.
  \item[ngevalsBBDP] (\id{long int})
    the number of calls to the user \id{Gloc} function.
  \end{args}
}
{
  The return value \id{flag} (of type \id{int}) is one of:
  \begin{args}[KIN\_PDATA\_NULL]
  \item[\Id{KINBBDPRE\_SUCCESS}] 
    The optional output value has been successfully set.
  \item[\Id{KINBBDPRE\_PDATA\_NULL}]
    The {\kinbbdpre} preconditioner has not been initialized.
  \end{args}
}
{}
%%
%%
\ucfunction{KINBBDPrecGetReturnFlagName}
{
  name = KINBBDPrecGetReturnFlagName(flag);
}
{
  The function \ID{KINBBDPrecGetReturnFlagName} returns the
  name of the {\kinbbdpre} constant corresponding to \id{flag}.
}
{
  The only argument, of type \id{int} is a return flag from a {\kinbbdpre} function.
}
{
  The return value is a string containing the name of the corresponding constant.
}
{}
\index{KINBBDPRE@{\kinbbdpre} preconditioner!optional output|)}
\index{optional output!band-block-diagonal preconditioner|)}

In addition to the \id{ngevalsBBDP} \id{Gloc} evaluations,
the costs associated with {\kinbbdpre} also include \id{nlinsetups} LU
factorizations, \id{nlinsetups} calls to \id{Gcomm}, \id{npsolves} banded
backsolve calls, and \id{nfevalsLS} right-hand side function evaluations,
where \id{nlinsetups} is an optional {\kinsol} output and \id{npsolves} and 
\id{nfevalsLS} are linear solver optional outputs (see \S\ref{ss:optional_output}).

%%
%%
%%
%%
%%

%%==============================================================================
\chapter{FKINSOL, an Interface Module for FORTRAN Applications}\label{s:fcmix}
%%==============================================================================

The {\fkinsol} interface module is a package of {\C} functions which support
the use of the {\kinsol} solver, for the solution nonlinear systems
$F(u)=0$, in a mixed {\F}/{\C} setting.  While {\kinsol} is written
in {\C}, it is assumed here that the user's calling program and
user-supplied problem-defining routines are written in {\F}.
This package provides the necessary interface to {\kinsol} for both the
serial and the parallel {\nvector} implementations.

%%==============================================================================
\section{Important note on portability}
%%==============================================================================
\index{portability!Fortran}

In this package, the names of the interface functions, and the names of
the {\F} user routines called by them, appear as dummy names
which are mapped to actual values by a series of definitions in the
header files \id{fkinsol.h}and \id{fkinbbd.h}.
By default, those mapping definitions depend in turn on the {\C} macro
\id{F77\_FUNC} defined in the header file \id{sundials\_config.h} by \id{configure}. However,
the set of flags \Id{SUNDIALS\_CASE\_UPPER}, \Id{SUNDIALS\_CASE\_LOWER},
\Id{SUNDIALS\_UNDERSCORE\_NONE}, \Id{SUNDIALS\_UNDERSCORE\_ONE}, and
\Id{SUNDIALS\_UNDERSCORE\_TWO} can be explicitly defined in the header file
\id{sundials\_config.h} when
configuring {\sundials} via the options \id{--with-f77underscore} and
\id{--with-f77case} to override the default behavior if necessary
(see Chapter \ref{s:install}). Either way, the names into which the dummy names
are mapped are in upper or lower case and have up to two underscores appended.

The user must also ensure that variables in the user {\F} code are
declared in a manner consistent with their counterparts in {\kinsol}.
All real variables must be declared as \id{REAL}, \id{DOUBLE PRECISION},
or perhaps as \id{REAL*}{\em n}, where {\em n} denotes the number of bytes,
depending on whether {\kinsol} was built in single, double or extended precision 
(see Chapter \ref{s:install}). Moreover, some of the {\F} integer variables
must be declared as \id{INTEGER*4} or \id{INTEGER*8} according to the 
{\C} type \id{long int}. These integer variables include: the array
of integer optional inputs and outputs (\id{IOUT}), problem dimensions (\id{NEQ},
\id{NLOCAL}, \id{NGLOBAL}), and Jacobian half-bandwidths (\id{MU} and \id{ML}).
This is particularly important when using
{\kinsol} and the {\fkinsol} package on 64-bit architectures.

%%==============================================================================
\section{FKINSOL routines}\label{sss:fkinroutines}
%%==============================================================================

\index{FKINSOL@{\fkinsol} interface module!user-callable functions|(}
The user-callable functions, with the corresponding {\kinsol} functions,
are as follows:
\begin{itemize}
\item
  Interface to the {\nvector} modules
  \begin{itemize}
  \item \id{FNVINITS} (defined by {\nvecs}) 
    interfaces to \id{N\_VNewEmpty\_Serial}.
  \item \id{FNVINITP} (defined by {\nvecp}) 
    interfaces to \id{N\_VNewEmpty\_Parallel}.
  \end{itemize}
\item Interface to the main {\kinsol} module
  \begin{itemize}
  \item \id{FKINMALLOC}
    interfaces to \id{KINCreate}, \id{KINSetFdata}, and \id{KINMalloc}.
  \item \id{FKINSETIIN} and \id{FKINSETRIN}
    interface to \id{KINSet*} functions.
  \item \id{FKINSETVIN}
    interfaces to \id{KINSetConstraints}.
  \item \id{FKINSOL}
    interfaces to \id{KINSol}, \id{KINGet*} functions, and to the optional
    output functions for the selected linear solver module.
  \item \id{FKINFREE}    
    interfaces to \id{KINFree}.
  \end{itemize}  
\item Interface to the linear solver modules
  \begin{itemize}
  \item \id{FKINDENSE}
    interfaces to \id{KINDense}.
  \item \id{FKINDENSESETJAC}
    interfaces to \id{KINDenseSetJacFn}.
  \item \id{FKINBAND}
    interfaces to \id{KINBand}.
  \item \id{FKINBANDSETJAC}
    interfaces to \id{KINBandSetJacFn}.
  \item \id{FKINSPGMR}
    interfaces to \id{KINSpgmr} and {\spgmr} optional input functions.
  \item \id{FKINSPBCG}
    interfaces to \id{KINSpbcg} and {\spbcg} optional input functions.
  \item \id{FKINSPTFQMR}
    interfaces to \id{KINSptfqmr} and {\sptfqmr} optional input functions.
  \item \id{FKINSPILSSETJAC}
    interfaces to \id{KINSpilsSetJacTimesVecFn}.
  \item \id{FKINSPILSSETPREC}
    interfaces to \id{KINSpilsSetPreconditioner}.
 \end{itemize}

\end{itemize}
\index{FKINSOL@{\fkinsol} interface module!user-callable functions|)}

\index{FKINSOL@{\fkinsol} interface module!user-supplied functions}
The user-supplied functions, each listed with the corresponding interface
function which calls it (and its type within {\kinsol}), are as follows:
\begin{center}
\begin{tabular}{l|l|l}
{\fkinsol} routine ({\F})  &  {\kinsol} function ({\C}) & {\kinsol} function type \\\hline
\id{FKFUN}    & \id{FKINfunc}     & \id{KINSysFn} \\
\id{FKDJAC}   & \id{FKINDenseJac} & \id{KINDenseJacFn} \\
\id{FKBJAC}   & \id{FKINBandJac}  & \id{KINBandJacFn} \\
\id{FKPSET}   & \id{FKINPSet}     & \id{KINSpilsPrecSetupFn} \\
\id{FKPSOL}   & \id{FKINPSol}     & \id{KINSpilsPrecSolveFn} \\
\id{FKJTIMES} & \id{FKINJtimes}   & \id{KINSpilsJacTimesVecFn} \\
\end{tabular}
\end{center}
In contrast to the case of direct use of {\kinsol}, the names of all 
user-supplied routines here are fixed, in order to maximize portability 
for the resulting mixed-language program.

%%==============================================================================
\section{Usage of the FKINSOL interface module}\label{ss:fkinsol_usage}
%%==============================================================================
\index{FKINSOL@{\fkinsol} interface module!usage|(}

The usage of {\fkinsol} requires calls to a few different interface
functions, depending on the method options selected, and one or more
user-supplied routines which define the problem to be solved.  These
function calls and user routines are summarized separately below.
Some details are omitted, and the user is referred to the description
of the corresponding {\kinsol} functions for information on the arguments 
of any given user-callable interface routine, or of a given user-supplied 
function called by an interface function.

Steps marked with {\s} in the instructions below apply to the serial
{\nvector} implementation ({\nvecs}) only, while those marked with {\p}
apply to {\nvecp}.

\index{User main program!FKINSOL@{\fkinsol} usage}
\begin{Steps}
 
%%====================
\item {\bf Nonlinear system function specification}
%%====================
  
  The user must in all cases supply the following {\F} routine
  \index{FKFUN@\texttt{FKFUN}}
\begin{verbatim}
      SUBROUTINE FKFUN (U, FVAL, IER)
      DIMENSION U(*), FVAL(*)
\end{verbatim}
  It must set the \id{FVAL} array to $F(u)$, the system function,
  as a function of \id{U}$=u$.  
  \id{IER} is an error return flag which should be set to $0$ if successful,
  a positive value if a recoverable error occurred (in which case {\kinsol}
  will attempt to corret), or a negative value if it failed unrecoverably
  (in which case the the solution process is halted).

%%====================
\item  {\bf {\nvector} module initialization}
%%====================

  {\s} To initialize the serial {\nvector} module, the user must make the
  following call:
  \index{FNVINITS@\texttt{FNVINITS}}
\begin{verbatim}
      CALL FNVINITS (KEY, NEQ, IER)
\end{verbatim}
  where 
  \id{KEY} is the solver id (\id{KEY} = 3 for {\kinsol}),
  \id{NEQ} is the size of vectors, and
  \id{IER} is a  return completion flag which is $0$ on success and $-1$ 
  if a failure occurred.
  
  {\p} To initialize the parallel vector module, the user must make the
  following call:
  \index{FNVINITP@\texttt{FNVINITP}}
\begin{verbatim}
      CALL FNVINITP (COMM, KEY, NLOCAL, NGLOBAL, IER)
\end{verbatim}
  in which the arguments are: \id{COMM} = {\mpi} communicator, \id{KEY} = 3,
  \id{NLOCAL} = the local size of
  vectors on this processor, and \id{NGLOBAL} = the system size (and the global
  size of all vectors, equal to the sum of all values of \id{NLOCAL}).
  The return flag \id{IER} is set to $0$ on a successful return and to $-1$
  otherwise.

  {\warn}If the header file \id{sundials\_config.h} defines
  \id{SUNDIALS\_MPI\_COMM\_F2C} to be $1$ (meaning the {\mpi}
  implementation used to build {\sundials} includes the
  \id{MPI\_Comm\_f2c} function), then \id{COMM} can be any valid
  {\mpi} communicator. Otherwise, \id{MPI\_COMM\_WORLD} will be used, so
  just pass an integer value as a placeholder.

%%====================
\item {\bf Problem specification}
%%====================

  To set various problem and solution parameters and allocate
  internal memory, make the following call:
  \index{FKINMALLOC@\texttt{FKINMALLOC}}
  \ucfunction{FKINMALLOC}
  {
    \begin{tabular}[t]{@{}r@{}l@{}l@{}}
        &CALL FKINMALLOC (IOUT, ROUT, IER)
    \end{tabular}
  }
  {
    This function specifies the optional output arrays,
    allocates internal memory, and initializes {\kinsol}.
  }
  {
    \begin{args}[IOUT\,]
    \item[IOUT] is an integer array for integer optional outputs.
    \item[ROUT] is a real array for real optional outputs.
    \end{args}
  }
  {
    \id{IER} is the return completion flag. Values are $0$ for successful return
    and $-1$ otherwise. See printed message for details in case of failure.
  }
  {
    The user integer data array \id{IOUT} must be declared as \id{INTEGER*4} or
    \id{INTEGER*8} according to the {\C} type \id{long int}.

    The optional outputs associated with the main {\kinsol} integrator
    are listed in Table~\ref{t:fkinsol_out}.
  }

%%====================
\item\label{i:fkinsol_lin_solv_spec}{\bf Linear solver specification} 
%%====================

  The solution method in {\kinsol} involves the solution of linear systems 
  related to the Jacobian of the nonlinear system. {\kinsol} presently
  includes five choices for the treatment of these systems, and the user of
  {\fkinsol} must call a routine with a specific name to make the desired
  choice.

  %%-------------------------------------------------

  {\s} {\bf Dense treatment of the linear system}
  \index{KINDENSE@{\kindense} linear solver!use in {\fkinsol}}
  
  The user must make the call:
  \index{FKINDENSE@\texttt{FKINDENSE}}
\begin{verbatim}
      CALL FKINDENSE (NEQ, IER)
\end{verbatim}
  where \id{NEQ} is the size of the nonlinear system.
  The argument \id{IER} is an error return flag which is $0$ 
  for success , $-1$ if a memory allocation failure occurred, or $-2$ for illegal
  input.  \index{Jacobian approximation function!dense!use in {\fkinsol}}
  As an option when using the {\dense} linear solver, the user may supply a
  routine that computes a dense approximation of the system Jacobian 
  $J = \partial F / \partial u$. If supplied, it must have the following form:
  \index{FKDJAC@\texttt{FKDJAC}}
\begin{verbatim}
      SUBROUTINE FKDJAC (NEQ, U, FVAL, DJAC,
     &                   WK1, WK2, IER)
      DIMENSION U(*), FVAL(*), DJAC(NEQ,*),
     &          WK1(*), WK2(*)
\end{verbatim}
  Typically this routine will use only \id{NEQ}, \id{U}, and \id{DJAC}. 
  It must compute the Jacobian and store it columnwise in \id{DJAC}.
  The input arguments \id{U} and \id{FVAL} contain the current
  values of $u$, and $F(u)$, respectively.
  The vectors \id{WK1} and \id{WK2}
  of length \id{NEQ} are provided as work space for use in \id{FKDJAC}.
  \id{IER} is an error return flag which should be set to $0$ if successful,
  a positive value if a recoverable error occurred (in which case {\kinsol}
  will attempt to correct), or a negative value if \id{FKDJAC} failed
  unrecoverably (in which case the solution process is halted).

  If the \id{FKDJAC} routine is provided, then, 
  following the call to \id{FKINDENSE}, the user must make the call:
  \index{FKINDENSESETJAC@\texttt{FKINDENSESETJAC}}
\begin{verbatim}
      CALL FKINDENSESETJAC (FLAG, IER)
\end{verbatim}
  with \id{FLAG} $\neq 0$ to specify use of the user-supplied Jacobian
  approximation.  The argument \id{IER} is an error return flag which is $0$ 
  for success or non-zero if an error occurred.
  
  Optional outputs specific to the {\dense} case are listed in
  Table~\ref{t:fkinsol_out}.

  %%-------------------------------------------------

  {\s} {\bf Band treatment of the linear system}
  \index{KINBAND@{\kinband} linear solver!use in {\fkinsol}}
  
  The user must make the call:
  \index{FKINBAND@\texttt{FKINBAND}}
\begin{verbatim}
      CALL FKINBAND (NEQ, MU, ML, IER)
\end{verbatim}
  The arguments are: \id{MU}, the upper half-bandwidth; \id{ML}, 
  the lower half-bandwidth; and \id{IER} an error return flag which is 
  $0$ for success , $-1$ if a memory allocation failure occurred, or $-2$ 
  in case an input has an illegal value.     
  
  \index{Jacobian approximation function!band!use in {\fkinsol}}
  As an option when using the {\band} linear solver, the user may supply a
  routine that computes a band approximation of the system Jacobian 
  $J = \partial F / \partial u$. If supplied, it must have the following form:
  \index{FKBJAC@\texttt{FKBJAC}}
\begin{verbatim}
      SUBROUTINE FKBJAC (NEQ, MU, ML, MDIM, U, FVAL, BJAC,
     &                   WK1, WK2, IER)
      DIMENSION U(*), FVAL(*), BJAC(MDIM,*),
     &          WK1(*), WK2(*)
\end{verbatim}
  Typically this routine will use only \id{NEQ}, \id{MU}, \id{ML},
  \id{U}, and \id{BJAC}.
  It must load the \id{MDIM} by \id{N} array \id{BJAC} with the Jacobian matrix
  at the current $u$ in band form.  Store in \id{BJAC}$(k,j)$ the Jacobian
  element $J_{i,j}$ with $k = i - j + $ \id{MU} $ + 1$ ($k = 1 \cdots $
  \id{ML + MU + 1}) and $j = 1 \cdots N$.
  The input arguments \id{U}, and \id{FVAL} contain the current
  values of $u$, and $F(u)$, respectively.
  The vectors \id{WK1} and \id{WK2}
  of length \id{NEQ} are provided as work space for use in
  \id{FKBJAC}.
  \id{IER} is an error return flag which should be set to $0$ if successful,
  a positive value if a recoverable error occurred (in which case {\kinsol} will
  attempt to correct), or a negative value if \id{FKBJAC} failed unrecoverably
  (in which case the solution process is halted).

  If the \id{FKBJAC} routine is provided, then, following the call to \id{FKINBAND},
  the user must make the call:
  \index{FKINBANDSETJAC@\texttt{FKINBANDSETJAC}}
\begin{verbatim}
      CALL FKINBANDSETJAC (FLAG, IER)
\end{verbatim}
  with \id{FLAG} $\neq 0$ to specify use of the user-supplied Jacobian approximation.
  The argument \id{IER} is an error return flag which is $0$ 
  for success or non-zero if an error occurred.
  
  Optional outputs specific to the {\band} case are listed in Table~\ref{t:fkinsol_out}.

  %%-------------------------------------------------
  
  {\s}{\p} {\bf SPGMR treatment of the linear systems}
  \index{KINSPGMR@{\kinspgmr} linear solver!use in {\fkinsol}}

  For the Scaled Preconditioned GMRES solution of the linear systems,
  the user must make the call
  \index{FKINSPGMR@\texttt{FKINSPGMR}}
\begin{verbatim}
      CALL FKINSPGMR (MAXL, MAXLRST, IER)
\end{verbatim}
  The arguments are as follows.
  \id{MAXL} is the maximum Krylov subspace dimension.
  \id{MAXLRST} is the maximum number of restarts.
  \id{IER} is an error return flag which is $0$ to indicate success, $-1$
  if a memory allocation failure occurred, or $-2$ to indicate an illegal input.
  
  Optional outputs specific to the {\spgmr} case are listed in
  Table~\ref{t:fkinsol_out}.

  For descriptions of the relevant optional user-supplied routines, see 
  {\bf User-supplied routines for SPGMR/SPBCG/SPTFQMR} below.
  
  %%-------------------------------------------------
  
  {\s}{\p} {\bf SPBCG treatment of the linear systems}
  \index{KINSPBCG@{\kinspbcg} linear solver!use in {\fkinsol}}
  
  For the Scaled Preconditioned Bi-CGStab solution of the linear systems,
  the user must make the call
  \index{FKINSPBCG@\texttt{FKINSPBCG}}
\begin{verbatim}
      CALL FKINSPBCG (MAXL, IER)
\end{verbatim}
  Its arguments are the same as those with the same names for \id{FKINSPGMR}.

  Optional outputs specific to the {\spbcg} case are listed in
  Table~\ref{t:fkinsol_out}.

  For descriptions of the relevant optional user-supplied routines, see 
  {\bf User-supplied routines for SPGMR/SPBCG/SPTFQMR} below.
  
  %%-------------------------------------------------
  
  {\s}{\p} {\bf SPTFQMR treatment of the linear systems}
  \index{KINSPTFQMR@{\kinsptfqmr} linear solver!use in {\fkinsol}}
  
  For the Scaled Preconditioned Transpose-Reee Quasi-Minimal Residual solution 
  of the linear systems, the user must make the call
  \index{FKINSPTFQMR@\texttt{FKINSPTFQMR}}
\begin{verbatim}
      CALL FKINSPTFQMR (MAXL, IER)
\end{verbatim}
  Its arguments are the same as those with the same names for \id{FKINSPGMR}.

  Optional outputs specific to the {\sptfqmr} case are listed in
  Table~\ref{t:fkinsol_out}.

  For descriptions of the relevant optional user-supplied routines, see below.

  %%-------------------------------------------------

  {\s}{\p} {\bf Functions used by SPGMR/SPBCG/SPTFQMR}

  An optional user-supplied routine, \id{FKINJTIMES} (see below), can be provided for
  Jacobian-vector products.  If it is, then, following the call to
  \id{FKINSPGMR}, \id{FKINSPBCG}, or \id{FKINSPTFQMR}, the user must make the call:
  \index{FKINSPILSSETJAC@\texttt{FKINSPILSSETJAC}}
\begin{verbatim}
      CALL FKINSPILSSETJAC (FLAG, IER)
\end{verbatim}
  with \id{FLAG} $\neq 0$ to specify use of the user-supplied
  Jacobian-times-vector approximation.
  The argument \id{IER} is an error return flag which is $0$ 
  for success or non-zero if an error occurred.
  
  If preconditioning is to be done, then the user must call
  \index{FKINSPILSSETPREC@\texttt{FKINSPILSSETPREC}}
\begin{verbatim}
      CALL FKINSPILSSETPREC (FLAG, IER)
\end{verbatim}
  with \id{FLAG} $\neq 0$.  The return flag \id{IER} is 0 if
  successful, or negative if a memory error occurred.
  In addition, the user program must include
  preconditioner routines \id{FKPSOL} and \id{FKPSET} (see below).

  %%-------------------------------------------------

 {\s}{\p} {\bf User-supplied routines for SPGMR/SPBCG/SPTFQMR}

  With treatment of the linear systems by any of the Krylov iterative
  solvers, there are three optional user-supplied routines ---
  \id{FKINJTIMES}, \id{FKPSOL}, and \id{FKPSET}.  The specifications
  for these routines are given below.

  \index{Jacobian approximation function!Jacobian times vector!use in {\fkinsol}}
  As an option when using the {\spgmr}, {\spbcg}, or {\sptfqmr}  linear solvers, 
  the user may supply a routine that computes the product of the system Jacobian 
  $J = \partial F / \partial u$ 
  and a given vector $v$.  If supplied, it must have the following form:
  \index{FKINJTIMES@\texttt{FKINJTIMES}}
\begin{verbatim}
      SUBROUTINE FKINJTIMES (V, FJV, NEWU, U, IER)
      DIMENSION V(*), FJV(*), U(*)
\end{verbatim}
  Typically this routine will use only \id{NEQ}, \id{U}, \id{V}, and
  \id{FJV}.  It must compute the product vector $Jv$, where the vector $v$ is
  stored in \id{V}, and store the product in \id{FJV}.  
  The input argument \id{U} contains the current
  value of $u$.  On return, set
  \id{IER = 0} if \id{FKINJTIMES} was successful, and nonzero otherwise.

  If preconditioning is to be included, the following routine must be
  supplied, for solution of the preconditioner linear system:
  \index{FKPSOL@\texttt{FKPSOL}}
\begin{verbatim}
      SUBROUTINE FKPSOL (U, USCALE, FVAL, FSCALE, VTEM, FTEM, IER)
      DIMENSION U(*), USCALE(*), FVAL(*), FSCALE(*), VTEM(*), FTEM(*)
\end{verbatim}
  Typically this routine will use only \id{U}, \id{FVAL}, \id{VTEM} and \id{FTEM}.
  It must solve the preconditioned linear system $Pz = r$, where
  $r = $ \id{VTEM} is input, and store the solution $z$ in \id{VTEM} as well. 
  Here $P$ is the right preconditioner. If scaling is being used, the
  routine supplied must also account for scaling on either coordinate
  or function value, as given in the arrays \id{USCALE} and
  \id{FSCALE}, respectively.
  
  If the user's preconditioner requires that any Jacobian-related data be evaluated
  or preprocessed, then the following routine can be used for the evaluation and
  preprocessing of the preconditioner:
  \index{FKPSET@\texttt{FKPSET}}
\begin{verbatim}
      SUBROUTINE FKPSET (U, USCALE, FVAL, FSCALE, VTEMP1, VTEMP2, IER)
      DIMENSION U(*), USCALE(*), FVAL(*), FSCALE(*), VTEMP1(*), VTEMP2(*)
\end{verbatim}
  It must perform any evaluation of Jacobian-related data and
  preprocessing needed for the solution of the preconditioned linear
  systems by \id{FKPSOL}. The variables \id{U} through \id{FSCALE} are for use in the
  preconditioning setup process. Typically, the system function \id{FKFUN} is
  called before any calls to \id{FKPSET}, so that \id{FVAL} will have
  been updated. \id{U} is the current solution
  iterate. The arrays \id{VTEMP1} and \id{VTEMP2} are available for work space. 
  If scaling is being used, \id{USCALE} and \id{FSCALE} are available for those operations
  requiring scaling. \id{NEQ} is the problem size.
  
  On return, set \id{IER} $= 0$ if \id{FKPSET} was successful or set \id{IER} $= 1$
  if an error occurred.
  
  {\warn} If the user calls \id{FKINSPILSSETPREC}, the routine \id{FKPSET} must
  be provided, even if it is not needed and must return \id{IER=0}.

%%====================
\item {\bf Problem solution}
%%====================

  Solving the nonlinear system is accomplished by making the following call:
  \index{FKINSOL@\texttt{FKINSOL}}
\begin{verbatim}
      CALL FKINSOL (U, GLOBALSTRAT, USCALE, FSCALE, IER)
\end{verbatim}
  The arguments are as follows.
  \id{U} is an array containing the initial guess on input, and the
  solution on return.
  \id{GLOBALSTRAT} is an integer (type \id{INTEGER}) defining the global strategy 
  choice ($1$ specifies Inexact Newton, while $2$ indicates line search).
  \id{USCALE} is an array of scaling factors for the \id{U} vector.
  \id{FSCALE} is an array of scaling factors for the \id{FVAL} vector.
  \id{IER} is an integer completion flag and will have one of the following values:
  $0$ to indicate success,
  $1$ to indicate that the initial guess satisfies $F(u) = 0$ within tolerances,
  $2$ to indicate apparent stalling (small step), or a negative value to indicate
  an error or failure. These values correspond to the \id{KINSol} returns
  (see \S\ref{sss:kinsol} and \S\ref{s:kinsol_out_constants}). The values of
  the optional outputs are available in \id{IOPT} and \id{ROPT}
  (see Table~\ref{t:fkinsol_out}).

%%====================
\item {\bf Memory deallocation}
%%====================

  To free the internal memory created by the call to \id{FKINMALLOC},
  make the call
  \index{FKINFREE@\texttt{FKINFREE}}
\begin{verbatim}
      CALL FKINFREE
\end{verbatim}

\end{Steps}
\index{FKINSOL@{\fkinsol} interface module!usage|)}


%%==============================================================================
\section{FKINSOL optional input and output}
%%==============================================================================
\index{FKINSOL@{\fkinsol} interface module!optional input and output}

In order to keep the number of user-callable {\fkinsol} interface routines to
a minimum, optional inputs to the {\kinsol} solver are passed through only two
routines: \Id{FKINSETIIN} for integer optional inputs and \ID{FKINSETRIN}
for real optional inputs. These functions should be called as follows:
\begin{verbatim}
      CALL FKINSETIIN (KEY, IVAL, IER)
      CALL FKINSETRIN (KEY, RVAL, IER)
\end{verbatim}
where \id{KEY} is a quoted string indicating which optional input is set
(see Table \ref{t:fkinsol_in}),
\id{IVAL} is the integer input value to the used,
\id{RVAL} is the real input value to be used, and
\id{IER} is an integer return flag which is set to $0$ on success and
a negative value if a filure occurred.

The optional outputs from the {\kinsol} solver are accessed not through
individual functions, but rather through a pair of arrays, \Id{IOUT}
(integer type) of dimension at least $15$, and \Id{ROUT} (real type) of
dimension at least $2$.  These arrays are owned (and allocated) by the user
and are passed as arguments to \id{FKINMALLOC}.
Table \ref{t:fkinsol_out} lists the entries in these two arrays and specifies the
optional variable as well as the {\kinsol} function which is actually called to
extract the optional output.

For more details on the optional inputs and outputs, see \S\ref{ss:optional_input}
and \S\ref{ss:optional_output}.

\begin{table}
\centering
\caption{Keys for setting {\fkinsol} optional inputs}
\label{t:fkinsol_in}
\medskip
\begin{tabular}{|r|c|l|}
\multicolumn{3}{c}{Integer optional inputs \id{FKINSETIIN}}\\
\hline
{\bf Key} & {\bf Optional input} & Default value \\
\hline
\Id{PRNT\_LEVEL}     & Verbosity level of output & $0$ \\
\Id{MAX\_NITER}      & Maximum no. of nonlinear iterations & $200$ \\
\Id{ETA\_FORM}       & Form of $\eta$ coefficient & $1$ (\id{KIN\_ETACHOICE1}) \\
\Id{MAX\_SETUPS}     & Maximum no. of iterations without prec. setup & $10$ \\
\Id{MAX\_SP\_SETUPS} & Maximum no. of iterations without residual check & $5$ \\
\Id{NO\_INIT\_SETUP} & No initial preconditioner setup & \id{FALSE} \\
\Id{NO\_MIN\_EPS}    & Lower bound on $\epsilon$ & \id{FALSE} \\
\Id{NO\_RES\_MON}    & No residual monitoring & \id{FALSE} \\
\hline
\multicolumn{3}{c}{}\\
\multicolumn{3}{c}{Real optional inputs (\id{FKINSETRIN})}\\
\hline
{\bf Key} & {\bf Optional input} & Default value \\
\hline
\Id{FNORM\_TOL}   & Function-norm stopping tolerance & $\sqrt[3]{\text{uround}}$ \\
\Id{SSTEP\_TOL}   & Scaled-step stopping tolerance & ${\text uround}^{2/3}$ \\
\Id{MAX\_STEP}    & Max. scaled length of Newton step & $1000 \| D_u u_0 \|_2$ \\
\Id{RERR\_FUNC}   & Relative error for F.D. $Jv$ & $\sqrt{\text{uround}}$ \\
\Id{ETA\_CONST}   & Constant value of $\eta$ & $0.1$ \\
\Id{ETA\_PARAMS}  & Values of $\gamma$ and $\alpha$ & $0.9$ and $2.0$ \\
\Id{RMON\_CONST}  & Constant value of $\omega$ & $0.9$ \\
\Id{RMON\_PARAMS} & Values of $\omega_{min}$ and $\omega_{max}$ & $0.00001$ and $0.9$ \\
\hline
\end{tabular}
\end{table}

\begin{table}
\centering
\caption{Description of the {\fkinsol} optional output arrays \Id{IOUT} and \Id{ROUT}}
\label{t:fkinsol_out}
\medskip
\begin{tabular}{|r|c|l|}
\multicolumn{3}{c}{Integer output array \id{IOUT}} \\
\hline
{\bf Index} & {\bf Optional output} & {\kinsol} {\bf function} \\
\hline
\multicolumn{3}{|c|}{{\kinsol} main solver} \\
\hline
1  & \id{LENRW}  & \id{KINGetWorkSpace} \\
2  & \id{LENIW}  & \id{KINGetWorkSpace} \\
3  & \id{NNI}    & \id{KINGetNumNonlinSolvIters} \\
4  & \id{NFE}    & \id{KINGetNumFuncEvals} \\
5  & \id{NBCF}   & \id{KINGetNumBetaCondFails} \\
6  & \id{NBKTRK} & \id{KINGetNumBacktrackOps} \\
\hline
\multicolumn{3}{|c|}{{\kindense} linear solver}\\
\hline
7  & \id{LENRWLS}  & \id{KINDenseGetWorkSpace} \\ 
8  & \id{LENIWLS}  & \id{KINDenseGetWorkSpace} \\ 
9  & \id{LS\_FLAG} & \id{KINDenseGetLastFlag} \\ 
10 & \id{NFELS}    & \id{KINDenseGetNumFuncEvals} \\ 
11 & \id{NJE}      & \id{KINDenseGetNumJacEvals} \\ 
\hline
\multicolumn{3}{|c|}{{\kinband} linear solver}\\
\hline
7  & \id{LENRWLS}  & \id{KINBandGetWorkSpace} \\
8  & \id{LENIWLS}  & \id{KINBandGetWorkSpace} \\
9  & \id{LS\_FLAG} & \id{KINBandGetLastFlag} \\
10 & \id{NFELS}    & \id{KINBandGetNumFuncEvals} \\
11 & \id{NJE}      & \id{KINBandGetNumJacEvals} \\
\hline
\multicolumn{3}{|c|}{{\kinspgmr}, {\kinspbcg}, {\kinsptfqmr}  linear solvers}\\
\hline
7  & \id{LENRWLS}  & \id{KINSpilsGetWorkSpace} \\
8  & \id{LENIWLS}  & \id{KINSpilsGetWorkSpace} \\
9  & \id{LS\_FLAG} & \id{KINSpilsGetLastFlag} \\
10 & \id{NFELS}    & \id{KINSpilsGetNumFuncEvals} \\
11 & \id{NJTV}     & \id{KINSpilsGetNumJacEvals} \\
12 & \id{NPE}      & \id{KINSpilsGetNumPrecEvals} \\
13 & \id{NPS}      & \id{KINSpilsGetNumPrecSolves} \\
14 & \id{NLI}      & \id{KINSpilsGetNumLinIters} \\
15 & \id{NCFL}     & \id{KINSpilsGetNumConvFails} \\
\hline
\multicolumn{3}{c}{}\\
\multicolumn{3}{c}{Real output array \id{ROUT}}\\
\hline
{\bf Index} & {\bf Optional output} & {\kinsol} {\bf function} \\ 
\hline
1  & \id{FNORM} & \id{KINGetFuncNorm} \\
2  & \id{SSTEP} & \id{KINGetStepLength} \\
\hline
\end{tabular}
\end{table}                                                                  


%%==============================================================================
\section{Usage of the FKINBBD interface to KINBBDPRE}
%%==============================================================================
\index{FKINBBD@{\fkinbbd} interface module!interface to the {\kinbbdpre} module|(}

The {\fkinbbd} interface sub-module is a package of {\C} functions which,
as part of the {\fkinsol} interface module, support the use of the
{\kinsol} solver with the parallel {\nvecp} module and the {\kinbbdpre} 
preconditioner module (see \S\ref{sss:kinbbdpre}), for the solution of 
nonlinear problems in a mixed {\F}/{\C} setting.  

The user-callable functions in this package, with the corresponding
{\kinsol} and {\kinbbdpre} functions, are as follows: 
\begin{itemize}
\item \id{FKINBBDINIT}
  interfaces to \id{KINBBDPrecAlloc}.
\item \id{FKINBBDSPGMR}
  interfaces to \id{KINBBDSpgmr} and {\spgmr} optional input functions.
\item \id{FKINBBDSPBCG}
  interfaces to \id{KINBBDSpbcg} and {\spbcg} optional input functions.
\item \id{FKINBBDSPTFQMR}
  interfaces to \id{KINBBDSptfmqr} and {\sptfqmr} optional input functions.
\item \id{FKINBBDOPT}
  interfaces to {\kinbbdpre} optional output functions.
\item \id{FKINBBDFREE}
  interfaces to \id{KINBBDPrecFree}.
\end{itemize}

In addition to the {\F} right-hand side function \id{FKFUN}, the
user-supplied functions used by this package, are listed below,
each with the corresponding interface function which calls it (and its
type within {\kinbbdpre} or {\kinsol}):
\begin{center}
\begin{tabular}{l|l|l}
{\fkinbbd} routine ({\F})  &  {\kinsol} function ({\C}) & {\kinsol} function type \\\hline
\id{FKLOCFN}  & \id{FKINgloc}     & \id{KINLocalFn} \\
\id{FKCOMMF}  & \id{FKINgcomm}      & \id{KINCommFn} \\
\id{FKJTIMES} & \id{FKINJtimes}   & \id{KINSpilsJacTimesVecFn} \\
\end{tabular}
\end{center}
As with the rest of the {\fkinsol} routines, the names of all user-supplied routines
here are fixed, in order to maximize portability for the resulting mixed-language
program. Additionally, based on flags discussed above in \S\ref{sss:fkinroutines},
the names of the user-supplied routines are mapped to actual values through a
series of definitions in the header file \id{fkinbbd.h}.

The following is a summary of the usage of this module. Steps that are unchanged
from the main program described in \S\ref{ss:fkinsol_usage} are grayed-out.

\index{User main program!FKINBBD@{\fkinbbd} usage}
\begin{Steps}
  
\item \textcolor{gray}{\bf Nonlinear system function specification}

\item \textcolor{gray}{\bf {\nvector} module initialization}

\item \textcolor{gray}{\bf Problem specification}

\item {\bf Linear solver specification}

  To initialize the {\kinbbdpre} preconditioner, make the following call:
  \index{FKINBBDINIT@\texttt{FKINBBDINIT}}
\begin{verbatim}
      CALL FKINBBDINIT (NLOCAL, MUDQ, MLDQ, MU, ML, IER)
\end{verbatim}
  The arguments are as follows.
  \id{NLOCAL} is the local size of vectors for this process.
  \id{MUDQ} and \id{MLDQ} are the upper and lower half-bandwidths to be used
  in the computation of the local Jacobian blocks by difference quotients;
  these may be smaller than the true half-bandwidths of the Jacobian of the
  local block of $G$, when smaller values may provide greater efficiency.
  \id{MU} and \id{ML} are the upper and lower half-bandwidths of the band
  matrix that is retained as an approximation of the local Jacobian block;
  these may be smaller than \id{MUDQ} and \id{MLDQ}.
  \id{IER} is a return completion flag.  A value of $0$ indicates success, while
  a value of $-1$ indicates that a memory failure occurred or that an input had
  an illegal value.

  To specify the {\spgmr} linear system solver and use the {\kinbbdpre}
  preconditioner, make the following call:
  \index{FKINBBDSPGMR@\texttt{FKINBBDSPGMR}}
\begin{verbatim}
      CALL FKINBBDSPGMR (MAXL, MAXLRST, IER)
\end{verbatim}
  Its arguments are the same as those of \id{FKINSPGMR}
  (see step \ref{i:fkinsol_lin_solv_spec} in \S\ref{ss:fkinsol_usage}).

  To specify the {\spbcg} linear system solver and use the {\kinbbdpre}
  preconditioner, make the following call:
  \index{FKINBBDSPBCG@\texttt{FKINBBDSPBCG}}
\begin{verbatim}
      CALL FKINBBDSPBCG (MAXL, IER)
\end{verbatim}
  Its arguments are the same as those of \id{FKINSPGMR}
  (see step \ref{i:fkinsol_lin_solv_spec} in \S\ref{ss:fkinsol_usage}).

  To specify the {\sptfqmr} linear system solver and use the {\kinbbdpre}
  preconditioner, make the following call:
  \index{FKINBBDSPTFQMR@\texttt{FKINBBDSPTFQMR}}
\begin{verbatim}
      CALL FKINBBDSPTFQMR (MAXL, IER)
\end{verbatim}
  Its arguments are the same as those of \id{FKINSPGMR}
  (see step \ref{i:fkinsol_lin_solv_spec} in \S\ref{ss:fkinsol_usage}).

  Optionally, to specify that {\spgmr}, {\spbcg}, or {\sptfqmr} should use
  the supplied \id{FKJTIMES}, make the call
  \index{FKINSPILSSETJAC@\texttt{FKINSPILSSETJAC}}
\begin{verbatim}
      CALL FKINSPILSSETJAC (FLAG, IER)
\end{verbatim}
  with \id{FLAG} $\neq 0$.
  (see step \ref{i:fkinsol_lin_solv_spec} in \S\ref{ss:fkinsol_usage}).


\item \textcolor{gray}{\bf Problem solution}

\item {\bf {\kinbbdpre} Optional outputs}

  Optional outputs specific to the {\spgmr}, {\spbcg}, or {\sptfqmr} solver are
  listed in Table \ref{t:fkinsol_out}.
  To obtain the optional outputs associated with the {\kinbbdpre} module, make
  the following call:
  \index{FKINBBDOPT@\texttt{FKINBBDOPT}}
\begin{verbatim}
      CALL FKINBBDOPT (LENRBBD, LENIBBD, NGEBBD)
\end{verbatim}
  The arguments returned are as follows.
  \id{LENRBBD} is the length of real preconditioner work space, in \id{realtype}
  words. \id{LENIBBD} is the length of integer preconditioner work space, in
  integer words. These sizes are local to the current process.
  \id{NGEBBD} is the cumulative number of $G(u)$ evaluations (calls to \id{FKLOCFN})
  so far.
  
\item {\bf Memory deallocation}

  To free the internal memory created by the call to \id{FKINBBDINIT}, before
  calling \id{FKINFREE} and \id{FNVFREEP}, the user must call
  \index{FKINBBDFREE@\texttt{FKINBBDFREE}}
\begin{verbatim}
      CALL FKINBBDFREE
\end{verbatim}
\index{FKINSOL@{\fkinsol} interface module!interface to the {\kinbbdpre} module|)}

\item {\bf User-supplied routines}

  The following two routines must be supplied for use with the {\kinbbdpre}
  module:
  \index{FKLOCFN@\texttt{FKLOCFN}}
\begin{verbatim}
      SUBROUTINE FKLOCFN (NLOC, ULOC, GLOC, IER)
      DIMENSION ULOC(*), GLOC(*)
\end{verbatim}
  This routine is to evaluate the function $G(u)$ approximating $F$
  (possibly identical to $F$), in terms of the array
  \id{ULOC} (of length \id{NLOC}), which is the sub-vector
  of $u$ local to this processor.  The resulting (local) sub-vector
  is to be stored in the array \id{GLOC}.
  \id{IER} is an error return flag which should be set to $0$ if successful,
  a positive value if a recoverable error occurred (in which case {\kinsol} will
  attempt to correct), or a negative value if \id{FKLOCFN} failed unrecoverably
  (in which case the solution process is halted).

  \index{FKCOMMFN@\texttt{FKCOMMFN}}
\begin{verbatim}
      SUBROUTINE FKCOMMFN (NLOC, ULOC, IER)
      DIMENSION ULOC(*)
\end{verbatim}
  This routine is to perform the inter-processor communication necessary
  for the \id{FKLOCFN} routine.
  Each call to \id{FKCOMMFN} is preceded by a call to the system function
  routine \id{FKFUN} with the same argument \id{ULOC}.
  \id{IER} is an error return flag which should be set to $0$ if successful,
  a positive value if a recoverable error occurred (in which case {\kinsol} will
  attempt to correct), or a negative value if \id{FKCOMMFN} failed recoverably
  (in which case the solution process is halted).

  {\warn}The subroutine \id{FKCOMMFN} must be supplied even if it is not needed
  and must return \id{IER=0}.

  \index{FKINJTIMES@\texttt{FKINJTIMES}}
  Optionally, the user can supply a routine \id{FKINJTIMES} for the evaluation of
  Jacobian-vector products, as described above in step \ref{i:fkinsol_lin_solv_spec}
  in \S\ref{ss:fkinsol_usage}.

\end{Steps}
