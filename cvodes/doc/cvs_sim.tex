%%===================================================================================
\chapter{Using {\cvodes} for IVP Solution}\label{s:simulation}
%%===================================================================================

This section is concerned with the use of {\cvodes} for the integration of IVPs.
The following subsections treat the header files, the layout of the user's main
program, description of the {\cvodes} user-callable functions, and user-supplied functions. 
The listings of the sample programs in the companion document \cite{} may also be helpful. 
Those codes are intended to serve as templates and are included in the {\cvodes} package.

The user should be aware that not all linear solver modules are compatible 
with all {\nvector} implementations. 
\index{CVODES@{\cvodes} linear solvers!NVECTOR@{\nvector} compatibility}
For example, {\nvecp} is not compatible with the direct dense or direct band 
linear solvers since these linear solver modules need to form the system Jacobian.
The following {\cvodes} modules can only be used with {\nvecs}:
{\cvdense}, {\cvband}, and {\cvbandpre}. The preconditioner module {\cvbbdpre}
can only be used with {\nvecp}. 

%%===================================================================================
\section{Data Types}\label{s:types}
%%===================================================================================
% This is a shared SUNDIALS TEX file with description of
% types used in llntyps.h
%
\index{portability}
The \ID{sundials\_types.h} file contains the definition of the type \ID{realtype},
which is used by the {\sundials} solvers for all floating-point data, the definition 
of the integer type \ID{sunindextype}, which is used for vector and matrix indices,
and \ID{booleantype}, which is used for certain logic operations within {\sundials}.


\subsection{Floating point types}

The type \id{realtype} can be \id{float}, \id{double}, or \id{long double}, with
the default being \id{double}.
The user can change the precision of the {\sundials} solvers arithmetic at the
configuration stage (see \S\ref{ss:configuration_options_nix}).

Additionally, based on the current precision, \id{sundials\_types.h} defines 
\Id{BIG\_REAL} to be the largest value representable as a \id{realtype},
\Id{SMALL\_REAL} to be the smallest value representable as a \id{realtype}, and
\Id{UNIT\_ROUNDOFF} to be the difference between $1.0$ and the minimum \id{realtype}
greater than $1.0$.

Within {\sundials}, real constants are set by way of a macro called
\Id{RCONST}.  It is this macro that needs the ability to branch on the
definition \id{realtype}.  In ANSI {\CC}, a floating-point constant with no
suffix is stored as a \id{double}.  Placing the suffix ``F'' at the
end of a floating point constant makes it a \id{float}, whereas using the suffix
``L'' makes it a \id{long double}.  For example,
\begin{verbatim}
#define A 1.0
#define B 1.0F
#define C 1.0L
\end{verbatim}
defines \id{A} to be a \id{double} constant equal to $1.0$, \id{B} to be a
\id{float} constant equal to $1.0$, and \id{C} to be a \id{long double} constant
equal to $1.0$.  The macro call \id{RCONST(1.0)} automatically expands to \id{1.0}
if \id{realtype} is \id{double}, to \id{1.0F} if \id{realtype} is \id{float},
or to \id{1.0L} if \id{realtype} is \id{long double}.  {\sundials} uses the
\id{RCONST} macro internally to declare all of its floating-point constants. 

A user program which uses the type \id{realtype} and the \id{RCONST} macro
to handle floating-point constants is precision-independent except for
any calls to precision-specific standard math library
functions.  (Our example programs use both \id{realtype} and
\id{RCONST}.)  Users can, however, use the type \id{double}, \id{float}, or
\id{long double} in their code (assuming that this usage is consistent
with the typedef for \id{realtype}).  Thus, a previously existing
piece of ANSI {\CC} code can use {\sundials} without modifying the code
to use \id{realtype}, so long as the {\sundials} libraries use the
correct precision (for details see \S\ref{ss:configuration_options_nix}).


\subsection{Integer types used for vector and matrix indices}

The type \id{sunindextype} can be either a 64- or 32-bit \emph{signed} integer.
The default is the portable \id{int64\_t} type, and the user can change it
to \id{int32\_t} at the configuration stage. The configuration system
will detect if the compiler does not support portable types, and will
replace \id{int64\_t} and \id{int32\_t} with \id{long long} and \id{int},
respectively, to ensure use of the desired sizes on Linux, Mac OS X and Windows
platforms. {\sundials} currently does not support \emph{unsigned} integer types 
for vector and matrix indices, although these could be added in the future if there 
is sufficient demand.


%%===================================================================================
\section{Header files}\label{ss:header_sim}
%%===================================================================================

The calling program must include several header files so that various macros
and data types can be used. The header file that is always required is:
%%
\begin{itemize}
\item  \Id{cvodes.h}, 
  the header file for {\cvodes}, which defines the several
  types and various constants, and includes function prototypes.
\end{itemize}
%%
Note that \id{cvodes.h} includes \Id{sundialstypes.h}, 
which defines the types \id{realtype, integertype, booleantype}
and constants \id{FALSE} and \id{TRUE}.

The calling program must also include an {\nvector} implementation header file
(see \S\ref{s:nvector} for details).
For the two {\nvector} implementations that are included in the {\cvodes} package,
the corresponding header files are:
%%
\begin{itemize}
\item \Id{nvector\_serial.h}, 
  which defines the serial implementation {\nvecs};
\item \Id{nvector\_parallel.h}, 
  which defines the parallel MPI implementation, {\nvecp}.
\end{itemize}
%%
Note that both these files include in turn the header file \Id{nvector.h} which 
defines the abstract \Id{N\_Vector} and \Id{N\_VSpec} types. 

Finally, if the user chooses Newton iteration for the solution of the nonlinear systems,
then a linear solver module header file will be required. 
\index{CVODES@{\cvodes} linear solvers!header files}
The header files corresponding to the various linear solver options in {\cvodes} are:
%%
\begin{itemize}
\item \Id{cvsdense.h}, 
  which is used with the dense direct linear solver in 
  the context of {\cvodes}. This in turn includes a header file (\id{dense.h})
  which defines the \Id{DenseMat} type and corresponding accessor macros; 
\item \Id{cvsband.h}, 
  which is used with the band direct linear solver in the
  context of {\cvodes}. This in turn includes a header file (\id{band.h})
  which defines the \Id{BandMat} type and corrsponding accessor macros;
\item \Id{cvsdiag.h}, which is used with a diagonal linear solver in the
  context of {\cvodes};
\item \Id{cvsspgmr.h}, 
  which is used with the Krylov solver {\spgmr} in the
  context of {\cvodes}. This in turn includes a header file (\id{iterative.h})
  which enumerates the kind of preconditioning and the choices for the
  Gram-Schmidt process.
\end{itemize}

Other headers may be needed, according as to the choice of
preconditioner, etc. In one of the examples in \cite{}, preconditioning
is done with a block-diagonal matrix. For this, the header
\id{smalldense.h} is included.

%%===================================================================================
\section{A skeleton of the user's main program}\label{ss:skeleton_sim}
%%===================================================================================

A high-level view of the combined user program and {\cvodes} package is
shown in Figure~\ref{f:sim_overview}.
%%
\begin{figure}
\centerline{\psfig{figure=cvssim.eps,width=\textwidth}}
\caption {Diagram of the user program and 
  {\cvodes} package for integration of IVP}\label{f:sim_overview}
\end{figure}
%%
The following is a skeleton of the user's main program (or calling
program) for the integration of an ODE IVP. Some steps are independent of the {\nvector}
implementation used; where this is not the case, usage specifications are given for the
two implementations provided with {\cvodes}: steps marked with {\p} correspond to 
{\nvecp}, while steps marked with {\s} correspond to {\nvecs}.
%%
%%
%%
\begin{Steps}
  
\item 
  {\bf {\p} Initialize MPI}

  \id{MPI\_Init(\&argc, \&argv);} to initialize MPI if used by
  the user's program, aside from the internal use in {\nvecp}.  
  Here \id{argc} and \id{argv} are the command line argument 
  counter and array received by \id{main}.
  
\item
  {\bf Set problem dimensions}

  {\s} Set \id{N}, the problem size $N$.

  {\p} Set \id{Nlocal}, the local vector length (the sub-vector
  length for this processor); \id{N}, the global vector length (the
  problem size $N$, and the sum of all the values of \id{Nlocal});
  and the active set of processors.
  
\item\label{i:nv_spec_init}
  {\bf Initialize vector specification}

  {\s} \id{nvSpec = }\Id{NV\_SpecInit\_Serial}\id{(N);}

  {\p} \id{nvSpec = }\Id{NV\_SpecInit\_Parallel}\id{(comm, Nlocal, N, \&argc, \&argv);}
  Here \id{comm} is the MPI communicator, set in one of two ways: 
  If a proper subset of active processors is to be used, \id{comm} 
  must be set by suitable MPI calls. Otherwise, to specify that all 
  processors are to be used, \id{comm} must be \id{MPI\_COMM\_WORLD}.
  
\item
  {\bf Set intial values}
 
  To set the vector \id{y0} of initial values, use macros defined by a particular 
  {\nvector} implementation:

  {\s} \id{NV\_MAKE\_S(y0, ydata, machEnv);}

  {\p} \id{NV\_MAKE\_P(y0, ydata, machEnv);}

  if an existing real array \id{ydata} contains the initial values of $y$.  
  Otherwise, make the call \id{y0 = }\Id{N\_VNew}\id{(machEnv);} and load 
  initial values into the real array defined by:

  {\s} \id{NV\_DATA\_S(y0)}

  {\p} \id{NV\_DATA\_P(y0)}
  
\item\label{i:cvode_create} 
  {\bf Create {\cvodes} object}

  Call \id{cvode\_mem = }\id{CVodeCreate}\id{(...);} 
  to create the {\cvodes} memory block and specify the solution method
  (linear multistep method and nonlinear solver iteration type).
  \id{CVodeCreate} returns a pointer to the {\cvodes} memory structure.

\item
  {\bf Set optional inputs}

  Call \id{CVodeSet*} functions to change from their default values any
  optional inputs that control the behavior of {\cvodes}.

\item\label{i:cvode_malloc} 
  {\bf Allocate internal memory}

  Call \id{cvode\_mem = }\id{CVodeMalloc}\id{(...);} 
  to provide required problem specifications,
  allocate internal memory for {\cvodes}, 
  and initialize {\cvodes}.
  \id{CVodeMalloc} returns an error flag to indicate success or an illegal argument value.
  
\item\label{i:lin_solver} 
  {\bf Attach linear solver module}

  If Newton iteration is chosen, initialize the linear solver module
  with one of the following calls (for details see \S\ref{sss:lin_solv_init}):

  {\s} \id{ier = }\Id{CVDense}\id{(...);}

  {\s} \id{ier = }\Id{CVBand}\id{(...);}

  \id{ier = }\Id{CVDiag}\id{(...);}

  \id{ier = }\Id{CVSpgmr}\id{(...);}
  
\item
  {\bf Set linear solver optional inputs}

  Call \id{CV*Set*} functions from the selected linear solver module to
  change optional inputs specific to that linear solver.

\item
  {\bf Advance solution in time}

  For each point at which output is desired, call
  \id{ier = }\Id{CVode}\id{(cvode\_mem, tout, yout, \&tret, itask);}
  Set \id{itask} to specify the return mode.
  The vector \id{y} (which can be the same as
  the vector \id{y0} above) will contain $y(t)$.
  
\item
  {\bf Deallocate memory for solution vector}

  Upon completion of the integration, deallocate memory for the vector \id{y}
  by either calling a macro defined by the {\nvector} implementation:

  {\s} \id{NV\_DISPOSE\_S(y);}

  {\p} \id{NV\_DISPOSE\_P(y);}

  if \id{y} was created from \id{ydata}, or by making the call 
  \Id{N\_VFree}\id{(y);} if \id{y} was created by a call to \id{N\_VNew}.
  
\item
  {\bf Free solver memory}

  \Id{CVodeFree}\id{(cvode\_mem);} to free the memory allocated for {\cvodes}.
  
\item
  {\bf Free vector specification memory}

  {\s} \Id{NV\_SpecFree\_Serial}\id{(nvSpec);}

  {\p} \Id{NV\_SpecFree\_Parallel}\id{(nvSpec);}

\item 
  {\bf {\p} Finalize MPI}
  
\end{Steps}

%%===================================================================================
\section{User-callable functions for IVP solution}
\label{ss:cvodes_fct_sim}
%%===================================================================================

This section describes the {\cvodes} functions that are called by the user to set up 
and solve an IVP. Some of these are required. However, starting with \S\ref{ss:optional_input},
the functions listed involve optional inputs/outputs or restarting, and those paragraphs can 
be skipped for a casual use of {\cvodes}. In any case, refer to \S\ref{ss:skeleton_sim} for
the correct order of these calls.

\subsection{{\cvodes} initialization and deallocation functions}
\label{sss:cvodemalloc}
%%
The following three functions must be called in the order listed. The last one is to be 
called only after the IVP solution is complete, as it frees the {\cvodes} memory block
created and allocated by the first two calls.
%%
\ucfunction{CVodeCreate}
{
  cvode\_mem = CVodeCreate(lmm, iter);
}
{
  Instantiates a {\cvodes} solver object and specifies the solution method.
}
{
  \begin{args}[iter]
  \item[lmm] (\id{int})
    specifies the linear multistep method and may be one of two
    possible values: \Id{ADAMS} or \Id{BDF}.     
  \item[iter] (\id{int})
    specifies the type of nonlinear solver iteration and may be
    either \Id{NEWTON} or \Id{FUNCTIONAL}. 
  \end{args}
  The recommended choices for (\id{lmm}, \id{iter}) are
  (\id{ADAMS}, \id{FUNCTIONAL}) for nonstiff problems and
  (\id{BDF}, \id{NEWTON}) for stiff problems.
}
{
  If successful, \id{CVodeCreate} returns a pointer to the newly created 
  {\cvodes} memory block (of type \id{void *}).
  If an error occured, \id{CVodeCreate} prints an error message to \id{stdout}
  and returns \id{NULL}.
}
{}
%%
%%
\ucfunction{CVodeMalloc}
{
flag = CVodeMalloc(cvode\_mem, f, t0, y0, itol, reltol, abstol, nvSpec);
}
{
  Provides required problem and solution specifications, allocates internal memory,
  and initializes {\cvodes}.
}
{
  \begin{args}[abstol]
  \item[cvode\_mem] (\id{void *})
    pointer to the {\cvodes} memory block returned by \id{CVodeCreate}.
  \item[f] (\Id{RhsFn})
    is the {\C} function which computes $f$ in the ODE. This function has the form 
    \id{f(N, t, y, ydot, f\_data)} (for full details see \S\ref{ss:user_fct_sim}).
  \item[t0] (\id{realtype})
    is the initial value of $t$.
  \item[y0] (\id{N\_Vector})
    is the initial value of $y$. 
  \item[itol] (\id{int}) 
    is either \Id{SS} or \Id{SV}, where \id{SS} indicates scalar relative error 
    tolerance and scalar absolute error tolerance, while \id{SV} indicates scalar
    relative error tolerance and vector absolute error tolerance. 
    The latter choice is important when the absolute error tolerance needs to
    be different for each component of the ODE. 
  \item[reltol] (\id{realtype *})
    is a pointer to the relative error tolerance.
  \item[abstol] (\id{void *})
    is a pointer to the absolute error tolerance.
  \item[nvSpec] (\id{NV\_Spec})
    is a pointer to the vector specification structure. See \S\ref{ss:skeleton_sim}, 
    step \ref{i:nv_spec_init}.
  \end{args}
}
{
  The return flag \id{flag} (of type \id{int}) will be one of the following:.
  \begin{args}[CVM\_ILL\_INPUT]
  \item[\Id{SUCCESS}]
    The call to \id{CVodeMalloc} was successful.
  \item[\Id{CVM\_NO\_MEM}] 
    The {\cvodes} memory block was not initialized through a previous call to \id{CVodeCreate}.
  \item[\Id{CVM\_MEM\_FAIL}] 
    A memory allocation request has failed.
  \item[\Id{CVM\_ILL\_INPUT}] 
    An input argument to \id{CVodeMalloc} has an illegal value.
  \end{args}
}
{
  If an error occured, \id{CVodeMalloc} also prints an error message to the
  file specified by the optional input \id{errfp}.
}
%%
%%
\ucfunction{CVodeFree}
{
  CVodeFree(cvode\_mem);
}
{
  The function \ID{CVodeFree} frees the pointer allocated by
  a previous call to \id{CVodeMalloc}.
}
{
  The argument is the pointer to the {\cvodes} memory block (of type \id{void *}).
}
{
  The function \id{CVodeFree} has no return value.
}
{}
%%
%%===================================================================================
%%
\subsection{Linear solver specification functions}\label{sss:lin_solv_init}

As previously explained, Newton iteration requires the solution of
linear systems of the form (\ref{e:Newton}).  There are four {\cvodes} linear
solvers currently available for this task: {\cvdense}, {\cvband}, {\cvdiag},
and {\cvspgmr}.  The first three are direct solvers and derive their name
from the type of approximation used for the Jacobian 
$J = \partial{f}/\partial{y}$.  {\cvdense}, {\cvband}, and {\cvdiag} work with
dense, banded, and diagonal approximations to $J$, respectively.  The
fourth {\cvodes} linear solver, {\cvspgmr}, is an iterative solver.  The {\spgmr}
in the name indicates that it uses a scaled preconditioned
GMRES method.

\index{CVODES@{\cvodes} linear solvers!selecting one|(} 
To specify a {\cvodes} linear solver, after the call to \id{CVodeCreate}
but before any calls to \id{CVode}, the user's program must call one
of the functions \Id{CVDense}, \Id{CVBand}, \Id{CVDiag}, \Id{CVSpgmr},
as documented below. The first argument passed to these functions is the {\cvodes}
memory pointer returned by \id{CVodeCreate}.  A call to one of these
functions links the main {\cvodes} integrator to a linear solver and
allows the user to specify parameters which are specific to a
particular solver, such as the bandwidths in the {\cvband} case.
%%
The use of each of the linear solvers involves certain constants and possibly 
some macros, that are likely to be needed in the user code.  These are
available in the corresponding header file associated with the linear
solver, as specified below.
\index{CVODES@{\cvodes} linear solvers!selecting one|)}

\index{CVODES@{\cvodes} linear solvers!built on generic solvers|(} 
In each case except the diagonal approximation case {\cvdiag}, the linear
solver module used by {\cvodes} is actually built on top of a generic
linear system solver, which may be of interest in itself.  These
generic solvers, denoted {\dense}, {\band}, and {\spgmr}, are described
separately in \S\ref{s:gen_linsolv}.
\index{CVODES@{\cvodes} linear solvers!built on generic solvers|)}
%%
%%
%%
\index{CVODES@{\cvodes} linear solvers!CVDENSE@{\cvdense}}
\index{CVDENSE@{\cvdense} linear solver!selection of}
\index{CVDENSE@{\cvdense} linear solver!NVECTOR@{\nvector} compatibility}
\ucfunction{CVDense}
{
  flag = CVDense(cvode\_mem, N);
}
{
  The function \ID{CVDense} selects the {\cvdense} linear solver. 

  The user's main function must include the \id{cvsdense.h} header file.
}
{
  \begin{args}[cvode\_mem]
  \item[cvode\_mem] (\id{void *})
    pointer to the {\cvodes} memory block.
  \item[N] (\id{integertype})
    problem dimension.
  \end{args}
}
{
  The return value \id{flag} (of type \id{int}) is one of
  \begin{args}[LIN\_ILL\_INPUT]
  \item[\Id{SUCCESS}] 
    The {\cvdense} initialization was successful.
  \item[\Id{LIN\_NO\_MEM}]
    The \id{cvode\_mem} pointer is \id{NULL}.
  \item[\Id{LIN\_ILL\_INPUT}]
    The {\cvdense} solver is not compatible with the current {\nvector} module.
  \item[\Id{LMEM\_FAIL}]
    A memory allocation request failed.
  \end{args}
}
{
  The {\cvdense} linear solver may not be compatible with a particular
  implementation of the {\nvector} module. 
  Of the two {\nvector} modules provided by {\sundials}, only {\nvecs} is 
  compatible, while {\nvecp} is not.
}
%%
%%
%%
\index{CVODES@{\cvodes} linear solvers!CVBAND@{\cvband}}
\index{CVBAND@{\cvband} linear solver!selection of}
\index{CVBAND@{\cvband} linear solver!NVECTOR@{\nvector} compatibility}
\ucfunction{CVBand}
{
  flag = CVBand(cvode\_mem, N, mupper, mlower);
}
{
  The function \ID{CVBand} selects the {\cvband} linear solver. 

  The user's main function must include the \id{cvsband.h} header file.
}
{
  \begin{args}[cvode\_mem]
  \item[cvode\_mem] (\id{void *})
    pointer to the {\cvodes} memory block.
  \item[N] (\id{integertype})
    problem dimension.
  \item[mupper] (\id{integertype})
    upper half-bandwidth of the problem Jacobian (or of the approximation of it).
  \item[mlower] (\id{integertype})
    lower half-bandwidth of the problem Jacobian (or of the approximation of it).
  \end{args}
}
{
  The return value \id{flag} (of type \id{int}) is one of
  \begin{args}[LIN\_ILL\_INPUT]
  \item[\Id{SUCCESS}] 
    The {\cvband} initialization was successful.
  \item[\Id{LIN\_NO\_MEM}]
    The \id{cvode\_mem} pointer is \id{NULL}.
  \item[\Id{LIN\_ILL\_INPUT}]
    The {\cvband} solver is not compatible with the current {\nvector} module, or
    one of the Jacobian half-bandwidths is outside its valid range ($0 \ldots$ \id{N}$-1$).
  \item[\Id{LMEM\_FAIL}]
    A memory allocation request failed.
  \end{args}
}
{
  The {\cvband} linear solver may not be compatible with a particular
  implementation of the {\nvector} module. Of the two {\nvector} modules 
  provided by {\sundials}, only {\nvecs} is compatible, while {\nvecp} is not.
  The half-bandwidths are to be set so that the nonzero locations $(i,j)$ in the
  banded (approximate) Jacobian satisfy $-$\id{mlower} $\leq j-i \leq$ \id{mupper}.
}
%%
%%
%%
\index{CVODES@{\cvodes} linear solvers!CVDIAG@{\cvdiag}}
\index{CVDIAG@{\cvdiag} linear solver!selection of}
\index{CVDIAG@{\cvdiag} linear solver!Jacobian approximation used by}
\index{Jacobian approximation function!diagonal!difference quotient}
\ucfunction{CVDiag}
{
  flag = CVDiag(cvode\_mem);
}
{
  The function \ID{CVDiag} selects the {\cvdiag} linear solver. 

  The user's main function must include the \id{cvsdiag.h} header file.
}
{
  \begin{args}[cvode\_mem]
  \item[cvode\_mem] (\id{void *})
    pointer to the {\cvodes} memory block.
  \end{args}
}
{
  The return value \id{flag} (of type \id{int}) is one of
  \begin{args}[LMEM\_FAIL]
  \item[\Id{SUCCESS}]
    The {\cvdiag} initialization was successful.
  \item[\Id{LIN\_NO\_MEM}]
    The \id{cvode\_mem} pointer is \id{NULL}.
  \item[\Id{LMEM\_FAIL}]
    A memory allocation request failed.
  \end{args}
}
{
  The {\cvdiag} solver is the simplest of all the current {\cvodes} linear
  solvers. 
  The {\cvdiag} solver uses an approximate diagonal Jacobian formed by way of a difference 
  quotient. The user does {\em not} have the option to supply a function to compute
  an approximate diagonal Jacobian.
}
%%
%%
%%
\index{CVODES@{\cvodes} linear solvers!CVSPGMR@{\cvspgmr}}
\index{CVSPGMR@{\cvspgmr} linear solver!selection of} 
\ucfunction{CVSpgmr}
{
  flag = CVSpgmr(cvode\_mem, pretype, maxl);
}
{
  The function \ID{CVSpgmr} selects the {\cvspgmr} linear solver. 

  The user's main function must include the \id{cvsspgmr.h} header file.
}
{
  \begin{args}[cvode\_mem]
  \item[cvode\_mem] (\id{void *})
    pointer to the {\cvodes} memory block.
  \item[pretype] (\id{int})
    preconditioning type. Can be one of: \Id{NONE}, \Id{LEFT}, \Id{RIGHT}, or \Id{BOTH}.
  \item[maxl] (\id{int})
    maximum dimension of the Krylov subspace to be used. Pass $0$ to use the 
    default value \id{CVSPGMR\_MAXL}$=5$.
  \end{args}
}
{
  The return value \id{flag} (of type \id{int}) is one of
  \begin{args}[LIN\_ILL\_INPUT]
  \item[\Id{SUCCESS}] 
    The {\cvspgmr} initialization was successful.
  \item[\Id{LIN\_NO\_MEM}]
    The \id{cvode\_mem} pointer is \id{NULL}.
  \item[\Id{LIN\_ILL\_INPUT}]
    The preconditioner type \id{pretype} is not valid.
  \item[\Id{LMEM\_FAIL}]
    A memory allocation request failed.
  \end{args}
}
{
  The {\cvspgmr} solver uses a scaled preconditioned GMRES\index{GMRES method}
  iterative method to solve the linear system (\ref{e:Newton}).\\
  \index{preconditioning!advice on|(}With this {\spgmr} method, preconditioning 
  can be done on the left only, on the right only, on both the left and the right, 
  or not at all.  For a given preconditioner matrix, the merits of left vs. right
  preconditioning are unclear in general, and the user should experiment
  with both choices.  Performance will differ because the inverse of the
  left preconditioner is included in the linear system residual whose
  norm is being tested in the {\spgmr} algorithm.  As a rule, however, if
  the preconditioner is the product of two matrices, we recommend that
  preconditioning be done either on the left only or the right only,
  rather than using one factor on each side. For specification of preconditioner,
  see \S\ref{ss:optional_input} and \S\ref{ss:user_fct_sim}.

  If preconditioning is done, user-supplied routines define left and right 
  preconditioner matrices $P_1$ and $P_2$ (either of which could be the identity matrix),
  such that the product $P_1 P_2$ approximates the Newton matrix $M=I-\gamma J$ of 
  (\ref{e:Newtonmat}).
  \index{preconditioning!advice on|)}
}

%%===================================================================================

\subsection{{\cvodes} solver function}\label{sss:cvode}
%
This is the central step in the solution process - the call to perform the integration 
of the IVP.
%
\ucfunction{CVode}
{
  flag = CVode(cvode\_mem, tout, yout, tret, itask);
}
{
  The function \ID{CVode} integrates the ODE over an interval in $t$.
}
{
  \begin{args}[cvode\_mem]
  \item[cvode\_mem] (\id{void *})
    pointer to the {\cvodes} memory block.
  \item[tout] (\id{realtype})
    the next time at which a computed solution is desired.
  \item[yout] (\id{N\_Vector})
    the computed solution vector.
  \item[tret] (\id{realtype *})
    the time reached by the solver.
  \item[itask] (\id{int})
    a flag indicating the job of the solver for the next user step. 
    The \Id{NORMAL} task is to have the solver take internal steps until   
    it has reached or just passed the user specified \id{tout}
    parameter. The solver then interpolates in order to   
    return an approximate value of $y($\id{tout}$)$. 
    The \Id{ONE\_STEP} option tells the solver to just take one internal step  
    and return the solution at the point reached by that step. 
    The \Id{NORMAL\_TSTOP} and \Id{ONE\_STEP\_TSTOP} modes are     
    similar to \id{NORMAL} and \id{ONE\_STEP}, respectively, except    
    that the integration never proceeds past the value      
    \id{tstop} (specified through the function \id{CVodeSetStopTime}).
  \end{args}
}
{
  On return, \id{CVode} returns a vector \id{yout} and a corresponding 
  independent variable value $t=$\id{*tret}, such that \id{yout} is the computed 
  value of $y(t)$.

  In \id{NORMAL} mode with no errors, \id{*tret} will be equal to \id{tout} 
  and \id{yout} = $y($\id{tout}$)$.

  The return value \id{flag} (of type \id{int}) will be one of the following:
  \begin{args}[SOLVER\_FAILURE]
  \item[\Id{SUCCESS}]
    \id{CVode} succeeded.
  \item[\Id{TSTOP\_RETURN}]
    \id{CVode} succeeded by reaching the stopping point specified through
    the optional input function \id{CVodeSetStopTime} (see \S\ref{ss:optional_input}).
  \item[\Id{CVODE\_NO\_MEM}]
    The \id{cvode\_mem} argument was \id{NULL}.
  \item[\Id{ILL\_INPUT}]
    One of the inputs to CVode is illegal. This includes the situation when a 
    component of the error weight vectors becomes negative during internal 
    time-stepping. The \id{ILL\_INPUT} flag will also be returned if the linear 
    solver function initialization (called by the user after calling 
    \id{CVodeMalloc}) failed to set one of the linear solver-related fields 
    in \id{cvode\_mem} or if the linear solver's initialization function failed. 
    In any case, the user should see the printed error message for more details.
  \item[\Id{TOO\_MUCH\_WORK}] 
    The solver took \id{mxstep} internal steps but could not reach tout. 
    The default value for \id{mxstep} is \id{MXSTEP\_DEFAULT = 500}.
  \item[\Id{TOO\_MUCH\_ACC}] 
    The solver could not satisfy the accuracy demanded by the user for some 
    internal step.
  \item[\Id{ERR\_FAILURE}]
    Error test failures occurred too many times (\id{MXNEF = 7}) during one 
    internal time step or occurred with $|h| = h_{min}$.
  \item[\Id{CONV\_FAILURE}] 
    Convergence test failures occurred too many times (\id{MXNCF = 10}) during 
    one internal time step or occurred with $|h| = h_{min}$.             
  \item[\Id{SETUP\_FAILURE}] 
    The linear solver's setup function failed in an unrecoverable manner.
  \item[\Id{SOLVE\_FAILURE}] 
    The linear solver's solve function failed in an unrecoverable manner.
  \end{args} 
}
{
  The vector \id{yout} can occupy the same space as the \id{y0} vector of 
  initial conditions that was passed to \id{CVodeMalloc}. 

  In the \id{ONE\_STEP} mode, \id{tout} is used on the first call only, 
  to get the direction and rough scale of the independent variable.

  All failure return values are negative and therefore a test \id{ier}$< 0$
  will trap all \id{CVode} failures.
}

%%===================================================================================

\subsection{Optional input functions}\label{ss:optional_input}

{\cvodes} provides an extensive list of functions that can be used to change
from their default values various optional input parameters that control the
behavior of the {\cvodes} solver. 
Table \ref{t:optional_input} lists all optional input functions in {\cvodes} which 
are then described in detail in the remainder of this section.
For the most casual use of {\cvodes}, the reader can skip to \S\ref{ss:user_fct_sim}.

We note that, on error return, all these functions also print an error message to \id{stdout} 
(or to the file pointed to by \id{errfp} if already specified).
We also note that all error return values are negative, so a test \id{flag}$==0$
will catch any error.

\begin{table}
\centering
\caption{Optional inputs for {\cvodes}, {\cvdense}, {\cvband}, and {\cvspgmr}}
\label{t:optional_input}
\medskip
\begin{tabular}{|l|l|l|}\hline
{\bf Optional input} & {\bf Routine name} & {\bf Default} \\
\hline
\multicolumn{3}{|c|}{\bf CVODES main solver} \\
\hline
Data for right-hand side routine & \id{CVodeSetFdata} & NULL \\
Pointer to an error file & \id{CVodeSetErrFile} & NULL  \\
Maximum order for BDF method & \id{CVodeSetMaxOrd} & 5 \\
Maximum order for Adams method & \id{CVodeSetMaxOrd} & 12  \\
Maximum no. of internal steps before $t_{\mbox{\scriptsize out}}$ & \id{CVodeSetMaxNumSteps} & 500 \\
Maximum no. of warnings for $h < U$ & \id{CVodeSetMaxHnilWarns} & 10 \\
Flag to activate stability limit detection & \id{CVodeSetStabLimDet} & FALSE \\
Initial step size & \id{CVodeSetInitStep} & estimated \\
Minimum absolute step size & \id{CVodeSetMinStep} & 0.0 \\
Maximum absolute step size & \id{CVodeSetMaxStep} & $\infty$ \\
Value of $t_{stop}$ & \id{CVodeSetStopTime} & $\infty$ \\
Maximum no. of error test failures & \id{CVodeSetMaxErrTestFails} & 7 \\
Maximum no. of nonlinear iterations & \id{CVodeSetMaxNonlinIters} & 3 \\
Maximum no. of convergence failures & \id{CVodeSetMaxConvFails} & 10 \\
Coefficient in the nonlinear convergence test & \id{CVodeSetNonlinConvCoef} & 0.1 \\
Error control on quadrature variables & \id{CVodeSetQuadErrCon} & FULL \\
Data for quadrature right-hand side routine & \id{CVodeSetQuadFdata} & NULL\\
Sensitivity right-hand side routine & \id{CVodeSetSensRhsFn}  & internal DQ  \\
Data for sensitivity right-hand side routine & \id{CVodeSetSensFdata} & NULL \\
Error control on sensitivity variables & \id{CVodeSetSensErrCon} & FULL \\
Control for difference quotient approximation & \id{CVodeSetSensRho}   & 0.0 \\
Vector of problem parameter scalings & \id{CVodeSetSensPbar}  & NULL \\
Relative tolerance for sensitivity variables & \id{CVodeSetSensReltol} & estimated \\
Absolute tolerance for sensitivity variables & \id{CVodeSetSensAbstol} & estimated \\
\hline
\multicolumn{3}{|c|}{\bf CVDENSE linear solver} \\
\hline
Dense Jacobian routine & \id{CVDenseSetJacFn} & internal DQ \\
Data for Jacobian routine & \id{CVDenseSetJacData} & NULL \\
\hline
\multicolumn{3}{|c|}{\bf CVBAND linear solver} \\
\hline
Band Jacobian routine & \id{CVBandSetJacFn} & internal DQ \\
Data for Jacobian routine & \id{CVBandSetJacData} & NULL \\
\hline
\multicolumn{3}{|c|}{\bf CVSPGMR linear solver} \\
\hline
Type of Gram-Schmidt orthogonalization & \id{CVSpgmrSetGSType} & classical GS \\
Ratio between linear and nonlinear tolerances & \id{CVSpgmrSetDelt} & 0.05 \\
Preconditioner setup routine & \id{CVSpgmrSetPrecSetupFn} & NULL \\
Preconditioner solve routine & \id{CVSpgmrSetPrecSolveFn} & NULL \\
Data for preconditioner routines & \id{CVSpgmrSetPrecData} & NULL \\
Jacobian times vector routine & \id{CVSpgmrSetJacTimesVecFn} & NULL \\
Data for Jacobian times vector routine &\id{CVSpgmrSetJacData} & NULL \\ \hline
\end{tabular}
\end{table}

\subsubsection{Main solver optional input functions}
\index{optional input!solver|(}
The calls listed here can be executed in any order. However, if \id{CVodeSetErrFile} 
is to be called, that call should be first, in order to take effect for any later 
error message.

\ucfunction{CVodeSetErrFile}
{
flag = CVodeSetErrFile(cvode\_mem, errfp);
}
{
  The function \ID{CVodeSetErrFile} specifies the pointer to the file
  where all {\cvodes} messages should be directed.
}
{
  \begin{args}[cvode\_mem]
  \item[cvode\_mem] (\id{void *})
    pointer to the {\cvodes} memory block.
  \item[errfp] (\id{FILE *})
    pointer to output file.
  \end{args}
}
{
  The return value \id{flag} (of type \id{int}) is one of
  \begin{args}[CVS\_NO\_MEM]
  \item[\Id{SUCCESS}] 
    The optional value has been successfuly set.
  \item[\Id{CVS\_NO\_MEM}]
    The \id{cvode\_mem} pointer is \id{NULL}.
  \end{args}
}
{
  The default value for \id{errfp} is \id{NULL}, in which case
  all output is directed to \id{stdout}.
}
%%
%%
\ucfunction{CVodeResetIterType}
{
flag = CVodeResetIterType(cvode\_mem, iter);
}
{
  The function \ID{CVodeResetIterType} resets the nonlinear solver
  iteration type.
}
{
  \begin{args}[cvode\_mem]
  \item[cvode\_mem] (\id{void *})
    pointer to the {\cvodes} memory block.
  \item[iter] (\id{int})
    specifies the type of nonlinear solver iteration and may be
    either \Id{NEWTON} or \Id{FUNCTIONAL}. 
  \end{args}
}
{
  The return value \id{flag} (of type \id{int}) is one of
  \begin{args}[CVS\_ILL\_INPUT]
  \item[\Id{SUCCESS}] 
    The optional value has been successfuly set.
  \item[\Id{CVS\_NO\_MEM}]
    The \id{cvode\_mem} pointer is \id{NULL}.
  \item[\Id{CVS\_ILL\_INPUT}]
    The \id{iter} value passed is neither \Id{NEWTON} nor \Id{FUNCTIONAL}.
  \end{args}
}
{
  The nonlinear solver iteration type is initially specified in the call
  to \id{CVodeCreate} (see \S\ref{sss:cvodemalloc}). This function call is
  needed only if \id{iter} is being changed from its value in the prior call 
  to \id{CVodeCreate}.
}
%%
%%
\ucfunction{CVodeSetFdata}
{
  flag = CVodeSetFdata(cvode\_mem, f\_data);
}
{
  The function \ID{CVodeSetFdata} specifies the user data block
  and attaches it to the main {\cvodes} memory block.
}
{
  \begin{args}[cvode\_mem]
  \item[cvode\_mem] (\id{void *})
    pointer to the {\cvodes} memory block.
  \item[f\_data] (\id{void *})
    pointer to the user data.
  \end{args}
}
{
  The return value \id{flag} (of type \id{int}) is one of
  \begin{args}[CVS\_ILL\_INPUT]
  \item[\Id{SUCCESS}] 
    The optional value has been successfuly set.
  \item[\Id{CVS\_NO\_MEM}]
    The \id{cvode\_mem} pointer is \id{NULL}.
  \end{args}
}
{
  If \id{f\_data} is not specified, a \id{NULL} pointer is
  passed to all user functions that have it as an argument.
}
%%
%%
\ucfunction{CVodeSetMaxOrd}
{
flag = CVodeSetMaxOrder(cvode\_mem, maxord);
}
{
  The function \ID{CVodeSetMaxOrder} specifies the maximum order of the 
  linear multistep method.
}
{
  \begin{args}[cvode\_mem]
  \item[cvode\_mem] (\id{void *})
    pointer to the {\cvodes} memory block.
  \item[maxord] (\id{int})
    value of the maximum method order.
  \end{args}
}
{
  The return value \id{flag} (of type \id{int}) is one of
  \begin{args}[CVS\_ILL\_INPUT]
  \item[\Id{SUCCESS}] 
    The optional value has been successfuly set.
  \item[\Id{CVS\_NO\_MEM}]
    The \id{cvode\_mem} pointer is \id{NULL}.
  \item[\Id{CVS\_ILL\_INPUT}]
    The specified value \id{maxord} is negative, or larger than 
    its previous value.
  \end{args}
}
{
  The default value is \Id{ADAMS\_Q\_MAX}$= 12$ for
  the Adams-Moulton method and \Id{BDF\_Q\_Max}$= 5$
  for the BDF method.
  Since \id{maxord} affects the memory requirements
  for the internal {\cvodes} memory block, its value
  can not be increased past its previous value.
}
%%
%%
\ucfunction{CVodeSetMaxNumSteps}
{
flag = CVodeSetMaxNumSteps(cvode\_mem, mxsteps);
}
{
  The function \ID{CVodeSetMaxNumSteps} specifies the maximum number
  of steps to be taken by the solver in its attemt to reach 
  the final time.
}
{
  \begin{args}[cvode\_mem]
  \item[cvode\_mem] (\id{void *})
    pointer to the {\cvodes} memory block.
  \item[mxsteps] (\id{int})
    maximum allowed number of steps.
  \end{args}
}
{
  The return value \id{flag} (of type \id{int}) is one of
  \begin{args}[CVS\_ILL\_INPUT]
  \item[\Id{SUCCESS}] 
    The optional value has been successfuly set.
  \item[\Id{CVS\_NO\_MEM}]
    The \id{cvode\_mem} pointer is \id{NULL}.
  \item[\Id{CVS\_ILL\_INPUT}]
    \id{mxsteps} is non-positive.
  \end{args}
}
{
  The default value is $500$.
}
%%
%%
\ucfunction{CVodeSetMaxHnilWarns}
{
flag = CVodeSetMaxHnilWarns(cvode\_mem, mxhnil);
}
{
  The function \ID{CVodeSetMaxHnilWarns} specifies the maximum number of warning messages
  issued by the solver that $t+h=t$ on the next internal step.
}
{
  \begin{args}[cvode\_mem]
  \item[cvode\_mem] (\id{void *})
    pointer to the {\cvodes} memory block.
  \item[mxhnil] (\id{int})
    maximum number of warning messages
  \end{args}
}
{
  The return value \id{flag} (of type \id{int}) is one of
  \begin{args}[CVS\_NO\_MEM]
  \item[\Id{SUCCESS}] 
    The optional value has been successfuly set.
  \item[\Id{CVS\_NO\_MEM}]
    The \id{cvode\_mem} pointer is \id{NULL}.
  \end{args}
}
{
  The default value is $10$.
  A negative \id{mxhnil} value indicates that no warning messages should
  be issued.
}
%%
%%
\ucfunction{CVodeSetStabLimDet}
{
flag = CVodeSetstabLimDet(cvode\_mem, stldet);
}
{
  The function \ID{CVodeSetStabLimDet} indicates to turn on/off
  the BDF stability limit detection algorithm. See \S\ref{s:bdf_stab}.
}
{
  \begin{args}[cvode\_mem]
  \item[cvode\_mem] (\id{void *})
    pointer to the {\cvodes} memory block.
  \item[stldet] (\id{booleantype})
    flag to control stability limit detection (\id{TRUE} = on; \id{FALSE} = off).
  \end{args}
}
{
  The return value \id{flag} (of type \id{int}) is one of
  \begin{args}[CVS\_ILL\_INPUT]
  \item[\Id{SUCCESS}] 
    The optional value has been successfuly set.
  \item[\Id{CVS\_NO\_MEM}]
    The \id{cvode\_mem} pointer is \id{NULL}.
  \item[\Id{CVS\_ILL\_INPUT}]
    The linear multistep method is not set to \id{BDF}.
  \end{args}
}
{
  The default value is \id{FALSE}. If \id{stldet = TRUE}, when BDF is used
  and the method order is 3 or greater, an internal function, \id{CVsldet},
  is called to detect stability limit. If limit is detected, the order is reduced.
}
%%
%%
\ucfunction{CVodeSetInitStep}
{
flag = CVodeSetInitStep(cvode\_mem, hin);
}
{
  The function \ID{CVodeSetInitStep} specifies the initial step size.
}
{
  \begin{args}[cvode\_mem]
  \item[cvode\_mem] (\id{void *})
    pointer to the {\cvodes} memory block.
  \item[hin] (\id{realtype})
    value of the initial step size.
  \end{args}
}
{
  The return value \id{flag} (of type \id{int}) is one of
  \begin{args}[CVS\_NO\_MEM]
  \item[\Id{SUCCESS}] 
    The optional value has been successfuly set.
  \item[\Id{CVS\_NO\_MEM}]
    The \id{cvode\_mem} pointer is \id{NULL}.
  \end{args}
}
{
  By default, {\cvodes} estimates the inital step as solution of 
  $\| 0.5 h^2 \ddot y \|_{\mbox{\scriptsize WRMS}} = 1$,
  where $\ddot y$ is an estimated second derivative of the solution at the
  initial time.
}
%%
%%
\ucfunction{CVodeSetMinStep}
{
flag = CVodeSetMinStep(cvode\_mem, hmin);
}
{
  The function \ID{CVodeSetMinStep} specifies the minimum absolute
  value of the step size.
}
{
  \begin{args}[cvode\_mem]
  \item[cvode\_mem] (\id{void *})
    pointer to the {\cvodes} memory block.
  \item[hmin] (\id{realtype})
    minimum absolute value of the step size.
  \end{args}
}
{
  The return value \id{flag} (of type \id{int}) is one of
  \begin{args}[CVS\_ILL\_INPUT]
  \item[\Id{SUCCESS}] 
    The optional value has been successfuly set.
  \item[\Id{CVS\_NO\_MEM}]
    The \id{cvode\_mem} pointer is \id{NULL}.
  \item[\Id{CVS\_ILL\_INPUT}]
    Either \id{hmin} is not positive or it is larger than the maximum allowable step.
  \end{args}
}
{
  The default value is $0.0$.
}
%%
%%
\ucfunction{CVodeSetMaxStep}
{
flag = CVodeSetMaxStep(cvode\_mem, hmax);
}
{
  The function \ID{CVodeSetMaxStep} specifies the maximum absolute
  value of the step size.
}
{
  \begin{args}[cvode\_mem]
  \item[cvode\_mem] (\id{void *})
    pointer to the {\cvodes} memory block.
  \item[hmax] (\id{realtype})
    maximum absolute value of the step size.
  \end{args}
}
{
  The return value \id{flag} (of type \id{int}) is one of
  \begin{args}[CVS\_ILL\_INPUT]
  \item[\Id{SUCCESS}] 
    The optional value has been successfuly set.
  \item[\Id{CVS\_NO\_MEM}]
    The \id{cvode\_mem} pointer is \id{NULL}.
  \item[\Id{CVS\_ILL\_INPUT}]
    Either \id{hmax} is not positive or it is smaller than the minimum allowable step.
  \end{args}
}
{
  The default value is $\infty$.
}
%%
%%
\ucfunction{CVodeSetStopTime}
{
flag = CVodeSetStopTime(cvode\_mem, tstop);
}
{
  The function \ID{CVodeSetStopTime} specifies the value of the
  independent variable $t$ past which the solution is not to proceed.
}
{
  \begin{args}[cvode\_mem]
  \item[cvode\_mem] (\id{void *})
    pointer to the {\cvodes} memory block.
  \item[tstop] (\id{realtype})
    value of the independent variable past which the solution should
    not proceed.
  \end{args}
}
{
  The return value \id{flag} (of type \id{int}) is one of
  \begin{args}[CVS\_NO\_MEM]
  \item[\Id{SUCCESS}] 
    The optional value has been successfuly set.
  \item[\Id{CVS\_NO\_MEM}]
    The \id{cvode\_mem} pointer is \id{NULL}.
  \end{args}
}
{
  The default value is $\infty$.
}
%%
%%
\ucfunction{CVodeSetMaxErrTestFails}
{
flag = CVodeSetMaxErrTestFails(cvode\_mem, maxnef);
}
{
  The function \ID{CVodeSetMaxErrTestFails} specifies the
  maximum number of error test failures in attempting one step.
}
{
  \begin{args}[cvode\_mem]
  \item[cvode\_mem] (\id{void *})
    pointer to the {\cvodes} memory block.
  \item[maxnef] (\id{int})
    maximum number of error test failures allowed on one step.
  \end{args}
}
{
  The return value \id{flag} (of type \id{int}) is one of
  \begin{args}[CVS\_NO\_MEM]
  \item[\Id{SUCCESS}] 
    The optional value has been successfuly set.
  \item[\Id{CVS\_NO\_MEM}]
    The \id{cvode\_mem} pointer is \id{NULL}.
  \end{args}
}
{
  The default value is $7$.
}
%%
%%
\ucfunction{CVodeSetMaxNonlinIters}
{
flag = CVodeSetMaxNonlinIters(cvode\_mem, maxcor);
}
{
  The function \ID{CVodeSetNonlinIters} specifies the maximum
  number of nonlinear solver iterations at one step.
}
{
  \begin{args}[cvode\_mem]
  \item[cvode\_mem] (\id{void *})
    pointer to the {\cvodes} memory block.
  \item[maxcor] (\id{int})
    maximum number of nonlinear solver iterations allowed on one step.
  \end{args}
}
{
  The return value \id{flag} (of type \id{int}) is one of
  \begin{args}[CVS\_NO\_MEM]
  \item[\Id{SUCCESS}] 
    The optional value has been successfuly set.
  \item[\Id{CVS\_NO\_MEM}]
    The \id{cvode\_mem} pointer is \id{NULL}.
  \end{args}
}
{
  The default value is $3$.
}
%%
%%
\ucfunction{CVodeSetMaxConvFails}
{
flag = CVodeSetMaxConvFails(cvode\_mem, maxncf);
}
{
  The function \ID{CVodeSetMaxConvFails} specifies the
  maximum number of nonlinear solver convergence failures at one step.
}
{
  \begin{args}[cvode\_mem]
  \item[cvode\_mem] (\id{void *})
    pointer to the {\cvodes} memory block.
  \item[maxncf] (\id{int})
    maximum number of allowable nonlinear solver convergence failures
    on one step.
  \end{args}
}
{
  The return value \id{flag} (of type \id{int}) is one of
  \begin{args}[CVS\_NO\_MEM]
  \item[\Id{SUCCESS}] 
    The optional value has been successfuly set.
  \item[\Id{CVS\_NO\_MEM}]
    The \id{cvode\_mem} pointer is \id{NULL}.
  \end{args}
}
{
  The default value is $10$.
}
%%
%%
\ucfunction{CVodeSetNonlinConvCoef}
{
flag = CVodeSetNonlinConvCoef(cvode\_mem, nlscoef);
}
{
  The function \ID{CVodeSetNonlinConvCoef} specifies the safety factor
  in the nonlinear convergence test (see \S\ref{ss:ivp_sol}).
}
{
  \begin{args}[cvode\_mem]
  \item[cvode\_mem] (\id{void *})
    pointer to the {\cvodes} memory block.
  \item[nlscoef] (\id{realtype})
    coefficient in nonlinear convergence test.
  \end{args}
}
{
  The return value \id{flag} (of type \id{int}) is one of
  \begin{args}[CVS\_NO\_MEM]
  \item[\Id{SUCCESS}] 
    The optional value has been successfuly set.
  \item[\Id{CVS\_NO\_MEM}]
    The \id{cvode\_mem} pointer is \id{NULL}.
  \end{args}
}
{
  The default value is $0.1$.
}
%%
\index{optional input!solver|)}
%%
%%
%%===================================================================================
%%
\subsubsection{Linear solver optional input functions}
%%
The linear solver modules, with one exception, allow for various optional 
inputs, which are described here. The diagonal linear solver module has no
optional inputs.
%%
\paragraph\noindent{\bf Dense Linear solver.}
\index{optional input!dense linear solver|(}
\index{CVDENSE@{\cvdense} linear solver!optional input|(}
The \index{CVDENSE@{\cvdense} linear solver!Jacobian approximation used by}
{\cvdense} solver needs a function to compute a dense approximation to
the Jacobian matrix $J(t,y)$.  This function must be of type \id{CVDenseJacFn}. 
The user can supply his/her own dense Jacobian function, or use the default 
difference quotient function \Id{CVDenseDQJac} 
\index{Jacobian approximation function!dense!difference quotient}
that comes with the {\cvdense} solver.
To specify a user-supplied Jacobian function \id{djac} and associated user 
data \id{jac\_data}, {\cvdense} provides the functions \id{CVDenseSetJacFn}
and \id{CVDenseSetJacData}, respctively.
The {\cvdense} solver passes the pointer it receives through \id{CVDenseSetJacData} 
to its dense Jacobian function. This allows the user to
create an arbitrary structure with relevant problem data and access it
during the execution of the user-supplied Jacobian function, without
using global data in the program.  The pointer \id{jac\_data} may be
identical to \id{f\_data}, if the latter was specified through \id{CVodeSetFdata}.
%%
\index{Jacobian approximation function!dense!user-supplied}
\ucfunction{CVDenseSetJacFn}
{
  flag = CVDenseSetJacFn(cvode\_mem, CVDenseJacFn djac);
}
{
  The function \ID{CVDenseSetJacFn} specifies the dense Jacobian
  approximation function to be used.
}
{
  \begin{args}[cvode\_mem]
  \item[cvode\_mem] (\id{void *})
    pointer to the {\cvodes} memory block.
  \item[djac] (\id{CVDenseJacFn})
    user-defined dense Jacobian approximation function.
  \end{args}
}
{
  The return value \id{flag} (of type \id{int}) is one of
  \begin{args}[LIN\_NO\_LMEM]
  \item[\Id{SUCCESS}] 
    The optional value has been successfuly set.
  \item[\Id{LIN\_NO\_MEM}]
    The \id{cvode\_mem} pointer is \id{NULL}.
  \item[\Id{LIN\_NO\_LMEM}]
    The {\cvdense} linear solver has not been initialized.
  \end{args}
}
{
  By default, {\cvdense} uses the difference quotient function \id{CVDenseDQJac}.
  If \id{NULL} is passed to \id{djac}, this default function is used.

  The function type \id{CVDenseJacFn} is described in \S\ref{ss:djacFn}.
}
%%
\ucfunction{CVDenseSetJacData}
{
  flag = CVDenseSetJacData(cvode\_mem, void *jac\_data);
}
{
  The function \ID{CVDenseSetJacData} specifies the data structure
  to be passed to the user supplied dense Jacobian approximation 
  function each time it is called.
}
{
  \begin{args}[cvode\_mem]
  \item[cvode\_mem] (\id{void *})
    pointer to the {\cvodes} memory block.
  \item[jac\_data] (\id{void *})
    pointer to the user-defined data structure.
  \end{args}
}
{
  The return value \id{flag} (of type \id{int}) is one of
  \begin{args}[LIN\_NO\_LMEM]
  \item[\Id{SUCCESS}] 
    The optional value has been successfuly set.
  \item[\Id{LIN\_NO\_MEM}]
    The \id{cvode\_mem} pointer is \id{NULL}.
  \item[\Id{LIN\_NO\_LMEM}]
    The {\cvdense} linear solver has not been initialized.
  \end{args}
}
{}
\index{CVDENSE@{\cvdense} linear solver!optional input|)}
\index{optional input!dense linear solver|)}
%%
%%---------------------------------------------------------
%%
\paragraph\noindent{\bf Band Linear solver.}
\index{optional input!band linear solver|(}
\index{CVBAND@{\cvband} linear solver!optional input|(}
The \index{CVBAND@{\cvband} linear solver!Jacobian approximation used by}
{\cvdense} solver needs a function to compute a banded approximation to
the Jacobian matrix $J(t,y)$.  This function must be of type \id{CVBandJacFn}. 
The user can supply his/her own banded Jacobian approximation function, 
or use the default difference quotient function \Id{CVBandDQJac} 
\index{Jacobian approximation function!band!difference quotient}
that comes with the {\cvband} solver.
To specify a user-supplied Jacobian function \id{bjac} and associated user 
data \id{jac\_data}, {\cvband} provides the functions \id{CVBandSetJacFn}
and \id{CVBandSetJacData}, respctively.
The {\cvband} solver passes the pointer it receives through \id{CVBandSetJacData} 
to its banded Jacobian approximaton function. This allows the user to
create an arbitrary structure with relevant problem data and access it
during the execution of the user-supplied Jacobian function, without
using global data in the program.  The pointer \id{jac\_data} may be
identical to \id{f\_data}, if the latter was specified through \id{CVodeSetFdata}.
%%
\index{Jacobian approximation function!band!user-supplied}
\ucfunction{CVBandSetJacFn}
{
  flag = CVBandSetJacFn(cvode\_mem, CVBandJacFn bjac);
}
{
  The function \ID{CVBandSetJacFn} specifies the banded Jacobian
  approximation function to be used.
}
{
  \begin{args}[cvode\_mem]
  \item[cvode\_mem] (\id{void *})
    pointer to the {\cvodes} memory block.
  \item[bjac] (\id{CVBandJacFn})
    user-defined banded Jacobian approximation function.
  \end{args}
}
{
  The return value \id{flag} (of type \id{int}) is one of
  \begin{args}[LIN\_NO\_LMEM]
  \item[\Id{SUCCESS}] 
    The optional value has been successfuly set.
  \item[\Id{LIN\_NO\_MEM}]
    The \id{cvode\_mem} pointer is \id{NULL}.
  \item[\Id{LIN\_NO\_LMEM}]
    The {\cvband} linear solver has not been initialized.
  \end{args}
}
{
  By default, {\cvband} uses the difference quotient function \id{CVBandDQJac}.
  If \id{NULL} is passed to \id{bjac}, this default function is used.

  The function type \id{CVBandJacFn} is described in \S\ref{ss:bjacFn}.
}
%%
\ucfunction{CVBandSetJacData}
{
  flag = CVBandSetJacData(cvode\_mem, void *jac\_data);
}
{
  The function \ID{CVBandSetJacData} specifies the data structure
  to be passed to the user supplied banded Jacobian approximation 
  function each time it is called.
}
{
  \begin{args}[cvode\_mem]
  \item[cvode\_mem] (\id{void *})
    pointer to the {\cvodes} memory block.
  \item[jac\_data] (\id{void *})
    pointer to the user-defined data structure.
  \end{args}
}
{
  The return value \id{flag} (of type \id{int}) is one of
  \begin{args}[LIN\_NO\_LMEM]
  \item[\Id{SUCCESS}] 
    The optional value has been successfuly set.
  \item[\Id{LIN\_NO\_MEM}]
    The \id{cvode\_mem} pointer is \id{NULL}.
  \item[\Id{LIN\_NO\_LMEM}]
    The {\cvdense} linear solver has not been initialized.
  \end{args}
}
{}
\index{CVBAND@{\cvband} linear solver!optional input|)}
\index{optional input!band linear solver|)}
%%
%%---------------------------------------------------------
%%
\paragraph\noindent{\bf SPGMR Linear solver.}
\index{optional input!iterative linear solver|(}
\index{CVSPGMR@{\cvspgmr} linear solver!optional input|(}
The call to \id{CVSpgmr} is used to communicate the type of
preconditioning (\id{pretype}) and the maximum dimension of the Krylov 
subspace to be used (\id{maxl}).
The \id{pretype} parameter can be \id{NONE}, \id{LEFT}, \id{RIGHT}, or \id{BOTH}.
If no preconditioning is desired, pass \id{NONE} for \id{pretype}.
Otherwise, a preconditioner solve function \id{psolve} is required.  
Regardless of the type of preconditioning, a preconditioner setup function 
\id{psetup} is sometimes useful, but is {\em not} required. 

If any type of preconditioning is to be done within the {\spgmr} method,
then the user must supply a preconditioner solve function \id{psolve}
and specify it through a call to \id{CVSpgmrSetPrecSolveFn}.
\index{CVSPGMR@{\cvspgmr} linear solver!preconditioner solve function}
%%
The evaluation and preprocessing of any Jacobian-related data needed
by the user's preconditioner solve function is done in the optional
user-supplied function \id{psetup}. Both of these functions are
fully specified in \S\ref{ss:user_fct_sim}.
If used, the \id{psetup} function should be specified through a call to
\id{CVSpgmrSetPrecSetupFn}.
\index{CVSPGMR@{\cvspgmr} linear solver!preconditioner setup function}
%%
Optionally, the {\cvspgmr} solver passes the pointer it receives through 
\id{CVSpgmrSetPrecData} to the preconditioner setup and solve functions.  
This allows the user to create an arbitrary structure with relevant problem data 
and access it during the execution of the user-supplied preconditioner functions
without using global data in the program.  
The pointer \id{P\_data} may be identical to \id{f\_data}, if the latter was 
specified through \id{CVodeSetFdata}.

The \index{CVSPGMR@{\cvspgmr} linear solver!Jacobian approximation used by}
{\cvspgmr} solver requires a function to compute an approximation to the
product between the Jacobian matrix $J(t,y)$ and a vector $v$.
The user can supply his/her own Jacobian times vector approximation function, 
or use the difference quotient function \Id{CVSpgmrDQJtimes} 
\index{Jacobian approximation function!Jacobian times vector!difference quotient}
that comes with the {\cvspgmr} solver.
A user-defined Jacobian-vector function must be of type \id{CVSpgmrJtimesFn} and 
can be specified through a call to \id{CVSpgmrSetJacTimesVecFn} 
(see \S\ref{ss:user_fct_sim} for specification details).
%%
As with the preconditioner user data structure \id{P\_data}, 
the user can specify, through a call to \id{CVSpgmrSetJacData}, a pointer to a 
user-defined data structure, \id{jac\_data}, which
the {\cvspgmr} solver passes to the Jacobian times vector function \id{jtimes} each
time it is called.  
The pointer \id{jac\_data} may be identical to \id{P\_data} and/or \id{f\_data}.
%%
%%
\ucfunction{CVSpgmrSetPrecSolveFn}
{
  flag = CVSpgmrSetPrecSolveFn(cvode\_mem, CVSpgmrPrecSolveFn psolve);
}
{
  The function \ID{CVSpgmrSet} specifies the preconditioner
  solve function.
}
{
  \begin{args}[cvode\_mem]
  \item[cvode\_mem] (\id{void *})
    pointer to the {\cvodes} memory block.
  \item[psolve] (\id{CVSpgmrPrecSolveFn})
    user-defined preconditioner solve function.
  \end{args}
}
{
  The return value \id{flag} (of type \id{int}) is one of
  \begin{args}[LIN\_NO\_LMEM]
  \item[\Id{SUCCESS}] 
    The optional value has been successfuly set.
  \item[\Id{LIN\_NO\_MEM}]
    The \id{cvode\_mem} pointer is \id{NULL}.
  \item[\Id{LIN\_NO\_LMEM}]
    The {\cvspgmr} linear solver has not been initialized.
  \end{args}
}
{
   The function type \id{CVSpgmrPrecSolveFn} is described in \S\ref{ss:psolveFn}.
}
%%
%%
\ucfunction{CVSpgmrSetPrecSetupFn}
{
  flag = CVSpgmrSetPrecSetupFn(cvode\_mem, CVSpgmrPrecSetupFn psetup);
}
{
  The function \ID{CVSpgmrSetPrecSetupFn} specifies the preconditioner
  preprocessing function.
}
{
  \begin{args}[cvode\_mem]
  \item[cvode\_mem] (\id{void *})
    pointer to the {\cvodes} memory block.
  \item[psetup] (\id{CVSpgmrPrecSetupFn})
    user-defined preconditioner setup function.
  \end{args}
}
{
  The return value \id{flag} (of type \id{int}) is one of
  \begin{args}[LIN\_NO\_LMEM]
  \item[\Id{SUCCESS}] 
    The optional value has been successfuly set.
  \item[\Id{LIN\_NO\_MEM}]
    The \id{cvode\_mem} pointer is \id{NULL}.
  \item[\Id{LIN\_NO\_LMEM}]
    The {\cvspgmr} linear solver has not been initialized.
  \end{args}
}
{
   The function type \id{CVSpgmrPrecSetupFn} is described in \S\ref{ss:precondFn}.
}
%%
%%
\ucfunction{CVSpgmrSetPrecData}
{
  flag = CVSpgmrSetJacData(cvode\_mem, void *P\_data);
}
{
  The function \ID{CVSpgmrSetPrecData} specifies the data structure
  to be passed to the user supplied preconditioner setup and solve
  functions each time they are called.
}
{
  \begin{args}[cvode\_mem]
  \item[cvode\_mem] (\id{void *})
    pointer to the {\cvodes} memory block.
  \item[P\_data] (\id{void *})
     pointer to the user-defined data structure.
  \end{args}
}
{
  The return value \id{flag} (of type \id{int}) is one of
  \begin{args}[LIN\_NO\_LMEM]
  \item[\Id{SUCCESS}] 
    The optional value has been successfuly set.
  \item[\Id{LIN\_NO\_MEM}]
    The \id{cvode\_mem} pointer is \id{NULL}.
  \item[\Id{LIN\_NO\_LMEM}]
    The {\cvspgmr} linear solver has not been initialized.
  \end{args}
}
{}
%%
%%
\index{Jacobian approximation function!Jacobian times vector!user-supplied}
\ucfunction{CVSpgmrSetJacTimesVecFn}
{
  flag = CVSpgmrSetJacTimesVecFn(cvode\_mem, CVSpgmrJacTimesVecFn jtimes);
}
{
  The function \ID{CVSpgmrSetJacTimesFn} specifies the Jacobian-vector 
  function to be used.
}
{
  \begin{args}[cvode\_mem]
  \item[cvode\_mem] (\id{void *})
    pointer to the {\cvodes} memory block.
  \item[jtimes] (\id{CVSpgmrJacTimesVecFn})
    user-defined Jacobian-vector product function.
  \end{args}
}
{
  The return value \id{flag} (of type \id{int}) is one of
  \begin{args}[LIN\_NO\_LMEM]
  \item[\Id{SUCCESS}] 
    The optional value has been successfuly set.
  \item[\Id{LIN\_NO\_MEM}]
    The \id{cvode\_mem} pointer is \id{NULL}.
  \item[\Id{LIN\_NO\_LMEM}]
    The {\cvspgmr} linear solver has not been initialized.
  \end{args}
}
{
  By default, {\cvspgmr} uses the difference quotient function \id{CVSpgmrDQJtimes}.
  If \id{NULL} is passed to \id{jtimes}, this default function is used.

  The function type \id{CVSpgmrJacTimesVecFn} is described in \S\ref{ss:jtimesFn}.
}
%%
%%
\ucfunction{CVSpgmrSetJacData}
{
  flag = CVSpgmrSetJacData(cvode\_mem, void *jac\_data);
}
{
  The function \ID{CVSpgmrSetJacData} specifies the data structure
  to be passed to the user supplied Jacobian-vector
  function each time it is called.
}
{
  \begin{args}[cvode\_mem]
  \item[cvode\_mem] (\id{void *})
    pointer to the {\cvodes} memory block.
  \item[jac\_data] (\id{void *})
     pointer to the user-defined data structure.
  \end{args}
}
{
  The return value \id{flag} (of type \id{int}) is one of
  \begin{args}[LIN\_NO\_LMEM]
  \item[\Id{SUCCESS}] 
    The optional value has been successfuly set.
  \item[\Id{LIN\_NO\_MEM}]
    The \id{cvode\_mem} pointer is \id{NULL}.
  \item[\Id{LIN\_NO\_LMEM}]
    The {\cvspgmr} linear solver has not been initialized.
  \end{args}
}
{}
%%
%%
\ucfunction{CVSpgmrSetGSType}
{
  flag = CVSpgmrSetGSType(cvode\_mem, int gstype);
}
{
  The function \ID{CVSpgmrSetGSType} specifies the 
  Gram-Schmidt orthogonalization to be used. 
  This must be one of the enumeration constants \ID{MODIFIED\_GS}
  or \ID{CLASSICAL\_GS}. These correspond to using modified Gram-Schmidt 
  and classical Gram-Schmidt, respectively. 
  \index{Gram-Schmidt procedure}
}
{
  \begin{args}[cvode\_mem]
  \item[cvode\_mem] (\id{void *})
    pointer to the {\cvodes} memory block.
  \item[gstype] (\id{int})
    type of Gram-Schmidt orthogonalization.
  \end{args}
}
{
  The return value \id{flag} (of type \id{int}) is one of
  \begin{args}[LIN\_ILL\_INPUT]
  \item[\Id{SUCCESS}] 
    The optional value has been successfuly set.
  \item[\Id{LIN\_NO\_MEM}]
    The \id{cvode\_mem} pointer is \id{NULL}.
  \item[\Id{LIN\_NO\_LMEM}]
    The {\cvspgmr} linear solver has not been initialized.
  \item[\Id{LIN\_ILL\_INPUT}]
    The Gram-Schmidt orthogonalization type \id{gstype} is not valid.
  \end{args}
}
{
  The default value is \id{MODIFIED\_GS}.
}
%%
%%
\ucfunction{CVSpgmrSetDelt}
{
  flag = CVSpgmrSetDelt(cvode\_mem, realtype delt);
}
{
  The function \ID{CVSpgmrSetDelt} specifies the factor by
  which the GMRES\index{GMRES method} convergence test constant is reduced
  from the Newton iteration test constant.
}
{
  \begin{args}[cvode\_mem]
  \item[cvode\_mem] (\id{void *})
    pointer to the {\cvodes} memory block.
  \item[delt] (\id{realtype})

  \end{args}
}
{
  The return value \id{flag} (of type \id{int}) is one of
  \begin{args}[LIN\_NO\_LMEM]
  \item[\Id{SUCCESS}] 
    The optional value has been successfuly set.
  \item[\Id{LIN\_NO\_MEM}]
    The \id{cvode\_mem} pointer is \id{NULL}.
  \item[\Id{LIN\_NO\_LMEM}]
    The {\cvspgmr} linear solver has not been initialized.
  \item[\Id{LIN\_ILL\_INPUT}]
    The factor \id{delt} is negative.  
  \end{args}
}
{
  The default value is $0.05$.

  Passing a value \id{delt}$ = 0.0$ also indicates using the default value.
}
%%
%%
\ucfunction{CVSpgmrResetPrecType}
{
  flag = CVSpgmrResetPrecType(cvode\_mem, int pretype);
}
{
  The function \ID{CVSpgmrResetPrecType} resets the type
  of preconditioning to be used.
}
{
  \begin{args}[cvode\_mem]
  \item[cvode\_mem] (\id{void *})
    pointer to the {\cvodes} memory block.
  \item[pretype] (\id{int})
    specifies the type of prconditioning and may be
    \Id{NONE}, \Id{LEFT}, \Id{RIGHT}, or \Id{BOTH}.
  \end{args}
}
{
  The return value \id{flag} (of type \id{int}) is one of
  \begin{args}[LIN\_ILL\_INPUT]
  \item[\Id{SUCCESS}] 
    The optional value has been successfuly set.
  \item[\Id{LIN\_NO\_MEM}]
    The \id{cvode\_mem} pointer is \id{NULL}.
  \item[\Id{LIN\_NO\_LMEM}]
    The {\cvspgmr} linear solver has not been initialized.
  \item[\Id{LIN\_ILL\_INPUT}]
    The preconditioner type \id{pretype} is not valid.
  \end{args}
}
{
  The preconditioning type is initially specified in the call
  to \id{CVSpgmr} (see \S\ref{sss:lin_solv_init}). This function call is
  needed only if \id{pretype} is being changed from its value in the
  previous call to \id{CVSpgmr}.
}
%%
\index{CVSPGMR@{\cvspgmr} linear solver!optional input|)}
\index{optional input!iterative linear solver|)}

%%
%%===================================================================================
%%

\subsection{Interpolated output function}\label{ss:optional_dky}
\index{optional output!interpolated solution}

An optional function \ID{CVodeGetDky} is available to obtain additional output values.  
This function must be called after a successful return from \id{CVode} and provides 
interpolated values of $y$ or its derivatives, up to the current order of the 
integration method, interpolated to any value of $t$ in the last internal step 
taken by {\cvodes}.

The call to the \id{CVodeGetDky} function has the following form:
%%
\ucfunction{CVodeGetDky}
{
  flag = CVodeDky(cvode\_mem, t, k, dky);
}
{
  Computes the \id{k}-th derivative of the \id{y} function at      
  time \id{t}, i.e. $d^{(k)}y/dt^{(k)} (t)$, where $t_n - h_u \le$ \id{t} $\le t_n$, 
  $t_n$ denotes the current internal time reached, and $h_u$ is the 
  last internal step size successfully used by the solver. 
  The user may request \id{k} $= 0, 1, ..., q_u$, where $q_u$ is the 
  current order. 
}
{
  \begin{args}[cvode\_mem]
  \item[cvode\_mem] (\id{void *})
    pointer to the {\cvodes} memory block.
  \item[t] (\id{realtype})
  \item[k] (\id{int})
  \item[dky] (\id{N\_Vector})
    The derivative vector is returned in \id{dky}. 
    This vector must be allocated by the caller. 
  \end{args}
}
{
  The return value \id{flag} (of type \id{int}) is one of
  \begin{args}[DKY\_NO\_MEM] 
  \item[\Id{OKAY}]
    \id{CVodeDky} succeeded;
  \item[\Id{BAD\_K}] 
    \id{k} is not in the range $0, 1, ..., q_u$;
  \item[\Id{BAD\_T}] 
    \id{t} is not in the interval $[t_n - h_u , t_n]$;
  \item[\Id{BAD\_DKY}] 
    the \id{dky} argument was \id{NULL};
  \item[\Id{DKY\_NO\_MEM}] 
    the \id{cvode\_mem} argument was \id{NULL}.
  \end{args}

}
{
  It is only legal to call the function \id{CVodeDky} after a 
  successful return from \id{CVode}. See \id{CVodeGetLastOrder} 
  and \id{CVodeGetLastStep} in the next section for access to 
  $q_u$ and $h_u$.
}

%%
%%===================================================================================
%%

\subsection{Optional output functions}\label{ss:optional_output}

{\cvodes} provides an extensive list of functions that can be used to obtain
solver performance statistics.
Table \ref{t:optional_output} lists all optional output functions in {\cvodes} which 
are then described in detail in the remainder of this section.

\begin{table}
\centering
\caption{Optional outputs from {\cvodes}, {\cvdense}, {\cvband}, {\cvdiag}, and {\cvspgmr}}
\label{t:optional_output}
\medskip
\begin{tabular}{|l|l|}\hline
{\bf Optional output} & {\bf Routine name} \\
\hline
\multicolumn{2}{|c|}{\bf CVODES main solver} \\
\hline
Size of {\cvodes} integer workspace & \id{CVodeGetIntWorkSpace} \\
Size of {\cvodes} real workspace & \id{CVodeGetRealWorkSpace} \\
Cumulative number of internal steps & \id{CVodeGetNumSteps} \\
No. of calls to r.h.s. function & \id{CVodeGetNumRhsEvals} \\
No. of calls to linear solver setup routine & \id{CVodeGetNumLinSolvSetups} \\
No. of local error test failures that have occurred & \id{CVodeGetNumErrTestFails} \\
Order used during the last step & \id{CVodeGetLastOrder} \\
Order to be attempted on the next step & \id{CVodeGetCurrentOrder} \\
Order reductions due to stability limit detection & \id{CVodeGetNumStabLimOrderReds} \\
Actual initial step size used & \id{CVodeGetActualInitStep} \\
Step size used for the last step & \id{CVodeGetLastStep} \\
Step size to be attempted on the next step & \id{CVodeGetCurrentStep} \\
Current internal time reached by the solver & \id{CVodeGetCurrentTime} \\
Suggested factor for tolerance scaling  & \id{CVodeGetTolScaleFactor} \\
Error weight vector for state variables & \id{CVodeGetErrWeights} \\
Estimated local error vector & \id{CVodeGetEstLocalErrors} \\
No. of nonlinear solver iterations & \id{CVodeGetNumNonlinSolvIters} \\
No. of nonlinear convergence failures & \id{CVodeGetNumNonlinSolvConvFails} \\ 
\hline
\multicolumn{2}{|c|}{\bf CVDENSE linear solver} \\
\hline
Size of {\cvdense} integer workspace & \id{CVDenseGetIntWorkSpace} \\
Size of {\cvdense} real workspace & \id{CVDenseGetRealWorkSpace} \\
No. of Jacobian evaluations & \id{CVDenseGetNumJacEvals} \\
No. of r.h.s. calls for finite diff. Jacobian evals. & \id{CVDenseGetNumRhsEvals} \\ 
\hline
\multicolumn{2}{|c|}{\bf CVBAND linear solver} \\
\hline
Size of {\cvband} integer workspace & \id{CVBandGetIntWorkSpace} \\
Size of {\cvband} real workspace & \id{CVBandGetRealWorkSpace} \\
No. of Jacobian evaluations & \id{CVBandGetNumJacEvals} \\
No. of r.h.s. calls for finite diff. Jacobian evals. & \id{CVBandGetNumRhsEvals} \\ 
\hline
\multicolumn{2}{|c|}{\bf CVDIAG linear solver} \\
\hline
Size of {\cvdiag} integer workspace & \id{CVDiagGetIntWorkSpace} \\
Size of {\cvdiag} real workspace & \id{CVDiagGetRealWorkSpace} \\
No. of r.h.s. calls for finite diff. Jacobian evals. & \id{CVDiagGetNumRhsEvals} \\ 
\hline
\multicolumn{2}{|c|}{\bf CVSPGMR linear solver} \\
\hline
Size of {\cvspgmr} integer workspace & \id{CVSpgmrGetIntWorkSpace} \\
Size of {\cvspgmr} real workspace & \id{CVSpgmrGetRealWorkSpace} \\
No. of linear iterations & \id{CVSpgmrGetNumLinIters} \\
No. of linear convergence failures & \id{CVSpgmrGetNumConvFails} \\
No. of preconditioner evaluations & \id{CVSpgmrGetNumPrecEvals} \\
No. of preconditioner solves & \id{CVSpgmrGetNumPrecSolves} \\
No. of Jacobian-vector product evaluations & \id{CVSpgmrGetNumJtimesEvals} \\
No. of r.h.s. calls for finite diff. Jacobian-vector evals. & \id{CVSpgmrGetNumRhsEvals} \\ 
\hline
\end{tabular}
\end{table}


\subsubsection{Main solver optional output functions}
\index{optional output!solver|(}
%%
{\cvodes} provides several user-callable functions that can be used to obtain
different quantities that may be of interest to the user, such as solver workspace
requirements, solver performance statistics, as well as additional data from
the {\cvodes} memory block (a suggested tolerance scaling factor, the error weight
vector, and the vector of estimated local errors). Also provided are functions to
extract statistics related to the performance of the {\cvodes} nonlinear solver
being used. As a convenience, additional extraction functions provide the optional 
outputs in groups.
%%
These optional output functions are described next.
%%
\index{memory requirements!CVODES@{\cvodes} solver|(}
%%
\ucfunction{CVodeGetIntWorkSpace}
{
  flag = CVodeGetIntWorkSpace(cvode\_mem, leniw);
}
{
  The function \ID{CVodeGetIntWorkSpace} returns the
  {\cvodes} integer workspace size.
}
{
  \begin{args}[cvode\_mem]
  \item[cvode\_mem] (\id{void *})
    pointer to the {\cvodes} memory block.
  \item[leniw] (\id{long int *})
    number of \id{integertype} values in the {\cvodes} workspace.
  \end{args}
}
{
  The return value \id{flag} (of type \id{int}) is one of
  \begin{args}[CVG\_NO\_MEM]
  \item[OKAY] 
    The optional output value has been successfuly set.
  \item[\Id{CVG\_NO\_MEM}]
    The \id{cvode\_mem} pointer is \id{NULL}.
  \end{args}
}
{}
%%
\ucfunction{CVodeGetRealWorkSpace}
{
  flag = CVodeGetRealWorkSpace(cvode\_mem, lenrw);
}
{
  The function \ID{CVodeGetRealWorkSpace} returns the
  {\cvodes} real workspace size.
}
{
  \begin{args}[cvode\_mem]
  \item[cvode\_mem] (\id{void *})
    pointer to the {\cvodes} memory block.
  \item[lenrw] (\id{long int *})
    number of \id{realtype} values in the {\cvodes} workspace.
  \end{args}
}
{
  The return value \id{flag} (of type \id{int}) is one of
  \begin{args}[CVG\_NO\_MEM]
  \item[OKAY] 
    The optional output value has been successfuly set.
  \item[\Id{CVG\_NO\_MEM}]
    The \id{cvode\_mem} pointer is \id{NULL}.
  \end{args}
}
{}
%%
%%
\ucfunction{CVodeGetWorkSpace}
{
  flag = CVodeGetWorkSpace(cvode\_mem, leniw, lenrw);
}
{
  The function \ID{CVodeGetWorkSpace} returns the
  {\cvodes} integer and real workspace sizes.
}
{
  \begin{args}[cvode\_mem]
  \item[cvode\_mem] (\id{void *})
    pointer to the {\cvodes} memory block.
  \item[leniw] (\id{long int *})
    number of \id{realtype} values in the {\cvodes} workspace.
  \item[lenrw] (\id{long int *})
    number of \id{realtype} values in the {\cvodes} workspace.
  \end{args}
}
{
  The return value \id{flag} (of type \id{int}) is one of
  \begin{args}[CVG\_NO\_MEM]
  \item[OKAY] 
    The optional output value has been successfuly set.
  \item[\Id{CVG\_NO\_MEM}]
    The \id{cvode\_mem} pointer is \id{NULL}.
  \end{args}
}
{}
%%
\index{memory requirements!CVODES@{\cvodes} solver|)}
%%
\ucfunction{CVodeGetNumSteps}
{
  flag = CVodeGetNumSteps(cvode\_mem, nsteps);
}
{
  The function \ID{CVodeGetNumSteps} returns the cumulative number of internal 
  steps taken by the solver (total so far).
}
{
  \begin{args}[cvode\_mem]
  \item[cvode\_mem] (\id{void *})
    pointer to the {\cvodes} memory block.
  \item[nsteps] (\id{int *})
    number of steps taken by {\cvodes}.
  \end{args}
}
{
  The return value \id{flag} (of type \id{int}) is one of
  \begin{args}[CVG\_NO\_MEM]
  \item[OKAY] 
    The optional output value has been successfuly set.
  \item[\Id{CVG\_NO\_MEM}]
    The \id{cvode\_mem} pointer is \id{NULL}.
  \end{args}
}
{}
%%
%%
\ucfunction{CVodeGetNumRhsEvals}
{
  flag = CVodeGetNumRhsEvals(cvode\_mem, nfevals);
}
{
  The function \ID{CVodeGetNumRhsEvals} returns the 
  number of calls to the user's right-hand side evaluation function.
}
{
  \begin{args}[cvode\_mem]
  \item[cvode\_mem] (\id{void *})
    pointer to the {\cvodes} memory block.
  \item[nfevals] (\id{int *})
    number of calls to the user's \id{f} function.
  \end{args}
}
{
  The return value \id{flag} (of type \id{int}) is one of
  \begin{args}[CVG\_NO\_MEM]
  \item[OKAY] 
    The optional output value has been successfuly set.
  \item[\Id{CVG\_NO\_MEM}]
    The \id{cvode\_mem} pointer is \id{NULL}.
  \end{args}
}
{
  The \id{nfevals} value returned by \id{CVodeGetNumRhsEvals} does not
  account for calls made to \id{f} from a linear solver or preconditioner 
  module. 
}
%%
%%
\ucfunction{CVodeGetNumLinSolvSetups}
{
  flag = CVodeGetNumLinSolvSetups(cvode\_mem, nlinsetups);
}
{
  The function \ID{CVodeGetNumLinSolvSetups} returns the
  number of calls made to the linear solver's setup function.
}
{
  \begin{args}[nlinsetups]
  \item[cvode\_mem] (\id{void *})
    pointer to the {\cvodes} memory block.
  \item[nlinsetups] (\id{int *})
    number of calls made to the linear solver setup routine.
  \end{args}
}
{
  The return value \id{flag} (of type \id{int}) is one of
  \begin{args}[CVG\_NO\_MEM]
  \item[OKAY] 
    The optional output value has been successfuly set.
  \item[\Id{CVG\_NO\_MEM}]
    The \id{cvode\_mem} pointer is \id{NULL}.
  \end{args}
}
{}
%%
%%
\ucfunction{CVodeGetNumErrTestFails}
{
  flag = CVodeGetNumErrTestFails(cvode\_mem, netfails);
}
{
  The function \ID{CVodeGetNumErrTestFails} returns the
  number of local error test failures that have occurred.
}
{
  \begin{args}[cvode\_mem]
  \item[cvode\_mem] (\id{void *})
    pointer to the {\cvodes} memory block.
  \item[netfails] (\id{int *})
    number of error test failures.
  \end{args}
}
{
  The return value \id{flag} (of type \id{int}) is one of
  \begin{args}[CVG\_NO\_MEM]
  \item[OKAY] 
    The optional output value has been successfuly set.
  \item[\Id{CVG\_NO\_MEM}]
    The \id{cvode\_mem} pointer is \id{NULL}.
  \end{args}
}
{}
%%
%%
\ucfunction{CVodeGetLastOrder}
{
  flag = CVodeGetLastOrder(cvode\_mem, qlast);
}
{
  The function \ID{CVodeGetLastOrder} returns the
  integration method order used during the last internal step.
}
{
  \begin{args}[cvode\_mem]
  \item[cvode\_mem] (\id{void *})
    pointer to the {\cvodes} memory block.
  \item[qlast] (\id{int *})
    method order used on the last internal step.
  \end{args}
}
{
  The return value \id{flag} (of type \id{int}) is one of
  \begin{args}[CVG\_NO\_MEM]
  \item[OKAY] 
    The optional output value has been successfuly set.
  \item[\Id{CVG\_NO\_MEM}]
    The \id{cvode\_mem} pointer is \id{NULL}.
  \end{args}
}
{}
%%
%%
\ucfunction{CVodeGetCurrentOrder}
{
  flag = CVodeGetCurrentOrder(cvode\_mem, qcur);
}
{
  The function \ID{CVodeGetCurrentOrder} returns the
  integration method order to be used on the next internal step.
}
{
  \begin{args}[cvode\_mem]
  \item[cvode\_mem] (\id{void *})
    pointer to the {\cvodes} memory block.
  \item[qcur] (\id{int *})
    method order to be used on the next internal step.
  \end{args}
}
{
  The return value \id{flag} (of type \id{int}) is one of
  \begin{args}[CVG\_NO\_MEM]
  \item[OKAY] 
    The optional output value has been successfuly set.
  \item[\Id{CVG\_NO\_MEM}]
    The \id{cvode\_mem} pointer is \id{NULL}.
  \end{args}
}
{}
%%
%%
\ucfunction{CVodeGetLastStep}
{
  flag = CVodeGetLastStep(cvode\_mem, hlast);
}
{
  The function \ID{CVodeGetLastStep} returns the
  integration step size taken on the last internal step.
}
{
  \begin{args}[cvode\_mem]
  \item[cvode\_mem] (\id{void *})
    pointer to the {\cvodes} memory block.
  \item[hlast] (\id{realtype *})
    step size taken on the last internal step.
  \end{args}
}
{
  The return value \id{flag} (of type \id{int}) is one of
  \begin{args}[CVG\_NO\_MEM]
  \item[OKAY] 
    The optional output value has been successfuly set.
  \item[\Id{CVG\_NO\_MEM}]
    The \id{cvode\_mem} pointer is \id{NULL}.
  \end{args}
}
{}
%%
%%
\ucfunction{CVodeGetCurrentStep}
{
  flag = CVodeGetCurrentStep(cvode\_mem, hcur);
}
{
  The function \ID{CVodeGetCurrentStep} returns the
  integration step size to be attempted on the next internal step.
}
{
  \begin{args}[cvode\_mem]
  \item[cvode\_mem] (\id{void *})
    pointer to the {\cvodes} memory block.
  \item[hcur] (\id{realtype 8})
    step size to be attempted on the next internal step.
  \end{args}
}
{
  The return value \id{flag} (of type \id{int}) is one of
  \begin{args}[CVG\_NO\_MEM]
  \item[OKAY] 
    The optional output value has been successfuly set.
  \item[\Id{CVG\_NO\_MEM}]
    The \id{cvode\_mem} pointer is \id{NULL}.
  \end{args}
}
{}
%%
%%
\ucfunction{CVodeGetActualInitStep}
{
  flag = CVodeGetActualInitStep(cvode\_mem, hinused);
}
{
  The function \ID{CVodeGetActualInitStep} returns the
  value of the integration step size used on the first step.
}
{
  \begin{args}[cvode\_mem]
  \item[cvode\_mem] (\id{void *})
    pointer to the {\cvodes} memory block.
  \item[hinused] (\id{realtype *})
    actual value of initial step size.
  \end{args}
}
{
  The return value \id{flag} (of type \id{int}) is one of
  \begin{args}[CVG\_NO\_MEM]
  \item[OKAY] 
    The optional output value has been successfuly set.
  \item[\Id{CVG\_NO\_MEM}]
    The \id{cvode\_mem} pointer is \id{NULL}.
  \end{args}
}
{
  Even if the value of the initial integration step size was specified
  by the user through a call to \id{CVodeSetInitStep}, this value might have 
  been changed by {\cvodes} to ensure that the step size is within the 
  prescribed bounds ($h_{\min} \le h_0 \le h_{\max}$), or to meet the
  local error test.
}
%%
%%
\ucfunction{CVodeGetCurrentTime}
{
  flag = CVodeGetCurrentTime(cvode\_mem, tcur);
}
{
  The function \ID{CVodeGetCurrentTime} returns the
  current internal time reached by the solver.
}
{
  \begin{args}[cvode\_mem]
  \item[cvode\_mem] (\id{void *})
    pointer to the {\cvodes} memory block.
  \item[tcur] (\id{realtype *})
    current internal time reached.
  \end{args}
}
{
  The return value \id{flag} (of type \id{int}) is one of
  \begin{args}[CVG\_NO\_MEM]
  \item[OKAY] 
    The optional output value has been successfuly set.
  \item[\Id{CVG\_NO\_MEM}]
    The \id{cvode\_mem} pointer is \id{NULL}.
  \end{args}
}
{}
%%
%%
\ucfunction{CVodeGetNumStabLimOrderReds}
{
  flag = CVodeGetNumStabLimOrderReds(cvode\_mem, nslred);
}
{
  The function \ID{CVodeGetNumStabLimOrderReds} returns the
  number of order reductions dictated by the BDF stability limit 
  detection algorithm (see \S\ref{s:bdf_stab}).
}
{
  \begin{args}[cvode\_mem]
  \item[cvode\_mem] (\id{void *})
    pointer to the {\cvodes} memory block.
  \item[nslred] (\id{int *})
    number of order reductions due to stability limit detection.
  \end{args}
}
{
  The return value \id{flag} (of type \id{int}) is one of
  \begin{args}[CVG\_NO\_SLDET]
  \item[OKAY] 
    The optional output value has been successfuly set.
  \item[\Id{CVG\_NO\_MEM}]
    The \id{cvode\_mem} pointer is \id{NULL}.
  \item[\Id{CVG\_NO\_SLDET}]
    The stability limit detection algorithm was not activated 
    through a call to \id{CVodeSetStabLimDet}.
  \end{args}
}
{}
%%
%%
\ucfunction{CVodeGetTolScaleFactor}
{
  flag = CVodeGetTolScaleFactor(cvode\_mem, tolsfac);
}
{
  The function \ID{CVodeGetTolScaleFactor} returns a
  suggested factor by which the user's tolerances 
  should be scaled when too much accuracy has been 
  requested for some internal step.
}
{
  \begin{args}[cvode\_mem]
  \item[cvode\_mem] (\id{void *})
    pointer to the {\cvodes} memory block.
  \item[tolsfac] (\id{realtype *})
    suggested scaling factor for user tolerances.
  \end{args}
}
{
  The return value \id{flag} (of type \id{int}) is one of
  \begin{args}[CVG\_NO\_MEM]
  \item[OKAY] 
    The optional output value has been successfuly set.
  \item[\Id{CVG\_NO\_MEM}]
    The \id{cvode\_mem} pointer is \id{NULL}.
  \end{args}
}
{}
%%
%%
\ucfunction{CVodeGetErrWeights}
{
  flag = CVodeGetErrWeights(cvode\_mem, eweight);
}
{
  The function \ID{CVodeGetErrWeights} returns the solution error weights 
  at the current time. These are the reciprocals of the $W_i$ of (\ref{e:errwt}).
}
{
  \begin{args}[cvode\_mem]
  \item[cvode\_mem] (\id{void *})
    pointer to the {\cvodes} memory block.
  \item[eweight] (\id{N\_Vector *})
    solution error weights at the current time.
  \end{args}
}
{
  The return value \id{flag} (of type \id{int}) is one of
  \begin{args}[CVG\_NO\_MEM]
  \item[OKAY] 
    The optional output value has been successfuly set.
  \item[\Id{CVG\_NO\_MEM}]
    The \id{cvode\_mem} pointer is \id{NULL}.
  \end{args}
}
{
  The user need not allocate space for \id{eweight} and should not modify
  any of its components.
}
%%
%%
\ucfunction{CVodeGetEstLocalErrors}
{
  flag = CVodeGetEstLocalErrors(cvode\_mem, ele);
}
{
  The function \ID{CVodeGetEstLocalErrors} returns the
  vector of estimated local errors.
}
{
  \begin{args}[cvode\_mem]
  \item[cvode\_mem] (\id{void *})
    pointer to the {\cvodes} memory block.
  \item[ele] (\id{N\_Vector *})
    
  \end{args}
}
{
  The return value \id{flag} (of type \id{int}) is one of
  \begin{args}[CVG\_NO\_MEM]
  \item[OKAY] 
    The optional output value has been successfuly set.
  \item[\Id{CVG\_NO\_MEM}]
    The \id{cvode\_mem} pointer is \id{NULL}.
  \end{args}
}
{
  The user need not allocate space for \id{ele}.
}
%%
%%
\ucfunction{CVodeGetIntegratorStats}
{
  \begin{tabular}[t]{@{}r@{}l@{}}
    flag = CVodeGetIntegratorStats(&cvode\_mem, nsteps, nfevals, nlinsetups, \\
                                   &netfails, qlast, qcur, hinused, hlast, \\
                                   &hcur, tcur);
  \end{tabular}
}
{
  The function \ID{CVodeGetIntegratorStats} returns the {\cvodes} integrator statistics
  as a group.
}
{
  \begin{args}[cvode\_mem]
  \item[cvode\_mem] (\id{void *})
    pointer to the {\cvodes} memory block.
  \item[nsteps] (\id{int *})
    number of steps taken by {\cvodes}.
  \item[nfevals] (\id{int *})
    number of calls to the user's \id{f} function.
  \item[nlinsetups] (\id{int *})
    number of calls made to the linear solver setup routine.
  \item[netfails] (\id{int *})
    number of error test failures.
  \item[qlast] (\id{int *})
    method order used on the last internal step.
  \item[qcur] (\id{int *})
    method order to be used on the next internal step.
  \item[hinused] (\id{realtype *})
    actual value of initial step size.
  \item[hlast] (\id{realtype *})
    step size taken on the last internal step.
  \item[hcur] (\id{realtype *})
    step size to be attempted on the next internal step.
  \item[tcur] (\id{realtype *})
    current internal time reached.
  \end{args}
}
{
  The return value \id{flag} (of type \id{int}) is one of
  \begin{args}[CVG\_NO\_MEM]
  \item[OKAY] 
    the optional output value has been successfuly set.
  \item[\Id{CVG\_NO\_MEM}]
    the \id{cvode\_mem} pointer is \id{NULL}.
  \end{args}
}
{}
%%
%%
\ucfunction{CVodeGetNumNonlinSolvIters}
{
  flag = CVodeGetNumNonlinSolvIters(cvode\_mem, nniters);
}
{
  The function \ID{CVodeGetNumNonlinSolvIters} returns the
  number of nonlinear (functional or Newton) iterations performed. 
}
{
  \begin{args}[cvode\_mem]
  \item[cvode\_mem] (\id{void *})
    pointer to the {\cvodes} memory block.
  \item[nniters] (\id{int *})
    number of nonlinear iterations performed.
  \end{args}
}
{
  The return value \id{flag} (of type \id{int}) is one of
  \begin{args}[CVG\_NO\_MEM]
  \item[OKAY] 
    The optional output value has been successfuly set.
  \item[\Id{CVG\_NO\_MEM}]
    The \id{cvode\_mem} pointer is \id{NULL}.
  \end{args}
}
{}
%%
%%
\ucfunction{CVodeGetNumNonlinSolvConvFails}
{
  flag = CVodeGetNumNonlinSolvConvFails(cvode\_mem, nncfails);
}
{
  The function \ID{CVodeGetNumNonlinSolvConvFails} returns the
  number of nonlinear convergence failures that have occurred.
}
{
  \begin{args}[cvode\_mem]
  \item[cvode\_mem] (\id{void *})
    pointer to the {\cvodes} memory block.
  \item[nncfails] (\id{int *})
    number of nonlinear convergence failures.
  \end{args}
}
{
  The return value \id{flag} (of type \id{int}) is one of
  \begin{args}[CVG\_NO\_MEM]
  \item[OKAY] 
    The optional output value has been successfuly set.
  \item[\Id{CVG\_NO\_MEM}]
    The \id{cvode\_mem} pointer is \id{NULL}.
  \end{args}
}
{}
%%
%%
\ucfunction{CVodeGetNonlinSolvStats}
{
  flag = CVodeGetNonlinSolvStats(cvode\_mem, nniters, nncfails);
}
{
  The function \ID{CVodeGetNonlinSolvStats} returns the the
  {\cvodes} nonlinear solver statistics as a group.
}
{
  \begin{args}[cvode\_mem]
  \item[cvode\_mem] (\id{void *})
    pointer to the {\cvodes} memory block.
  \item[nniters] (\id{int *})
    number of nonlinear iterations performed.
  \item[nncfails] (\id{int *})
    number of nonlinear convergence failures.
  \end{args}
}
{
  The return value \id{flag} (of type \id{int}) is one of
  \begin{args}[CVG\_NO\_MEM]
  \item[OKAY] 
    The optional output value has been successfuly set.
  \item[\Id{CVG\_NO\_MEM}]
    The \id{cvode\_mem} pointer is \id{NULL}.
  \end{args}
}
{}
%%
\index{optional output!solver|)}
%%
%%===================================================================================

\subsubsection{Linear solver optional output functions}

For each of the linear system solver modules, there are various optional 
outputs that describe the performance of the module. The functions available 
to access these are described below.

\vspace{0.1in}\noindent{\bf Dense Linear solver}
\index{optional output!dense linear solver|(}
\index{CVDENSE@{\cvdense} linear solver!optional output|(}
%%
\index{CVDENSE@{\cvdense} linear solver!memory requirements|(} 
\index{memory requirements!CVDENSE@{\cvdense} linear solver|(}
%%
\ucfunction{CVDenseGetIntWorkSpace}
{
  flag = CVDenseGetIntWorkSpace(cvode\_mem, leniwD);
}
{
  The function \ID{CVDenseGetIntWorkSpace} returns the
  {\cvdense} integer workspace size.
}
{
  \begin{args}[cvode\_mem]
  \item[cvode\_mem] (\id{void *})
    pointer to the {\cvodes} memory block.
  \item[leniwD] (\id{long int *})
    the number of \id{integertype} values in the {\cvdense} workspace.
  \end{args}
}
{
  The return value \id{flag} (of type \id{int}) is one of
  \begin{args}[LIN\_NO\_LMEM]
  \item[OKAY] 
    The optional output value has been successfuly set.
  \item[\Id{LIN\_NO\_MEM}]
    The \id{cvode\_mem} pointer is \id{NULL}.
  \item[\Id{LIN\_NO\_LMEM}]
    The {\cvdense} linear solver has not been initialized.
  \end{args}
}
{
  In terms of the problem size $N$, the actual size of the integer workspace
  is $N$ \id{integertype} words.
}
%%
\ucfunction{CVDenseGetRealWorkSpace}
{
  flag = CVDenseGetRealWorkSpace(cvode\_mem, lenrwD);
}
{
  The function \ID{CVDenseGetRealWorkSpace} returns the
  {\cvdense} real workspace size.
}
{
  \begin{args}[cvode\_mem]
  \item[cvode\_mem] (\id{void *})
    pointer to the {\cvodes} memory block.
  \item[lenrwD] (\id{long int *})
    the number of \id{realtype} values in the {\cvdense} workspace.
  \end{args}
}
{
  The return value \id{flag} (of type \id{int}) is one of
  \begin{args}[LIN\_NO\_LMEM]
  \item[OKAY] 
    The optional output value has been successfuly set.
  \item[\Id{LIN\_NO\_MEM}]
    The \id{cvode\_mem} pointer is \id{NULL}.
  \item[\Id{LIN\_NO\_LMEM}]
    The {\cvdense} linear solver has not been initialized.
  \end{args}
}
{
  In terms of the problem size $N$, the actual size of the real workspace
  is $2N^2$ \id{realtype} words.
}
%%
\index{CVDENSE@{\cvdense} linear solver!memory requirements|)} 
\index{memory requirements!CVDENSE@{\cvdense} linear solver|)}
%%
%%
\ucfunction{CVDenseGetNumJacEvals}
{
  flag = CVDenseGetNumJacEvals(cvode\_mem, njevalsD);
}
{
  The function \ID{CVDenseGetNumJacEvals} returns the
  number of calls to the dense Jacobian approximation function.
}
{
  \begin{args}[cvode\_mem]
  \item[cvode\_mem] (\id{void *})
    pointer to the {\cvodes} memory block.
  \item[njevalsD] (\id{int *})
    the number of calls to the Jacobian function.
  \end{args}
}
{
  The return value \id{flag} (of type \id{int}) is one of
  \begin{args}[LIN\_NO\_LMEM]
  \item[OKAY] 
    The optional output value has been successfuly set.
  \item[\Id{LIN\_NO\_MEM}]
    The \id{cvode\_mem} pointer is \id{NULL}.
  \item[\Id{LIN\_NO\_LMEM}]
    The {\cvdense} linear solver has not been initialized.
  \end{args}
}
{}
%%
%%
\ucfunction{CVDenseGetNumRhsEvals}
{
  flag = CVDenseGetNumRhsEvals(cvode\_mem, nfevalsD);
}
{
  The function \ID{CVDenseGetNumRhsEvals} returns the
  number of calls to the user right-hand side function due to the 
  finite difference dense Jacobian approximation.
}
{
  \begin{args}[cvode\_mem]
  \item[cvode\_mem] (\id{void *})
    pointer to the {\cvodes} memory block.
  \item[nfevalsD] (\id{int *})
    the number of calls to the user right-hand side function.
  \end{args}
}
{
  The return value \id{flag} (of type \id{int}) is one of
  \begin{args}[LIN\_NO\_LMEM]
  \item[OKAY] 
    The optional output value has been successfuly set.
  \item[\Id{LIN\_NO\_MEM}]
    The \id{cvode\_mem} pointer is \id{NULL}.
  \item[\Id{LIN\_NO\_LMEM}]
    The {\cvdense} linear solver has not been initialized.
  \end{args}
}
{
  The value \id{nfevalsD} is incremented only if the default 
  \id{CVDenseDQJac} difference quotient function is used.
}
%%
%%
\index{CVDENSE@{\cvdense} linear solver!optional output|)}
\index{optional output!dense linear solver|)}
%
%--------------------------------
%
\noindent{\bf Band Linear solver}
\index{optional output!band linear solver|(}
\index{CVBAND@{\cvband} linear solver!optional output|(}
%%
\index{CVBAND@{\cvband} linear solver!memory requirements|(} 
\index{memory requirements!CVBAND@{\cvband} linear solver|(}
%%
\ucfunction{CVBandGetIntWorkSpace}
{
  flag = CVBandGetIntWorkSpace(cvode\_mem, leniwB);
}
{
  The function \ID{CVBandGetIntWorkSpace} returns the
  {\cvband} integer workspace size.
}
{
  \begin{args}[cvode\_mem]
  \item[cvode\_mem] (\id{void *})
    pointer to the {\cvodes} memory block.
  \item[leniwB] (\id{long int *})
    the number of \id{integertype} values in the {\cvband} workspace.
  \end{args}
}
{
  The return value \id{flag} (of type \id{int}) is one of
  \begin{args}[LIN\_NO\_LMEM]
  \item[OKAY] 
    The optional output value has been successfuly set.
  \item[\Id{LIN\_NO\_MEM}]
    The \id{cvode\_mem} pointer is \id{NULL}.
  \item[\Id{LIN\_NO\_LMEM}]
    The {\cvband} linear solver has not been initialized.
  \end{args}
}
{
  In terms of the problem size $N$, the actual size of the integer workspace
  is $N$ \id{integertype} words.
}
%%
\ucfunction{CVBandGetRealWorkSpace}
{
  flag = CVBandGetRealWorkSpace(cvode\_mem, lenrwB);
}
{
  The function \ID{CVBandGetRealWorkSpace} returns the
  {\cvband} real workspace size.
}
{
  \begin{args}[cvode\_mem]
  \item[cvode\_mem] (\id{void *})
    pointer to the {\cvodes} memory block.
  \item[lenrwB] (\id{long int *})
    the number of \id{realtype} values in the {\cvband} workspace.
  \end{args}
}
{
  The return value \id{flag} (of type \id{int}) is one of
  \begin{args}[LIN\_NO\_LMEM]
  \item[OKAY] 
    The optional output value has been successfuly set.
  \item[\Id{LIN\_NO\_MEM}]
    The \id{cvode\_mem} pointer is \id{NULL}.
  \item[\Id{LIN\_NO\_LMEM}]
    The {\cvband} linear solver has not been initialized.
  \end{args}
}
{
  In terms of the problem size $N$ and Jacobian half-bandwidths, 
  the actual size of the real workspace
  $N\,(2$ \id{mupper}$+ 3$ \id{mlower} $+ 2)$ \id{realtype} words.
}
%%
\index{CVBAND@{\cvband} linear solver!memory requirements|)} 
\index{memory requirements!CVBAND@{\cvband} linear solver|)}
%%
%%
\ucfunction{CVBandGetNumJacEvals}
{
  flag = CVBandGetNumJacEvals(cvode\_mem, njevalsB);
}
{
  The function \ID{CVBandGetNumJacEvals} returns the
  number of calls to the banded Jacobian approximation function.
}
{
  \begin{args}[cvode\_mem]
  \item[cvode\_mem] (\id{void *})
    pointer to the {\cvodes} memory block.
  \item[njevalsB] (\id{int *})
    the number of calls to the Jacobian function.
  \end{args}
}
{
  The return value \id{flag} (of type \id{int}) is one of
  \begin{args}[LIN\_NO\_LMEM]
  \item[OKAY] 
    The optional output value has been successfuly set.
  \item[\Id{LIN\_NO\_MEM}]
    The \id{cvode\_mem} pointer is \id{NULL}.
  \item[\Id{LIN\_NO\_LMEM}]
    The {\cvband} linear solver has not been initialized.
  \end{args}
}
{}
%%
%%
\ucfunction{CVBandGetNumRhsEvals}
{
  flag = CVBandGetNumRhsEvals(cvode\_mem, nfevalsB);
}
{
  The function \ID{CVBandGetNumRhsEvals} returns the
  number of calls to the user right-hand side function due to the 
  finite difference banded Jacobian approximation.
}
{
  \begin{args}[cvode\_mem]
  \item[cvode\_mem] (\id{void *})
    pointer to the {\cvodes} memory block.
  \item[nfevalsB] (\id{int *})
    the number of calls to the user right-hand side function.
  \end{args}
}
{
  The return value \id{flag} (of type \id{int}) is one of
  \begin{args}[LIN\_NO\_LMEM]
  \item[OKAY] 
    The optional output value has been successfuly set.
  \item[\Id{LIN\_NO\_MEM}]
    The \id{cvode\_mem} pointer is \id{NULL}.
  \item[\Id{LIN\_NO\_LMEM}]
    The {\cvband} linear solver has not been initialized.
  \end{args}
}
{
  The value \id{nfevalsB} is incremented only if the default 
  \id{CVBandDQJac} difference quotient function is used.
}
%%
%%
\index{CVBAND@{\cvband} linear solver!optional output|)}
\index{optional output!band linear solver|)}
%
%--------------------------------
%
\noindent{\bf Diagonal Linear solver.}
\index{optional output!diagonal linear solver|(}
\index{CVDIAG@{\cvdiag} linear solver!optional output|(}
%%
\index{CVDIAG@{\cvdiag} linear solver!memory requirements|(} 
\index{memory requirements!CVDIAG@{\cvdiag} linear solver|(}
%%
\ucfunction{CVDiagGetIntWorkSpace}
{
  flag = CVDiagGetIntWorkSpace(cvode\_mem, leniwDI);
}
{
  The function \ID{CVDiagGetIntWorkSpace} returns the
  {\cvdiag} integer workspace size.
}
{
  \begin{args}[cvode\_mem]
  \item[cvode\_mem] (\id{void *})
    pointer to the {\cvodes} memory block.
  \item[leniwDI] (\id{long int *})
    the number of \id{integertype} values in the {\cvdiag} workspace.
  \end{args}
}
{
  The return value \id{flag} (of type \id{int}) is one of
  \begin{args}[LIN\_NO\_LMEM]
  \item[OKAY] 
    The optional output value has been successfuly set.
  \item[\Id{LIN\_NO\_MEM}]
    The \id{cvode\_mem} pointer is \id{NULL}.
  \item[\Id{LIN\_NO\_LMEM}]
    The {\cvdiag} linear solver has not been initialized.
  \end{args}
}
{}
%%
\ucfunction{CVDiagGetRealWorkSpace}
{
  flag = CVDiagGetRealWorkSpace(cvode\_mem, lenrwDI);
}
{
  The function \ID{CVDiagGetRealWorkSpace} returns the
  {\cvdiag} real workspace size.
}
{
  \begin{args}[cvode\_mem]
  \item[cvode\_mem] (\id{void *})
    pointer to the {\cvodes} memory block.
  \item[lenrwDI] (\id{long int *})
    the number of \id{realtype} values in the {\cvdiag} workspace.
  \end{args}
}
{
  The return value \id{flag} (of type \id{int}) is one of
  \begin{args}[LIN\_NO\_LMEM]
  \item[OKAY] 
    The optional output value has been successfuly set.
  \item[\Id{LIN\_NO\_MEM}]
    The \id{cvode\_mem} pointer is \id{NULL}.
  \item[\Id{LIN\_NO\_LMEM}]
    The {\cvdiag} linear solver has not been initialized.
  \end{args}
}
{
  In terms of the problem size $N$, the actual size of the real workspace
  $3 N$ \id{realtype} words.
}
%%
\index{CVDIAG@{\cvdiag} linear solver!memory requirements|)} 
\index{memory requirements!CVDIAG@{\cvdiag} linear solver|)}
%%
%%
\ucfunction{CVDiagGetNumRhsEvals}
{
  flag = CVDiagGetNumRhsEvals(cvode\_mem, nfevalsDI);
}
{
  The function \ID{CVDiagGetNumRhsEvals} returns the
  number of calls to the user right-hand side function due to the 
  finite difference Jacobian approximation.
}
{
  \begin{args}[cvode\_mem]
  \item[cvode\_mem] (\id{void *})
    pointer to the {\cvodes} memory block.
  \item[nfevalsDI] (\id{int *})
    the number of calls to the user right-hand side function.
  \end{args}
}
{
  The return value \id{flag} (of type \id{int}) is one of
  \begin{args}[LIN\_NO\_LMEM]
  \item[OKAY] 
    The optional output value has been successfuly set.
  \item[\Id{LIN\_NO\_MEM}]
    The \id{cvode\_mem} pointer is \id{NULL}.
  \item[\Id{LIN\_NO\_LMEM}]
    The {\cvdiag} linear solver has not been initialized.
  \end{args}
}
{
  The number of diagonal approximate Jacobians formed is
  equal to the number of calls to the linear solver setup function
  (available by calling \id{CVodeGetNumLinsolvSetups}).
}
%%
%%
\index{CVDIAG@{\cvdiag} linear solver!optional output|)}
\index{optional output!diagonal linear solver|)}
%
%--------------------------------
%
\noindent{\bf SPGMR Linear solver.}
\index{optional output!iterative linear solver|(}
\index{CVSPGMR@{\cvspgmr} linear solver!optional output|(} 
%%
\index{CVSPGMR@{\cvspgmr} linear solver!memory requirements|(} 
\index{memory requirements!CVSPGMR@{\cvspgmr} linear solver|(}
%%
\ucfunction{CVSpgmrGetIntWorkSpace}
{
  flag = CVSpgmrGetIntWorkSpace(cvode\_mem, leniwSG);
}
{
  The function \ID{CVSpgmrGetIntWorkSpace} returns the
  {\cvspgmr} integer workspace size.
}
{
  \begin{args}[cvode\_mem]
  \item[cvode\_mem] (\id{void *})
    pointer to the {\cvodes} memory block.
  \item[leniwSG] (\id{long int *})
    the number of \id{integertype} values in the {\cvspgmr} workspace.
  \end{args}
}
{
  The return value \id{flag} (of type \id{int}) is one of
  \begin{args}[LIN\_NO\_LMEM]
  \item[OKAY] 
    The optional output value has been successfuly set.
  \item[\Id{LIN\_NO\_MEM}]
    The \id{cvode\_mem} pointer is \id{NULL}.
  \item[\Id{LIN\_NO\_LMEM}]
    The {\cvspgmr} linear solver has not been initialized.
  \end{args}
}
{}
%%
\ucfunction{CVSpgmrGetRealWorkSpace}
{
  flag = CVSpgmrGetRealWorkSpace(cvode\_mem, lenrwSG);
}
{
  The function \ID{CVSpgmrGetRealWorkSpace} returns the
  {\cvspgmr} real workspace size.
}
{
  \begin{args}[cvode\_mem]
  \item[cvode\_mem] (\id{void *})
    pointer to the {\cvodes} memory block.
  \item[lenrwSG] (\id{long int *})
    the number of \id{realtype} values in the {\cvspgmr} workspace.
  \end{args}
}
{
  The return value \id{flag} (of type \id{int}) is one of
  \begin{args}[LIN\_NO\_LMEM]
  \item[OKAY] 
    The optional output value has been successfuly set.
  \item[\Id{LIN\_NO\_MEM}]
    The \id{cvode\_mem} pointer is \id{NULL}.
  \item[\Id{LIN\_NO\_LMEM}]
    The {\cvspgmr} linear solver has not been initialized.
  \end{args}
}
{
  In terms of the problem size $N$ and maximum subspace size \id{maxl}, 
  the actual sizes of the real workspace is
  $N*($ \id{maxl} $+ 5) +$ \id{maxl} $*($ \id{maxl} $ + 4) + 1$ \id{realtype} words.
}
%%
\index{CVSPGMR@{\cvspgmr} linear solver!memory requirements|)} 
\index{memory requirements!CVSPGMR@{\cvspgmr} linear solver|)}
%%
%%
\ucfunction{CVSpgmrGetNumLinIters}
{
  flag = CVSpgmrGetNumLinIters(cvode\_mem, nliters);
}
{
  The function \ID{CVSpgmrGetNumLinIters} returns the
  cumulative number of linear iterations.
}
{
  \begin{args}[cvode\_mem]
  \item[cvode\_mem] (\id{void *})
    pointer to the {\cvodes} memory block.
  \item[nliters] (\id{int *})
    the current number of linear iterations.
  \end{args}
}
{
  The return value \id{flag} (of type \id{int}) is one of
  \begin{args}[LIN\_NO\_LMEM]
  \item[OKAY] 
    The optional output value has been successfuly set.
  \item[\Id{LIN\_NO\_MEM}]
    The \id{cvode\_mem} pointer is \id{NULL}.
  \item[\Id{LIN\_NO\_LMEM}]
    The {\cvspgmr} linear solver has not been initialized.
  \end{args}
}
{}
%%
%%
\ucfunction{CVSpgmrGetNumConvFails}
{
  flag = CVSpgmrGetNumConvFails(cvode\_mem, nlcfails);
}
{
  The function \ID{CVSpgmrGetNumConvFails} returns the
  cumulative number of linear convergence failures.
}
{
  \begin{args}[cvode\_mem]
  \item[cvode\_mem] (\id{void *})
    pointer to the {\cvodes} memory block.
  \item[nlcfails] (\id{int *})
    the current number of linear convergence failures.
  \end{args}
}
{
  The return value \id{flag} (of type \id{int}) is one of
  \begin{args}[LIN\_NO\_LMEM]
  \item[OKAY] 
    The optional output value has been successfuly set.
  \item[\Id{LIN\_NO\_MEM}]
    The \id{cvode\_mem} pointer is \id{NULL}.
  \item[\Id{LIN\_NO\_LMEM}]
    The {\cvspgmr} linear solver has not been initialized.
  \end{args}
}
{}
%%
%%
\ucfunction{CVSpgmrGetNumPrecEvals}
{
  flag = CVSpgmrGetNumPrecEvals(cvode\_mem, npevals);
}
{
  The function \ID{CVSpgmrGetNumPrecEvals} returns the
  number of preconditioner evaluations, i.e., the number of 
  calls made to \id{psetup} with \id{jok=FALSE}.
}
{
  \begin{args}[cvode\_mem]
  \item[cvode\_mem] (\id{void *})
    pointer to the {\cvodes} memory block.
  \item[npevals] (\id{int *})
    the current number of calls to \id{psetup}.
  \end{args}
}
{
  The return value \id{flag} (of type \id{int}) is one of
  \begin{args}[LIN\_NO\_LMEM]
  \item[OKAY] 
    the optional output value has been successfuly set.
  \item[\Id{LIN\_NO\_MEM}]
    the \id{cvode\_mem} pointer is \id{NULL}.
  \item[\Id{LIN\_NO\_LMEM}]
    the {\cvspgmr} linear solver has not been initialized.
  \end{args}
}
{}
%%
%%
\ucfunction{CVSpgmrGetNumPrecSolves}
{
  flag = CVSpgmrGetNumPrecSolves(cvode\_mem, npsolves);
}
{
  The function \ID{CVSpgmrGetNumPrecSolves} returns the
  cumulative number of calls made to the preconditioner 
  solve function, \id{psolve}.
}
{
  \begin{args}[cvode\_mem]
  \item[cvode\_mem] (\id{void *})
    pointer to the {\cvodes} memory block.
  \item[npsolves] (\id{int *})
    the current number of calls to \id{psolve}.
  \end{args}
}
{
  The return value \id{flag} (of type \id{int}) is one of
  \begin{args}[LIN\_NO\_LMEM]
  \item[OKAY] 
    The optional output value has been successfuly set.
  \item[\Id{LIN\_NO\_MEM}]
    The \id{cvode\_mem} pointer is \id{NULL}.
  \item[\Id{LIN\_NO\_LMEM}]
    The {\cvspgmr} linear solver has not been initialized.
  \end{args}
}
{}
%%
%%
\ucfunction{CVSpgmrGetNumJtimesEvals}
{
  flag = CVSpgmrGetNumJtimesEvals(cvode\_mem, njvevals);
}
{
  The function \ID{CVSpgmrGetNumJtimesEvals} returns the
  cumulative number made to the Jacobian-vector function,
  \id{jtimes}.
}
{
  \begin{args}[cvode\_mem]
  \item[cvode\_mem] (\id{void *})
    pointer to the {\cvodes} memory block.
  \item[njvevals] (\id{int *})
    the current number of calls to \id{jtimes}.
  \end{args}
}
{
  The return value \id{flag} (of type \id{int}) is one of
  \begin{args}[LIN\_NO\_LMEM]
  \item[OKAY] 
    The optional output value has been successfuly set.
  \item[\Id{LIN\_NO\_MEM}]
    The \id{cvode\_mem} pointer is \id{NULL}.
  \item[\Id{LIN\_NO\_LMEM}]
    The {\cvspgmr} linear solver has not been initialized.
  \end{args}
}
{}
%%
%%
\ucfunction{CVSpgmrGetNumRhsEvals}
{
  flag = CVSpgmrGetNumRhsEvals(cvode\_mem, nfevalsSG);
}
{
  The function \ID{CVSpgmrGetNumRhsEvals} returns the
  number of calls to the user right-hand side function for
  finite difference Jacobian-vector product approximation.
}
{
  \begin{args}[cvode\_mem]
  \item[cvode\_mem] (\id{void *})
    pointer to the {\cvodes} memory block.
  \item[nfevalsSG] (\id{int *})
    the number of calls to the user right-hand side function.
  \end{args}
}
{
  The return value \id{flag} (of type \id{int}) is one of
  \begin{args}[LIN\_NO\_LMEM]
  \item[OKAY] 
    The optional output value has been successfuly set.
  \item[\Id{LIN\_NO\_MEM}]
    The \id{cvode\_mem} pointer is \id{NULL}.
  \item[\Id{LIN\_NO\_LMEM}]
    The {\cvspgmr} linear solver has not been initialized.
  \end{args}
}
{
  The value \id{nfevalsSG} is incremented only if the default 
  \id{CVSpgmrDQJtimes} difference quotient function is used.
}
%%
\index{CVSPGMR@{\cvspgmr} linear solver!optional output|)} 
\index{optional output!iterative linear solver|)}
%%
%%
%%===================================================================================

\subsection{{\cvodes} reinitialization function}\label{sss:cvreinit}
\index{reinitialization}

The function \ID{CVodeReInit} reinitializes the main {\cvodes} solver for
the solution of a problem, where a prior call to \Id{CVodeMalloc} has
been made. The new problem must have the same size as the previous one.
\id{CVodeReInit} performs the same input checking and initializations 
that \id{CVodeMalloc} does, but does no memory allocation, assuming that the 
existing internal memory is sufficient for the new problem.             
                                                                 
The use of \id{CVodeReInit} requires that the maximum method order,    
\Id{maxord}, is no larger for the new problem than for the problem  
specified in the last call to \id{CVodeMalloc}.  This condition is  
automatically fulfilled if the multistep method parameter \id{lmm}  
is unchanged (or changed from \Id{ADAMS} to \Id{BDF}) and the default    
value for \id{maxord} is specified.

The {\nvector} module specification \id{nvSpec} set for the previous problem
will be reused for the new problem.
%%
%%
\ucfunction{CVodeReInit}
{
  flag = CVodeReInit(cvode\_mem, f, t0, y0, itol, reltol, abstol);
}
{
  The function \id{CVodeReInit} provides required problem specifications 
  and reinitializes {\cvodes}.
}
{
  \begin{args}[abstol]
  \item[cvode\_mem] (\id{void *})
    pointer to the {\cvodes} memory block.
  \item[f] (\Id{RhsFn})
    is the {\C} function which computes $f$ in the ODE. This function has the form 
    \id{f(N, t, y, ydot, f\_data)} (for full details see \S\ref{ss:user_fct_sim}).
  \item[t0] (\id{realtype})
    is the initial value of $t$.
  \item[y0] (\id{N\_Vector})
    is the initial value of $y$. 
  \item[itol] (\id{int}) 
    is either \Id{SS} or \Id{SV}, where \id{SS} indicates scalar relative error 
    tolerance and scalar absolute error tolerance, while \id{SV} indicates scalar
    relative error tolerance and vector absolute error tolerance. 
    The latter choice is important when the absolute error tolerance needs to
    be different for each component of the ODE. 
  \item[reltol] (\id{realtype *})
    is a pointer to the relative error tolerance.
  \item[abstol] (\id{void *})
    is a pointer to the absolute error tolerance.
  \end{args}
}
{
  The return flag \id{flag} (of type \id{int}) will be one of the following:.
  \begin{args}[CVREI\_NO\_MALLOC]
  \item[\Id{SUCCESS}]
    The call to \id{CVodeReInit} was successful.
  \item[\Id{CVREI\_NO\_MEM}] 
    The {\cvodes} memory block was not initialized through a 
    previous call to \id{CVodeCreate}.
  \item[\Id{CVREI\_NO\_MALLOC}] 
    Memory space for the {\cvodes} memory block was not allocated through a 
    previous call to \id{CVodeMalloc}.
  \item[\Id{CVREI\_ILL\_INPUT}] 
    An input argument to \id{CVodeReInit} has an illegal value.
  \end{args}
}
{
  If an error occured, \id{CVodeReInit} also prints an error message to the
  file specified by the optional input \id{errfp}.
}
%%
%%
%%===================================================================================
\section{User-supplied functions for IVP solution}\label{ss:user_fct_sim}
%%===================================================================================

The user-supplied functions consist of one function defining the ODE, 
(optionally) a function that provides Jacobian related information for the linear 
solver (if Newton iteration is chosen), and (optionally) one or two functions 
that define the preconditioner for use in the {\spgmr} algorithm. 
%%
%%--------------
%%
\subsection{ODE right-hand side} 
\label{ss:rhsFn}
\index{right hand side function!initial value problem|(}
The user must provide a function of type \ID{RhsFn} defined as follows:
\usfunction{RhsFn}
{
  typedef void (*RhsFn)(realtype t, N\_Vector y, N\_Vector ydot, void *f\_data);
}
{
  This function computes the ODE right-hand side for a given value
  of the independent variable $t$ and state vector $y$.
}
{
  \begin{args}[f\_data]
  \item[t]
    is the current value of the independent variable.
  \item[y]
    is the current value of the dependent variable vector, $y(t)$.
  \item[ydot]
    is the output vector $f(t,y)$.
  \item[f\_data]
    is a pointer to user data - the same as the \id{f\_data}      
    parameter passed to \id{CVodeSetFdata}.   
  \end{args}
}
{
  A \id{RhsFn} function type does not have a return value.                        
}
{
Allocation of memory for \id{ydot} is handled within {\cvodes}.
}
\index{right hand side function!initial value problem|)}
%%
%%--------------
%%
\subsection{Jacobian information (direct method with dense Jacobian)}
\label{ss:djacFn}
\index{Jacobian approximation function!dense!user-supplied|(}
If the direct linear solver with dense treatment of the Jacobian is used 
(i.e. \Id{CVDense} is called in Step \ref{i:lin_solver} of \S\ref{ss:skeleton_sim}), 
the user may provide a function of type \ID{CVDenseJacFn} defined by
\usfunction{CVDenseJacFn}
{
  typedef void (*CVDenseJacFn)(&integertype N, DenseMat J, realtype t, \\
                               &N\_Vector y, N\_Vector fy, void *jac\_data, \\
                               &N\_Vector tmp1, N\_Vector tmp2, N\_Vector tmp3);
}
{
  This function computes the dense Jacobian $J = \partial f / \partial y$ 
  (or an approximation to it).
}
{
  \begin{args}[jac\_data]
  \item[N]
    is the problem size.
  \item[J]
    is the output Jacobian matrix.  
  \item[t]
    is the current value of the independent variable.
  \item[y]
    is the current value of the dependent variable vector, 
    namely the predicted value of $y(t)$.
  \item[fy]
    is the vector $f(t,y)$.
  \item[jac\_data]
    is a pointer to user data - the same as the \id{jac\_data}      
    parameter passed to \id{CVDenseSetJacData}.   
  \item[tmp1]
  \item[tmp2]
  \item[tmp3]
    are pointers to memory allocated    
    for vectors of length \id{N} which can be used by           
    \id{CVDenseJacFn} as temporary storage or work space.    
  \end{args}
}
{
  A \id{CVDenseJacFn} function type does not have a return value.                        
}
{
  A user-supplied dense Jacobian function must load the \id{N} by \id{N}
  dense matrix \id{J} with an approximation to the Jacobian matrix $J$
  at the point (\id{t}, \id{y}).  Only nonzero elements need to be loaded
  into \id{J} because \id{J} is set to the zero matrix before the call
  to the Jacobian function. The type of \id{J} is \Id{DenseMat}. 
  
  The accessor macros \Id{DENSE\_ELEM} and \Id{DENSE\_COL} allow the user to
  read and write dense matrix elements without making explicit
  references to the underlying representation of the \id{DenseMat}
  type. \id{DENSE\_ELEM(J, i, j)} references the (\id{i}, \id{j})-th
  element of the dense matrix \id{J} (\id{i}, \id{j} = 0..N-1). This macro
  is for use in small problems in which efficiency of access is not a major
  concern.  Thus, in terms of indices $m$ and $n$ running from $1$ to
  $N$, the Jacobian element $J_{m,n}$ can be loaded with the statement
  \id{DENSE\_ELEM(J, m-1, n-1) =} $J_{m,n}$.  Alternatively,
  \id{DENSE\_COL(J, j)} returns a pointer to the storage for
  the \id{j}th column of \id{J}, and the elements of the \id{j}th column
  are then accessed via ordinary array indexing.  Thus $J_{m,n}$ can be 
  loaded with the statements \id{col\_n = DENSE\_COL(J, n-1);}
  \id{col\_n[m-1] =} $J_{m,n}$.  For large problems, it is more 
  efficient to use \id{DENSE\_COL} than to use \id{DENSE\_ELEM}. 
  Note that both of these macros number rows and columns
  starting from $0$, not $1$.  

  The \id{DenseMat} type and the accessor macros \id{DENSE\_ELEM} and 
  \id{DENSE\_COL} are documented in \S\ref{ss:dense}.

  If the user's \id{CVDenseJacFn} function uses difference quotient approximations,
  it may need to access quantities not in the call list. These include the current
  stepsize, the error weights, etc. To obtain these, use the \id{CVodeGet} functions 
  described in \S\ref{ss:optional_output}. The unit roundoff can be accessed through
  the macro \id{DBL\_EPSILON} defined in \id{float.h}.
}
\index{Jacobian approximation function!dense!user-supplied|)}
%%
%%--------------
%%
\subsection{Jacobian information (direct method with banded Jacobian)}
\label{ss:bjacFn}
\index{Jacobian approximation function!band!user-supplied|(}
If the direct linear solver with banded treatment of the Jacobian is used 
(i.e. \Id{CVBand} is called in Step \ref{i:lin_solver} of \S\ref{ss:skeleton_sim}), 
the user may provide a function of type \ID{CVBandJacFn} defined as follows:
\usfunction{CVBandJacFn}
{
 typedef void (*CVBandJacFn)(&integertype N, integertype mupper, \\
                             &integertype mlower, BandMat J, realtype t, \\ 
                             &N\_Vector y, N\_Vector fy, void *jac\_data, \\
                             &N\_Vector tmp1, N\_Vector tmp2, N\_Vector tmp3);
}
{
  This function computes the banded Jacobian $J = \partial f / \partial y$ 
  (or a banded approximation to it).
}
{
  \begin{args}[jac\_data]
  \item[N]
    is the problem size.
  \item[mlower]
  \item[mupper]
    are the lower and upper halh bandwidth of the Jacobian.
  \item[J]
    is the output Jacobian matrix.  
  \item[t]
    is the current value of the independent variable.
  \item[y]
    is the current value of the dependent variable vector, 
    namely the predicted value of $y(t)$.
  \item[fy]
    is the vector $f(t,y)$.
  \item[jac\_data]
    is a pointer to user data - the same as the \id{jac\_data}      
    parameter passed to \id{CVBandSetJacData}.   
  \item[tmp1]
  \item[tmp2]
  \item[tmp3]
    are pointers to memory allocated    
    for vectors of length \id{N} which can be used by           
    \id{CVBandJacFn} as temporary storage or work space.    
  \end{args}
}
{
  A \id{CVBandJacFn} function type does not have a return value.                        
}
{
  A user-supplied band Jacobian function must load the band matrix \id{J}
  of type \Id{BandMat} with the elements of the Jacobian $J(t,y)$ at the
  point (\id{t},\id{y}).  Only nonzero elements need to be loaded into
  \id{J} because \id{J} is preset to zero before the call to the
  Jacobian function.  

  The accessor macros \Id{BAND\_ELEM}, \Id{BAND\_COL}, and \Id{BAND\_COL\_ELEM} 
  allow the user to read and write band matrix elements without making specific 
  references to the underlying representation of the \id{BandMat} type.
  \id{BAND\_ELEM(J, i, j)} references the (\id{i}, \id{j})th element of the band matrix \id{J}.
  This macro is for use in small problems in which efficiency of access is not
  a major concern.  Thus, in terms of indices $m$ and $n$ running from $1$ to
  $N$ with $(m,n)$ within the band defined by \id{mupper} and
  \id{mlower}, the Jacobian element $J_{m,n}$ can be loaded with the 
  statement \id{BAND\_ELEM(J, m-1, n-1) =} $J_{m,n}$. The elements within
  the band are those with \id{-mupper} $\le$ \id{m-n} $\le$ \id{mlower}.
  Alternatively, \id{BAND\_COL(J, j)} returns a pointer to the diagonal element of the
  \id{j}th column of \id{J}, and if we assign this address to 
  \id{realtype *col\_j}, then the \id{i}th element of the \id{j}th column is
  given by \id{BAND\_COL\_ELEM(col\_j, i, j)}.
  Thus for $(m,n)$ within the band, $J_{m,n}$ can be loaded by setting 
  \id{col\_n = BAND\_COL(J, n-1);} \id{BAND\_COL\_ELEM(col\_n, m-1, n-1) =} $J_{m,n}$.
  The elements of the \id{j}th column can also be accessed
  via ordinary array indexing, but this approach requires knowledge of
  the underlying storage for a band matrix of type \id{BandMat}.  
  The array \id{col\_n} can be indexed from $-$\id{mupper} to \id{mlower}.
  For large problems, it is more efficient to use the combination of
  \id{BAND\_COL} and \id{BAND\_COL\_ELEM} than to use the
  \id{BAND\_ELEM}.  As in the dense case, these macros all number rows
  and columns starting from $0$, not $1$.  

  The \id{BandMat} type and the accessor macros \id{BAND\_ELEM}, \id{BAND\_COL}, and
  \id{BAND\_COL\_ELEM} are documented in \S\ref{ss:band}.

  If the user's \id{CVBandJacFn} function uses difference quotient approximations,
  it may need to access quantities not in the call list. These include the current
  stepsize, the error weights, etc. To obtain these, use the \id{CVodeGet} functions 
  described in \S\ref{ss:optional_output}. The unit roundoff can be accessed through
  the macro \id{DBL\_EPSILON} defined in \id{float.h}.
}
\index{Jacobian approximation function!band!user-supplied|)}
%%
%%----------------
%%
\subsection{Jacobian information ({\spgmr} matrix-vector product)}
\label{ss:jtimesFn}
\index{Jacobian approximation function!Jacobian times vector!user-supplied|(}
If an iterative {\spgmr} linear solver is selected (\id{CVSpgmr} is called in step 
\ref{i:lin_solver} of \S\ref{ss:skeleton_sim}) the user may provide a function
of type \ID{CVSpgmrJacTimesVecFn} in the following form:
\usfunction{CVSpgmrJacTimesVecFn}
{
  typedef int (*CVSpgmrJacTimesVecFn)(&N\_Vector v, N\_Vector Jv, realtype t, \\
                                      &N\_Vector y, N\_Vector fy, void *jac\_data, \\
                                      &N\_Vector tmp);
}
{
  This function computes the product $J v = (\partial f / \partial y) v$ 
  (or an approximation to it).
}
{
  \begin{args}[jac\_data]
  \item[v]
    is the vector by which the Jacobian must be multiplied to the right.
  \item[Jv]
      is the output vector computed.
  \item[t]
    is the current value of the independent variable.       
  \item[y] 
    is the current value of the dependent variable vector. 
  \item[fy]
    is the vector $f(t,y)$.
  \item[jac\_data]
    is a pointer to user data - the same as the \id{jac\_data}      
    parameter passed to \id{CVSpgmrSetJacData}.   
  \item[tmp]
    is a pointer to memory allocated for a vector of        
    length \id{N} which can be used for work space.
  \end{args}
}
{  
  The value to be returned by the Jacobian times vector function should be
  $0$ if successful. Any other return value will result in an unrecoverable
  error of the {\spgmr} generic solver, in which case the integration is halted.
}
{}
\index{Jacobian approximation function!Jacobian times vector!user-supplied|)}
%%
%%--------------
%%
\subsection{Preconditioning ({\spgmr} linear system solution)}
\label{ss:psolveFn}
\index{CVSPGMR@{\cvspgmr} linear solver!preconditioner solve function}
If preconditioning is used, then the user must provide a {\C} function to
solve the linear system $Pz = r$ where $P$ may be either a left or a
right preconditioner matrix.
This function must be of type \ID{CVSpgmrPrecSolveFn}, defined as follows:
%%
%%
\usfunction{CVSpgmrPrecSolveFn}
{
  typedef int (*CVSpgmrPrecSolveFn)(&realtype t, N\_Vector y, N\_Vector fy, \\
                                    &N\_Vector r, N\_Vector z, \\ 
                                    &realtype gamma, realtype delta, \\
                                    &int lr, void *P\_data, N\_Vector tmp);
}
{
  This function solves the preconditioning system $Pz = r$.
}
{  
  \begin{args}[P\_data]
  \item[t]
    is the current value of the independent variable.
  \item[y] 
    is the current value of the dependent variable vector.  
  \item[fy]
    is the vector $f(t,y)$.
  \item[r]
    is the right-hand side vector of the linear system.
  \item[z]
    is the output vector computed.
  \item[gamma]
    is the scalar $\gamma$ appearing in the Newton matrix $M=I-\gamma J$.
  \item[delta]
    is an input tolerance to be used if an iterative method 
    is employed in the solution.  In that case, the residual 
    vector $Res = r - P z$ of the system should be made less than 
    \id{delta} in weighted $l_2$ norm,     
    i.e., $\sqrt{\sum_i (Res_i \cdot ewt_i)^2 } < delta$.
    To obtain the \id{N\_Vector} \id{ewt}, call \id{CVodeGetErrWeights} 
    (see \S\ref{ss:optional_output}).
  \item[lr]
    is an input flag indicating whether the preconditioner solve
    function is to use the left preconditioner (\id{lr=1}) or 
    the right preconditioner (\id{lr=2});
  \item[P\_data]
    is a pointer to user data - the same as the \id{P\_data}      
    parameter passed to the function \id{CVSpgmrSetPrecData}.
  \item[tmp]
    is a pointer to memory allocated for a vector of        
    length \id{N} which can be used for work space.
  \end{args}
}
{
  The value to be returned by the preconditioner solve function is a flag indicating 
  whether it was successful.  This value should be $0$ if successful, 
  positive for a recoverable error (in which case the step will be retried),     
  negative for an unrecoverable error (in which case the integration is halted). 
}
{}
%%
%%-----------------
%%
\subsection{Preconditioning ({\spgmr} Jacobian data)}
\label{ss:precondFn}
\index{CVSPGMR@{\cvspgmr} linear solver!preconditioner setup function}
If the user's preconditioner requires that any Jacobian related data
be evaluated or preprocessed, then this needs to be done in a
user-supplied {\C} function of type \ID{CVSpgmrPrecSetupFn} defined as follows:
\usfunction{CVSpgmrPrecSetupFn}
{
  typedef int (*CVSpgmrPrecSetupFn)(&realtype t, N\_Vector y, N\_Vector fy,  \\
                                    &booleantype jok, booleantype *jcurPtr, \\
                                    &realtype gamma, void *P\_data,\\
                                    &N\_Vector tmp1, N\_Vector tmp2,\\
                                    &N\_Vector tmp3);
}
{
  This function evaluates and/or preprocesses Jacobian related data needed
  by the preconditioner.
}
{
  The arguments of a \id{CVSpgmrPrecSetupFn} are as follows:
  \begin{args}[jcurPtr]
  \item[t]
    is the current value of the independent variable.
  \item[y]
    is the current value of the dependent variable vector, 
    namely the predicted value of $y(t)$.
  \item[fy]
    is the vector $f(t,y)$.                    
  \item[jok]
    is an input flag indicating whether Jacobian-related   
    data needs to be recomputed. The \id{jok} argument provides for 
    the re-use of Jacobian data in the preconditioner solve function.
    \id{jok == FALSE} means that Jacobian-related data   
    must be recomputed from scratch.                                 
    \id{jok == TRUE}  means that Jacobian data, if saved from 
    the previous call to this function, can be reused      
    (with the current value of \id{gamma}).            
    A call with \id{jok == TRUE} can only occur after   
    a call with \id{jok == FALSE}.
  \item[jcurPtr]
    is a pointer to an output integer flag which is        
    to be set to \id{TRUE} if Jacobian data was recomputed or   
    to \id{FALSE} if Jacobian data was not           
    recomputed, but saved data was reused.
  \item[gamma]
    is the scalar $\gamma$ appearing in the Newton matrix $M = I - \gamma P$.
  \item[P\_data]
    is a pointer to user data, the same as the \id{P\_data}      
    parameter passed to \id{CVSpgmrSetPrecData}.
  \item[tmp1]
  \item[tmp2]
  \item[tmp3]
    are pointers to memory allocated    
    for vectors of length \id{N} which can be used by           
    \id{CVSpgmrPrecSetupFn} as temporary storage or work space.    
  \end{args}
}
{
  The value to be returned by the preconditioner setup function is a flag indicating 
  whether it was successful.  This value should be $0$ if successful, 
  positive for a recoverable error (in which case the step will be retried),     
  negative for an unrecoverable error (in which case the integration is halted). 
}
{
  The operations performed by this function might include forming a crude 
  approximate Jacobian, and performing an LU factorization on the resulting            
  approximation to $M=I - \gamma J$.

  Each call to the preconditioner setup function is preceded by a call to     
  the \id{RhsFn} user function with the same \id{(t,y)} arguments.  
  Thus the preconditoner setup function can use any auxiliary data that is 
  computed and saved during the evaluation of the ODE right hand side.
  
  This function is not called in advance of every call to the preconditioner solve
  function, but rather is called only as often as needed to achieve convergence in the
  Newton iteration. 
}
%%===================================================================================
\section{Integration of pure quadrature equations}
%%===================================================================================

Note that if the system of ODEs contains {\em pure quadratures} it is more efficient
to treat them separately. To do this, exclude them from \id{y0} in the 
\id{CVodeMalloc} call (as well from within \id{f}) and instead use the following 
function to initialize quadrature integration with {\cvodes}:

%%===================================================================================

\subsection{Quadrature initialization function}\label{sss:cvodequadmalloc}

The function \ID{CVodeQuadMalloc} activates integration of quadrature equations 
and allocates internal memory related to these calculations.
The form of the call to this function is
\begin{verbatim}
ier = CVodeQuadMalloc(cvode_mem, fQ, errconQ, &rtolQ, atolQ, fQ_data, machEnvQ);  
\end{verbatim}
where \id{cvode\_mem} is the pointer to the {\cvodes} memory returned by \id{CVodeMalloc}
and \id{fQ} is a user-provided {\C} function to evaluate the righ-hand side of the
quadrature equations (for full details, see \S\ref{ss:user_fct_sim}). 
The flag \id{errconQ} specify whether partial or full error 
control is to be used. If \id{errconQ = }\Id{FULL} then both state variables and
quadrature variables are included in the error tests. 
If \id{errconQ = }\Id{PARTIAL} then the quadrature variables are excluded from the 
error tests. When full error control is used, \id{\&rtolQ} is the relative tolerance for
quadrature variables and \id{atolQ} is a pointer to a scalar absolute tolerance for
quadrature variables (if \id{itol = }\id{SS}) or to a vector of absolute tolerances 
(if \id{itol = }\id{SV}). The pointer to the user data \id{fQ\_data} is passed
directly to the user's \id{fQ} function. The final argument, \id{machEnvQ},
is a pointer to a machine environment structure corresponding to the
vector of quadrature variables, as obtained from a previous call to an
\id{M\_EnvInit\_*} function.

Note that there are no initial values for quadrature variables. The reason for this is 
that these variables are always $0$ at the initial time and {\cvodes} performs this
initialization by itself.

The return value \id{ier} of \id{CVodeQuadMalloc} is equal to \id{SUCCESS} if
there were no errors, or \Id{QCVM\_NO\_MEM} if \id{cvode\_mem} was NULL.  

%%===================================================================================

\subsection{Quadrature extraction function}\label{sss:cvodequadextract}

If forward sensitivity computations have been initialized by a call to \id{CVodeQuadMalloc},
or reinitialized by a call to \id{CVodeQuadReInit}, then {\cvodes} computes both solution
and quadratures at time \id{t}. However, \id{CVode} will still return only the solution
$y$ in \id{y}. Solution quadratures can be obtained through the function
\ID{CVodeQuadExtract}:
\begin{verbatim}
ier = CVodeQuadExtract(cvode_mem, t, yQ);
\end{verbatim}
Its arguments are as follows:
\begin{itemize}
\item \id{cvode\_mem} is the pointer to the memory previously allocated
  by \id{CVodeMalloc}.
\item \id{t} specifies the time at which quadrature information is 
  requested. The time \id{t} must fall within the interval defined by the last 
  succesful step taken by {\cvodes}.
\item \id{yQ} must be declared of type \id{N\_Vector}.
  If successful, \id{CVodeQuadExtract} will load \id{yQ} with the values of the
  solution quadratures at time \id{t}.
\end{itemize}
The return value \id{ier} of \id{CVodeQuadExtract} is equal to: 
\begin{itemize}
\item \Id{SUCCESS} if there were no errors; 
\item \Id{DKY\_NO\_MEM} if \id{cvode\_mem} was NULL;
\item \Id{DKY\_NO\_QUAD} if quadrature computation was not turned on
      by a call to \id{CVodeQuadMalloc};
\item \Id{BAD\_T} if the time \id{t} is not in the allowed range.
\end{itemize}
In case of an error return, an error message is also printed.  

%%===================================================================================

\subsection{Interpolated quadrature output function}

The function \ID{CVodeQuadDky} computes the \id{k}-th derivatives of the interpolating 
polynomials for the quadrature variables at time \id{t}.
This function is called by \id{CVodeQuadExtract} with \id{k} $= 0$, but may also be called 
directly by the user.
\begin{verbatim}
ier = CVodeQuadDky(cvode_mem, t, k, dkyQ)
\end{verbatim}
Its arguments and return value are the same as for \id{CVodeDky}.

%%===================================================================================

\subsection{Quadrature reinitialization function}\label{sss:cvqreinit}

The function \ID{CVodeQuadReInit}, useful during the solution of a sequence of problems of 
same size, reinitializes the quadrature related internal memory 
and must follow a call to \Id{CVodeQuadMalloc} (and maybe a call to \id{CVodeReInit}). 
The number \id{Nq} of quadratures is assumed to be unchanged since the call to 
\Id{CVodeQuadMalloc}.

The call to the \id{CVodeQuadReInit} function has the form
\begin{verbatim}
ier = CVodeQuadReInit(cvode_mem, fQ, errconQ, &rtolQ, atolQ, 
                      fQ_data, machEnvQ);
\end{verbatim}
The arguments have names and meanings identical to those of \id{CVodeQuadMalloc}.

The return value \id{ier} of \id{CVodeQuadReInit} is equal to: 
\begin{itemize}
\item \Id{SUCCESS} if there were no errors; 
\item \Id{QCVREI\_NO\_MEM} if \id{cvode\_mem} was NULL;
\item \Id{QCVREI\_NO\_QUAD} if quadrature computation was not turned on
      by a call to \id{CVodeQuadMalloc};
\item \Id{QCVREI\_ILL\_INPUT} if an input argument was illegal;
\end{itemize}
In case of an error return, an error message is also printed.  

\subsection{User-supplied functions for quadrature integration}

\index{right hand side function|quadarture equations|(}
For integration of quadrature equations, the user must provide a function of
type \ID{QuadRhsFn} defined by
\begin{verbatim}
typedef void (*QuadRhsFn)(realtype t, N_Vector y, N_Vector yQdot, 
                          void *fQ_data);
\end{verbatim}
to compute the right-hand side of these equations.

This function takes as input the independent variable value \id{t}
and the dependent variable vector \id{y}. It must store the
result of $fQ(t,y)$ in the vector \id{yQdot}. 
The \id{fQ\_data} parameter is the same as the \id{fQ\_data} parameter passed by 
the user to the \id{CVodeQuadMalloc} function. This user-supplied pointer is passed to 
the user's \id{fQ} function every time it is called.

Both \id{y} and \id{yQdot} arguments are of type \id{N\_Vector},
but they  typically all have different internal representations. It is the user's 
responsibility to access the vector data consistently (including the use of the 
correct accessor macros from each {\nvector} implementation). For the sake of 
computational efficiency, the vector kernels in the two {\nvector} implementations 
provided with {\cvodes} do not perform any consistency checking for their 
\id{N\_Vector} arguments (see \S\ref{ss:nvec_ser} and \S\ref{ss:nvec_par}).

A \id{QuadRhsFn} function type does not have a return value.
\index{right hand side function|quadarture equations|)}

%%===================================================================================
%%
\section{{\cvodes} preconditioner modules}\label{ss:preconds}
%%
%%===================================================================================

The efficiency of Krylov iterative methods for the solution of linear systems 
can be greatly enhanced through preconditioning. For problems in which the 
user cannot define a more effective, problem-specific preconditioner,
{\cvodes} provides a banded preconditioner in the module {\cvbandpre} and
a band-block-diagonal preconditioner module {\cvbbdpre}.

%%===================================================================================
\subsection{A serial banded preconditioner module}\label{sss:cvbandpre}
%%===================================================================================

\index{CVBANDPRE@{\cvbandpre} preconditioner!description}
This preconditioner provides a band matrix preconditioner based on
difference quotients of the ODE right-hand side function \id{f}.
It generates a band matrix of bandwidth $m_l + m_u + 1$, where
the number of super-diagonals ($m_u$, the upper half-bandwidth) and
sub-diagonals ($m_l$, the lower half-bandwidth) are specified by
the user and uses this to form a preconditioner for use with the Krylov
linear solver in {\cvspgmr}.  Although this matrix is intended
to approximate the Jacobian $\partial f / \partial y$, 
it may be a very crude approximation.  The true Jacobian need not be banded, or its
true bandwith may be larger than $m_l + m_u + 1$, as long as the
banded approximation generated here is sufficiently accurate
to speed convergence as a preconditioner. 

%%------------------------------------------------------------------------------------
%%
\index{CVBANDPRE@{\cvbandpre} preconditioner!usage|(}
In order to use the {\cvbandpre} module, the user need not define any
additional functions. 
%%
Besides the header files required for the integration of the ODE problem
(see \S\ref{ss:header_sim}),  to use the {\cvbandpre} module, the main program 
must include the header file \id{cvsbandpre.h} which declares the needed
function prototypes.
%%
The following is a summary of the usage of this module and describes the sequence 
of calls in the user main program. Steps that are unchanged from the user main
program presented in \S\ref{ss:skeleton_sim} are grayed-out.
%%
%%
\begin{Steps}
  
\item
  \textcolor{gray}{\bf Set problem dimensions}

\item
  \textcolor{gray}{\bf Initialize vector specification}

\item
  \textcolor{gray}{\bf Set intial values}
 
\item
  \textcolor{gray}{\bf Create {\cvodes} object}

\item
  \textcolor{gray}{\bf Set optional inputs}

\item
  \textcolor{gray}{\bf Allocate internal memory}

\item \label{i:bandpre_init}
  {\bf Initialize the {\cvbandpre} preconditioner module}

  Specify the upper and lower half-bandwidths \id{mu} and \id{ml} and call 

  \id{bp\_data = CVBandPrecAlloc(cvode\_mem, N, mu, ml);} 

  to allocate memory for and initialize a data structure \id{bp\_data} to be 
  passed to the {\cvspgmr} linear solver.

\item \label{i:bandpre_attach}
  {\bf Attach the {\cvspgmr} linear solver}

  \id{flag = CVBPSpgmr(cvode\_mem, pretype, maxl, bp\_data);}

  The function \Id{CVBPSpgmr} is a wrapper around the {\cvspgmr} specification
  function \id{CVSpgmr} and performs the following actions:
  \begin{itemize}
    \item Attaches the {\cvspgmr} linear solver to the main {\cvodes} solver memory;
    \item Sets the preconditioner data structure for {\cvbandpre};
    \item Sets the preconditioner setup function for {\cvbandpre};
    \item Sets the preconditioner solve function for {\cvbandpre};
  \end{itemize}
  The arguments \id{pretype} and \id{maxl} are described below.
  The last argument of \id{CVBPSpgmr} is the pointer to the {\cvbandpre} data
  returned by \id{CVBandPrecAlloc}.

\item
  \textcolor{gray}{\bf Set linear solver optional inputs}

  Note that the user should not overwrite the preconditioner data, setup function, 
  or solve function through calls to {\cvspgmr} optional input functions.

\item
  \textcolor{gray}{\bf Advance solution in time}

\item
  \textcolor{gray}{\bf Deallocate memory for solution vector}

\item \label{i:bandpre_free}
  {\bf Free the {\cvbandpre} data structure}

  \id{CVBandPrecFree(bp\_data);}

\item
  \textcolor{gray}{\bf Free solver memory}
  
\item
  \textcolor{gray}{\bf Free vector specification memory}

\end{Steps}
%%
%%
\index{CVBANDPRE@{\cvbandpre} preconditioner!usage|)}
%%
%%------------------------------------------------------------------------------------
%%
\index{CVBANDPRE@{\cvbandpre} preconditioner!user-callable functions|(}
The three user-callable functions that initialize, attach, and deallocate
the {\cvbandpre} preconditioner module (steps \ref{i:bandpre_init},
\ref{i:bandpre_attach}, and \ref{i:bandpre_free} above) are described
in more detail below.
%%
\ucfunction{CVBandPrecAlloc}
{
  bp\_data = CVBandPrecAlloc(cvode\_mem, N, mu, ml);
}
{
  The function \ID{CVBandPrecAlloc} initializes and allocates
  memory for the {\cvbandpre} preconditioner.
}
{
  \begin{args}[cvode\_mem]
  \item[cvode\_mem] (\id{void *})
    pointer to the {\cvodes} memory block.
  \item[N] (\id{integertype})
    problem dimension.
  \item[mu] (\id{integertype})
    upper half-bandwidth of the problem Jacobian approximation.
  \item[ml] (\id{integertype})
    lower half-bandwidth of the problem Jacobian approximation.
  \end{args}
}
{
  If successful, \id{CVBandPrecAlloc} returns a pointer to the newly created 
  {\cvbandpre} memory block (of type \id{void *}).
  If an error occured, \id{CVBandPrecAlloc} returns \id{NULL}.
}
{
  The banded approximate Jacobian will have nonzeros only in location $(i,j)$
  with $-$\id{ml} $\leq j-i \leq$ \id{mu}.
}
%%
%%
\ucfunction{CVBPSpgmr}
{
  flag = CVBPSpgmr(cvode\_mem, pretype, maxl, bp\_data);
}
{
  The function \ID{CVBPSpgmr} links the {\cvbandpre} data to the
  {\cvspgmr} linear solver and attaches the latter to the {\cvodes}
  memory block.
}
{
  \begin{args}[cvode\_mem]
  \item[cvode\_mem] (\id{void *})
    pointer to the {\cvodes} memory block.
  \item[pretype] (\id{int})
    preconditioning type. Can be one of \Id{LEFT} or \Id{RIGHT}.
  \item[maxl] (\id{int})
    maximum dimension of the Krylov subspace to be used. Pass $0$ to use the 
    default value \id{CVSPGMR\_MAXL}$=5$.
  \item[bp\_data] (\id{void *})
    pointer to the {\cvbandpre} data structure.
  \end{args}
}
{
  The return value \id{flag} (of type \id{int}) is one of
  \begin{args}[LIN\_ILL\_INPUT]
  \item[\Id{SUCCESS}] 
    The {\cvspgmr} initialization was successful.
  \item[\Id{LIN\_NO\_MEM}]
    The \id{cvode\_mem} pointer is \id{NULL}.
  \item[\Id{LIN\_ILL\_INPUT}]
    The preconditioner type \id{pretype} is not valid.
  \item[\Id{LMEM\_FAIL}]
    A memory allocation request failed.
  \item[\Id{BP\_NO\_DATA}]
    The {\cvbandpre} preconditioner has not been initialized.
  \end{args}
}
{}
%%
\ucfunction{CVBandPrecFree}
{
  CVBandPrecFree(bp\_data);
}
{
  The function \ID{CVBandPrecFree} frees the pointer allocated by
  \id{CVBandPrecAlloc}.
}
{
  The only argument of \id{CVBandPrecFree} is the pointer to the {\cvbandpre} 
  data structure (of type \id{void *}).
}
{
  The function \id{CVBandPrecFree} has no return value.
}
{}
%%
\index{CVBANDPRE@{\cvbandpre} preconditioner!user-callable functions|)}
%%
%%------------------------------------------------------------------------------------
%%
\index{optional output!banded preconditioner|(}
\index{CVBANDPRE@{\cvbandpre} preconditioner!optional output|(}
The following three optional output functions are available for use with 
the {\cvbandpre} module:
%%
\index{memory requirements!CVBANDPRE@{\cvbandpre} preconditioner|(}
\ucfunction{CVBandPrecGetIntWorkSpace}
{
  flag = CVBandPrecGetIntWorkSpace(bp\_data, leniwBP);
}
{
  The function \ID{CVBandPrecGetIntWorkSpace} returns the
  {\cvbandpre} integer workspace size.
}
{
  \begin{args}[leniwBP]
  \item[bp\_data] (\id{void *})
    pointer to the {\cvbandpre} data structure.
  \item[leniwBP] (\id{long int *})
    the number of \id{integertype} values in the {\cvbandpre} workspace.
  \end{args}
}
{
  The return value \id{flag} (of type \id{int}) is one of
  \begin{args}[BP\_NO\_DATA]
  \item[OKAY] 
    The optional output value has been successfuly set.
  \item[\Id{BP\_NO\_DATA}]
    The {\cvbandpre} preconditioner has not been initialized.
  \end{args}
}
{
  In terms of the problem size $N$, the actual sizes of the integer workspace
  is $N$ \id{integertype} words.
}
%%
%%
\ucfunction{CVBandPrecGetRealWorkSpace}
{
  flag = CVBandPrecGetRealWorkSpace(bp\_data, lenrwBP);
}
{
  The function \ID{CVBandPrecGetRealWorkSpace} returns the
  {\cvbandpre} real workspace size.
}
{
  \begin{args}[lenrwBP]
  \item[bp\_data] (\id{void *})
    pointer to the {\cvbandpre} data structure.
  \item[lenrwBP] (\id{long int *})
    the number of \id{realtype} values in the {\cvbandpre} workspace.
  \end{args}
}
{
  The return value \id{flag} (of type \id{int}) is one of
  \begin{args}[BP\_NO\_DATA]
  \item[OKAY] 
    The optional output value has been successfuly set.
  \item[\Id{BP\_NO\_DATA}]
    The {\cvbandpre} preconditioner has not been initialized.
  \end{args}
}
{
  In terms of the problem size $N$, the actual sizes of the real workspace is
  $N\,(2$ \id{ml} $+$ \id{mu} $+$ \id{smu} $+2)$ \id{realtype} words,
  where \id{smu} = $\min(N-1 ,$ \id{mu} $+$ \id{ml}$)$.
}
%%
\index{memory requirements!CVBANDPRE@{\cvbandpre} preconditioner|)}
%%
\ucfunction{CVBandPrecGetNumRhsEvals}
{
  flag = CVBandPrecGetNumRhsEvals(bp\_data, nfevalsBP);
}
{
  The function \ID{CVBandPrecGetNumRhsEvals} returns the
  number of calls to the user right-hand side function for
  finite difference banded Jacobian approximation used within
  {\cvbandpre}'s preconditioner setup function.
}
{
  \begin{args}[nfevalsBP]
  \item[bp\_data] (\id{void *})
    pointer to the {\cvbandpre} data structure.
  \item[nfevalsBP] (\id{long int *})
    the number of calls to the user right-hand side function.
  \end{args}
}
{
  The return value \id{flag} (of type \id{int}) is one of
  \begin{args}[BP\_NO\_DATA]
  \item[OKAY] 
    The optional output value has been successfuly set.
  \item[\Id{BP\_NO\_DATA}]
    The {\cvbandpre} preconditioner has not been initialized.
  \end{args}
}
{}
%%
\index{CVBANDPRE@{\cvbandpre} preconditioner!optional output|)}
\index{optional output!banded preconditioner|)}
%%
%%
%-------------------------------------------------------

%%===================================================================================
\subsection{A parallel band-block-diagonal preconditioner module}\label{sss:cvbbdpre}
%%===================================================================================

A principal reason for using a parallel ODE solver such as {\cvodes} lies
in the solution of partial differential equations (PDEs).  Moreover,
the use of a Krylov iterative method for the solution of many such
problems is motivated by the nature of the underlying linear system of
equations (\ref{e:Newton}) that must be solved at each time step.  The
linear algebraic system is large, sparse, and structured. However, if
a Krylov iterative method is to be effective in this setting, then a
nontrivial preconditioner needs to be used.  Otherwise, the rate of
convergence of the Krylov iterative method is usually unacceptably
slow.  Unfortunately, an effective preconditioner tends to be
problem-specific.

However, we have developed one type of preconditioner that treats a
rather broad class of PDE-based problems.  It has been successfully
used for several realistic, large-scale problems \cite{HiTa:98} and is
included in a software module within the {\cvodes} package. This module
works with the parallel vector module {\nvecp} and 
generates a preconditioner that is a block-diagonal matrix with each
block being a band matrix. The blocks need not have the same number of
super- and sub-diagonals and these numbers may vary from block to
block. This Band-Block-Diagonal Preconditioner module is called
{\cvbbdpre}.

\index{CVBBDPRE@{\cvbbdpre} preconditioner!description|(}
One way to envision these preconditioners is to think of the domain of
the computational PDE problem as being subdivided into $M$ non-overlapping
subdomains.  Each of these subdomains is then assigned to one of the
$M$ processors to be used to solve the ODE system. The basic idea is
to isolate the preconditioning so that it is local to each processor,
and also to use a (possibly cheaper) approximate right-hand side
function. This requires the definition of a new function $g(t,y)$
which approximates the function $f(t, y)$ in the definition of the ODE
system (\ref{e:ivp}). However, the user may set $g = f$.  Corresponding
to the domain decomposition, there is a decomposition of the solution
vector $y$ into $M$ disjoint blocks $y_m$, and a decomposition of $g$
into blocks $g_m$.  The block $g_m$ depends on $y_m$ and also on
components of blocks $y_{m'}$ associated with neighboring subdomains
(so-called ghost-cell data).  Let $\bar{y}_m$ denote $y_m$ augmented
with those other components on which $g_m$ depends.  Then we have
\begin{equation}
  g(t,y) = [g_1(t,\bar{y}_1), g_2(t,\bar{y}_2), \ldots, g_M(t,\bar{y}_M)]^T
\end{equation}
and each of the blocks $g_m(t, \bar{y}_m)$ is uncoupled from the others.

The preconditioner associated with this decomposition has the form 
\begin{equation}
  P= diag[P_1, P_2, \ldots, P_M]
\end{equation}
where 
\begin{equation}
  P_m \approx I - \gamma J_m
\end{equation}
and $J_m$ is a difference quotient approximation to 
$\partial g_m/\partial y_m$. This matrix is taken to be banded, with
upper and lower half-bandwidths \id{mudq} and \id{mldq} defined as
the number of non-zero diagonals above and below the main diagonal,
respectively. The difference quotient approximation is computed using
\id{mudq} $+$ \id{mldq} $+ 2$ evaluations of $g_m$, but only a matrix
of bandwidth \id{mu} $+$ \id{ml} $+ 1$ is retained. 
Neither pair of parameters need be the true half-bandwidths of the Jacobian of the
local block of $g$, if smaller values provide a more efficient
preconditioner. The solution of the complete linear system
\begin{equation}
  Px = b
\end{equation}
reduces to solving each of the equations 
\begin{equation}
  P_m x_m = b_m
\end{equation}
and this is done by banded LU factorization of $P_m$ followed by a banded
backsolve.
\index{CVBBDPRE@{\cvbbdpre} preconditioner!description|)}

%%
%%------------------------------------------------------------------------------------
%%

\index{CVBBDPRE@{\cvbbdpre} preconditioner!user-supplied functions|(}
To use this {\cvbbdpre} module, the user must supply two functions which the
module calls to construct $P$: a function \id{gloc} (of type \id{CVLocalFn}) 
which approximates the right-hand side function $g(t,y) \approx f(t,y)$ and which 
is computed locally, and a function \id{cfn} (of type \id{CVCommFn}) which performs 
all inter-process communication necessary to evaluate the approximate right-hand side $g$.
These are in addition to the user-supplied right-hand side function \id{f}.
Both functions take as input the same pointer \id{f\_data} as that passed
by the user to \id{CVodeSetFdata} and passed to the user's function \id{f},
and neither function has a return value. The user is responsible for
providing space (presumably within \id{f\_data}) for components of \id{y}
that are communicated by \id{cfn} from the other processors, and that are
then used by \id{gloc}, which is not expected to do any communication.
%%
%%
\usfunction{CVLocalFn}
{
  typedef void (*CVLocalFn)(&integertype Nlocal, realtype t,  \\
                            &N\_Vector y, N\_Vector glocal, \\
                            &void *f\_data);
}
{
  This function computes $g(t,y)$. It loads the vector
  \id{glocal} as a function of \id{t} and \id{y}.  
}
{
  \begin{args}[Nlocal]
  \item[Nlocal] 
    is the local vector length.
  \item[t]
    is the value of the independent variable.
  \item[y]
    is the dependent variable. 
  \item[glocal]
    is the output vector.
  \item[f\_data]
    is a pointer to user data - the same as the \id{f\_data}      
    parameter passed to \id{CVodeSetFdata}.  
  \end{args}
}
{
  A \id{CVLocalFn} function type does not have a return value.
}
{
  This routine assumes that all inter-processor communication of data needed to 
  calculate \id{glocal} has already been done, and this data is accessible within
  \id{f\_data}.

  The case where $g$ is mathematically identical to $f$ is allowed.
}
%%
%%
\usfunction{CVCommFn}
{
  typedef void (*CVCommFn)(&integertype Nlocal, realtype t,  \\
                           &N\_Vector y, void *f\_data);
}
{
  This function performs all inter-processor communications necessary 
  for the execution of the \id{gloc} function above, using the input vector \id{y}.
}
{
  \begin{args}[Nlocal]
  \item[Nlocal] 
    is the local vector length.
  \item[t]
    is the value of the independent variable.
  \item[y]
    is the dependent variable. 
  \item[f\_data]
    is a pointer to user data - the same as the \id{f\_data}      
    parameter passed to \id{CVodeSetFdata}.  
  \end{args}
}
{
  A \id{CVCommFn} function type does not have a return value.
}
{
  The \id{cfn} function is expected to save communicated data in space defined within the
  structure \id{f\_data}. 

  Each call to the \id{cfn} function is preceded by a call to the right-hand side
  function \id{f} with the same (\id{t}, \id{y}) arguments.  Thus \id{cfn} can omit 
  any communications done by \id{f} if relevant to the evaluation of \id{glocal}.  
}
%%
\index{CVBBDPRE@{\cvbbdpre} preconditioner!user-supplied functions|)}

%%
%%------------------------------------------------------------------------------------
%%

\index{CVBBDPRE@{\cvbbdpre} preconditioner!usage|(}
%%
Besides the header files required for the integration of the ODE problem
(see \S\ref{ss:header_sim}),  to use the {\cvbbdpre} module, the main program 
must include the header file \id{cvsbbdpre.h} which declares the needed
function prototypes.

The following is a summary of the usage of this module and describes the sequence 
of calls in the user main program. Steps that are unchanged from the user main
program presented in \S\ref{ss:skeleton_sim} are grayed-out.
%%
%%
\begin{Steps}
\item 
  \textcolor{gray}{\bf Initialize MPI}

\item
  \textcolor{gray}{\bf Set problem dimensions}

\item
  \textcolor{gray}{\bf Initialize vector specification}

\item
  \textcolor{gray}{\bf Set intial values}
 
\item
  \textcolor{gray}{\bf Create {\cvodes} object}

\item
  \textcolor{gray}{\bf Set optional inputs}

\item
  \textcolor{gray}{\bf Allocate internal memory}

\item \label{i:bbdpre_init}
  {\bf Initialize the {\cvbbdpre} preconditioner module}

  Specify the upper and lower bandwidths \id{mudq}, \id{mldq} and
  \id{mukeep}, \id{mlkeep} and call 

   \id{
     \begin{tabular}[t]{@{}r@{}l@{}}
       bbd\_data = CVBBDPrecAlloc(&cvode\_mem, local\_N, mudq, mldq, \\
                                  &mukeep, mlkeep, dqrely, gloc, cfn);
     \end{tabular}
   }

  to allocate memory for and initialize a data structure \id{bbd\_data} to be 
  passed to the {\cvspgmr} linear solver. The last two arguments of \id{CVBBDPrecAlloc}
  are the two mandatory user-supplied functions described above.

\item \label{i:bbdpre_attach}
  {\bf Attach the {\cvspgmr} linear solver}

  \id{flag = CVBBDSpgmr(cvode\_mem, pretype, maxl, bbd\_data);}

  The function \Id{CVBPSpgmr} is a wrapper around the {\cvspgmr} specification
  function \id{CVSpgmr} and performs the following actions:
  \begin{itemize}
    \item Attaches the {\cvspgmr} linear solver to the main {\cvodes} solver memory;
    \item Sets the preconditioner data structure for {\cvbbdpre};
    \item Sets the preconditioner setup function for {\cvbbdpre};
    \item Sets the preconditioner solve function for {\cvbbdpre};
  \end{itemize}
  The arguments \id{pretype} and \id{maxl} are described below.
  The last argument of \id{CVBPSpgmr} is the pointer to the {\cvbbdpre} data
  returned by \id{CVBBDPrecAlloc}.

\item
  \textcolor{gray}{\bf Set linear solver optional inputs}

  Note that the user should not overwrite the preconditioner data, setup function, 
  or solve function through calls to {\cvspgmr} optional input functions.

\item
  \textcolor{gray}{\bf Advance solution in time}

\item
  \textcolor{gray}{\bf Deallocate memory for solution vector}

\item \label{i:bbdpre_free}
  {\bf Free the {\cvbbdpre} data structure}

  \id{CVBBDPrecFree(bbd\_data);}

\item
  \textcolor{gray}{\bf Free solver memory}
  
\item
  \textcolor{gray}{\bf Free vector specification memory}

\item 
  \textcolor{gray}{\bf Finalize MPI}

\end{Steps}
%%
\index{CVBBDPRE@{\cvbbdpre} preconditioner!usage|)}
%%
%%------------------------------------------------------------------------------------
%%
\index{CVBBDPRE@{\cvbbdpre} preconditioner!user-callable functions|(}
%%
The three user-callable functions that initialize, attach, and deallocate
the {\cvbbdpre} preconditioner module (steps \ref{i:bbdpre_init},
\ref{i:bbdpre_attach}, and \ref{i:bbdpre_free} above) are described
next.
%%
\ucfunction{CVBBDPrecAlloc}
{
   \begin{tabular}[t]{@{}r@{}l@{}}
     bbd\_data = CVBBDPrecAlloc(&cvode\_mem, local\_N, mudq, mldq, \\
                                &mukeep, mlkeep, dqrely, gloc, cfn);
   \end{tabular}
}
{
  The function \ID{CVBBDPrecAlloc} initializes and allocates
  memory for the {\cvbbdpre} preconditioner.
}
{
  \begin{args}[cvode\_mem]
  \item[cvode\_mem] (\id{void *})
    pointer to the {\cvodes} memory block.
  \item[local\_N] (\id{integertype})
    local vector dimension.
  \item[mudq] (\id{integertype})
    upper half-bandwidth to be used in the difference-quotient Jacobian approximation.
  \item[mldq] (\id{integertype})
    lower half-bandwidth to be used in the difference-quotient Jacobian approximation.
  \item[mukeep] (\id{integertype})
    upper half-bandwidth of the retained banded approximate Jacobian block.
  \item[mlkeep] (\id{integertype})
    lower half-bandwidth of the retained banded approximate Jacobian block.
  \item[dqrely] (\id{realtype})
    the relative increment in components of \id{y} used in the difference quotient
    approximations.  The default is \id{dqrely}$ = \sqrt{\text{unit roundoff}}$, which
    can be specified by passing \id{dqrely}$ = 0.0$.
  \item[gloc] (\id{CVLocalFn})
    the {\C} function which computes the approximation $g(t,y) \approx f(t,y)$. 
  \item[cfn] (\id{CVCommFn})
    the {\C} function which performs all inter-process communication required for
    the computation of $g(t,y)$.
  \end{args}
}
{
  If successful, \id{CVBBDPrecAlloc} returns a pointer to the newly created 
  {\cvbbdpre} memory block (of type \id{void *}).
  If an error occured, \id{CVBBDPrecAlloc} returns \id{NULL}.
}
{
  If one of the half-bandwidth \id{mudq} or \id{mldq} to be used in the 
  difference-quotient calculation of the approximate Jacobian is negative or 
  exceeds the value \id{local\_N}$-1$, it is replaced with 0.

  The half-bandwiths \id{mudq} and \id{mldq} need not be the true 
  half-bandwidths of the Jacobian of the local block of $g$,    
  when smaller values may provide a greater efficiency.       

  Also, the half-bandwidths \id{mukeep} and \id{mlkeep} of the retained 
  banded approximate Jacobian block may be even smaller,      
  to reduce storage and computation costs further.            

  For all four half-bandwidths, the values need not be the    
  same on every processor.
}
%%
%%
\ucfunction{CVBBDSpgmr}
{
  flag = CVBBDSpgmr(cvode\_mem, pretype, maxl, bbd\_data);
}
{
  The function \ID{CVBBDSpgmr} links the {\cvbbdpre} data to the
  {\cvspgmr} linear solver and attaches the latter to the {\cvodes}
  memory block.
}
{
  \begin{args}[cvode\_mem]
  \item[cvode\_mem] (\id{void *})
    pointer to the {\cvodes} memory block.
  \item[pretype] (\id{int})
    preconditioning type. Can be one of \Id{LEFT} or \Id{RIGHT}.
  \item[maxl] (\id{int})
    maximum dimension of the Krylov subspace to be used. Pass $0$ to use the 
    default value \id{CVSPGMR\_MAXL}$=5$.
  \item[bbd\_data] (\id{void *})
    pointer to the {\cvbbdpre} data structure.
  \end{args}
}
{
  The return value \id{flag} (of type \id{int}) is one of
  \begin{args}[LIN\_ILL\_INPUT]
  \item[\Id{SUCCESS}] 
    The {\cvspgmr} initialization was successful.
  \item[\Id{LIN\_NO\_MEM}]
    The \id{cvode\_mem} pointer is \id{NULL}.
  \item[\Id{LIN\_ILL\_INPUT}]
    The preconditioner type \id{pretype} is not valid.
  \item[\Id{LMEM\_FAIL}]
    A memory allocation request failed.
  \item[\Id{BBDP\_NO\_DATA}]
    The {\cvbbdpre} preconditioner has not been initialized.
  \end{args}
}
{}
%%
\ucfunction{CVBBDPrecFree}
{
  CVBBDPrecFree(bbd\_data);
}
{
  The function \ID{CVBBDPrecFree} frees the pointer allocated by
  \id{CVBBDPrecAlloc}.
}
{
  The only argument of \id{CVBBDPrecFree} is the pointer to the {\cvbbdpre} 
  data structure (of type \id{void *}).
}
{
  The function \id{CVBBDPrecFree} has no return value.
}
{}
%%
The {\cvbbdpre} module also provides a reinitialization routine to allow
solving  a sequence of problems of the same size with {\cvspgmr}/{\cvbbdpre},
provided there is no change in \id{local\_N}, \id{mukeep}, or \id{mlkeep}.
After solving one problem, and after calling \id{CVodeReInit} to re-initialize 
{\cvodes} for a subsequent problem, a call to \id{CVBBDPrecReInit} can be made
to change any of the following: the half-bandwidths \id{mudq} and \id{mldq} 
used in the difference-quotient Jacobian approximations, the relative increment \id{dqrely}, 
or one of the user-supplied functions \id{gloc} and \id{cfn}.
%%
\ucfunction{CVBBDPrecReInit}
{
  flag = CVBBDPrecReInit(bbd\_data, mudq, mldq, dqrely, gloc, cfn);
}
{
  The function \ID{CVBBDPrecReInit} reinitializes the {\cvbbdpre} preconditioner.
}
{
  \begin{args}[bbd\_data]
  \item[bbd\_data] (\id{void *})
    pointer to the {\cvbbdpre} data structure.
  \item[mudq] (\id{integertype})
    upper half-bandwidth to be used in the difference-quotient Jacobian approximation.
  \item[mldq] (\id{integertype})
    lower half-bandwidth to be used in the difference-quotient Jacobian approximation.
  \item[dqrely] (\id{realtype})
    the relative increment in components of \id{y} used in the difference quotient
    approximations.  The default is \id{dqrely} $= \sqrt{\text{unit roundoff}}$, which
    can be specified by passing \id{dqrely} $= 0.0$.
  \item[gloc] (\id{CVLocalFn})
    the {\C} function which computes the approximation $g(t,y) \approx f(t,y)$. 
  \item[cfn] (\id{CVCommFn})
    the {\C} function which performs all inter-process communication required for
    the computation of $g(t,y)$.
  \end{args}
}
{
  The return value of \id{CVBBDPrecReInit} is always \Id{SUCCESS}.
}
{
  If one of the half-bandwidth \id{mudq} or \id{mldq} is negative or it
  exceeds the value \id{local\_N}$-1$, it is replaced with 0.
}
%%
\index{CVBBDPRE@{\cvbbdpre} preconditioner!user-callable functions|)}
%%
%%------------------------------------------------------------------------------------
%%
\index{optional output!band-block-diagonal preconditioner|(}
\index{CVBBDPRE@{\cvbbdpre} preconditioner!optional output|(}
The following three optional output functions are available for use with
the {\cvbbdpre} module:
%%
\index{memory requirements!CVBBDPRE@{\cvbbdpre} preconditioner|(}
\ucfunction{CVBBDPrecGetIntWorkSpace}
{
  flag = CVBBDPrecGetIntWorkSpace(bbd\_data, leniwBP);
}
{
  The function \ID{CVBBDPrecGetIntWorkSpace} returns the
  {\cvbbdpre} integer workspace size.
}
{
  \begin{args}[leniwBBDP]
  \item[bbd\_data] (\id{void *})
    pointer to the {\cvbbdpre} data structure.
  \item[leniwBBDP] (\id{long int *})
    the number of \id{integertype} values in the {\cvbbdpre} workspace.
  \end{args}
}
{
  The return value \id{flag} (of type \id{int}) is one of
  \begin{args}[BBDP\_NO\_DATA]
  \item[OKAY] 
    The optional output value has been successfuly set.
  \item[\Id{BBDP\_NO\_DATA}]
    The {\cvbbdpre} preconditioner has not been initialized.
  \end{args}
}
{
  In terms of the local vector dimension $N_{l}$, the actual size of the integer workspace
  is $N_l$ \id{integertype} words.
}
%%
%%
\ucfunction{CVBBDPrecGetRealWorkSpace}
{
  flag = CVBBDPrecGetRealWorkSpace(bbd\_data, lenrwBBDP);
}
{
  The function \ID{CVBBDPrecGetRealWorkSpace} returns the
  {\cvbbdpre} real workspace size.
}
{
  \begin{args}[lenrwBBDP]
  \item[bbd\_data] (\id{void *})
    pointer to the {\cvbbdpre} data structure.
  \item[lenrwBBDP] (\id{long int *})
    the number of \id{realtype} values in the {\cvbbdpre} workspace.
  \end{args}
}
{
  The return value \id{flag} (of type \id{int}) is one of
  \begin{args}[BBDP\_NO\_DATA]
  \item[OKAY] 
    The optional output value has been successfuly set.
  \item[\Id{BBDP\_NO\_DATA}]
    The {\cvbbdpre} preconditioner has not been initialized.
  \end{args}
}
{
  In terms of the local vector dimension $N_l$, the actual size of the real workspace is
  $N_l \,(2$ \id{mlkeep} $+$ \id{mukeep} $+$ \id{smu} $+2)$ \id{realtype} words,
  where \id{smu} = $\min ( N_l - 1 ,$ \id{mukeep} $+$ \id{mlkeep}$)$.
}
%%
\index{memory requirements!CVBBDPRE@{\cvbbdpre} preconditioner|)}
%%
\ucfunction{CVBBDPrecGetNumGfnEvals}
{
  flag = CVBBDPrecGetNumGfnEvals(bbd\_data, ngevalsBBDP);
}
{
  The function \ID{CVBBDPrecGetNumGfnEvals} returns the
  number of calls to the user \id{gloc} function due to the 
  finite difference approximaton of the Jacobian blocks used within
  {\cvbbdpre}'s preconditioner setup function.
}
{
  \begin{args}[ngevalsBBDP]
  \item[bbd\_data] (\id{void *})
    pointer to the {\cvbbdpre} data structure.
  \item[ngevalsBBDP] (\id{long int *})
    the number of calls to the user \id{gloc} function.
  \end{args}
}
{
  The return value \id{flag} (of type \id{int}) is one of
  \begin{args}[BBDP\_NO\_DATA]
  \item[OKAY] 
    the optional output value has been successfuly set.
  \item[\Id{BBDP\_NO\_DATA}]
    The {\cvbbdpre} preconditioner has not been initialized.
  \end{args}
}
{}
%%
\index{CVBBDPRE@{\cvbbdpre} preconditioner!optional output|)}
\index{optional output!band-block-diagonal preconditioner|)}
%%
%%
%-------------------------------------------------------

The costs associated with {\cvbbdpre} also include \id{nlinsetups} LU
factorizations, \id{nlinsetups} calls to \id{cfn}, and \id{npsolves} banded
backsolve calls, where \id{nlinsetups} and \id{npsolves} are optional {\cvodes}
outputs (see \S\ref{ss:optional_output}).

Similar block-diagonal preconditioners could be considered with different
treatment of the blocks $P_m$. For example, incomplete LU factorization or
an iterative method could be used instead of banded LU factorization.

