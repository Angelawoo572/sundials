%===================================================================================
\section{Using {\cvodes} for Forward Sensitivity Analysis}\label{s:forward}
%===================================================================================

This section describes the use of {\cvodes} to compute solution sensitivities using
forward sensitivity analysis. One of our main guiding principles was to design 
the {\cvodes} user interface for forward sensitivity analysis as an extension of
that for IVP integration. Assuming a user main program and user-defined support 
routines for IVP integration have already been defined, in order to perform 
forward sensitivity analysis the user only has to insert a few more calls 
into the main program and (optionally) define an additional routine which
computes the right hand side of the sensitivity systems (\ref{e:sens_eqns}). 
The only departure from this philosophy is due to the \id{RhsFn} type definition
(\S\ref{ss:user_fct_sim}). Without changing the definition of this type, the
only way to pass values of the problem parameters to the ODE right hand side
function is to require the user data structure \id{f\_data} to contain a pointer
to the array of real parameters $p$.

We begin with a brief overview, in the form of a skeleton user program.
Following that are detailed descriptions of the interface to the
various user-callable routines and of the user-supplied routines that were not already
described in \S\ref{s:simulation}.

%-------------------------------------------------
\subsection{A Skeleton of the User's Main Program}\label{ss:forward_usage}
%-------------------------------------------------

The following is a skeleton of the user's main program (or calling
program) as an application of {\cvodes}. The user program is to have these 
steps in the order indicated, unless otherwise noted.
For the sake of brevity, we defer many of the details to the later sections.
As in \S\ref{ss:skeleton_sim}, most steps are independent of the {\nvector}
implementation used; where this is not the case, usage specifications are given for the
two implementations provided with {\cvodes}: steps marked with {\p} correspond to 
{\nvecp}, while steps marked with {\s} correspond to {\nvecs}.
Differences between the user main program in \S\ref{ss:skeleton_sim} and
the one below start only at step (\ref{i:fwd_start}).

\begin{enumerate}
  
\item 
  Include relevant header files. No additional header files need be
  included for forward sensitivity analysis beyond those for IVP solution 
  (\S\ref{ss:skeleton_sim});
  
\item 
  {\p} If MPI is needed by the user code, call \id{MPI\_Init(\&argc, \&argv);}
  
\item
  Set problem dimensions (excluding sensitivity equations):

  {\s} set \id{N}; {\p} set \id{N} and \id{Nlocal};
  
\item 
  Initialize the machine environment block by calling the appropriate
  {\nvector} routine:

  {\s} \id{M\_EnvInit\_Serial}; {\p} \id{M\_EnvInit\_Parallel};
  
\item
  Set the vector \id{y0} of initial values;
  
\item
  Call \id{cvode\_mem = }\Id{CVodeMalloc}\id{()} to allocate internal memory for 
  {\cvodes} not related to forward sensitivity computations and initializes {\cvodes};
  
\item
  If Newton iteration is chosen, initialize the linear solver module by
  calling the appropriate initialization routine;
  
\item \label{i:fwd_start}
  Define the sensitivity problem (see \S\ref{sss:cvodesensmalloc} for more details)
  \begin{itemize}
  \item 
    Set \id{p}, an array of \id{Np} real parameters upon which the IVP depends 
    (both through its right-hand side and initial conditions). Only parameters
    with respect to which sensitivities are (potentially) desired need to 
    be included. Also set \id{pbar}, an array of \id{Np} scaling factors.
  \item
    Attach \id{p} to the user data structure \id{f\_data}. 
    For example, \id{f\_data->p = p;}
  \item
    Set \id{Ns}, the number of parameters with respect to which sensitivities
    are to be computed. \id{Ns} must not be larger than \id{Np};
  \item
    Set \id{plist}, an array of \id{Ns} integer flags to specify the parameters
    \id{p} with respect to which solution sensitivities are to be computed.
  \end{itemize}
  Note that the names for \id{p}, \id{pbar}, \id{plist}, as well as the field
  {\em p} of \id{f\_data} are arbitrary, but they must agree with the arguments
  to \id{CVodeSensMalloc} below;

\item
  Set the \id{Ns} vectors \id{yS0[i]} of \id{N} initial values
  for sensitivities (for $i=0,\ldots,Ns-1$). 
  If an existing data array \id{ySdata} (of type \id{realtype**} and pointing to
  {\tt Ns} vectors of length {\tt N} each) contains the initial values
  {\tt yS0}, then use a macro of type \id{NVS\_MAKE} defined by the current 
  {\nvector} implementation:
  \begin{itemize}
  \item {\s} 
    \id{NVS\_MAKE\_S(Ns, yS0, ySdata, machEnv);}
  \item {\p} 
    \id{NVS\_MAKE\_P(Ns, yS0, ySdata, machEnv);}
  \end{itemize}
  Otherwise, make the call \id{yS0 = }\Id{N\_VNew\_S}\id{(Ns,machEnv);} 
  to create an array of \id{N\_Vector}'s and load initial values for 
  sensitivities \id{yS0[i]} into the array given by:
  \begin{itemize}
  \item {\s}
    \id{NV\_DATA\_S(yS0[i])}
  \item {\p}
    \id{NV\_DATA\_P(yS0[i])}
  \end{itemize}
  
\item
  Call \id{ier = }\Id{CVodeSensMalloc}\id{(\ldots);} to activate forward 
  sensitivity computations and allocate internal
  memory for {\cvodes} related to sensitivity calculations
  (see \S\ref{sss:cvodesensmalloc});
  
\item
  Call \Id{CVode} for each point at which output is desired. The 
  forward sensitivity equations will be integrated together with 
  the original IVP;
  
\item
  After each successful return from \id{CVode}, the solution of the
  original IVP is available in the \id{y} argument of \id{CVode},
  while the sensitivity solution can be extracted into \id{yS} (which can 
  be the same as \id{yS0}) by calling the routine 
  \id{ier = }\Id{CVodeSensExtract}\id{(cvode\_mem, t, yS);}
  (see \S\ref{sss:cvodesensextract});
  
\item 
  Upon completion of the integration, deallocate memory for the vector \id{y}
  and the vectors \id{yS}. If \id{yS} was created from \id{ysdata}, then use
  the implementation-dependent {\nvector} deallocation macro:
  \begin{itemize}
  \item {\s}
    \id{NVS\_DISPOSE\_S(yS, Ns);}
  \item {\p}
    \id{NVS\_DISPOSE\_P(yS, Ns);}
  \end{itemize}
  If \id{yS} was allocated through a call to \id{N\_VNew\_S} then deallocate
  it by calling \Id{N\_VFree\_S}\id{(Ns, yS);}
  
  
\item
  Before freeing the pointer to the user-defined data block 
  \id{f\_data}, release the array containing the real parameters \id{p}:
  \id{free(f\_data->p); free(f\_data);}
  
\item 
  Free the memory allocated for {\cvodes} by calling \Id{CVodeFree}\id{(cvode\_mem);}
  
\item 
  Free the machine environment block by calling the appropriate {\nvector}
  implementation-dependent routine;
  
\item
  {\p} If MPI was initialized by the user main program, call \id{MPI\_Finalize();}.
  
\end{enumerate}

%-------------------------------------------------------------------------------
\subsection{User-Callable Routines for Forward Sensitivity Analysis}
%-------------------------------------------------------------------------------

\subsubsection{Forward Sensitivity Initialization Routine}\label{sss:cvodesensmalloc}

The routine \ID{CVodeSensMalloc} activates forward sensitivity computations and
allocates internal memory related to sensitivity calculations.
The form of the call to this routine is
\begin{verbatim}
ier = CVodeSensMalloc(cvode_mem, Ns, ism, p, pbar, plist, ifS, fS,
                      errcon, rhomax, yS0, rtolS, atolS, fS_data);
\end{verbatim}

\begin{itemize}
  
\item \id{cvode\_mem} 
  is the pointer to the {\cvodes} memory returned by {\tt CVodeMalloc};
  
\item \id{Ns} 
  is the number of sensitivities to be computed;
  
\item \id{ism} \index{forward sensitivity analysis!correction strategies}
  is a flag used to select the sensitivity solution method and can 
  be \id{SIMULTANEOUS}, \id{STAGGERED}, or \id{STAGGERED1}:
  \begin{itemize}
  \item In the \ID{SIMULTANEOUS} approach, the state and sensitivity variables are
    corrected at the same time. If \id{NEWTON} was selected as the nonlinear system 
    solution method, this amounts to performing a modified Newton iteration on the
    combined nonlinear system;
  \item In the \ID{STAGGERED} approach, the correction step for the sensitivity
    variables takes place at the same time for all sensitivity equations, but only after 
    the correction of the state variables has converged and the state variables 
    have passed the local error test; 
  \item In the \ID{STAGGERED1} approach, all corrections are done sequentially, first
    for the state variables and then for the sensitivity variables, one parameter at
    a time. If the sensitivity variables are not included in the error control, this 
    approach is equivalent to \id{STAGGERED}. Note that the \id{STAGGERED1} approach 
    can be used only if \id{ifS = ONESENS}.
  \end{itemize}
  
\item \id{p} 
  is a pointer to the array of real problem parameters used to evaluate $f(t,y,p)$.
  The user data block \id{f\_data} must include a realtype pointer (e.g. \id{p})
  that points to \id{p}. For example, if the pointer to the data block has the 
  form \id{typedef struct}\{\id{\ldots, realtype *p;}\}\id{*f\_data;} 
  then \id{f\_data->p = p;} must point to the array in which 
  \id{p[i-1] =} $p_i$, for $i=1,\ldots,N_p$;
  
\item \id{pbar} 
  is an array of real values  that are used to scale the sensitivity absolute error 
  tolerance vectors. Each \id{pbar[i]} must be set to a nonzero constant that is
  dimensionally consistent with \id{p[i]}.
  Typically, \id{pbar[i]}$=$\id{p[i]} whenever \id{p[i]} is nonzero;
  
\item \id{plist} 
  is an array of \id{Ns} nonzero integer flags that serves two functions:
  it specifies parameter indeces in $\{1,\ldots,N_p\}$  
  with respect to which sensitivities are to be computed
  and indicates whether a given parameter affects the right hand side of the original
  IVP or only its initial conditions.
  More specifically, a positive \id{j = plist[i]} indicates that the sensitivity
  of the solution with respect to the \id{j}-th parameter \id{p[j-1]} is to be computed.
  A negative \id{j = plist[i]} indicates that the sensitivity of the solution with
  respect to the (\id{-j})-th parameter \id{p[-j-1]} is to be computed and indicates
  that \id{p[-j-1]} does not enter $f(t,y,p)$ thus increasing the efficiency of the
  difference quotient approximation routine for sensitivity right hand side evaluation;
  
\item \id{ifS} \index{forward sensitivity analysis!right hand side evaluation}
  is the type of sensitivity right hand side. The legal values are \Id{ALLSENS} 
  or \Id{ONESENS}. The \id{ALLSENS} type means that right hand sides for all   
  sensitivities are provided simultaneously. In this case 
  \id{fS} (if provided by the user) must be of type \Id{SensRhsFn}. 
  The \id{ONESENS} type means that \id{fS} (which, if provided by   
  the user, must be of type \Id{SensRhs1Fn}) computes one sensitivity right 
  hand side at a time. Note that \id{ism = STAGGERED1} requires \id{ifS = ONESENS}. 
  Either value for \id{ifS} is valid for \id{ism = SIMULTANEOUS} or 
  \id{ism = STAGGERED};
  
\item \id{fS}, 
  if not \id{NULL}, is a user-provided {\C} function
  to evaluate the right-hand sides of the sensitivity equations
  (\ref{e:sens_eqns}). If a \id{NULL} pointer is passed then {\cvodes}
  uses one the default difference quotient routines (\Id{CVSensRhsDQ}
  or \Id{CVSensRhs1DQ}, depending on the value \id{ifS})
  to evaluate these quantities. For more details, see \S\ref{ss:user_fct_fwd};
  
\item \id{errcon} 
  is a flag used to specify whether partial or full error control is to be used. 
  If \id{errcon = }\Id{FULL} then both state variables and
  sensitivity variables are included in the error tests. 
  If \id{errcon = }\Id{PARTIAL} then the sensitivity variables are excluded from the 
  error tests. Note that, in any event, all variables are considered in the convergence 
  tests;
  
\item \id{rhomax} 
  is a real scalar used to decide the finite differencing strategy
  in the case in which residuals of sensitivity equations are to be 
  computed by {\cvodes} (\id{fS = NULL}) using the internal difference
  quotient routines (see \S\ref{ss:fwd_sensi} for details);
  
\item \id{yS0} 
  is a pointer to an array of \id{Ns} vectors of length \id{N} containing the initial 
  values of the sensitivities;
  
\item \id{rtolS} 
  is a pointer to the user's relative error tolerance for sensitivity variables.
  If a \id{NULL} pointer is passed for \id{rtolS} then {\cvodes} uses the same
  relative tolerance for sensitivity variables as for the state variables;
  
\item \id{atolS} 
  must point to a vector of \id{Ns} absolute tolerance values for sensitivity variables 
  (if \id{itol = }\Id{SS}) or to an array of \id{Ns} vectors of sensitivity absolute tolerances
  (if \id{itol = }\Id{SV}).
  If a \id{NULL} pointer is passed for \id{atolS}, {\cvodes} defaults to using
  as absolute error tolerance for sensitivity \id{yS[i]} a multiple of the
  absolute error tolerance for the state variables, with the scale factor
  being $1/$\id{pbar[j-1]}, \id{j}$=|$\id{plist[i]}$|$.

\item \id{fS\_data}
  is a pointer to user-defined space passed directly to the user's \id{fS} function.
  
\end{itemize}
Note that, unlike the argument \id{y0} of \id{CVodeMalloc} - which could be deallocated
by the user before the call to \id{CVode}, the space \id{yS0} is used during the
integration of the sensitivity equations and should therefore not be deallocated until
after the last call to \id{CVode}.

The return value \id{ier} of \id{CVodeSensMalloc} is equal to: 
\begin{itemize}
\item \Id{SUCCESS}=0 if there were no errors; 
\item \Id{SCVM\_NO\_MEM} if \id{cvode\_mem} was NULL;
\item \Id{SCVM\_ILL\_INPUT} if an input argument was illegal;
\item \Id{SCVM\_MEM\_FAIL} if a memory request failed. 
\end{itemize}

%--------------------------
\subsubsection{Forward Sensitivity Extraction Routine}\label{sss:cvodesensextract}

If forward sensitivity computations have been initialized by a call to \id{CVodeSensMalloc},
or reinitialized by a call to \id{CVSensReInit}, then {\cvodes} computes both solution
and sensitivities at time \id{t}. However, \id{CVode} will still return only the solution
$y$ in \id{y}. Solution sensitivities can be obtained through the routine
\ID{CVodeSensExtract}:
\begin{verbatim}
ier = CVodeSensExtract(cvode_mem, t, yS);
\end{verbatim}
Its arguments are as follows:
\begin{itemize}
\item \id{cvode\_mem} is the pointer to the memory previously allocated
  by \id{CVodeMalloc}.
\item \id{t} specifies the time at which sensitivity information is 
  requested. The time \id{t} must fall within the interval defined by the last 
  succesful step taken by {\cvodes}.
\item \id{yS} must be declared of type \id{N\_Vector\_S} (i.e. a pointer to 
  \id{N\_Vector}). The user can use \id{yS = yS0}. 
  If successful, \id{CVodeSensExtract} will load \id{yS} with the values of the
  solution sensitivities at time \id{t}. Sensitivity with respect to the $i$-th
  sensitivity parameter (i.e. parameter \id{p[}$|$\id{plist[i]}$|$\id{-1]}) can be accessed in
  the \id{N\_Vector} \id{yS[i]}.
\end{itemize}
The return value \id{ier} of \id{CVodeSensExtract} is equal to: 
\begin{itemize}
\item \Id{SUCCESS}=0 if there were no errors; 
\item \Id{DKY\_NO\_MEM} if \id{cvode\_mem} was NULL;
\item \Id{DKY\_NO\_SENSI} if sensitivity computation was not turned on
      by a call to \id{CVodeSensMalloc};
\item \Id{BAD\_T} if the time \id{t} is not in the allowed range.
\end{itemize}
In case of an error return, an error message is also printed.  

%--------------------------
\subsubsection{Additional Optional Input/Output}\label{sss:more_optional_io}

As mentioned in \S\ref{sss:optional_io}, during forward sensitivity 
analysis, {\cvodes} stores some additional integer output values in \Id{iopt}.
These values are described in Table \ref{t:more_iopt}.

\begin{table}[htb]
\centering
\caption[Additional optional integer output from forward sensitivity]
{Additional optional integer output entries in the array \id{iopt} 
from forward sensitivity analysis}\label{t:more_iopt}
\medskip
\begin{tabular}{|l|p{4in}|}
\hline
{\bf Index} & {\bf Description} \\ 
\hline\hline
%
\id{NFSE} & 
Number of calls made to the sensitivity right hand side evaluation routine. 
\\ \hline
%
\id{NNIS} & 
Number of Newton iterations performed during sensitivity corrections (sum over all
sensitivities in the \id{STAGGERED1} case).
\\ \hline
%
\id{NCFNS} &
Number of nonlinear convergence failures during the sensitivity corrections (sum over
all sensitivities in the \id{STAGGERED1} case).
\\ \hline
%
\id{NETFS} &
Number of error test failures for sensitivity variables.
\\ \hline
\id{NSETUPSS} &
Number of calls to the linear solver's setup routine made during
the nonlinear solver phase in the \id{STAGGERED} or \id{STAGGERED1} case.
\\ \hline
%
%
\end{tabular}
\end{table}

%------------------------
\subsubsection{Interpolated Sensitivity Output Routines}
\index{interpolated output}

The two routines described here are available to obtain additional output values
for the sensitivity variables.

The routine \ID{CVodeSensDkyAll} computes the \id{k}-th derivatives of the interpolating 
polynomials for each sensitivity variable at time \id{t}.
This function is called by \id{CVodeSensExtract} with \id{k} $= 0$, but may also be called 
directly by the user.
\begin{verbatim}
ier = CVodeSensDkyAll(cvode_mem, t, k, dkyA)
\end{verbatim}

The arguments \id{cvode\_mem} and \id{t} are as above.
The argument \id{k} specifies the derivative order and must be 
$0 \le$ \id{k} $\le q$, where $q$ is the order of the LMM used on the last step.
If its value is illegal, the return value is \Id{BAD\_K}.
The argument \id{dkyA} must be declared as a pointer to \id{N\_Vector} (i.e. of type
\id{N\_Vector\_S}) and the user must allocate space for it. 
If \id{dkyA} or any of its component vectors is \id{NULL}, the return value is \Id{BAD\_DKY}.

The routine \ID{CVodeSensDky} computes the \id{k}-th derivatives of the interpolating 
polynomial for the \id{is}-th sensitivity variable at time \id{t}.
This function is called by \id{CVodeSensDkyAll} for all sensitivities, but may 
also be called directly by the user.
\begin{verbatim}
ier = CVodeSensDky(cvode_mem, t, k, is, dky)
\end{verbatim}

The arguments \id{cvode\_mem}, \id{t}, and \id{k} are as above.
The argument \id{is} specifies the sensitivity for which information
is requested and must be $0 \le$ \id{is} $< Ns$. If its value is illegal, the return
value is \Id{BAD\_IS}.
\id{dkyA} must be declared as an \id{N\_Vector} and the user
must allocate space for it. If \id{dkyA} is \id{NULL}, the return value 
is \Id{BAD\_DKY}.

%--------------------------
\subsubsection{Forward Sensitivity Reinitialization Routine}\label{sss:cvsreinit}
\index{reinitialization}

The routine \ID{CVodeSensReInit}, useful during the solution of a sequence of problems of 
same size, reinitializes the sensitivity related internal memory 
and must follow a call to \Id{CVodeSensMalloc} (and maybe a call to \id{CVReInit}). 
The number \id{Ns} of sensitivities is assumed to be unchanged since the call to 
\Id{CVodeSensMalloc}.

The call to the \id{CVodeSensReInit} function has the form
\begin{verbatim}
ier = CVodeSensReInit(cvode_mem, ism, p, pbar, plist, ifS, fS,
                      errcon, rhomax, yS0, rtolS, atolS, fS_data);
\end{verbatim}
The arguments have names and meanings identical to those of \id{CVodeSensMalloc}.
Note that the number of sensitivities \id{Ns} is not passed as an argument to 
\id{CVodeSensReInit}, as that is assumed to be unchanged since 
the \id{CVodeSensMalloc} call. 

The return value \id{ier} of \id{CVodeSensReInit} is equal to: 
\begin{itemize}
\item \Id{SUCCESS}=0 if there were no errors; 
\item \Id{SCVREI\_NO\_MEM} if \id{cvode\_mem} was NULL;
\item \Id{SCVREI\_NO\_SENSI} if sensitivity computation was not turned on
      by a call to \id{CVodeSensMalloc};
\item \Id{SCVREI\_ILL\_INPUT} if an input argument was illegal;
\item \Id{SCVREI\_MEM\_FAIL} if a memory request failed.
\end{itemize}
In case of an error return, an error message is also printed.  

Note that \id{CVodeSensReInit} is not a completely genuine reinitialization
routine as it may perform some memory allocation. This can happen
if \id{atolS = NULL} is passed to \id{CVodeSensReInit}, in which case {\cvodes}
must allocate and set its own internal absolute tolerances for the sensitivity
varaiables, or if \id{ism = STAGGERED1}, in which case some internal counter
arrays of length \id{Ns} are allocated by \id{CVodeSensReInit}.

\subsubsection{Additional Extraction Routines}\label{sss:more_extraction}
\index{access to additional data}

For user-provided sensitivity right-hand side routines based on finite
differences, additional quantities already available to {\cvodes} may
be of interest. The machine unit roundoff and the vector of error weights
for states can be accessed as described in \S\ref{sss:cvodegetewt}.
The vectors of error weights for sensitivity variables can be extracted
with a call to the function
\begin{verbatim}
ier = CVodeGetEwtS(cvode_mem, weightS);
\end{verbatim}
where \id{cvode\_mem} is the pointer to the {\cvodes} memory returned by 
{\tt CVodeMalloc} and \id{weightS} is of type \id{N\_Vector *}. 
The user need not allocate space \id{weightS}. The possible return
values for \id{ier} are \Id{OKAY} on successful return or \Id{GEWT\_NO\_MEM}
if \id{cvode\_mem} was \id{NULL}.

In the \Id{STAGGERED1} approach, the correction and error test phases are done sequentially
for each sensitivity parameter. In this case, {\cvodes} collects additional information 
regarding the performance of the nonlinear solvers for the each individual sensitivity 
system, in particular the number of nonlinear iterations and nonlinear 
solver convergence failures for each sensitivity system.
Upon return from \id{CVode}, the user can obtain this information 
with a call to the function \ID{CVodeMemExtract}. 

The call to \id{CVodeMemExtract} has the following form:
\begin{verbatim}
ier = CVodememExtract(cvode_mem, n_niS1, n_cfnS1);
\end{verbatim}
The vectors \id{n\_niS1} and \id{n\_cfnS1}, each of length \id{Ns} and type
\id{long int *} must be allocated by the user if the corresponding information
is desired. Upon a successful return, the return value is \Id{OKAY}, and
\id{n\_niS1}, if non-\id{NULL}, contains the nonlinear iteration counts,
while \id{n\_cfnS1}, if non-\id{NULL}, contains the nonlinear solver convergence failure counts.
If \id{cvode\_mem} is \id{NULL}, \id{CVodememExtract} returns \Id{MEXT\_NO\_MEM}.

%-------------------------------------------------------------------
\subsection{User-Supplied Routines for Forward Sensitivity Analysis}\label{ss:user_fct_fwd}
%-------------------------------------------------------------------

In addition to the required and optional user-supplied routines described
in \S\ref{ss:user_fct_sim}, when using {\cvodes} for forward sensitivity analysis,
the user has the option of providing a routine that calculates the right hand side 
of the sensitivity equations (\ref{e:sens_eqns}).

{\cvodes} provides difference quotient approximation routines for the right
hand sides of the sensitivity equations, \Id{CVSensRhsDQ} if \id{ifS = ALLSENS}
and \Id{CVSensRhs1DQ} if \id{ifS = ONESENS}. 
To use these routines, the user must pass \id{fS = NULL} in the call to
\id{CVodeSensMalloc} (see \S\ref{sss:cvodesensmalloc}). However, {\cvodes}
allows the option for user-defined sensitivity right hand side routines
(which also provides a mechanism for interfacing {\cvodes} to routines
generated by automatic differentiation).

Recall that, if sensitivity analysis is to be performed, the user-supplied 
data structure \id{f\_data} contains a pointer (e.g., \id{p}) that points 
to the array of real parameters upon which the original IVP depends.

\begin{itemize}
%
\item {\em Sensitivity equations right hand side (the case \Id{ALLSENS})}
  \index{forward sensitivity analysis!right hand side evaluation|(}
  \index{right hand side function!forward sensitivity|(}

  If the \Id{SIMULTANEOUS} or \Id{STAGGERED} approach was selected in the call to
  \id{CVodeSensMalloc}, the user may provide the right hand sides of the sensitivity equations
  (\ref{e:sens_eqns}), for all sensitivity parameters at once, through a function 
  of type \ID{SensRhsFn} defined by
\begin{verbatim}
typedef void (*SensRhsFn)(integertype Ns, realtype t, 
                          N_Vector y, N_Vector ydot, 
                          N_Vector *yS, N_Vector *ySdot, 
                          void *fS_data,  
                          N_Vector tmp1, N_Vector tmp2);
\end{verbatim}
  In this case, the argument \id{ifS} of \id{CVodeSensMalloc} must be set to \id{ALLSENS}.
  Note that a sensitivity right hand side function of type \id{SensRhsFn} is not
  compatible with the \id{STAGGERED1} approach.

  A function of type \id{SensRhsFn}  receives as input the value of the independent
  variable \id{t}, the ODE solution vector \id{y} and its derivative \id{ydot}, 
  and sensitivity vectors \id{yS}. It must compute the vectors
  $(\dfdyI) s_i(t) + (\dfdpiI)$ are store them in \id{ySdot[i]}. 
  There is no return value for a \id{SensRhsFn}.

  The complete list of arguments is given below.
  \begin{itemize}
  \item \id{Ns} is the number of sensitivity parameters;
  \item \id{t} is the value of the independent variable;
  \item \id{y} and \id{ydot} are the ODE solution and its derivative;
  \item \id{yS} contains the \id{Ns} sensitivity vectors;
  \item \id{ySdot} is the output of \id{SensRhsFn}. On exit it must contain
    the sensitivity right hand side vectors;
  \item \id{fS\_data} is a pointer to the user-data space passed to \id{CVodeSensMalloc};
  \item \id{tmp1} and \id{tmp2} are pointers to memory allocated
    for vectors of length \id{N} which can be used by a \id{SensRhsFn} 
    function as temporary storage or work space.
  \end{itemize}

\item Sensitivity equations right hand side (the case \Id{ONESENS})
  
  Alternatively, the user may provide the sensitivity right hand sides, one sensitivity
  parameter at a time through a function of type \ID{SensRhs1Fn}. 
  In this case, the argument \id{ifS} of \id{CVodeSensMalloc} must be set to \id{ONESENS}. 
  Note that a sensitivity right hand side function of type \id{SensRhs1Fn} is compatible with
  any legal value of the \id{CVodeSensMalloc} argument \id{ism}, and is 
  required if \id{ism = }\Id{STAGGERED1}.

  The type \id{SensRhs1Fn} is defined by
\begin{verbatim}
typedef void (*SensRhs1Fn)(integertype Ns, realtype t, 
                           N_Vector y, N_Vector ydot, 
                           integertype iS, N_Vector yS, N_Vector ySdot, 
                           void *fS_data,
                           N_Vector tmp1, N_Vector tmp2);
\end{verbatim}
  Except for \id{iS}, \id{yS}, and \id{ySdot}, the arguments are identical to those of 
  \id{SensRhsFn} functions.
  A function of type \id{SensRhs1Fn} receives as an argument \id{iS} specifying the
  sensitivity parameter for which the sensitivity right hand side vector must be
  evaluated ($0 \le$ \id{iS} $<$ \id{Ns}). The argument \id{yS} contains the \id{iS}-th
  sensitivity vector and, on return, \id{ySdot} must contain the right hand side of the
  \id{iS}-th sensitivity system.

  \index{right hand side function!forward sensitivity|)}
  \index{forward sensitivity analysis!right hand side evaluation|)}

%
\end{itemize}

%-------------------------------------------------------------------
\subsection{Note on Using \id{PARTIAL} Error Control}\label{ss:partial}
%-------------------------------------------------------------------
\index{partial error control!explanation of {\cvodes} behavior}
For some problems, when sensitivities are excluded from the error control test, 
the behavior of {\cvodes} may appear at first glance erroneous. One would
expect that, in such cases, the sensitivity variables will not influence in
any way the step size selection. A comparison of the solver diagnostics 
reported for the \id{cvdx} example in \S\ref{ss:serial_sim_ex} and the
second run of the \id{cvfdx} example in \S\ref{ss:serial_fwd_ex} indicates 
that this may not always be the case.

The short explanation of this behavior is that the step size selection
implemented by the error control mechanism in {\cvodes} is based on the 
magnitude of the correction calculated by the nonlinear solver. As mentioned
in \S\ref{sss:cvodesensmalloc}, even with partial error control selected
in the call to \id{CVodeSensMalloc}, the sensitivity variables are included
in the convergence tests of the nonlinear solver.

When using the simultaneous corrector method (\S\ref{ss:fwd_sensi}) 
the nonlinear system that is solved at each step involves both the states
and sensitivity equations. In this case, it is easy to see how the sensitivity 
variables may affect the convergence rate of the nonlinear solver and therefore
the step size selection. 
The case of the staggered corrector approach is more subtle. 
After all, in this case (\id{ism=STAGGERED} or \id{ism=STAGGERED1} in the call 
to \id{CVodeSensMalloc}), the sensitivity variables at a given step are 
computed only once the solver for the nonlinear state equations has converged.
However, if the nonlinear system corresponding to the sensitivity equations
has convergence problems, {\cvodes} will attempt to improve the initial guess
by reducing the step size in order to provide a better prediction of the
sensitivity variables. Moreover, even if there are no convergence failures in
the solution of the sensitivity system, {\cvodes} may trigger a call to the
linear solver's setup routine which typically involves re-evaluation of Jacobian
information (Jacobian approximation in the case of {\cvdense} and {\cvband} or
preconditioner data in the case of {\cvspgmr}). The new Jacobian information
will be used by subsequent calls to the nonlinear solver for the state equations
and, in this way, potentially affect the step size selection.

When using the simultaneous corrector method it is not possible to decide whether 
nonlinear solver convergence failures or calls to the linear solver setup routine
have been triggered by convergence problems due to the state or the sensitivity 
equations.
When using one of the staggered corrector methods however, these situations can be 
identified by carefully monitoring the diagnostic information provided through 
the vector of optional outputs \id{iopt}. If there are no convergence
failures in the sensitivity nonlinear solver (\id{iopt[NCFNS]=0}) and none
of the calls to the linear solver setup routine were made by the sensitivity
nonlinear solver (\id{iopt[NSETUPSS]=0}) then the step size selection is
not affected by the sensitivity variables.

Finally, the user must be warned that the effect of appending sensitivity 
equations to a given system of ODEs on the step size selection 
(through the mechanisms described above) is problem-dependent and can therefore
lead to either an increase or decrease of the total number of steps that {\cvodes} takes
to complete the simulation. At first glance, one would expect that the impact
of the sensitivity variables, if any, will be in the direction of increasing the
step size and therefore reducing the total number of steps. The argument for this
is that the presence of the sensitivity variables in the convergence test of the
nonlinear solver can only lead to additional iterations (and therefore a smaller
final correction) or to additional calls to the linear solver setup routine
(and therefore more up-to-date Jacobain information), both of which will lead
to larger steps being taken by {\cvodes}. However, this is true only locally.
Overall, a larger integration step taken at a given time may lead to 
step size reductions at later times (due to either nonlinear solver convergence
failures or error test failures).