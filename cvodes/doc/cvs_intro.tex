%===================================================================================
\chapter{Introduction}\label{s:intro}
%===================================================================================

{\cvodes} is part of a software family called {\sundials}: 
SUite of Nonlinear and DIfferential/ALgebraic equation Solvers.  
This suite consists of {\cvode}, {\kinsol} and {\ida}, and variants of these
with sensitivity analysis capabilities.
%
{\cvodes}\index{CVODES@{\cvodes}!brief description of} is a solver for stiff and
nonstiff initial value problems (IVPs) for systems of ordinary differential equation
(ODEs). In addition to solving stiff and nonstiff ODE systems, {\cvodes} has sensitivity 
analysis capabilities, using either the forward or the adjoint methods.

%---------------------------------
\section{Historical background}\label{ss:history}
%---------------------------------

\index{CVODES@{\cvodes}!relationship to {\vode}, {\vodpk}|(}
{\F} solvers for ODE initial value problems are widespread and heavily used. 
Two solvers that were previously written at LLNL are {\vode} \cite{BBH:89} 
and {\vodpk} \cite{Byr:92}.
{\vode}\index{VODE@{\vode}} is a general-purpose solver that includes methods for both stiff
and nonstiff systems, and in the stiff case uses direct methods (full or
banded) for the solution of the linear systems that arise at each implicit
step. Externally, {\vode} is very similar to the well known solver
{\lsode}\index{LSODE@{\lsode}} \cite{RaHi:94}.
{\vodpk}\index{VODPK@{\vodpk}} is a variant of {\vode} that uses a preconditioned Krylov 
(iterative) method for the solution of the linear systems. {\vodpk} is a powerful 
tool for large stiff systems because it combines established methods for stiff 
integration, nonlinear iteration, and Krylov (linear) iteration with a problem-specific
treatment of the dominant source of stiffness, in the form of the user-supplied
preconditioner matrix \cite{BrHi:89}.
The capabilities of both {\vode} and {\vodpk} were combined in the {\C}-language 
package {\cvode}\index{CVODE@{\cvode}} \cite{CoHi:94, CoHi:96}.

In the process of translating the {\vode} and {\vodpk} algorithms into {\C}, the overall 
{\cvode} organization has changed considerably.
One key feature of the {\cvode} organization is that the linear system solvers comprise a
layer of code modules that is separated from the integration algorithm, thus allowing for 
easy modification and expansion of the linear solver array.
A second key feature is a separate module devoted to vector operations; this 
facilitated the extension to multiprosessor environments with only a minimal impact 
on the rest of the solver, resulting in {\pvode}\index{PVODE@{\pvode}} \cite{ByHi:99}, 
the parallel variant of {\cvode}.
\index{CVODES@{\cvodes}!relationship to {\vode}, {\vodpk}|)}

\index{CVODES@{\cvodes}!relationship to {\cvode}, {\pvode}|(}
{\cvodes} is written with a functionality that is a superset of that of the pair
{\cvode}/{\pvode}. Sensitivity analysis capabilities, both forward and adjoint, 
have been added to the main integrator. Enabling forward sensititivity computations 
in {\cvodes} will result in the code integrating the so-called {\em sensitivity equations}
simultaneously with the original IVP, yielding both the solution and its sensitivity
with respect to parameters in the model. Adjoint sensitivity analysis, most useful
when the gradients of relatively few functionals of the solution with respect to
many parameters are sought, involves integration of the original IVP forward in time
followed by the integration of the so-called {\em adjoint equations} backward
in time. {\cvodes} provides the infrastructure needed to integrate any final-condition ODE
dependent on the solution of the original IVP (in particular the adjoint system). 

Development of {\cvodes} was concurrent with a redesign of the vector operations module
across the {\sundials} suite. The key feature of the new {\nvector} module is that it
is written in terms of abstract vector operations with the actual vector functions attached
by a particular implementation (such as serial or parallel) of {\nvector}. This allows
writing the {\sundials} solvers in a manner independent of the actual {\nvector} 
implementation (which can be user-supplied), as well as allowing more than one 
{\nvector} module to be linked into an executable file.
\index{CVODES@{\cvodes}!relationship to {\cvode}, {\pvode}|)}

\index{CVODES@{\cvodes}!motivation for writing in C|(}
There were several motivations for choosing the {\C} language for {\cvode} and later
for {\cvodes}.
First, a general movement away from {\F} and toward {\C} in scientific
computing was and still is apparent.  Second, the pointer, structure, and dynamic
memory allocation features in {\C} are extremely useful in software of
this complexity.
Finally, we prefer {\C} over {\CPP} for {\cvodes} because of the wider
availability of {\C} compilers, the potentially greater efficiency of {\C},
and the greater ease of interfacing the solver to applications written
in extended {\F}.
\index{CVODES@{\cvodes}!motivation for writing in C|)}

\section{Changes from previous versions}

\subsection*{Changes in v2.1.0}

The major changes from the previous version involve a redesign of the
user interface across the entire {\sundials} suite. We have eliminated the
mechanism of providing optional inputs and extracting optional statistics 
from the solver through the \id{iopt} and \id{ropt} arrays. Instead,
{\cvodes} now provides a set of routines (with prefix \id{CVodeSet})
to change the default values for various quantities controlling the
solver and a set of extraction routines (with prefix \id{CVodeGet})
to extract statistics after return from the main solver routine.
Similarly, each linear solver module provides its own set of {\id{Set}-}
and {\id{Get}-type} routines. For more details see \S\ref{ss:optional_input}
and \S\ref{ss:optional_output}.

Additionally, the interfaces to several user-supplied routines
(such as those providing Jacobians, preconditioner information, and
sensitivity right hand sides) were simplified by reducing the number
of arguments. The same information that was previously accessible
through such arguments can now be obtained through {\id{Get}-type}
functions.

Installation of {\cvodes} (and all of {\sundials}) has been completely 
redesigned and is now based on a configure script.

\subsection*{Changes in v2.1.1}

This {\cvodes} release includes bug fixes related to forward sensitivity
computations (possible loss of accuray on a BDF order increase and incorrect
logic in testing user-supplied absolute tolerances). 
In addition, we have added the option of activating and deactivating
forward sensitivity calculations on successive {\cvodes} runs without memory
allocation/deallocation.

Other changes in this minor {\sundials} release affect the build system.


\section{Reading this user guide}\label{ss:reading}

This user guide is a combination of general usage instructions.
Specific example programs are provided as a separate document.
We expect that some readers will want to concentrate on the general 
instructions, while others will refer mostly to the examples.

There are different possible levels of usage of {\cvodes}. The most casual
user, with an IVP problem only, can get by with reading \S\ref{ss:ivp_sol}, 
then \S\ref{s:simulation} through \S\ref{sss:cvode} only, and looking at examples 
in \cite{cvodes2.1.0_ex}.
In addition, to solve a forward sensitivity problem the user should read 
\S\ref{ss:fwd_sensi}, followed by \S\ref{s:forward} through 
\S\ref{ss:sensi_get} only, and look at examples in \cite{cvodes2.1.0_ex}.

%%
%%
In a different direction, a more advanced user with an IVP problem may want to 
(a) use a package preconditioner (\S\ref{ss:preconds}), 
(b) supply his/her own Jacobian or preconditioner routines (\S\ref{ss:user_fct_sim}),
(c) do multiple runs of problems of the same size (\S\ref{sss:cvreinit}), 
(d) supply a new {\nvector} module (\S\ref{s:nvector}), or even 
(e) supply a different linear solver module (\S\ref{ss:cvodes_org}).
%%
%%
An advanced user with a forward sensitivity problem may also want to
(a) provide his/her own sensitivity equations right-hand side routine
(\S\ref{s:user_fct_fwd}), (b) perform multiple runs with the same number of
sensitivity parameters (\S\ref{ss:sensi_malloc}), or (c) extract additional
diagnostic information (\S\ref{ss:sensi_get}).
%%
%%
A user with an adjoint sensitivity problem needs to understand the IVP 
solution approach at the desired level and also go through 
\S\ref{ss:adj_sensi} for a short mathematical description of the adjoint
approach, \S\ref{s:adjoint} for the usage of the adjoint module in {\cvodes},
and the examples in \cite{cvodes2.1.0_ex}.

The structure of this document is as follows:
\begin{itemize}
\item
  In Chapter \ref{s:install} we begin with instructions for the installation of 
  {\cvodes}, within the structure of {\sundials}.
\item
  In Chapter \ref{s:math}, we give short descriptions of the numerical 
  methods implemented by {\cvodes} for the solution of initial value problems
  for systems of ODEs, continue with an overview of the mathematical aspects 
  of sensitivity analysis, both forward (\S\ref{ss:fwd_sensi}) and adjoint
  (\S\ref{ss:adj_sensi}), and conclude with a description of stability limit
  detection (\S\ref{s:bdf_stab}).
\item
  The following chapter describes the structure of the {\sundials} suite
  of solvers (\S\ref{ss:sun_org}) and the software organization of the {\cvodes}
  solver (\S\ref{ss:cvodes_org}). 
\item
  In Chapter \ref{s:simulation}, we give an overview of the usage of {\cvodes},
  as well as a complete description of the user interface and of the 
  user-defined routines for integration of IVP ODEs. Readers that are not 
  interested in using {\cvodes} for sensitivity analysis can then 
  skip to the example programs in \cite{cvode2.2.0_ex}.
\item
  Chapter \ref{s:forward} describes the usage of {\cvodes} for forward
  sensitivity analysis as an extension of its IVP integration capabilities. 
  We begin with a skeleton of the user main program, with emphasis on the 
  steps that are required in addition to those already described in Chapter \ref{s:simulation}.
  Following that we provide detailed descriptions of the user-callable interface routines 
  specific to forward sensitivity analysis and of the additonal optional user-defined
  routines.
\item
  Chapter \ref{s:adjoint} describes the usage of {\cvodes} for adjoint
  sensitivity analysis. We begin by describing the {\cvodes} checkpointing 
  implementation for interpolation of the original IVP solution during
  integration of the adjoint system backward in time, and with 
  an overview of a user's main program. Following that we provide complete
  descriptions of the user-callable interface routines for adjoint sensitivity
  analysis as well as descriptions of the required additional user-defined routines.
\item
  Chapter \ref{s:nvector} gives a brief overview of the generic {\nvector} module 
  shared amongst the various components of {\sundials}, as well as details on the two {\nvector}
  implementations provided with {\sundials}: a serial implementation
  (\S\ref{ss:nvec_ser}) and a parallel implementation based on {\mpi}\index{MPI}
  (\S\ref{ss:nvec_par}).
\item
  Chapter \ref{s:new_linsolv} describes the specifications of linear solver modules as 
  supplied by the user.
\item
  Chapter \ref{s:gen_linsolv} describes in detail the generic linear solvers shared 
  by all {\sundials} solvers.
\item
  Finally, Chapter \ref{c:constants} lists the constants used for input to
  and output from {\cvodes}.
\end{itemize}

Finally, the reader should be aware of the following notational conventions
in this user guide:  Program listings and identifiers (such as \id{CVodeMalloc}) 
within textual explanations appear in typewriter type style; 
fields in {\C} structures (such as {\em content}) appear in italics;
and packages or modules, such as {\cvdense}, are written in all capitals. 
In the Index, page numbers that appear in bold indicate the main reference
for that entry.