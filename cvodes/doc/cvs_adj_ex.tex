%===================================================================================
\section{Example Problems for Adjoint Sensitivity Analysis}\label{s:adj_examples}
%===================================================================================
\index{adjoint sensitivity analysis!examples|see{examples, adjoint sensitivity}}

\index{examples, adjoint sensitivity!list of|(}
The {\cvodes} distribution contains, in the \id{sundials/cvodes/adj\_examples}
directory, the following five examples for forward sensitivity analysis:

\begin{itemize}
\item \id{cvadx}
  solves a chemical kinetics problem consisting of three rate equations.
  
  The adjoint capability of {\cvodes} is used to compute gradients
  of a quantity of the form (\ref{e:G}) with respect to the three
  reaction rate constants appearing in the model.
  This program solves both the forward and backward problems with the BDF method, 
  Newton iteration with the {\cvdense} linear solver, and user-supplied    
  Jacobian routines;
\item \id{cvabx}
  solves the semi-discrete form of an advection-diffusion equation in 2-D.

  The adjoint capability of {\cvodes} is used to compute gradients
  of the average (over both time and space) of the solution with respect to
  the initial conditions.
  This program solves both the forward and backward problems with the BDF method, 
  Newton iteration with the {\cvband} linear solver, and user-supplied     
  Jacobian routines;
\item \id{cvakx}
  solves a stiff ODE system that arises from a system     
  of partial differential equations.  The PDE system is a six-species
  food web population model, with predator-prey interaction and diffusion 
  on the unit square in two dimensions.

  The adjoint capability of {\cvodes} is used to compute gradients
  of the average (over both time and space) of the concentration of a selected species
  with respect to the initial conditions of all six species.
  Both the forward and backward problems are solved with the BDF/GMRES method 
  (i.e. using the {\cvspgmr} linear solver) and the block-diagonal part of the  
  Newton matrix as a left preconditioner;
\item \id{cvakxb}
  solves the same problem as \id{cvakx}, but computes gradients
  of the average over space at the final time of the concentration of a selected species
  with respect to the initial conditions of all six species;
\item \id{pvanx}
  solves the semi-discrete form of an advection-diffusion equation in 1-D.

  The adjoint capability of {\cvodes} is used to compute gradients
  of the average over space of the solution at the final time with
  respect to both the initial conditions and the advection and
  diffusion coefficients in the model.
  This program solves both the forward and backward problems with the option 
  for nonstiff systems, i.e. Adams method and functional iteration.
\end{itemize}
The first four are serial examples that use the {\nvecs} module and
the last one is a parallel example using the {\nvecp} module.
\index{examples, adjoint sensitivity!list of|)}

The next two sections describe in detail a serial example (\id{cvadx}) and
a parallel one (\id{pvanx}). For details on the other examples, the reader is
directed to the comments in their source files.

%------------------------------------------------------------------------
\subsection{A Serial Sample Problem}\label{ss:serial_adj_ex}

\index{examples, adjoint sensitivity!serial sample program!problem solved by|(}
As a first example of using {\cvodes} for adjoint sensitivity analysis
we examine the chemical kinetics problem of \S\ref{ss:serial_sim_ex} and
\S\ref{ss:serial_fwd_ex}:
\begin{equation}\label{e:chemkin:ivp}
  \begin{split}
    &{\dot y}_1 = -p_1 y_1 + p_2 y_2 y_3   \\
    &{\dot y}_2 =  p_1 y_1 - p_2 y_2 y_3 - p_3 y_2^2 \\
    &{\dot y}_3 =  p_3 y_2^2 \\
    &y(t_0) = y_0 \, ,
  \end{split}
\end{equation}
for which we want to compute the gradient with respect to $p$ of 
\begin{equation}\label{e:chemkin:G}
  G(p) = \int_{t_0}^{t_1}  \left( y_1 + p_2 y_2 y_3 \right) dt ,
\end{equation}
without having to compute the solution sensitivities ${dy}/{dp}$.
Following the derivation of \S\ref{ss:adj_sensi} and taking into account
the fact that the initial values of (\ref{e:chemkin:ivp}) do not depend on 
the parameters $p$, by (\ref{e:dGdp}) this gradient is simply
\begin{equation}\label{e:chemkin:dGdp}
\frac{dG}{dp} = \int_{t_0}^{t_1} 
\left( g_p + \lambda^T f_p \right) dt \, ,
\end{equation}
where $g(t,y,p) = y_1 + p_2 y_2 y_3$, $f$ is the vector valued function 
defining the right hand side of (\ref{e:chemkin:ivp}), and $\lambda$ is 
the solution of the adjoint problem (\ref{e:adj_eqns}),
\begin{equation}\label{e:chemkin:adj}
  \begin{split}
    &{\dot \lambda} = - (f_y)^T  \lambda - (g_y)^T \\
    &\lambda(t_1) = 0 \, .
  \end{split}
\end{equation}

In order to avoid saving intermediate $\lambda$ values just for the
evaluation of the integral in (\ref{e:chemkin:dGdp}), we extend the
backward problem with the following $N_p$ quadrature equations
\begin{equation}\label{e:chemkin:xi}
  \begin{split}
    &{\dot \xi} = g_p^T + f_p^T \lambda \\
    &\xi (t_1) = 0 \, ,
  \end{split}
\end{equation}
which yield $\xi(t_0) = - \int_{t_0}^{t_1} ( g_p^T + f_p^T \lambda) dt$
and thus ${dG}/{dp} = -\xi^T(t_0)$.
Similarly, the value of $G$ in (\ref{e:chemkin:G}) can be obtained as
$G = - \zeta(t_0)$, where $\zeta$ is solution of the following quadrature
equation:
\begin{equation}\label{e:chemkin:zeta}
  \begin{split}
    &{\dot\zeta} = g \\
    &\zeta(t_1) = 0 \, .
  \end{split}
\end{equation}
\index{examples, adjoint sensitivity!serial sample program!problem solved by|)}

The source code for this example is listed in \A\ref{ss:cvadx}.
The main program and the user-defined routines are described below, 
with emphasis on the aspects particular to adjoint sensitivity calculations.
\index{examples, adjoint sensitivity!serial sample program!explanation of|(}
The calling program must include the {\cvodes} header file \id{cvodea.h} which
in turn includes \id{cvodes.h} and thus provides {\cvodes} function prototypes
and constants, including those in the adjoint sensitivity module.
 
This program also includes two user-defined accessor macros,
\id{Ith} and \id{IJth}
\index{examples, adjoint sensitivity!serial sample program!user-defined accessor macros}
that are useful in writing the problem functions in a form closely
matching their mathematical description, i.e. with components numbered from 1 instead of from 0. 
Following that, the program defines problem-specific constants and a user-defined 
data structure which will be used to pass the values of the parameters $p$ to various
user routines. The constant \id{STEPS} defines the number of integration steps
between two consecutive check points.
The program prologue ends with the prototypes of four user-supplied functions that are
called by {\cvodes}. The first two provide the right hand side and dense Jacobian
for the forward problem and the last two provide the right hand side and dense Jacobian 
for the backward problem.

The \id{main} function begins with type declarations. Notice that we employ two machine 
environment variables, \id{machEnvF} for the forward problem and \id{machEnvB} for
the backward problem. The next code blocks allocate and initialize the user data
structure with the values of the parameters $p$, initialize \id{machEnvF} by calling
the serial machine environment initialization routine from {\nvecs}, allocate and
initialize \id{y} with the initial conditions of the forward problem, and finally
set the tolerances \id{rtol} and \id{atol}.

The call to \id{CVodeMalloc} sets-up the forward integration and specifies the
\id{BDF} integration method with \id{NEWTON} iteration. The linear solver is
selected to be {\cvdense} through the call to its initialization routine
\id{CVDense}, with a non-\id{NULL} Jacobian routine \id{Jac}.

Allocation for the memory block of the combined forward-backward problem is
acomplished through the call to \id{CVadjMalloc} which specifies \id{STEPS}=150,
the number of steps between two check points.

The call to \id{CVodeF} requests the solution of the forward problem to \id{TOUT}.
If successful, at the end of the integration, \id{CVodeF} will return the number
of saved check points in the argument \id{ncheck}. A list of the check points
is printed by \id{CVadjCheckPointsList}.

The next segment of code deals with the setup of the backward problem. First,
a serial machine environment variable \id{machEnvB} is initialized for vectors
of length \id{NEQ + NP + 1} (dimension of $\lambda$ $+$ dimension of $\xi$ $+$
one additional quadrature variable to evaluate $G$). Following that, the program 
allocates space and initializes the variables of the backward problem and
the relative and absolute tolerances for the backward integration.
{\cvodes} memory for the integration of the backward integration is allocated
by the call to the interface routine \id{CVodeMallocB} which specifies the
size of the problem, the right hand side user function \id{fB} and the \id{BDF}
integration method with \id{NEWTON} iteration, among other things.
The dense linear solver {\cvdense} is then initialized by calling the \id{CVDenseB}
interface routine with a non-\id{NULL} Jacobian routine \id{JacB}.

The actual solution of the backward problem is acomplished through the call to
\id{CVodeB}. If successful, \id{CVodeB} returns the solution of the backward 
problem at time \id{T0} in the vector \id{yB}. The values for $G$ and its gradient
are printed next.

The main program continues with a call to \id{CVReInitB} to reinitialize the 
backward memory block for a new adjoint computation with a different final 
time (\id{TB2}), followed by a second call to \id{CVodeB} and, upon successful
return, reporting of the new values for $G$ and its gradient.

The main program ends by freeing previously allocated memory through calls to 
\id{CVodeFree} (for the {\cvodes} memory for the forward problem), \id{CVadjFree} 
(for the memory allocated for the combined problem), \id{N\_VFree} 
(for the various vectors), and \id{M\_EnvFree\_Serial} (for the two machine 
environment variables \id{machEnvF} and \id{machEnvB}).

The user-supplied functions \id{f} and \id{Jac} for the right hand side and
Jacobian of the forward problem are straightforward expressions of its 
mathematical formulation (\ref{e:chemkin:ivp}), while \id{fB} and \id{JacB}
are mere translations of the backward problem
(\ref{e:chemkin:adj})--(\ref{e:chemkin:xi})--(\ref{e:chemkin:zeta}).
\index{examples, adjoint sensitivity!serial sample program!explanation of|)}

The output generated by \id{cvadx} is shown below.
\index{examples, adjoint sensitivity!serial sample program!output|(}
{\small\begin{verbatim}
Allocate CVODE memory for forward runs

Allocate global memory

Forward integration

List of Check Points (ncheck = 3)
Check point 3
  address   50e40
  t0        5210491.598010
  t1        40000000.000000
  next      50c98
Check point 2
  address   50c98
  t0        8078.421607
  t1        5210491.598010
  next      4fee0
Check point 1
  address   4fee0
  t0        66.985880
  t1        8078.421607
  next      4aa58
Check point 0
  address   4aa58
  t0        0.000000
  t1        66.985880
  next      0

Allocate CVODE memory for backward run


========================================================
G:            1.8219e+04 
Gp:          -7.8383e+05   3.1991e+00  -5.3301e-04
========================================================
lambda(t0):   3.4249e+04   3.4206e+04   3.4139e+04
========================================================

Free memory
\end{verbatim}}
\index{examples, adjoint sensitivity!serial sample program!output|)}

%--------------------------------------------------------------------------

\subsection{A Parallel Sample Program}\label{ss:parallel_adj_ex}

\index{examples, adjoint sensitivity!parallel sample program!problem solved by|(}
As an example of using the {\cvodes} adjoint sensitivity module with
the parallel vector module {\nvecp}, we describe a sample program
that solves the following problem: consider the 1-D advection-diffusion
equation
\begin{equation}\label{e:pvanx:orig_pde}
  \begin{split}
    & \frac{\partial u}{\partial t} = p_1 \frac{\partial^2 u}{\partial x^2} 
    + p_2 \frac{\partial u}{\partial x} \\
    & 0 = x_0 \le x \le x_1 = 2 \\
    & 0 = t_0 \le t \le t_1 = 2.5 \, ,
  \end{split}
\end{equation}
with boundary conditions $u(t,x_0) = u(t,x_1) = 0 ,\, \forall t$
and initial condition $u(t_0 , x) = u_0(x) = x(2-x)e^{2x}$. Also
consider the function
\begin{equation*}
  g(t) = \int_{x_0}^{x_1} u(t,x) dx \, .
\end{equation*}
We wish to find, through adjoint sensitivity analysis, the gradient of
$g(t_1)$ with respect to $p = [p_1 ; p_2]$ and the perturbation in $g(t_1)$
due to a perturbation $\delta u_0$ in $u_0$.

The approach we take in the program \id{pvanx} is to first derive an 
adjoint PDE which is then discretized in space and integrated backwards
in time to yield the desired sensitivities. A straight-forward extension 
to PDEs of the derivation given in \S\ref{ss:adj_sensi} gives
\begin{equation}\label{e:pvanx:dgdp}
  \frac{dg}{dp} (t_1) = \int_{t_0}^{t_1} dt 
  \int_{x_0}^{x_1} dx \mu \cdot 
  \left[
    \frac{\partial^2 u}{\partial x^2} ;
    \frac{\partial u}{\partial x}
  \right ]
\end{equation}
and
\begin{equation}\label{e:pvanx:delg}
  \delta g |_{t_1} = \int_{x_0}^{x_1} \mu(t_0,x) \delta u_0(x) dx \, , 
\end{equation}
where $\mu$ is the solution of the adjoint PDE
\begin{equation}\label{e:pvanx:adj_pde}
  \begin{split}
    & \frac{\partial \mu}{\partial t} + p_1 \frac{\partial^2 \mu}{\partial x^2} 
    - p_2 \frac{\partial \mu}{\partial x} = 0 \\
    & \mu(t_1 , x) = 1 \\
    & \mu(t , x_0) = \mu( t , x_1 ) = 0 \, .
  \end{split}
\end{equation}
Both the forward problem (\ref{e:pvanx:orig_pde}) and the backward problem 
(\ref{e:pvanx:adj_pde}) are discretized on a uniform spatial grid of size
$M_x + 2$ with central differencing and with boundary values eliminated,
leaving ODE systems of size $N = M_x$ each. 
As always, we deal with the time quadratures in (\ref{e:pvanx:dgdp}) by introducing
the additional equations
\begin{equation}\label{e:pvanx:quad}
  \begin{split}
    &{\dot\xi}_1 = \int_{x_0}^{x_1} dx \mu \frac{\partial^2 u}{\partial x^2} \, , \quad
    \xi_1(t_1) = 0 \, , \\
    &{\dot\xi}_2 = \int_{x_0}^{x_1} dx \mu \frac{\partial u}{\partial x} \, , \quad
    \xi_2(t_1) = 0 \, ,
  \end{split}
\end{equation}
yielding
\begin{equation*}
  \frac{dg}{dp} (t_1) = \left[ \xi_1(t_0) ; \xi_2(t_0) \right ]
\end{equation*}
The space integrals in (\ref{e:pvanx:delg}) and (\ref{e:pvanx:quad}) are
evaluated numerically, on the given spatial mesh, using the trapezoidal rule.

Note that $\mu(t_0 , x^*)$ is nothing but the perturbation in $g(t_1)$
due to a perturbation $\delta u_0(x) = \delta(x-x^*)$ in the initial conditions.
Therefore, $\mu(t_0,x)$ completely describes $\delta g(t_1)$ for any
perturbation $\delta u_0$.
\index{examples, adjoint sensitivity!parallel sample program!problem solved by|)}

\index{examples, adjoint sensitivity!parallel sample program!explanation of|(}
The source code for this example is listed in \A\ref{ss:pvanx}. Both the forward
and the backward problems are solved with the option for nonstiff systems,
i.e. using the Adams method with functional iteration for the solution of
the nonlinear systems. The overall structure of the \id{main} function is very
similar to that of the code \id{cvadx} (\S\ref{ss:serial_adj_ex}) with 
differences arising from the use of the parallel vector module.

Besides the parallelism implemented by {\cvodes} at the vector kernel level,
\id{pvanx} uses MPI calls to parallelize the calculations of the right-hand side
routines \id{f} and \id{fB} and of the spatial integrals involved.
The forward problem has size \id{NEQ = MX}, while the backward problem has
size \id{NB = NEQ + NP}, where \id{NP = 2} is the number of quadrature equations
in (\ref{e:pvanx:quad}).
The use of the total number of available processes on two problems of different 
sizes deserves some comments, as this is typical in adjoint sensitivity 
analysis. Out of the total number of available processes, namely \id{nprocs},
the first \id{npes = nprocs - 1} processes are dedicated to the integration of
the ODEs arising from the semi-discretization of the PDEs 
(\ref{e:pvanx:orig_pde}) and (\ref{e:pvanx:adj_pde}) and receive
the same load on both the forward and backward integration phases. 
The last process is reserved for the integration of the quadrature equations 
(\ref{e:pvanx:quad}), and is therefore inactive during the forward phases.
Of course, for problems involving a much larger number of quadrature equations,
more than one process could be reserved for their integration. 
An alternative would be to redistribute the \id{NB} backward problem variables 
over all available processes, without any relationship to the load distribution 
of the forward phase. However, the approach taken in \id{pvanx} has the 
advantage that the communication strategy adopted for the forward problem 
can be directly transfered to communication among the first \id{npes}
processes during the backward integration phase. 

We must also emphasize that, although inactive during the forward integration phase, 
the last process {\em must} participate in that phase with a 
{\em zero local array length}. 
This is because, during the backward integration phase, this process must
have its own local copy of variables (such as \id{cvadj\_mem}) that were set
only during the forward phase.
\index{examples, adjoint sensitivity!parallel sample program!explanation of|)}

Sample output generated by \id{pvanx} is shown below.
\index{examples, adjoint sensitivity!parallel sample program!output|(}
{\small\begin{verbatim}
 (PE# 0)
    mu(t0)[ 1] = 0.000277565
    mu(t0)[ 2] = 0.000561901
    mu(t0)[ 3] = 0.000847449
    mu(t0)[ 4] = 0.00112626
    mu(t0)[ 5] = 0.0013933

 (PE# 1)
    mu(t0)[ 6] = 0.00163938
    mu(t0)[ 7] = 0.00186055
    mu(t0)[ 8] = 0.00204712
    mu(t0)[ 9] = 0.00219668
    mu(t0)[10] = 0.00229996

 (PE# 2)
    mu(t0)[11] = 0.00235647
    mu(t0)[12] = 0.00235827
    mu(t0)[13] = 0.00230703
    mu(t0)[14] = 0.00219703
    mu(t0)[15] = 0.00203218

 (PE# 3)
    mu(t0)[16] = 0.00180971
    mu(t0)[17] = 0.00153563
    mu(t0)[18] = 0.00121073
    mu(t0)[19] = 0.000842711
    mu(t0)[20] = 0.000436183

 (PE# 4)
    g(t1) = 0.0199069

 (PE# 4)
    dgdp(t1) = [ -1.12075  -1.00896 ]
\end{verbatim}}
\index{examples, adjoint sensitivity!parallel sample program!output|)}
