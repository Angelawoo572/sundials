%===================================================================================
\section{Mathematical Considerations}\label{s:math}
%===================================================================================

{\cvodes} solves initial-value problems (IVPs) for systems of ODEs. 
Such problems can be stated as
\begin{equation}\label{e:ivp}
\begin{split}
&\dot{y} = f(t,\,y) \\
&y(t_0) = y_0 \, ,
\end{split}
\end{equation}
where $y \in {\bf R}^N$, $\dot{y}\,=dy/dt$ and ${\bf R}^N$ is the real $N$-dimensional
vector space. That is, (\ref{e:ivp}) represents a system of $N$ ordinary
differential equations and their initial conditions at some $t_0$. The
dependent variable is $y$ and the independent variable is $t$. The
independent variable need not appear explicitly in the $N$-vector valued
function $f$.

Additionally, if (\ref{e:ivp}) depends on some parameters $p \in {\bf R}^{N_p}$, i.e.
\begin{equation}\label{e:ivp_p}
\begin{split}
&\dot{y}  = f(t,\,y,\,p) \\
&y(t_0)  = y_0(p) \, ,
\end{split}
\end{equation}
{\cvodes} can also compute first order derivative information, performing either
{\em forward sensitivity analysis} or {\em adjoint sensitivity analysis}.
In the first case, {\cvodes} computes the sensitivities of the solution with respect to the 
parameters $p$, while in the second case, {\cvodes} computes the gradient of a 
{\em derived function} with respect to the parameters $p$.

%------------------------
\subsection{IVP Solution}\label{ss:ivp_sol}
%------------------------

The IVP is solved by one of two numerical methods. These are the
backward differentiation formula (BDF) \index{BDF method} and the 
Adams-Moulton formula \index{Adams method}. 
Both are implemented in a variable-stepsize, variable-order form. The BDF
uses a fixed-leading-coefficient form. These formulas can both be
represented by a linear multistep formula 
\begin{equation}\label{e:lmm}
\sum_{i=0}^{K_1}\alpha_{n,i}y_{n-i} + h_n\sum_{i=0}^{K_2}\beta_{n,i} 
\dot{y}_{n-i}=0
\end{equation}
where the $N$-vector $y_n$ is the computed approximation to $y(t_n)$,
the exact solution of (\ref{e:ivp}) at $t_n$. The stepsize is
$h_n=t_n-t_{n-1}$.  The coefficients $\alpha_{n,i}$ and $\beta_{n,i}$
are uniquely determined by the particular integration formula, the
history of the stepsize, and the normalization $\alpha_{n,0}=-1$. The
Adams-Moulton \index{Adams method} formula is recommended for nonstiff ODEs and is
represented by (\ref{e:lmm}) with $K_1=1$ and $K_2=q-1$. The order
of this formula is $q$ and its values range from 1 through 12. For
stiff ODEs, BDF \index{BDF method} should be selected and is represented by 
(\ref{e:lmm}) with $K_1=q$ and $K_2=0$. For BDF, the order $q$ may
take on values from 1 through 5. In the case of either formula, the
integration begins with $q=1$, and after that $q$ varies automatically
and dynamically.

For either BDF or the Adams formula, $\dot{y}_n$ denotes
$f(t_n,\,y_n)$. That is, (\ref{e:lmm}) is an implicit formula, and 
the nonlinear equation 
\begin{equation}\label{e:nonlinear}
\begin{split}
G(y_n) &\equiv  y_n-h_n\beta_{n,0}f(t_n,\,y_n) - a_n=0   \\
a_n &= \sum_{i>0}(\alpha_{n,i}y_{n-i}+h_n\beta_{n,i}\dot{y}_{n-i}) 
\end{split}
\end{equation}
must be solved for $y_{n}$ at each time step. For nonstiff problems,
functional (or fixpoint) iteration is normally used and does not
require the solution of a linear system of equations. For stiff
problems, a Newton iteration is used and for each iteration an
underlying linear system must be solved. This linear system of
equations has the form
\begin{equation}\label{e:Newton}
M[y_{n(m+1)}-y_{n(m)}]=-G(y_{n(m)}) \, ,
\end{equation}
where $y_{n(m)}$ is the $m$th approximation to $y_n$, and $M$
approximates $\partial G/ \partial y$:
\begin{equation} \label{e:N_Matrix}
M \approx I-\gamma J, ~~~~ J = \frac{\partial f}{\partial y}, ~~~~
    \gamma = h_n\beta_{n,0} ~.
\end{equation}
At present, aside from a diagonal Jacobian approximation, the other
options implemented in {\cvodes} for solving the linear systems
(\ref{e:Newton}) are:
\begin{itemize}
\item a direct method with dense treatment of the Jacobian;
\item a direct method with band treatment of the Jacobian;
\item an iterative method {\spgmr} (scaled, preconditioned
GMRES) \cite{BrHi89}, which is a Krylov subspace method. In most
cases, performance of {\spgmr} is improved by user-supplied
preconditioners. The user may precondition the system on the left, on
the right, on both the left and right, or use no preconditioner.
\end{itemize}
In most cases of interest to the {\cvodes} user, the technique of
integration will involve BDF and the Newton method coupled with one of the 
linear solver modules.

\index{error control|(}
The integrator computes an estimate $E_{n}$ of the local error at each time
step, and strives to satisfy the following inequality
\begin{equation*}%\label{e:Err}
\left\| E_n\right\|_{rms,w} < 1 ~.
\end{equation*}
Here the weighted root-mean-square norm is defined by
\begin{equation}\label{e:rms}
\left\| E_n\right\|_{rms,w}=\left[ \sum_{i=1}^N\frac{1}{N}\left(
w_iE_{n,i}\right) ^2\right] ^{1/2} \, ,
\end{equation}
where $E_{n,i}$ denotes the $i$th component of $E_n$, and the $i$th 
component of the weight vector is 
\begin{equation}\label{e:weight}
w_i=\frac{1}{rtol|y_i|+atol_i} \,.
\end{equation}
This permits an arbitrary combination of relative and absolute error control.
The user-specified relative error tolerance is the scalar $rtol$ and the
user-specified absolute error tolerance is $atol$ which may be an $N$-vector
(as indicated above) or a scalar. The value for $rtol$
indicates the number of digits of relative accuracy for a single time step.
The specified value for $atol_{i}\;$indicates the values of the
corresponding component of the solution vector which may be thought of as
being zero, or at the noise level. In particular, if we set 
$atol_i=rtol\times floor_i$ then $floor_i$ represents the floor value for the 
$i$th component of the solution and is that magnitude of the component for
which there is a crossover from relative error control to absolute error
control. Since these tolerances define the allowed error per step, they
should be chosen conservatively. Experience indicates that a conservative
choice yields a more economical solution than error tolerances that are too
large.
\index{error control|)}

The error control mechanism in {\cvodes} varies the stepsize and order
in an attempt to take minimum number of steps while satisfying the local
error test. The order control can be (optionally) modified with an algorithm
that attempts to detect limitations resulting from BDF stability properties.

%----------------------------------------
\subsection{Forward Sensitivity Analysis}\label{ss:fwd_sensi}
%----------------------------------------
\index{forward sensitivity analysis!mathematical background|(}
Typically, the governing equations of complex, large-scale models
depend on various parameters,  through the right-hand side vector 
and/or through the vector of initial conditions:
\begin{equation}\label{e:orig_eqns}
\begin{split}
&\dot{y}  = f(t,\,y,\,p) \\
&y(t_0)  = y_0(p) \, ,
\end{split}
\end{equation}
where $y \in {\bf R}^N$ and $p \in {\bf R}^{N_p}$.
In addition to numerically solving the ODEs, it may be desirable to
determine the sensitivity of the results with respect to the model
parameters. 
Such sensitivity information can be used to estimate which
parameters are most influential in affecting the behavior of the
simulation or to evaluate optimization gradients (in the setting of dynamic
optimization, parameter estimation, optimal control, etc.).

The {\em solution sensitivity} with respect to the model parameter
$p_i$ is defined as the vector:
\begin{equation}
s_i (t) = \frac{\partial y(t)}{\partial p_i}
\end{equation}
and satisfies the following {\em forward sensitivity equations}
(or in short {\em sensitivity equations}):
\begin{equation}\label{e:sens_eqns}
\begin{split}
&\dot{s_i}  = \frac{\partial f}{\partial y} s_i + \frac{\partial f}{\partial p_i} \\
&s_i(t_0)  = \frac{\partial y_{0}(p)}{\partial p_i} \, ,
\end{split}
\end{equation}
which are obtained by applying the chain rule of differentiation to the original
ODEs (\ref{e:orig_eqns}). The initial sensitivity vector $s_i(t_0)$ is either all zeros 
(if $p_i$ occurs only in $f$), or has nonzeros according to how $y_{0}(p)$
depends on $p_{i}$.

When performing forward sensitivity analysis, {\cvodes} carries out the time integration 
of the combined system, (\ref{e:orig_eqns}) and (\ref{e:sens_eqns}), by viewing it as an ODE
system of size $N(N_s+1)$, where $N_s$ represents a subset of model parameters $p_i$, 
with respect to which sensitivities are desired ($N_s \le N_p$). 
However, major efficiency improvements can be obtained by taking advantage of the special 
form of the sensitivity equations as linearizations of the original ODEs. 
In particular, for stiff systems, in which case {\cvodes} employs a Newton iteration, 
the original ODE system and all sensitivity systems share the same Jacobian matrix, 
and therefore the same iteration matrix $M$ in (\ref{e:N_Matrix}).

The sensitivity equations are solved with the same linear multistep formula that
was selected for the original ODEs and, if Newton iteration was selected, the
same linear solver is used in the correction phase for both state and sensitivity 
variables. In addition, {\cvodes} offers the option of including \index{error control}
({\em full error control}) or excluding
({\em partial error control}) the sensitivity variables from the local 
error test.

\paragraph{Forward sensitivity methods.}
\index{forward sensitivity analysis!correction strategies|(}
In what follows we briefly describe three methods that have been proposed for the 
solution of the combined ODE and sensitivity system. Due to its inefficiency, 
especially for large-scale problems, the first approach is not implemented in {\cvodes}.

\begin{itemize}

\item {\em Staggered Direct Method}
  
  In this method \cite{CS85}, the nonlinear system (\ref{e:nonlinear}) is first solved and, 
  once an acceptable numerical solution is obtained, the sensitivity variables 
  are found by the direct solution of 
  \begin{equation}\label{e:sensi_syst}
    \dot{s_i}  - \frac{\partial f}{\partial y} s_i = \frac{\partial f}{\partial p_i} \, ,
  \end{equation}
  where the BDF discretization is used to eliminate ${\dot s}_i$. Although the 
  system matrix of the above linear system is based on the exact same information
  as the matrix $M$ in (\ref{e:N_Matrix}), it must be updated and factored at every
  step of the integration as $M$ is updated only ocasionally. The computational cost 
  associated with these matrix updates and factrorizations makes this method 
  unattractive when compared with the methods described below and is therefore not
  implemented in {\cvodes}.
  
\item {\em Simultaneous Corrector Method}

  In this method \cite{MP96}, the BDF discretization is applied simultaneously
  to both the original equations (\ref{e:orig_eqns}) and the sensitivity systems
  (\ref{e:sens_eqns}) resulting in the following nonlinear system 
  \begin{equation*}
    {\hat G}({\hat y}_n) \equiv  
    {\hat y}_n - h_n\beta_{n,0} {\hat f}(t_n,\,{\hat y}_n) - {\hat a}_n = 0 \, ,
  \end{equation*}
  where ${\hat y} = [y, \ldots , s_i, \ldots ]$ and
  ${\hat f} = [ f(t,y,p), \ldots , (\dfdyI)(t,y,p) s_i + (\dfdpiI)(t,y,p) , \ldots ]$
  and ${\hat a}_n$ are the terms in the BDF discretization that depend on the
  solution at previous integration steps.
  This combined nonlinear system can be solved as in (\ref{e:Newton}) using
  a modified Newton method by solving the corrector equation
  \begin{equation}\label{e:Newton_sim}
    {\hat M}[{\hat y}_{n(m+1)}-{\hat y}_{n(m)}]=-{\hat G}({\hat y}_{n(m)})
  \end{equation}
  at each iteration, where 
  \begin{equation*}
    {\hat M} = 
    \begin{bmatrix}
      M              &        &        &        &   \\
      \gamma J_1     & M      &        &        &   \\
      \gamma J_2     & 0      & M      &        &   \\
      \vdots         & \vdots & \ddots & \ddots &   \\
      \gamma J_{N_s} & 0      & \ldots & 0      & M 
    \end{bmatrix} \, ,
  \end{equation*}
  $M$ is defined as in (\ref{e:N_Matrix}), and 
  $J_i = ({\partial}/{\partial y})\left[ (\dfdyI) s_i + (\dfdpiI) \right]$.
  It can be shown that a 2-step quadratic convergence can be attained by only
  using the block-diagonal portion of ${\hat M}$ in the corrector equation
  (\ref{e:Newton_sim}). This results in a decoupling that allows the reuse of 
  $M$ without additional matrix factorizations. However, the products
  $(\dfdyI)s_i$ as well as the vectors $\dfdpiI$ must still be reevaluated at 
  each step of the iterative process (\ref{e:Newton_sim}) to update the 
  sensitivity portions of the residual ${\hat G}$.
  
\item {\em Staggered Corrector Method}
  
  In the staggered corrector method \cite{FTB97}, as in the staggered direct method,
  the nonlinear system (\ref{e:nonlinear}) is solved first using the Newton iteration
  (\ref{e:Newton}). Then, a separate Newton iteration is used to solve the
  sensitivity system (\ref{e:sensi_syst}):
  \begin{equation}\label{e:stgr_iterations}
    M [s_{i , n(m+1)} - s_{i , n(m)}]=
    s_{i, n(m)} - 
    \gamma \left( \dfdy (t_n , y_n, p) s_{i , n(m)} + \dfdpi (t_n , y_n , p) \right)
    -a_{i,n} \, ,
  \end{equation}
  where $a_{i,n} = \sum_{j>0}(\alpha_{n,j}s_{i , n-j}+h_n\beta_{n,j}\dot{s}_{i , n-j})$.
  In other words, a modified-Newton iteration is used to solve a linear system.
  In this approach, the vectors $\dfdpiI$ need be updated only once per integration step, 
  after the state correction phase (\ref{e:Newton}) has converged. Note also that 
  Jacobian-related data can be reused at all iterations (\ref{e:stgr_iterations})
  to evaluate the products $(\dfdyI) s_i$.
  
\end{itemize}

{\cvodes} implements the simultaneous corrector method and two flavors of the 
staggered corrector method which differ only if the sensitivity variables are
included in the error control test. \index{error control} 
In the {\em full error control} case, 
the first variant of the staggered corrector method requires the convergence of 
the iterations (\ref{e:stgr_iterations}) for all $N_s$ sensitivity sytems and then 
performs the error test on the sensitivity variables. The second variant of the method
will perform the error test for each sensitivity vector $s_i,\,i=1,2,\ldots,N_s$
individually, as they pass the convergence test. Differences in performance
between the two variants may therefore be noticed whenever one of the sensitivity 
vectors $s_i$ fails a convergence or error test. 

An important observation is that the staggered corrector method, combined with 
the {\spgmr} linear solver effectively results in a staggered direct method. 
Indeed, {\spgmr} requires only the action of the matrix $M$ on a vector and
this can be provided with the current Jacobian information. Therefore, the
modified Newton procedure (\ref{e:stgr_iterations}) will theoretically converge 
after one iteration.
\index{forward sensitivity analysis!correction strategies|)}

\paragraph{Selection of the absolute tolerances for sensitivity variables.}
\index{forward sensitivity analysis!absolute tolerance selection|(}
If the sensitivities are considered in the error test, {\cvodes} provides an 
automated estimation of absolute tolerances for the sensitivity variables 
based on the absolute tolerance for the corresponding state variable.
The selection of $atol$ for the sensitivity variables is based on the observation
that the sensitivity vector $s_i$ will have units of $[y]/[p_i]$.
With this, the absolute tolerance for the $j$-th component of the sensitivity
vector $s_i$ is set to
\begin{equation*}
atolS_{i,j} = \frac{atol_j}{|{\bar p}_i|} \, ,
\end{equation*}
where $atol$ are the absolute tolerances for the state variables and $\bar p$
is a vector of scaling factors that are dimensionally consistent with
the model parameters $p$ and give indication of their order of magnitude.
Typically, if $p_i \ne 0$, then ${\bar p}_i = p_i$. The relative tolerance
$rtolS$ for sensitivity variables is set to be the same as for the state variables,
i.e. $rtolS = rtol$. This choice of relative and absolute tolerances is equivalent 
to requiring that the weighted root-mean-square norm (\ref{e:rms}) of the sensitivity 
vector $s_i$ with weights (\ref{e:weight}) based on $s_i$ is the same as the
weighted root-mean-square norm of the vector of scaled sensitivities 
${\bar s}_i = |{\bar p}_i| s_i$ with weights based on the state variables
(the scaled sensitivities ${\bar s}_i$ being dimensionally consistent with the
state variables).
\index{forward sensitivity analysis!absolute tolerance selection|)}

\paragraph{Evaluation of the sensitivity right-hand side.}
\index{forward sensitivity analysis!right hand side evaluation|(}
There are several methods for evaluating the right-hand side of the sensitivity 
systems $\left[ (\dfdyI) s_i + (\dfdpiI) \right]$: 
analytic evaluation, automatic differentiation, complex-step approximation, 
finite differences (or directional derivatives).
{\cvodes} provides all the software hooks for implementing interfaces to
automatic differentiation or complex-step approximation and future versions
will provide these capabilities.
At the present time, besides the option for analytical sensitivity right hand sides
(user-provided), {\cvodes} can evaluate these quantities using various
finite difference-based approximations.
The first option applies central finite differences to each term separately:
\begin{gather}
\dfdy s_i \approx 
\frac{f(t, y+\delta_y s_i, p)-
f(t, y-\delta_y s_i, p)}{2\,\delta_y} \label{e:fd2}\\
\dfdpi \approx 
\frac{f(t,y,p + \delta_i e_i)-
f(t,y,p - \delta_i e_i)}{2\,\delta_i} \, .\tag{\ref{e:fd2}'}
\end{gather}
As is typical for finite differences, the proper choice of
perturbations $\delta_y$ and $\delta_i$ is a delicate matter.
{\cvodes} uses $\delta_y$ and $\delta_i$ that take into account
several problem-related features;
namely, the relative ODE error tolerance $rtol$, the machine unit roundoff $\epsilon$,
the scale factor ${\bar p}_i$, and the weighted root-mean-square norm of the 
sensitivity vector $s_i$.
We then define
\begin{equation*}
\delta_i = |{\bar p}_i| \sqrt{\max(rtol, \epsilon)};
\quad
\delta_y = \frac{|{\bar p}_i|}{\max(1/\delta_i, \|s_i\|_{rms,w})}.
\end{equation*}
The terms $\epsilon$ and $1/\delta_i$ are included as
divide-by-zero safeguards in case $rtol = 0$ or $||s_i|| = 0$.
Roughly speaking (i.e., if the safeguard terms are ignored),
$\delta_i$ gives a $\sqrt{rtol}$ relative perturbation to the scaled parameter
$i$, and $\delta_y$ gives a unit weighted rms norm perturbation to $y$.
Of course, the main drawback of this approach is that it requires four
evaluations of $f(t,y,p)$.

Another technique for estimating the scaled sensitivity derivatives
via centered differences is by using directional derivatives:
\begin{equation}\label{e:dd2}
\dfdy s_i + \dfdpi \approx
\frac{f(t, y+\delta s_i, p + \delta e_i)-
      f(t, y-\delta s_i, p - \delta e_i)}{2\,\delta} \, ,
\end{equation}
in which
\begin{equation*}
\delta = \min(\delta_i, \delta_y) \, .
\end{equation*}
If $\delta_i = \delta_y = \delta$, a Taylor series analysis shows
that the sum of~(\ref{e:fd2})--(\ref{e:fd2}')
and~(\ref{e:dd2}) are equivalent to within $O(\delta^2)$.
However, the latter approach is half as costly since it only requires
two evaluations of $f(t,y,p)$.  
To take advantage of this savings, it may also be desirable to use the
latter formula when $\delta_i \approx \delta_y$.
{\cvodes} accommodates this possibility by allowing the user to
specify a threshold parameter~$\rhomax$.
In particular, if $\delta_i$ and $\delta_y$ are within a factor of
$|\rhomax|$ of each other then~(\ref{e:dd2}) is used to estimate the
scaled sensitivity derivatives.
Otherwise, the sum of (\ref{e:fd2})--(\ref{e:fd2}') is used since
$\delta_i$ and $\delta_y$ differ by a relatively large amount and the
use of separate perturbations is prudent.

These procedures for choosing the perturbations ($\delta_i$,
$\delta_y$, $\delta$) and switching ($\rhomax$) between
finite difference and directional derivative formulas have also been implemented 
for first-order formulas.
Forward finite differences can be applied to $\dfdy s_i$ and
$\dfdpi$ separately or the single directional derivative formula
\begin{equation*}
\dfdy s_i + \dfdpi \approx
\frac{f(t, y+\delta s_i, p + \delta e_i) - f(t, y, p)}{\delta}
\end{equation*}
can be used.
In {\cvodes}, the default value of $\rhomax=0$ indicates the use of
the second-order centered directional derivative formula ~(\ref{e:dd2}) exclusively.
Otherwise, the magnitude of $\rhomax$ and its sign (positive or
negative) indicates whether this switching is done with regard to
(centered or forward) finite differences, respectively.
\index{forward sensitivity analysis!right hand side evaluation|)}
\index{forward sensitivity analysis!mathematical background|)}

%----------------------------------------
\subsection{Adjoint Sensitivity Analysis}\label{ss:adj_sensi}
%----------------------------------------
\index{adjoint sensitivity analysis!mathematical background|(}
In the {\em forward sensitivity approach} described in the previous
section, obtaining sensitivities with respect to $N_s$ parameters is roughly
equivalent to solving an ODE system of size $(1+N_s) N$. This can become 
prohibitively expensive, especially for large-scale problems, if sensitivities
with respect to many parameters are desired.
In this situation, the {\em adjoint sensitivity method} is a very
attractive alternative, provided that we do not need the solution sensitivities
$s_i$, but rather the gradients with respect to model parameters of a relatively 
few derived functionals of the solution. In other words, if $y(t)$ is the solution
of (\ref{e:orig_eqns}), we wish to evaluate the gradient $\frac{dG}{dp}$ 
with respect to $p$ of
\begin{equation}\label{e:G}
G(p) = \int_{t_0}^{t_1} g(t, y, p) dt \, ,
\end{equation}
or, alternatively, the gradient $\frac{dg}{dp}$ of the function $g(t, x, p)$ 
at time $t_1$. The function $g$ must be smooth enough that $g_p$ and $g_x$ 
exist and are bounded. In what follows, we only sketch the analysis for the 
sensitivity problem for both $G$ and $g$.
For details on the derivation see \cite{CLPS02}.
Introducing a Lagrange multiplier $\lambda$, we form the augmented
objective function
\begin{equation}
I(p) = G(p) - \int_{t_0}^{t_1} \lambda^* 
\left( {\dot y} - f(t,y,p)\right) dt \, ,
\end{equation}
where $*$ denotes the transpose conjugate. The gradient of $G$ with respect to $p$ is
\begin{equation}
  \frac{dG}{dp} = \frac{dI}{dp} 
=\int_{t_0}^{t_1}(g_p + g_y s)dt - \int_{t_0}^{t_1} 
\lambda^* \left( {\dot s} - f_y s - f_p \right)dt \, ,
\end{equation}
where subscripts on functions such as $f$ or $g$ are used to denote partial 
derivatives and $s = [s_1,\ldots,s_{N_s}]$ is the matrix of solution sensitivities.
Applying integration by parts to the term $\lambda^* {\dot s}$ and selecting
$\lambda$ such that
\begin{equation}\label{e:adj_eqns}
\begin{split}
&{\dot \lambda} = -\left( \dfdy \right)^* \lambda - 
\left( \frac{\partial g}{\partial y} \right)^* \\
&\lambda(t_1) = 0 \, ,
\end{split}
\end{equation}
the gradient of $G$ with respect to $p$ is nothing but
\begin{equation}\label{e:dGdp}
\frac{dG}{dp} = \lambda^*(t_0) s(t_0) + 
\int_{t_0}^{t_1} \left( g_p + \lambda^* f_p \right) dt \, .
\end{equation}
The gradient of $g(t_1,y,p)$ with respect to $p$ can be then obtained
by using the Leibnitz differentiation rule. Indeed, from (\ref{e:G}),
\begin{equation*}
\frac{dg}{dp}(t_1) = \frac{d}{dt_1}\frac{dG}{dp}
\end{equation*}
and therefore, taking into account that $dG/dp$ in (\ref{e:dGdp}) depends on $t_1$
both through the upper integration limit and through $\lambda$ and that $\lambda(t_1) = 0$, 
\begin{equation}\label{e:dgdp}
\frac{dg}{dp}(t_1) = \mu^*(t_0) s(t_0) + g_p(t_1) +
\int_{t_0}^{t_1} \mu^* f_p dt \, ,
\end{equation}
where $\mu$ is the sensitivity of $\lambda$ with respect to the final integration 
limit and thus satisfies the following equation, obtained by taking the total derivative
with respect to $t_1$ of (\ref{e:adj_eqns}):
\begin{equation}\label{e:adj1_eqns}
\begin{split}
&{\dot \mu} = -\left( \dfdy \right)^* \mu \\ 
&\mu(t_1) = \left( \frac{\partial g}{\partial y} \right)^* \, .
\end{split}
\end{equation}
The final condition on $\mu(t_1)$ follows from 
$(\partial\lambda/\partial t) + (\partial\lambda/\partial t_1) = 0$ at $t_1$, and
therefore, $\mu(t_1) = -{\dot\lambda}(t_1)$. 

The first thing to notice about the adjoint system (\ref{e:adj_eqns}) is that there is 
no explicit specification of the parameters $p$; this implies that, once the solution
$\lambda$ is found, the formula (\ref{e:dGdp}) can then be used to find the gradient
of $G$ with respect to any of the parameters $p$. The same holds true for the system
(\ref{e:adj1_eqns}) and the formula (\ref{e:dgdp}) for gradients of $g(t_1,y,p)$. 
The second important remark is that the adjoint systems (\ref{e:adj_eqns}) and
(\ref{e:adj1_eqns}) are terminal value problems which depend on the solution $y(t)$ 
of the original IVP (\ref{e:orig_eqns}). Therefore, a procedure is needed for providing 
the states $y$ obtained during a forward integration phase of (\ref{e:orig_eqns}) to
{\cvodes} during the backward integration phase of (\ref{e:adj_eqns}) or (\ref{e:adj1_eqns}).
The approach adopted in {\cvodes}, based on {\em check-pointing} is described below.

\paragraph{Check-pointing scheme.}
\index{adjoint sensitivity analysis!check-pointing}
During the backward integration, the evaluation of the right hand side 
of the adjoint system requires, at the current time, the states $y$ which
were computed in the forward integration phase.
Since {\cvodes} implements variable-stepsize integration formulas,
it is unlikely that the states will be available at the desired time and
therefore some form of interpolation is needed. The {\cvodes} implementation
being also variable-order, it is possible that during the forward
integration phase the order may be reduced as low as 1st order,
which means that there may be points in time where only $y$ and ${\dot y}$
are available. Therefore, {\cvodes} employs a cubic Hermite interpolation
algorithm. However, especially for large-scale problems and long integration
intervals, the number and size of the vectors $y$ and ${\dot y}$ that would 
need to be stored make this approach computationally intractable. 

{\cvodes} settles for a compromise {\em storage space - execution time} by
implementing a so-called {\em check-pointing scheme}. At the cost of
one additional forward integration, this approach offers the best possible 
estimate of memory requirements for adjoint sensitivity analysis. To begin with,
based on the problem size $N$ and the available memory, the user decides on 
the number $N_d$ of data pairs $y$-${\dot y}$ that can be kept in memory for 
the purpose of interpolation. Then, during the first forward integration stage, 
every $N_d$ integration steps a check point is formed by saving enough information
(either in memory or on disk if needed) to allow for a hot restart, that is a restart
which will exactly reproduce the forward integration. In order to avoid storing
Jacobina-related data at each check point, a reevaluation of the iteration matrix
is forced before each check point. At the end of this stage, we are left with $N_c$ 
check points, including one at $t_0$.
During the backward integration stage, the adjoint variables are integrated
from $t_1$ to $t_0$ going from one check point to the previous one.
The backward integration from check point $i+1$ to check point $i$ is preceeded
by a forward integration from $i$ to $i+1$ during which $N_d$ data pairs 
$y$-${\dot y}$ are generated and stored in memory for interpolation.

This approach transfers the uncertainty in the number of integration
steps in the forward integration phase to uncertainty in the final number of check 
points. However, $N_c$ is much smaller than the number of steps taken during
the forward integration and there is no major penalty for writting and then reading
check point data to/from a temporary file.

\bigskip

\index{adjoint sensitivity analysis!implementation in {\cvodes}|(}
Finally, we note that the adjoint sensitivity module in {\cvodes} provides the 
infrastructure to integrate backwards in time any ODE terminal value problem
dependent on the solution of the IVP (\ref{e:orig_eqns}), including
adjoint systems (\ref{e:adj_eqns}) or (\ref{e:adj1_eqns}), as well as any other
quadrature ODEs that may be needed in evaluating the integrals in (\ref{e:dGdp}) 
or (\ref{e:dgdp}). In particular, for ODE systems arising from semi-discretization
of time-dependent PDEs, this feature allows for integration of both the 
discretized adjoint PDE system and the adjoint of the discretized PDE.
\index{adjoint sensitivity analysis!implementation in {\cvodes}|)}
\index{adjoint sensitivity analysis!mathematical background|)}

%-----------------------------------------------

\subsection{BDF Stability Limit Detection}

{\cvodes} includes an algorithm, {\stald} (STAbility Limit Detection),
which provides protection against potentially unstable behavior of the 
BDF multistep integration methods is certain situations, as described below.

When the BDF option is selected, {\cvodes} uses Backward Differentiation 
Formula methods of orders 1 to 5.  At order 1 or 2, the BDF
method is A-stable, meaning that for any complex constant $\lambda$ in
the open left half-plane, the method is unconditionally stable (for
any step size) for the standard scalar model problem $dy/dt = \lambda y$.
For an ODE system, this means that, roughly speaking, as long as all
modes in the system are stable, the method is also stable for any
choice of step size, at least in the sense of a local linear stability
analysis.

At orders 3 to 5, the BDF methods are not A-stable, although they are
{\em stiffly stable}. In each case, in order for the method to be stable
at step size $h$ on the scalar model problem, the product $h\lambda$ must
lie in a {\em region of absolute stability}. 
That region excludes a portion of the left half-plane that is concentrated 
near the imaginary axis.  The size of that region of instability grows as the order
increases from 3 to 5.  What this means is that, when running BDF at
any of these orders, if an eigenvalue $\lambda$ of the system lies close
enough to the imaginary axis, the step sizes $h$ for which the method is
stable are limited (at least according to the linear stability theory)
to a set that prevents $h\lambda$ from leaving the stability region.
The meaning of {\em close enough} depends on the order.  
At order 3, the unstable region is much narrower than at order 5, 
so the potential for unstable behavior grows with order.

System eigenvalues that are likely to run into this instability are
ones that correspond to weakly damped oscillations.  A pure undamped
oscillation corresponds to an eigenvalue on the imaginary axis.
Problems with modes of that kind call for different considerations,
since the oscillation generally must be followed by the solver, and
this requires step sizes ($h \sim 1/\nu$, where $\nu$ is the frequency) 
that are stable for BDF anyway.  But for a weakly damped oscillatory mode,
the oscillation in the solution is eventually damped to the noise level, 
and at that time it is important that the solver not be restricted to step 
sizes on the order of $1/\nu$.  It is in this situation that the new option may
be of great value.

In terms of partial differential equations, the typical problems for
which the stability limit detection option is appropriate are
semi-discrete ODE systems (i.e. discretized in space) from PDEs with
advection and diffusion, but with advection dominating over diffusion.
Diffusion alone produces pure decay modes, while advection tends to
produce undamped oscillatory modes.  A mix of the two with advection
dominant will have weakly damped oscillatory modes.

The {\stald} algorithm attempts to detect, in a direct
manner, the presence of a stability region boundary that is limiting
the step sizes in the presence of a weakly damped oscillation \cite{Hi92}.
The algorithm supplements (but differs greatly from) the existing
algorithms in {\cvodes} for choosing step size and order based on
estimated local truncation errors.  The {\stald} algorithm works directly
with history data that is readily available in {\cvodes}.  If it concludes
that the step size is in fact stability-limited, it dictates a
reduction in the method order, regardless of the outcome of the
error-based algorithm.  The {\stald} algorithm has been tested in
combination with the {\vode} solver on linear advection-dominated
advection-diffusion problems \cite{Hi95}, where it works well.  The
implementation in {\cvodes} has been successfully tested on linear 
and nonlinear advection-diffusion problems, among others.

This stability limit detection option adds some overhead computational
cost to the {\cvodes} solution.  (In timing tests, these overhead costs
have ranged from 2\% to 7\% of the total, depending on the size and
complexity of the problem, with lower relative costs for larger
problems.)  Therefore, it should be activated only when there is
reasonable expectation of modes in the user's system for which it is
appropriate.  In particular, if a {\cvodes} solution with this option
turned off appears to take an inordinately large number of steps at
orders 3-5 for no apparent reason in terms of the solution time scale,
then there is a good chance that step sizes are being limited by
stability, and that turning on the option will improve the efficiency
of the solution. 

