%%========================================================================
\chapter{CVODES Constants}\label{c:constants}
%%========================================================================

Below we list all input and output constants used by the main solver and 
linear solver modules, together with their numerical values and a short
description of their meaning.

%%-------------------------------------------------------------------------
%% Supertabular setings

\newlength{\tcolone}
\settowidth{\tcolone}{\id{SPTFQMR\_PSOLVE\_FAIL\_UNREC}}
\newlength{\tcoltwo}
\settowidth{\tcoltwo}{-109}
\newlength{\tcolthree}
\setlength{\tcolthree}{\textwidth}
\addtolength{\tcolthree}{-0.5in}
\addtolength{\tcolthree}{-\tcolone}
\addtolength{\tcolthree}{-\tcoltwo}

\tablefirsthead{}
\tablehead{}
\tabletail{}
\tablelasttail{}

%%-------------------------------------------------------------------------

\section{CVODES input constants}

\begin{supertabular*}{\textwidth}{p{\tcolone}@{\hspace*{2mm}\extracolsep{\fill}}rp{\tcolthree}}
%%
\hline
\multicolumn{3}{c}{\bf {\cvodes} main solver module}\\
\hline\\
%%
\id{CV\_ADAMS}            & 1 & Adams-Moulton linear multistep method. \\
\id{CV\_BDF}              & 2 & BDF linear multistep method. \\
\id{CV\_FUNCTIONAL}       & 1 & Nonlinear system solution through functional iterations. \\
\id{CV\_NEWTON}           & 2 & Nonlinear system solution through Newton iterations. \\
\id{CV\_SS}               & 1 & Scalar relative tolerance, scalar absolute tolerance. \\
\id{CV\_SV}               & 2 & Scalar relative tolerance, vector absolute tolerance. \\
\id{CV\_EE}               & 2 & Estimated relative tolerance and absolute tolerance for sensitivity variables. \\
\id{CV\_NORMAL}           & 1 & Solver returns at specified output time. \\
\id{CV\_ONE\_STEP}        & 2 & Solver returns after each successful step. \\
\id{CV\_NORMAL\_TSTOP}    & 3 & Solver returns at specified output time, but does not proceed past the specified stopping time. \\
\id{CV\_ONE\_STEP\_TSTOP} & 4 & Solver returns after each successful step, but does not proceed past the specified stopping time. \\
\id{CV\_SIMULTANEOUS}     & 1 & Simultaneous corrector forward sensitivity method. \\
\id{CV\_STAGGERED}        & 2 & Staggered corrector forward sensitivity method. \\
\id{CV\_STAGGERED1}       & 3 & Staggered (variant) corrector forward sensitivity method. \\
%%
\\\hline
\multicolumn{3}{c}{\bf {\cvodea} adjoint solver module}\\
\hline\\
%%
\id{CV\_HERMITE} & 1 & Use Hermite interpolation. \\
\id{CV\_POLYNOMIAL} & 2 & Use variable-degree polynomial interpolation. \\
%%
\\\hline
\multicolumn{3}{c}{\bf Iterative linear solver module}\\
\hline\\
%%
\id{PREC\_NONE} & 0 & No preconditioning \\
\id{PREC\_LEFT} & 1 & Preconditioning on the left only. \\
\id{PREC\_RIGHT} & 2 & Preconditioning on the right only. \\
\id{PREC\_BOTH} & 3 & Preconditioning on both the left and the right. \\
\id{MODIFIED\_GS} & 1 & Use modified Gram-Schmidt procedure. \\
\id{CLASSICAL\_GS} & 2 & Use classical Gram-Schmidt procedure.
%%
\end{supertabular*}


%%-------------------------------------------------------------------------

\section{CVODES output constants}

\begin{supertabular*}{\textwidth}{p{\tcolone}@{\hspace*{2mm}\extracolsep{\fill}}rp{\tcolthree}}

\hline
\multicolumn{3}{c}{\bf {\cvodes} main solver module}\\
\hline\\

\id{CV\_SUCCESS}         &  0  & Successful function return. \\
\id{CV\_TSTOP\_RETURN}   &  1  & \id{CVode} succeeded by reaching the specified stopping point. \\
\id{CV\_ROOT\_RETURN}    &  2  & \id{CVode} succeeded and found one or more roots. \\
\id{CV\_MEM\_NULL}       & -1  & The \id{cvode\_mem} argument was \id{NULL}. \\
\id{CV\_ILL\_INPUT}      & -2  & One of the function inputs is illegal. \\
\id{CV\_NO\_MALLOC}      & -3  & The {\cvode} memory block was not allocated by a call to \id{CVodeMalloc}. \\
\id{CV\_TOO\_MUCH\_WORK} & -4  & The solver took \id{mxstep} internal steps but could not reach tout.\\
\id{CV\_TOO\_MUCH\_ACC}  & -5  & The solver could not satisfy the accuracy demanded by the user for some internal step.\\
\id{CV\_ERR\_FAILURE}    & -6  & Error test failures occurred too many times during one internal time step or minimum step size was reached. \\
\id{CV\_CONV\_FAILURE}   & -7  & Convergence test failures occurred too many times during one internal time step or minimum step size was reached. \\
\id{CV\_LINIT\_FAIL}     & -8  & The linear solver's initialization function failed.  \\
\id{CV\_LSETUP\_FAIL}    & -9  & The linear solver's setup function failed in an unrecoverable manner. \\
\id{CV\_LSOLVE\_FAIL}    & -10 & The linear solver's solve function failed in an unrecoverable manner. \\
\id{CV\_MEM\_FAIL}       & -11 & A memory allocation failed. \\
\id{CV\_BAD\_K}          & -14 & The derivative order $k$ is larger than the order used. \\
\id{CV\_BAD\_T}          & -15 & The time $t$ s outside the last step taken. \\
\id{CV\_BAD\_DKY}        & -16 & The output derivative vector is \id{NULL}. \\
\id{CV\_BAD\_IS}         & -18 & The sensitivity index is larger than the number of sensitivities computed.\\
\id{CV\_NO\_QUAD}        & -20 & Quadrature integration was not activated. \\
\id{CV\_NO\_SENS}        & -21 & Forward sensitivity integration was not activated. \\

\\\hline
\multicolumn{3}{c}{\bf {\cvodea} adjoint solver module}\\
\hline\\

\id{CV\_ADJMEM\_NULL} & -101 & The \id{cvadj\_mem} argument was \id{NULL}. \\
\id{CV\_BAD\_TB0}     & -103 & The final time for the adjoint problem is outside the interval over which the forward problem was solved.\\
\id{CV\_BCKMEM\_NULL} & -104 & The \id{cvodes} memory for the backward problem was not created. \\
\id{CV\_REIFWD\_FAIL} & -105 & Reinitialization of the forward problem failed at the first checkpoint. \\
\id{CV\_FWD\_FAIL}    & -106 & An error occured during the integration of the forward problem.\\
\id{CV\_BAD\_ITASK}   & -107 & Wrong task for backward integration. \\
\id{CV\_BAD\_TBOUT}   & -108 & The desired output time is outside the interval over which the forward problem was solved.\\
\id{CV\_GETY\_BADT}   & -109 & Wrong time in Hermite interpolation function. \\

\\\hline
\multicolumn{3}{c}{\bf {\cvdense} linear solver module}\\
\hline\\

\id{CVDENSE\_SUCCESS}    &  0 & Successful function return. \\
\id{CVDENSE\_MEM\_NULL}  & -1 & The \id{cvode\_mem} argument was \id{NULL}.\\
\id{CVDENSE\_LMEM\_NULL} & -2 & The {\cvdense} linear solver has not been initialized.\\
\id{CVDENSE\_ILL\_INPUT} & -3 & The {\cvdense} solver is not compatible with the current {\nvector} module.\\
\id{CVDENSE\_MEM\_FAIL}  & -4 & A memory allocation request failed.\\
%%
\id{CVDENSE\_ADJMEM\_NULL}  & -101 & The \id{cvadj\_mem} argument was \id{NULL}. \\
\id{CVDENSE\_LMEMB\_NULL}  & -102 & The {\cvdense} linear solver has not been initialized for
                           the backward integration.\\

\\\hline
\multicolumn{3}{c}{\bf {\cvband} linear solver module}\\
\hline\\

\id{CVBAND\_SUCCESS}    &  0 & Successful function return. \\
\id{CVBAND\_MEM\_NULL}  & -1 & The \id{cvode\_mem} argument was \id{NULL}.\\
\id{CVBAND\_LMEM\_NULL} & -2 & The {\cvband} linear solver has not been initialized.\\
\id{CVBAND\_ILL\_INPUT} & -3 & The {\cvband} solver is not compatible with the
                         current {\nvector} module, or an input value was illegal.\\
\id{CVBAND\_MEM\_FAIL}  & -4 & A memory allocation request failed.\\
%%
\id{CVBAND\_ADJMEM\_NULL}  & -101 & The \id{cvadj\_mem} argument was \id{NULL}. \\
\id{CVBAND\_LMEMB\_NULL}  & -102 & The {\cvband} linear solver has not been initialized for
                           the backward integration.\\

\\\hline
\multicolumn{3}{c}{\bf {\cvdiag} linear solver module}\\
\hline\\

\id{CVDIAG\_SUCCESS}    &  0 & Successful function return. \\
\id{CVDIAG\_MEM\_NULL}  & -1 & The \id{cvode\_mem} argument was \id{NULL}.\\
\id{CVDIAG\_LMEM\_NULL} & -2 & The {\cvdiag} linear solver has not been initialized.\\
\id{CVDIAG\_ILL\_INPUT} & -3 & The {\cvdiag} solver is not compatible with the current {\nvector} module.\\
\id{CVDIAG\_MEM\_FAIL}  & -4 & A memory allocation request failed.\\
%%
\id{CVDIAG\_ADJMEM\_NULL}  & -101 & The \id{cvadj\_mem} argument was \id{NULL}. \\

\\\hline
\multicolumn{3}{c}{\bf {\cvspils} linear solver modules}\\
\hline\\

\id{CVSPILS\_SUCCESS}    &  0 & Successful function return. \\
\id{CVSPILS\_MEM\_NULL}  & -1 & The \id{cvode\_mem} argument was \id{NULL}.\\
\id{CVSPILS\_LMEM\_NULL} & -2 & The linear solver has not been initialized.\\
\id{CVSPILS\_ILL\_INPUT} & -3 & The solver is not compatible with the
                          current {\nvector} module, or an input value was illegal.\\
\id{CVSPILS\_MEM\_FAIL}  & -4 & A memory allocation request failed.\\
%%
\id{CVSPILS\_ADJMEM\_NULL}  & -101 & The \id{cvadj\_mem} argument was \id{NULL}. \\
\id{CVSPILS\_LMEMB\_NULL}  & -102 & The linear solver has not been initialized for
                           the backward integration.\\

\\\hline
\multicolumn{3}{c}{\bf {\spgmr} generic linear solver module}\\
\hline\\

\id{SPGMR\_SUCCESS}            &  0 & Converged. \\
\id{SPGMR\_RES\_REDUCED}       &  1 & No convergence, but the residual norm was reduced. \\
\id{SPGMR\_CONV\_FAIL}         &  2 & Failure to converge. \\
\id{SPGMR\_QRFACT\_FAIL}       &  3 & A singular matrix was found during the QR factorization. \\
\id{SPGMR\_PSOLVE\_FAIL\_REC}  &  4 & The preconditioner solve function failed recoverably.\\
\id{SPGMR\_MEM\_NULL}          & -1 & The {\spgmr} memory is \id{NULL}\\
\id{SPGMR\_ATIMES\_FAIL}       & -2 & The Jacobian-times-vector function failed. \\
\id{SPGMR\_PSOLVE\_FAIL\_UNREC} & -3 & The preconditioner solve function failed unrecoverably. \\
\id{SPGMR\_GS\_FAIL}           & -4 & Failure in the Gram-Schmidt procedure. \\
\id{SPGMR\_QRSOL\_FAIL}        & -5 & The matrix $R$ was found to be singular during the QR solve phase. \\

\\\hline
\multicolumn{3}{c}{\bf {\spbcg} generic linear solver module}\\
\hline\\

\id{SPBCG\_SUCCESS}            &  0 & Converged. \\
\id{SPBCG\_RES\_REDUCED}       &  1 & No convergence, but the residual norm was reduced. \\
\id{SPBCG\_CONV\_FAIL}         &  2 & Failure to converge. \\
\id{SPBCG\_PSOLVE\_FAIL\_REC}  &  3 & The preconditioner solve function failed recoverably.\\
\id{SPBCG\_MEM\_NULL}          & -1 & The {\spbcg} memory is \id{NULL}\\
\id{SPBCG\_ATIMES\_FAIL}       & -2 & The Jacobian-times-vector function failed. \\
\id{SPBCG\_PSOLVE\_FAIL\_UNREC}& -3 & The preconditioner solve function failed unrecoverably. \\

\\\hline
\multicolumn{3}{c}{\bf {\sptfqmr} generic linear solver module}\\
\hline\\

\id{SPTFQMR\_SUCCESS}            &  0 & Converged. \\
\id{SPTFQMR\_RES\_REDUCED}       &  1 & No convergence, but the residual norm was reduced. \\
\id{SPTFQMR\_CONV\_FAIL}         &  2 & Failure to converge. \\
\id{SPTFQMR\_PSOLVE\_FAIL\_REC}  &  3 & The preconditioner solve function failed recoverably.\\
\id{SPTFQMR\_MEM\_NULL}          & -1 & The {\sptfqmr} memory is \id{NULL}\\
\id{SPTFQMR\_ATIMES\_FAIL}       & -2 & The Jacobian-times-vector function failed. \\
\id{SPTFQMR\_PSOLVE\_FAIL\_UNREC}& -3 & The preconditioner solve function failed unrecoverably. \\

\\\hline
\multicolumn{3}{c}{\bf {\cvbandpre} preconditioner module}\\
\hline\\

\id{CVBANDPRE\_SUCCESS}         &  0 & Successful function return. \\
\id{CVBANDPRE\_PDATA\_NULL}     & -11 & The preconditioner module has not been initialized. \\
%%
\id{CVBANDPRE\_ADJMEM\_NULL}  & -101 & The \id{cvadj\_mem} argument was \id{NULL}. \\
\id{CVBANDPRE\_PMEMB\_NULL}  & -102 & The {\cvbandpre} preconditionr module has not been initialized for
                           the backward integration.\\
\id{CVBANDPRE\_MEM\_FAIL}  & -103 & A memory allocation failed.\\

\\\hline
\multicolumn{3}{c}{\bf {\cvbbdpre} preconditioner module}\\
\hline\\

\id{CVBBDPRE\_SUCCESS}         &  0 & Successful function return. \\
\id{CVBBDPRE\_PDATA\_NULL}     & -11 & The preconditioner module has not been initialized. \\
%%
\id{CVBBDPRE\_ADJMEM\_NULL}  & -101 & The \id{cvadj\_mem} argument was \id{NULL}. \\
\id{CVBBDPRE\_PMEMB\_NULL}  & -102 & The {\cvbbdpre} preconditionr module has not been initialized for
                           the backward integration.\\
\id{CVBBDPRE\_MEM\_FAIL}  & -103 & A memory allocation failed.

\end{supertabular*} 

