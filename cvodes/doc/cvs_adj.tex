%===================================================================================
\chapter{Using CVODES for Adjoint Sensitivity Analysis}\label{s:adjoint}
%===================================================================================

This section describes the use of {\cvodes} to compute sensitivities of derived 
functions using adjoint sensitivity analysis. As mentioned before, the adjoint
sensitivity module of {\cvodes} provides the infrastructure for integrating
backward in time any system of ODEs that depends on the solution of the original
IVP, by providing various interfaces to the main {\cvodes} integrator, as well 
as several supporting user-callable functions. For this reason, in the following
sections we refer to the {\em backward problem} and not to the 
{\em adjoint problem} when discussing details relevant to the ODEs that
are integrated backward in  time. The backward problem can be the adjoint problem
(\ref{e:adj_eqns}) or (\ref{e:adj1_eqns}), and
can be augmented with some quadrature differential equations.

We begin with a brief overview, in the form of a skeleton user program.
Following that are detailed descriptions of the interface to the
various user-callable functions and of the user-supplied functions that were not already
described in \S\ref{s:simulation}.
%%
%%----------------------------------------------------------------------
\section{A skeleton of the user's main program}\label{ss:skeleton_adj}
%%---------------------------------------------------------------------
%%
The following is a skeleton of the user's main program as an application of
{\cvodes}. The user program is to have these steps in the order indicated, 
unless otherwise noted. For the sake of brevity, we defer many of the details to 
the later sections.
As in \S\ref{ss:skeleton_sim}, most steps are independent of the {\nvector}
implementation used; where this is not the case, usage specifications are given for the
two implementations provided with {\cvodes}: steps marked with {\p} correspond to 
{\nvecp}, while steps marked with {\s} correspond to {\nvecs}.

\index{User main program!Adjoint sensitivity analysis}
\begin{Steps}

\item
  {\bf Include necessary header files}
  
  Besides \id{cvodes.h}, the main program must also include
  \Id{cvodea.h} which defines additional types and constants, and includes
  function prototypes for the adjoint sensitivity module user-callable functions and
  \id{cvodes.h}. In addition, the main program should include an {\nvector} 
  implementation header file (\id{nvector\_serial.h} or \id{nvector\_parallel.h}
  for the two implementations provided with {\cvodes}) and, if Newton iteration 
  was selected, a header file from the desired linear solver module.

\item
  \textcolor{gray}{\bf {\p} Initialize MPI}

  \vspace{0.2in}\centerline{\bf Forward problem}

\item
  \textcolor{gray}{\bf Set problem dimensions for the forward problem}

\item
  \textcolor{gray}{\bf Set initial conditions for the forward problem}

\item
  \textcolor{gray}{\bf Create {\cvodes} object for the forward problem}

\item
  \textcolor{gray}{\bf Set optional inputs for the forward problem}

\item
  \textcolor{gray}{\bf Allocate internal memory for the forward problem}

\item
  \textcolor{gray}{\bf Attach linear solver module for the forward problem}

\item
  \textcolor{gray}{\bf Set linear solver optional inputs for the forward problem}

\item
  {\bf Allocate space for the adjoint computation}

  Call \id{cvadj\_mem = }\Id{CVadjMalloc}\id{()} to allocate memory for the 
  combined forward-backward problem (see \S\ref{sss:cvadjmalloc} for more details). 
  This call requires \id{Nd}, the number of steps between two consecutive checkpoints.

\item
  {\bf Integrate forward problem}

  Call \Id{CVodeF}, a wrapper for the {\cvodes} main integration
  function \id{CVode}, either in \Id{CV\_NORMAL} mode to the time 
  \id{tout} or in \Id{CV\_ONE\_STEP} mode inside a loop (if intermediate
  solutions of the forward problem are desired (see \S\ref{sss:cvodef})).
  The final value of \id{tret}, denoted \id{tfinal}, is then
  the maximum allowable value for the endpoint $t_1$.

  \vspace{0.2in}\centerline{\bf Backward problem}

\item \label{i:back_start}
  {\bf Set problem dimensions for the backward problem}

  {\s} set \id{NB}, the number of variables in the backward problem \\
  {\p} set \id{NB} and \id{NBlocal}

\item
  {\bf Set final values for the backward problem}

  Set the vector \id{yB0} of final values for the backward problem.

\item
  {\bf Create {\cvodes} object for the backward problem}

  Call \Id{CVodeCreateB}, a wrapper for \id{CVodeCreate}, to
  create the {\cvodes} memory block and specify the solution method
  (linear multistep method and nonlinear solver iteration type) for
  the backward problem. Unlike \id{CVodeCreate}, the function \id{CVodeCreateB}
  does not return a pointer to the newly created memory block. Instead,
  this pointer is attached to the adjoint memory block (returned by
  \id{CVadjMalloc} and passed as the first argument to \id{CVodeCreateB}).

\item
  {\bf Set optional inputs for the backward problem}

  Call \id{CVodeSet*B} functions to change from their default values
  any optional inputs that control the behavior of {\cvodes}. Unlike
  their counterparts for the forward problem, these functions take as
  their first argument the adjoint memory block returned by \id{CVadjMalloc}.

\item
  {\bf Allocate memory for the backward problem}

  Call \Id{CVodeMallocB}, a wrapper for \id{CVodeMalloc}, to 
  allocate internal memory and initialize {\cvodes} at \id{tB0} for 
  the backward problem (see \S\ref{sss:cvodemallocb}).

\item \label{i:lin_solverB}
  {\bf Attach linear solver module for the backward problem}

  If Newton iteration is chosen, initialize the linear solver module 
  for the backward problem by calling the appropriate wrapper
  function: \id{CVDenseB}, \id{CVBandB}, \id{CVDiagB},
  or \id{CVSpgmrB} (see \S\ref{sss:lin_solv_b}). Note that it is not required
  to use the same linear solver module for both the forward and the 
  backward problems; for example, the forward problem could be solved
  with the {\cvdense} linear solver and the backward problem with {\cvspgmr}.

\item \label{i:quadB}
  {\bf Initialize quadrature calculation}

  If additional quadrature equations must be evaluated , 
  call \id{CVodeQuadMallocB}, a wrapper for \id{CVodeQuadMalloc},
  to initialize and allocate memory for quadrature integration.
  Optionally, call \id{CVodeSetQuad*B} functions to change from their
  default values optional inputs that control the integration of
  quadratures during the backward phase.

\item
  {\bf Integrate backward problem}

  Call \Id{CVodeB}, a second wrapper for the {\cvodes} main integration
  function \id{CVode}, to integrate the backward problem from \id{tB0}
  (see \S\ref{sss:cvodeb}). This function can be called either in \id{CV\_NORMAL}
  or \id{CV\_ONE\_STEP} mode. Typically, \id{CVodeB} will be called in \id{CV\_NORMAL}
  mode with an end time equal to the initial time of the forward problem.

\item \label{i:back_end}
  {\bf Extract quadrature variables}

  If applicable, call \Id{CVodeGetQuadB}, a wrapper for \id{CVodeGetQuad},
  to extract the values of the quadrature variables at the time returned
  by the last call to \id{CVodeB}.

\item
  {\bf Deallocate memory}

  Upon completion of the backward integration, call all necessary deallocation
  functions. These include appropriate destructors for the vectors 
  \id{y} and \id{yB}, a call to \id{CVodeFree} to free the {\cvodes} memory block 
  for the forward problem, and a call to \id{CVadjFree} (see \S\ref{sss:cvadjmalloc}) to free 
  the memory allocated for the combined problem. 
  Note that \id{CVadjFree} also deallocates the {\cvodes} memory for the backward problem.

\item
  \textcolor{gray}{\bf Finalize MPI}

  {\p} If {\mpi} was initialized by the user main program, call \id{MPI\_Finalize();}.

\end{Steps}

The above user interface to the adjoint sensitivity module in {\cvodes} was motivated by
the desire to keep it as close as possible in look and feel to the one for ODE IVP 
integration. Note that if steps (\ref{i:back_start})-(\ref{i:back_end}) are not present, 
a program with the above structure will have the same functionality as one described in
\S\ref{ss:skeleton_sim} for integration of ODEs, albeit with some overhead due to 
the checkpointing scheme.

%%
%%-------------------------------------------------------------------------------
\section{User-callable functions for adjoint sensitivity analysis}
%%-------------------------------------------------------------------------------
%%

\subsection{Adjoint sensitivity allocation and deallocation functions}\label{sss:cvadjmalloc}

After the setup phase for the forward problem, but before the call
to \id{CVodeF}, memory for the combined forward-backward problem must be
allocated by a call to the function \id{CVadjMalloc}.
The form of the call to this function is
%%
%%
\ucfunction{CVadjMalloc}
{
  cvadj\_mem = CVadjMalloc(cvode\_mem, Nd);
}
{
  The function \ID{CVadjMalloc} allocates internal memory for the combined
  forward and backward integration, other than the {\cvodes} memory block. 
  Space is allocated for the $N_d$ interpolation data points and a linked 
  list of checkpoints is initialized.
}
{
  \begin{args}[cvode\_mem]
  \item[cvode\_mem] (\id{void *}) 
    is the {\cvodes} memory block for the forward problem
    returned by a previous call to \id{CVodeCreate}.
  \item[Nd] (\id{long int}) 
    is the number of integration steps between two consecutive checkpoints.
  \end{args}
}
{
  If successful, \id{CVadjMalloc} returns a pointer of type \id{void *}. The user
  does not need to access this memory block but must pass it to other adjoint module
  user-callable functions. In case of failure (\id{cvode\_mem} is \id{NULL}, \id{Nd} is
  nonpositive, or a memory request fails), \id{CVadjMalloc} prints an error
  message to the standard output stream \id{stderr} and returns \id{NULL}.
}
{
  The user must set \id{Nd} so that all data needed for interpolation of the 
  forward problem solution between two checkpoints fits in memory. \id{CVadjMalloc}
  attempts to allocate space for $(2$\id{Nd}$+3)$ variables of type \id{N\_Vector}.
}
%%
%%
\ucfunction{CVadjFree}
{
  CVadjFree(cvadj\_mem);
}
{
  The function \ID{CVadjFree} frees the memory allocated by
  a previous call to \id{CVadjMalloc}.
}
{
  The argument is the pointer to the adjoint memory block (of type \id{void *}).
}
{
  The function \id{CVadjFree} has no return value.
}
{
  This function frees all memory allocated by \id{CVadjMalloc}. This includes workspace memory, 
  the linked list of checkpoints, memory for the interpolation data, the {\cvodes} memory 
  for the backward integration phase, and possibly memory for backward-phase preconditioners.
}

%%---------------------------------------------------------------------

\subsection{Forward integration function}\label{sss:cvodef}

The function \ID{CVodeF} is very similar to the {\cvodes} function \id{CVode}
(see \S\ref{sss:cvode}) in that it integrates the solution of the forward
problem and returns the solution in \id{y}. At the same time,
however, \id{CVodeF} stores checkpoint data every \id{Nd} integration steps.
\id{CVodeF} can be called repeatedly by the user.
%%
The call to this function has the form
%%
\ucfunction{CVodeF}
{
  flag = CVodeF(cvadj\_mem, tout, yout, tret, itask, ncheck);
}
{
  The function \ID{CVodeF} integrates the forward problem over an interval in $t$
  and saves checkpointing data.
}
{
  \begin{args}[cvadj\_mem]
  \item[cvadj\_mem] (\id{void *})
    pointer to the adjoint memory block.
  \item[tout] (\id{realtype})
    the next time at which a computed solution is desired.
  \item[yout] (\id{N\_Vector})
    the computed solution vector.
  \item[tret] (\id{realtype *})
    the time reached by the solver.
  \item[itask] (\id{int})
    \index{itask@\texttt{itask}}
    a flag indicating the job of the solver for the next step. 
    The \Id{CV\_NORMAL} task is to have the solver take internal steps until   
    it has reached or just passed the user-specified \id{tout}
    parameter. The solver then interpolates in order to   
    return an approximate value of $y($\id{tout}$)$. 
    The \Id{CV\_ONE\_STEP} option tells the solver to just take one internal step  
    and return the solution at the point reached by that step. 
    The \Id{CV\_NORMAL\_TSTOP} and \Id{CV\_ONE\_STEP\_TSTOP} modes are     
    similar to \id{CV\_NORMAL} and \id{CV\_ONE\_STEP}, respectively, except    
    that the integration never proceeds past the value      
    \id{tstop} (specified through the function \id{CVodeSetStopTime}).
  \end{args}
}
{
  On return, \id{CVodeF} returns a vector \id{yout} and a corresponding 
  independent variable value $t=$\id{*tret}, such that \id{yout} is the computed 
  value of $y(t)$. Additionally, it returns in \id{ncheck} the number of
  checkpoints saved.

  The return value \id{flag} (of type \id{int}) will be one of the following.
  For more details see \S\ref{sss:cvode}.
  \begin{args}[CV\_TOO\_MUCH\_WORK]
  \item[\Id{CV\_SUCCESS}]
    \id{CVodeF} succeeded.
  \item[\Id{CV\_TSTOP\_RETURN}]
    \id{CVodeF} succeeded by reaching the optional stopping point.
  \item[\Id{CV\_NO\_MALLOC}]
    The function \id{CVodeMalloc} has not been previously called.
  \item[\Id{CV\_ILL\_IPUT}]
    One of the inputs to \id{CVodeF} is illegal.
  \item[\Id{CV\_TOO\_MUCH\_WORK}] 
    The solver took \id{mxstep} internal steps but could not reach tout. 
  \item[\Id{CV\_TOO\_MUCH\_ACC}] 
    The solver could not satisfy the accuracy demanded by the user for some 
    internal step.
  \item[\Id{CV\_ERR\_FAILURE}]
    Error test failures occurred too many times during one 
    internal time step or occurred with $|h| = h_{min}$.
  \item[\Id{CV\_CONV\_FAILURE}] 
    Convergence test failures occurred too many times during 
    one internal time step or occurred with $|h| = h_{min}$.             
  \item[\Id{CV\_LSETUP\_FAIL}] 
    The linear solver's setup function failed in an unrecoverable manner.
  \item[\Id{CV\_LSOLVE\_FAIL}] 
    The linear solver's solve function failed in an unrecoverable manner.
  \item[\Id{CV\_ADJMEM\_NULL}]
    The \id{cvadj\_mem} argument was \id{NULL}.
  \item[\Id{CV\_MEM\_FAIL}]
    A memory allocation request has failed (in an attempt to allocate space
    for a new checkpoint).
  \end{args} 
}
{
  All failure return values are negative and therefore a test \id{flag}$< 0$
  will trap all \id{CVodeF} failures.

  At this time, \id{CVodeF} stores checkpoint information in memory
  only. Future versions will provide for a safeguard option of dumping checkpoint
  data into a temporary file as needed. The data stored at each checkpoint is basically
  a snapshot of the {\cvodes} internal memory block and contains enough information
  to restart the integration from that time and to proceed with the same step size and
  method order sequence as during the forward integration.
  
  In addition, \id{CVodeF} also stores interpolation data between consecutive checkpoints
  so that, at the end of this first forward integration phase, information for the Hermite 
  interpolation is already available from the last checkpoint forward. In particular,
  if no check points were necessary, there is no need for the second forward integration phase.

  Although it is legal to define a value for \id{tstop} and then call \id{CVodeF}
  with \id{itask} $=$ \id{CV\_NORMAL\_TSTOP} or \id{itask} $=$ \id{CV\_ONE\_STEP\_TSTOP}, 
  after a return with \id{flag} $=$ \id{TSTOP\_RETURN}, the integration should not be 
  continued (no \id{tstop} information is stored at checkpoints).
}  

%%---------------------------------------------------------------------

\subsection{Backward problem initialization functions}\label{sss:cvodemallocb}

The functions \id{CVodeCreateB} and \id{CVodeMallocB} must be called in the order listed. 
They instantiate a {\cvodes} solver object, provide problem and solution specifications, 
and allocate internal memory for the backward problem.
%%
%%
\ucfunction{CVodeCreateB}
{
  flag = CVodeCreateB(cvadj\_mem, lmm, iter);
}
{
  The function \ID{CVodeCreateB} instantiates a {\cvodes} solver object and specifies 
  the solution method for the backward problem.
}
{
  \begin{args}[cvadj\_mem]
  \item[cvadj\_mem] (\id{void *})
    pointer to the adjoint memory block returned by \id{CVadjMalloc}.
  \item[lmm] (\id{int})
    specifies the linear multistep method and may be one of two
    possible values: \Id{CV\_ADAMS} or \Id{CV\_BDF}.     
  \item[iter] (\id{int})
    specifies the type of nonlinear solver iteration and may be
    either \Id{CV\_NEWTON} or \Id{CV\_FUNCTIONAL}. 
  \end{args}
}
{
  If successful, \id{CVodeCreateB} stores a pointer to the newly created 
  {\cvodes} memory block (of type \id{void *}) for the backward problem.
   The return value \id{flag} (of type \id{int}) is one of:
  \begin{args}[CV\_ADJMEM\_NULL]
  \item[\Id{CV\_SUCCESS}]
    The call to \id{CVodeCreateB} was successful.
  \item[\Id{CV\_ADJMEM\_NULL}]
    The \id{cvadj\_mem} argument was \id{NULL}.
  \item[\Id{CV\_MEM\_FAIL}]
    An error occured while trying to create the {\cvodes} memory block for the backward 
    problem.
  \end{args}
}
{}
%%
%%
The function \id{CVodeMallocB} is essentially a call to \id{CVodeMalloc} with some 
particularization for backward integration as described below.
\ucfunction{CVodeMallocB}
{
  flag = CVodeMallocB(cvadj\_mem, fB, tB0, yB0, itolB, reltolB, abstolB);
}
{
  The function \ID{CVodeMallocB} provides required problem and solution specifications, 
  allocates internal memory, and initializes {\cvodes} for the backward problem.
}
{
  \begin{args}[cvadj\_mem]
  \item[cvadj\_mem] (\id{void *})
    pointer to the adjoint memory block returned by \id{CVadjMalloc}.
  \item[fB] (\Id{CVRhsFnB})
    is the {\C} function which computes $fB$, the right-hand side of the 
    backward ODE problem. This function has the form 
    \id{fB(t, y, yB, yBdot, f\_dataB)} (for full details see \S\ref{ss:user_fct_adj}).
  \item[tB0] (\id{realtype})
    specifies the endpoint where final conditions are provided for the 
    backward problem.
  \item[yB0] (\id{N\_Vector})
    is the final value of the backward problem. 
  \item[itolB] (\id{int}) 
    is either \Id{CV\_SS} or \Id{CV\_SV}, where \Id{itol}$=$\id{CV\_SS} indicates scalar relative error 
    tolerance and scalar absolute error tolerance, while \id{itol}$=$\id{CV\_SV} indicates scalar
    relative error tolerance and vector absolute error tolerance. 
  \item[reltolB] (\id{realtype *})
    \index{tolerances}
    is a pointer to the relative error tolerance.
  \item[abstolB] (\id{void *})
    is a pointer to the absolute error tolerance.
  \end{args}
}
{
  The return value \id{flag} (of type \id{int}) will be one of the following:
  \begin{args}[CV\_ADJMEM\_NULL]
  \item[\Id{CV\_SUCCESS}]
    The call to \id{CVodeMallocB} was successful.
  \item[\Id{CV\_ADJMEM\_NULL}]
    The \id{cvadj\_mem} argument was \id{NULL}.
  \item[\Id{CV\_BAD\_TB0}]
    The final time \id{tB0} is outside the interval over which the forward problem
    was solved.
  \item[\Id{CV\_MEM\_NULL}] 
    The {\cvodes} memory block for the backward problem was not initialized through 
    a previous call to \id{CVodeCreateB}.
  \item[\Id{CV\_MEM\_FAIL}] 
    A memory allocation request has failed.
  \item[\Id{CV\_ILL\_INPUT}] 
    An input argument to \id{CVodeMallocB} has an illegal value.
  \end{args}
}
{}
%%
%%
For the case when it is needed to solve several different backward problems 
corresponding to the same original problem, {\cvodes} provides a mechanism 
to reuse the existing checkpoints. The function \ID{CVodeReInitB} reinitializes
the {\cvodes} memory block for the backward problem, where a prior call to
\Id{CVodeMallocB} has been made with the same problem size \id{NB}.
\id{CVodeReInitB} performs the same input checking and initializations that 
\id{CVodeMallocB} does, but does no memory allocation, 
assuming that the existing internal memory is sufficient for the new problem. 
Note that \id{CVReInitB} is essentially a wrapper for \id{CVodeReInit}
and so all details given for \id{CVodeReInit} in \S\ref{sss:cvreinit}
apply.\index{reinitialization}
The call to the \id{CVodeReInitB} function has the form
%%
%%
\ucfunction{CVodeReInitB}
{
  flag = CVodeReInitB(cvadj\_mem, fB, tB0, yB0, itolB, reltolB, abstolB);
}
{
  The function \id{CVodeReInitB} provides required problem specifications 
  and reinitializes {\cvodes} for the backward problem.
}
{
  \begin{args}[cvadj\_mem]
  \item[cvadj\_mem] (\id{void *})
    pointer to the adjoint memory block returned by \id{CVadjMalloc}.
  \item[fB] (\Id{CVRhsFnB})
    is the {\C} function which computes $fB$, the right-hand side of the 
    backward ODE problem. This function has the form 
    \id{fB(t, y, yB, yBdot, f\_dataB)} (for full details see \S\ref{ss:user_fct_adj}).
  \item[tB0] (\id{realtype})
    specifies the endpoint where final conditions are provided for the 
    backward problem.
  \item[yB0] (\id{N\_Vector})
    is the final value of the backward problem. 
  \item[itolB] (\id{int}) 
    is either \Id{CV\_SS} or \Id{CV\_SV}, where \Id{itol}$=$\id{CV\_SS} indicates scalar relative error 
    tolerance and scalar absolute error tolerance, while \id{itol}$=$\id{CV\_SV} indicates scalar
    relative error tolerance and vector absolute error tolerance. 
  \item[reltolB] (\id{realtype *})
    \index{tolerances}
    is a pointer to the relative error tolerance.
  \item[abstolB] (\id{void *})
    is a pointer to the absolute error tolerance.
  \end{args}
}
{
  The return value \id{flag} (of type \id{int}) will be one of the following:
  \begin{args}[CV\_ADJMEM\_NULL]
  \item[\Id{CV\_SUCCESS}]
    The call to \id{CVodeReInitB} was successful.
  \item[\Id{CV\_ADJMEM\_NULL}]
    The \id{cvadj\_mem} argument was \id{NULL}.
  \item[\Id{CV\_BAD\_TB0}]
    The final time \id{tB0} is outside the interval over which the forward problem
    was solved.
  \item[\Id{CV\_MEM\_NULL}] 
    The {\cvodes} memory block for the backward problem was not initialized through a 
    previous call to \id{CVodeCreateB}.
  \item[\Id{CV\_NO\_MALLOC}] 
    Memory space for the {\cvodes} memory block for the backward problem was not 
    allocated through a previous call to \id{CVodeMallocB}.
  \item[\Id{CV\_ILL\_INPUT}] 
    An input argument to \id{CVodeReInitB} has an illegal value.
  \end{args}
}
{}
%%
%%
Note that memory allocated by \id{CVodeMallocB} is deallocated by the function 
\id{CVadjFree}.
%%

%%---------------------------------------------------------------------
\subsection{Linear solver initialization functions for backward problem}
\label{sss:lin_solv_b}

\index{CVODES@{\cvodes} linear solvers!usage with adjoint module|(} 
The adjoint sensitivity module in {\cvodes} provides interfaces to the initialization
functions of all supported linear solver modules for the case in which Newton
iteration is selected for the solution of the backward problem. The initialization
functions described in \S\ref{sss:lin_solv_init} cannot be directly used since the
optional user-defined Jacobian-related functions have different prototypes for the
backward problem than for the forward problem.
\index{CVODES@{\cvodes} linear solvers!usage with adjoint module|)} 

The following four wrapper functions can be used to initialize one of the linear
solver modules for the backward problem. Their arguments are identical to those
of the functions in \S\ref{sss:lin_solv_init} with the exception of their first
argument which must be the pointer to the adjoint memory block returned by
\id{CVadjMalloc}.
\begin{verbatim}
  flag = CVDenseB(cvadj_mem, nB);
  flag = CVDiagB(cvadj_mem);
  flag = CVBandB(cvadj_mem, nB, mupperB, mlowerB);
  flag = CVSpgmrB(cvadj_mem, pretypeB, maxlB);
\end{verbatim}
Their return value \id{flag} (of type \id{int}) can have any of the return values 
of their counterparts, but it can also be \Id{CV\_ADJMEM\_NULL} if the \id{cvadj\_mem} 
argument was \id{NULL}. 

%%---------------------------------------------------------------------
\subsection{Backward integration function}\label{sss:cvodeb}

The function \ID{CVodeB} performs the integration of the backward problem.
It is essentially a wrapper for the {\cvodes} main integration function 
\id{CVode} and, in the case in which checkpoints were needed, it evolves 
the solution of the backward problem through a sequence of forward-backward 
integrations between consecutive checkpoints. 
The first run integrates the original IVP forward in time and
stores interpolation data; the second run integrates the backward problem 
backward in time and performs Hermite interpolation to provide
the solution of the IVP to the backward problem.

The call to this function has the form
%%
%%
\ucfunction{CVodeB}
{
  flag = CVodeB(cvadj\_mem, tBout, yBout, tBret, itaskB);
}
{
  The function \ID{CVodeB} integrates the backward ODE problem over an 
  interval in $t$.
}
{
  \begin{args}[cvadj\_mem]
  \item[cvadj\_mem] (\id{void *})
    pointer to the adjoint memory block returned by \id{CVadjMalloc}.
  \item[tBout] (\id{realtype})
    the next time at which a computed solution is desired.
  \item[yBout] (\id{N\_Vector})
    the computed solution vector of the backward problem.
  \item[tBret] (\id{realtype *})
    the time reached by the solver.
  \item[itaskB] (\id{int})
    a flag indicating the job of the solver for the next step. 
    The \Id{CV\_NORMAL} task is to have the solver take internal steps until   
    it has reached or just passed the user specified \id{tBout}
    parameter. The solver then interpolates in order to   
    return an approximate value of $yB($\id{tBout}$)$. 
    The \Id{CV\_ONE\_STEP} option tells the solver to just take one internal step  
    and return the solution at the point reached by that step. 
  \end{args}
}
{
  On return, \id{CVodeB} returns a vector \id{yBout} and a corresponding 
  independent variable value $t=$\id{*tBret}, such that \id{yBout} is the computed 
  value of the solution of the backward problem.

  The return value \id{flag} (of type \id{int}) will be one of the following.
  For more details see \S\ref{sss:cvode}.
  \begin{args}[CV\_TOO\_MUCH\_WORK]
  \item[\Id{CV\_SUCCESS}]
    \id{CVodeB} succeeded.
  \item[\Id{CV\_NO\_MALLOC}]
    The {\cvodes} memory for the backward problem was \id{NULL}.
  \item[\Id{CV\_ILL\_INPUT}]
    One of the inputs to \id{CVode} is illegal.
  \item[\Id{CV\_TOO\_MUCH\_WORK}] 
    The solver took \id{mxstep} internal steps but could not reach \id{tBout}. 
  \item[\Id{CV\_TOO\_MUCH\_ACC}] 
    The solver could not satisfy the accuracy demanded by the user for some 
    internal step.
  \item[\Id{CV\_ERR\_FAILURE}]
    Error test failures occurred too many times during one internal time step.
  \item[\Id{CV\_CONV\_FAILURE}] 
    Convergence test failures occurred too many times during one internal time step.
  \item[\Id{CV\_LSETUP\_FAIL}] 
    The linear solver's setup function failed in an unrecoverable manner.
  \item[\Id{CV\_SOLVE\_FAIL}] 
    The linear solver's solve function failed in an unrecoverable manner.
  \item[\Id{CV\_ADJMEM\_NULL}]
    The \id{cvadj\_mem} argument was \id{NULL}.
  \item[\Id{CV\_BCKMEM\_NULL}]
    The \id{cvodes} memory for the backward problem was not created through
    a call to \id{CVodeCreateB}.
  \item[\Id{CV\_BAD\_TBOUT}]
    The desired output time \id{tBout} is outside the interval over which the 
    forward problem was solved.
  \item[\Id{CV\_REIFWD\_FAIL}]
    Reinitialization of the forward problem failed at the first checkpoint
    (corresponding to the initial time of the forward problem).
  \item[\Id{CV\_FWD\_FAIL}]
    An error occured during the integration of the forward problem.
  \end{args} 
}
{
  All failure return values are negative and therefore a test \id{flag}$< 0$
  will trap all \id{CVodeB} failures.
}

%%---------------------------------------------------------------------
\subsection{Optional input and output functions for the backward problem}
\label{ss:optional_io_B}

\subsubsection{Optional input functions}
%%
The adjoint module in {\cvodes} provides wrappers for most of the optional 
input functions defined in \S\ref{ss:optional_input}. The only difference is 
that the first argument of the optional input functions for the backward problem
is the pointer to the adjoint memory block, \id{cvadj\_mem} of type \id{void *},
returned by \id{CVadjMalloc}. 
The optional input functions defined for the backward problem are:
\begin{verbatim}
  flag = CVodeSetIterTypeB(cvadj_mem, iterB);
  flag = CVodeSetFdataB(cvadj_mem, f_dataB);
  flag = CVodeSetErrFileB(cvadj_mem, errfpB);
  flag = CVodeSetMaxOrdB(cvadj_mem, maxordB);
  flag = CVodeSetMaxNumStepsB(cvadj_mem, mxstepsB);
  flag = CVodeSetStabLimDetB(cvadj_mem, stldetB);
  flag = CVodeSetInitStepB(cvadj_mem, hinB);
  flag = CVodeSetMinStepB(cvadj_mem, hminB);
  flag = CVodeSetMaxStepB(cvadj_mem, hmaxB);
\end{verbatim}
Their return value \id{flag} (of type \id{int}) can have any of the return values 
of their counterparts, but it can also be \Id{CV\_ADJMEM\_NULL} if the \id{cvadj\_mem} 
argument was \id{NULL}.

Optional inputs for the {\cvdense} linear solver module can be set for the backward
problem through the following functions:
%%
\ucfunction{CVDenseSetJacFnB}
{
  flag = CVDenseSetJacFnB(cvadj\_mem, djacB);
}
{
  The function \ID{CVDenseSetJacFnB} specifies the dense Jacobian
  approximation function to be used for the backward problem.
}
{
  \begin{args}[cvadj\_mem]
  \item[cvadj\_mem] (\id{void *})
    pointer to the adjoint memory block.
  \item[djacB] (\id{CVDenseJacFnB})
    user-defined dense Jacobian approximation function.
  \end{args}
}
{
  The return value \id{flag} (of type \id{int}) is one of:
  \begin{args}[CVDENSE\_LMEM\_NULL]
  \item[\Id{CVDENSE\_SUCCESS}] 
    The optional value has been successfuly set.
  \item[\Id{CVDENSE\_MEM\_NULL}]
    The {\cvodes} solver memory block was not created through a call to \id{CVodeCreateB}.
  \item[\Id{CVDENSE\_LMEM\_NULL}]
    The {\cvdense} linear solver has not been initialized through a call to \id{CVDenseB}.
  \item[\Id{CV\_ADJMEM\_NULL}]
    The \id{cvadj\_mem} argument was \id{NULL}.
  \end{args}
}
{
  The function type \id{CVDenseJacFnB} is described in \S\ref{ss:user_fct_adj}.
}
%%
\ucfunction{CVDenseSetJacDataB}
{
  flag = CVDenseSetJacDataB(cvadj\_mem, jac\_dataB);
}
{
  The function \ID{CVDenseSetJacDataB} specifies the data structure
  to be passed to the user-supplied dense Jacobian approximation 
  function for the backward problem each time it is called.
}
{
  \begin{args}[cvadj\_mem]
  \item[cvadj\_mem] (\id{void *})
    pointer to the adjoint memory block.
  \item[jac\_dataB] (\id{void *})
    pointer to the user-defined data structure.
  \end{args}
}
{
  The return value \id{flag} (of type \id{int}) is one of:
  \begin{args}[CVDENSE\_LMEM\_NULL]
  \item[\Id{CVDENSE\_SUCCESS}] 
    The optional value has been successfuly set.
  \item[\Id{CVDENSE\_MEM\_NULL}]
    The {\cvodes} solver memory block was not created through a call to \id{CVodeCreateB}.
  \item[\Id{CVDENSE\_LMEM\_NULL}]
    The {\cvdense} linear solver has not been initialized through a call to \id{CVDenseB}.
  \item[\Id{CV\_ADJMEM\_NULL}]
    The \id{cvadj\_mem} argument was \id{NULL}.
  \end{args}
}
{}
%%
%%
Optional inputs for the {\cvband} linear solver module can be set for the backward
problem through the following functions:
%%
\ucfunction{CVBandSetJacFnB}
{
  flag = CVBandSetJacFnB(cvadj\_mem, bjacB);
}
{
  The function \ID{CVBandSetJacFnB} specifies the banded Jacobian
  approximation function to be used for the backward problem.
}
{
  \begin{args}[cvadj\_mem]
  \item[cvadj\_mem] (\id{void *})
    pointer to the adjoint memory block.
  \item[bjacB] (\id{CVBandJacFnB})
    user-defined banded Jacobian approximation function.
  \end{args}
}
{
  The return value \id{flag} (of type \id{int}) is one of:
  \begin{args}[CVBAND\_LMEM\_NULL]
  \item[\Id{CVBAND\_SUCCESS}] 
    The optional value has been successfuly set.
  \item[\Id{CVBAND\_MEM\_NULL}]
    The {\cvodes} solver memory block was not created through a call to \id{CVodeCreateB}.
  \item[\Id{CVBAND\_LMEM\_NULL}]
    The {\cvband} linear solver has not been initialized through a call to \id{CVBandB}.
  \item[\Id{CV\_ADJMEM\_NULL}]
    The \id{cvadj\_mem} argument was \id{NULL}.
  \end{args}
}
{
  The function type \id{CVBandJacFnB} is described in \S\ref{ss:user_fct_adj}.
}
%%
\ucfunction{CVBandSetJacDataB}
{
  flag = CVBandSetJacDataB(cvadj\_mem, jac\_dataB);
}
{
  The function \ID{CVBandSetJacDataB} specifies the data structure
  to be passed to the user-supplied banded Jacobian approximation 
  function for the backward problem each time it is called.
}
{
  \begin{args}[cvadj\_mem]
  \item[cvadj\_mem] (\id{void *})
    pointer to the adjoint memory block.
  \item[jac\_dataB] (\id{void *})
    pointer to the user-defined data structure.
  \end{args}
}
{
  The return value \id{flag} (of type \id{int}) is one of:
  \begin{args}[CVBAND\_LMEM\_NULL]
  \item[\Id{CVBAND\_SUCCESS}] 
    The optional value has been successfuly set.
  \item[\Id{CVBAND\_MEM\_NULL}]
    The {\cvodes} solver memory block was not created through a call to \id{CVodeCreateB}.
  \item[\Id{CVBAND\_LMEM\_NULL}]
    The {\cvband} linear solver has not been initialized through a call to \id{CVBandB}.
  \item[\Id{CV\_ADJMEM\_NULL}]
    The \id{cvadj\_mem} argument was \id{NULL}.
  \end{args}
}
{}
%%
%%
Optional inputs for the {\cvspgmr} linear solver module can be set for the backward
problem through the following functions:
%%
\ucfunction{CVSpgmrSetPrecSolveFnB}
{
  flag = CVSpgmrSetPrecSolveFnB(cvadj\_mem, psolveB);
}
{
  The function \ID{CVSpgmrSetPrecSolveFnB} specifies the preconditioner
  solve function.
}
{
  \begin{args}[cvode\_mem]
  \item[cvadj\_mem] (\id{void *})
    pointer to the adjoint memory block.
  \item[psolveB] (\id{CVSpgmrPrecSolveFnB})
    user-defined preconditioner solve function.
  \end{args}
}
{
  The return value \id{flag} (of type \id{int}) is one of:
  \begin{args}[CVSPGMR\_LMEM\_NULL]
  \item[\Id{CVSPGMR\_SUCCESS}] 
    The optional value has been successfuly set.
  \item[\Id{CVSPGMR\_MEM\_NULL}]
    The {\cvodes} solver memory block was not created through a call to \id{CVodeCreateB}.
  \item[\Id{CVSPGMR\_LMEM\_NULL}]
    The {\cvspgmr} linear solver has not been initialized through a call to \id{CVSpgmrB}.
  \item[\Id{CV\_ADJMEM\_NULL}]
    The \id{cvadj\_mem} argument was \id{NULL}.
  \end{args}
}
{
   The function type \id{CVSpgmrPrecSolveFnB} is described in \S\ref{ss:user_fct_adj}.
}
%%
%%
\ucfunction{CVSpgmrSetPrecSetupFnB}
{
  flag = CVSpgmrSetPrecSetupFnB(cvadj\_mem, psetupB);
}
{
  The function \ID{CVSpgmrSetPrecSetupFnB} specifies the preconditioner
  preprocessing function.
}
{
  \begin{args}[cvadj\_mem]
  \item[cvadj\_mem] (\id{void *})
    pointer to the adjoint memory block.
  \item[psetupB] (\id{CVSpgmrPrecSetupFnB})
    user-defined preconditioner setup function.
  \end{args}
}
{
  The return value \id{flag} (of type \id{int}) is one of:
  \begin{args}[CVSPGMR\_LMEM\_NULL]
  \item[\Id{CVSPGMR\_SUCCESS}] 
    The optional value has been successfuly set.
  \item[\Id{CVSPGMR\_MEM\_NULL}]
    The {\cvodes} solver memory block was not created through a call to \id{CVodeCreateB}.
  \item[\Id{CVSPGMR\_LMEM\_NULL}]
    The {\cvspgmr} linear solver has not been initialized through a call to \id{CVSpgmrB}.
  \item[\Id{CV\_ADJMEM\_NULL}]
    The \id{cvadj\_mem} argument was \id{NULL}.
  \end{args}
}
{
   The function type \id{CVSpgmrPrecSetupFnB} is described in \S\ref{ss:user_fct_adj}.
}
%%
%%
\ucfunction{CVSpgmrSetPrecDataB}
{
  flag = CVSpgmrSetPrecDataB(cvadj\_mem, P\_dataB);
}
{
  The function \ID{CVSpgmrSetPrecDataB} specifies the data structure
  to be passed to the user-supplied preconditioner setup and solve
  functions each time they are called.
}
{
  \begin{args}[cvadj\_mem]
  \item[cvadj\_mem] (\id{void *})
    pointer to the adjoint memory block.
  \item[P\_dataB] (\id{void *})
     pointer to the user-defined data structure.
  \end{args}
}
{
  The return value \id{flag} (of type \id{int}) is one of:
  \begin{args}[CVSPGMR\_LMEM\_NULL]
  \item[\Id{CVSPGMR\_SUCCESS}] 
    The optional value has been successfuly set.
  \item[\Id{CVSPGMR\_MEM\_NULL}]
    The {\cvodes} solver memory block was not created through a call to \id{CVodeCreateB}.
  \item[\Id{CVSPGMR\_LMEM\_NULL}]
    The {\cvspgmr} linear solver has not been initialized through a call to \id{CVSpgmrB}.
  \item[\Id{CV\_ADJMEM\_NULL}]
    The \id{cvadj\_mem} argument was \id{NULL}.
  \end{args}
}
{}
%%
%%
\ucfunction{CVSpgmrSetJacTimesVecFnB}
{
  flag = CVSpgmrSetJacTimesVecFnB(cvadj\_mem, jtimesB);
}
{
  The function \ID{CVSpgmrSetJacTimesFnB} specifies the Jacobian-vector 
  product function to be used.
}
{
  \begin{args}[cvadj\_mem]
  \item[cvadj\_mem] (\id{void *})
    pointer to the adjoint memory block.
  \item[jtimesB] (\id{CVSpgmrJacTimesVecFnB})
    user-defined Jacobian-vector product function.
  \end{args}
}
{
  The return value \id{flag} (of type \id{int}) is one of:
  \begin{args}[CVSPGMR\_LMEM\_NULL]
  \item[\Id{CVSPGMR\_SUCCESS}] 
    The optional value has been successfuly set.
  \item[\Id{CVSPGMR\_MEM\_NULL}]
    The {\cvodes} solver memory block was not created through a call to \id{CVodeCreateB}.
  \item[\Id{CVSPGMR\_LMEM\_NULL}]
    The {\cvspgmr} linear solver has not been initialized through a call to \id{CVSpgmrB}.
  \item[\Id{CV\_ADJMEM\_NULL}]
    The \id{cvadj\_mem} argument was \id{NULL}.
  \end{args}
}
{
  The function type \id{CVSpgmrJacTimesVecFnB} is described in \S\ref{ss:user_fct_adj}.
}
%%
%%
\ucfunction{CVSpgmrSetJacDataB}
{
  flag = CVSpgmrSetJacDataB(cvadj\_mem, jac\_dataB);
}
{
  The function \ID{CVSpgmrSetJacDataB} specifies the data structure
  to be passed to the user-supplied Jacobian-vector product
  function each time it is called.
}
{
  \begin{args}[cvadj\_mem]
  \item[cvadj\_mem] (\id{void *})
    pointer to the adjoint memory block.
  \item[jac\_dataB] (\id{void *})
     pointer to the user-defined data structure.
  \end{args}
}
{
  The return value \id{flag} (of type \id{int}) is one of:
  \begin{args}[CVSPGMR\_LMEM\_NULL]
  \item[\Id{CVSPGMR\_SUCCESS}] 
    The optional value has been successfuly set.
  \item[\Id{CVSPGMR\_MEM\_NULL}]
    The {\cvodes} solver memory block was not created through a call to \id{CVodeCreateB}.
  \item[\Id{CVSPGMR\_LMEM\_NULL}]
    The {\cvspgmr} linear solver has not been initialized through a call to \id{CVSpgmrB}.
  \item[\Id{CV\_ADJMEM\_NULL}]
    The \id{cvadj\_mem} argument was \id{NULL}.
  \end{args}
}
{}
%%
%%
\ucfunction{CVSpgmrSetGSTypeB}
{
  flag = CVSpgmrSetGSType(cvadj\_mem, gstypeB);
}
{
  The function \ID{CVSpgmrSetGSTypeB} specifies the type of
  Gram-Schmidt orthogonalization to be used. 
  This must be one of the enumeration constants \ID{MODIFIED\_GS}
  or \ID{CLASSICAL\_GS}. These correspond to using modified Gram-Schmidt 
  and classical Gram-Schmidt, respectively. 
  \index{Gram-Schmidt procedure}
}
{
  \begin{args}[cvadj\_mem]
  \item[cvadj\_mem] (\id{void *})
    pointer to the adjoint memory block.
  \item[gstypeB] (\id{int})
    type of Gram-Schmidt orthogonalization.
  \end{args}
}
{
  The return value \id{flag} (of type \id{int}) is one of:
  \begin{args}[CVSPGMR\_ILL\_INPUT]
  \item[\Id{CVSPGMR\_SUCCESS}] 
    The optional value has been successfuly set.
  \item[\Id{CVSPGMR\_MEM\_NULL}]
    The {\cvodes} solver memory block was not created through a call to \id{CVodeCreateB}.
  \item[\Id{CVSPGMR\_LMEM\_NULL}]
    The {\cvspgmr} linear solver has not been initialized through a call to \id{CVSpgmrB}.
  \item[\Id{CVSPGMR\_ILL\_INPUT}]
    The Gram-Schmidt orthogonalization type \id{gstypeB} is not valid.
  \item[\Id{CV\_ADJMEM\_NULL}]
    The \id{cvadj\_mem} argument was \id{NULL}.
  \end{args}
}
{
  The default value is \id{MODIFIED\_GS}.
}
%%
%%
\ucfunction{CVSpgmrSetDeltB}
{
  flag = CVSpgmrSetDeltB(cvadj\_mem, deltB);
}
{
  The function \ID{CVSpgmrSetDeltB} specifies the factor by
  which the GMRES\index{GMRES method} convergence test constant is reduced
  from the Newton iteration test constant.
}
{
  \begin{args}[cvadj\_mem]
  \item[cvadj\_mem] (\id{void *})
    pointer to the adjoint memory block.
  \item[delt] (\id{realtype})
    the value of the convergence test constant reduction factor.
  \end{args}
}
{
  The return value \id{flag} (of type \id{int}) is one of:
  \begin{args}[CVSPGMR\_ILL\_INPUT]
  \item[\Id{CVSPGMR\_SUCCESS}] 
    The optional value has been successfuly set.
  \item[\Id{CVSPGMR\_MEM\_NULL}]
    The {\cvodes} solver memory block was not created through a call to \id{CVodeCreateB}.
  \item[\Id{CVSPGMR\_LMEM\_NULL}]
    The {\cvspgmr} linear solver has not been initialized through a call to \id{CVSpgmrB}.
  \item[\Id{CVSPGMR\_ILL\_INPUT}]
    The factor \id{deltB} is negative.  
  \item[\Id{CV\_ADJMEM\_NULL}]
    The \id{cvadj\_mem} argument was \id{NULL}.
  \end{args}
}
{
  The default value is $0.05$.
  Passing a value \id{deltB}$ = 0.0$ also indicates using the default value.
}
%%
%%
\ucfunction{CVSpgmrSetPrecTypeB}
{
  flag = CVSpgmrSetPrecTypeB(cvadj\_mem, pretypeB);
}
{
  The function \ID{CVSpgmrSetPrecTypeB} resets the type
  of preconditioning to be used.
}
{
  \begin{args}[cvadj\_mem]
  \item[cvadj\_mem] (\id{void *})
    pointer to the adjoint memory block.
  \item[pretypeB] (\id{int})
    \index{pretypeB@\texttt{pretypeB}}
    specifies the type of prconditioning and may be
    \Id{PREC\_NONE}, \Id{PREC\_LEFT}, \Id{PREC\_RIGHT}, or \Id{PREC\_BOTH}.
  \end{args}
}
{
  The return value \id{flag} (of type \id{int}) is one of:
  \begin{args}[CVSPGMR\_ILL\_INPUT]
  \item[\Id{CVSPGMR\_SUCCESS}] 
    The optional value has been successfuly set.
  \item[\Id{CVSPGMR\_MEM\_NULL}]
    The {\cvodes} solver memory block was not created through a call to \id{CVodeCreateB}.
  \item[\Id{CVSPGMR\_LMEM\_NULL}]
    The {\cvspgmr} linear solver has not been initialized through a call to \id{CVSpgmrB}.
  \item[\Id{CVSPGMR\_ILL\_INPUT}]
    The preconditioner type \id{pretype} is not valid.
  \item[\Id{CV\_ADJMEM\_NULL}]
    The \id{cvadj\_mem} argument was \id{NULL}.
  \end{args}
}
{
  The preconditioning type is initially specified in the call
  to \id{CVSpgmrB} (see \S\ref{sss:lin_solv_b}). This function call is
  needed only if \id{pretypeB} is being changed from its value in the
  previous call to \id{CVSpgmrB}.
}
%%

\subsubsection{Optional output functions}
%%
The user of the adjoint module in {\cvodes} has access to any of the optional output 
functions described in \S\ref{ss:optional_output}, both for the main solver and for
the linear solver modules. The first argument of these \id{CVodeGet*} functions is
the {\cvodes} memory block for the backward problem. In order to call any of these
functions, the user must first call the following function to obtain a pointer to
this memory block:
%%
\ucfunction{CVadjGetCVodeBmem}
{
  cvode\_memB = CVadjGetCVodeBmem(cvadj\_mem);
}
{
  The function \ID{CvadjGetCVodeBmem} returns a pointer to the {\cvodes}
  memory block for the backward problem.
}
{
  The argument \id{cvadj\_mem} (of type \id{void *}) is a pointer to the
  adjoint memory block returned by \id{CVadjMalloc}.
}
{
  The return value, \id{cvode\_memB} (of type \id{void *}), is a pointer to the
  {\cvodes} memory for the backward problem.
}
{}
%%
%%

%%---------------------------------------------------------------------
\subsection{Backward integration of pure quadrature equations}

\subsubsection{Backward quadrature initialization functions}\label{sss:cvodequadmallocb}

The function \id{CVodeQuadMallocB} initializes and allocates memory for the backward
integration of quadrature equations. It has the following form:
%%
%%
\ucfunction{CVodeQuadMallocB}
{
flag = CVodeQuadMallocB(cvadj\_mem, fQB, yQB0);
}
{
  The function \ID{CVodeQuadMallocB} provides required problem specifications,
  allocates internal memory, and initializes backward quadrature integration.
}
{
  \begin{args}[cvadj\_mem]
  \item[cvadj\_mem] (\id{void *})
    pointer to the adjoint memory block returned by \id{CVadjMalloc}.
  \item[fQB] (\Id{CVQuadRhsFnB})
    is the {\C} function which computes $fQB$, the right-hand side of the 
    backward quadrature equations. This function has the form 
    \id{fQB(t, y, yB, qBdot, fQ\_dataB)} (for full details see below).
  \item[yQB0] (\id{N\_Vector})
    is the value of the quadrature variables at \id{tB0}.
  \end{args}
}
{
  The return value \id{flag} (of type \id{int}) will be one of the following:
  \begin{args}[CV\_ADJMEM\_NULL]
  \item[\Id{CV\_SUCCESS}]
    The call to \id{CVodeQuadMallocB} was successful.
  \item[\Id{CV\_MEM\_NULL}] 
    The {\cvodes} solver memory block was not created through a previous 
    call to \id{CVodeCreateB}.
  \item[\Id{CV\_MEM\_FAIL}] 
    A memory allocation request has failed.
  \item[\Id{CV\_ADJMEM\_NULL}]
    The \id{cvadj\_mem} argument was \id{NULL}.
  \end{args}
}
{}
%%
%%
The integration of quadrature equations during the backward phase can be
re-initialized by calling
%%
%%
\ucfunction{CVodeQuadReInitB}
{
  flag = CVodeQuadReInitB(cvadj\_mem, fQB, yQB0);
}
{
  The function \id{CVodeQuadReInitB} provides required problem specifications 
  and re-initializes the backward quadrature integration.
}
{
  \begin{args}[cvadj\_mem]
  \item[cvadj\_mem] (\id{void *})
    pointer to the adjoint memory block.
  \item[fQB] (\Id{CVQuadRhsFnB})
    is the {\C} function which computes $fQB$, the right-hand side of the 
    backward quadrature equations.
  \item[yQB0] (\id{N\_Vector})
    is the value of the quadrature variables at \id{tB0}.
  \end{args}
}
{
  The return value \id{flag} (of type \id{int}) will be one of the following:
  \begin{args}[CV\_ADJMEM\_NULL]
  \item[\Id{CV\_SUCCESS}]
    The call to \id{CVodeReInitB} was successful.
  \item[\Id{CV\_MEM\_NULL}] 
    The {\cvodes} solver memory block was not created through a 
    previous call to \id{CVodeCreateB}.
  \item[\Id{CV\_NO\_QUAD}] 
    Quadrature integration was not activated through a 
    previous call to \id{CVodeQuadMallocB}.
  \item[\Id{CV\_ADJMEM\_NULL}]
    The \id{cvadj\_mem} argument was \id{NULL}.
  \end{args}
}
{}

\subsubsection{Backward Quadrature Extraction Function}

To extract the values of the quadrature variables at the last return time
of \id{CVodeB}, {\cvodes} provides a wrapper for the function \ID{CVodeGetQuad}
(see \S\ref{ss:quad_get}). The call to this function has the form
%%
%%
\ucfunction{CVodeGetQuadB}
{
  flag = CVodeGetQuadB(cvadj\_mem, yQB);
}
{
  The function \id{CVodeGetQuadB} returns the quadrature solution vector after
  a successful return from \id{CVodeB}.
}
{
  \begin{args}[cvadj\_mem]
  \item[cvadj\_mem] (\id{void *})
    pointer to the adjoint memory returned by \id{CVadjMalloc}.
  \item[yQB] (\id{N\_Vector})
    the computed backward quadrature vector.
  \end{args}
}
{
  The return value \id{flag} of \id{CVodeGetQuadB} is one of:
  \begin{args}[CV\_ADJMEM\_NULL]
  \item[\Id{CV\_SUCCESS}] 
    \id{CVodeGetQuadB} was successful.
  \item[\Id{CV\_MEM\_NULL}] 
    The {\cvodes} solver memory block was not created through a 
    previous call to \id{CVodeCreateB}.
  \item[\Id{CV\_NO\_QUAD}] 
    Quadrature integration was not initialized through a 
    previous call to \id{CVodeQuadMallocB}..
  \item[\Id{Cv\_BAD\_DKY}] 
    \id{yQB} is \id{NULL}.
  \item[\Id{CV\_ADJMEM\_NULL}]
    The \id{cvadj\_mem} argument was \id{NULL}.
  \end{args}
}
{}

\subsubsection{Optional input and output functions for backward quadrature integration}

Optional values controlling the backward integration of quadrature equations can be changed
from their default values through calls to one of the following functions which are wrappers for
the corresponding optional input functions defined in \S\ref{ss:quad_optional_input}:
\begin{verbatim}
  flag = CVodeSetQuadFdataB(cvadj_mem, fQ_dataB);
  flag = CVodeSetQuadErrConB(cvadj_mem, errconQB);
  flag = CVodeSetQuadTolerancesB(cvadj_mem, itolQB, reltolQB, abstolQB);
\end{verbatim}
Their return value \id{flag} (of type \id{int}) can have any of the return values 
of their counterparts, but it can also be \Id{CV\_ADJMEM\_NULL} if the \id{cvadj\_mem} 
argument was \id{NULL}. 

Access to optional outputs related to backward quadrature integration can be obtained
by calling the corresponding \id{CVodeGetQuad*} functions 
(see \S\ref{ss:quad_optional_output}). 
A pointer to the {\cvodes} memory block for the backward problem, required as the first 
argument of these functions, can be obtained through a call to the functions 
\id{CVadjGetCVodeBmem} (see \S\ref{ss:optional_io_B}).

%%---------------------------------------------------------------------

\subsection{Check Point Listing Function}

For debugging purposes, {\cvodes} provides a function \ID{CVadjGetCheckPointsList} which
displays partial information from the linked list of checkpoints generated by
\id{CVodeF}. The call to this function has the form:
%%
%%
\ucfunction{CVadjGetCheckPointsList}
{
  CVadjCheckPointsList(cvadj\_mem);
}
{
  The function \ID{CVadjGetCheckPointsList} prints
  the memory addresses of the checkpoints and the time points at
  which the checkpoints were defined.
}
{
  The argument \id{cvadj\_mem} (of type \id{void *}) is a pointer to the
  adjoint memory block returned by \id{CVadjMalloc}.
}
{
  This function has no return value.
}
{
  For a typical output of \id{CVadjGetCheckPointsList}, see \cite{cvodes2.1.0_ex}.
}
%%
%%
%%-------------------------------------------------------------------------------------
\section{User-supplied functions for adjoint sensitivity analysis}\label{ss:user_fct_adj}
%%-------------------------------------------------------------------------------------
%%
In addition to the required ODE right-hand side function and any optional functions
for the forward problem, when using the adjoint sensitivity module in {\cvodes},
the user must supply one function defining the backward problem ODE and, optionally,
functions to supply Jacobian-related information (if Newton iteration is chosen) and
one or two functions that define the preconditioner (if the {\cvspgmr} solver is
selected) for the backward problem.
Type definitions for all these user-supplied functions are given below.

\subsubsection{ODE right-hand side for the backward problem}
\index{adjoint sensitivity analysis!right-hand side evaluation}
\index{right-hand side function!backward problem}

The user must provide a function of type \ID{CVRhsFnB} defined as follows:
\usfunction{CVRhsFnB}
{
  typedef void (*CVRhsFnB)(&realtype t, N\_Vector y, N\_Vector yB, \\
                           &N\_Vector yBdot, void *f\_dataB);
}
{
  This function computes the right-hand side of the backward problem ODE system.
  This could be (\ref{e:adj_eqns}) or (\ref{e:adj1_eqns}).
}
{
  \begin{args}[f\_dataB]
  \item[t]
    is the current value of the independent variable.
  \item[y]
    is the current value of the forward solution vector.
  \item[yB]
    is the current value of the dependent variable vector.
  \item[yBdot]
    is the output vector containing the right-hand side of the backward ODE problem.
  \item[f\_dataB]
    is a pointer to user data - the same as the \Id{f\_dataB}      
    parameter passed to \id{CVodeSetFdataB}.   
  \end{args}
}
{
  A \id{CVRhsFnB} function type does not have a return value.                        
}
{
  Allocation of memory for \id{yBdot} is handled within {\cvodes}.

  The \id{y}, \id{yB}, and \id{yBdot} arguments are all of type \id{N\_Vector},
  but \id{yB} and \id{yBdot}  typically have different internal representations
  from \id{y}. It is the user's 
  responsibility to access the vector data consistently (including the use of the 
  correct accessor macros from each {\nvector} implementation). For the sake of 
  computational efficiency, the vector functions in the two {\nvector} implementations 
  provided with {\cvodes} do not perform any consistency checks with respect to their 
  \id{N\_Vector} arguments (see \S\ref{ss:nvec_ser} and \S\ref{ss:nvec_par}).

  The \id{f\_dataB} pointer is passed to 
  the user's \id{fB} function every time it is called and can be the same as the 
  \id{f\_data} pointer used for the forward problem.
}

%%-------------------------------------------------------------------------------------
\subsubsection{Quadrature right-hand side for the backward problem}
\index{right-hand side function!quadrature backward problem}
\index{adjoint sensitivity analysis!quadrature evaluation}
The user must provide a function of type \ID{CVQuadRhsFnB} defined by
\usfunction{CVQuadRhsFnB}
{
  typedef void (*CVQuadRhsFnB)(&realtype t, N\_Vector y, N\_Vector yB, \\
                               &N\_Vector qBdot, void *fQ\_dataB);
}
{
  This function computes the quadrature equation right-hand side for the
  backward problem.
}
{
  \begin{args}[fQ\_dataB]
  \item[t]
    is the current value of the independent variable.
  \item[y]
    is the current value of the forward solution vector.
  \item[yB]
    is the current value of the dependent variable vector.
  \item[qBdot]
    is the output vector containing the right-hand side of the backward quadrature
    equations.
  \item[fQ\_dataB]
    is a pointer to user data - the same as the \Id{fQ\_dataB}      
    parameter passed to \id{CVodeSetQuadFdataB}.   
  \end{args}
}
{
  A \id{CVQuadRhsFnB} function type does not have a return value.                        
}
{
  Allocation of memory for \id{qBdot} is handled within {\cvodes}.

  The \id{y}, \id{yB}, and \id{yQBdot} arguments are all of type \id{N\_Vector},
  but they  typically all have different internal representations. It is the user's 
  responsibility to access the vector data consistently (including the use of the 
  correct accessor macros from each {\nvector} implementation). For the sake of 
  computational efficiency, the vector functions in the two {\nvector} implementations 
  provided with {\cvodes} do not perform any consistency checks with repsect to their 
  \id{N\_Vector} arguments (see \S\ref{ss:nvec_ser} and \S\ref{ss:nvec_par}).

  The \id{fQ\_dataB} pointer is passed to the user's \id{fQB} function every time 
  it is called and can be the same as the \id{f\_data} pointer used for the forward problem.
}

%%-------------------------------------------------------------------------------------
\subsubsection{Jacobian information for the backward problem 
  (direct method with dense Jacobian)}
  
If the direct linear solver with dense treatment of the Jacobian is selected
for the backward problem (i.e. \Id{CVDenseB} is called in step \ref{i:lin_solverB} 
of \S\ref{ss:skeleton_adj}), the user may provide, through a call to \id{CVDenseSetJacFnB}
(see \S\ref{ss:optional_io_B}), a function of the following type:
%%
\usfunction{CVDenseJacFn}
{
  typedef void (*CVDenseJacFnB)(&long int nB, DenseMat JB, realtype t, \\
                                &N\_Vector y, N\_Vector yB, N\_Vector fyB, \\
                                &void *jac\_dataB, N\_Vector tmp1B, \\
                                &N\_Vector tmp2B, N\_Vector tmp3B);
}
{
  This function computes the dense Jacobian of the backward problem (or an approximation
  to it). If the backward problem is the adjoint of the original IVP, then
  this Jacobian is just the transpose of $J = \dfdyI$ with
  a change in sign.
}
{
  \begin{args}[jac\_dataB]
  \item[nB]
    is the backward problem size.
  \item[J]
    is the output Jacobian matrix.  
  \item[t]
    is the current value of the independent variable.
  \item[y]
    is the current value of the forward solution vector.
  \item[yB]
    is the current value of the dependent variable vector.
  \item[fyB]
    is the current value of the right-hand side of the backward problem.
  \item[jac\_dataB]
    is a pointer to user data - the same as the \id{jac\_dataB}      
    parameter passed to \id{CVDenseSetJacDataB}.   
  \item[tmp1B]
  \item[tmp2B]
  \item[tmp3B]
    are pointers to memory allocated    
    for variables of type \id{N\_Vector} which can be used by           
    \id{CVDenseJacFnB} as temporary storage or work space.    
  \end{args}
}
{
  A \id{CVDenseJacFnB} function type does not have a return value.                        
}
{
  A user-supplied dense Jacobian function must load the \id{nB} by \id{nB}
  dense matrix \id{JB} with an approximation to the Jacobian matrix
  at the point (\id{t},\id{y},\id{yB}), where \id{y} is the solution
  of the original IVP at time \id{t} and \id{yB} is the solution of the
  backward problem at the same time.
  Only nonzero elements need to be loaded into \id{JB} as this matrix 
  is set to zero before the call to the Jacobian function. 
  The type of \id{JB} is \Id{DenseMat}. The user is referred to 
  \S\ref{ss:djacFn} for details regarding accessing a \id{DenseMat} object.
}

%%-------------------------------------------------------------------------------------
\subsubsection{Jacobian information for the backward problem
  (direct method with banded Jacobian)}

If the direct linear solver with banded treatment of the Jacobian is selected
for the backward problem (i.e. \Id{CVBandB} is called in step \ref{i:lin_solverB} 
of \S\ref{ss:skeleton_adj}), the user may provide, through a call to \id{CVBandSetJacFnB}
(see \S\ref{ss:optional_io_B}), a function the following type:
%%
\usfunction{CVBandJacFnB}
{
 typedef void (*CVBandJacFnB)(&long int nB, \\
                              &long int mupperB, long int mlowerB, \\
                              &BandMat JB, realtype t, N\_Vector y, \\
                              &N\_Vector yB, N\_Vector fyB, void *jac\_dataB, \\
                              &N\_Vector tmp1B, N\_Vector tmp2B, \\
                              &N\_Vector tmp3B);
}
{
  This function computes the banded Jacobian of the backward problem
  (or a banded approximation to it).
}
{
  \begin{args}[jac\_dataB]
  \item[nB]
    is the backward problem size.
  \item[mlowerB]
  \item[mupperB]
    are the lower and upper half-bandwidth of the Jacobian.
  \item[JB]
    is the output Jacobian matrix.  
  \item[t]
    is the current value of the independent variable.
  \item[y]
    is the current value of the forward solution vector.
  \item[yB]
    is the current value of the dependent variable vector.
  \item[fyB]
    is the current value of the right-hand side of the backward problem.
  \item[jac\_dataB]
    is a pointer to user data - the same as the \id{jac\_dataB}      
    parameter passed to \id{CVBandSetJacDataB}.   
  \item[tmp1B]
  \item[tmp2B]
  \item[tmp3B]
    are pointers to memory allocated    
    for variables of type \id{N\_Vector} which can be used by           
    \id{CVBandJacFnB} as temporary storage or work space.    
  \end{args}
}
{
  A \id{CVBandJacFnB} function type does not have a return value.                        
}
{
  A user-supplied band Jacobian function must load the band matrix \id{JB}
  (of type \Id{BandMat}) with the elements of the Jacobian at the
  point (\id{t},\id{y},\id{yB}), where \id{y} is the solution
  of the original IVP at time \id{t} and \id{yB} is the solution of the
  backward problem at the same time.  
  Only nonzero elements need to be loaded into
  \id{JB} because \id{JB} is preset to zero before the call to the
  Jacobian function. More details on the accessor macros provided for
  a \id{BandMat} object and on the rest of the arguments passed to a function
  of type \id{CVBandJacFnB} are given in \S\ref{ss:bjacFn}.
}

%%-------------------------------------------------------------------------------------
\subsubsection{Jacobian information for the backward problem
  ({\spgmr} case)}

If an iterative {\spgmr} linear solver is selected (\id{CVSpgmrB} is called in step 
\ref{i:lin_solverB} of \S\ref{ss:skeleton_adj}), the user may provide, through a call 
to \id{CVSpgmrSetJacTimesVecFnB} (see \S\ref{ss:optional_io_B}), a function
of the following type:
%%
\usfunction{CVSpgmrJacTimesVecFnB}
{
  typedef int (*CVSpgmrJacTimesVecFnB)(&N\_Vector vB, N\_Vector JvB, \\
                                       &realtype t, N\_Vector y, \\
                                       &N\_Vector yB, N\_Vector fyB, \\
                                       &void *jac\_dataB, N\_Vector tmpB);
}
{
  This function computes the action of the Jacobian on a given vector \id{vB} for
  the backward problem (or an approximation to it).
}
{
  \begin{args}[jac\_dataB]
  \item[vB]
    is the vector by which the Jacobian must be multiplied to the right.
  \item[JvB]
      is the output vector computed.
  \item[t]
    is the current value of the independent variable.
  \item[y]
    is the current value of the forward solution vector.
  \item[yB]
    is the current value of the dependent variable vector.
  \item[fyB]
    is the current value of the right-hand side of the backward problem.
  \item[jac\_dataB]
    is a pointer to user data - the same as the \id{jac\_dataB}      
    parameter passed to \id{CVSpgmrSetJacDataB}.   
  \item[tmpB]
    is a pointer to memory allocated for a vector which can be used for work space.
  \end{args}
}
{  
  The return value of a function of type \id{CVSpgmrJtimesFnB} should be
  $0$ if successful or nonzero if an error was encountered, in which case
  the integration is halted.
}
{
  A user-supplied Jacobian-vector product function must load the vector \id{JvB}
  with the result of the product between the Jacobian of the backward problem 
  at the point (\id{t},\id{y}, \id{yB}) and the vector \id{vB}. 
  Here, \id{y} is the solution of the original IVP at time \id{t} and 
  \id{yB} is the solution of the backward problem at the same time.  
  The rest of the arguments are equivalent to those passed to a function of type
  \id{CVSpgmrJtimesFn} (see \S\ref{ss:jtimesFn}).
  If the backward problem is the adjoint of ${\dot y} = f(t, y)$, then this 
  function is to compute $-(\dfdyI)^T v_B$.
}

%%-------------------------------------------------------------------------------------
\subsubsection{Preconditioning for the backward problem (linear system solution)}

If preconditioning is used during integration of the backward problem, 
then the user must provide a {\C} function to solve the linear system 
$Pz = r$, where $P$ may be either a left or a right preconditioner matrix.
This function must be of type \ID{CVSpgmrPSolveFnB} defined by
%%
\usfunction{CVSpgmrPrecSolveFnB}
{
  typedef int (*CVSpgmrPrecSolveFnB)(&realtype t, N\_Vector y, N\_Vector yB, \\
                                     &N\_Vector fyB, N\_Vector rB, \\
                                     &N\_Vector zB, realtype gammaB, \\
                                     &realtype deltaB, int lrB, \\
                                     &void *P\_dataB, N\_Vector tmpB);
}
{
  This function solves the preconditioning system $Pz = r$ for the backward problem.
}
{  
  \begin{args}[P\_dataB]
  \item[t]
    is the current value of the independent variable.
  \item[y]
    is the current value of the forward solution vector.
  \item[yB]
    is the current value of the dependent variable vector.
  \item[fyB]
    is the current value of the right-hand side of the backward problem.
  \item[rB]
    is the right-hand side vector of the linear system.
  \item[zB]
    is the output vector computed.
  \item[gammaB]
    is the scalar $\gamma$ appearing in the Newton matrix.
  \item[deltaB]
    is an input tolerance to be used if an iterative method 
    is employed in the solution.
  \item[lrB]
    is an input flag indicating whether the preconditioner solve
    function is to use the left preconditioner (\id{lr=1}) or 
    the right preconditioner (\id{lr=2});
  \item[P\_dataB]
    is a pointer to user data - the same as the \id{P\_dataB}      
    parameter passed to the function \id{CVSpgmrSetPrecDataB}.
  \item[tmpB]
    is a pointer to memory allocated for a vector which can be used for work space.
  \end{args}
}
{
  The return value of a preconditioner solve function for the backward
  problem should be $0$ if successful, 
  positive for a recoverable error (in which case the step will be retried), or
  negative for an unrecoverable error (in which case the integration is halted).
}
{}

%%-------------------------------------------------------------------------------------
\subsubsection{Preconditioning for the backward problem (Jacobian data)}

If the user's preconditioner requires that any Jacobian-related data
be preprocessed or evaluated, then this needs to be done in a
user-supplied {\C} function of type \ID{CVSpgmrPrecondFnB} defined by
%%
\usfunction{CVSpgmrPrecSetupFnB}
{
  typedef int (*CVSpgmrPrecSetupFnB)(&realtype t, N\_Vector y, N\_Vector yB, \\ 
                                     &N\_Vector fyB, booleantype jokB, \\
                                     &booleantype *jcurPtrB, \\
                                     &realtype gammaB, void *P\_dataB,\\
                                     &N\_Vector tmp1B, N\_Vector tmp2B, \\
                                     &N\_Vector tmp3B);
}
{
  This function preprocesses and/or evaluates Jacobian-related data needed
  by the preconditioner for the backward problem.
}
{
  The arguments of a \id{CVSpgmrPrecSetupFnB} are as follows:
  \begin{args}[jcurPtrB]
  \item[t]
    is the current value of the independent variable.
  \item[y]
    is the current value of the forward solution vector.
  \item[yB]
    is the current value of the dependent variable vector.
  \item[fyB]
    is the current value of the right-hand side of the backward problem.
  \item[jokB]
    is an input flag indicating whether Jacobian-related   
    data needs to be recomputed.
  \item[jcurPtrB]
    is a pointer to an output integer flag which is        
    to be set to \id{TRUE} if Jacobian data was recomputed, or   
    to \id{FALSE} if Jacobian data was not recomputed but saved data was still reused.
  \item[gammaB]
    is the scalar $\gamma$ appearing in the Newton matrix.
  \item[P\_dataB]
    is a pointer to user data - the same as the \id{P\_dataB}      
    parameter passed to \id{CVSpgmrSetPrecDataB}.
  \item[tmp1B]
  \item[tmp2B]
  \item[tmp3B]
    are pointers to memory allocated for vectors which can be used           
    as temporary storage or work space.    
  \end{args}
}
{
  The return value of a preconditioner setup function for the backward
  problem should be $0$ if successful, 
  positive for a recoverable error (in which case the step will be retried),     
  or negative for an unrecoverable error (in which case the integration is halted).
}
{}

%%
%%-------------------------------------------------------------------
\section{Using CVODES preconditioner modules for the backward problem}
%%-------------------------------------------------------------------
%%

\subsection{Using the banded preconditioner CVBANDPRE}

\index{CVBANDPRE@{\cvbandpre} preconditioner!usage with adjoint module|(}
The adjoint module in {\cvodes} offers an interface to the banded 
preconditioner module {\cvbandpre} described in section \S\ref{sss:cvbandpre}.
This preconditioner provides a band matrix preconditioner based on difference
quotients of the backward problem right-hand side function \id{fB}. It generates
a banded approximation to the Jacobian with $m_{lB}$ sub-diagonals and $m_{uB}$
super-diagonals to be used with the Krylov linear solver in {\cvspgmr}.

In order to use the {\cvbandpre} module in the solution of the backward problem,
the user need not define any additional functions. First, the user must initialize the
{\cvbandpre} module by calling
%%
%%
\ucfunction{CVBandPrecAllocB}
{
  flag = CVBandPrecAlloc(cvadj\_mem, nB, muB, mlB);
}
{
  The function \ID{CVBandPrecAllocB} initializes and allocates
  memory for the {\cvbandpre} preconditioner for the backward problem.
}
{
  \begin{args}[cvadj\_mem]
  \item[cvadj\_mem] (\id{void *})
    pointer to the adjoint memory block returned by \id{CVadjMalloc}.
  \item[nB] (\id{long int})
    backward problem dimension.
  \item[muB] (\id{long int})
    upper half-bandwidth of the backward problem Jacobian approximation.
  \item[mlB] (\id{long int})
    lower half-bandwidth of the backward problem Jacobian approximation.
  \end{args}
}
{
  If successful, \id{CVBandPrecAlloc} stores a pointer to the newly created 
  {\cvbandpre} memory block.
  The return value \id{flag} (of type \id{int}) is one of:
  \begin{args}[CV\_ADJMEM\_NULL]
  \item[\Id{CV\_SUCCESS}]
    The call to \id{CVBandPrecAllocB} was successful.
  \item[\Id{CV\_PDATA\_NULL}]
    An error occured while trying to create the {\cvbandpre} memory block.
  \item[\Id{CV\_ADJMEM\_NULL}]
    The \id{cvadj\_mem} argument was \id{NULL}.
  \end{args}
}
{}
%%
%%
To specify the use of the {\cvspgmr} linear solver module with the {\cvbandpre} 
preconditioner module, make the following call:
%%
\ucfunction{CVBPSpgmrB}
{
  flag = CVBPSpgmrB(cvadj\_mem, pretypeB, maxlB);
}
{
  The function \ID{CVBPSpgmrB} links the {\cvbandpre} data to the
  {\cvspgmr} linear solver and attaches the latter to the {\cvodes}
  memory block for the backward problem.
}
{
  \begin{args}[cvadj\_mem]
  \item[cvadj\_mem] (\id{void *})
    pointer to the adjoint memory block returned by \id{CVadjMalloc}.
  \item[pretypeB] (\id{int})
    \index{pretypeB@\texttt{pretypeB}}
    preconditioning type. Can be one of \Id{PREC\_LEFT} or \Id{PREC\_RIGHT}.
  \item[maxlB] (\id{int})
    maximum dimension of the Krylov subspace to be used. Pass $0$ to use the 
    default value \id{CVSPGMR\_MAXL}$=5$.
  \end{args}
}
{
  The return value \id{flag} (of type \id{int}) is one of:
  \begin{args}[CVSPGMR\_ILL\_INPUT]
  \item[\Id{CV\_SUCCESS}]
    The {\cvspgmr} initialization was successful.
  \item[\Id{CVSPGMR\_MEM\_NULL}]
    The {\cvodes} memory block for the backward problem was not initialized through a 
    previous call to \id{CVodeCreateB}.
  \item[\Id{CVSPGMR\_ILL\_INPUT}]
    The preconditioner type \id{pretypeB} is not valid.
  \item[\Id{CVSPGMR\_MEM\_FAIL}]
    A memory allocation request failed.
  \item[\Id{CV\_PDATA\_NULL}]
    The {\cvbandpre} preconditioner has not been initialized.
  \item[\Id{CV\_ADJMEM\_NULL}]
    The \id{cvadj\_mem} argument was \id{NULL}.
  \end{args}
}
{}
%%
%%
Note that memory allocated by \id{CVBandPrecAllocB} is deallocated 
by the function \id{CVadjFree}.
For more details on {\cvbandpre} see \S\ref{sss:cvbandpre}.
\index{CVBANDPRE@{\cvbandpre} preconditioner!usage with adjoint module|)}

%%-------------------------------------------------------------------
\subsection{Using the band-block-diagonal preconditioner CVBBDPRE}

\index{CVBBDPRE@{\cvbbdpre} preconditioner!usage with adjoint module|(}
The adjoint module in {\cvodes} offers an interface to the band-block-diagonal
preconditioner module {\cvbbdpre} described in section \S\ref{sss:cvbbdpre}.
This generates a preconditioner that is a block-diagonal matrix with each
block being a band matrix and can be used with the Krylov linear solver in {\cvspgmr}
and with the parallel vector module {\nvecp}.

In order to use the {\cvbbdpre} module in the solution of the backward problem,
the user must define one or two additional functions, described at the end of this section. 

\subsubsection{Usage of {\cvbbdpre} for the backward problem}
%%
The {\cvbbdpre} module is initialized by calling
%%
%%
\ucfunction{CVBBDPrecAllocB}
{
  \begin{tabular}[t]{@{}r@{}l@{}}
    flag = CVBBDPrecAllocB(&cvadj\_mem, NlocalB, mudqB, mldqB, \\
                           &mukeepB, mlkeepB, dqrelyB, glocB, cfnB);
  \end{tabular}
}
{
  The function \ID{CVBBDPrecAllocB} initializes and allocates
  memory for the {\cvbbdpre} preconditioner for the backward problem.
}
{
  \begin{args}[cvadj\_mem]
  \item[cvadj\_mem] (\id{void *})
    pointer to the adjoint memory block returned by \id{CVadjMalloc}.
  \item[NlocalB] (\id{long int})
    local vector dimension for the backward problem.
  \item[mudqB] (\id{long int})
    upper half-bandwidth to be used in the difference-quotient Jacobian approximation.
  \item[mldqB] (\id{long int})
    lower half-bandwidth to be used in the difference-quotient Jacobian approximation.
  \item[mukeepB] (\id{long int})
    upper half-bandwidth of the retained banded approximate Jacobian block.
  \item[mlkeepB] (\id{long int})
    lower half-bandwidth of the retained banded approximate Jacobian block.
  \item[dqrelyB] (\id{realtype})
    the relative increment in components of \id{yB} used in the difference quotient
    approximations.  The default is \id{dqrelyB}$ = \sqrt{\text{unit roundoff}}$, which
    can be specified by passing \id{dqrely}$ = 0.0$.
  \item[glocB] (\id{CVLocalFnB})
    the {\C} function which computes the approximation $g_B(t,y)$ to the right-hand
    side of the backward problem. 
  \item[cfnB] (\id{CVCommFnB})
    the optional {\C} function which performs all interprocess communication required for
    the computation of $g_B(t,y)$.
  \end{args}
}
{
  If successful, \id{CVBBDPrecAlloc} stores a pointer to the newly created 
  {\cvbbdpre} memory block.
  The return value \id{flag} (of type \id{int}) is one of:
  \begin{args}[CV\_ADJMEM\_NULL]
  \item[\Id{CV\_SUCCESS}]
    The call to \id{CVBBDPrecAllocB} was successful.
  \item[\Id{CV\_PDATA\_NULL}]
    An error occured while trying to create the {\cvbbdpre} memory block.
  \item[\Id{CV\_ADJMEM\_NULL}]
    The \id{cvadj\_mem} argument was \id{NULL}.
  \end{args}
}
{}
%%
%%
To specify the use of the {\cvspgmr} linear solver module with the {\cvbbdpre} 
preconditioner module, make the following call:
%%
\ucfunction{CVBBDSpgmrB}
{
  flag = CVBBDSpgmrB(cvadj\_mem, pretypeB, maxlB);
}
{
  The function \ID{CVBBDSpgmrB} links the {\cvbbdpre} data to the
  {\cvspgmr} linear solver and attaches the latter to the {\cvodes}
  memory block for the backward problem.
}
{
  \begin{args}[cvadj\_mem]
  \item[cvadj\_mem] (\id{void *})
    pointer to the adjoint memory block returned by \id{CVadjMalloc}.
  \item[pretypeB] (\id{int})
    \index{pretypeB@\texttt{pretypeB}}
    preconditioning type. Can be one of \Id{PREC\_LEFT} or \Id{PREC\_RIGHT}.
  \item[maxlB] (\id{int})
    maximum dimension of the Krylov subspace to be used. Pass $0$ to use the 
    default value \id{CVSPGMR\_MAXL}$=5$.
  \end{args}
}
{
  The return value \id{flag} (of type \id{int}) is one of:
  \begin{args}[CVSPGMR\_ILL\_INPUT]
  \item[\Id{CVSPGMR\_SUCCESS}] 
    The {\cvspgmr} initialization was successful.
  \item[\Id{CVSPGMR\_MEM\_NULL}]
    The {\cvodes} memory block for the backward problem was not initialized through a 
    previous call to \id{CVodeCreateB}.
  \item[\Id{CVSPGMR\_ILL\_INPUT}]
    The preconditioner type \id{pretypeB} is not valid.
  \item[\Id{CVSPGMR\_MEM\_FAIL}]
    A memory allocation request failed.
  \item[\Id{CV\_PDATA\_NULL}]
    The {\cvbbdpre} preconditioner has not been initialized.
  \item[\Id{CV\_ADJMEM\_NULL}]
    The \id{cvadj\_mem} argument was \id{NULL}.
  \end{args}
}
{}
%%
%%
To reinitialize the {\cvbbdpre} preconditioner module for the backward problem
call the following function:
%%
\ucfunction{CVBBDPrecReInitB}
{
  flag = CVBBDPrecReInitB(cvadj\_mem, mudqB, mldqB, dqrelyB, glocB, cfnB);
}
{
  The function \ID{CVBBDPrecReInitB} reinitializes the {\cvbbdpre} preconditioner
  for the backward problem.
}
{
  \begin{args}[cvadj\_mem]
  \item[cvadj\_mem] (\id{void *})
    pointer to the adjoint memory block returned by \id{CVadjMalloc}.
  \item[mudqB] (\id{long int})
    upper half-bandwidth to be used in the difference-quotient Jacobian approximation.
  \item[mldqB] (\id{long int})
    lower half-bandwidth to be used in the difference-quotient Jacobian approximation.
  \item[dqrelyB] (\id{realtype})
    the relative increment in components of \id{yB} used in the difference quotient
    approximations.
  \item[glocB] (\id{CVLocalFnB})
    the {\C} function which computes the approximation $g_B(t,y)$  to the right-hand
    side of the backward problem. 
  \item[cfnB] (\id{CVCommFnB})
    the optional {\C} function which performs all interprocess communication required for
    the computation of $g_B(t,y)$.
  \end{args}
}
{
  The return value \id{flag} (of type \id{int}) is one of:
  \begin{args}[CV\_ADJMEM\_NULL]
  \item[\Id{CV\_SUCCESS}] 
    The {\cvspgmr} initialization was successful.
  \item[\Id{CV\_ADJMEM\_NULL}]
    The \id{cvadj\_mem} argument was \id{NULL}.
  \end{args}
}
{}
%%
%%
Note that memory allocated by \id{CVBBDPrecAllocB} is deallocated 
by the function \id{CVadjFree}.
For more details on {\cvbbdpre} see \S\ref{sss:cvbbdpre}.

\subsubsection{User-supplied functions for {\cvbbdpre}}
To use the {\cvbbdpre} module, the user must supply one or two functions which the
module calls to construct the preconditioner: a required function \id{glocB} 
(of type \id{CVLocalFnB}) which approximates the right-hand side of the backward
problem and which is computed locally, and an optional function \id{cfnB} 
(of type \id{CVCommFnB}) which performs all interprocess communication necessary 
to evaluate this approximate right-hand side (see \S\ref{sss:cvbbdpre}).
The prototypes for these two functions are described below.
%%
%%
\usfunction{CVLocalFnB}
{
  typedef void (*CVLocalFnB)(&long int NlocalB, realtype t,  \\
                             &N\_Vector y, N\_Vector yB, N\_Vector gB, \\
                             &void *f\_dataB);
}
{
  This function loads the vector
  \id{gB} as a function of \id{t}, \id{y}, and \id{yB}.  
}
{
  \begin{args}[NlocalB]
  \item[NlocalB] 
    is the local vector length for the backward problem.
  \item[t]
    is the value of the independent variable.
  \item[y]
    is the current value of the forward solution vector.
  \item[yB]
    is the current value of the dependent variable vector.
  \item[gB]
    is the output vector.
  \item[f\_dataB]
    is a pointer to user data - the same as the \Id{f\_dataB}      
    parameter passed to \id{CVodeSetFdataB}.  
  \end{args}
}
{
  A \id{CVLocalFnB} function type does not have a return value.
}
{
  This routine assumes that all interprocess communication of data needed to 
  calculate \id{gB} has already been done, and this data is accessible within
  \id{f\_dataB}.
}
%%
%%
\usfunction{CVCommFnB}
{
  typedef void (*CVCommFnB)(&long int NlocalB, realtype t,  \\
                            &N\_Vector y, N\_Vector yB, void *f\_dataB);
}
{
  This function performs all interprocess communications necessary 
  for the execution of the \id{glocB} function above, using the input 
  vectors \id{y} and \id{yB}.
}
{
  \begin{args}[NlocalB]
  \item[NlocalB] 
    is the local vector length.
  \item[t]
    is the value of the independent variable.
  \item[y]
    is the current value of the forward solution vector.
  \item[yB]
    is the current value of the dependent variable vector.
  \item[f\_dataB]
    is a pointer to user data - the same as the \Id{f\_dataB}
    parameter passed to \id{CVodeSetFdataB}.
  \end{args}
}
{
  A \id{CVCommFnB} function type does not have a return value.
}
{
  The \id{cfnB} function is expected to save communicated data in space defined within the
  structure \id{f\_dataB}. 

  Each call to the \id{cfnB} function is preceded by a call to the function that 
  evaluates the right-hand side of the backward problem with the same \id{t}, \id{y}, 
  and \id{yB} arguments. If there is no additional communication needed, then pass
  \id{cfnB} $=$ \id{NULL} to \id{CVBBDPrecAllocB}.
}
%%
%%
\index{CVBBDPRE@{\cvbbdpre} preconditioner!usage with adjoint module|)}
