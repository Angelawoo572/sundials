%===================================================================================
\section{Using {\cvodes} for Adjoint Sensitivity Analysis}\label{s:adjoint}
%===================================================================================

This section describes the use of {\cvodes} to compute sensitivities of derived 
functions using adjoint sensitivity analysis. As mentioned before, the adjoint
sensitivity module of {\cvodes} provides the infrastructure for integrating
backwards in time any system of ODEs that depends on the solution of the original
IVP, by providing various interfaces to the main {\cvodes} integrator, as well 
as several supporting user-callable routines. For this reason, in the following
sections we refer to the {\em backward problem} and not to the 
{\em adjoint problem} when discussing details relevant to the ODEs that
are integrated backwards in  time. The backward problem can be the adjoint problem
(\ref{e:adj_eqns}) or (\ref{e:adj1_eqns}),
maybe augmented with some quadrature differential equations.

We begin with a brief overview, in the form of a skeleton user program.
Following that are detailed descriptions of the interface to the
various user-callable routines and of the user-supplied routines that were not already
described in \S\ref{s:simulation}.
%-----------------------------
\subsection{A Skeleton of the User's Main Program}\label{ss:skeleton_adj}
%-----------------------------

The following is a skeleton of the user's main program as an application of
{\cvodes}. The user program is to have these steps in the order indicated, 
unless otherwise noted. For the sake of brevity, we defer many of the details to 
the later sections.
As in \S\ref{ss:skeleton_sim}, most steps are independent of the {\nvector}
implementation used; where this is not the case, usage specifications are given for the
two implementations provided with {\cvodes}: steps marked with {\p} correspond to 
{\nvecp}, while steps marked with {\s} correspond to {\nvecs}.

\begin{enumerate}

\item
  Include necessary header files. The header file that is always required
  is \Id{cvodea.h} which defines additional types and constants, and includes
  function prototypes for the adjoint sensitivity module user-callable routines.
  The header file \id{cvodes.h} need not be included by the user, as it is included 
  by \id{cvodea.h}. In addition, the main program should include an {\nvector} 
  implementation header file (\id{nvector\_serial.h} or \id{nvector\_parallel.h}
  for the two implementations provided with {\cvodes}) and, if Newton iteration 
  was selected, a header file from the desired linear solver module.

\item
  {\p} If MPI is needed by the user code, call \id{MPI\_Init(\&argc, \&argv);}

\item
  Set problem dimensions for the forward problem:

  {\s} set \id{N}; {\p} set \id{N} and \id{Nlocal};

\item
  Initialize the machine environment block for the forward problem by calling 
  the appropriate implementation-dependent {\nvector} routine:

  {\s} \id{M\_EnvInit\_Serial}; {\p} \id{M\_EnvInit\_Parallel};

\item
  Set the vector \id{y0} of initial values for the forward problem; 

\item
  Call \id{cvode\_mem = }\Id{CVodeMalloc}\id{()} to allocate internal memory for 
  {\cvodes} and initializes {\cvodes} for the forward problem;

\item
  If Newton iteration is chosen, initialize the linear solver module 
  for the forward problem, by calling the appropriate initialization routine;

\item
  Call \id{cvadj\_mem = }\Id{CVadjMalloc}\id{()} to allocate memory for the 
  combined forward-backward problem (see \S\ref{sss:cvadjmalloc} for more details). 
  This call requires \id{Nd}, the number of steps between two consecutive check points;

\item
  Call \Id{CVodeF}, a wrapper around the {\cvodes} main integration
  routine \id{CVode}, either in \Id{NORMAL} mode to the time 
  \id{tout} or in \Id{ONE\_STEP} mode inside a loop (if intermediate
  solutions of the forward problem are desired (see \S\ref{sss:cvodef}).
  The final value of of \id{t}, denoted \id{tfinal}, is then
  the maximmum allowable value for the endpoint $t_1$;

\item \label{i:back_start}
  Set problem dimensions for the backward problem:

  {\s} set \id{NB}, the number of variables in the backward problem;
  {\p} set \id{NB} and \id{NBlocal};

\item
  Initialize the machine environment block for the backward problem by calling 
  the appropriate implementation-dependent {\nvector} routine
  (typically the same as for the forward problem):
  
  {\s} \id{M\_EnvInit\_Serial}; {\p} \id{M\_EnvInit\_Parallel};

\item
  Set the vector \id{yB0} of final values for the backward problem;

\item
  Call \Id{CVodeMallocB}, a wrapper around the \id{CVodeMalloc}, to 
  allocate internal memory and initialize {\cvodes} at \id{tB0} for 
  the backward problem (see \S\ref{sss:cvodemallocb});

\item \label{i:lin_solverB}
  If Newton iteration is chosen, initialize the linear solver module 
  for the backward problem, by calling the appropriate wrapper around 
  the initialization routines \id{CVDenseB}, \id{CVBandB}, \id{CVDiagB},
  or \id{CVSpgmrB} (see \S\ref{sss:lin_solv_b}). Note that it is not required
  to use the same linear solver module for both the forward and the 
  backward problems; for example, the forward problem could be solved
  with the {\cvdense} linear solver and the backward problem with {\cvspgmr};

\item \label{i:back_end}
  Call \Id{CVodeB}, a second wrapper around the {\cvodes} main integration
  routine \id{CVode}, to integrate the backward problem from \id{tB0}, 
  returning the solution of the backward problem at \id{t0} into
  \id{yB} (see \S\ref{sss:cvodeb});

\item
  Upon completion of the backward integration, call all necessary deallocation
  routines. These include calls to \id{NV\_Free} for the vectors \id{y} and \id{yB},
  calls to the appropriate implementation-dependent {\nvector} free routine for
  the forward and backward machine environment blocks, and calls to \id{CVodeFree}
  to free the {\cvodes} memory block for the forward problem and to
  \id{CVadjFree} (see \S\ref{sss:cvadjfree}) to free the memory allocated for the 
  combined problem. Note that \id{CVadjFree} also deallocates the {\cvodes} memory 
  for the backward problem;

\item
  {\p} If MPI was initialized by the user main program, call \id{MPI\_Finalize();}.

\end{enumerate}

The above user interface to the adjoint sensitivity module in {\cvodes} was motivated by
the desire to keep it as close as possible in look and feel to the one for ODE IVP 
integration. Note that if steps (\ref{i:back_start})-(\ref{i:back_end}) are not present, 
a program with the above structure will have the same functionality as one described in
\S\ref{ss:skeleton_sim} for integration of ODEs, albeit with some overhead due to 
the check pointing scheme.

%-------------------------------------------------------------------------------
\subsection{User-Callable Routines for Adjoint Sensitivity Analysis}
%-------------------------------------------------------------------------------

\subsubsection{Adjoint Sensitivity Allocation Routine}\label{sss:cvadjmalloc}

The routine \ID{CVadjMalloc} allocates internal memory for the combined
forward-backward integration, other than the {\cvodes} memory block. Space is allocated 
for the $N_d$ interpolation data points and a linked list of check points is 
initialized.
The form of the call to this routine is
\begin{verbatim}
cadj_mem = CVadjMalloc(cvode_mem, Nd);
\end{verbatim}
where \id{cvode\_mem} is the {\cvodes} memory block for the forward problem
returned by a previous call to \id{CVodeMalloc} and \id{Nd} is the number of
integration steps between two consecutive check points.
If successful, \id{CVadjMalloc} returns a pointer of type \id{void *}. The user
doesn't need to access this memory block but must pass it to other adjoint module
user-callable routines. In case of failure (\id{cvode\_mem} is \id{NULL}, \id{Nd} is
non-positive, or a memory request fails), \id{CVadjMalloc} prints an error
message to the standard output stream \id{stdout} and returns \id{NULL}.

The user must set \id{Nd} so that all data needed for interpolation of the 
forward problem solution between two check points fits in memory. \id{CVadjMalloc}
attempts to allocate space for $(2$\id{Nd}$+3)$ vectors of length \id{N}, the
dimension of the forward problem.

\subsubsection{Forward Integration Routine}\label{sss:cvodef}

The routine \ID{CVodeF} is very similar to the {\cvodes} routine \id{CVode}
(see \S\ref{sss:cvode}) in that it integrates the solution of the forward
problem and returns the solution in \id{y}. At the same time,
however, \id{CVodeF} stores check point data every \id{Nd} integration steps.
\id{CVodeF} can be called repeatedly by the user. The last value of \id{tout} 
(or of \id{t}), \id{tlast}, will be used as the starting time for the backward 
integration.

The call to this routine has the form
\begin{verbatim}
ier = CVodeF(cadj_mem, tout, y, &t, itask, &ncheckPtr);
\end{verbatim}
Most of its arguments have names and meanings identical to those of \id{CVode}.
In addition, \id{CVodeF} takes as first argument the memory pointer \id{cvad\_mem}
returned by \id{CVadjMalloc} and returns in its last argument the final number of 
check points. If an error occured during the memory allocation of a new check
point, \id{CVodeF} returns \Id{CVODEF\_MEM\_FAIL} and prints an error message to 
\id{stdout}. Otherwise, since \id{CVodeF} wraps around \id{CVode}, its return 
value \id{ier} is the return value of \id{CVode} (see \S\ref{sss:cvode}): 
\Id{SUCCESS}\id{=0} or \Id{TSTOP\_RETURN}\id{=1} for a successful return, 
or a negative value in case of failure.

Note that, at this time, \id{CVodeF} stores check point information into memory
only. Future versions will provide for a safe-guard option of dumping check point
data to a temporary file as needed. The data stored at each check point is basically
a snapshot of the {\cvodes} internal memory block and contains enough information
to restart the integration from that time and proceed with the same stepsize and
method order sequence as during the forward integration.

In addition, \id{CVodeF} also stores interpolation data between consecutive check points
so that, at the end of this first forward integration phase, information for the Hermite 
interpolation is already available from the last check point to \id{tfial}. In particular,
if no check points were necessary there is no need for the second forward integration phase.

\subsubsection{Backward Problem Initialization Routine}\label{sss:cvodemallocb}

The routine \ID{CVodeMallocB} initializes and allocates memory for the backward
problem. It has the following form
\begin{verbatim}
ier = CVodeMallocB(cvadj_mem, NB, fB, tB0, yB0, lmmB, iterB, itolB, &rtolB, 
                   atolB, f_dataB, errfpB, optInB, ioptB, roptB, machEnvB);
\end{verbatim}
and is essentially a call to \id{CVodeMalloc} with some particularization for 
backward integration. First, \id{CVodeMallocB} takes as an argument 
\id{cvadj\_mem}, the memory pointer returned by \id{CVadjMalloc}.
Secondly, the argument \id{fB}, the {\C} function to compute the right hand side 
of the backward problem, must be of type \id{RhsFnB} and
has the form \id{fB(NB, t, y, yB, yBdot, f\_dataB)} 
(see \S\ref{ss:user_fct_adj} for details). 
All other arguments are equivalent to those of \id{CVodeMalloc}. The argument
\id{tB0} specifies the end point $t_1$ where final conditions are provided for
the backward problem. If \id{tB0} is not in the interval $[$\id{t0},\id{tfinal}$]$
an error message is printed and \id{CVodeMallocB} returns \id{CVBM\_BAD\_TB0}.
The relative and absolute tolerances \id{rtolB} and \id{atolB} for \id{yB} need
to be set appropriately; this may require some experimentation. Data that could
help in the selection of appropriate tolerances includes the time scale of the
forward problem (including the actual initial stepsize used, available in the
optional I/O real array at \id{ropt[HOU]}), the \id{yB} values at $t_0$ after
the backward integration phase, and (if an adjoint problem is integrated 
backwards in time) the value of the gradient $g_y$ at time $t_1$.

The return value of \id{CVodeMallocB} is \Id{SUCCESS} if there were no errors,
\Id{CVBM\_NO\_MEM} if \id{cvadj\_mem} was \id{NULL}, \Id{CVBM\_MEM\_FAIL}
if a memory request failed, \Id{CVBM\_BAD\_TB0} if \id{tB0} had an illegal value,
or any other error return value from \id{CVodeMalloc} (see \S\ref{sss:cvodemalloc}).

\subsubsection{Linear Solver Initialization Routines for Backward Problem}\label{sss:lin_solv_b}

\index{CVODES@{\cvodes} linear solvers!usage with adjoint module|(} 
The adjoint sensitivity module in {\cvodes} provides interfaces to the initialization
routines of all supported linear solver modules for the case in which Newton
iteration is selected for the solution of the backward problem. The initialization
routines described in \S\ref{sss:lin_solv_init} cannot be directly used since the
optional user-defined Jacobian-related routines have different prototypes for the
backward problem than for the forward problem. The initialization routine
\id{CVDiag} for the diagonal linear solver can be used directly, since {\cvdiag}
does not provide for a user-defined diagonal Jacobian approximation routine. 
\index{CVODES@{\cvodes} linear solvers!usage with adjoint module|)} 

\begin{itemize}

\item {\em Dense linear solver initialization}
  \index{CVDENSE@{\cvdense} linear solver!usage with adjoint module|(}

  In order to use the {\cvdense} solver for the backward problem, after the call to 
  \id{CVodeMallocB}, the calling program must make the call
\begin{verbatim}
ier = CVDenseB(cvadj_mem, djacB, jac_dataB);
\end{verbatim}
  The argument \id{cvadj\_mem} is a pointer to the memory block
  returned by \id{CVadjMalloc}. The {\cvdense} solver routine for computing a
  dense approximation to the Jacobian matrix of the backward problem must be
  of type \id{CVDenseJacFnB}, and is communicated through the argument
  \id{djacB}. The user can supply his/her own dense Jacobian routine
  (see \S\ref{ss:user_fct_adj}), or use the difference quotient routine 
  \Id{CVDenseDQJac} that comes with the {\cvdense} solver.  To use 
  \id{CVDenseDQJac}, the user must pass \id{NULL} for the \id{djacB} parameter.
  As with \id{CVDense}, \id{jac\_dataB} is a pointer to a user-defined 
  data structure that the {\cvdense} solver passes to its Jacobian function
  and which can be used to store data relevant to the Jacobian computation.
  
  The possible return values \id{ier} are identical to those of \id{CVDense}
  (see \S\ref{sss:lin_solv_init}).
  \index{CVDENSE@{\cvdense} linear solver!usage with adjoint module|)}

\item {\em Banded linear solver initialization}
  \index{CVBAND@{\cvband} linear solver!usage with adjoint module|(}

  In order to use the {\cvband} solver for the backward problem, after the call to 
  \id{CVodeMallocB}, the calling program must make the call
\begin{verbatim}
ier = CVBandB(cvadj_mem, mupperB, mlowerB, bjacB, jac_dataB);
\end{verbatim}  
  The \ID{CVBandB} argument \id{cvadj\_mem} is a pointer to the memory block
  returned by \id{CVadjMalloc}. The upper and lower bandwidths of the 
  Jacobian of the backward problem (or its approximation) are specified 
  in this call through the \id{mupperB} and \id{mlowerB} parameters.
  The {\cvband} solver routine for computing a
  banded approximation to the Jacobian matrix of the backward problem must be
  of type \id{CVBandJacFnB}, and is communicated through the argument
  \id{bjacB}. The user can supply his/her own banded Jacobian routine
  (see \S\ref{ss:user_fct_adj}), or use the difference quotient routine 
  \Id{CVBandDQJac} that comes with the {\cvband} solver.  To use 
  \id{CVBandDQJac}, the user must pass \id{NULL} for the \id{bjacB} parameter.
  As before, \id{jac\_dataB} is a pointer to a user-defined 
  data structure that the {\cvband} solver passes to its Jacobian function
  and which can be used to store data relevant to the Jacobian computation.

  The possible return values \id{ier} are identical to those of \id{CVBand}
  (see \S\ref{sss:lin_solv_init}).
  \index{CVBAND@{\cvband} linear solver!usage with adjoint module|)}

\item {\em {\spgmr} linear solver initialization}
  \index{CVSPGMR@{\cvspgmr} linear solver!usage with adjoint module|(} 
  
  The {\cvspgmr} solver can be linked to {\cvodes} for use during the
  backward problem solution by calling the routine \ID{CVSpgmrB}. The
  call to this routine has the following form:
\begin{verbatim}
ier = CVSpgmrB(cvadj_mem, pretypeB, gstypeB, maxlB, deltB, PrecondB, 
               PSolveB, P_dataB, jtimesB, jac_dataB);
\end{verbatim}
  The first argument, \id{cvadj\_mem}, is a pointer to the memory block
  returned by \id{CVadjMalloc}. The rest of the arguments are equivalent 
  to those of \id{CVSpgmr} (see \S\ref{sss:lin_solv_init}) with the 
  following exceptions:
  \begin{itemize}
  \item \id{PrecondB}, the optional user-defined preconditioner setup
    routine, must now have function type \id{CVSpgmrPrecondFnB};
  \item \id{PSolveB}, the user-defined preconditioner solve routine,
    must now have function type \id{CVSpgmrPSolveFnB};
  \item \id{jtimesB}, the optional user-defined Jacobian times vector 
    routine, must now be of type \id{CVSpgmrJtimesFnB}. The user can 
    supply his/her own Jacobian times vector approximation routine, 
    or use the difference quotient routine \Id{CVSpgmrDQJtimes}.
    To use the \id{CVSpgmrDQJtimes}, the user must pass \id{NULL} 
    for \id{jtimesB}.
  \end{itemize}
  For details on the prototypes of these user-defined functions, 
  see \S\ref{ss:user_fct_adj}.
  The possible return values \id{ier} are identical to those of \id{CVSpgmr}
  (see \S\ref{sss:lin_solv_init}).
  \index{CVSPGMR@{\cvspgmr} linear solver!usage with adjoint module|)} 


\end{itemize}

\subsubsection{Backward Integration Routine}\label{sss:cvodeb}

The routine \ID{CVodeB} performs the integration of the backward problem
from the final to the initial time. It is essentially a wrapper
around the {\cvodes} main integration routine \id{CVode} and, in the case
in which check points were needed, it evolves the solution of the backward 
problem through a sequence of forward-backward integrations between consecutive
check points. The first run integrates the original IVP forward in time and
stores interpolation data; the second run integrates the backward problem 
backward in time and performs Hermite interpolation to provide
the solution of the IVP to the backward problem.

The call to this routine has the form
\begin{verbatim}
ier = CVodeB(cvadj_mem, yB);
\end{verbatim}
and loads the solution of the backward problem at the initial time into
\id{yB}. 

If \id{CVodeB} successfully integrates the backward problem, the return value
is \Id{TSTOP\_RETURN}=1, as the \id{TSTOP} option is automatically turned on in this
phase. If an error occurred, \id{CVodeB} prints an error message indicating whether 
the error occurred during the forward or the backward integration phase and returns 
\id{ier} from \id{CVode} (see \S\ref{sss:cvode}).

\subsubsection{Adjoint Sensitivity Deallocation Routine}\label{sss:cvadjfree}

To free the \id{cvadj\_mem} memory block allocated by \id{CVadjMalloc}, the
user must call \ID{CVadjFree}. The call to this routine has the form
\begin{verbatim}
CVadjFree(cvadj_mem);
\end{verbatim}
and it frees check-point-related memory, as well as the {\cvodes} memory block
allocated for the integration of the backward problem. There is no return 
value for \id{CVadjFree}.

\subsubsection{Backward Problem Reinitialization Routine}\label{sss:cvreinitb}
\index{reinitialization}

For the case when it is needed to solve several different backward problems 
corresponding to the same original problem, {\cvodes} provides a mechanism 
to reuse the existing check points. The function \ID{CVReInitB} reinitializes
the {\cvodes} memory block for the backward problem, where a prior call to
\Id{CVodeMallocB} has been made with the same problem size \id{NB}.
\id{CVReInitB} performs the same input checking and initializations that 
\id{CVodeMallocB} does (except for \id{NB}), but does no memory allocation, 
assuming that the existing internal memory is sufficient for the new problem. 
Note that \id{CVReInitB} is essentially a wrapper around \id{CVReInit}
and hence all details given for \id{CVReInit} in \S\ref{sss:cvreinit}
apply.

The call to the \id{CVReInitB} function has the form
\begin{verbatim}
ier = CVReInitB(cvadj_mem, fB, tB0, yB0, lmmB, iterB, itolB, &rtolB, 
                atolB, f_dataB, errfpB, optInB, ioptB, roptB, machEnvB);
\end{verbatim}
All its arguments  have names and meanings identical to those of \id{CVodeMallocB}.
Note that the backward problem size \id{NB} is not passed as an argument to 
\id{CVReInitB}, as that is assumed to be unchanged since the \id{CVodeMallocB} call.

The return value \id{ier} of \id{CVReInitB} is  
\id{SUCCESS}=0 if there were no errors,
\id{CVBM\_NO\_MEM} if \id{cvadj\_mem} was NULL,
\id{CVBM\_BAD\_TB0} if \id{tB0} has an illegal value,
or any other error return value from \id{CVReInit} (see \S\ref{sss:cvreinit}).
In case of an error return, an error message is also printed.  

\subsubsection{Check Point Listing Routine}

For debugging purposes, {\cvodes} provides a routine \ID{CVadjCheckPointsList} which
displays partial information from the linked list of check points generated by
\id{CVodeF}. The call to this routine has the form:
\begin{verbatim}
CVadjCheckPointsList(cvadj_mem);
\end{verbatim}
For a typical output of \id{CVadjCheckPointsList}, see the example section
\S\ref{ss:serial_adj_ex}.

%-------------------------------------------------------------------
\subsection{User-Supplied Routines for Adjoint Sensitivity Analysis}\label{ss:user_fct_adj}
%-------------------------------------------------------------------

In addition to the required ODE right hand side routine and any optional routines
for the forward problem, when using the adjoint sensitivity module in {\cvodes},
the user must supply one function defining the backward problem ODE and, optionally,
routines to supply Jacobian related information (if Newton iteration is chosen) and
one or two functions that define the preconditioner (if the {\cvspgmr} solver is
selected) for the backward problem.

Type definitions for all these user-supplied routines are given below.

\begin{itemize}

\item{\em ODE right hand side for the backward problem}
  \index{adjoint sensitivity analysis!right hand side evaluation|(}
  \index{right hand side function!adjoint backward problem|(}

  The user must provide a function of type \ID{RhsFnB} defined by
\begin{verbatim}
typedef void (*RhsFnB)(integertype NB, realtype t, N_Vector y, 
                       N_Vector yB, N_Vector yBdot, void *f_dataB);
\end{verbatim}
  to compute the right hand side of the backward problem ODE system.
  This could be (\ref{e:adj_eqns}) or (\ref{e:adj1_eqns}), 
  possibly with one or more quadrature ODEs appended.

  This function takes as input the size \id{NB} of the backward problem, the 
  independent variable value \id{t}, and the dependent variable vector \id{yB},
  as well as the solution of the original IVP \id{y} at time \id{t}.  
  It must store the backward problem right hand side in the vector \id{yBdot}.  
  Allocation of memory for \id{yBdot} is handled within {\cvodes}.
  
  The \id{y}, \id{yB}, and \id{yBdot} arguments are all of type \id{N\_Vector},
  but \id{yB} and \id{yBdot}  typically have different internal representations
  from \id{y}. It is the user's 
  responsibility to access the vector data consistently (including the use of the 
  correct accessor macros from each {\nvector} implementation). For the sake of 
  computational efficiency, the vector kernels in the two {\nvector} implementations 
  provided with {\cvodes} do not perform any consistency checking for their 
  \id{N\_Vector} arguments (see \S\ref{ss:nvec_ser} and \S\ref{ss:nvec_par}).

  The \id{f\_dataB} parameter is the same as the \id{f\_dataB} parameter passed by 
  the user to the \id{CVodeMallocB} routine. This user-supplied pointer is passed to 
  the user's \id{fB} function every time it is called and can be the same as the 
  \id{f\_data} pointer used for the forward problem.
  A \id{RhsFnB} function type does not have a return value.
  \index{right hand side function!adjoint backward problem|)}
  \index{adjoint sensitivity analysis!right hand side evaluation|)}

\item {\em Jacobian information for the backward problem
    (direct method with dense Jacobian)}
  
  If the direct linear solver with dense treatment of the Jacobian is selected
  for the backward problem (i.e. \Id{CVDenseB} is called in step \ref{i:lin_solverB} 
  of \S\ref{ss:skeleton_adj}), the user may provide a function of type \ID{CVDenseJacFnB} 
  defined by
\begin{verbatim}
typedef void (*CVDenseJacFnB)(integertype NB, DenseMat JB, RhsFnB fB, 
                           void *f_dataB, realtype t, N_Vector y, 
                           N_Vector yB, N_Vector fyB, N_Vector ewtB,
                           realtype hB, realtype uroundB, void *jac_dataB,
                           long int *nfePtrB, N_Vector vtemp1B,
                           N_Vector vtemp2B, N_Vector vtemp3B);
\end{verbatim}
  to compute the dense Jacobian of the backward problem (or an approximation
  to it). If the backward problem is the adjoint of the original IVP then
  this Jacobian is nothing else than the transpose of $J = \dfdyI$ with
  a change in sign.

  A user-supplied dense Jacobian routine must load the \id{NB} by \id{NB}
  dense matrix \id{JB} with an approximation to the Jacobian matrix
  at the point (\id{t},\id{y},\id{yB}), where \id{y} is the solution
  of the original IVP at time \id{t} and \id{yB} is the solution of the
  backward problem at the same time.
  Only nonzero elements need to be loaded into \id{JB} as this matrix 
  is set to zero before the call to the Jacobian routine. 
  The type of \id{JB} is \Id{DenseMat}. The user is referred to 
  \S\ref{ss:user_fct_sim} (page \pageref{p:djac}) for details regarding 
  accessing a \id{DenseMat} object as well as details on the rest of the 
  arguments of a function of type \id{CVDenseJacFnB}.

  Typically, a user-supplied Jacobian function \id{djacB} would be
  expected to access the arguments \id{NB}, \id{t}, \id{y}, \id{yB}, \id{JB}, 
  \id{f\_dataB}, and \id{jac\_dataB}, at most. The remaining arguments 
  would not typically be accessed, but appear in the call list because
  they are needed by the function \id{CVDenseDQJac} that computes a
  difference quotient dense Jacobian approximation, when the user specifies that
  option.

\item {\em Jacobian information for the backward problem
    (direct method with banded Jacobian)}

  If the direct linear solver with banded treatment of the Jacobian is selected
  for the backward problem (i.e. \Id{CVBandB} is called in step \ref{i:lin_solverB} 
  of \S\ref{ss:skeleton_adj}), the user may provide a function of type \ID{CVBandJacFnB} 
  defined by
\begin{verbatim}
typedef void (*CVBandJacFnB)(integertype NB, integertype mupperB, 
                             integertype mlowerB, BandMat JB, RhsFnB fB, 
                             void *f_dataB, realtype t, N_Vector y, 
                             N_Vector yB, N_Vector fyB, N_Vector ewtB, 
                             realtype hB, realtype uroundB, void *jac_dataB, 
                             long int *nfePtrB, N_Vector vtemp1B, 
                             N_Vector vtemp2B, N_Vector vtemp3B);
\end{verbatim}
  to compute the banded Jacobian of the backward problem (or an approximation
  to it).

  A user-supplied band Jacobian routine must load the band matrix \id{JB}
  of type \Id{BandMat} with the elements of the Jacobian at the
  point (\id{t},\id{y},\id{yB}), where \id{y} is the solution
  of the original IVP at time \id{t} and \id{yB} is the solution of the
  backward problem at the same time.  
  Only nonzero elements need to be loaded into
  \id{JB} because \id{JB} is preset to zero before the call to the
  Jacobian routine. More details on the accessor macros provided for
  a \id{BandMat} object and on the rest of the arguments of a function
  of type \id{CVBandJacFnB} are given in \S\ref{ss:user_fct_sim} on page
  \pageref{p:bjac}.

  Typically, a user-supplied Jacobian function \id{bjacB} would be
  expected to access the arguments \id{NB}, \id{mupperb}, \id{mlowerB},
  \id{t}, \id{y}, \id{yB}, \id{JB}, \id{f\_dataB}, and \id{jac\_dataB}, at most.
  The remaining arguments would not typically be accessed, but appear in
  the call list because they are needed by the function \id{CVBandDQJac}
  that computes a difference quotient banded Jacobian approximation, when the
  user specifies that option.

\item {\em Jacobian information for the backward problem
    ({\spgmr} case)}

  If an iterative {\spgmr} linear solver is selected (\id{CVSpgmrB} is called in step 
  \ref{i:lin_solverB} of \S\ref{ss:skeleton_adj}) the user may provide a function
  of type \ID{CVSpgmrJtimesFnB} in the form 
\begin{verbatim}
typedef int (*CVSpgmrJtimesFnB)(integertype NB, N_Vector vB, N_Vector JvB, 
                                RhsFnB fB, void *f_dataB, realtype t, 
                                N_Vector y, N_Vector yB, N_Vector fyB,
                                realtype vnrmB, N_Vector ewtB, realtype hB, 
                                realtype uroundB, void *jac_dataB, 
                                long int *nfePtrB, N_Vector workB);
\end{verbatim}
  to compute the action of the Jacobian on a given vector \id{vB} for
  the backward problem (or an approximation to it).
  A user-supplied Jacobian times vector routine must load the vector \id{JvB}
  with the result of the product between the Jacobian of the backward problem 
  at the point (\id{t},\id{y}, \id{yB}) and the vector \id{vB} of dimension 
  \id{NB}. Here, \id{y} is the solution of the original IVP at time \id{t} and 
  \id{yB} is the solution of the backward problem at the same time.  
  The rest of the arguments are equivalent to those of a function of type
  \id{CVSpgmrJtimesFn} (see \S\ref{ss:user_fct_sim} on page \pageref{p:jtimes}).
  If the backward problem is the adjoint of ${\dot y} = f(t, y)$, then this 
  routine is to compute $-(\dfdyI)^T v_B$.

  The return value of a function of type \id{CVSpgmrJtimesFnB} should be
  $0$ if successful or non-zero if an error was encountered, in which case
  the integration is halted.

\item {\em Preconditioning for the backward problem
    (linear system solution)}

  If preconditioning is used during integration of the backward problem, 
  then the user must provide a {\C} function to solve the linear system 
  $Pz = r$ where $P$ may be either a left or a right preconditioner matrix.
  This function must be of type \ID{CVSpgmrPSolveFnB} defined by
\begin{verbatim}
typedef int (*CVSpgmrPSolveFnB)(integertype NB, realtype t, N_Vector y, 
                                N_Vector yB, N_Vector fyB, 
                                N_Vector vtempB,  realtype gammaB, 
                                N_Vector ewtB, realtype deltaB, 
                                long int *nfePtrB, N_Vector rB, 
                                int lrB, void *P_dataB, N_Vector zB);
\end{verbatim}
  The only difference between this function type and \id{CVSpgmrPSolveFn}
  (defined in \S\ref{ss:user_fct_sim} on page \pageref{p:psolve}) is that
  a function of type \id{CVSpgmrPSolveFnB} also receives as an argument
  \id{y}, the solution of the forward problem at time \id{t}.

  The return value of a preconditioner solve routine for the backward
  problem should be $0$ if successful, 
  positive for a recoverable error (in which case the step will be retried),     
  negative for an unrecoverable error (in which case the integration is halted).

\item {\em Preconditioning for the backward problem
    (Jacobian data)}

  If the user's preconditioner requires that any Jacobian related data
  be evaluated or preprocessed, then this needs to be done in a
  user-supplied {\C} function of type \ID{CVSpgmrPrecondFnB} 
  as defined by
\begin{verbatim}
typedef int (*CVSpgmrPrecondFnB)(integertype NB, realtype t, N_Vector y, 
                             N_Vector yB, N_Vector fyB, booleantype jokB, 
                             booleantype *jcurPtrB, realtype gammaB,
                             N_Vector ewtB, realtype hB, realtype uroundB,
                             long int *nfePtrB, void *P_dataB,
                             N_Vector vtemp1B, N_Vector vtemp2B,
                             N_Vector vtemp3B);
\end{verbatim}
  This function type is identical to the type \id{CVSpgmrPrecondFn}
  (\S\ref{ss:user_fct_sim} on page \pageref{p:precond}) with the exception
  of the additional argument \id{y} which contains the solution of the 
  forward problem at time \id{t}.

  The return value of a preconditioner setup routine for the backward
  problem should be $0$ if successful, 
  positive for a recoverable error (in which case the step will be retried),     
  negative for an unrecoverable error (in which case the integration is halted).

\end{itemize}

%-------------------------------------------------------------------
\subsection{Using the Banded Preconditioner Module for Adjoint Sensitivity Analysis}
%-------------------------------------------------------------------

\index{CVBANDPRE@{\cvbandpre} preconditioner!usage with adjoint module|(}
The adjoint sensitivity module in {\cvodes} offers an interface to the banded 
preconditioner module {\cvbandpre} described in section \S\ref{sss:cvbandpre}.
This preconditioner provides a band matrix preconditioner based on difference
quotients of the backward problem right hand side function \id{fB}. It generates
a banded approximation to the Jacobian with $m_{lB}$ sub-diagonals and $m_{uB}$
super-diagonals to be used with the Krylov linear solver in {\cvspgmr}.

In order to use the {\cvbandpre} module in the solution of the backward problem,
the user need not define any additional routines. Before the call to \Id{CVSpgmrB}
(step \ref{i:lin_solverB} in \S\ref{ss:skeleton_adj}), the user must initialize the
{\cvbandpre} module by calling
\begin{verbatim}
bp_dataB = CVBandPreAllocB(cvadj_mem, NB, muB, mlB);
\end{verbatim}
where \id{cvadj\_mem} is the pointer to the memory block allocated by
\id{CVadjMalloc}, \id{NB} is the size of the backward problem, and \id{muB}
and \id{mlB} are the upper and lower half-bandwidths, respectively. 
\id{CVBandPreAlloc} returns an object of type \id{CVBandPreData} which must
then be passed to \id{CVSpgmrB} as the \id{P\_dataB} argument:
\begin{verbatim}
ier = CVSpgmrB(cvadj_mem, pretypeB, gstypeB, maxlB, deltB,
               CVBandPrecondB, CVBandPSolveB, bp_dataB,
               jtimesB, jac_dataB);
\end{verbatim}
Here the preconditioner setup and solve routines \id{CVBandPSolveB} and 
\id{CVBandPrecondB} are provided in the interface to the {\cvbandpre} module.
They are never called by the user explicitly but are simply passed as arguments
to \id{CVSpgmrB}. 
Although these functions simply call \id{CVBandPSolve} and \id{CVBandPrecond},
without any other processing, they need to be defined because they must be
of the correct function types expected by \id{CVSpgmrB}. For example,
\id{CVBandPSolveB} must be of type \id{CVSpgmrPSolveFnB} and thus receives
\id{y}, the solution of the original IVP, as an argument 
(which is consequently ignored as it is not needed by {\cvbandpre}).
\index{CVBANDPRE@{\cvbandpre} preconditioner!usage with adjoint module|)}
