%===============================================================================
\section{Introduction}\label{s:ex_intro}
%===============================================================================

This report is intended to serve as a companion document to the User
Documentation of {\cvodes} \cite{cvodes2.1.0_ug}.  It provides details, with
listings, on the example programs supplied with the {\cvode} distribution
package.

The {\cvode} distribution contains examples of the following types: 
serial and parallel examples for IVP integration, 
serial and parallel examples for forward sensitivity analysis, and 
serial and parallel examples for adjoint sensitivity analysis.
These examples, summarized below, are shortly described next.

\newlength{\colone}
\settowidth{\colone}{em*3}
%\newlength{\coltwo}
%\setlength{\coltwo}{(\textwidth-\colone-0.5in)/2}
\begin{center}
%  \begin{tabular}{|p{\colone}|p{\coltwo}|p{\coltwo}|} \hline
  \begin{tabular}{|p{\colone}|l|l|} \hline
    & Serial examples & Parallel examples \\ \hline
    IVP & 
    \id{cvbx} \id{cvdx} \id{cvdemd} \id{cvkx} \id{cvkxb} \id{cvdemk} &
    \id{pvkx} \id{pvkxb} \id{pvfnx} \\ \hline
    FSA &  
    \id{cvfdx} \id{cvfkx} \id{cvfnx}  & 
    \id{pvfnx} \id{pvfkx}\\ \hline
    ASA & 
    \id{cvabx} \id{cvadx} \id{cvakx} \id{cvakxb} & 
    \id{pvanx} \id{pvakx} \\ \hline
  \end{tabular}
\end{center}

\vspace{0.2in}
\noindent Supplied in the \id{sundials/cvodes/examples\_ser} directory are the
following thirteen serial examples (using the {\nvecs} module):

%%
%%--------------------------------------
%%
\begin{itemize}

\item \id{cvdx}
  solves a chemical kinetics problem consisting of three rate equations.

  This program solves the problem with the BDF method and Newton          
  iteration, with the {\cvdense} linear solver and a user-supplied    
  Jacobian routine.  It also uses the rootfinding feature of {\cvode}.

\item \id{cvbx}
  solves the semi-discrete form of an advection-diffusion equation in 2-D. 

  This program solves the problem with the BDF method and Newton          
  iteration, with the {\cvband} linear solver and a user-supplied     
  Jacobian routine.

\item \id{cvkx}
  solves the semi-discrete form of a two-species diurnal kinetics
  advection-diffusion PDE system in 2-D.

  The problem is solved with the BDF/GMRES method (i.e.    
  using the {\cvspgmr} linear solver) and the block-diagonal part of the  
  Newton matrix as a left preconditioner. A copy of the block-diagonal 
  part of the Jacobian is saved and conditionally reused within the    
  preconditioner setup routine.

\item \id{cvkxb}
  solves the same problem as \id{cvkx}, with the BDF/GMRES method 
  and a banded preconditioner, generated by difference quotients, 
  using the module {\cvbandpre}.

  The problem is solved twice---with preconditioning on the left,
  then on the right.

\item \id{cvdemd}
  is a demonstration program for {\cvode} with direct linear solvers.

  Two separate problems are solved using both the Adams and BDF linear
  multistep methods in combination with functional and Newton
  iterations. 

  The first problem is the Van der Pol oscillator for which 
  the Newton iteration cases use the following types of Jacobian approximations:
  (1) dense, user-supplied, (2) dense, difference-quotient approximation, 
  (3) diagonal approximation. The second problem is a linear ODE with a
  banded lower triangular matrix derived from a 2-D advection PDE. In this
  case, the Newton iteration cases use the following types of Jacobian
  approximation: (1) banded, user-supplied, (2) banded, difference-quotient
  approximation, (3) diagonal approximation.

\item \id{cvdemk}
  is a demonstration program for {\cvode} with the Krylov linear solver.

  This program solves a stiff ODE system that arises from a system     
  of partial differential equations.  The PDE system is a six-species
  food web population model, with predator-prey interaction and diffusion 
  on the unit square in two dimensions.

  The ODE system is solved using Newton iteration and the      
  {\cvspgmr} linear solver (scaled preconditioned GMRES).

  The preconditioner matrix used is the product of two matrices:         
  (1) a matrix, only defined implicitly, based on a fixed number of     
  Gauss-Seidel iterations using the diffusion terms only; and               
  (2) a block-diagonal matrix based on the partial derivatives of the   
  interaction terms only, using block-grouping.                          

  Four different runs are made for this problem.                        
  The product preconditoner is applied on the left and on the right.    
  In each case, both the modified and classical Gram-Schmidt options    
  are tested.

\item \id{cvfdx}
  solves a chemical kinetics problem consisting of three rate equations.

  {\cvodes} computes both its solution and solution sensitivities with respect
  to the three reaction rate constants appearing in the model. 
  This program solves the problem with the BDF method, Newton          
  iteration with the {\cvdense} linear solver, and a user-supplied    
  Jacobian routine.

\item \id{cvfkx}
  solves the semi-discrete form of a two-species diurnal kinetics
  advection-diffusion PDE system in 2-D space.

  {\cvodes} computes both its solution and solution sensitivities with respect
  to two parameters affecting the kinetic rate terms.
  The problem is solved with the BDF/GMRES method (i.e.    
  using the {\cvspgmr} linear solver) and the block-diagonal part of the  
  Newton matrix as a left preconditioner.

\item \id{cvfnx}
  solves the semi-discrete form of an advection-diffusion equation in 1-D.

  {\cvodes} computes both its solution and solution sensitivities with respect
  to the advection and diffusion coefficients.
  This program solves the problem with the option for nonstiff systems,
  i.e. Adams method and functional iteration.

\item \id{cvabx}
  solves the semi-discrete form of an advection-diffusion equation in 2-D.

  The adjoint capability of {\cvodes} is used to compute gradients
  of the average (over both time and space) of the solution with respect to
  the initial conditions.
  This program solves both the forward and backward problems with the BDF method, 
  Newton iteration with the {\cvband} linear solver, and user-supplied     
  Jacobian routines.

\item \id{cvadx}
  solves a chemical kinetics problem consisting of three rate equations.
  
  The adjoint capability of {\cvodes} is used to compute gradients
  of a functional of the solution with respect to the three
  reaction rate constants appearing in the model.
  This program solves both the forward and backward problems with the BDF method, 
  Newton iteration with the {\cvdense} linear solver, and user-supplied    
  Jacobian routines.

\item \id{cvakx}
  solves a stiff ODE system that arises from a system     
  of partial differential equations.  The PDE system is a six-species
  food web population model, with predator-prey interaction and diffusion 
  on the unit square in two dimensions.

  The adjoint capability of {\cvodes} is used to compute gradients
  of the average (over both time and space) of the concentration of a selected species
  with respect to the initial conditions of all six species.
  Both the forward and backward problems are solved with the BDF/GMRES method 
  (i.e. using the {\cvspgmr} linear solver) and the block-diagonal part of the  
  Newton matrix as a left preconditioner.

\item \id{cvakxb}
  solves the same problem as \id{cvakx}, but computes gradients
  of the average over space at the final time of the concentration of a selected species
  with respect to the initial conditions of all six species.

\end{itemize}
%%
%%--------------------------------------
%%

\vspace{0.2in}
\noindent Supplied in the \id{sundials/cvode/examples\_par} directory are
the following six parallel examples (using the {\nvecp} module):
%%
%%--------------------------------------
%%
\begin{itemize}

\item \id{pvnx}
  solves the semi-discrete form of an advection-diffusion equation in 1-D.

  This program solves the problem with the option for nonstiff systems,
  i.e. Adams method and functional iteration.

\item \id{pvkx}
  is the parallel implementation of \id{cvkx}.

\item \id{pvkxb}
  solves the same problem as \id{pvkx}, with the BDF/GMRES method 
  and a block-diagonal matrix with banded blocks as a preconditioner, 
  generated by difference quotients, using the module {\cvbbdpre}.

\item \id{pvfnx}
  is the parallel version of \id{cvfnx}.

\item \id{pvfkx}
  is the parallel version of \id{cvfkx}.

\item \id{pvanx}
  solves the semi-discrete form of an advection-diffusion equation in 1-D.

  The adjoint capability of {\cvodes} is used to compute gradients
  of the average over space of the solution at the final time with
  respect to both the initial conditions and the advection and
  diffusion coefficients in the model.
  This program solves both the forward and backward problems with the option 
  for nonstiff systems, i.e. Adams method and functional iteration.

\item \id{pvakx}
  solves an adjoint sensitivity problem for an advection-diffusion PDE in 2-D 
  or 3-D using the BDF/GMRES method and the {\cvbbdpre} preconditioner module
  on both the forward and backward phases.

  The adjoint capability of {\cvodes} is used to compute the gradient of the
  space-time average of the squared solution norm with respect to problem 
  parameters which parametrize a distributed volume source.

\end{itemize}
%%
%%--------------------------------------

\vspace{0.2in}\noindent 
In the following sections, we give detailed descriptions of some (but
not all) of the sensitivity analysis examples. We do not discuss any of the 
examples for IVP integration. The interested reader should consult the
{\cvode} Examples Document~\cite{cvode2.2.0_ex}. Any {\cvode} problem
will work with {\cvodes} with only one modification: the main program
should include the header file \id{cvodes.h} instead of \id{cvode.h}.

The Appendices contain complete listings
of the examples described below.  We also give our output files for
each of these examples, but users should be cautioned that their
results may differ slightly from these.  Differences in solution
values may differ within the tolerances, and differences in cumulative
counters, such as numbers of steps or Newton iterations, may differ
from one machine environment to another by as much as 10\% to 20\%.

The final section of this report describes a set of tests done with
{\cvodes} in a parallel environment, using {\nvecp},on a modification of
the \id{pvkx} example.

In the descriptions below, we make frequent references to the {\cvodes}
User Guide~\cite{cvodes2.1.0_ug}.  All citations to specific sections
(e.g. \S\ref{s:types}) are references to parts of that user guide, unless
explicitly stated otherwise.

\vspace{0.2in}\noindent {\bf Note}. 
The examples in the {\cvodes} distribution are written in such a way as
to compile and run for any combination of configuration options during
the installation of {\sundials} (see \ugref{s:install}). As a consequence,
they contain portions of code that will not be typically present in a
user program. For example, all example programs make use of the
variable \id{SUNDIALS\_EXTENDED\_PRECISION} to test if the solver libraries
were built in extended precision and use the appropriate conversion 
specifiers in \id{printf} functions. Similarly, all forward sensitivity
examples can be run with or without sensitivity computations enabled and,
in the former case, with various combinations of methods and error control 
strategies. This is achieved in these example through the program arguments.

