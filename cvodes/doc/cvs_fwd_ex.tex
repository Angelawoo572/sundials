%===================================================================================
\section{Example Problems for Forward Sensitivity Analysis}\label{s:fwd_examples}
%===================================================================================
\index{forward sensitivity analysis!examples|see{examples, forward sensitivity}}

\index{examples, forward sensitivity!list of|(}
The {\cvodes} distribution contains, in the \id{sundials/cvodes/fwd\_examples}
directory, the following five examples for forward sensitivity analysis:

\begin{itemize}
\item \id{cvfnx}
  solves the semi-discrete form of an advection-diffusion equation in 1-D.

  {\cvodes} computes both its solution and solution sensitivities with respect
  to the advection and diffusion coefficients.
  This program solves the problem with the option for nonstiff systems,
  i.e. Adams method and functional iteration;
\item \id{cvfdx}
  solves a chemical kinetics problem consisting of three rate equations.

  {\cvodes} computes both its solution and solution sensitivities with respect
  to the three reaction rate constants appearing in the model. 
  This program solves the problem with the BDF method, Newton          
  iteration with the {\cvdense} linear solver, and a user-supplied    
  Jacobian routine;
\item \id{cvfkx}
  solves the semi-discrete form of a two-species diurnal kinetics
  advection-diffusion PDE system in 2-D space.

  {\cvodes} computes both its solution and solution sensitivities with respect
  to two parameters affecting the kinetic rate terms.
  The problem is solved with the BDF/GMRES method (i.e.    
  using the {\cvspgmr} linear solver) and the block-diagonal part of the  
  Newton matrix as a left preconditioner;
\item \id{pvfnx}
  is the parallel version of \id{cvfnx};
\item \id{pvfkx}
  is the parallel version of \id{cvfkx}.
\end{itemize}
The first three are serial examples that use the {\nvecs} module and
the last two are parallel examples using the {\nvecp} module.
For all the above examples, any of three sensitivity methods (\id{SIMULTANEOUS}, 
\id{STAGGERED}, and \id{STAGGERED1}) can be used and sensitivities may be 
included in the error test or not (error control set on \id{FULL} or \id{PARTIAL},
respectively).
\index{examples, forward sensitivity!list of|)}

The next two sections describe in detail a serial example (\id{cvfdx}) and
a parallel one (\id{pvfkx}). For details on the other examples, the reader is
directed to the comments in their source files.

%--------------------------------------------------------------------
\subsection{A Serial Sample Problem}\label{ss:serial_fwd_ex}

As a first example of using {\cvodes} for forward sensitivity analysis,
we provide a modification of the chemical kinetics problem described in
\S\ref{ss:serial_sim_ex} which computes, in addition to the solution of the
IVP, sensitivities of the solution with respect to the three reaction rates 
involved in the model.
\index{examples, forward sensitivity!serial sample program!problem solved by|(}
The ODEs are written as:
\begin{equation}\label{e:chemkin}
  \begin{split}
    {\dot y}_1 &= -p_1 y_1 + p_2 y_2 y_3   \\
    {\dot y}_2 &=  p_1 y_1 - p_2 y_2 y_3 - p_3 y_2^2 \\
    {\dot y}_3 &=  p_3 y_2^2 \, ,
  \end{split}
\end{equation}
with initial conditions at $t_0 = 0$, $y_1 = 1$ and $y_2 = y_3 = 0$. 
The nominal values of the reaction rate constants are 
$p_1 = 0.04$, $p_2 = 10^4$, and $p_3 = 3\cdot 10^7$.
The sensitivity systems that are solved together with (\ref{e:chemkin}) are
\begin{equation}
  \begin{split}
    & {\dot s}_i = 
    \begin{bmatrix}
      - p_1 &   p_2 y_3             &   p_2 y_2 \\
        p_1 & - p_2 y_3 - 2 p_3 y_2 & - p_2 y_2 \\
        0   &             2 p_3 y_2 &  0              
    \end{bmatrix}
    s_i + \frac{\partial f}{\partial p_i} ~,
    \quad s_i(t_0) = \begin{bmatrix} 0 \\ 0 \\ 0 \end{bmatrix}  ~,
    \quad i = 1,2,3 \\
    & \frac{\partial f}{\partial p_1} = \begin{bmatrix} -y_1 \\ y_1 \\ 0 \end{bmatrix}, \quad
    \frac{\partial f}{\partial p_2} = \begin{bmatrix} y_2 y_3 \\ -y_2 y_3 \\ 0 \end{bmatrix}, \quad
    \frac{\partial f}{\partial p_3} = \begin{bmatrix} 0 \\ - y_2^2 \\ y_2^2 \end{bmatrix} \, .
  \end{split}
\end{equation}
\index{examples, forward sensitivity!serial sample program!problem solved by|)}

The source code for this example is listed in \A\ref{ss:cvfdx}. The main program is described 
below with emphasis on the sensitivity related components. 
\index{examples, forward sensitivity!serial sample program!explanation of|(}
These explanations, together with those given for the code \id{cvdx}
in \S\ref{ss:serial_sim_ex}, will also provide the user with a template for instrumenting 
an existing simulation code to perform forward sensitivity analysis.
As will be seen from this example, an existing simulation code can be modified to compute 
sensitivity variables (in addition to state variables) by only inserting a few {\cvodes} 
calls in the main program. 

First note that no new header files need be included. In addition to the constants already
defined in \id{cvdx} we define the number of model parameters, \id{NP} ($=3$),
the number of sensitivity parameters, \id{NS} ($=3$), and a constant \id{ZERO}$=0.0$. 

As mentioned in \S\ref{ss:forward_usage} and \S\ref{sss:cvodesensmalloc}, the user data
structure \id{f\_data} must provide access to the array of model parameters 
\index{examples, forward sensitivity!serial sample program!user data}
as the only way for {\cvodes} to communicate parameter values to the right hand side 
function \id{f}. In the \id{cvfdx} example this is done by defining \id{f\_data} to be 
of type \id{UserData}, i.e. a pointer to a structure which contains an array of 
\id{NP} \id{realtype} values.

The program prologue ends by defining three additional private helper functions. The first
one, \id{WrongArgs} (which would not be present in a typical user code) prints a usage
message and stops execution if the command line arguments to \id{cvfdx} are wrong.
After each successful return from the main {\cvodes} integrator, the functions 
\id{PrintOutput} and \id{PrintOutputS} print the state and sensitivity variables,
respectively. The program does not define any additional user-supplied functions 
since this example uses the {\cvodes} internal difference quotient routines to compute
the sensitivity equations right hand sides (see below).

The \id{main} function begins with definitions and type declarations. In addition to 
those in \id{cvdx.s}, it defines the vector \id{pbar} of \id{NP} scaling factors for
the model parameters \id{p}, the array \id{plist} of \id{integertype} flags specifying
the sensitivity parameters among the model parameters, and the array \id{yS} of 
\id{N\_Vector}'s which will contain the initial conditions and solutions for the sensitivity
variables. It also declares the variable \id{data} of type \id{UserData} 
which will contain the user-data structure to be passed to {\cvodes} and used in the 
evaluation of the ODEs right hand sides.

The first code block in \id{main} deals with reading and interpreting the
command line arguments. \id{cvfdx} can be run with or without sensitivity computations
turned on and with different selections for the sensitivity method and error control strategy.

Next, the serial machine environment variable \id{machEnv} is initialized by calling
the {\nvecs} function \id{M\_EnvInit\_Serial} and the user-data structure is allocated
and initialized with the model parameter values.
\index{examples, forward sensitivity!serial sample program!user data}

The next block of code is identical to that in \id{cvdx.c} and involves allocation and 
initialization of the state variables and integration tolerances for the state variables,
initialization of \id{cvode\_mem} and of the {\cvodes} solver memory. It also attaches {\cvdense},
with a non-\id{NULL} Jacobian function, as the linear solver to be used in the Newton 
nonlinear solver.

If sensitivity analysis is enabled (through the command line arguments), 
the main program will then set the scaling parameters
\id{pbar} (\id{pbar}$_i$ = \id{p}$_i$, which can typically be used for 
non-zero model parameters) and the array of flags \id{plist}. The choice \id{plist}$_i=i+1$
indicates that sensitivities with respect to all model parameters will be computed.
Next, the program allocates memory for \id{yS}, by calling the {\nvector} function 
\id{N\_VNew\_S}, and initializaes all sensitivity variables to $0.0$.

The call to \id{CVodeSensMalloc} specifies the sensitivity solution method through 
\id{sensi\-meth} (read from the command line arguments) as \id{SIMULTANEOUS}, 
\id{STAGGERED}, or \id{STAGGERED1} and the error control strategy through \id{err\_con}
(also read from the command line arguments) as either \id{FULL} or \id{PARTIAL}.
The \id{ifS} parameter indicates the type of sensitivity right hand side function
(\id{ALLSENS}, or \id{ONESENS} if the \id{STAGGERED1} approach was selected),
while the first \id{NULL} parameter indicates that the {\cvodes} internal difference quotient 
routines (\id{CVSensRhsDQ} or \id{CVSensRhsDQ1}, depending on the value of \id{ifS})
should be used. The last two \id{NULL} parameters in the call to \id{CVodeSensMalloc}
force {\cvodes} to set the relative and absolute
tolerances \id{rtolS} and \id{atolS} for sensitivity variables based on the tolerances for 
state variables and the scaling parameters \id{pbar} (see \S\ref{ss:fwd_sensi} for details).

The return value of \id{CVodeSensMalloc} is an \id{int} flag. In case of failure
(\id{flag != SUCCESS}), the main program prints an error message and stops execution.

Next, in a loop over the \id{NOUT} output times, the program calls the integration
routine \id{CVode} which, if sensitivity analysis was initialized through the call
to \id{CVodeSensMalloc}, computes both state and sensitivity variables. However,
\id{CVode} returns only the state solution at \id{tout} in the vector \id{y}.
The program tests the return from \id{CVode} for a value other than \id{SUCCESS} and
prints the state variables.
Sensitivity variables at \id{tout} are loaded into \id{yS} by calling \id{CVodeSensExtract}.
The program tests the return from \id{CVodeSensExtract} for a value other than \id{SUCCESS} 
and then prints the sensitivity variables.

Finally, the program prints some final statistics and deallocates memory through calls
to \id{N\_VFree}, \id{N\_VFree\_S}, \id{CVodeFree}, and \id{M\_EnvFree\_Serial}.

The user-supplied functions \id{f} for the right hand side of the original ODEs and
\id{Jac} for the system Jacobian are identical to those in \id{cvdx.c} with the 
notable exeption that model parameters are extracted from the user-data structure
\id{f\_data}, which must be first cast to the \id{UserData} type.
\index{examples, forward sensitivity!serial sample program!explanation of|)}

Sample outputs from \id{cvfdx}, for two different combinations of command line arguments, 
follows. The command to execute this program must have the form:
\begin{verbatim}
% cvfdx -nosensi
\end{verbatim} 
if no sensitivity calculations are desired, or
\begin{verbatim}
% cvfdx -sensi sensi_meth err_con
\end{verbatim}
where \id{sensi\_meth} must be one of \id{sim}, \id{stg}, or \id{stg1} to
indicate the \id{SIMULTANEOUS}, \id{STAGGERED}, or \id{STAGGERED1} method,
respectively and \id{err\_con} must be one of \id{full} or \id{partial} to
select the \id{FULL} or \id{PARTIAL} error control strategy, respectively.

\index{examples, forward sensitivity!serial sample program!output|(}
The output generated by \id{cvfdx} when computing sensitivities with the \id{SIMULTANEOUS}
method and \id{FULL} error control (\id{cvfdx -sensi sim full}) is:
{\small\begin{verbatim}
3-species chemical kinetics problem

=====================================================================================
     T     Q       H      NST                    y1           y2           y3    
=====================================================================================
4.000e-01  3  3.507e-02   106
                                Solution         9.8517e-01   3.3864e-05   1.4794e-02 
                                Sensitivity 1   -3.5595e-01   3.9026e-04   3.5556e-01 
                                Sensitivity 2    9.5428e-08  -2.1310e-10  -9.5215e-08 
                                Sensitivity 3   -1.5832e-11  -5.2900e-13   1.6361e-11 
-------------------------------------------------------------------------------------
4.000e+00  4  1.705e-01   145
                                Solution         9.0552e-01   2.2405e-05   9.4458e-02 
                                Sensitivity 1   -1.8761e+00   1.7922e-04   1.8759e+00 
                                Sensitivity 2    2.9615e-06  -5.8304e-10  -2.9609e-06 
                                Sensitivity 3   -4.9336e-10  -2.7627e-13   4.9363e-10 
-------------------------------------------------------------------------------------
4.000e+01  3  1.872e+00   226
                                Solution         7.1582e-01   9.1854e-06   2.8417e-01 
                                Sensitivity 1   -4.2476e+00   4.5906e-05   4.2476e+00 
                                Sensitivity 2    1.3731e-05  -2.3572e-10  -1.3731e-05 
                                Sensitivity 3   -2.2884e-09  -1.1381e-13   2.2885e-09 
-------------------------------------------------------------------------------------
4.000e+02  3  1.006e+01   317
                                Solution         4.5051e-01   3.2228e-06   5.4949e-01 
                                Sensitivity 1   -5.9584e+00   3.5418e-06   5.9584e+00 
                                Sensitivity 2    2.2738e-05  -2.2595e-11  -2.2738e-05 
                                Sensitivity 3   -3.7897e-09  -4.9947e-14   3.7898e-09 
-------------------------------------------------------------------------------------
4.000e+03  2  1.486e+02   487
                                Solution         1.8319e-01   8.9415e-07   8.1681e-01 
                                Sensitivity 1   -4.7500e+00  -5.9942e-06   4.7500e+00 
                                Sensitivity 2    1.8809e-05   2.3128e-11  -1.8809e-05 
                                Sensitivity 3   -3.1348e-09  -1.8758e-14   3.1348e-09 
-------------------------------------------------------------------------------------
4.000e+04  3  2.000e+03   567
                                Solution         3.8978e-02   1.6215e-07   9.6102e-01 
                                Sensitivity 1   -1.5748e+00  -2.7622e-06   1.5749e+00 
                                Sensitivity 2    6.2870e-06   1.1003e-11  -6.2870e-06 
                                Sensitivity 3   -1.0478e-09  -4.5363e-15   1.0478e-09 
-------------------------------------------------------------------------------------
4.000e+05  3  1.765e+04   625
                                Solution         4.9391e-03   1.9853e-08   9.9506e-01 
                                Sensitivity 1   -2.3638e-01  -4.5854e-07   2.3639e-01 
                                Sensitivity 2    9.4523e-07   1.8330e-12  -9.4523e-07 
                                Sensitivity 3   -1.5752e-10  -6.3633e-16   1.5752e-10 
-------------------------------------------------------------------------------------
4.000e+06  4  2.252e+05   678
                                Solution         5.1687e-04   2.0685e-09   9.9948e-01 
                                Sensitivity 1   -2.5669e-02  -5.1067e-08   2.5669e-02 
                                Sensitivity 2    1.0267e-07   2.0425e-13  -1.0267e-07 
                                Sensitivity 3   -1.7111e-11  -6.8516e-17   1.7111e-11 
-------------------------------------------------------------------------------------
4.000e+07  3  1.457e+06   740
                                Solution         5.2046e-05   2.0820e-10   9.9995e-01 
                                Sensitivity 1   -2.6001e-03  -5.1968e-09   2.6001e-03 
                                Sensitivity 2    1.0400e-08   2.0787e-14  -1.0400e-08 
                                Sensitivity 3   -1.7333e-12  -6.9338e-18   1.7333e-12 
-------------------------------------------------------------------------------------
4.000e+08  3  2.570e+07   796
                                Solution         5.2108e-06   2.0843e-11   9.9999e-01 
                                Sensitivity 1   -2.6043e-04  -5.2069e-10   2.6043e-04 
                                Sensitivity 2    1.0417e-09   2.0828e-15  -1.0417e-09 
                                Sensitivity 3   -1.7367e-13  -6.9470e-19   1.7367e-13 
-------------------------------------------------------------------------------------
4.000e+09  3  3.805e+08   828
                                Solution         5.1965e-07   2.0786e-12   1.0000e+00 
                                Sensitivity 1   -2.6131e-05  -5.2722e-11   2.6131e-05 
                                Sensitivity 2    1.0452e-10   2.1089e-16  -1.0452e-10 
                                Sensitivity 3   -1.7322e-14  -6.9288e-20   1.7322e-14 
-------------------------------------------------------------------------------------
4.000e+10  3  7.725e+09   851
                                Solution         5.8386e-08   2.3355e-13   1.0000e+00 
                                Sensitivity 1   -2.7256e-06  -5.0248e-12   2.7257e-06 
                                Sensitivity 2    1.0903e-11   2.0099e-17  -1.0903e-11 
                                Sensitivity 3   -1.9441e-15  -7.7764e-21   1.9441e-15 
-------------------------------------------------------------------------------------


========================================================
Final Statistics
Sensitivity: YES ( SIMULTANEOUS + FULL ERROR CONTROL )

nst     =   851                

nfe     =  7987    nfSe  =  1141  
nni     =  1138    nniS  =     0  
ncfn    =     4    ncfnS =     0  
netf    =    17    netfS =     0

nsetups =   136                  
nje     =    24                  
========================================================
\end{verbatim}}

The output generated by \id{cvfdx} when computing sensitivities with the \id{STAGGERED1}
method and \id{PARTIAL} error control (\id{cvfdx -sensi stg1 partial}) is:
{\small\begin{verbatim}
3-species chemical kinetics problem

=====================================================================================
     T     Q       H      NST                    y1           y2           y3    
=====================================================================================
4.000e-01  3  1.153e-01    60
                                Solution         9.8517e-01   3.3863e-05   1.4797e-02 
                                Sensitivity 1   -3.5609e-01   3.9023e-04   3.5570e-01 
                                Sensitivity 2    9.4895e-08  -2.1323e-10  -9.4682e-08 
                                Sensitivity 3   -1.5744e-11  -5.2897e-13   1.6273e-11 
-------------------------------------------------------------------------------------
4.000e+00  4  4.988e-01    74
                                Solution         9.0552e-01   2.2404e-05   9.4458e-02 
                                Sensitivity 1   -1.8761e+00   1.7923e-04   1.8759e+00 
                                Sensitivity 2    2.9615e-06  -5.8303e-10  -2.9610e-06 
                                Sensitivity 3   -4.9339e-10  -2.7625e-13   4.9366e-10 
-------------------------------------------------------------------------------------
4.000e+01  3  1.099e+00   119
                                Solution         7.1583e-01   9.1860e-06   2.8416e-01 
                                Sensitivity 1   -4.2476e+00   4.5906e-05   4.2476e+00 
                                Sensitivity 2    1.3731e-05  -2.3574e-10  -1.3730e-05 
                                Sensitivity 3   -2.2883e-09  -1.1381e-13   2.2884e-09 
-------------------------------------------------------------------------------------
4.000e+02  3  1.992e+01   181
                                Solution         4.5044e-01   3.2218e-06   5.4955e-01 
                                Sensitivity 1   -5.9590e+00   3.5329e-06   5.9590e+00 
                                Sensitivity 2    2.2741e-05  -2.2531e-11  -2.2741e-05 
                                Sensitivity 3   -3.7902e-09  -4.9938e-14   3.7903e-09 
-------------------------------------------------------------------------------------
4.000e+03  3  1.340e+02   242
                                Solution         1.8320e-01   8.9428e-07   8.1680e-01 
                                Sensitivity 1   -4.7500e+00  -5.9938e-06   4.7500e+00 
                                Sensitivity 2    1.8809e-05   2.3123e-11  -1.8809e-05 
                                Sensitivity 3   -3.1348e-09  -1.8758e-14   3.1348e-09 
-------------------------------------------------------------------------------------
4.000e+04  4  2.337e+03   326
                                Solution         3.8979e-02   1.6216e-07   9.6102e-01 
                                Sensitivity 1   -1.5748e+00  -2.7618e-06   1.5748e+00 
                                Sensitivity 2    6.2869e-06   1.1002e-11  -6.2869e-06 
                                Sensitivity 3   -1.0478e-09  -4.5363e-15   1.0478e-09 
-------------------------------------------------------------------------------------
4.000e+05  4  1.747e+04   365
                                Solution         4.9378e-03   1.9848e-08   9.9506e-01 
                                Sensitivity 1   -2.3629e-01  -4.5827e-07   2.3629e-01 
                                Sensitivity 2    9.4476e-07   1.8316e-12  -9.4477e-07 
                                Sensitivity 3   -1.5750e-10  -6.3623e-16   1.5750e-10 
-------------------------------------------------------------------------------------
4.000e+06  4  2.064e+05   407
                                Solution         5.1680e-04   2.0683e-09   9.9948e-01 
                                Sensitivity 1   -2.5640e-02  -5.0960e-08   2.5640e-02 
                                Sensitivity 2    1.0255e-07   2.0381e-13  -1.0256e-07 
                                Sensitivity 3   -1.7117e-11  -6.8540e-17   1.7117e-11 
-------------------------------------------------------------------------------------
4.000e+07  4  1.690e+06   450
                                Solution         5.2014e-05   2.0807e-10   9.9995e-01 
                                Sensitivity 1   -2.5993e-03  -5.1968e-09   2.5994e-03 
                                Sensitivity 2    1.0397e-08   2.0787e-14  -1.0397e-08 
                                Sensitivity 3   -1.7328e-12  -6.9320e-18   1.7328e-12 
-------------------------------------------------------------------------------------
4.000e+08  5  4.266e+07   498
                                Solution         5.2115e-06   2.0846e-11   9.9999e-01 
                                Sensitivity 1   -2.6013e-04  -5.1938e-10   2.6013e-04 
                                Sensitivity 2    1.0405e-09   2.0775e-15  -1.0405e-09 
                                Sensitivity 3   -1.7358e-13  -6.9434e-19   1.7358e-13 
-------------------------------------------------------------------------------------
4.000e+09  3  6.156e+08   531
                                Solution         5.1852e-07   2.0741e-12   1.0000e-00 
                                Sensitivity 1   -2.6093e-05  -5.2518e-11   2.6093e-05 
                                Sensitivity 2    1.0437e-10   2.1007e-16  -1.0437e-10 
                                Sensitivity 3   -1.7284e-14  -6.9137e-20   1.7284e-14 
-------------------------------------------------------------------------------------
4.000e+10  4  3.281e+09   550
                                Solution         4.8604e-08   1.9442e-13   1.0000e-00 
                                Sensitivity 1   -2.5704e-06  -5.4213e-12   2.5704e-06 
                                Sensitivity 2    1.0282e-11   2.1685e-17  -1.0282e-11 
                                Sensitivity 3   -1.6230e-15  -6.4920e-21   1.6230e-15 
-------------------------------------------------------------------------------------


========================================================
Final Statistics
Sensitivity: YES ( STAGGERED + PARTIAL ERROR CONTROL )

nst     =   550                

nfe     =  1293    nfSe  =  2649  
nni     =   740    nniS  =   882  
ncfn    =     0    ncfnS =     0  
netf    =    26    netfS =     0

nsetups =   115                  
nje     =    13                  
========================================================
\end{verbatim}}
\index{examples, forward sensitivity!serial sample program!output|)}

%----------------------------------------------------------------------------------

\subsection{A Parallel Sample Program}\label{ss:parallel_fwd_ex}

\index{examples, forward sensitivity!parallel sample program!problem solved by|(}
The second example program for forward sensitivity analysis, \id{pvfkx} is the
semi-discrete form of a two-species diurnal kinetics advection-diffusion PDE 
system in 2-D space (\ref{e:pvkx:pde}), described in \S\ref{ss:parallel_sim_ex}, 
for which we compute solution sensitivities with respect to the parameters $q_1$ 
and $q_2$ that appear in the kinetic rate terms (\ref{e:pvkx:r}).
\index{examples, forward sensitivity!parallel sample program!problem solved by|)}

\index{examples, forward sensitivity!parallel sample program!explanation of|(}
The source code for this example is listed in \A\ref{ss:pvfkx}.
The overall structure of the \id{main} function is very
similar to that of the code \id{cvfdx} (\S\ref{ss:serial_fwd_ex}) with 
differences arising from the use of the parallel vector module.
On the other hand, the user-supplied routines in \id{pvfkx},
\id{f} for the right-hand side of the original system,
\id{Precond} for the preconditioner set-up, and \id{PSolve} for the
preconditioner solve, are identical to those defined for the sample program
\id{pvkx} (see \S\ref{ss:parallel_sim_ex}). The only difference is in the
routine \id{fcalc}, which operates on local data only and contains the actual 
calculation of $f(t,u)$, where the problem parameters are first extracted from
the user data structure \id{data}. The program \id{pvfkx} defines no additional
user-supplied routines, as it uses the {\cvodes} internal difference quotient routines 
to compute the sensitivity equation right hand sides (as indicated by passing
\id{NULL} for \id{fS} in the call to \id{CVodeSensMalloc}).
\index{examples, forward sensitivity!parallel sample program!explanation of|)}

Sample outputs from \id{pvfkx}, for two different combinations of command line arguments, 
follow. The command to execute this program must have the form:
\begin{verbatim}
% mpirun -np nproc pvfkx -nosensi
\end{verbatim} 
if no sensitivity calculations are desired, or
\begin{verbatim}
% mpirun -np nproc pvfkx -sensi sensi_meth err_con
\end{verbatim}
where \id{nproc} is the number of processes, \id{sensi\_meth} must be one of \id{sim}, 
\id{stg}, or \id{stg1} to
indicate the \id{SIMULTANEOUS}, \id{STAGGERED}, or \id{STAGGERED1} method,
respectively and \id{err\_con} must be one of \id{full} or \id{partial} to
select the \id{FULL} or \id{PARTIAL} error control strategy, respectively.

\index{examples, forward sensitivity!parallel sample program!output|(}
The output generated by \id{pvfkx} when computing sensitivities with the \id{SIMULTANEOUS}
method and \id{FULL} error control (\id{mpirun -np 4 pvfkx -sensi sim full}) is:
{\small\begin{verbatim}
2-species diurnal advection-diffusion problem

========================================================================
     T     Q       H      NST                    Bottom left  Top right 
========================================================================
7.200e+03  3  3.601e+01   341
                                Solution         1.0468e+04   1.1185e+04 
                                                 2.5267e+11   2.6998e+11 
                                ----------------------------------------
                                Sensitivity 1   -6.4201e+19  -6.8598e+19 
                                                 7.1177e+19   7.6556e+19 
                                ----------------------------------------
                                Sensitivity 2   -4.3853e+14  -5.0065e+14 
                                                -2.4407e+18  -2.7843e+18 
------------------------------------------------------------------------
1.440e+04  3  9.479e+01   546
                                Solution         6.6590e+06   7.3008e+06 
                                                 2.5819e+11   2.8329e+11 
                                ----------------------------------------
                                Sensitivity 1   -4.0848e+22  -4.4785e+22 
                                                 5.9549e+22   6.7173e+22 
                                ----------------------------------------
                                Sensitivity 2   -4.5235e+17  -5.4318e+17 
                                                -6.5418e+21  -7.8315e+21 
------------------------------------------------------------------------
2.160e+04  3  4.631e+01   940
                                Solution         2.6650e+07   2.9308e+07 
                                                 2.9928e+11   3.3134e+11 
                                ----------------------------------------
                                Sensitivity 1   -1.6346e+23  -1.7976e+23 
                                                 3.8203e+23   4.4991e+23 
                                ----------------------------------------
                                Sensitivity 2   -7.6601e+18  -9.4433e+18 
                                                -7.6459e+22  -9.4501e+22 
------------------------------------------------------------------------
2.880e+04  3  4.488e+01  1114
                                Solution         8.7021e+06   9.6501e+06 
                                                 3.3804e+11   3.7510e+11 
                                ----------------------------------------
                                Sensitivity 1   -5.3375e+22  -5.9188e+22 
                                                 5.4487e+23   6.7430e+23 
                                ----------------------------------------
                                Sensitivity 2   -4.8855e+18  -6.1040e+18 
                                                -1.7194e+23  -2.1518e+23 
------------------------------------------------------------------------
3.600e+04  3  1.185e+01  1298
                                Solution         1.4040e+04   1.5609e+04 
                                                 3.3868e+11   3.7652e+11 
                                ----------------------------------------
                                Sensitivity 1   -8.6141e+19  -9.5761e+19 
                                                 5.2718e+23   6.6030e+23 
                                ----------------------------------------
                                Sensitivity 2   -8.4328e+15  -1.0549e+16 
                                                -1.8439e+23  -2.3096e+23 
------------------------------------------------------------------------
4.320e+04  3  7.455e+01  1431
                                Solution        -1.1264e-06  -1.3709e-06 
                                                 3.3823e+11   3.8035e+11 
                                ----------------------------------------
                                Sensitivity 1    1.5351e+10   1.7978e+10 
                                                 5.2753e+23   6.7448e+23 
                                ----------------------------------------
                                Sensitivity 2   -4.9528e+05  -6.6937e+05 
                                                -1.8454e+23  -2.3595e+23 
------------------------------------------------------------------------
5.040e+04  4  1.472e+02  1478
                                Solution        -2.4251e-06  -2.8291e-06 
                                                 3.3582e+11   3.8645e+11 
                                ----------------------------------------
                                Sensitivity 1   -1.5598e+10  -1.8305e+10 
                                                 5.2067e+23   6.9664e+23 
                                ----------------------------------------
                                Sensitivity 2   -1.3900e+08  -1.8680e+08 
                                                -1.8214e+23  -2.4371e+23 
------------------------------------------------------------------------
5.760e+04  5  3.006e+02  1517
                                Solution        -2.4915e-09  -2.8647e-09 
                                                 3.3203e+11   3.9090e+11 
                                ----------------------------------------
                                Sensitivity 1    1.9998e+09   2.2998e+09 
                                                 5.0825e+23   7.1205e+23 
                                ----------------------------------------
                                Sensitivity 2    2.0183e+04   2.7043e+04 
                                                -1.7780e+23  -2.4910e+23 
------------------------------------------------------------------------
6.480e+04  2  7.538e+01  1628
                                Solution         1.3570e-09   1.4501e-09 
                                                 3.3130e+11   3.9634e+11 
                                ----------------------------------------
                                Sensitivity 1   -5.9393e+08  -6.3445e+08 
                                                 5.0442e+23   7.3274e+23 
                                ----------------------------------------
                                Sensitivity 2   -4.2721e+06  -5.0168e+06 
                                                -1.7646e+23  -2.5633e+23 
------------------------------------------------------------------------
7.200e+04  5  4.437e+02  1654
                                Solution        -3.8398e-11  -4.0723e-11 
                                                 3.3297e+11   4.0389e+11 
                                ----------------------------------------
                                Sensitivity 1    2.2933e+06   2.4329e+06 
                                                 5.0783e+23   7.6382e+23 
                                ----------------------------------------
                                Sensitivity 2    1.1312e+04   1.3201e+04 
                                                -1.7765e+23  -2.6721e+23 
------------------------------------------------------------------------
7.920e+04  5  6.664e+02  1669
                                Solution         2.5611e-11   2.6926e-11 
                                                 3.3344e+11   4.1203e+11 
                                ----------------------------------------
                                Sensitivity 1   -7.8910e+06  -8.3733e+06 
                                                 5.0731e+23   7.9960e+23 
                                ----------------------------------------
                                Sensitivity 2   -8.4799e+01  -1.0851e+02 
                                                -1.7747e+23  -2.7972e+23 
------------------------------------------------------------------------
8.640e+04  5  6.664e+02  1680
                                Solution         3.9957e-12   4.2057e-12 
                                                 3.3518e+11   4.1625e+11 
                                ----------------------------------------
                                Sensitivity 1   -5.6675e+06  -5.9636e+06 
                                                 5.1171e+23   8.2143e+23 
                                ----------------------------------------
                                Sensitivity 2   -5.1929e+01  -6.7654e+01 
                                                -1.7901e+23  -2.8736e+23 
------------------------------------------------------------------------


========================================================
Final Statistics
Sensitivity: YES ( SIMULTANEOUS + FULL ERROR CONTROL )

nst     =  1680                

nfe     = 17419    nfSe  =  2429  
nni     =  2427    nniS  =     0  
ncfn    =     2    ncfnS =     0  
netf    =   128    netfS =     0

nsetups =   328                  
nli     =  5274    ncfl  =     0  
npe     =    34    nps   = 11262  
========================================================
\end{verbatim}}

The output generated by \id{pvfkx} when computing sensitivities with the \id{STAGGERED1}
method and \id{PARTIAL} error control (\id{mpirun -np 4 pvfkx -sensi stg1 partial}) is:
{\small\begin{verbatim}
2-species diurnal advection-diffusion problem

========================================================================
     T     Q       H      NST                    Bottom left  Top right 
========================================================================
7.200e+03  5  1.587e+02   219
                                Solution         1.0468e+04   1.1185e+04 
                                                 2.5267e+11   2.6998e+11 
                                ----------------------------------------
                                Sensitivity 1   -6.4201e+19  -6.8598e+19 
                                                 7.1178e+19   7.6555e+19 
                                ----------------------------------------
                                Sensitivity 2   -4.3853e+14  -5.0065e+14 
                                                -2.4407e+18  -2.7842e+18 
------------------------------------------------------------------------
1.440e+04  5  3.772e+02   251
                                Solution         6.6590e+06   7.3008e+06 
                                                 2.5819e+11   2.8329e+11 
                                ----------------------------------------
                                Sensitivity 1   -4.0848e+22  -4.4785e+22 
                                                 5.9550e+22   6.7173e+22 
                                ----------------------------------------
                                Sensitivity 2   -4.5235e+17  -5.4317e+17 
                                                -6.5418e+21  -7.8315e+21 
------------------------------------------------------------------------
2.160e+04  5  2.746e+02   277
                                Solution         2.6650e+07   2.9308e+07 
                                                 2.9928e+11   3.3134e+11 
                                ----------------------------------------
                                Sensitivity 1   -1.6346e+23  -1.7976e+23 
                                                 3.8203e+23   4.4991e+23 
                                ----------------------------------------
                                Sensitivity 2   -7.6601e+18  -9.4433e+18 
                                                -7.6459e+22  -9.4502e+22 
------------------------------------------------------------------------
2.880e+04  3  1.225e+02   311
                                Solution         8.7021e+06   9.6500e+06 
                                                 3.3804e+11   3.7510e+11 
                                ----------------------------------------
                                Sensitivity 1   -5.3375e+22  -5.9187e+22 
                                                 5.4487e+23   6.7430e+23 
                                ----------------------------------------
                                Sensitivity 2   -4.8854e+18  -6.1042e+18 
                                                -1.7194e+23  -2.1518e+23 
------------------------------------------------------------------------
3.600e+04  4  7.367e+01   355
                                Solution         1.4040e+04   1.5609e+04 
                                                 3.3868e+11   3.7652e+11 
                                ----------------------------------------
                                Sensitivity 1   -8.6140e+19  -9.5761e+19 
                                                 5.2718e+23   6.6029e+23 
                                ----------------------------------------
                                Sensitivity 2   -8.4328e+15  -1.0549e+16 
                                                -1.8439e+23  -2.3096e+23 
------------------------------------------------------------------------
4.320e+04  5  6.190e+02   411
                                Solution        -6.2931e-08  -2.6406e-07 
                                                 3.3823e+11   3.8035e+11 
                                ----------------------------------------
                                Sensitivity 1   -3.9700e+09  -1.1579e+09 
                                                 5.2753e+23   6.7448e+23 
                                ----------------------------------------
                                Sensitivity 2    2.5549e+08  -4.8020e+08 
                                                -1.8454e+23  -2.3595e+23 
------------------------------------------------------------------------
5.040e+04  5  3.708e+02   425
                                Solution        -3.6771e-08  -3.8820e-08 
                                                 3.3582e+11   3.8644e+11 
                                ----------------------------------------
                                Sensitivity 1   -8.6075e+09  -1.7695e+10 
                                                 5.2067e+23   6.9664e+23 
                                ----------------------------------------
                                Sensitivity 2   -3.4343e+08   9.3179e+08 
                                                -1.8214e+23  -2.4370e+23 
------------------------------------------------------------------------
5.760e+04  5  3.958e+02   435
                                Solution         7.3905e-07   7.3693e-07 
                                                 3.3203e+11   3.9090e+11 
                                ----------------------------------------
                                Sensitivity 1   -8.6842e+09  -1.1898e+10 
                                                 5.0825e+23   7.1205e+23 
                                ----------------------------------------
                                Sensitivity 2   -1.0786e+08   2.9492e+08 
                                                -1.7780e+23  -2.4910e+23 
------------------------------------------------------------------------
6.480e+04  4  4.215e+02   455
                                Solution        -2.5320e-07  -2.5443e-07 
                                                 3.3130e+11   3.9634e+11 
                                ----------------------------------------
                                Sensitivity 1    1.3779e+09   1.3864e+09 
                                                 5.0442e+23   7.3274e+23 
                                ----------------------------------------
                                Sensitivity 2    1.7998e+07  -5.2410e+07 
                                                -1.7646e+23  -2.5633e+23 
------------------------------------------------------------------------
7.200e+04  5  6.445e+02   467
                                Solution         3.3224e-08   3.3434e-08 
                                                 3.3297e+11   4.0388e+11 
                                ----------------------------------------
                                Sensitivity 1   -6.0966e+08  -6.1281e+08 
                                                 5.0783e+23   7.6382e+23 
                                ----------------------------------------
                                Sensitivity 2    5.0541e+08  -1.4775e+09 
                                                -1.7765e+23  -2.6721e+23 
------------------------------------------------------------------------
7.920e+04  5  6.445e+02   478
                                Solution         8.5096e-10   8.5716e-10 
                                                 3.3344e+11   4.1203e+11 
                                ----------------------------------------
                                Sensitivity 1    7.9158e+09   7.9490e+09 
                                                 5.0730e+23   7.9960e+23 
                                ----------------------------------------
                                Sensitivity 2    9.1731e+06  -2.6812e+07 
                                                -1.7747e+23  -2.7972e+23 
------------------------------------------------------------------------
8.640e+04  5  6.445e+02   489
                                Solution         1.0731e-11   1.0824e-11 
                                                 3.3518e+11   4.1625e+11 
                                ----------------------------------------
                                Sensitivity 1   -2.0114e+08  -2.0194e+08 
                                                 5.1171e+23   8.2142e+23 
                                ----------------------------------------
                                Sensitivity 2    5.6451e+04  -1.6467e+05 
                                                -1.7901e+23  -2.8736e+23 
------------------------------------------------------------------------


========================================================
Final Statistics
Sensitivity: YES ( STAGGERED + PARTIAL ERROR CONTROL )

nst     =   489                

nfe     =  5293    nfSe  =   597  
nni     =   600    nniS  =   596  
ncfn    =     0    ncfnS =     0  
netf    =    32    netfS =     0

nsetups =    91                  
nli     =  1813    ncfl  =     1  
npe     =     9    nps   =  3379  
========================================================
\end{verbatim}}
\index{examples, forward sensitivity!parallel sample program!output|)}